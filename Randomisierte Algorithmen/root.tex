\documentclass[12pt]{book}

\usepackage[T1]{fontenc}
\usepackage[utf8]{inputenc}
\usepackage[ngerman]{babel}

\usepackage{tikz-cd}
\usetikzlibrary{babel}

\usepackage{amsfonts}
\usepackage{amssymb}
\usepackage{amsmath}
\usepackage{mathtools}
\usepackage{wasysym}
\usepackage{dsfont}
\usepackage{geometry}
\usepackage{makeidx}
\usepackage{booktabs}
\usepackage{hyperref}

\usepackage{enumerate}
\usepackage{adjustbox}

\newcommand{\ifLeer}[3]{\ifx&#1&\relax#2\relax\else\relax#3\relax\fi\relax}

\newcommand{\Def}[1]{\subsection{Definition\ifLeer{#1}{}{: #1}}}
\newcommand{\Bsp}[1]{\subsection{Beispiel\ifLeer{#1}{}{: #1}}}
\newcommand{\Lem}[1]{\subsection{Lemma\ifLeer{#1}{}{: #1}}}
\newcommand{\Bem}[1]{\subsection{Bemerkung\ifLeer{#1}{}{: #1}}}
\newcommand{\Kor}[1]{\subsection{Korollar\ifLeer{#1}{}{: #1}}}
\newcommand{\Satz}[1]{\subsection{Satz\ifLeer{#1}{}{: #1}}}
\newcommand{\Prop}[1]{\subsection{Proposition\ifLeer{#1}{}{: #1}}}

\newcommand{\QED}{\hfill $\square$}
\newcommand{\qed}{\hfill $\blacksquare$}

\newenvironment{Beweis}[1]{\paragraph{Beweis\ifLeer{#1}{}{: #1}\\}}{\QED}
\newenvironment{Beweisskizze}[1]{\paragraph{Beweisskizze\ifLeer{#1}{}{: #1}\\}}{\qed}

\newcommand{\df}[1]{\index{#1}\textbf{#1}}

\newcommand{\klam}[1]{\left(#1\right)}
\newcommand{\bet}[1]{\left|#1\right|}
\newcommand{\norm}[1]{\bet{\bet{#1}}}
\newcommand{\brak}[1]{\left[#1\right]}
\newcommand{\curv}[1]{\left\lbrace#1\right\rbrace}
\newcommand{\shrp}[1]{\left<#1\right>}
\newcommand{\quot}[1]{\glqq #1 \grqq\relax}
\newcommand{\set}[2]{\curv{\ifLeer{#2}{#1}{#1 ~ | ~ #2}}}
\newcommand{\grp}[2]{\shrp{\ifLeer{#2}{#1}{#1 ~ | ~ #2}}}

\newcommand{\A}{\mathcal{A}}
\newcommand{\B}{\mathcal{B}}
\newcommand{\C}{\mathbb{C}}
\newcommand{\D}{\mathcal{D}}
\newcommand{\E}{\mathcal{E}}
\newcommand{\F}{\mathcal{F}}
\newcommand{\G}{\mathcal{G}}
\renewcommand{\H}{\mathbb{H}}
\newcommand{\I}{\mathcal{I}}
\newcommand{\J}{\mathcal{J}}
\newcommand{\K}{\mathbb{K}}
\renewcommand{\L}{\mathcal{L}}
\newcommand{\M}{\mathcal{M}}
\newcommand{\N}{\mathbb{N}}
\renewcommand{\O}{\mathcal{O}}
\renewcommand{\P}{\mathcal{P}}
\newcommand{\Q}{\mathbb{Q}}
\newcommand{\R}{\mathbb{R}}
\renewcommand{\S}{\mathcal{S}}
\newcommand{\T}{\mathcal{T}}
\newcommand{\U}{\mathcal{U}}
\newcommand{\V}{\mathcal{V}}
\newcommand{\W}{\mathcal{W}}
\newcommand{\X}{\mathcal{X}}
\newcommand{\Y}{\mathcal{Y}}
\newcommand{\Z}{\mathbb{Z}}

\newcommand{\id}[1]{\text{Id}_{#1}}
\newcommand{\Ker}{\textsf{Kern}}
\newcommand{\Coker}{\textsf{Kokern}}
\newcommand{\Img}{\textsf{Bild}}
\newcommand{\Coimg}{\textsf{Kobild}}
\newcommand{\Hom}[3]{\textsf{Hom}_{#1}\left(#2, #3\right)}
\newcommand{\Aut}[2]{\textsf{Aut}_{#1}\left(#2\right)}
\newcommand{\Sym}[1]{\textsf{Symm}_{#1}}

\newcommand{\e}{\varepsilon}

\newcommand{\Pfeil}[1]{\overset{#1}{\longrightarrow}}
\newcommand{\pfeil}[1]{\overset{#1}{\rightarrow}}
\newcommand{\inj}[1]{\overset{#1}{\hookrightarrow}}
\newcommand{\Inj}[1]{\overset{#1}{\lhook\joinrel\longrightarrow}}
\newcommand{\surj}[1]{\overset{#1}{\twoheadrightarrow}}

\newcommand{\impl}[1]{\overset{#1}{\Rightarrow}}
\newcommand{\Impl}[1]{\overset{#1}{\Longrightarrow}}
\newcommand{\gdw}[1]{\overset{#1}{\Leftrightarrow}}
\newcommand{\Gdw}[1]{\overset{#1}{\Longleftrightarrow}}

\newcommand{\off}{\overset{o}{\subset}}
\newcommand{\abg}{\overset{c}{\subset}}

\newcommand{\gl}[1]{\overset{#1}{=}}
\newcommand{\grgl}[1]{\overset{#1}{\geq}}
\newcommand{\klgl}[1]{\overset{#1}{\leq}}
\newcommand{\gr}[1]{\overset{#1}{>}}
\newcommand{\kl}[1]{\overset{#1}{<}}
\newcommand{\isom}[1]{\overset{#1}{\cong}}

\newcommand{\supp}{\text{supp}}

\renewcommand{\i}{^{-1}}
\renewcommand{\phi}{\varphi}
\renewcommand{\d}{\text{d}}

\newcommand{\rot}{\text{rot}}

\renewcommand{\epsilon}{\varepsilon}
\newcommand{\sgn}{\text{sign}}

\setlength{\marginparwidth}{20mm}

\makeindex
\date{\today}
\author{\href{mailto:tensor.produkt@gmx.de}{tensor.produkt@gmx.de}}

\makeindex

\begin{document}
\title{Mitschrieb: Randomisierte Algorithmen\\
SS 18}
\maketitle
\section*{Vorwort}
Dies ist ein Mitschrieb der Vorlesungen vom 16.04.18 bis zum ... des Kurses \textsc{Randomisierte Algorithmen} an der Universität Heidelberg.\\
Dieses Dokument wurde \glqq{live}\grqq\ in der Vorlesung getext. Sämtliche Verantwortung für Fehler übernimmt alleine der Autor dieses Dokumentes.\\
Auf Fehler kann gerne hingewiesen werden bei folgende E-Mail-Adresse
\begin{center}
	\href{mailto:tensor.produkt@gmx.de}{tensor.produkt@gmx.de}
\end{center}
Ferner kann bei dieser E-Mail-Adresse auch der Tex-Code für dieses Dokument erfragt werden.

\setcounter{tocdepth}{1}
\tableofcontents

%Prüfungstag: Mittwoch, 18. Juli

\chapter{Einführung in die Riemannsche Geometrie}
\section{Überblick und Ideen}
\marginpar{Vorlesung vom 16.04.18}

Bisher können wir durch die äußere Ableitung
\[ \d : \Omega^p(M) \pfeil{} \Omega^{p+1}(M) \]
nur Differentialformen auf glatten Mannigfaltigkeiten ableiten, aber keine anderen Objekte wie zum Beispiel Vektorfelder. Wir können also auch nicht über Phänomene aus der Physik wie Beschleunigung zum Beispiel sprechen.

\paragraph{Ziel}
Wir wollen einen Rahmen finden, in dem Objekte wie zum Beispiel Vektorfelder abgeleitet werden können.

\Bsp{}
Sei $f:M \pfeil{} \R$ eine glatte Funktion. Gilt $\d f = 0$ und ist $M$ zusammenhängend, so ist $f$ konstant.\\
Hätten wir für ein Vektorfeld $\xi$ eine Ableitung $\d \xi$, dann sollte die Gleichung $\d \xi = 0$ implizieren, dass $\xi$ \textsl{konstant} ist.\\
Ist zum Beispiel $\xi$ auf $M = \R^n$ konstant, so ist $\xi$ parallel, im Sinne von, die einzelnen Tangentialvektoren, die im Bild von $\xi$ liegen, sind parallel.\\
Somit impliziert eine Ableitung für Vektorfelder ein Konzept von \textsl{Parallelismus}.

\paragraph{Problem}
Ein Konzept von Parallelismus kann nicht über Karten erklärt werden, weil Kartenwechsel im Allgemeinen nicht winkeltreu sind.

\Bsp{}
Sei $M = S^2 \subset \R^3$ die zweidimensionale Einheitssphäre. Sei $p \in S^2$ und $\xi(p) \in T_pS^2$.\\
$\gamma$ sei ein Großkreis, der durch $p$ in Richtung $\xi(p)$ geht. Ist $p_1$ ein weiterer Punkt auf $\gamma$, so lässt sich $\xi(p)$ \textsl{naiv} wie gewohnt in $\R^3$ von $p$ auf $p_1$ verschieben. Dies hat das offensichtliche Problem, das der so parallel verschobene Vektor im Allgemeinem nicht tangential an $S^2$ anliegt.\\
Diesen kann man nun orthogonal auf den Tangentialraum $T_{p_1}S^2$ projizieren. Dadurch erhält man einen Tangentialvektor $\xi(p_1) \in T_{p_1}S^2$. Durch dieses Prozedere lässt sich $\xi$ glatt auf $S^2$ fortsetzen. Wählt man weitere Punkte $p_i$ auf $\gamma$, die gegen einen Punkt $q$ am Äquator konvergieren und für die gilt
\[ d(p_i, p_{i+1}) \Pfeil{} 0 \]
dann erhalten wir einen Vektor $\xi(q) \in T_qS^2$. Dies nennt man den \df{Paralleltransport} von $\xi(p)$ entlang $\gamma$ zu $\xi(q)$.\\
Allerdings kann man $\xi(p)$ auch entlang eines weiteren Großkreises $\gamma_1$ verschieben. Verschiebt man entlang $\gamma_1$ wieder auf den Äquator und von dort wieder auf $q$, so erhält man einen anderen Tangentialvektor auf $q$.

\paragraph{Neues Phänomen}
Für allgemeine Mannigfaltigkeiten hängt der Paralleltransport vom Weg $\gamma$ ab; im Gegensatz zum Euklidischen Raum.

\subsection{Zurück zu Ableitungen von Vektorfeldern $\xi$}
Auf $M$ sei Parallelismus gegeben (zum Beispiel ist $M$ eingebettet im $\R^n$). $p\in M$ sei ein Punkt und $v \in T_pM$ sei ein Tangentialvektor. $\xi$ sei ein Vektorfeld auf $M$.\\
Sei $\gamma$ eine glatte Kurve mit $\gamma(0) = p$ und $\dot{\gamma} (0) = v$. $q$ sei ein Punkt auf $\gamma$. Durch den vorgegebenen Parallelismus lässt sich $\xi(p)$ entlang $\gamma$ verschieben. D.\,h., im Punkt $q$ haben wir die Vektoren $\xi(q)$ und $\tau^q_p\xi(p)$, wobei $\tau^q_p\xi(p)$ der Paralleltransport von $\xi(p)$ nach $q$ entlang $\gamma$ ist.

\paragraph{Idee}
Betrachte
\[ \xi(q) - \tau^q_p\xi(p) \in T_pM  \]
für $\d(p,q) \pfeil{} 0$. Dies bezeichnet man dann auch als die \df{kovariante Ableitung} von $\xi$ in Richtung $v$
\[ \nabla_v\xi \in T_qM \]
$\nabla_v$ nennt man dabei einen \df{Zusammenhang.} Diese hat folgende Eigenschaften:
\begin{itemize}
	\item $\nabla_v$ ist $\Omega^0(M)$-linear in $v$, d.\,h.
	\[ \nabla_{\lambda v + w}(\xi) = \lambda \nabla_{v} (\xi) + \nabla_{w} (\xi) \]
	für glatte Funktionen $\lambda : M \pfeil{} \R$.
	\item Sie ist $\R$-linear im zweiten Argument
	\[ \nabla_{v}(\xi + \eta) = \nabla_{v}(\xi) + \nabla_{v}(\eta) \]
	\item Ist $f : M \pfeil{} \R$ linear, so liegt folgende Produktregel vor
	\[ \nabla_{v}(f\cdot \xi) = f \cdot \nabla_{v}(\xi) + \nabla_{v}(f) \cdot \xi \]
	wobei
	\[ \nabla_{v} f := v(f) \]
\end{itemize}


\subsection{Geodätische}
Sei $\gamma$ eine (glatte) Kurve auf $M$. $\gamma$ heißt eine \df{Geodätische}, falls gilt
\[ \nabla_{\dot{\gamma}}\dot{\gamma} = 0 \]
Obige Bedingung ist in lokalen Koordinaten eine Differentialgleichung zweiter Ordnung.\\
Physikalisch gesprochen verschwindet die Beschleunigung. Geometrisch gesprochen ist $\gamma$ parallel entlang $\gamma$.

\Bsp{}
Sei $M$ eine Riemannsche Fläche im $\R^3$. Die Gleichung
\[ \nabla_{\dot{\gamma}}\dot{\gamma} = 0 \]
bedeutet
\[ \ddot{\gamma} \bot M \]
D.\,h., die Euklidische zweite Ableitung steht orthogonal auf der Fläche $M$.

\Bsp{}
\begin{itemize}
	\item Geraden sind Geodätische im Euklidischen Raum.
	\item Großkreise sind Geodätische auf Sphären.
	\item Allgemein sind Geodätische lokal kürzeste Kurven.
\end{itemize}

\newpage
\subsection{Parallelogramme}
Sei $p\in M$. $\mu, \lambda$ seien zwei Geodätische, die sich im Punkt $p$ schneiden mit $\mu(0) = \lambda(0) = p$.\\
$\mu, \lambda$ seien parametrisiert durch die Bogenlänge, d.\,h.,
\[ \norm{\dot{\lambda}(t)} = \norm{\dot{\mu}(t)} = 1 \]
für alle $t$. Setze $v := \dot{\mu}(0)$ und $w := \dot{\lambda}(0)$. Sei $\e > 0$.\\
Indem wir $w$ entlang $\mu$ verschieben, erhalten wir einen Vektor $\overline{w}$ auf $\mu(\e)$ und analog einen Vektor $\overline{v}$ auf $\lambda(\e)$.\\
Es gilt
\[ \norm{\overline{v}} = \norm{\overline{w}} = 1 \]
da der Paralleltransport eine Isometrie ist, wenn die Riemannsche Metrik kompatibel ist zum Zusammenhang $\nabla$.\\
Indem man $\overline{v}$ und $\overline{w}$ durch durch Bogenlänge parametrisierte Geodätische fortsetzt, erhält man Geodätische $\overline{\mu}$ und $\overline{\lambda}$. Dadurch erhält man dann Punkte $\overline{\lambda}(\e) $ und $\overline{\mu}(\e)$. Im Euklidischen würden die beiden Punkte zusammen fallen und das Parallelogramm schließen. Für allgemeine Riemannsche Mannigfaltigkeiten muss dies nicht der Fall sein, aber es gilt
\[ d(\overline{\mu}(\e), \overline{\lambda}(\e)) \in O(\e^2) \]

\printindex
\end{document}