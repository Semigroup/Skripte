\documentclass{book}

\usepackage{../Package/latexa}
\usepackage{../Package/algebra}
\usepackage{../Package/theorema}
\usepackage{../Package/diagramma}
\usepackage{../Package/categoria}

\usepackage{tikz}
\usepackage{tikz-cd}
\usetikzlibrary{arrows}

\newcommand{\qi}{\backsimeq_{\textsc{QI}}}
\newcommand{\ba}{\backsimeq_{\textsc{BA}}}
\newcommand{\normal}{\vartriangleleft}
\newcommand{\tm}{\subset}
\newcommand{\Stab}[2]{\textsf{Stab}_{#1}(#2)}
\newcommand{\Cay}[2]{\textsf{Cay}(#1,#2)}


\renewcommand{\A}{\mathbb{A}}
\newcommand{\Nc}{\mathcal{N}}

\renewcommand{\i}{^{-1}}
\renewcommand{\d}{\textsf{d}}
\renewcommand{\P}{\mathbb{P}}

\newcommand{\af}{\mathfrak{a}}
\newcommand{\Af}{\mathfrak{A}}
\renewcommand{\bf}{\mathfrak{b}}
\newcommand{\Bf}{\mathfrak{B}}
\newcommand{\cf}{\mathfrak{c}}
\newcommand{\Cf}{\mathfrak{C}}
\newcommand{\ff}{\mathfrak{f}}
\newcommand{\Ff}{\mathfrak{F}}
\newcommand{\pf}{\mathfrak{p}}
\newcommand{\Pf}{\mathfrak{P}}
\newcommand{\mf}{\mathfrak{m}}
\newcommand{\Mf}{\mathfrak{M}}
\newcommand{\qf}{\mathfrak{q}}
\newcommand{\Qf}{\mathfrak{Q}}

\renewcommand{\M}{\mathbb{M}}

\newcommand{\Ac}{\mathcal{A}}
\newcommand{\Rc}{\mathcal{R}}
\renewcommand{\O}{\mathcal{O}}

\newcommand{\Frob}{\textsf{Frob}}

\newcommand{\Leg}[2]{\left(\frac{#1}{#2}\right)}

\newcommand{\ric}{\textsf{ric}}

\usepackage{enumerate}

\makeindex

\begin{document}

\chapter{Erste Woche}
\section{Krümmungen}

\Def{}
Es sei $M$ eine Riemannsche Mannigfaltigkeit und $\kappa_p(\sigma)$ bezeichne die Schnittkrümmung in einem Punkt $p$ und einer Ebene $\sigma \subset T_pM$.\\
$M$ heißt \df{von positiver Schnittkrümmung}, falls $\kappa_p(\sigma) > 0$ für alle $p \in M, \sigma \subset T_pM$.
Bezeichnet $\ric$ die Ricci-Krümung, so heißt $M$ \df{von positiver Ricci-Krümmung}, falls $\ric_p(v,v) > 0$ für alle $p \in M, v \in T_pM$.\\
Die Ricci-Krümmung von $M$ wird von $\alpha > 0$ nach unten beschränkt, falls
\[ \ric_p(v,v) \geq \alpha g_p(v,v) \]
für alle $p \in M, v \in T_pM$.

\Def{Isometrien}
Eine glatte Abbildung $f : M \pfeil{} N$ heißt \df{lokal isometrisch}, falls für alle $p \in M$ das Differential $\d_p : T_pM \pfeil{} T_pN$ eine lineare Isometrie metrischer Räume ist.

\Bem{}
$f$ ist genau dann isometrisch, wenn $f$ lokal isometrisch und diffeomorph ist.

\Satz{}
Jede glatte Abbildung $f : M \pfeil{} M$, die für einen Punkt eine Isometrie auf den Tangentialräumen induziert, ist eine Isometrie.\\
Jede glatte, abstandserhaltende Funktion $f : M \pfeil{} M$ ist eine Isometrie.

\Def{}
Die \df{Kompakt-Offen-Topologie} auf $\textsf{Isom}(M,M)$ wird durch eine Basis folgender Mengen generiert
\[ W(K,U) := \set{f}{ f(K) \subseteq U } \]
für alle $K \subseteq M$ kompakt und $ U \subseteq M$ offen.

\Prop{Cartan}
Seien $v,w$ normierte, orthogonale Vektoren in $T_pM$. Es gilt
\[ d(c_v(t), c_w(t))^2 = 2t^2 - \frac{\kappa_p(\sigma_{v,w})}{6} t^4 + o(t^5)  \]

\Lem{}
Ist $Y$ ein Jacobifeld entlang $c$, so gilt
\[ D_t(\Rc (\dot{c}, Y), \dot{c})(0) = \Rc (\dot{c}, D_tY)\dot{c}  \]
\begin{Beweis}{}
\begin{align*}
\frac{\d}{\d t} g(\Rc(\dot{c} , Y) \dot{c}, w ) &= g(D_t \Rc(\dot{c}, Y) \dot{c}, w ) + g(\Rc (\dot{c}, Y) \dot{c}, D_tw )\\
\frac{\d}{\d t} g(\Rc(\dot{c} , w) \dot{c}, Y ) &= g(D_t \Rc(\dot{c}, w) \dot{c}, Y ) + g(\Rc (\dot{c}, w) \dot{c}, D_tY )\\
g(D_t \Rc(\dot{c}, Y) \dot{c}, w ) &= g(D_t \Rc(\dot{c}, w) \dot{c}, Y ) + g(\Rc(\dot{c}, w) \dot{c}, D_tY ) =  g(\Rc(\dot{c}, D_tY) \dot{c}, w )
\end{align*}
\end{Beweis}


\Prop{}
Seien $v,w \in T_pM$ mit $\bet{w} = 1$. Setze $Y(t) = (\d \exp_p)_{tv}(tw) \in J_{c_v}$. Dann gilt
\[ \bet{Y(t)}^2 = g(Y(t), Y(t)) = t^2 - \frac{1}{3} g(\Rc(v,w)w, v)t^4 + o(t^5) \]

\begin{Beweis}{}
Es gilt
\[ Y(0) = 0, D_t(Y)(0) = w \]
Man rechnet nach
\begin{align*}
	g(Y,Y)(0)     & = 0                                                                    \\
	g(Y,Y)'(0)    & = \frac{\d}{\d t}_{|t = 0} g(Y(t), Y(t)) = 2g(D_t(Y),Y) (0) = 0        \\
	g(Y,Y)''(0)   & = 2g(D_tY, D_tY)(0) + 2g(D_tD_tY,Y)(0) = 2g(w,w) = 2                   \\
	g(Y,Y)'''(0)  & = 6g(D_tD_tY, D_tY)(0) = 6g(-\Rc(\dot{c}, Y) \dot{c}, D_t Y ) (0) = 0  \\
	g(Y,Y)''''(0) & = 8g(\Rc(\dot{c}, D_tY )\dot{c}, D_tY  ) - \frac{8}{24} g(\Rc(v,w)w,v) = - \frac{1}{3}g(\Rc(v,w)w,v)
\end{align*}
\end{Beweis}

\newpage
\chapter{Ls Vorlesungen}
\section{Räume konstanter Krümmung}
\Def{}
Unter $\M_k^n$ verstehen wir die bis auf Isomorphie eindeutige vollständige, einfache zusammenhängende, $n$-dimensionale Riemannsche Mannigfaltigkeit von konstanter Schnittkrümmung $k$.
\Bem{}
Skaliert man die Metrik einer Riemannschen Mannigfaltigkeit mt $c > 0$, so skaliert sich ihre Krümmung mit $\frac{1}{c}$.
\Bsp{}
Man führe auf $\R^{n+1}$ folgende Bilinearformen ein
\begin{align*}
(x,y) &:= \sum_{i = 1}^{n+1}x_i \cdot y_i\\
\shrp{x,y} &:= \sum_{i = 1}^{n}x_i \cdot y_i - x_{n+1} \cdot y_{n+1}\\
\end{align*}
Dann gilt
\begin{align*}
\M_0^n = \R^n && \M_1^n = S^n = \{x \in \R^{n+1} ~|~ (x,x) = 1\} && \M_{-1}^n = \H^n = \{ x \in \R^{n+1} ~|~ (x,x) = 1, x_{n+1} > 0 \} 
\end{align*}

\Def{}
Der \df{Index} einer symmetrischen Bilinearform ist die größte Dimension aller Unterräume, auf denen jene Bilinearform negativ definit ist.

\Def{}
Sei $(M,g)$ eine semi-Riemannsche Untermannigfaltigkeit einer Riemannschen Mannigfaltigkeit $(M',g')$. Bezeichnet $\iota : M \inj{} M'$ die Einbettung, so verstehen wir unter $\V'(M) := \Gamma(\iota^*(TM'))$ den Pullback der Vektorfelder auf $M'$, d.h.
\[ \V'(M) = \set{Y_{|M}}{Y \in \V(M')} \]
Unter $\V(M)^\bot \subset \V'(M)$ verstehen wir die Vektorfelder auf $M$, die normal zu $M$ liegen.\\
Wir schränken den Levi-Civita-Zusammenhang von $M'$ auf $\V(M) \times \V'(M)$ ein und erhalten
\[ \D' : \V(M) \times \V'(M) \Pfeil{} \V'(M) \]
Betrachte folgende Zerlegung
\[ T_pM' = T_pM \oplus (T_pM)^\bot \]
Unter dem \df{Koindex von $M$} verstehen wir den Index auf $(T_pM)^\bot$. Es gilt
\[ \text{Index}(T_pM') = \text{Index}(T_pM) + \text{Index}(T_pM)^\bot \]

\Lem{}
$\D'$ ist wohldefiniert.
\begin{Beweis}{}
Sei $p \in M$ ein Punkt und $U \subset M$ eine Koordinatennachbarschaft von $p$. Seien ferner $V \in \V(M), X \in \V'(M)$ mit Liftungen $V',X' \in \V(M')$. $X'$ lässt sich auf $U$ darstellen durch
\[ X' = \sum_{i} f^i \frac{\d}{\d x_i} \]
\[ \D'_{V'}X' = \sum_i V'(f^i) \frac{\d}{\d x_i} + \sum_i f^i D'_{V'}\klam{\frac{\d}{\d x_i}} \]
Für $q \in U\cap M$ gilt
\[V'(f^i)(q) = V_q(f^i) = V_q(f^i_{|U\cap M}) \]
und
\[ \D'_{V'}\klam{\frac{\d}{\d x_i}}(q) = \D'_{V'_q}\klam{\frac{\d}{\d x_i}} \]
\end{Beweis}
\Kor{}
Seien $V,W \in \V(M), X,Y \in V'(M)$.
\begin{itemize}
\item $D'_VX$ ist $\F(M)$-linear in $V$.
\item $D'_VX$ ist $\R$-linear in $X$.
\item $D'_V(fX) = (Vf) X + f\D'_VX$
\item $[V,W] = \D'_VW - \D'_WV$
\item $Vg(X,Y) = g(D'_VX, Y) + g(X, D'_VY)$
\end{itemize}
\Lem{}
Sind $V,W \in \V(M)$, dann gilt
\[ \D_VW = (\D'_VW)^T \]
\begin{Beweis}{}
Sei $X \in \V(M)$ beliebig
\begin{align*}
2g(\D'_{V'}W', X') = V'g(W',X') + W'g(X',V') - X'g(V',W') + g(W',[X',V']) + g(X', [V',W'])
\end{align*}
Mit dem Vorhergehenden Korollar erhalten wir durch Einschränkung auf $M$
\[ g(\D'_{V}W, X) = g(\D_{V}W, X) \]
Es gilt ferner
\[ g(\D'_{V}W, X) = g((\D'_{V}W)^T, X) \]
da $X$ tangential an $M$ liegt.
\end{Beweis}

\Lem{}
Definiere den \df{Gestalttensor} bzw. \df{zweiten Fundamentaltensor} durch
\begin{align*}
\Pi : \V(M) \times \V(M) &\Pfeil{} \V(M)^\bot\\
(V,W) & \longmapsto (\D'_VW)^N = \D'_VW - \D_VW
\end{align*}
Dann ist dieser Tensor $\F(M)$-bilinear und symmetrisch.
\begin{Beweis}{}
\begin{align*}
(\D'_V(fW))^N = (f\D'_VW)^N + (V(f) W)^N = (f\D'_VW)^N\\
\Pi(V,W) - \Pi(W,V) = (\D'(V,W) - \D'(W,V))^N = [V,W]^N = 0
\end{align*}
\end{Beweis}

\Satz{Gauss-Gleichung}
Seien $V,W,X,Y \in \V(M)$. Es gilt
\[ g(\Rc(V,W)X,Y) = g(\Rc'(V,W)X,Y) + g(\Pi(V,X), \Pi(W,Y )) - g(\Pi(V,Y), \Pi(W,X )) \]
\begin{Beweis}{}
\begin{align*}
\Rc(X,Y)Z &= -\D_{[X,Y]}Z + [\D_X,\D_Y]Z\\
g(\D'_{[V,W]}X,Y) &= g(\D_{[V,W]}X,Y)\\
g(\D'_V\D'_WX,Y) &= g(\D'_V\D_WX,Y) + g(\D'_V \Pi(W,X), Y)\\
&= g(\D_V\D_WX,Y) + (Vg(\Pi(W,X), Y) - g(\Pi(W,X), \D'_VY ))\\
\end{align*}
Nun gilt aber
\[ g(\Pi(W,X), Y) = 0 \]
Ergo
\begin{align*}
g(\D'_V\D'_WX,Y) &= g(\D_V\D_WX,Y) + (Vg(\Pi(W,X), Y) - g(\Pi(W,X), \D'_VY ))\\
&= g(\D_V\D_WX,Y) -g(\Pi(W,X), \Pi(V,Y) )
\end{align*}
\end{Beweis}

\Kor{}
Sind $v,w \in T_pM$ linear unabhängig, so gilt
\[ \kappa(v,w) = \kappa'(v,w) + \frac{g(\Pi(v,v), \Pi(w,w)) - g(\Pi(v,w), \Pi(v,w))}{g(v,v) g(w,w) - g(v,w)^2} \]

\Prop{}
Sei $Y$ ein Vektorfeld tangential zu $M$ entlang einer Kurve $\alpha(t)$ in $M$. Dann gilt
\[ \D_t' Y = \D_tY + \Pi(\dot{\alpha}, Y) \]

\Kor{}
$\alpha$ ist genau dann eine Geodäte von $M$, wenn $\D'_t(\dot{\alpha})$ normal zu $M$ steht.

\Def{}
Eine \df{semi-Riemannsche Hyperfläche} ist eine \df{semi-Riemannsche Untermannigfaltigkeit} der Kodimension Eins.

\Def{}
Das \df{Vorzeichen} einer semi-Riemannschen Hyperfläche $M$ ist definiert durch
\[ \e := \left\lbrace 
\begin{aligned}
+1 && \textsf{coindex}M = 0\\
-1 && \textsf{coindex}M = 1
\end{aligned}
\right. \]

\Def{}
Sei $U$ ein normierter Normalenvektorfeld einer semi-Riemannschen Hyperfläche $M$. Dann wird durch
\[ g(S(V),W)  = g(\Pi(V,W), U)\]
ein (1,1)-Tensorfeld $S$ auf $M$ definiert, der sogenannte \df{Gestaltoperator}, der durch $U$ induziert wird.

\Lem{}
Für alle $v \in T_pM$ gilt
\[ S(v) = -\D'_vU \]
Ferner ist $S_p$ selbstadjungiert.
\begin{Beweis}{}
$g(U,U)$ ist konstant gleich 1. Ergo gilt
\[ g(\D'_VU,U) = 0 \]
Daraus folgt, dass $\D'_VU$ für alle $V \in \V(M)$ tangent an $M$ liegt.\\
Sei nun $W \in \V(M)$, dann gilt
\[ g(S(V),W) = g(\Pi(V,W), U) =g(\D'_VW, U) \gl{g(U,W) = 0} g(\D'_VU,W)   \]
\end{Beweis}

\Kor{}
\[ \kappa(v,w) = \kappa'(v,w) + \e \frac{g(Sv,v)g(Sw,w) - g(Sv,w)^2}{g(v,v) g(w,w) - g(v,w)^2} \]
\begin{Beweis}{}
\[ \Pi(v,w) = \e g(Sv,w) U \text{ und } g(U,U) = \e \]
\end{Beweis}

% % % Vorlesung
\newpage
\Def{3 Shades of Artig-Sein}
Sei $x \in \R^{n+1}$.
\begin{align*}
x \text{ heißt } \left\lbrace \begin{aligned}
\text{zeitartig, falls } &\shrp{x,x} < 0\\
\text{lichtartig, falls } &\shrp{x,x} = 0\\
\text{raumartig, falls } &\shrp{x,x} > 0
\end{aligned} \right.
\end{align*}

\Prop{}
Sei $M = \set{p \in \R^{n+1}}{\shrp{p,p} = -r^2}$. Dann ist $M$ eine Riemannsche Mannigfaltigkeit der konstanten Schnittkrümmung $-\frac{1}{r^2}$ für $r > 0$.
\begin{Beweis}{}
Definiere $f : \R^{n+1} \pfeil \R$ durch $x \mapsto \shrp{x,x}$. Dann ist $\d f_p(v) = 2\shrp{v,p}$. Ergo ist $\d f_p$ surjektiv für alle $p \in M$. Ergo ist $f\i(-r^2)$ eine glatte Hyperfläche von $\R^{n+1}$.\\
Ferner ist
\[ T_pM = \Ker \d f_p  = \set{v \in \R^{n+1}}{\shrp{p,v} = 0} = p^\bot \]
Da $p$ zeitartig ist, ist $T_pM$ positiv definit. Ferner ist der vom Einheiten-Normalen Vektorfeld $U = \frac{p}{r}$ abgeleitete Gestaltoperator gegeben durch $S(V) = - \D'_VU = \frac{-V}{r}$. Es folgt
\[ \kappa(v,w) = 0 - \frac{1}{r^2} = - \frac{1}{r^2} \] 
\end{Beweis}

\Def{}
Unter einer \df{Paar-Isometrie} $\Phi : (M,M') \pfeil{} (N,N')$, wobei $M \subset M', N \subset N'$, ist eine Isometrie $M' \pfeil{} N'$, dergestalt, dass auch ihre Restriktion $M \pfeil{} N$ eine Isometrie ist.

\Lem{}
Eine Paar-Isometrie $\Phi : (M,M') \pfeil{} (N,N')$ erhält den Gestalttensor, d.\,h.
\[ \d \Phi (\Pi (v,w)) = \Pi ( \d \Phi(v),\d \Phi(w) ) \]
für alle $p \in M, v,w \in T_pM$.

\Prop{}
Seien $p,q \in M = \set{p \in \R^{n+1}}{\shrp{p,p} = -r^2}$ mit ONBs $e_i \in T_pM, f_i \in T_qM$. Dann existiert eine eindeutige Paar-Isometrie
\[ \phi : (\R^{n+1}, M) \Pfeil{} (\R^{n+1}, M) \]
mit
\begin{align*}
\phi(p) = q \text{ und } \d \phi_p(e_i) = f_i
\end{align*}


\Def{}
Eine semi-Riemannsche Untermannigfaltigkeit $M \subset M'$ heißt \df{total geodätisch}, falls $\Pi = 0$.

\Prop{}
Für eine semi-Riemannsche Untermannigfaltigkeit $M \subset M'$ sind folgende Aussagen äquivalent:
\begin{enumerate}[1.)]
\item $M$ ist total geodätisch in $M'$
\item Jede Geodäte von $M$ bleibt eine Geodäte in $M'$
\item Für $v \in T_pM \subset T_pM'$ liegt das Anfangstück von $\gamma(t) = \exp'_p(tv)$ in $M$
\item Für jeden Weg in $M$ stimmt der Paralleltransport in $M$ mit dem in $M'$ überein.
\end{enumerate}

\Prop{}
Seien $M, N \subset M'$ vollständige, zusammenhängende, total geodätische semi-Riemannsche Mannigfaltigkeiten. Existiert ein Punkt $p \in M\cap N$ mit
\[ T_pM = T_pN \] 
so folgt
\[ M = N\]

\Prop{}
Bezeichnet $H^n \subset \R^{n+1}$ das Hyperboloidenmodell des hyperbolischen Raumes, so sind die $k$-dimensionalen, vollständigen, total geodätischen, zusammenhängenden Riemannschen Untermannigfaltigkeiten von $H^n$ genau die Schnitte
\[ H^n \cap W^{k+1}\]
wobei $W^{k+1}$ ein $k+1$-dimensionaler Untervektorraum von $\R^{n+1}$ ist.

\Kor{}
$H^n$ ist vollständig.

\Lem{Poincare-Scheiben-Modell}
Definiere
\[ H^n = \set{p \in \R^{n+1}}{\shrp{p,p} = -1} \text{ und } D^n = \set{x \in \R^n}{\norm{x} < 1} \]
und folgenden Diffeomorphismus
\begin{align*}
p : H^n & \Pfeil{} D^n\\
(x_1, \ldots, x_n, x_{n+1}) & \longmapsto \frac{1}{x_{n+1} + 1} (x_1,\ldots, x_n) 
\end{align*}
Dann ist die Metrik auf $D^n$ gerade gegeben durch
\[ g^D_p = \klam{\frac{2}{1 - \norm{p}^2}}^2g^E_p \]
wobei $g^E$ die euklidische Metrik von $\R^n$ bezeichnet.


% % % % 3.Vorlesung von L
\Lem{}
Es seien
\[ O(n,1) = \set{A \in \R^{n+1\times n + 1}}{ \shrp{v,w} = \shrp{Av, Aw} }\]
und
\[ O(n,1)^+ = \set{A \in O(n,1)}{A (H^n) \subset H^n } \]
Dann ist $O(n,1)^+$ eine Index-2-Gruppe von $O(n,1)$ und
\[ Isom(H^n) = O(n,1)^+ \]
Ferner gilt
\[ Isom(S^n) = O(n) \text{ und } Isom(\R^n) = \set{x\mapsto Ax + b}{A \in O(n), b\in \R^n } \]

\Def{}
Eine \df{$k$-Ebene} von $H^n$ ist eine vollständige, zusammenhängende, total geodätische, $k$-dimensionale Riemannsche Untermannigfaltigkeit von $H^n$.\\
Ist $S = H^n\cap W^n$ eine Hyperebene von $H^n$, so definiere die \df{Reflektion} an $S$ durch
\begin{align*}
r_S : \R^{n+1} & \Pfeil{} \R^{n+1}\\
\klam{r_S}_{|W^n} &= \id{W^n}\\
\klam{r_S}_{|{W^n}^\bot} &= -\id{{W^n}^\bot}
\end{align*}

\Prop{}
Die Reflektionen an Hyperebenen erzeugen $Isom(H^n)$.

\Def{}
Ein Diffeomorphismus
\[ f : (M,g) \Pfeil{} (N,h) \] 
heißt \df{konform}, falls eine glatte Funktion $f : M \pfeil{} \R_{>0}$ existiert, sodass
\[ f^*(h_{f(p)}) = \lambda(p)\cdot g_p  \]

\Bem{}
Die Poincare-Scheibe ist ein konformes Modell von $\H^n$, d.\,h., $(D^n, g^E)$ und $(D^n, g^D)$ sind zueinander konform.

\Lem{}
Die $k$-dimensionalen, vollständigen, zusammenhängenden, total geodätischen Untermannigfaltigkeiten der Poincare-Scheibe sind ihre Schnitte mit $k$-Sphären und $k$-Ebenen von $\R^n$, die orthogonal zum Rand der Poincare-Scheibe liegen.

\Def{}
Sei $S_{p}(r) \subset \R^n$ eine Sphäre mit Radius $r$ um $p$. Definiere die \df{Inversion} an $S_p(r)$ durch
\begin{align*}
\phi : \R^n\setminus\{p\} & \Pfeil{} \R^n\setminus\{p\}\\
x & \longmapsto p + r^2\frac{x- p}{\norm{x-p}^2}
\end{align*}
\Prop{}
Jede Inversion ist konform und bildet Sphären und Ebenen auf Sphären und Ebenen ab.

\Def{}
Das \df{obere Halbebenen-Modell}
\[ U^n = \set{x \in \R^{n}}{x_n > 0} \]
ergibt sich durch eine Inversion der Poincare-Scheibe an der Sphäre
\[ S = S_{(0,\ldots,0,-1)}(\sqrt{2}) \]
Insofern ist die obere Halbebene ein konformes Modell von $\H^n$.

\Prop{}
Die $k$-Ebenen von $U^n$ sind die $k$-Ebenen und $k$-Sphären von $\R^n$, die orthogonal zu $\partial U^n$ sind.

\Prop{}
Die Metrik auf $U^n$ ist gegeben durch
\[ g_x^U =  \frac{1}{x_n^2} g^E \]


\Prop{}
Folgende Abbildungen sind Isometrien von $U^n$:
\begin{enumerate}[1.)]
\item Horizontale Translationen:
\begin{align*}
x \longmapsto x + (b_1,\ldots, b_{n-1}, 0)
\end{align*}
\item Dilationen:
\[ x \longmapsto x \cdot \lambda \]
\item Inversionen an Sphären orthogonal zu $\partial U^n$
\end{enumerate}

\Prop{}
Die Isometrien der Poincare-Scheibe und der oberen Halbebene werden durch Inversionen an Sphären und Reflektion an Euklidischen Ebenen, die alle orthogonal zum Rand stehen, erzeugt.
\Prop{}
In den konformen Modellen sind Kugeln genau die euklidischen Kugeln mit exzentrischen Mittelpunkten.

% % % Ls fünfte VL
\Satz{Thurston}
Ein Knoten in $S^3$ ist entweder ein Torus-Knoten, d.\,h., er kann auf die Oberfläche eines Torus platziert werden, ein Satellitknoten, d.\,h., es gibt einen echt einfacheren Knoten, sodass der Satellitknoten in der Verdickung des einfacheren lebt, oder ein hyperbolischer Knoten, d.\,h., sein Komplement ist eine hyperbolische Mannigfaltigkeit.

\Satz{Cartan} 
Seien $M,N$ zwei Riemannsche Mannigfaltigkeiten der Dimension $n$, $p \in M, q \in N$. Ferner sei eine lineare Isometrie
\begin{align*}
i : T_pM & \Pfeil{} T_pN
\end{align*}
gegeben und eine offene Umgebung $p \in V\subset_o M$, sodass
\begin{align*}
f:= \exp_q \circ i \circ \exp_p\i : V \Pfeil{} f(V)
\end{align*}
wohldefiniert ist. Definiere für $p'\in V$ folgende Abbildung
\begin{align*}
\phi_{p'} : T_pM & \Pfeil{} T_qN\\
x &\longmapsto P^N_t \circ i \circ \klam{P^M_t}\i
\end{align*}
wobei $P_t^M : T_pM \pfeil{} T_{p'}M$ den Paralleltransport entlang der Geodäte 
\[\gamma : \gamma(0) = p \mapsto \gamma(t) =  p'\]
und $P_N$ den Paralleltransport entlang der Geodäte $\delta$ mit
\[ \delta(0) = q \text{ und } \dot{\delta}(0) = i(\dot{\gamma}(0)) \]
bezeichnet.\\
Gilt nun für alle $p'\in V,x,y,z,w \in T_{p'}M$
\[ g(\Rc(x,y)z, w) = g(\Rc(\phi_t(x), \phi_t(y)) \phi_t(z),\phi_t(w)) \]
Dann ist $f$ eine lokale Isometrie mit $\d f_p = i$.

\begin{Beweis}{}
\begin{enumerate}[1.)]
\item Sei $\gamma : [0,t] \pfeil{} M$ eine Geodäte von $p$ nach $p'$ und $v \in T_pM$ beliebig. Definiere ein Jacobi-Feld $J$ entlang $\gamma$ durch
\[ J(0) = 0 \text{ und } J(l) = v \]
Sei $\{e_1(t), \ldots, e_{n-1}(t), e_n(t) = \dot{\gamma}(t)\}$ eine ONB von parallelen Vektorfeldern entlang $\gamma$. Es gilt
\[ J(t) = \sum_{i = 1}^n y_i(t) e_i(t) \]
und
\begin{align*}
&\D_t\D_tJ(t) + \Rc( \dot{\gamma}(t), J(t)) \dot{\gamma}(t) = 0\\
\Gdw{} & \sum_{i = 1}^{n} y_i''(t) e_i(t) + \sum_{i = 1}^ny_i(t)\Rc (e_n(t), e_i(t) ) e_n(t) = 0\\
\Gdw{} &  \forall j:~ y_j''(t) + \sum_{i = 1}^ny_i(t)g(e_j(t), \Rc (e_n(t), e_i(t) ) e_n(t)) = 0 
\end{align*}
Definiere nun $\delta : [0,l] \pfeil{} N$ durch
\[ \delta(0) = q \text{ und } \dot{\delta}(0) = i(\dot{\gamma}(0)) \]
und folgende ONB entlang $\delta$ durch
\[ e_i'(t) := \phi_t(e_i(t)) \]
und das Vektorfeld $K$ entlang $\delta$ durch
\[ K(t) := \phi_t(J(t)) = \sum_{i = 1}^n y_i(t)e_i'(t) \]
Nun gilt aufgrund unserer Voraussetzung $\phi_t^*g(\Rc,) = g(\Rc,)$
\begin{align*}
&  \forall j:~ y_j''(t) + \sum_{i = 1}^ny_i(t)g(e_j(t), \Rc (e_n(t), e_i(t) ) e_n(t)) = 0 \\
\Gdw{} &  \forall j:~ y_j''(t) + \sum_{i = 1}^ny_i(t)g(e'_j(t), \Rc (e'_n(t), e'_i(t) ) e'_n(t)) = 0\\
\Gdw{} & \sum_{i = 1}^{n} y_i''(t) e'_i(t) + \sum_{i = 1}^ny_i(t)\Rc (e'_n(t), e'_i(t) ) e'_n(t) = 0\\
\Gdw{} &\D_t\D_tK(t) + \Rc( \dot{\delta}(t), K(t)) \dot{\delta}(t) = 0
\end{align*}
Ergo ist $K$ ein Jacobifeld.
\item Da Paralleltransporte isometrisch sind, gilt
\[ g(J(l), J(l)) = g(\phi_t\circ J(l),\phi_t\circ J(l)) = g(K(l), K(l)) \]
\item Da $v = J(l) \in T_{p'}M$ beliebig war, bleibt nun folgendes zu zeigen
\[ K(l) = \d f_{p'}(J(l)) \]
Es gilt
\begin{align*}
\dot{K}(0) &= (\D_tK)(0) \\
&= \frac{\d}{\d t} {P^N_t}\i(K(t))_{|t = 0}\\
&= i\klam{ \frac{\d}{\d t} {P^M_t}\i (J(t))_{|t = 0} }\\
&= i(\D_tJ)(0) = i\dot{J}(0) 
\end{align*}
und
\begin{align*}
K(l) &= (\d \exp_q)_{l\dot{\delta}(0)} (l\dot{K}(0))\\
&= ((\d\exp_q)_{l\dot{\delta}(0)}\circ i)(l\dot{J}(0))\\
&= (\d \exp_q)_{l\dot{\delta}(0)} \circ i \circ (\d \exp_p)\i_{l\dot{\gamma}(0)} (J(l))\\
&= \d f_{p'}(J(l))
\end{align*}
\end{enumerate}
\end{Beweis}

% % %Ls sechste
\newpage
\Kor{}
Seien $M,N$ Riemannsche Mannigfaltigkeiten derselben konstanten Schnittkrümmung und Dimension. Dann existiert für jede Wahl von Punkten $p\in M, q\in N$ und ONBs $e_i \in T_pM$ und $f_i \in T_qN$ eine lokale Isometrie
\begin{align*}
g : p\in V\subset_oM &\Pfeil{} g(V)
\end{align*}
mit
\[ g(p) = q \text{ und } \d g_p(e_i) = f_i \]

\Def{}
Für $p,q \in M$ bezeichne $P(p,q)$ die Menge aller \textbf{stetigen Wege} $c : [0,1] \pfeil{} \R$ von $p$ nach $q$. Sind $a,b \in P(p,q)$ so verstehen wir unter einer \df{Homotopie} von $a$ nach $b$ eine \textbf{stetige} Abbildung
\[ H : I^2 \Pfeil{} M \]
mit
\begin{align*}
H_s(0) &= p~~ \forall s\\
H_s(1) &= q~~ \forall s\\
H_0(t) &= a(t)~~ \forall t\\
H_1(t) &= b(t)~~ \forall t\\
\end{align*}
Existierte eine Homotopie $a\mapsto b$, so heißen $a,b$ zueinander \df{homotop}.\\
Homotop Sein ist eine Äquivalenzrelation von $P(p,q)$.\\
Definiere ferner für $a \in P(p,q), b \in P(q,r)$
\begin{align*}
a * b(t) := \left\lbrace
\begin{aligned}
a(2t) && t \leq \frac{1}{2}\\
b(2t - 1) && t \geq \frac{1}{2}
\end{aligned}
\right.
\text{ und }
a\i(t) := a(1-t)
\end{align*} 
Diese Operationen sind unter Homotopie invariant.

\Def{Die X.te Definition der Fundamentalgruppe...}
Definiere die \df{Fundamentalgruppe} von $M$ in $p$ durch
\begin{align*}
\pi_1(M,p) := \klam{P(p,p) / \text{Homotopie}, *}
\end{align*}

\Bem{}
Ist $M$ wegzusammenhängend, so ist die Fundamentalgruppe bis auf Isomorphie unabhängig von der Wahl des Basispunktes.

\Def{}
$M$ heißt einfach zusammenhängend, wenn $M$ wegzusammenhängend ist und eine triviale Fundamentalgruppe hat.

\Def{}
Eine glatte, surjektive Abbildung von Riemannschen Mannigfaltigkeiten $M' \pfeil{} M$ heißt \df{Überlagerung}, falls jeder Punkt $p \in M$ eine Umgebung $U \subset_o M$ besitzt, sodass das Urbild dieser Umgebung ausschließlich aus Zusammenhangkomponenten besteht, die diffeomorph zu $U$ sind.

\Lem{}
Eine Überlagerung von Riemannschen Mannigfaltigkeiten $\varphi : M' \pfeil{} M$ gehorcht folgender \df{Universellen Abbildungseigenschaft}:\\
Ist $\alpha : I \pfeil{} M$ eine glatte Kurve mit Anfangspunkt $p$, so existiert für jedes Urbild $q \in \varphi\i(p)$ genau eine \df{Liftung} $\alpha' : I \pfeil{} M'$ mit Anfangspunkt $q$ und
\[ \varphi \circ \alpha' = \alpha \]

\Lem{}
Sei $M' \pfeil{} M $ eine Überlagerung und $a,b : I \pfeil{} M$ homotope Kurven in $M$. Besitzen ihre Liftungen $a', b' : I \pfeil{} M'$ denselben Anfangspunkt, so sind diese ebenfalls homotop.

\Kor{}
Sei $\varphi : M' \pfeil{} M$ eine Überlagerung von Riemannschen Mannigfaltigkeiten.\\
Ist $f : P \pfeil{} M$ eine glatte Abbildung, $P$ wegzusammenhängend und $p \in f(P)$, so existiert für jedes Urbild $q \in \varphi\i(p)$ höchstens eine \df{Liftung} $f' : P \pfeil{} M'$ mit Anfangspunkt $q$ und
\[ \varphi \circ f' = f \]
Ist $P$ einfach zusammenhängend, so existiert diese Liftung immer.

\Satz{}
Jede zusammenhänge Mannigfaltigkeit besitzt eine einfach zusammenhängende Überdeckung.
\begin{Beweis}{Skizze}
Für $p \in M$ sei $\Omega_p$ die Mannigfaltigkeit aller Pfade in $M$, die in $p$ starten. Setze $M' := \Omega_p / \text{Homotopie}$, dann ist folgende Überdeckung gegeben
\begin{align*}
M' & \Pfeil{} M\\
[\gamma]_H & \longmapsto \gamma(1)
\end{align*}
\end{Beweis}

\Kor{}
Ist $M$ eine zusammenhängende Riemannsche Mannigfaltigkeit, so gibt es bis auf Diffeomorphie nur eine einfach zusammenhängende Überlagerung $M' \pfeil{} M$. In diesem Sinne nennen wir $M'$ die \df{universelle Überlagerung} von $M$.

\Def{}
Eine Überlagerung $M' \pfeil{} M$ heißt \df{trivial}, wenn das Urbild jeder Zusammenhangskomponente von $M$ ausschließlich aus Zusammenhangskomponenten besteht, die jeweils diffeomorph zu ihrem Bild sind.

\Kor{}
Jede Überlagerung einer einfach zusammenhängenden Mannigfaltigkeit ist trivial.

\Def{}
Sei $k : M' \pfeil{} M$ eine Überlagerung. Unter einer \df{Deck-Transformation} verstehen wir einen Diffeomorphismus $\phi : M' \pfeil{} M'$ mit
\[ k \circ \phi = k \]
Bezeichnet $D$ die Gruppe aller Deck-Transformationen, so heißt $k$ \df{normal}, wenn für alle $p,q \in M'$ mit $k(p) = k(q)$ ein $\phi \in D$ mit $\phi(p) = q$ existiert.

\Kor{}
Jede einfach zusammenhängende Überlagerung ist normal.

\Satz{}
Ist $k: M' \pfeil{} M$ eine universelle Überlagerung, so ist $D \isom{} \pi_1(M)$.

\Def{}
Sei $G \subset \text{Diffeo}(M)$ eine Untergruppe. Wir sagen $G$ agiert \df{eigentlich diskontinuierlich (und frei)}, falls gilt
\begin{enumerate}[PD1)]
\item Jedes $p \in M$ hat eine Umgebung $U \subset M$, sodass für alle $g \in G$ gilt
\[ g.U \cap U \neq \emptyset \Impl{} g = 1 \]
\item Punkte $p,q \in M$ aus verschiedenen $G$-Orbiten haben Umgebungen $U,V\subset M$, sodass für alle $g \in G$ gilt
\[ g.U \cap V = \emptyset \]
\end{enumerate}

\Bem{}
Die Deck-Transformationsgruppe ist immer eigentlich diskontinuierlich.

\Lem{}
Sei $G \subset \text{Diffeo}(M)$ eine eigentlich diskontinuierliche Untergruppe. Es gibt genau eine Struktur auf $M/G$, durch die
\[ M \Pfeil{} M/G\]
zu einer Überlagerung (Riemannscher) Mannigfaltigkeiten wird. Ist $M$ zusammenhängend, so gilt $D = G$ und die Überlagerung ist normal.

\newpage
\chapter{Ws Vorlesungen}
% % % Vorlesung 18.01.17
\section{Symmetrische Räume}
\Def{Lokal Symmetrisch}
Eine zusammenhängende Riemannsche Mannigfaltigkeit $M$ heißt \df{lokal symmetrisch}, falls für jeden Punkt $P \in M$ eine Zahl $r > 0$ und eine Isometrie
\[ s_p : B_p(r) \Pfeil{} B_p(r) \]
sodass
\begin{enumerate}[1.)]
\item $s_p(p) = p$
\item $\klam{\d s_p}_{|p} = - \id{T_pM}$
\end{enumerate}

\Bem{}
Für jede Isometrie $L : T_pM \pfeil T_pM$ lässt sich lokal eine Abbildung durch
\[ l_p := \exp_p \circ L \circ \exp_{p}\i \]
definieren, die im Allgemeinem keine Isometrie ist.

\Satz{}
\label{180117}
Für eine zusammenhängende Riemannsche Mannigfaltigkeit $M$ sind folgende Aussagen äquivalent:
\begin{enumerate}[1.)]
\item $M$ ist lokal symmetrisch
\item der Krümmungstensor ist \df{parallel}, d.\,h., $\nabla \Rc = 0$, d.\,h., für alle $X,Y,Z,U \in \T M$
\[ \nabla_U \Rc(X,Y)Z = 0 \]
\item Für jede Geodäte $c$ ist die Abbildung
\[ \Rc_{\dot{c}}X := \Rc(X, \dot{c})\dot{c} \]
parallel entlang $c$.
\end{enumerate}
\begin{Beweis}{}
\begin{enumerate}
\item $1.) \impl{} 2.)$\\
Betrachte $s_p^*(\nabla \Rc) = (\nabla \Rc) \circ \d s_p$. Da $s_p$ eine Isometrie ist, gilt
\[ s_p^*(\nabla \Rc) = \nabla \Rc \]
Ferner gilt
\[ - \nabla_U \Rc(X,Y) Z = \d s_p (\nabla_U \Rc(X,Y) Z) = s_p^*(\nabla_U \Rc(X,Y) Z) \]
Es folgt ergo
\[ \nabla \Rc = 0 \]
\item $3.) \impl{} 1.)$\\
Sei $p \in M$, $r < inj_p(M)$.
\[ s_p := \exp_p \circ (-\id{T_pM}) \exp_p\i : B_p(r) \Pfeil{} B_p(r) \]
Wir zeigen durch Jacobi-Felder, dass $s_p$ isometrisch ist:\\
Sei dazu $E_1, \ldots, E_n$ eine Basis paralleler Vektorfelder entlang $c$, die Eigenvektoren von $\Rc_{\dot{c}}$ an der Stelle $p$ sind, d.\,h.
\[ \klam{\Rc_{\dot{c}}(E_i)} (p) = \lambda_i E_i(p) \]
Da diese Vektorfelder parallel entlang $c$ sind, bleiben sie Eigenvektoren entlang $c$ mit konstanten Eigenwerten $\lambda_i$, da $\Rc$ parallel entlang $c$ ist. Definiere folgendes Jacobi-Feld entlang $c$
\[ J(t) = \sum_{i = 1} y_i(t) E_i(t) \]
mit
\[ y_i''(t) = - \lambda_i y_i(t) \]
Da $J(0) = 0$, gilt
\[ y_i(t) = y_i(-t) \]
Für $J$ und $q = \exp_p(v)$ gilt nun
\[ \norm{ (\d s_p)_q (J(t)) } = \norm{ (\d s_p)_q \klam{ (\d \exp_p)_v (tJ'(0)) } } = \norm{ (\d \exp_p)_v (-tJ'(0)) } = \norm{J(-t)} = \norm{J(t)}  \]
Da diese Gleichung für alle Jacobi-Felder in $p$ gilt, folgt für alle $w \in T_qM$
\[ \norm{(\d s_p)_q w} = \norm{w} \]
ergo ist $s_p$ isometrisch.
\end{enumerate}
\end{Beweis}

\Bem{}
Analog gilt für eine Isometrie $L : T_pM \pfeil{} T_pM$ mit
\[l_p = \exp_p \circ L \circ \exp_p\i \]
folgende Implikation
\[ L^*(\Rc_p) = \Rc_p \Impl{} l_p \text{ ist isometrisch} \] % % L^*(\nabla\Rc_p) = \nabla\Rc_p ?

\Def{}
Ein lokal symmetrischer Raum heißt \df{(global) symmetrisch}, falls jede punktweise Spiegelung zu einer globalen Isometrie fortgesetzt werden kann.

\Satz{}
Folgende Aussagen sind für eine einfach zusammenhängende Riemannsche Mannigfaltigkeit $M$ äquivalent:
\begin{enumerate}[1.)]
\item $M$ ist lokal symmetrisch
\item $M$ ist symmetrisch
\item Jede lineare Isometrie
\[L : T_pM \Pfeil{} T_qM \]
die den Krümmungstensor erhält, d.\,h.
\[ L^*\Rc_q = \Rc_p \]
wird durch eine eindeutig bestimmte globale Isometrie induziert.
\end{enumerate}

\Kor{}
Die universelle Überdeckung einer lokal symmetrischen Mannigfaltigkeit ist symmetrisch.

\Bem{}
Noch allgemeiner kann man durch den Beweis von Satz \ref{180117} Folgendes zeigen:\\
Ist $L : T_pM \pfeil{} T_q N$ eine lineare Isometrie von Tangentialräumen lokal symmetrischer Mannigfaltigkeiten mit der Eigenschaft
\[ L^*\Rc_q =\Rc_p \]
so ist
\begin{align*}
f : B_p(r) & \Pfeil{} B_q(r)\\
x & \longmapsto \exp_q \circ L \circ \exp_p\i(x)
\end{align*}
eine Isometrie für $r > 0$ klein genug.

\Lem{}
Sind $M, N$ lokal symmetrisch, $M$ einfach zusammenhängend und $N$ vollständig, so lässt sich die Abbildung $f$ von obiger Bemerkung fortsetzen zu folgender lokalen Isometrie
\[ f : M \Pfeil{} N \]
\begin{Beweis}{}
\begin{enumerate}[1.)]
\item Setze $f : B_p(r) \pfeil{} B_q(r) $ durch Wege $c : p \mapsto x$ fort für alle $x \in M$
\item Zeige, dass die Fortsetzung $f(x)$ ausschließlich vom Homotopietyp des Weges $c$ abhängt
\item Da $M$ einfach zusammenhängend ist, ist die Fortsetzung von $f$ wohldefiniert 
\end{enumerate}

Zu 1.)\\
Setze $I = \{ s \in [0,1] ~|~ f \text{ kann fortgesetzt werden auf }c([0,s]) \}$. Dann ist $I$ offen in $[0,1]$, da der Definitionsbereich von $f$ immer offen ist. Es bleibt also die Abgeschlossenheit zu zeigen, d.\,h.
\[ S:= \sup I \in I \]
Sei $(s_i)_i \in I$ eine gegen $S$ konvergente Folge. Dann konvergiert
\[ f(c(s_i)) \pfeil{} q \]
da $f$ lokal isometrisch ist. Setze $r > 0$ so, dass $B_{c(s)}(3r), B_{q}(3r)$ normale Nachbarschaften sind. Dann existiert ein $s \in I$, sodass
\[ c(s) \in B_{c(S)}(r) \text{ und } f(c(s)) \in B_q(r) \]
Dann setzt sich aber $f$ fort auf $B_{c(S)}(2r)$, ergo liegt $S \in I$.\\

Zu 2.)\\
Ähnliches Argument. Sei $H : [0,1]^2 \pfeil{} M$ eine Homotopie und $f^s$ die Fortsetzung entlang $H(s,[0,1])$. Setze
\[ I = \set{s \in [0,1]}{ \forall\alpha \in [0,s]: f^\alpha (1) = f^s(1) } \]
und zeige analog, dass $I$ abgeschlossen ist.
\end{Beweis}

% % Vorlesung 20.01.17
\Prop{}
Ist $M$ symmetrisch, so ist $M$ vollständig.

\Def{}
Eine Isometrie $f$ von $M$ heißt \df{Transvektion}, falls ein Punkt $p \in M$ und ein Weg $c : p \mapsto f(p)$, sodass $\d f_p : T_pM \pfeil{} T_{f(p)}M$ der Paralleltransport entlang $c$ ist.

\Prop{}
Ist $M$ symmetrisch, so existiert für jedes Paar von Punkten $p,q \in M$, eine Transvektion, die $p$ auf $q$ abbildet.

\Kor{}
Ist $M$ symmetrisch, so ist jede vollständige Geodäte ist der Orbit einer 1-Parameter Untegruppe von $Isom(M)$.
\begin{Beweis}{}
Sei $\gamma : \R \pfeil{} M$ vollständig geodätisch. Definiere
\[ \tau_s := s_{\gamma(\frac{s}{2})} \circ s_{\gamma(0)} \]
Dann ist
\begin{align*}
\R &\Pfeil{} Isom(M)\\
s & \longmapsto \tau_s
\end{align*}
ein Gruppenhomomorphismus.
\end{Beweis}

\Kor{}
Sei $M$ symmetrisch.\\
Bezeichnet $G = Isom(M)^0$ die Zusammenhangskomponente der Eins in $Isom(M)$, so agiert $G$ transitiv auf $M$.\\
Ferner ist die Untergruppe
\[ K = Stab_G(m) = \set{g \in G}{gm = m} \]
kompakt. Ferner ist folgender Homöomorphismus gegeben
\begin{align*}
G / K &\Pfeil{} M\\
gK &\longmapsto g.m
\end{align*}
Fasst man $G/K$ als Lie-Gruppe auf, so handelt es sich hierbei sogar um einen Diffeomorphismus.

\Bsp{}
Es sei $M$ die Mannigfaltigkeit der symmetrisch positiv definiten reellen $n\times n$-Matrizen. $M$ besitzt dann eine Dimension von $\frac{n(n+1)}{2}$.\\
Für $p \in M$ gilt
\begin{align*}
T_pM &= \text{Symm}(n,\R)\\
g_p(A,B) &=spur(p\i A p\i B)
\end{align*}
Die Symmetrie gestaltet sich nun durch
\begin{align*}
S_p : M &\Pfeil{} M\\
A &\longmapsto pA\i p
\end{align*}
$GL(n,\R)$ agiert auf $M$ durch
\begin{align*}
g.p := gpg^T
\end{align*}
$G = GL(n,\R)^+, K = SO(n,\R)$

\Def{}
Eine \df{Lie-Algebra} $V$ ist ein reeller Vektorraum mit einer Bilinearform
\[ [,] : V\otimes_\R V \Pfeil{} V \]
mit
\begin{enumerate}[1.)]
\item $[X,Y] = -[Y,X]$
\item \df{Jacobi-Identität}: $[X,[Y,Z]] + [Z,[X,Y]] + [Y,[Z,X]] = 0$
\end{enumerate}

\newcommand{\g}{\mathfrak{g}}
\Bem{}
Sei $G$ eine Lie-Gruppe, setze
\[ \g = T_1G \]
Dann ist $\g$ isomorph zum Raum der \df{links-invarianten Vektorfelder} auf $G$, d.\,h. der Vektorfelder $X$, für die gilt
\[ X(gp) = \d g_{|p}X(p) \]
Ergo können wir die Lie-Klammer der Vektorfelder auf $\g$ einschränken, wodurch $\g$ zu einer Lie-Algebra wird.\\
Für $x\in\g$ definieren wir die \df{Adjunktion} durch
\begin{align*}
ad(x) : \g & \Pfeil{} \g\\
v & \longmapsto [x,v]
\end{align*}
Und die \df{Killing-Form} durch
\begin{align*}
B : \g \otimes \g &\Pfeil{} \g\\
x \otimes y & \longmapsto spur(ad(x) \circ ad(y))
\end{align*}
Beachte, für jedes $x \in \g$ existiert genau ein glatter Gruppenmorphismus
\[ \Theta : \R \Pfeil{} G \]
sodass $\Theta'(0) = x, \Theta(0) = 1$.\\
Wir definieren hierdurch folgende Exponentialabbildung
\begin{align*}
\exp : \g &\Pfeil{} G\\
x &\longmapsto \Theta(1)
\end{align*}
Dann existiert ein Zusammenhang auf $G$, sodass $\exp$ der Riemannschen Exponential-Abbildung entspricht und jedes $\Theta$ eine Geodäte ist.\\
Ist $h\in G$, so definiere
\begin{align*}
Int(h) :G& \Pfeil{} G\\
g & \longmapsto hgh\i
\end{align*}
und die \df{adjungierte Aktion} durch
\[ Ad(h) := \d Int(h)_{|1} : \g \Pfeil{} \g \]
Dann gilt
\begin{align*}
\exp(Ad(h)x) &= h\exp(x)h\i\\
Ad(\exp(x)) = e^{ad(x)}
\end{align*}

\Bem{}
Ist $G = GL(n,\R)$, so ist $\g = \R^{n\times n}$, $B(x,y) = c \cdot spur(xy)$ und $[x,y] = xy -yx$. 

\Bem{}
Sei $M$ ein symmetrischer Raum, $G = Isom(M)^0$. Definiere folgende Involution
\begin{align*}
\sigma : G &\Pfeil{} G\\
g &\longmapsto s_mgs_m
\end{align*}
Dann ist auch folgende Abbildung eine Involution für $m \in M$
\[ Ad(s_m) = \d\sigma_{|1} : \g \Pfeil{} \g \]
Wir erhalten dann folgende \df{Aktionszerlegung}
\begin{align*}
\g &= K \oplus P\\
&= \{ x \in \g ~|~ \d\sigma_{|1}x = x \} \oplus \{ x \in \g ~|~ \d\sigma_{|1}x = -x \}
\end{align*}
Dabei ist $K \subset \g$ eine Lie-Subalgebra.
\end{document}