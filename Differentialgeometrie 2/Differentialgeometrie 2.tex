\documentclass{article}

\usepackage{../Package/latexa}
\usepackage{../Package/algebra}
\usepackage{../Package/theorema}
\usepackage{../Package/diagramma}
\usepackage{../Package/categoria}

\usepackage{tikz}
\usepackage{tikz-cd}
\usetikzlibrary{arrows}

\newcommand{\qi}{\backsimeq_{\textsc{QI}}}
\newcommand{\ba}{\backsimeq_{\textsc{BA}}}
\newcommand{\normal}{\vartriangleleft}
\newcommand{\tm}{\subset}
\newcommand{\Stab}[2]{\textsf{Stab}_{#1}(#2)}
\newcommand{\Cay}[2]{\textsf{Cay}(#1,#2)}


\renewcommand{\A}{\mathbb{A}}
\newcommand{\Nc}{\mathcal{N}}

\renewcommand{\i}{^{-1}}
\renewcommand{\d}{\textsf{d}}
\renewcommand{\P}{\mathbb{P}}

\newcommand{\af}{\mathfrak{a}}
\newcommand{\Af}{\mathfrak{A}}
\renewcommand{\bf}{\mathfrak{b}}
\newcommand{\Bf}{\mathfrak{B}}
\newcommand{\cf}{\mathfrak{c}}
\newcommand{\Cf}{\mathfrak{C}}
\newcommand{\ff}{\mathfrak{f}}
\newcommand{\Ff}{\mathfrak{F}}
\newcommand{\pf}{\mathfrak{p}}
\newcommand{\Pf}{\mathfrak{P}}
\newcommand{\mf}{\mathfrak{m}}
\newcommand{\Mf}{\mathfrak{M}}
\newcommand{\qf}{\mathfrak{q}}
\newcommand{\Qf}{\mathfrak{Q}}

\renewcommand{\M}{\mathbb{M}}

\newcommand{\Ac}{\mathcal{A}}
\newcommand{\Rc}{\mathcal{R}}
\renewcommand{\O}{\mathcal{O}}

\newcommand{\Frob}{\textsf{Frob}}

\newcommand{\Leg}[2]{\left(\frac{#1}{#2}\right)}

\newcommand{\ric}{\textsf{ric}}

\makeindex

\begin{document}

\section{Krümmungen}

\Def{}
Es sei $M$ eine Riemannsche Mannigfaltigkeit und $\kappa_p(\sigma)$ bezeichne die Schnittkrümmung in einem Punkt $p$ und einer Ebene $\sigma \subset T_pM$.\\
$M$ heißt \df{von positiver Schnittkrümmung}, falls $\kappa_p(\sigma) > 0$ für alle $p \in M, \sigma \subset T_pM$.
Bezeichnet $\ric$ die Ricci-Krümung, so heißt $M$ \df{von positiver Ricci-Krümmung}, falls $\ric_p(v,v) > 0$ für alle $p \in M, v \in T_pM$.\\
Die Ricci-Krümmung von $M$ wird von $\alpha > 0$ nach unten beschränkt, falls
\[ \ric_p(v,v) \geq \alpha g_p(v,v) \]
für alle $p \in M, v \in T_pM$.

\Def{Isometrien}
Eine glatte Abbildung $f : M \pfeil{} N$ heißt \df{lokal isometrisch}, falls für alle $p \in M$ das Differential $\d_p : T_pM \pfeil{} T_pN$ eine lineare Isometrie metrischer Räume ist.

\Bem{}
$f$ ist genau dann isometrisch, wenn $f$ lokal isometrisch und diffeomorph.

\Satz{}
Jede glatte Abbildung $f : M \pfeil{} M$, die für einen Punkt eine Isometrie auf den Tangentialräumen induziert, ist eine Isometrie.\\
Jede glatte, abstandserhaltende Funktion $f : M \pfeil{} M$ ist eine Isometrie.

\Def{}
Die \df{Kompakt-Offen-Topologie} auf $\textsf{Isom}(M,M)$ wird durch eine Basis folgender Mengen generiert
\[ W(K,U) := \set{f}{ f(K) \subseteq U } \]
für alle $K \subseteq M$ kompakt und $ U \subseteq M$ offen.

\Prop{Cartan}
Seien $v,w$ normierte, orthogonale Vektoren in $T_pM$. Es gilt
\[ d(c_v(t), c_w(t))^2 = 2t^2 - \frac{\kappa_p(\sigma_{v,w})}{6} t^4 + o(t^5)  \]

\Lem{}
Ist $Y$ ein Jacobifeld entlang $c$, so gilt
\[ D_t(\Rc (\dot{c}, Y), \dot{c})(0) = \Rc (\dot{c}, D_tY)\dot{c}  \]
\begin{Beweis}{}
\begin{align*}
\frac{\d}{\d t} g(\Rc(\dot{c} , Y) \dot{c}, w ) &= g(D_t \Rc(\dot{c}, Y) \dot{c}, w ) + g(\Rc (\dot{c}, Y) \dot{c}, D_tw )\\
\frac{\d}{\d t} g(\Rc(\dot{c} , w) \dot{c}, Y ) &= g(D_t \Rc(\dot{c}, w) \dot{c}, Y ) + g(\Rc (\dot{c}, w) \dot{c}, D_tY )\\
g(D_t \Rc(\dot{c}, Y) \dot{c}, w ) &= g(D_t \Rc(\dot{c}, w) \dot{c}, Y ) + g(\Rc(\dot{c}, w) \dot{c}, D_tY ) =  g(\Rc(\dot{c}, D_tY) \dot{c}, w )
\end{align*}
\end{Beweis}


\Prop{}
Seien $v,w \in T_pM$ mit $\bet{w} = 1$. Setze $Y(t) = (\d \exp_p)_{tv}(tw) \in J_{c_v}$. Dann gilt
\[ \bet{Y(t)}^2 = g(Y(t), Y(t)) = t^2 - \frac{1}{3} g(\Rc(v,w)w, v)t^4 + o(t^5) \]

\begin{Beweis}{}
Es gilt
\[ Y(0) = 0, D_t(Y)(0) = w \]
Man rechnet nach
\begin{align*}
	g(Y,Y)(0)     & = 0                                                                    \\
	g(Y,Y)'(0)    & = \frac{\d}{\d t}_{|t = 0} g(Y(t), Y(t)) = 2g(D_t(Y),Y) (0) = 0        \\
	g(Y,Y)''(0)   & = 2g(D_tY, D_tY)(0) + 2g(D_tD_tY,Y)(0) = 2g(w,w) = 2                   \\
	g(Y,Y)'''(0)  & = 6g(D_tD_tY, D_tY)(0) = 6g(-\Rc(\dot{c}, Y) \dot{c}, D_t Y ) (0) = 0  \\
	g(Y,Y)''''(0) & = 8g(\Rc(\dot{c}, D_tY )\dot{c}, D_tY  ) - \frac{8}{24} g(\Rc(v,w)w,v) = - \frac{1}{3}g(\Rc(v,w)w,v)
\end{align*}
\end{Beweis}

\newpage
\section{Räume konstanter Krümmung}
\Def{}
Unter $\M_k^n$ verstehen wir die bis auf Isomorphie eindeutige vollständige, einfache zusammenhängende, $n$-dimensionale Riemannsche Mannigfaltigkeit von konstanter Schnittkrümmung $k$.
\Bem{}
Skaliert man die Metrik einer Riemannschen Mannigfaltigkeit mt $c > 0$, so skaliert sich ihre Krümmung mit $\frac{1}{c}$.
\Bsp{}
Man führe auf $\R^{n+1}$ folgende Bilinearformen ein
\begin{align*}
(x,y) &:= \sum_{i = 1}^{n+1}x_i \cdot y_i\\
\shrp{x,y} &:= \sum_{i = 1}^{n}x_i \cdot y_i - x_{n+1} \cdot y_{n+1}\\
\end{align*}
Dann gilt
\begin{align*}
\M_0^n = \R^n && M_1^n = S^n = \{x \in \R^{n+1} ~|~ (x,x) = 1\} && M_{-1}^n = \H^n = \{ x \in \R^{n+1} ~|~ (x,x) = 1, x_{n+1} > 0 \} 
\end{align*}

\Def{}
Der \df{Index} einer symmetrischen Bilinearform ist die größte Dimension aller Unterräume, auf denen jene Bilinearform negativ definit ist.

\Def{}
Sei $(M,g)$ eine semi-Riemannsche Untermannigfaltigkeit einer Riemannschen Mannigfaltigkeit $(M',g')$. Bezeichnet $\iota : M \inj{} M'$ die Einbettung, so verstehen wir unter $\V'(M) := \Gamma(\iota^*(TM'))$ den Pullback der Vektorfelder auf $M'$, d.h.
\[ \V'(M) = \set{Y_{|M}}{Y \in \V(M)} \]
Unter $\V(M)^\bot \subset \V'(M)$ verstehen wir die Vektorfelder auf $M$, die normal zu $M$ liegen.\\
Wir schränken den Levi-Civita-Zusammenhang von $M'$ auf $\V(M) \times \V'(M)$ ein und erhalten
\[ \D' : \V(M) \times \V'(M) \Pfeil{} \V'(M) \]

\Lem{}
$\D'$ ist wohldefiniert.
\begin{Beweis}{}
Sei $p \in M$ ein Punkt und $U \subset M$ eine Koordinatennachbarschaft von $p$. Seien ferner $V \in \V(M), X \in \V'(M)$ mit Liftungen $V',X' \in \V(M')$. $X'$ lässt sich auf $U$ darstellen durch
\[ X' = \sum_{i} f^i \frac{\d}{\d x_i} \]
\[ \D'_{V'}X' = \sum_i V'(f^i) \frac{\d}{\d x_i} + \sum_i f^i D'_{V'}\klam{\frac{\d}{\d x_i}} \]
Für $q \in U\cap M$ gilt
\[V'(f^i)(q) = V_q(f^i) = V_q(f^i_{|U\cap M}) \]
und
\[ \D'_{V'}\klam{\frac{\d}{\d x_i}}(q) = \D'_{V'_q}\klam{\frac{\d}{\d x_i}} \]
\end{Beweis}
\Kor{}
Seien $V,W \in \V(M), X,Y \in V'(M)$.
\begin{itemize}
\item $D'_VX$ ist $\F(M)$-linear in $V$.
\item $D'_VX$ ist $\R$-linear in $X$.
\item $D'_V(fX) = (Vf) X + f\D'_VX$
\item $[V,W] = \D'_VW - \D'_WV$
\item $Vg(X,Y) = g(D'_VX, Y) + g(X, D'_VY)$
\end{itemize}
\Lem{}
Sind $V,W \in \V(M)$, dann gilt
\[ \D_VW = (\D'_VW)^T \]
\begin{Beweis}{}
Sei $X \in \V(M)$ beliebig
\begin{align*}
2g(\D'_{V'}W', X') = V'g(W',X') + W'g(X',V') - X'g(V',W') + g(W',[X',V']) + g(X', [V',W'])
\end{align*}
Mit dem Vorhergehenden Korollar erhalten wir durch Einschränkung auf $M$
\[ g(\D'_{V}W, X) = g(\D_{V}W, X) \]
Es gilt ferner
\[ g(\D'_{V}W, X) = g((\D'_{V}W)^T, X) \]
da $X$ tangential an $M$ liegt.
\end{Beweis}

\Lem{}
Definiere den \df{Gestalttensor} bzw. \df{zweiten Fundamentaltensor} durch
\begin{align*}
\Pi : \V(M) \times \V(M) &\Pfeil{} \V(M)^\bot\\
(V,W) & \longmapsto (\D'_VW)^N = \D'_VW - \D_VW
\end{align*}
Dann ist dieser Tensor $\F(M)$-bilinear und symmetrisch.
\begin{Beweis}{}
\begin{align*}
(\D'_V(fW))^N = (f\D'_VW)^N + (V(f) W)^N = (f\D'_VW)^N\\
\Pi(V,W) - \Pi(W,V) = (\D'(V,W) - \D'(W,V))^N = [V,W]^N = 0
\end{align*}
\end{Beweis}

\Satz{Gauss-Gleichung}
Seien $V,W,X,Y \in \V(M)$. Es gilt
\[ g(\R(V,W)X,Y) = g(\R'(V,W)X,Y) + g(\Pi(V,X), \Pi(W,Y )) - g(\Pi(V,Y), \Pi(W,X )) \]
\begin{Beweis}{}
\begin{align*}
\R(X,Y)Z &= -\D_{[X,Y]}Z + [\D_X,\D_Y]Z\\
g(\D'_{[V,W]}X,Y) &= g(\D_{[V,W]}X,Y)\\
g(\D'_V\D'_WX,Y) &= g(\D'_V\D_WX,Y) + g(\D'_V \Pi(W,X), Y)\\
&= g(\D_V\D_WX,Y) + (Vg(\Pi(W,X), Y) - g(\Pi(W,X), \D'_VY ))\\
\end{align*}
Nun gilt aber
\[ g(\Pi(W,X), Y) = 0 \]
Ergo
\begin{align*}
g(\D'_V\D'_WX,Y) &= g(\D_V\D_WX,Y) + (Vg(\Pi(W,X), Y) - g(\Pi(W,X), \D'_VY ))\\
&= g(\D_V\D_WX,Y) -g(\Pi(W,X), \Pi(V,Y) )
\end{align*}
\end{Beweis}

\Kor{}
Sind $v,w \in T_pM$ linear unabhängig, so gilt
\[ \kappa(v,w) = \kappa'(v,w) + \frac{g(\Pi(v,v), \Pi(w,w)) - g(\Pi(v,w), \Pi(v,w))}{g(v,v) g(w,w) - g(v,w)^2} \]

\Prop{}
Sei $Y$ ein Vektorfeld tangential zu $M$ entlang einer Kurve $\alpha(t)$ in $M$. Dann gilt
\[ \D_t' Y = \D_tY + \Pi(\dot{\alpha}, Y) \]

\Kor{}
$\alpha$ ist genau dann eine Geodäte von $M$, wenn $\D'_t(\dot{\alpha})$ normal zu $M$ steht.

\Def{}
Eine \df{semi-Riemannsche Hyperfläche} ist eine \df{semi-Riemannsche Untermannigfaltigkeit} der Kodimension Eins.

\Def{}
Das \df{Vorzeichen} einer semi-Riemannschen Hyperfläche $M$ ist definiert durch
\[ \e := \left\lbrace 
\begin{aligned}
+1 && \textsf{coindex}M = 0\\
-1 && \textsf{coindex}M = 1
\end{aligned}
\right. \]

\Def{}
Sei $U$ ein normierter Normalenvektorfeld einer semi-Riemannschen Hyperfläche $M$. Dann wird durch
\[ g(S(V),W)  = g(\Pi(V,W), U)\]
ein (1,1)-Tensorfeld $S$ auf $M$ definiert, der sogenannte \df{Gestaltoperator}, der durch $U$ induziert wird.

\Lem{}
Für alle $v \in T_pM$ gilt
\[ S(v) = -\D'_vU \]
Ferner ist $S_p$ selbstadjungiert.
\begin{Beweis}{}
$g(U,U)$ ist konstant gleich 1. Ergo gilt
\[ g(\D'_VU,U) = 0 \]
Daraus folgt, dass $\D'_VU$ für alle $V \in \V(M)$ tangent an $M$ liegt.\\
Sei nun $W \in \V(M)$, dann gilt
\[ g(S(V),W) = g(\Pi(V,W), U) =g(\D'_VW, U) \gl{g(U,W) = 0} g(\D'_VU,W)   \]
\end{Beweis}

\Kor{}
\[ \kappa(v,w) = \kappa'(v,w) + \e \frac{g(Sv,v)g(Sw,w) - g(Sv,w)^2}{g(v,v) g(w,w) - g(v,w)^2} \]
\begin{Beweis}{}
\[ \Pi(v,w) = \e g(Sv,w) U \text{ und } g(U,U) = \e \]
\end{Beweis}

\Def{3 Shades of Artig-Sein}
Sei $x \in \R^{n+1}$.
\begin{align*}
x \text{ heißt } \left\lbrace \begin{aligned}
\text{zeitartig, falls } &\shrp{x,x} < 0\\
\text{lichtartig, falls } &\shrp{x,x} = 0\\
\text{raumartig, falls } &\shrp{x,x} > 0
\end{aligned} \right.
\end{align*}

\Prop{}
Sei $M = \set{p \in \R^{n+1}}{\shrp{p,p} = -r^2}$. Dann ist $M$ eine Riemannsche Mannigfaltigkeit der konstanten Schnittkrümmung $-\frac{1}{r^2}$ für $r > 0$.
\begin{Beweis}{}
Definiere $f : \R^{n+1} \pfeil \R$ durch $x \mapsto \shrp{x,x}$. Dann ist $\d f_p(v) = 2\shrp{v,p}$. Ergo ist $\d f_p$ surjektiv für alle $p \in M$. Ergo ist $f\i(-r^2)$ eine glatte Hyperfläche von $\R^{n+1}$.\\
Ferner ist
\[ T_pM = \Ker \d f_p  = \set{v \in \R^{n+1}}{\shrp{p,v} = 0} = p^\bot \]
Da $p$ zeitartig ist, ist $T_pM$ positiv definit. Ferner ist der vom Einheiten-Normalen Vektorfeld $U = \frac{p}{r}$ abgeleitete Gestaltoperator gegeben durch $S(V) = - \D'_VU = \frac{-V}{r}$. Es folgt
\[ \kappa(v,w) = 0 - \frac{1}{r^2} = - \frac{1}{r^2} \] 
\end{Beweis}

\Def{}
Unter einer \df{Paar-Isometrie} $\Phi : (M,M') \pfeil{} (N,N')$, wobei $M \subset M', N \subset N'$, ist eine Isometrie $M' \pfeil{} N'$, dergestalt, dass auch ihre Restriktion $M \pfeil{} N$ eine Isometrie ist.

\Lem{}
Eine Paar-Isometrie $\Phi : (M,M') \pfeil{} (N,N')$ erhält den Gestalttensor, d.\,h.
\[ \d \Phi (\Pi (v,w)) = \Pi ( \d \Phi(v),\d \Phi(w) ) \]
für alle $p \in M, v,w \in T_pM$.
\end{document}