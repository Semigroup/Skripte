\documentclass{book}

\usepackage[T1]{fontenc}
\usepackage[UTF8]{inputenc}
\usepackage{amsmath}
\usepackage{amsfonts}
\usepackage{amssymb}
\usepackage{stmaryrd}
\usepackage{lmodern}
\usepackage{ngerman}
\usepackage{xcolor}
\usepackage{geometry}
\usepackage[siunitx]{circuitikz}
\usepackage{tikz-cd}
\usepackage{makeidx}
\usepackage{hyperref}

\newcommand{\df}[1]{\textbf{#1}\index{#1}}
\newcommand{\Z}{\mathbb{Z}}
\newcommand{\Top}{\textbf{Top}}
\newcommand{\Mod}{\textbf{Mod}}

\begin{document}

%	13.05.15, 10.VL
\paragraph{Example}
$X = S^1\times [0,1]/\sim$, $\sim$ generated by
\[(z,0) \sim (e^{2\pi /n}z,0) ; (z,1) \sim (e^{2\pi /m}z,1) \]
identifying points that are  an angle $2\pi /n$, $2\pi/m$ apart\\
It follows:
\[top \cong S^1; bottom \cong S^1 \]
Pushout: (1)\\

The inclusions
\[top \hookrightarrow X_{T}, bottom \hookrightarrow X_B \]
are deformation retracts, in particular homotopy equivalences.\\

In the first case,
\[r: X_T \longrightarrow Top, [(z,t)] \longmapsto [(z,1)]\]
\[h: X_T\times [0,1]\longrightarrow X_T, ([(z,s)],t) \longmapsto [(z,s\cdot t)] \]
provides the data of a deformation retract. Similiar for bottom.\\

van Kampen yields a pushout of groups, when applying $\pi_1$ to (1).
(2)\\

How do the induced morphisms look like?\\
In the case of top:
\[r_* : \pi_1(X_T) \longrightarrow \pi_1(Top), \gamma \longmapsto \gamma^n \]m
One gains
\[\pi_1(S^1\times \{\frac{1}{2}\}) \longrightarrow \pi_1(X_T) \overset{\sim}{\longrightarrow} \pi_1(Top) \overset{\sim}{\longrightarrow} \pi_1(S^1) \cong \Z  \]
Be $\gamma$ a generator of $\pi_1(Top)$, $r$ applied to the generator of $\pi_1(S^1\times \frac{1}{2})$ wraps around $m$ times the top circle.\\

Because of (2) we obtain a group presentation
\[\pi_1(X) \cong \left\langle a,b~|~a^m = b^n \right\rangle \]
We have an epimorphism
\[\pi_1(X) \longrightarrow > \Z/m \star \Z /n \]
\[a \longmapsto 1_{\Z/m}\]
\[b \longmapsto 1_{\Z/n}\]

\newpage

\chapter{Homology - the axiomatic approach 2}

\section{2.1 The Eilenberg-Steenrod axioms}
Let $R$ be a commutative Ring.\\

A sequence of moprhisms of $R$-modules
\[M_{i+1} \overset{f_{i+1}}{\longrightarrow} M_i \overset{f_{i}}{\longrightarrow} M_{i-1} \overset{f_{i-1}}{\longrightarrow}\]
is called \df{exact} if
\[Kern f_i = im f_{i+1} \forall i\]

\subsubsection{Definition}
A \df{homology theory} $(H_*, \partial_*)$ with values in $R$-modules consists of a family $(H_n)_{n\in\Z}$ of functors
\begin{align*}
H_n : \Top^2 & \longrightarrow R-\Mod
\end{align*}
from the category of pairs of spaces to the category of $R$-modules and a family of natural transformations $(\partial_n)_{n\in \Z}$
\[\partial_n : H_n \longrightarrow H_{n-1} \circ J \]
where $J$ is the functor 
\[J : \Top^2 \longrightarrow \Top^2\]
\[(X,A) \longmapsto (X,\emptyset)\]
such that the following axioms are true
\begin{itemize}
\item \df{Homotopy invariance}\\
If $f,g : (X,A) \rightarrow (Y,B)$ are maps of pairs of spaces and $h_t : f \backsimeq g$ is a homotopy with $h_t(A) \subset B \forall t \in [0,1]$ then
\[H_n(f) = H_n(g)\]
\item \df{Long exact  sequence}\\
For every  pair $(X,A)$ the following sequence of $R$-modules is exact:
\[ \ldots\longrightarrow H_{n+1}(X,A)\overset{\partial_{n+1}(X,A)}{\longrightarrow }H_n(A,\emptyset)\overset{H_n(\iota)}{ \longrightarrow} H_n(X,\emptyset) \overset{H_n(j)}{ \longrightarrow}H_n(X,A) \longrightarrow H_{n-1}(A,\emptyset) \longrightarrow \ldots \]
is exact where
\[\iota : (A,\emptyset) \longrightarrow (X,\emptyset) \]
\[j : (X,\emptyset) \longrightarrow (X,A)\]
This maps $\partial_i(X,A)$ are called \df{boundary homomorphisms}.

\item \df{Excision axiom}\\
Let $X$ be a space and $A,B\subset X$ be a subspace such that $\overline{A} \subset B^o$. Then the $R$-homomorphisms
\[H_n(X\setminus A, B\setminus A) \longrightarrow H_n(X,B)\]
induced by
\[(X\setminus A, B\setminus A) \hookrightarrow (X,A)\]
is an isomorphism for all $n$.

\end{itemize}
If $(H_*, \partial_*)$ in addition satisfies the following, we say  $(H_*, \partial_*)$ satisfies the \df{dimension axiom}:
\[H_n(\{*\}, \emptyset) \cong R, if n = 0; 0 if n \neq 0\]

Notation: In the sequel we write $H_n(X)$ instead of $H_n(X,\emptyset)$.

\subsubsection{Remark}
In a nutshell, $(H_*, \partial_*)$ is the following:
\begin{itemize}
	\item $(X,A) \rightsquigarrow R-\text{Modules} H_n(X,A), n \in \Z$
	\item $(X,A) \overset{f}{\rightarrow} (Y,B) \rightsquigarrow H_n(X,A) \overset{H_n(f)}{\rightarrow} H_n(Y,B)$
	\item boundary homom. (3)
\end{itemize}

\subsubsection{Remark}
Long exact sequence for $(X,X)$
\[H_{n+1}(X,X) \overset{\partial_{n+1}}{\longrightarrow} H_n(X) \overset{\sim}{\longrightarrow} H_n(X) \overset{H_nj}{\longrightarrow} H_n(X,X)\overset{\partial_{n}}{\longrightarrow} H_{n-1}(X) \overset{\sim}{\longrightarrow} H_{n-1}(X)  \]

\[Kern id = im(\partial_n) = 0 \Longrightarrow \partial_n = 0 \]
\[im H_n(j) = Kern \partial_n = Hn(X,X)\]
\[Ker H_n(j) = im id = H_n(X) \]
Therefore $H_n(X,X) = 0$

\section{2.2 First conclusions from the axioms}
\subsubsection{Fünferlemma}
Consider the following commuting diagram of $R$-modules
(4)
such that both rows are exact and $f_1$ is surjective, $f_5$ is injective and $f_2$ and $f_4$ are isomorphisms. Then $f_3$ is an isomorphism.

\paragraph{Proof by Diagrammjagd}
$f_3$ is injective:
Let $x \in M_3$, if $f_3(x) = 0$

\end{document}