\documentclass{book}

\usepackage{../Package/latexa}
\usepackage{../Package/algebra}
\usepackage{../Package/theorema}
\usepackage{../Package/diagramma}
\usepackage{../Package/categoria}

\newcommand{\qi}{\backsimeq_{\textsc{QI}}}
\newcommand{\ba}{\backsimeq_{\textsc{BA}}}
\newcommand{\normal}{\vartriangleleft}
\newcommand{\tm}{\subset}
\newcommand{\Stab}[2]{\textsf{Stab}_{#1}(#2)}
\newcommand{\Cay}[2]{\textsf{Cay}(#1,#2)}


\renewcommand{\A}{\mathbb{A}}
\newcommand{\Nc}{\mathcal{N}}


\renewcommand{\d}{\textsf{d}}
\renewcommand{\P}{\mathbb{P}}

\newcommand{\af}{\mathfrak{a}}
\newcommand{\Af}{\mathfrak{A}}
\renewcommand{\bf}{\mathfrak{b}}
\newcommand{\Bf}{\mathfrak{B}}
\newcommand{\cf}{\mathfrak{c}}
\newcommand{\Cf}{\mathfrak{C}}
\newcommand{\ff}{\mathfrak{f}}
\newcommand{\Ff}{\mathfrak{F}}
\newcommand{\pf}{\mathfrak{p}}
\newcommand{\Pf}{\mathfrak{P}}
\newcommand{\mf}{\mathfrak{m}}
\newcommand{\Mf}{\mathfrak{M}}
\newcommand{\qf}{\mathfrak{q}}
\newcommand{\Qf}{\mathfrak{Q}}

\renewcommand{\M}{\mathbb{M}}
\renewcommand{\l}[1]{\overline{#1}}

\newcommand{\Ac}{\mathcal{A}}
\newcommand{\Rc}{\mathcal{R}}
\renewcommand{\O}{\mathcal{O}}

\newcommand{\Frob}{\textsf{Frob}}

\newcommand{\Leg}[2]{\left(\frac{#1}{#2}\right)}

\newcommand{\ric}{\textsf{ric}}
\newcommand{\g}{\mathfrak{g}}
\newcommand{\kf}{\mathfrak{k}}
\newcommand{\p}{\mathfrak{p}}
\newcommand{\m}{\mathfrak{m}}

\newcommand{\GL}{\textsf{GL}(n,\R)}
\newcommand{\glf}{\mathfrak{gl}(n,\R)}
\newcommand{\SO}{\textsf{SO}(n,\R)}
\newcommand{\so}{\mathfrak{so}(n,\R)}
\newcommand{\SL}{\textsf{SL}(n,\R)}
\newcommand{\slf}{\mathfrak{sl}(n,\R)}
\newcommand{\POS}{\textsf{Pos}(n,\R)}
\newcommand{\symm}{\textsf{symm}(n,\R)}
\newcommand{\of}{\mathfrak{o}}
\renewcommand{\epsilon}{\varepsilon}

\usepackage{enumerate}

\makeindex

\begin{document}

\title{
\begin{huge}
Geometrie der Mannigfaltigkeiten\\
\end{huge}
\begin{large}
Kurzskript, SS 17
\end{large}}


%\author{Ak\i n Ünal}
\maketitle
\renewcommand{\i}{^{-1}}

%Das folgende Kurzskript orientiert sich an einer Vorlesung, die im Wintersemester 2016 / 2017 in Heidelberg gehalten wurde. Für alle Fehler im Text trägt ausschließlich der Autor die Verantwortung.

\setcounter{tocdepth}{1}
\tableofcontents

% % % Vorlesung 1
\newpage
\chapter{Hyperbolische Modelle}
\section{Das Hyperboloidenmodell}
\Def{}
Definiere die \df{Lorentzform} auf $\R^{n+1}$ durch
\[ \shrp{x,y} := x_1y_1 + \ldots + x_{n}y_n - x_{n+1} y_{n+1} \]
Ein Vektor$x \in \R^{n+1}$ heißt
\begin{align*}
\left\lbrace \begin{aligned}
\text{\df{zeitartig}, falls } &\shrp{x,x} < 0\\
\text{\df{lichtartig}, falls } &\shrp{x,x} = 0\\
\text{\df{raumartig}, falls } &\shrp{x,x} > 0
\end{aligned} \right.
\end{align*}
Definiere das \df{Hyperboloidenmodell} von $\H^n$ durch
\[ I^n = \set{p \in \R^{n+1}}{\shrp{p,p} = -1, p_{n+1 > 0}} \]
\Prop{}
$I^n$ ist eine Riemannsche Mannigfaltigkeit.
\begin{Beweis}{}
Definiere $f : \R^{n+1} \pfeil{} \R$ durch $x \mapsto \shrp{x,x}$. Dann ist $\d f_p(v) = 2\shrp{v,p}$. Ergo ist $\d f_p$ surjektiv für alle $p \in M :=f\i(-1)$. Ergo ist $M$ eine glatte Hyperfläche von $\R^{n+1}$.\\
Ferner ist
\[ T_pM = \Ker ~\d f_p  = \set{v \in \R^{n+1}}{\shrp{p,v} = 0} = p^\bot \]
Da $p$ zeitartig ist, ist $\shrp{\_, \_}$ auf $T_pM$ positiv definit. $I^n$ ist nun gerade die obere Zusammenhangskomponente von $M$.
\end{Beweis}

\Lem{}
Definiere
\[ O(n,1) = \set{A \in \R^{n+1\times n + 1}}{ \shrp{v,w} = \shrp{Av, Aw} }\]
und
\[ O(n,1)^+ = \set{A \in O(n,1)}{A (I^n) \subset I^n } \]
Dann ist $O(n,1)^+$ eine Index-2-Gruppe von $O(n,1)$ und
\[ Isom(I^n) = O(n,1)^+ \]
Ferner gilt
\[ Isom(S^n) = O(n) \text{ und } Isom(\R^n) = \set{x\mapsto Ax + b}{A \in O(n), b\in \R^n } \]


\Prop{}
Die $k$-dimensionalen, vollständigen, total geodätischen, zusammenhängenden Riemannschen Untermannigfaltigkeiten von $I^n$ sind genau die Schnitte
\[ I^n \cap W^{k+1}\]
wobei $W^{k+1}$ ein $k+1$-dimensionaler Untervektorraum von $\R^{n+1}$ ist, der keinen leeren Schnitt mit $I^n$ hat.\\
Folgende Aussagen sind für einen $k+1$-dimensionalen Untervektorraum von $\R^{n+1}$ äquivalent:
\begin{enumerate}[]
	\item $W^{k+1}\cap I^n \neq \emptyset$
	\item $W^{k+1}$ besitzt einen zeit-ähnlichen Vektor
	\item $\shrp{\_,\_}$ besitzt auf $W^{k+1}$ die Signatur $(k,1)$
\end{enumerate}

\Bem{}
Ein $k$-Unterraum von $I^n$ ist isometrisch zu $I^k$.

\Prop{}
Jede nach Bogenlänge parametrisierte Geodäte von $I^n$ ist von der Gestalt
\[ \gamma(t) = \cosh(t) \gamma(0) + \sinh(t)\dot{\gamma}(0) \]

\Kor{}
$H^n$ ist vollständig.

\section{Die Poincare Scheibe}
\Lem{Poincare-Scheiben-Modell}
Definiere die \df{Poincare-Scheibe} durch
\[D^n = \set{x \in \R^n}{\norm{x} < 1} \]
und folgenden Diffeomorphismus
\begin{align*}
p : I^n & \Pfeil{} D^n\\
(x_1, \ldots, x_n, x_{n+1}) & \longmapsto \frac{1}{x_{n+1} + 1} (x_1,\ldots, x_n) 
\end{align*}
Dann ist die Metrik auf $D^n$ gerade gegeben durch
\[ g^D_x = \klam{\frac{2}{1 - \norm{x}^2}}^2g^E_x \]
wobei $g^E$ die euklidische Metrik von $\R^n$ bezeichnet.
\begin{Beweis}{}
	Die Umkehrabbildung von $p$ ist gerade
	\[ p\i(x) = \frac{(2x, 1 + \norm{x}^2)}{1 - \norm{x}^2} \]
	Ihre Derivation ist
	\[ \d_xp\i(u) = \frac{ 2 (u (1 - \norm{x}^2) + 2 x(x|u), 2 (x|u) )}{ (1 - \norm{x}^2)^2 } \]
	Es gilt
	\[ \shrp{d_xp\i(u), d_xp\i(u)} = (\frac{2}{1 - \norm{x}})^2 \norm{u}^2 \]
	Da $p\i$ eine Isometrie sein soll und das Verhalten einer Metrik durch ihre Norm bestimmt ist, folgt nun
	\[ g^D_x = g^I_{p\i(x)} \circ \d_xp\i  = \klam{\frac{2}{1 - \norm{x}^2}}^2g^E_x \]
\end{Beweis}

\Def{}
Ein Diffeomorphismus
\[ f : (M,g) \Pfeil{} (N,h) \] 
heißt \df{konform}, falls eine glatte Funktion $f : M \pfeil{} \R_{>0}$ existiert, sodass
\[ f^*(h_{f(p)}) = \lambda(p)\cdot g_p  \]

\Bem{}
Die Poincare-Scheibe ist ein konformes Modell von $\H^n$, d.\,h., $(D^n, g^E)$ und $(D^n, g^D)$ sind zueinander konform.\\
Daraus folgt nun insbesondere, dass Winkel von sich schneidenden Geodäten in $(D^n, g^D)$ genauso wie in $(D^n, g^E)$ gemessen werden dürfen.

\Lem{}
Die $k$-dimensionalen, vollständigen, zusammenhängenden, total geodätischen Untermannigfaltigkeiten der Poincare-Scheibe sind ihre Schnitte mit $k$-Sphären und $k$-Ebenen von $\R^n$, die orthogonal zum Rand der Poincare-Scheibe liegen.

\Def{}
Sei $S_{p}(r) \subset \R^n$ eine Sphäre mit Radius $r$ um $p$. Definiere die \df{Inversion} an $S_p(r)$ durch
\begin{align*}
\phi : \R^n\setminus\{p\} & \Pfeil{} \R^n\setminus\{p\}\\
x & \longmapsto p + r^2\frac{x- p}{\norm{x-p}^2}
\end{align*}
\Prop{}
Jede Inversion ist \df{anti-konform}, d.\,h. konform und Orientierung umkehrend, und bildet Sphären und Ebenen auf Sphären und Ebenen ab.

\section{Das obere Halbraum-Modell}
\Def{}
Das \df{obere Halbraum-Modell}
\[ H^n = \set{x \in \R^{n}}{x_n > 0} \]
ergibt sich durch eine Inversion der Poincare-Scheibe an der Sphäre
\[ S = S_{(0,\ldots,0,-1)}(\sqrt{2}) \]
Insofern ist die obere Halbebene ein konformes Modell von $\H^n$.

\Prop{}
Die $k$-Ebenen von $H^n$ sind die $k$-Ebenen und $k$-Sphären von $\R^n$, die orthogonal zu $\partial H^n$ sind.

\Prop{}
Die Metrik auf $H^n$ ist gegeben durch
\[ g_x^H =  \frac{1}{x_n^2} g^E \]


\Prop{}
Folgende Abbildungen sind Isometrien von $H^n$:
\begin{enumerate}[1.)]
\item Horizontale Translationen:
\begin{align*}
x \longmapsto x + (b_1,\ldots, b_{n-1}, 0)
\end{align*}
\item Dilationen:
\[ x \longmapsto x \cdot \lambda \]
\item Inversionen an Sphären orthogonal zu $\partial H^n$
\end{enumerate}

\Prop{}
Die Isometrien der Poincare-Scheibe und der oberen Halbebene werden durch Inversionen an Sphären und Reflektion an Euklidischen Ebenen, die alle orthogonal zum Rand stehen, erzeugt.
\Prop{}
In den konformen Modellen sind Kugeln genau die euklidischen Kugeln mit exzentrischen Mittelpunkten.

\section{Die Kleinsche Scheibe}
\Def{}
Die \df{Kleinsche Ebene} besitzt dieselbe Trägermenge $K^n = D^n$ wie die Poincare-Scheibe. Allerdings entsteht die Kleinsche-Ebene durch einen Diffeomorphismus
\begin{align*}
I^n & \Pfeil{} K^n\\
x & \longmapsto \frac{(x_1,\ldots,x_n)}{x_n}
\end{align*}
Die Kleinsche Scheibe ist nicht konform, weswegen ihre Winkel nicht durch eine euklidische Einbettung gemessen werden können. Allerdings sind ihre Geodäten genau die Geraden des $\R^n$.

\section{Ränder}
\Def{}
Zwei nach Bogenlänge parametrisierte Geodäten $\alpha, \beta : [0, \infty) \pfeil{} M$ heißen \df{asymptotisch äquivalent}, falls die Funktion
\[ t \longmapsto d(\alpha(t), \beta(t)) \]
beschränkt ist.\\
Asymptotisch äquivalent Sein ist eine Äquivalenzrelation auf der Menge aller geodätischer Halbgeraden.\\
Teilt man diese Relation heraus, erhält man den \df{Rand} $\partial M$ einer Mannigfaltigkeit $M$. Insbesondere schreibt man
\[ \overline{M} = M \cup \partial M \]

\Prop{}
Es gibt eine Bijektion zwischen $\partial D^n$ als Rand einer Mannigfaltigkeit und $S^{n-1}$.\\
Da ferner $\overline{D^n}$ eine naheliegende Topologie besitzt, können wir diese auf $\overline{\H^n}$ zurückführen.
\begin{Beweis}{}
	Sei $\gamma : [0, \infty) \pfeil{} D^n$ ein geodätischer Strahl. Da $\gamma$ sich orthogonal mit $S^{n-1}$ im Unendlichen schneiden muss, folgt
	\[ \lim\limits_{t \pfeil{} \infty} \gamma(t) \in S^{n-1} \]
	Hierdurch erhalten wir eine surjektive Abbildung
	\[ R(\gamma) :=  \lim\limits_{t \pfeil{} \infty} \gamma(t) \]
	Wir müssen nun zeigen, dass zwei nach Bogenlänge parametrisierte Strahlen $\gamma$, $\beta$ unter $R$ genau dann dasselbe Bild haben, wenn sie asymptotisch äquivalent sind.\\
Wir transformieren das Problem zu einem Problem auf $H^n$ und rechnen die geforderte Eigenschaft dort konstruktiv nach.
\end{Beweis}

\Bem{}
Man kann auch alternativ wie folgt eine Basis der Topologie von $\overline{\H^n}$ definieren: Dazu nimmt man alle offenen Mengen von $\H^n$ und schmeißt alle Mengen der Gestalt
\[ \{ \alpha(t) \in \H^n~|~ \alpha(0) = \gamma(0), \dot{\alpha}(0)  \in V ,t > r\} \cup \set{ [\alpha] \in \partial \H^n }{\alpha(0) = \gamma(0),\dot{\alpha}(0)  \in V} \]
für alle $[\gamma] \in \partial \H^n, V \subseteq_o T_{\gamma(0)}M, r > 0$. Die hierdurch entstehende Topologie stimmt der durch obige Proposition überein.

\Prop{}
Seien $S,S'$ zwei geodätisch vollständige Teilräume von $\H^n$. Dann tretet genau einer der folgenden Fälle ein:
\begin{itemize}
	\item $S$ und $S'$ sind \df{inzident}, d.\,h., $S\cap S'\neq \emptyset$.
	\item $S$ und $S'$ sind \df{asymptotisch parallel}, d.\,h., $S\cap S' = \emptyset$ und $d(S, S') = 0$. Ferner ist dann $\overline{S} \cap \overline{S'}$ ein Punkt in $\H^n$ und existiert keine Geodäte, die zu beiden Räumen orthogonal ist.
	\item $S$ und $S'$ sind \df{ultra-parallel}, d.\,h., $\overline{S}\cap \overline{S'} = \emptyset$ und $d(S, S') > 0$. In diesem Fall existiert genau eine Geodäte, die orthogonal zu beiden Teilräumen steht und den Abstand zwischen beiden realisiert.
\end{itemize}
\begin{Beweis}{}
	Enthält $\overline{S} \cap \overline{S'}$ mindestens zwei Punkte, so enthält der Schnitt auch eine Geodäte zwischen beiden Punkten. Wir können also annehmen, dass sich $S$ und $S'$ wenn überhaupt nur im Unendlichen schneiden und dort höchstens einen Schnittpunkt haben.\\
	Besteht $\overline{S} \cap \overline{S'}$ aus genau einem Punkt, so können wir die Situation in den $H^n$ transformieren und fordern, dass der gemeinsame Schnittpunkt gerade $\infty$ ist. In diesem Fall stehen $S$ und $S'$ parallel zur imaginären Achse, weswegen die Eigenschaften des zweiten Falles folgen.\\
	Im zweiten Fall finden wir $x\in S, x' \in S'$ mit $d(x,x') = d(S, S')$, da $\overline{S}$ und $\overline{S'}$ kompakt sind. Die Geodäte, die $x$ und $x'$ verbindet, muss orthogonal sein, da wir sie sonst verschieben könnten, um den Abstand zwischen $S$ und $S'$ zu minimieren. Es kann keine weitere Geodäte zwischen $S$ und $S'$ mit Abstand $d(S, S')$ geben, da wir sonst einen flachen Bereich gefunden hätten.
\end{Beweis}

\Lem{}
Eine Isometrie $\phi : \H^n \pfeil{} \H^n$ lässt sich zu einem Homöomorphismus $\overline{\phi} : \overline{\H^n} \pfeil{} \overline{\H^n} $ fortsetzen. $\phi$ ist durch $\overline{\phi}_{\partial \H^n}$ eindeutig festgelegt.

\section{Isometrien}
\Prop{}
Sei $\phi : \H^n \pfeil{} \H^n$ eine nichttriviale Isometrie. Dann tritt genau einer der folgenden Fälle ein:
\begin{itemize}
	\item $\phi$ ist \df{elliptisch}, d.\,h., $\phi$ hat einen oder mehrere Fixpunkte in $\H^n$.
	\item $\phi$ ist \df{parabolisch}, d.\,h., $\phi$ hat keinen Fixpunkt in $\H^n$, aber genau einen in $\partial \H^n$.
	\item $\phi$ ist \df{hyperbolisch}, d.\,h., $\phi$ hat keinen Fixpunkt in $\H^n$, aber genau zwei in $\partial \H^n$.
\end{itemize}
\begin{Beweis}{}
Wir können $\overline{\phi}$ als einen Homöomorphismus von $\overline{D^n}$ auf sich selbst auffassen. Nach Brauers Fixpunktsatz muss $\phi$ dann mindestens einen Fixpunkt haben. Hat $\phi$ keinen Fixpunkt in $D^n$ aber mindestens drei auf dem Rand, so fixiert $\phi$ jeden Punkt in $\H^n$, da jeder Punkt in $D^n$ durch seine Winkel zu den drei verschiedenen Randpunkten eindeutig determiniert ist.
\end{Beweis}

\Def{}
Eine hyperbolische Isometrie fixiert zwei Randpunkte und damit auch die Geodäte, die zwischen beiden verläuft. Diese eindeutig bestimmte Geodäte nenne wir die \df{Achse} von $\phi$.

\Def{}
Eine \df{Horosphäre} $S \subset \overline{\H^n}$ um den Punkt $p \in \partial H^n\setminus\{\infty\}$ ist eine $n-1$-dimensionale euklidische Sphäre, die $\partial H^n$ tangential in $p$ schneidet. Eine Horosphäre um $\infty$ ist eine $n-1$-dimensionale euklidische Hyperebene, die orthogonal zur imaginären Achse steht.\\
Beide Horosphären sind flache Untermannigfaltigkeiten mit der Eigenschaft, dass jede Geodäte, die ihr Zentrum verlässt, die Horosphäre orthogonal schneidet.

\Prop{}
Wir stellen Punkte aus $H^n$ in der Form $(x,t)$ dar. Sei $\phi$ eine nichttriviale Isometrie von $\H^n$.
\begin{itemize}
	\item Ist $\phi$ elliptisch mit Fixpunkt $0 \in D^n$, so gibt es ein $A \in O(n)$, sodass sich $\phi$ darstellen lässt durch
	\begin{align*}
\phi:	D^n & \Pfeil{} D^n\\
	x & \longmapsto Ax
	\end{align*}
	\item Ist $\phi$ parabolisch mit Fixpunkt $\infty \in \partial H^n$, so gibt es $A \in O(n-1)$ und $b\in \R^{n-1}$, sodass
	\begin{align*}
\phi:	H^n & \Pfeil{} H^n\\
	(x,t) & \longmapsto (Ax + b, t)
	\end{align*}
	\item Ist $\phi$ hyperbolisch mit Fixpunkten $0,\infty \in \partial H^n$, so gibt es ein $\lambda > 0, \neq 1$ und ein $A\in O(n-1)$, sodass
		\begin{align*}
	\phi:	H^n & \Pfeil{} H^n\\
	(x,t) & \longmapsto (\lambda Ax, \lambda t)
	\end{align*}
\end{itemize}
\begin{Beweis}{}
Der elliptische Fall ist klar.\\
Sei der zweite Fall gegeben. Ist $O$ eine Horosphäre um $\infty$, so muss $\phi(O)$ wieder eine Horosphäre um $\infty$ liefern. Dann gibt es ein $(x,t) \in O$, sodass $\phi(x,t) = (x, t')$. Ist $t' \neq t$, so erhält $\phi$ eine Geodäte durch $(x,t)$ und $\infty$ und ist nicht mehr parabolisch. Ergo ist $\phi(O) = O$. D.\,h., $\phi$ ist auf jeder Horosphäre um $\infty$ durch eine euklidische Isometrie gegeben.\\
Sei nun der dritte Fall gegeben. $\phi$ erhält die imaginäre Achse, ergo gibt es ein $\lambda$ mit $\phi(0,1) = (0, \lambda)$. Setzt man $\psi(x,t) = \lambda\i \phi(x,t)$, so gibt es ein $A\in O(n-1)$ mit
\[ \d_{(0,1)} \psi = \klam{
\begin{matrix}
A & 0\\
0 & 1
\end{matrix}
} \]
womit folgt
\[ \psi(x,t) = (Ax,t) \]
\end{Beweis}

\Def{}
Für eine Isometrie $\phi : M \pfeil{} M$ sei die \df{Versetzung} definiert durch
\begin{align*}
d(\phi) := \inf_{p\in M}d(p, \phi(p))
\end{align*}

\Kor{}
\begin{itemize}
	\item Eine elliptische Isometrie hat eine Versetzung von 0, die an ihren Fixpunkten verwirklicht wird.
	\item Eine parabolische Isometrie hat eine Versetzung von 0, die nirgendwo realisiert wird, und fixiert jede Horosphäre um ihren Fixpunkt.
	\item Eine hyperbolische Isometrie hat eine Versetzung von $d > 0$, die genau auf ihrer Achse realisiert wird.
\end{itemize}

\section{Möbiusgeschichten}
\Def{}
Es bezeichne $S = \C \cup \infty$ die \df{Riemannsche Zahlenkugel}. Die Gruppe $PSL_2(\C)$ agiert auf $S$ durch die \df{Möbiustransformation}
\[ \klam{
\begin{matrix}
a & b\\
c & d
\end{matrix}
}.z := \frac{az + b}{cz + d} \]
Die Möbiustransformation ist ein orientierungserhaltender Diffeomorphismus auf $S$.\\
Die \df{Anti-Möbiustransformation} gegeben durch
\[ \klam{
	\begin{matrix}
	a & b\\
	c & d
	\end{matrix}
}.z := \frac{a\overline{z} + b}{c\overline{z} + d} \]
ist ein orientierungsumkehrender Diffeomorphismus auf $S$.\\
Unter $Conf(S) \subset Diffeo(S)$ verstehen wir die Menge aller Möbius- und Anti-Möbiustransformationen, die durch Elemente aus $PSL_2(\C)$ induziert werden.

\Prop{}
Inversionen entlang Sphären und Spiegelungen entlang Geraden sind beides Anti-Möbiustransformationen und erzeugen $Conf(S)$.

\Lem{}
Betrachte
\[ H^2 = \set{ x + iy}{x \in \R, y > 0} \]
und setze
\[ Conf(H^2) = \set{\phi \in Conf(S)}{\phi(H^2) \subseteq H^2} \]
Dann ist jede Transformation aus $Conf(H^2)$ induziert durch eine Matrix mit reellen Einträgen, deren Determinante gleich 1 ist, falls die Transformation orientierungserhaltend ist, anderenfalls -1 ist.\\
Es gilt
\[ Conf^+(H^2) = PSL_2(\R) \]

\Prop{}
Inversionen entlang Kreisen und Reflexionen entlang Geraden, die beide orthogonal zu $\R$ sind, generieren $Conf(H^2)$.
\Kor{}
\[ Conf(H^2) = Isom(H^2) \]

\Prop{}
Eine nichttriviale Transformation $A \in PSL_2(\R)$ ist
\begin{itemize}
	\item elliptisch, falls $\bet{tr(A)} < 2$
	\item parabolisch, falls $\bet{tr(A)} = 2$
	\item hyperbolisch, falls $\bet{tr(A)} > 2$
\end{itemize}

\Prop{}
Da $\partial H^3 = S$, gilt
\[ Isom(H^2) = Conf(S) \]

\Prop{}
Eine nichttriviale Transformation $A \in PSL_2(\C)$ ist
\begin{itemize}
	\item elliptisch, falls $tr(A) \in (-2, 2)$
	\item parabolisch, falls $tr(A) = \pm 2$
	\item hyperbolisch, falls $tr(A) \in \C\setminus [-2, 2]$
\end{itemize}

\Prop{}
Die Distanzfunktion ist auf $\H^n$ \df{strikt konvex}, d.\,h., für je zwei nach Bogenlänge parametrisierte, disjunkte Geodäten $\alpha, \beta : \R \pfeil{} \H^n$ gilt
\[ d(\alpha(t  x_1 + (1-t)  y_1), \beta(t  x_2 + (1-t)  y_2) ) < t  d(\alpha(x_1), \beta(x_2)) + (1-t)  d(\alpha(y_1), \beta(y_2) ) \]
für alle $x_1 \neq y_1, x_2 \neq y_2, t \in (0,1)$.
\begin{Beweis}{}
	Da die Distanzfunktion stetig ist, genügt es die Aussage für $t = \frac{1}{2}$ zu zeigen, also
	\[ d(\alpha(\frac{x_1 + y_1}{2}), \beta(\frac{x_2 + y_2}{2})) < \frac{d(\alpha(x_1), \beta(x_2) ) + d(\alpha(y_1), \beta(y_2))}{2}  \]
	Setze
	\[ m:= \alpha(\frac{x_1 + y_1}{2})  \text{ und }n:= \beta(\frac{x_2 + y_2}{2}) \]
	Für einen Punkt $p \in \H^n$ bezeichne $s_p$ die Punktspiegelung bei $p$. Dies ist eine globale Isometrie. Für zwei verschiedene Punkte $p,q$ ist $s_p \circ s_q$ eine orientierungserhaltende Isometrie. Insbesondere ist diese hyperbolisch, denn ist $l$ die Geodäte, die $p$ und $q$ enthält, so wird jeder Punkt auf $l$ um $2d(p,q)$ entlang $l$ bewegt.\\
	Setze nun
	\[ \tau = s_n \circ s_m \]
	Dann gilt
	\begin{align*}
	d(\alpha(x_1), \tau(\alpha(x_1)) ) & \leq d(\alpha(x_1), \beta(x_2) ) + d( \beta(x_2), \tau(\alpha(x_1)) )\\
	&=d(\alpha(x_1), \beta(x_2) ) + d(  s_n(\beta(y_2))  , s_n(\alpha(y_1))  )\\
		&=d(\alpha(x_1), \beta(x_2) ) + d(  \beta(y_2)  , \alpha(y_1)  )\\
	\end{align*}
	Da $\alpha(x_1)$ nicht auf der hyperbolischen Achse von $\tau$ liegt, gilt ferner
		\begin{align*}
2d(m,n) &= d(m, \tau(m))\\
&< d(\alpha(x_1), \tau(\alpha(x_1)))\\
&\leq d(\alpha(x_1), \beta(x_2) ) + d(  \beta(y_2)  , \alpha(y_1)  )
	\end{align*}
\end{Beweis}

\chapter{Werkzeuge der Hyperbolischen Mannigfaltigkeiten}
\section{Hyperbolische Mannigfaltigkeiten}
\Def{}
Eine Mannigfaltigkeit heißt \df{hyperbolisch}, wenn sie durch offene, lokal isometrische Karten $\phi : U \pfeil{} \H^n$ überdeckt wird und zusammenhängend ist.

\Satz{}
Eine hyperbolische, einfach zusammenhängende, vollständige Mannigfaltigkeit ist isometrisch zu $\H^n$.
\begin{Beweis}{Skizze}
	Wir konstruieren eine lokal isometrische Abbildung $D : M \pfeil{} \H^n$. Zuerst wählen wir eine Karte $ U \pfeil{} \H^n$ und einen Punkt $p \in U$. Für ein beliebiges $q \in M$ wählen wir einen Weg $\gamma : p \mapsto q$ und überdecken diesen Weg mit endlich vielen hyperbolischen, kompatiblen Karten. Dadurch können wir $q$ einem Punkt aus $D(q) \in \H^n$ zuordnen.\\
	Um die Wohldefiniertheit von $D$ zu zeigen, zeigt man zuerst, dass die Wahl der Karten, die $\gamma$ überdecken unerheblich für die Wahl des Bildpunktes $D(q)$ sind.\\
	Danach zeigt man, dass $D(q)$ unabhängig von der Wahl des Weges $\gamma:p\mapsto q$ ist. Denn ist $H : \gamma \sim \gamma'$ eine Homotopie mit fixierten Endpunkten, so ist $D(H)$ ebenfalls eine Homotopie mit fixierten Endpunkten.\\
	Da $D$ lokal isometrisch und $M$ geodätisch vollständig ist, ist $D$ surjektiv. Da $\H^n$ ferner einfach zusammenhängend ist, ist $D$ injektiv. 
\end{Beweis}

\Kor{}
Ist $M$ eine hyperbolische, einfach zusammenhängende Mannigfaltigkeit, so gibt es eine lokal isometrische Abbildung $D : M \pfeil{} \H^n$, die sogenannte \df{Entfaltung}, die eindeutig bis auf Nachschaltung einer Isometrie auf $\H^n$ ist.

\newpage
\Prop{}
Sei $M$ eine hyperbolische Mannigfaltigkeit und $D : M \pfeil{} \H^n$ eine lokale Isometrie. Für jedes $g\in Isom(M)$ existiert genau ein $\rho(g) \in Isom(\H^n)$, sodass folgendes Diagramm kommutiert
\begin{center}
	\begin{tikzpicture}
	\node (M1) at (0,2) [] {$M$};
	\node (H1) at (4,2) [] {$\H^n$};
	\node (H2) at (4,0) [] {$\H^n$};
	\node (M2) at (0,0) [] {$M$};
	
	\draw [->] (M1) -> (H1) node[midway, above] {$D$};
	\draw [->] (M2) -> (H2) node[midway, above] {$D$};
	\draw [->] (M1) -> (M2) node[midway, right] {$g$};
	\draw [->] (H1) -> (H2) node[midway, right] {$\rho(g)$};
	\end{tikzpicture}
\end{center}
Die Abbildung $\rho : Isom(M) \pfeil{} Isom(\H^n)$ ist eine Gruppenhomomorphismus und heißt die \df{Holonomie} zu $D$.

\Bem{}
Sei $M$ eine hyperbolische, vollständige Mannigfaltigkeit. Ist $\widetilde{M} \pfeil{} M$ die universelle Überlagerung von $M$, so ist $\widetilde{M}$ hyperbolisch, vollständig und einfach zusammenhängend und deswegen isometrisch zu $\H^n$. Wir erhalten also eine lokal isometrische Überlagerung $\H^n \pfeil{} M$ und wissen dadurch
\[ \pi_1(M) = Deck(\H^n \pfeil{} M) \subseteq Isom(\H^n) \]

\section{Vollständige Hyperbolische Mannigfaltigkeiten}
\Def{}
Sei $G$ eine Gruppe, die auf einen topologischen Raum $X$ wirkt. Diese Wirkung heißt \df{eigentlich diskontinuierlich}, falls es für jedes Paar von zwei verschiedenen Punkten $x,y \in X$ Umgebungen $U_x, U_y$ existieren, sodass
\[ \#\set{g \in G}{g.U_x \cap U_y \neq \emptyset} < \infty \]
Die Wirkung heißt \df{frei}, falls jedes nichttriviale $g\in G$ fixpunktfrei auf $X$ wirkt.

\Lem{}
Wirkt $\Gamma \subset Isom(\H^n)$ frei und eigentlich diskontinuierlich auf $\H^n$, so besitzt $\H^n / \Gamma$ genau eine Struktur als hyperbolische, vollständige Mannigfaltigkeit, durch die $\H^n \pfeil{} \H^n/\Gamma$ zu einer \df{Riemannschen Überlagerung}, d.\,h., lokal isometrischen Überlagerung wird.

\Prop{}
Jede vollständige, hyperbolische Mannigfaltigkeit ist isometrisch zu $\H^n / \Gamma$ für eine passende Gruppe $\Gamma \subset Isom(\H^n)$, die frei und eigentlich diskontinuierlich wirkt.

\Bem{}
$\Gamma \subset  Isom(\H^n)$ wirkt genau dann eigentlich diskontinuierlich, wenn $\Gamma$ diskret ist.
\begin{Beweis}{}
	$\Gamma$ wirke eigentlich diskontinuierlich. Seien $x,y \in \H^n$, dann gibt es Umgebungen $U_x, U_y$, sodass nur endlich viele Elemente aus $\Gamma$ $U_x$ über $U_y$ bewegen.\\
	Definiert man die Abbildung
	\begin{align*}
	\phi : \Gamma& \Pfeil{} \H^n\\
	g & \longmapsto g(x)
	\end{align*}
	so besitzt die offene Menge $\phi\i(U_x)$ nur endlich viele Elemente. Ergo besitzt jeder Punkt aus $\Gamma$ eine endliche Umgebung.\\
	Sei $\Gamma$ nun diskret. Seien $x,y \in \H^n$ zwei verschiedene Punkte. Definiere $A:= \set{g \in \Gamma}{g.U_x \cap U_y \neq \emptyset}$. Indem wir $U_x$ und $U_y$ sukzessive kleiner machen, finden wir eine Folge in $\Gamma$, die durch ein Kompaktum beschränkt wird. Da $\Gamma$ endlich ist, muss dieses Kompaktum endlich sein.
\end{Beweis}

\Prop{}
$\H^n / \Gamma$ und $\H^n / \Sigma$ sind genau dann isometrisch, wenn $\Sigma$ und $\Gamma$ konjugiert in $\H^n$ sind.
\begin{Beweis}{}
	$\H^n / \Gamma$ und $\H^n / \Sigma$ sind genau dann isometrisch, wenn es eine Isometrie $\phi : \H^n / \Gamma\pfeil{}\H^n / \Sigma$ gibt. Da $\H^n$ beide Räume überlagert, gibt es diese Isometrie genau dann, wenn es eine Isometrie $\widetilde{\phi}$ gibt, die folgendes Diagramm kommutieren lässt
	\begin{center}
		\begin{tikzpicture}
		\node (M1) at (0,2) [] {$\H^n $};
		\node (H1) at (4,2) [] {$\H^n$};
		\node (H2) at (4,0) [] {$\H^n/ \Gamma$};
		\node (M2) at (0,0) [] {$\H^n / \Sigma$};
		
		\draw [->] (M1) -> (H1) node[midway, above] {$\widetilde{\phi}$};
		\draw [->] (M2) -> (H2) node[midway, above] {$\phi$};
		\draw [->] (M1) -> (M2) node[midway, right] {};
		\draw [->] (H1) -> (H2) node[midway, right] {};
		\end{tikzpicture}
	\end{center}
Dies ist genau dann der Fall, wenn es ein Element $\widetilde{\phi}$ in $Isom(\H^n)$ gibt, sodass für es für jedes $\sigma \in \Sigma$ ein $\gamma \in \Gamma$ gibt mit
\[ \widetilde{\phi}  \circ \sigma = \gamma \circ \widetilde{\phi} \]
\end{Beweis}

\Bem{}
Wirkt $\Gamma \subset Isom(\H^n)$ eigentlich diskontinuierlich, so ist für jedes $p\in \H^n$ der \df{Stabilisator}
\[ Stab_\Gamma(p) := \set{g \in \Gamma}{g(p) = p} \]
endlich und jeder \df{Orbit} von $\Gamma$ in $p$
\[ \Gamma.p := \set{g(p)}{g\in \Gamma} \]
diskret.

\Def{}
Eine Menge von Teilmengen von $\H^n$ heißt \df{lokal endlich}, falls sie für jedes Kompaktum $K \subset \H^n$ nur endlich viele Teilmengen besitzt, die einen nicht-leeren Schnitt mit $K$ haben.

\Prop{}
$\Gamma\subset Isom(\H^n)$ wirke eigentlich diskontinuierlich. Dann liegt die Menge
\[ \set{p \in \H^n}{ Stab_\Gamma(p) \text{ ist trivial} } \]
offen und dicht in $\H^n$.
\begin{Beweis}{}
	Setze
	\[ \mathcal{A}:= \set{Fix(g)}{g \in \Gamma \setminus \{\id{\H^n}\}} \]
	$\mathcal{A}$ ist lokal endlich, da $\Gamma$ diskret ist. Setze nun
	\[ B:= \H^n\setminus \klam{\bigcup_{X \in \mathcal{A}} X} \]
	Dann ist $B$ offen und es gilt
	\[ B = \set{p \in \H^n}{ Stab_\Gamma(p) \text{ ist trivial} } \]
	Jedes $X \in \mathcal{A}$ ist eine echte Untermannigfaltigkeit von $\H^n$. Deswegen und weil $\mathcal{A}$ lokal endlich ist, liegt $B$ dicht in $\H^n$.
\end{Beweis}

\Prop{}
Eine diskrete Gruppe $\Gamma \subset Isom(\H^n)$ agiert genau dann frei, wenn sie \df{torsionsfrei} ist, d.\,h., kein Element endlicher Ordnung besitzt.
\begin{Beweis}{}
	Es sei $\Gamma$ diskret und besitze ein Torsionselement $\phi$. Für einen Punkt $p$ ist dann die Menge $\set{\phi^k(p)}{k\geq 0}$ endlich. Der Mittelpunkt von dieser Menge wird dann durch $\phi$ fixiert.\\
	Ist $\Gamma$ nicht frei, so gibt es ein $p$ mit $Stab_\Gamma(p) \neq 1$. Da $\Gamma$ diskret ist, ist $Stab_\Gamma(p)$ eine endliche, nichttriviale Untergruppe von $\Gamma$ und besitzt deswegen Torsionselemente.
\end{Beweis}

\Satz{}
Es gibt eine natürliche Eins-zu-Eins-Korrespondenz zwischen der Menge der vollständigen hyperbolischen Mannigfaltigkeiten modulo Isometrie und der Menge der torsionsfreien, diskreten Untergruppen von $Isom(\H^n)$ modulo Konjugation.

\Satz{}
Agiert $\Gamma < Isom(\H^n)$ frei und eigentlich diskontinuierlich, so tut dies auch jede Untergruppe von $\Gamma$. Wir erhalten eine Eins-zu-Eins-Korrespondenz zwischen Überlagerungen $\H^n / \Gamma' \pfeil{}\H^n / \Gamma$ und den Untergruppen $\Gamma' < \Gamma$, wobei der Grad der Überlagerung gerade mit dem Index der Untergruppe übereinstimmt.

\Lem{Selberg}
Ist $G \subset GL_n(\C)$ endlich erzeugt, so existiert eine normale, torsionsfreie Untergruppe $H \subset G$ von endlichem Index.

\Kor{}
Jede endlich erzeugte, diskrete Untergruppe $\Gamma \subset Isom(\H^n)$ enthält eine normale Untergruppe endlichen Index, die frei und eigentlich diskontinuierlich auf $\H^n$ agiert.

\Def{}
Die spezielle lineare Untergruppe $PSL_2(\Z) < PSL_2(\R) = Isom^+(\H^2)$ ist diskret, agiert aber nicht frei auf $\H^2$.\\
Betrachte aber folgenden Gruppenhomomorphismus für $m\geq 2$
\begin{align*}
\mod m : PSL_2(\Z) & \Pfeil{} PSL_2(\Z / m \Z)
\end{align*}
Den Kern $\Gamma(m)$ dieser Abbildung nennt man die \df{Kongruenzenuntergruppe}. Er ist eine diskrete Untergruppe in $PSL_2(\Z)$ von endlichem Index.

\Prop{}
Für $m\geq 4$ agiert $\Gamma(m)$ frei auf $\H^2$.

\section{Polyeder}
\Def{}
Wir nennen eine Menge von Halbräumen in $\H^n$ \df{lokal endlich}, wenn die Menge der korrespondierenden Hyperebenen lokal endlich ist.\\
Eine $n$-dimensionaler \df{Polyeder} in $\H^n$ ist eine Teilmenge, die sich als Schnitt von einer lokal endlichen Menge von Halbräumen schreiben lässt und eine offene Menge enthält.

\Bem{}
Ist $P \subset \H^n$ ein Polyeder, so sind $P$ und $\overline{P}\subset \overline{\H^n}$ konvex.

\Def{}
Für eine beliebige $S \subset \overline{\H^n}$ definieren wir ihre \df{konvexe Hülle} als die kleinste konvexe Menge, die $S$ enthält.

\Def{}
Unter einem \df{endlichen Polyeder} $P$ verstehen wir die konvexe Hülle von endlich vielen Punkten in $\H^n$, die nicht alle im unendlichen Abschluss einer Hyperebene liegen.\\
Liegt eine Ecke von $P$ in $\partial \H^n$, so nennen wir sie \df{ideal}, ansonsten \df{endlich}.\\
Sind alle Ecken von $P$ ideal, so nennen wir $P$ einen \df{idealen Polyeder}.

\Lem{}
Sei $O \subset \overline{\H^n}$ eine Horosphäre um $p \in \partial \H^n$. $D \subset O$ sei eine Menge endlichen Volumens und $C$ der \df{Kegel} über $D$, d.\,h., $C$ besteht aus allen Geodätischen Strahlen, die von einem Punkt aus $D$ nach $p$ führen. Es gilt
\[ vol(C) = \frac{vol_O(D)}{n-1} \]
wobei $vol_O(D)$ das $n-1$-dimensionale Volumen im affinen Raum $O$ bezeichnet.
\begin{Beweis}{}
	Wir betrachten die Situation im oberen Halbraum-Modell mit $p = \infty$. $O$ ist dann eine Hyperebene, die orthogonal zur imaginären Achse steht und eine Höhe $h$ hat. Es gilt
	\begin{align*}
	Vol(C) &= \int_{C} \frac{1}{t^n} \d x_1 \ldots \d x_{n-1} \d t\\
	&= \int_{D}\d x_1 \ldots \d x_{n-1} \int_{h}^{\infty} \frac{1}{t^n} \d t\\
		&= \int_{D}\d x_1 \ldots \d x_{n-1} \cdot \frac{1}{h^{n-1}} \frac{1}{n-1} \d t\\
	&= \frac{Vol_O(D)}{n-1}
	\end{align*}
\end{Beweis}

\Prop{}
Jeder endliche Polyeder hat endliches Volumen.
\begin{Beweis}{}
	Besitzt ein endlicher Polyeder $P$ keine idealen Ecken, so ist die Aussage klar. Ansonsten kann man um jede ideale Ecke eine Horosphäre ziehen. Hierdurch wird $P$ in zwei Mengen aufgeteilt: $P$ geschnitten mit dem Inneren aller Horosphären und $P$ ohne das Innere der Horosphären.\\
	Durch obiges Lemma lässt sich das Volumen der ersten Menge abschätzen.
\end{Beweis}

\Def{}
Eine \df{Kachelung} ist eine lokal endliche Menge von Polyedern, die $\H^n$ überdecken und sich jeweils nur in gemeinsamen Facetten schneiden.

\Prop{}
Ist $S \subset \H^n$ diskret, so definiere die \df{Voronoi-Kachelung} für $p \in S$ durch
\[ D(p) := \set{q \in \H^n}{d(p,q) \leq d(p',q) \forall p' \in S } \]
\begin{Beweis}{}
	Wir müssen zeigen, dass die Voronoi-Kachelung tatsächlich eine Kachelung ist. Sind $p,p'\in S$ zwei verschiedene Punkte, so ist
	\[ \set{q \in \H^n}{d(p,q) = d(p',q)} \]
	eine Hyperebene und
	\[ H(p,p') := \set{q \in \H^n}{d(p,q) \leq d(p',q)} \]
	ein Halbraum. Es gilt
	\[ D(p) = \bigcap_{p'\in S} H(p,p')  \]
	und da $S$ diskret ist, sind all diese Halbräume lokal endlich.\\
	Sind $p,p'$ in $S$ verschieden, so ist $D(p) \cap D(p')$ in der Hyperebene
	\[\set{q \in \H^n}{d(p,q) = d(p',q)} \]
	enthalten.
\end{Beweis}

\Def{}
Ein \df{Fundamentalbereich} für eine diskrete Gruppe $\Gamma \subset Isom(\H^n)$ ist ein Polyeder $D \subset \H^n$, sodass alle $g(D)$ verschieden sind und eine Kachelung von $\H^n$ bilden.

\Bem{}
Ist $\Gamma \subset Isom(\H^n)$ eine diskrete Gruppe und $p\in \H^n$ ein Punkt mit trivialen Stabilisator, so erhält man durch $S = \Gamma.p$ eine Voronoi-Kachelung, deren Kacheln gerade einen Fundamentalbereich für $\Gamma$ bilden. Diese Kacheln nenne wir \df{Dirichlet-Bereiche}. Es gilt
\[ D(g(p)) = g.D(p) \]

\Prop{}
Ist $M = \H^n / \Gamma$ und $D$ ein Fundamentalbereich für $\Gamma$, so gilt
\[ Vol_{\H^n}(D) = Vol_M(M) \]
Ist ferner $D$ ein Dirichlet-Bereich, so ist $D$ genau dann kompakt, wenn $M$ kompakt ist.

\Def{}
Eine $n$-dimensionale \df{Verklebung} besteht aus einer endlichen Menge von $n$-Simplizes und aus affinen Verklebungsabbildungen, die jede Facette eines Simplex mit genau einer anderen Facette identifizieren.\\
Hieraus entsteht ein $n$-dimensionaler Raum, indem man alle Simplizes disjunkt vereinigt und dann die Relation, die durch die Verklebungsabbildungen erzeugt wird, heraus teilt.

\Def{}
Sei $\sigma \in \Sigma$ ein Simplex eines Simplizialkomplexes. Ist $\tau \supset \sigma$ ein weiterer Simplex, so heißt $\overline{\sigma} \subset \tau$ \df{opposit} zu $\sigma$, falls
\[ \sigma \cap \overline{\sigma} = \emptyset \text{ und } \textsf{conv}(\sigma \cup \overline{\sigma}) = \tau \]
Definiere den \df{Link} von $\sigma$ als den Simplizialkomplex aller Simplizes, die opposit zu $\sigma$ liegen, d.\,h.
\[ lk(\sigma) := \set{\overline{\sigma}}{\sigma \subset \tau}  \]

\Prop{}
Der Raum einer drei-dimensionalen Verklebung ist genau dann eine Mannigfaltigkeit, wenn jeder Link einer Ecke homöomorph zu $S^2$ ist.
\Prop{}
Der Raum einer drei-dimensionalen Verklebung ist genau dann eine Mannigfaltigkeit, wenn seine Eulerzahl gleich Null ist.
\begin{Beweis}{}
	Es sei $X$ der Raum der Verklebung. $X$ habe $k$ Ecken, $e$ Kanten, $f$ Facetten und $t$ Tetraeder. Jeder Tetraeder bringt vier Facetten ins Spiel; allerdings werden immer zwei verschieden Facetten gepaart, ergo gilt
	\[ f = 2t \]
	Es bezeichne $v_1,\ldots, v_k$ die Ecken des Simplizes. Jede Kante enthält zwei Ecken, die opposit zueinender sind. Jede Facette enthält drei Kanten, die opposit zu den Ecken der Facette stehen. Jeder Tetraeder enthält vier Facetten, die opposit zu seinen Ecken stehen. Es folgt
	\[ \sum_{i=1}^k \chi(lk(v_i)) = 2e - 3f + 4t = 2e - 2t \]
	und
	\[ \chi(X) = k - e+f-t = k -e + t = k - \frac{1}{2}(\sum_{i=1}^k \chi(lk(v_i))) \]
	Jedes $lk(v_i)$ bildet eine geschlossene, zwei-dimensionale Fläche, somit gilt
	\[ lk(v_i) \leq 2 \]
	Gleichheit gilt genau dann, wenn $lk(v_i)$ homöomorph zu $S^2$ ist.
\end{Beweis}

\Bem{Poincares Homologiesphäre}
Betrachte einen regelmäßigen Dodekaeder. Unter dem \df{Torsionswinkel} verstehen wir den Winkel zwischen zwei den Normalen zweier anliegender Facetten.\\
Wenn wir die Raumkrümmung erhöhen, steigen die Torsionswinkel unseres Dodekaeders an. Insbesondere können wir hierdurch erreichen, dass diese Winkel den Wert $\frac{2}{3}\pi$ erreichen. Wir können nun gegenüberliegende Seiten um $\frac{1}{3}\pi$ drehen und verkleben. Hierdurch erhalten wir eine kompakte, sphärische Mannigfaltigkeiten, die nicht homöomorph zu $S^3$ ist, aber dieselben Homologiegruppen hat.

\Bem{Der Dodekaeder von Seifert und Weber}
Wir können in obiger Situation die Raumkrümmung erniedrigen anstatt zu erhöhen und dadurch Torsionswinkel von $\frac{2}{5} \pi$ für unseren Dodekaeder erreichen. Indem wir gegenüberliegende Facetten um $\frac{3}{5}\pi$ verdrehen und verkleben, erhalten wir wieder eine kompakte, hyperbolische Mannigfaltigkeit.

\chapter{Dick-Dünn-Zerlegung}
\section{Tuben und Spitzen}
\Prop{}
Sei $M = \H^n / \Gamma$ eine Mannigfaltigkeit. Dann gilt
\[ inj_x(M) = \frac{1}{2}\inf_{\gamma \in \Gamma} d(x, \gamma(x))  \]
\Kor{}
\[ inj(M) = \frac{1}{2}\inf_{\gamma \in \Gamma} d(\gamma) \]

\Kor{}
Ist $M = \H^n / \Gamma$ eine kompakte Mannigfaltigkeit, so besteht $\Gamma$ nur aus hyperbolischen Elementen.
\begin{Beweis}{}
	$\Gamma$ kann von vornerein keine elliptischen Elemente besitzen.\\
	Da $M$ kompakt ist, ist $inj_x(M)$ nach unten beschränkt für $x \in M$. Allerdings ist $d(\gamma) = 0$ für eine parabolische Isometrie $\gamma$.
\end{Beweis}

\Def{Tuben}
Sei $\phi \in Isom(\H^n)$ hyperbolisch mit Achse $l$ und minimaler Versetzung $d$. Teilt man $\Gamma = \shrp{\gamma}$ aus $\H^n$ heraus, erhält man eine \df{unendliche Tube}. $l$ wird dabei auf eine geschlossene Geodäte der Länge $d$ abgebildet, der \df{Nabe} der Tube.

\Prop{}
Jede unendlich Tube ist diffeomorph zu $S^1 \times \R^{n-1}$ oder $S^1 \widetilde{\times} \R^{n-1}$, wenn $\phi$ orientierungsumkehrend ist.\\
$S^1 \widetilde{\times} \R^{n-1}$ entsteht aus $[0,1] \times  \R^{n-1}$, indem man die Relation
\[ (0,x) \sim (1, -x) \]
heraus teilt.

\Def{}
Eine \df{Tube der Länge $R$} entsteht, indem man $\Gamma = \shrp{\gamma}$ aus der $R$-Nachbarschaft von $l$ heraus teilt. Der entstehende Raum ist diffeomorph zu $S^1 \times D^{n-1}$ bzw. $S^1 \widetilde{\times} D^{n-1}$

\Def{Spitze}
Sei $\Gamma < Isom(\R^{n-1})$ eine nichttriviale diskrete Untergruppe, die frei auf $\R^n$ agiert.\\
Jedes Element in $\Gamma$ induziert durch
\[ \phi(x,t) := (Ax + b, t) \]
eine parabolische Wirkung auf $H^n$, durch die $\infty$ fixiert wird. Teilt man diese Wirkung heraus, erhält man eine hyperbolische Mannigfaltigkeit, die diffeomorph ist zu
\[ \R_{> 0} \times (\R^{n-1} / \Gamma) \]
Wir nennen diese hyperbolische Mannigfaltigkeit eine \df{Spitze}.\\
Eine \df{beschränkte Spitze} erhält man, indem man das diffeomorphe Äquivalent zu $[a,+\infty) \times (\R^{n-1} / \Gamma)$ für $a > 0$ nimmt.

\Prop{}
Für eine beschränkte Spitze $C$ gilt
\[ vol(C) = \frac{vol(\partial C)}{n-1} \]

\Lem{}
\label{Kommutieren}
Seien $\phi_1,\phi_2$ zwei hyperbolische oder parabolische Isometrien auf $\H^n$. Kommutieren diese, so gilt
\[ Fix(\phi_1) = Fix(\phi_2) \]
\begin{Beweis}{}
	Allgemein gilt für jedes Paar von Bijektionen $a,b$
	\[ Fix(aba\i) = a(Fix(b)) \]
	In unserem Fall folgt hierdurch
	\[ Fix(\phi_1) = \phi_2(Fix(\phi_1)) \]
	Ist also $\phi_1$ parabolisch, so besitzt $\phi_2$ unter Anderem denselben Fixpunkt wie $\phi_1$. Aus symmetrischen Gründen ist $\phi_2$ aber dann auch parabolisch.\\
	Sei $\phi_1$ also hyperbolisch. Dann sind $\phi_1$ und $\phi_2$ beide hyperbolisch und beide erhalten dieselbe Achse. Ergo müssen sie beide dieselben Punkte im Unendlichen fixieren.
\end{Beweis}

\Lem{}
\label{Lemma 422}
Seien $\phi_1, \phi_2 \in \Gamma$ zwei Isometrien einer diskreten und torsionsfreien Gruppe. Dann tritt genau einer der folgenden Fälle ein:
\begin{itemize}
	\item $Fix(\phi_1) \cap Fix(\phi_2) = \emptyset$.
	\item $\phi_1$ und $\phi_2$ sind beide parabolisch und haben denselben Fixpunkt.
	\item $\phi_1$ und $\phi_2$ sind beide hyperbolisch und lassen sich beide als eine Potenz derselben hyperbolischen Isometrie schreiben.
\end{itemize}
\begin{Beweis}{}
	Wir nehmen an, dass $Fix(\phi_1) \cap Fix(\phi_2) \neq \emptyset$.
\begin{itemize}
	\item Angenommen $\phi_1$ ist hyperbolisch und $\phi_2$ ist parabolisch. Wir wählen dann das obere Halbraummodell mit $Fix(\phi_1) = \{0, \infty\}$ und $Fix(\phi_2) = \infty$.\\
	$\phi_1$ und $\phi_2$ agieren dann durch
	\[ \phi_1(x,t) = \lambda(Ax, t) \text{ und } \phi_2(x,t) = (Bx + b, t) \]
	Ohne Beschränkung der Allgemeinheit ist $\lambda < 1$. Betrachte
	\[ \phi_1^n\circ \phi_2 \circ \phi_1^{-n}(0,t) = (\lambda^n A^nb, t) \Pfeil{n \pfeil{} \infty} (0,t) \]
	Dies ist ein Widerspruch zur Annahme, dass $\Gamma$ diskret wäre.
	\item $\phi_1$ und $\phi_2$ seien beide hyperbolisch und besitzen einen gemeinsamen Fixpunkt. Der Kommutator $[\phi_1, \phi_2] = \phi_1\phi_2 \phi_1\i \phi_2\i$ erhält jede Horosphäre um $\infty$. Da er aber nicht parabolisch sein kann, muss er trivial sein. Ergo kommutieren $\phi_1$ und $\phi_2$.
	\item Besitzen $\phi_1$ und $\phi_2$ dieselbe hyperbolisch Achse, so sind beide durch $w_i = d(\phi_i)$ bestimmt. $w_1$ und $w_2$ spannen als Elemente von $\R$ eine abelsche, additive Gruppe auf. Diese muss von einem Element erzeugt werden, da sonst $\Gamma$ nicht diskret wäre.
\end{itemize}
\end{Beweis}

\Kor{}
Ist $\Gamma$ diskret und torsionsfrei, so gelten folgende Eigenschaften:
\begin{itemize}
	\item Die Achsen zweier hyperbolischer Elemente aus $\Gamma$ sind entweder inzident oder ultraparallel.
	\item Jede Untergruppe von $\Gamma$, die isomorph zu $\Z\times \Z$ ist, muss von zwei parabolischen Elementen aufgespannt werden.
\end{itemize}

\Kor{}
Die Fundamentalgruppe einer geschlossenen hyperbolischen Mannigfaltigkeit kann keine Untergruppe enthalten, die isomorph zu $\Z\times \Z$ ist.

\section{Exkurs zu Lie-Gruppen und Lie-Algebren}

\Def{}
Eine \df{Lie-Gruppe} ist eine glatte Mannigfaltigkeit, die eine Gruppenstruktur trägt, durch die Inversion und Multiplikation zu glatten Abbildungen werden.

\Def{}
Eine \df{Lie-Algebra} $V$ ist ein reeller Vektorraum mit einer Bilinearform
\[ [,] : V\otimes_\R V \Pfeil{} V \]
die folgende Eigenschaften erfüllt:
\begin{enumerate}[1.)]
	\item \df{Anti-Symmetrie}: $[X,Y] = -[Y,X]$
	\item \df{Jacobi-Identität}: $[X,[Y,Z]] + [Z,[X,Y]] + [Y,[Z,X]] = 0$
\end{enumerate}


\Bem{}
Sei $G$ eine Lie-Gruppe, setze
\[ \g = T_eG \]
Dann ist $\g$, der Tangentialraum des Neutralelements von $G$, isomorph zum Raum der \df{links-invarianten Vektorfelder} auf $G$, d.\,h., der Vektorfelder $X$, für die gilt
\[ X(gp) = \d g_{|p}X(p) \]
Ergo können wir die Lie-Klammer der Vektorfelder auf $\g$ einschränken, wodurch $\g$ zu einer Lie-Algebra wird. Für $x\in\g$ definieren wir die \df{Adjunktion} durch
\begin{align*}
ad(x) : \g & \Pfeil{} \g\\
v & \longmapsto [x,v]
\end{align*}
Beachte, für jedes $x \in \g$ existiert genau ein glatter Gruppenmorphismus
\[ \Theta : \R \Pfeil{} G \]
sodass $\Theta'(0) = x, \Theta(0) = 1$.\\
Wir definieren hierdurch folgende \df{Exponentialabbildung}
\begin{align*}
\exp : \g &\Pfeil{} G\\
x &\longmapsto \Theta(1)
\end{align*}
Dann existiert ein Zusammenhang auf $G$, sodass $\exp$ der Riemannschen Exponential-Abbildung entspricht und jedes $\Theta$ eine Geodäte ist.\\
Ist $h\in G$, so definiere
\begin{align*}
Int(h) :G& \Pfeil{} G\\
g & \longmapsto hgh\i
\end{align*}
und die \df{adjungierte Aktion} durch
\[ Ad(h) := (\d Int(h))_{e} : \g \Pfeil{} \g \]
Dann gilt
\begin{align*}
\exp(Ad(h)x) &= h\exp(x)h\i\\
Ad(\exp(x)) &= e^{ad(x)}
\end{align*}

\Bem{}
Ist $G = \GL$, so ist $\g = \glf = \R^{n\times n}$ und $[x,y] = xy -yx$. 



\section{Zassenhaus}
\Def{}
Ist $U \subset G$ Teilmenge einer Gruppe, so definiere rekursiv
\begin{align*}
U_{(0)} &:= U\\
U_{(k)} &:= [U_{(k-1)}, U]
\end{align*}
Ist $U$ eine Umgebung der 1 in einer Lie-Gruppe $G$, so nennen wir $U$ eine \df{Zassenhaus-Umgebung}, falls
\begin{align*}
U_{(1)} &\subset U\\
U_{(k)} &\Pfeil{k \pfeil{{}} \infty} \{1\}
\end{align*}

\Prop{}
Jede reelle Lie-Gruppe besitzt eine Zassenhaus-Umgebung.
\begin{Beweis}{}
	\begin{itemize}
		\item Wir zeigen die Aussage zuerst für $GL_n(\R)$.\\
		Für $v \in \R^{n}$ bezeichne $\norm{v}$ die euklidische 2-Norm und für $A \in \R^{n\times n}$ bezeichne $\norm{A} := \sup_{v\in \R^n \setminus \{0\}} \frac{\norm{Av}}{\norm{v}}$ die dazugehörige Operatornorm. Insbesondere gilt für diese Norm
		\[ \norm{A B} \leq \norm{A} \cdot \norm{B} \]
		Definiere ferner
		\[ m(A) := \norm{I_n-A} \]
		Seien $A, B \in \R^{n\times n}$ mit $m(A) <1$ und $m(B) <1$. Schreibe $\alpha = A - I_n$ und $\beta = B - I_n$.\\
		Da $\norm{\alpha} < 1$, konvergiert folgende Reihe
		\[ A\i = \sum_{n = 0}^{\infty} (-\alpha)^n \]
		Ferner gilt
		\[ \norm{A\i v} \leq \sum_{n = 0}^{\infty} \norm{v}\norm{\alpha}^n = \frac{\norm{v}}{1 - \norm{\alpha}}  \]
		Ergo
		\[ \norm{A\i} \leq \frac{1}{1 - m(A)} \]
		Es folgt
		\begin{align*}
		m([A,B]) &= \norm{ [A,B] - I } \\
		&= \norm{ABA\i B\i - I}\\
		&= \norm{ ((A-I)(B-I) - (B-I)(A-I))A\i B\i }\\
		&\leq 2 \norm{A-I}\norm{B-I} \norm{A\i }\norm{B\i}\\
		& \leq 2 \frac{m(A)m(B)}{(1-m(A)) (1-m(B))}
		\end{align*}
		Definiere nun folgende Umgebung der Eins in $GL_n(\R)$
		\[ U:= \set{A\in \R^{n\times n}}{m(A) < \frac{1}{8}} \]
		Induktiv rechnet man dann für $n\geq 0$ für alle $C \in U_{(n)}$ nach
		\[ m(C) < \frac{1}{8} \klam{\frac{3}{8}}^n \]
		Ergo ist $U$ eine Zassenhaus-Umgebung.
		\item Ist $G$ eine Lie-Untergruppe von $GL_n(\R)$, so ist $U\cap G$ eine Zassenhaus-Umgebung für $G$.
		\item Ist $G$ irgendeine reelle Lie-Gruppe, so folgt aus Ados Theorem, dass eine Injektion von Lie-Algebren
		\[ \g = \T_eG \Inj{f} \mathfrak{gl}_n(\R) = T_eGL_n(\R) \]
		existiert. Es bezeichne $\mathcal{G}$ die Untergruppe von $GL_n(\R)$, deren Lie-Algebra gerade dass Bild von $f$ ist.\\
		Dann existiert eine lokale Isometrie
		\[ \phi : V \Pfeil{} W \]
		wobei $V \subset G, W \subset \mathcal{G}$ jeweils Umgebungen der Eins sind. Wir können nun eine Zassenhaus-Umgebung von $\mathcal{G}$ wählen, sie mit $W$ schneiden, und dann über $\phi$ zurückziehen.
	\end{itemize}
\end{Beweis}

\Def{}
Eine Gruppe heißt \df{nilpotent}, falls ein $n \in \N$ existiert mit
\[ G_{(n)} = 1 \]

\Prop{}
Sei $G$ eine Gruppe, die durch eine Menge $S$ erzeugt wird. Gilt für alle $a_1,\ldots, a_{n+1} \in S$
\[ [a_1, [a_2,  \ldots [a_n, a_{n+1} ] \ldots ]] = 1 \]
so folgt
\[ g_{(n)} = 1 \]

\Lem{}
Sei $G$ eine Lie-Gruppe. Dann existiert eine Umgebung $U$ der 1, sodass jede diskrete Untergruppe, die durch Elemente aus $U$ generiert wird, nilpotent ist.

\section{Margulis Lemma}

\Def{}
Sei $S$ eine Menge von Symbolen. $S^*$ bezeichne das durch $S$ frei erzeugte Monoid. Ferner sei ein surjektiver Homomorphismus $\pi : S^* \pfeil{} G$ in eine Gruppe $G$ gegeben.\\
Jedes Element aus $G$ besitzt dann eine Darstellung als ein \df{Wort} in $S^*$. Definiere folgende \df{Norm} auf $G$
\[ \norm{g}_S := \set{\bet{w}}{w\in S^*, \pi(w) = g} \]

\Lem{}
Es agiere $G$ transitiv auf einer Menge $A$, die mindestens $m+1$ Elemente habe. Für jedes $a \in A$ gibt es mindestens $m+1$ Elemente aus $G$ mit Norm $\leq m$, die $a$ auf $m+1$ verschiedene Elemente abbilden.

\Def{}
Ist $H \subset G$ eine Untergruppe endlichen Index, so heißt $G$ \df{virtuell} $H$.\\
Ist $H$ abelsch, nilpotent, etc., so heißt $G$ virtuell abelsch, nilpotent,etc..

\Lem{Margulis Lemma}
Sei $\Gamma \subset Isom(\H^n)$ diskret und $x \in \H^n$ beliebig. Für $\epsilon > 0$ setze
\[ \Gamma_\epsilon(x) := \set{\gamma \in \Gamma}{d(x, \gamma(x)) \leq \epsilon} \]
Dann existiert für jedes $n\geq 2$ eine sogenannte \df{Margulis-Konstante} $\epsilon_n > 0$, sodass für alle $x \in \H^n$ die Gruppe $\Gamma_{\epsilon_n}(x)$ virtuell nilpotent ist.
\begin{Beweis}{}
	\begin{itemize}
		\item $G = Isom(\H^n)$ ist eine Lie-Gruppe.
		Sei $U$ eine Zassenhaus-Umgebung der Eins in $G$. $V$ sei eine kompakte Umgebung von $Stab_G(x) \isom{} O(n-1)$ in $G$. Dann existiert eine Zahl $m$, sodass $V$ durch $m$ Translate von $U$ überdeckt wird.\\
		Sei ferner $W$ eine Umgebung von $Stab_G(x)$ in $G$ mit $W = W\i$ und $W^m \subset V$.\footnote{
			So ein $W$ kann mithilfe des verallgemeinerten Tuben-Lemmas gefunden werden: Seien Räume $A\subset X$ und $B \subset Y$ gegeben und eine offene Umgebung $N \subset X \times Y$ von $A\times B$. Sind $A$ und $B$ kompakt, so gibt es offene Mengen $U \subset X$ und $V \subset Y$ mit $A\times B \subset U \times V\subset N$.
		}
	\item Wir nehmen zuerst an, dass $\Gamma$ durch eine Menge $S \subset W$ erzeugt wird, und zeigen in diesem Fall, dass $\Gamma' := \Gamma \cap U$ einen Index von höchstens $m$ in $\Gamma$ hat.\\
	Betrachte hierzu die Wirkung von $\Gamma$ auf $\Gamma / \Gamma'$. Besitzt diese Menge mehr als $m$ Elemente, so gibt es $m+1$ Nebenklassen aus $\Gamma / \Gamma'$, die einen Repräsentanten mit Norm $\leq m$ haben.\\
	Da $S^{\leq m} \subset W^m \subset V$, liegen alle diese Repräsentanten in $V$. Da $V$ durch $m$ Translate von $U$ überdeckt wird, liegen mindestens zwei der $m+1$ Repräsentanten im denselben Translat. Deswegen repräsentieren diese beiden Elemente dieselbe Nebenklasse in $\Gamma / \Gamma'$, was einen Widerspruch zur Annahme, dass der Index von $\Gamma'$ in $\Gamma$ echt größer als $m$ ist, darstellt.
	\item Wir wollen nun $\epsilon_n > 0$ so wählen, dass gilt
	\[ \set{g \in G}{d(x, g(x)) \leq \epsilon_n} \subset W \]
Das geht, denn betrachte das Faserbündel
	\begin{align*}
	\pi : G & \Pfeil{} \H^n\\
	g & \longmapsto g(x)
	\end{align*}
	Die Faser von $x$ ist gerade $Stab_G(x) \isom{} O(n-1)$. Da $O(n-1)$ notorischerweise kompakt ist, ist $\pi$ eine eigentliche Abbildung\footnote{
Ein Faserbündel ist genau dann eigentlich, wenn ihre Faser kompakt ist. Eine Abbildung heißt eigentlich, wenn das Urbild kompakter Mengen wieder kompakt ist.	
}. Ergo ist auch die Menge
\[ K_l := \pi\i\set{y \in \H^n}{ d(x,y) \leq \frac{1}{l} } = \set{g \in G}{d(x, g(x)) \leq \frac{1}{l}} \]
kompakt. Ferner gilt
\[ \bigcap_l K_l = Stab_G(x) \subset W \]
Da die $K_l$ alle kompakt sind und $W$ eine Umgebung von $Stab_G(x)$ ist, existiert ein $l$ mit
\[ K_l \subset W \]
Ergo kann $\epsilon_n \leq \frac{1}{l}$ gewählt werden.
	\end{itemize}
	
\end{Beweis}

\Def{}
Eine nichttriviale, diskrete Untergruppe $\Gamma \subset Isom(\H^n)$ heißt \df{elementar}, wenn es eine endliche $\Gamma$-invariante Menge $\emptyset \neq S \subset \overline{\H^n}$ gibt, d.\,h.,
\[ \set{\phi(s) }{\phi \in \Gamma, s\in S} = S \]
Dies ist äquialent dazu zu fordern, dass es einen Punkt $x \in \overline{\H^n}$ gibt, dessen $\Gamma$-Orbit endlich ist.

\Prop{}
Eine elementare Gruppe $\Gamma$, die frei auf $\H^n$ operiert, wird entweder durch eine hyperbolische Isometrie oder durch beliebig viele parabolische Isometrien mitdemselben Fixpunkt erzeugt.
\begin{Beweis}{}
$\Gamma$ agiert frei und enthält deswegen kein elliptischen Elemente. Angenommen $S$ besäße Punkte in $\H^n$. Dann würden Elemente aus $\Gamma$ diese Punkte vertauschen, sprich, aufeinander spiegeln. Insbesondere könnte man gemeinsame Schwerpunkte finden, die von nichttriviale Elementen aus $\Gamma$ fixiert werden würden, was einen Widerspruch darstellen würde. Ergo kann $S$ nur aus Punkten im Unendlichen bestehen.\\
Hyperbolische und parabolische Elemente können Punkte im Unendlichen nicht zyklisch vertauschen, sondern nur fixieren bzw. unendlich oft bewegen. $S$ muss ergo in $\partial \H^n$ liegen und aus einem, oder zwei Elementen bestehen. Der Rest folgt nun aus Lemma \ref{Lemma 422}.
\end{Beweis}

\Prop{}
Sei $\Gamma \subset \H^n$ diskret, torsionsfrei und nichttrivial. $\Gamma$ ist genau dann elementar, wenn $\Gamma$ virtuell elementar ist.
\begin{Beweis}{}
	Sei $\Gamma' \subset \Gamma$ elementar und von endlichem Index. $\Gamma'$ besteht entweder aus hyperbolischen Elementen, die dieselbe Achse erhalten, oder aus parabolischen mit demselben Fixpunkt. Ist $\phi \in \Gamma$, so muss eine Potenz von $\phi^k$ in $\Gamma'$ liegen und ergo ebenfalls dieselbe Achse erhalten oder denselben Fixpunkt fixieren.
\end{Beweis}

\Kor{}
Sei $\Gamma \subset \H^n$ diskret und torsionsfrei. Ist $\Gamma $ virtuell nilpotent, so ist $\Gamma$ entweder trivial oder elementar.
\begin{Beweis}{}
	Sei $H \subset G$ von endlichem Index und nilpotent. Ist $H$ trivial, so ist $G$ endlich. Da $G$ aber auch torsionsfrei ist, muss $G$ somit auch trivial sein.\\
	Sei $H$ nicht trivial. Da $H$ nilpotent ist, existieren nichttriviale Elemente in $H$, die mit allen anderen kommutieren. Ergo müssen alle Elemente in $H$ dieselben Fixpunkte haben. Ergo ist $H$ elementar.
\end{Beweis}

\Kor{}
Sei $\Gamma$ diskret und torsionsfrei. Für jedes $x\in \H^n$ ist dann $\Gamma_{\epsilon_n}(x)$ entweder trivial oder elementar.

\section{Dick-Dünn-Zerlegung}

\Def{}
\begin{itemize}
	\item Eine \df{sternförmige Menge um einen Randpunkt} $p \in \partial\H^n$ ist eine Menge $U \subset \H^n$, die mit jedem Strahl, der zu $p$ läuft, einen nicht-leeren zusammenhängenden Schnitt besitzt.
	\item Eine \df{sternförmige Umgebung um eine Gerade} $l \subset \H^n$ ist eine Menge $U \subset \H^n$, die einen nicht-leeren zusammenhängenden Schnitt mit jeder Geodäte besitzt, die $l$ orthogonal schneidet.
	\item Sei $\Gamma$ eine Gruppe, die frei und diskontinuierlich agiert und aus parabolischen Isometrien bestehet, die einen Randpunkt $p\in \partial\H^n$ fixieren. Ist $U$ eine sternförmige Umgebung um $p$, die invariant unter $\Gamma$ ist, so heißt $U / \Gamma$ eine \df{sternförmige Spitzenumgebung}.
	\item Sei $\Gamma$ eine Gruppe, die durch eine hyperbolische Isometrie $\phi$ mit Achse $l$ erzeugt wird. Ist $V$ eine $\Gamma$-invariante sternförmige Umgebung um $l$, so heißt $V / \Gamma$ eine \df{sternförmige Umgebung einer einfachen geschlossenen Geodäten} der Länge $d(\phi)$.
\end{itemize}
\Bem{}
Die beschränkten Tuben und Spitzen stellen Beispiele für sternförmige einfache geschlossene geodätische Umgebungen bzw. sternförmige Spitzenumgebungen dar.

\newcommand{\thick}{M_{[\epsilon_n, \infty)}}
\newcommand{\thin}{M_{(0,\epsilon_n]}}
\Def{}
Sei $M = \H^n / \Gamma$ eine Mannigfaltigkeit. $\epsilon_n$ sei eine Margulis-Konstante. Definiere den \df{dicken Teil} von $M$ durch
\[ \thick := \set{ x \in M }{inj_x(M) \geq \frac{\epsilon_n}{2}} \]
und den \df{dünnen Teil} durch
\[ \thin := \overline{ M/\thick } \]

\Satz{Dick-Dünn-Zerlegung}
Sei $M$ eine vollständige, hyperbolische Mannigfaltigkeit der Dimension $n$.\\
Der dünne Part $\thin$ besteht aus einer disjunkten Vereinigung von sternförmigen Spitzenumgebungen und sternförmigen Umgebungen von geschlossenen Geodäten der Länge $< \epsilon_n$.
\begin{Beweis}{}
	\begin{itemize}
		\item 	Sei $M = \H^n / \Gamma$. Definiere für $\phi \in \Gamma, \epsilon > 0$
		\[ S_\phi(\epsilon) = \set{x \in \H^n}{d(x, \phi(x)) \leq \epsilon} \]
		Dann ist der dünne Teil gerade der Quotient von
		\[ S := \bigcup_{\phi \in \Gamma} S_\phi(\epsilon_n)  \]
		\item Ist $\phi$ eine Parabolische, die $p$ fixiert, so ist $S_\phi(\epsilon)$ eine sternförmige Umgebung um $p\in\partial \H^n$.
		\item Ist $\phi$ eine Hyperbolische mit Achse $l$, so ist $S_\phi(\epsilon)$ eine sterförmige Umgebung um $l$, falls $\epsilon > d(\phi)$.
		\item Es bleibt zu zeigen, dass $S_\phi(\epsilon_n)$ und $S_\psi(\epsilon_n)$ disjunkt sind für verschiedene $\phi, \psi$.\\
		Sei $x \in S_\phi(\epsilon_n)\cap S_\psi(\epsilon_n)$. Dann gilt
		\[ \phi, \psi \in \Gamma_{\epsilon_n}(x) \]
		Margulis Lemma sagt aus, dass $\Gamma_{\epsilon_n}(x)$ elementar ist. Ergo sind $\phi$ und $\psi$ entweder beide hyperbolisch und Potenz desselben Elements oder parabolisch und fixieren denselben Punkt im Unendlichen.
		\item Ist nun $S_0 \subset S$ eine Zusammenhangskomponente, so existiert eine maximale elementare Untergruppe $\Gamma_0 \subset \Gamma$, sodass
		\[S_0 = \bigcup_{\phi \in \Gamma_0\setminus \{ \id{\H^n}\}} S_\phi(\epsilon_n) \]
		$S_0$ ist nun die Vereinigung von sternförmigen Nachbarschaften um $l$ bzw. $p$, und deswegen vom selben Typ. 
	\end{itemize}
\end{Beweis}

\Prop{}
Sei $M$ eine vollständige, orientierbare hyperbolische Mannigfaltigkeit der Dimension $\leq 3$. Der dünne Teil von $M$ besteht aus beschränkten Spitzen und Tuben endlicher Länge.
\begin{Beweis}{}
	\begin{itemize}
		\item Sei $\phi \in Isom^+(\H^n)$. Es genügt zu zeigen, dass $S_\phi(\epsilon)$ entweder leer, oder eine $R$-Nachbarschaft um $l$ oder eine Horokugel um $p$ für alle $\epsilon > 0$ ist.
		\item Sei also $\phi$ hyperbolisch mit Achse $l$. Dann hängt $d(x,\phi(x))$ stetig und monoton von $d(x,l)$ ab. Ergo gilt
		\[ S_\phi(\epsilon) = \set{ x\in\H^n }{d(x,\phi(x)) \leq \epsilon} = \set{x\in \H^n}{d(x,l) \leq s(\epsilon)} \]
		\item Ist $\phi$ parabolisch, so agiert $\phi$ auf jeder Horosphäre um $p$ wie eine euklidische fixpunktfreie orientierungserhaltende Isometrie. Für $n\leq 3$ kann dies nur eine Translation sein. Es folgt
		\[ S_\phi(\epsilon) = \set{ x\in\H^n }{d(x,\phi(x)) \leq \epsilon} \isom{} \set{(x_1, t)\in H^n}{ t > s(\epsilon) } \]
	\end{itemize}
\end{Beweis}

\Prop{}
Sei $M$ eine vollständige hyperbolische Mannigfaltigkeit. Dann sind folgende Aussagen äquivalent:
\begin{enumerate}[(1)]
	\item $M$ hat endliches Volumen.
	\item $M_{[\epsilon, \infty)}:= \set{x\in M}{ inj_x(M) \geq \frac{\epsilon}{2} }$ ist kompakt für alle $\epsilon > 0$.
	\item $\thick$ ist kompakt.
\end{enumerate}
\begin{Beweis}{}
	\begin{itemize}
		\item Von (1) nach (2):\\
		Sei $S$ eine maximale Teilmenge von $M_{[\epsilon,\infty)}$ mit der Eigenschaft, dass je zwei Punkte aus einen Mindestabstand von ${\epsilon}$ haben. Es gilt dann
		\[ M_{[\epsilon, \infty)} \subset \bigcup_{s \in S} \overline{B_{\epsilon}(s)} =: D \]
		Andererseits müssen die offenen Bälle mit Radius $\frac{\epsilon}{2}$ und Zentrum $s \in S$ alle disjunkt sein. Da $M$ nur endliches Volumen hat, kann $S$ somit nur endlich viele Elemente enthalten. Dadurch ist $D$ kompakt und $ M_{[\epsilon, \infty)}$ als abgeschlossene Teilmenge eines Kompakti wieder kompakt.
		\item Von (3) nach (1):\\
		Sei $\thick$ kompakt. Dann ist auch der Rand von $\thick$ kompakt. Ergo hat $\thin$ nur endliche viele Zusammenhangskomponenten. Es genügt ergo zu zeigen, dass eine sternförmige Umgebung um eine Spitze bzw. eine geschlossene Geodäte aus $\thin$ endliches Volumen hat. Dies ist erfüllt, wenn eine solche Komponente kompakt ist.
		\item Eine abgeschlossene Umgebung um eine geschlossene Geodäte ist immer kompakt, da sie homöomorph zu $I \times S^1$ ist.
		\item Sei ergo $M_0$ eine nicht kompakte Zusammenhangskomponente von $\thin$. Sei $S_0$ das dazugehörige Urbild in $\H^n$. Sei $\Gamma_0 < \Gamma$ eine maximale elementare Untergruppe, die nur aus Parabolischen besteht, die $\infty$ fixieren, sodass
		\[ S_0 = \bigcup_{\phi \in \Gamma_0 \setminus \{ \id{\H^n} \}}\set{ x\in \H^n }{ d(x,\phi(x) ) \leq \epsilon_n } \]
		Sei eine \df{Blätterung} auf $S_0$ gegeben, d.\,h., 
		eine Übderdeckung von $S_0$ durch geodätische Halbstrahlen, die jeweils \df{Blätter} der Blätterung genannt werden. $\Gamma_0$ erhält diese Foliation, weswegen wir eine Foliation in $M_0$ erhalten.
		\item Jedes Blatt beginnt in einem Punkt in $\partial \thick$, welches kompakt ist. Wir wählen eine zusammenhängende Menge $P \subset \H^n$, sodass der Anfangspunkt jedes Blattes in $M_0$ einmal in $P$ vertreten ist. Das Volumen von $M_0$ errechnet sich nun durch die Fläche, die von $P$ und $\infty$ aufgespannt ist und endlich ist.
	\end{itemize}
\end{Beweis}

\Kor{}
Jede vollständige, hyperbolische Mannigfaltigkeit endlichen Volumens ist diffeomorph zum Inneren einer kompakten Mannigfaltigkeit mit Rand. Der Rand besteht hierbei aus flachen Mannigfaltigkeiten.
\begin{Beweis}{}
	Sei $U$ eine sternförmige Spitzenumgebung in $M$. $U$ hat eine kompakte Basis $X$ und enthält eine beschränkte Spitze diffeomorph zu $X \times [t, \infty)$. Schmeißt man alle solche beschränkten Spitzen aus $M$ raus, bleibt was kompaktes mit Rand übrig: Der dicke Teil plus kompakten Umgebungen von geschlossenen Geodäten und kompakten Teilen von Spitzenumgebungen.\\
	Die beschränkte Spitzenumgebung lässt sich zu $X\times [t,\infty]$ kompaktieren. $X$ entspricht einer Horosphäre in $\H^n$ und ist somit flach.
\end{Beweis}

\Bem{}
\begin{itemize}
	\item Ist in obiger Situation $n = 2$, so ist $M$ das Innere einer kompakten Fläche mit Rändern diffeomorph zu $S^1$.
	\item Ist $n = 3$ und $M$ orientierbar, so sind die Ränder von $N$ flache Tori.
	\item Wie die Ränder für allgemeine Fälle aussehen, beantworten folgende Theoreme.
\end{itemize}

\Satz{}
\label{IsomRn}
Jede diskrete Teilgruppe von $Isom(\R^n)$ ist virtuell abelsch.
\Satz{Bieberbach}
Jede geschlossene vollständige flache Mannigfaltigkeit wird durch endlich viele Tori überdeckt.

\chapter{Mostows Rigiditäts-Theorem}
\section{Inside group! Outside group! Which side group? I don't know!}
\Def{}
Seien $H\subset G$ Gruppen. Definiere den \df{Normalisierer} von $H$ durch
\[ N(H) := \set{g \in G}{ gH = Hg } \]
Der Normalisierer ist die größte Untergruppe von $G$, in der $H$ normal ist. Insbesondere ist $N(H) /H$ wohldefiniert.

\Prop{}
Sei $M = \H^n / \Gamma$ eine hyperbolische Mannigfaltigkeit. Dann gilt
\[ Isom(M) \isom{} N(\Gamma) / \Gamma \]
\begin{Beweis}{}
	Sei $\phi_M : M \pfeil{} M$ eine Isometrie. Wir erhalten folgendes kommutative Diagramm
		\begin{center}
		\begin{tikzpicture}
		\node (M1) at (0,2) [] {$\H^n $};
		\node (H1) at (4,2) [] {$\H^n$};
		\node (H2) at (4,0) [] {$M$};
		\node (M2) at (0,0) [] {$M$};
		
		\draw [->, dashed] (M1) -> (H1) node[midway, above] {$\phi_{\H^n}$};
		\draw [->] (M2) -> (H2) node[midway, above] {$\phi_M$};
		\draw [->] (M1) -> (M2) node[midway, right] {};
		\draw [->] (H1) -> (H2) node[midway, right] {};
		\end{tikzpicture}
	\end{center}
Aus der Kommutativität folgt zwangsläufig $\phi_{\H^n} \in N(\Gamma)$. Ferner ist $\phi_{\H^n}$ bis auf ein Element aus $\Gamma$ eindeutig. Umgekehrt steigt jedes Element aus $N(\Gamma)/\Gamma$ auf $M$ ab. 
\end{Beweis}

\Def{}
Das \df{Zentrum} von $H\subset G$ besteht aus allen Elementen aus $G$, die mit allen Elementen aus $H$ kommutieren, d.\,h.
\[ Z(H) := \set{g \in G}{gh = hg \forall h \in H} \]



\Def{}
Sei $G$ eine Gruppe.
\begin{itemize}
	\item $Aut(G)$ bezeichne die Gruppe aller Gruppenisomorphismen vom Typ $G\pfeil{} G$.
	\item Definiere die Gruppe der \df{inneren Automorphismen} durch
	\[ Int(G) := \set{h \mapsto ghg\i}{g \in G} \]
	Diese Gruppe ist ein Normalteiler von $Aut(G)$.
	\item Definiere die Gruppe der \df{äußeren Automorphismen} durch
	\[ Out(G) := Aut(G) / Int(G) \]
\end{itemize}

\Lem{}
Es sei $X$ ein zusammenhängender Raum. $Homöo(X)$ bezeichne die Menge der Homöomorphismen auf $X$. Dann erhalten wir folgenden Homomorphismus von Gruppen
\begin{align*}
\nu : Homöo(X) & \Pfeil{} Out(\pi_1(X,x_0))\\
f & \longmapsto [x_0 \mapsto f(x_0)]_* \circ f_*
\end{align*}
wobei $x_0 \mapsto f(x_0)$ einen Weg von $x_0$ nach $f(x_0)$ bezeichnet.\\
$\nu(f)$ ist hängt nur von der Homotopieklasse von $f$ ab.
\begin{Beweis}{}
	\begin{itemize}
		\item Ein Homöomorphismus $f : X \pfeil{} X$ induziert uns einen Isomorphismus
		\begin{align*}
		f_* : \pi_1(X,x_0) & \Pfeil{} \pi_1(X,f(x_0))\\
		\gamma & \longmapsto f\circ \gamma
		\end{align*}
		\item Ist $\alpha$ ein Weg, der $x_0$ mit $f(x_0)$ verbindet, so erhalten wir einen Isomorphismus
		\begin{align*}
		\alpha_* : \pi_1(X,f(x_0)) & \Pfeil{} \pi_1(X,x_0)\\
		\gamma & \longmapsto \alpha\i \gamma \alpha
		\end{align*}
		\item Ergo ist $\alpha_* \circ f_*$ ein Automorphismus von $\pi(X,x_0)$.
		\item Dieser Automorphismus ist eindeutig bis auf Elemente aus $Int(\pi(X,x_0))$.
	\end{itemize}
\end{Beweis}

\Prop{}
Ist $M$ eine hyperbolische, vollständige Mannigfaltigkeit endlichen Volumens, so ist
\[ \nu : Isom(M) \Pfeil{} Out(\pi_1(M)) \]
injektiv.
\paragraph{Bemerkung} Für $n\geq 3$ ist $\nu$ sogar ein Isomorphismus.
\begin{Beweis}{}
	Sei $M = \H^n /\Gamma$. Identifiziere $\pi_1(M)$ mit $\Gamma$ und $Isom(M) = N(\Gamma) / \Gamma$. Wir erhalten
	\begin{align*}
	\nu : N(\Gamma) / \Gamma &\Pfeil{} Out(\Gamma)\\
	g & \longmapsto [ \gamma \mapsto g\i \gamma g ]
	\end{align*}
Sei $g \in N(\Gamma)$, sodass ein $\delta \in \Gamma$ existert, sodass für alle $\gamma\in \Gamma$ gilt
\[ g\i \gamma g = \delta\i \gamma \delta \]
d.\,h., für alle $\gamma$ gilt
\[ (\delta g\i) \gamma = \gamma (\delta g\i) \]
D.\,h., $\delta g\i$ liegt im Zentrum von $\Gamma$ in $Isom(\H^n)$. Dieses ist trivial, ergo gilt $g = \delta  \in \Gamma$, ergo ist $\nu$ injektiv.
\end{Beweis}

\Kor{}
Sei $M$ eine hyperbolische, vollständige Mannigfaltigkeit endlichen Volumens. Dann sind verschiedene Isometrien auf $M$ nicht homotop zueinander.

\section{Limes Geschichten}
\Def{}
Sei $\Gamma \subset Isom(\H^n)$ nichttrivial und diskret. Sei $x \in \H^n$. Die \df{Limes-Menge} von $\Gamma$ ist definiert als alle Häufungspunkte von $\Gamma.x$ im Unendlichen, d.\,h.
\[ \Lambda(\Gamma) := \overline{\Gamma.x} \cap \partial \H^n \]
\Bem{}
\begin{itemize}
	\item Die Limes-Menge ist unabhängig von der Wahl von $x$.
	\item Ist $\Gamma$ virtuell $\Gamma'$, so gilt $\Lambda(\Gamma) = \Lambda(\Gamma')$.
	\item Ist $\Gamma$ elementar und torsionsfrei, so besteht $\Lambda(\Gamma) = Fix(\Gamma)$ aus einem oder zwei Punkten.
\end{itemize}

\Prop{}
Sei $\Gamma \subset Isom(\H^n)$ nichttrivial und diskret. Folgende Aussagen sind äquivalent:
\begin{enumerate}[(1)]
	\item $\Gamma$ ist elementar.
	\item $\Gamma$ erhält entweder einen Punkt in $\H^n$, oder eine Gerade in $\H^n$, oder einen Punkt in $p \in \partial \H^n$ und jede Horosphäre um $p$.
	\item $\Gamma$ ist virtuell abelsch.
	\item $\Lambda(\Gamma)$ besteht aus null, einem oder zwei Punkten.
\end{enumerate}
\begin{Beweis}{}
	\begin{enumerate}
		\item[(1) $\impl{}$ (2)] $\Gamma$ erhalte also eine endliche Menge $S$ an Punkten in $\overline{\H^n}$. Liegen einige davon in $\H^n$, so fixiert $\H^n$ deren Schwerpunkt.\\
		Wir können also annehmen, dass $S\subset \partial \H^n$. Enthält $S$ drei Punkte oder mehr, so lässt sich zu diesen wieder ein Schwerpunkt in $\H^n$ konstruieren, der von $\Gamma$ wieder fixiert werden muss.\\
		Ist $S$ zweielementig, so erhält $\Gamma$ die korrespondierende geodätische Achse.\\
		Sei $S = \{x\}$ ergo einelementig. Wenn $\Gamma$ ein hyperbolisches Element enthalten würde, so müssten andere hyperbolische und elliptische in $\Gamma$ dieselbe Achse erhalten, da $\Gamma$ diskret ist. Ergo wäre dann $S$ zweielementig.\\
		Ergo kann $\Gamma$ nur aus Parabolischen und Elliptischen bestehen, die $x$ fixieren. Diese erhalten jede Horosphäre um $x$.
		\item[(2) $\impl{}$ (3)] $\Gamma$ erhält eine Menge von Punkten im Unendlichen. Sei $x$ aus dieser Menge und $\Gamma' = \set{g\in \Gamma}{g(x) = x}$. Enthält $\Gamma$ keine Hyperbolischen, so ist es  virtuell $\Gamma'$, ansonsten ist es virtuell ein semi-direktes Produkt aus $\Gamma'$ und $\Z$.\\
		$\Gamma'$ erhält Horosphären und lässt sich deswegen als eine diskrete Untergruppe von $Isom(\R^{n-1})$ auffassen. Nach Satz \ref{IsomRn} ist $\Gamma'$ deswegen virtuell abelsch.
		\item[(3) $\impl{}$ (4)] Wir können annehmen, dass $\Gamma$ abelsch ist. Lemma \ref{Kommutieren} lässt sich für elliptische Isometrien erweitern, weswegen Punkt (2) und somit auch (4) folgt.
		\item[(4) $\impl{}$ (1)] $\Lambda(\Gamma)$ besteht aus $\Gamma$-Orbiten. Falls $\Lambda(\Gamma)$ leer ist, so fixiert $\Gamma$ einen Punkt in $\H^n$ wegen Brouwers Fixpunktsatz. Da $\Gamma$ diskret ist, muss $\Gamma$ endlich sein.
	\end{enumerate}
\end{Beweis}

\Def{}
Eine Gruppe $G$ agiere auf einem topologischen Raum $X$. Eine geschlossene, nicht-leere, $G$-invariante Menge $A \subset X$ heißt \df{minimal}, falls keine abgeschlossene, nicht-leere, $G$-invariante Teilmenge $B \subsetneq A$ existiert.

\Prop{}
Sei $\Gamma \subset Isom(\H^n)$ nichttrivial, nicht elementar und diskret.
Dann ist $\Lambda(\Gamma)$ minimal.
\begin{Beweis}{}
	Sei $S \subset \Lambda(\Gamma)$ nicht-leer und $\Gamma$-invariant. $S$ muss unendlich viele Punkte enthalten, da $\Gamma$ nicht elementar ist. $C(S)$ bezeichne die konvexe Hülle von $S$, die ihrerseits wieder $\Gamma$-invariant ist. Sei $x \in C(S) \cap \H^n$, dann gilt
	\[ \Lambda(\Gamma) = \overline{\Gamma.x} \cap \partial \H^n \subset C(S) \cap \partial \H^n = S \] 
\end{Beweis}

\Kor{}
Sei $\Gamma \subset Isom(\H^n)$ nichttrivial, nicht elementar und diskret. $\Gamma' \subset \Gamma$ sei eine unendliche normale Untergruppe. Dann gilt $\Lambda(\Gamma') \subset \Lambda(\Gamma)$
\begin{Beweis}{}
	Da $\Gamma'$ normal ist, schickt $g \in \Gamma$ den Orbit $\Gamma.x$ auf $\Gamma.g(x)$, weswegen $\Lambda(\Gamma')$ stabil unter $\Gamma$ bleibt. Da $\Gamma'$ unendlich und $\overline{\H^n}$ kompakt ist, ist $\Lambda(\Gamma')$ nicht leer. Da $\Lambda(\Gamma)$ minimal ist, gilt $\Lambda(\Gamma) = \Lambda(\Gamma')$.
\end{Beweis}

\Lem{}
Sei $C \subset \overline{\H^n}$ abgeschlossen, konvex und nicht leer. Für $x \in \H^n$ bezeichne $r(x)$ den nächsten Punkt in $C$. Für $x\in \partial\H^n$ bezeichne $r(x)$ den Punkt in $C$, der als erstes von einer Horosphäre um $x$ tangiert wird. $r : \overline{\H^n} \Pfeil{} C$ ist wohldefiniert.
\begin{Beweis}{}
	Sei $x\in {\H^n}$. Da $C$ nicht leer ist, gibt es möglich Kandidaten für $r(x)$. Wir nehmen uns zwei mögliche Kandidaten $y,z \in C$ für $r(x)$ heraus und konstruieren das geodätische Dreieck $\Delta_{xyz}$. Da $C$ konvex ist, liegt die Seite $\overline{yz}$ in $C$. Auf dieser Seite muss sich ein Punkt $x_1$ befinden, der von allen Punkten auf dieser Seite $x$ am nächsten ist.\\
	Dieser Punkt ist eventuell nicht $r(x)$, aber indem wir die Konstruktion sukezessive durchführen, erhalten wir eine konvergente Folge $(x_n)_n$, deren Grenzwert $r(x)$ ist.
\end{Beweis}

\Lem{}
$r$ ist stetig auf $\H^n \cup (\partial \overline{\H^n} \cap C)$.
\begin{Beweis}{}
	Seien $x,y \in D^n \cup (\partial \overline{D^n} \cap C) \setminus C$ verschieden. Wir wollen Folgendes zeigen
	\[ \norm{r(x) - r(y)} \leq \norm{x-y} \]
	wobei $\norm{\cdot}$ die euklidische Norm bezeichnet. Die geodätischen Segmente $[x,r(x)]$ und $[y,r(y)]$ stehen orthogonal zu $[r(x), r(y)]$. Deswegen gilt die Gleichung in $D^n$.
\end{Beweis}

\Def{}
Sei $\Gamma \subset Isom(\H^n)$ nicht trivial und diskret. $conv(\Lambda(\Gamma)) \cap \H^n$ ist dann eine konvexe, $\Gamma$-invariante, abgeschlossene Teilmenge.\\
Definiere den \df{konvexen Kern} von $M = \H^n / \Gamma$ durch
\[ \klam{conv(\Lambda(\Gamma)) \cap \H^n} / \Gamma \]

\Bem{}
Im allgemeinem Fall liefert $r$ einen Deformationsretrakt von $\H^n$ auf $C$.\\
In obigem Fall liefert $r$ einen $\Gamma$-invarianten Deformationsretrakt, weswegen wir einen Deformationsretrakt von $M$ auf seinen konvexen Kern erhalten. Deswegen ist $M$ homotopie-äquivalent zu seinem konvexen Kern.


\Def{}
Sei $\Gamma \subset Isom(\H^n)$ nicht trivial und diskret. Wir definieren den \df{Bereich der Diskontinuität} durch
\[ \Omega(\Gamma) := \partial \H^n \setminus \Lambda(\Gamma) \]

\Lem{}
Die Wirkung von $\Gamma$ auf $\H^n \cup \Omega(\Gamma)$ ist eigentlich diskontiniuerlich.
\begin{Beweis}{}
	Da $\Gamma$ diskret ist, ist die Wirkung auf $\H^n$ sowieso eigentlich diskontinuierlich. $r$ retrahiert $\H^n \cup \Omega(\Lambda)$ stetig zu $conv(\Lambda(\Gamma)) \setminus \Lambda(\Gamma)$ und kommutiert mit Elementen aus $\Gamma$.\\
	Sei $K \subset \H^n \cup \Omega(\Lambda)$ kompakt. Wir wollen zeigen, dass es nur endlich viele $g \in \Gamma$ gibt mit
	\[ g.K \cap K \neq \emptyset \]
	Da $K$ kompakt ist, ist es auch $r(K) \subset conv(\Lambda(\Gamma)) \setminus \Lambda(\Gamma)$. Gilt $g.K \cap K \neq \emptyset$, so folgt $g.r(K) \cap r(K) \neq \emptyset$. Da $\Gamma$ eigentlich diskontinuierlich auf $\H^n$ agiert, kann es für die zweite Schnittbedingung aber nur endlich viele $g$ geben.
\end{Beweis}

\Lem{}
Für jedes $x\in \Omega(\Gamma)$ gibt es eine offene Umgebung $U\subset \overline{\H^n}$, sodass für alle $g\in \Gamma$ entweder
\[ g.U = U \text{ falls } g(x) = x \]
oder
\[ g.U \cap U =\emptyset \]
gilt.
\begin{Beweis}{}
	Da $\Gamma$ eigentlich diskontinuierlich auf $\H^n\cup \Omega(\Gamma)$ wirkt, gibt es einen geschlossenen Halbraum $H \subset \H^n \cup \Omega(\Gamma)$ mit $H \cap g.H = \emptyset$, falls $g(x)\neq x$. Da $Stab_\Gamma(x)$ endlich ist, können wir einfach
	\[ U:= \bigcap_{g\in Stab_\Gamma(x)} g.H^o \]
	setzen.
\end{Beweis}

\Prop{}
Gilt $Vol(\H^n / \Gamma) < \infty$, so folgt $\Lambda(\Gamma) = \partial\H^n$.
\begin{Beweis}{}
	Sei $x \in \Omega(\Lambda)$. $U$ sei wie aus obigen Lemma. Zum Einen gilt
	\[ (U\cap \H^n) / Stab_\Gamma(x) \subset \H^n / \Gamma \]
	Es gilt aber
	\[ Vol((U\cap \H^n) / Stab_\Gamma(x)) \geq  \frac{Vol(U)}{\# Stab_\Gamma(x)} = \infty \]
	Da $Stab_\Gamma(x)$ endlich ist.
\end{Beweis}

\Kor{}
Ist $Vol(\H^n / \Gamma)$ endlich, so enthält $\Gamma$ keine nichttriviale, virtuell abelsche, normale Untergruppe.
\begin{Beweis}{}
	Sei $H \subset \Gamma$ normal und virtuell abelsch. Ist $H$ unendlich, dann gilt
	\[ \Lambda(H) = \Lambda(\Gamma)= \partial \H^n \]
	Andererseits muss $\Lambda(H)$ null, eins oder zwei Elemente enthalten, da $H$ virtuell abelsch ist. Ergo erhalten wir für diesen Fall einen Widerspruch.\\
	Wir können also annehmen, dass $H$ endlich ist. In diesem Fall ist $Fix(H)$ ein nicht-leerer echter Unterraum von $\H^n$. Da $H$ normal ist, agiert $\Gamma$ auf $Fix(H)$. Es folgt
	\[ \Lambda(\Gamma) \subset \partial Fix(H) \]
	Ein Widerspruch!
\end{Beweis}

\Kor{}
Die Fundamentalgruppe einer hyperbolischen, vollständigen Mannigfaltigkeit endlichen Volumens ist nicht auflösbar.

\Def{}
Ein Punkt aus $\partial \H^n$ heißt \df{parabolisch} bzw. \df{hyperbolisch} für $\Gamma$, falls er von einer Parabolischen bzw. einer Hyperbolischen aus $\Gamma$ fixiert wird.

\Prop{}
Sei $\Gamma \subset Isom(\H^n)$ nicht elementar und diskret. Dann ist die Menge der hyperbolischen bzw. der parabolischen Punkte entweder dicht in $\Lambda(\Gamma)$ oder leer.
\begin{Beweis}{}
	$\Lambda(\Gamma)$ ist minimal und die Menge der parabolischen bzw. hyperbolischen Punkte ist $\Gamma$ invariant.
\end{Beweis}

\Kor{}
Besitzt $\H^n / \Gamma$ endliches Volumen, so ist die Menge aller hyperbolischen bzw. parabolischen Punkte entweder leer oder dicht in $\partial \H^n$.

\Lem{}
Sei $\Gamma \subset Isom(\H^n)$ diskret, nicht elementar und endlich erzeugt. $\Gamma$ soll keinen $m$-Raum von $\H^n$ invariant lassen für $m < n- 1$.\\
Dann ist $N(\Gamma)\subset Isom(\H^n)$ diskret.
\begin{Beweis}{}
\begin{itemize}
	\item 	Es seien $g_1,\ldots, g_k$ Erzeuger von $\Gamma$. $x \in D^n$ sei ein Punkt, der von keinem Element aus $\Gamma$ außer der Identität fixiert wird. Setze
	\[ s = d(x, \Gamma.x \setminus \{x\}) \]
	und
	\[ U = \set{\phi \in Isom(D^n)}{d(g_ix, \phi(g_ix)) < \frac{s}{2} \forall i } \]
	$U$ ist eine offene Umgebung der Eins in $Isom(\H^n)$. Wir wollen zeigen, dass $N(\Gamma) \cap U$ (fast) trivial ist.
	\item Sei $h \in N(\Gamma) \cap U$. Dann
	\begin{align*}
	d(g_i\i h\i g_i h x, x) = d(g_i h x, h g_i x) \leq d(g_i h x, g_i x) + d(g_i x, h g_i x) < d(hx, x) + \frac{s}{2} < s
	\end{align*}
	Ergo gilt $g_i\i h\i g_i h x = x$ für alle $i$, d.\,h.
	\[ g_i h = h g_i \]
	D.\,h., $h$ kommutiert mit jedem Element aus $\Gamma$.
	\item Sei $y \in \Lambda(\Gamma)$ und $f_i \in \Gamma$ mit
	\[ f_i(x) \Pfeil{i\pfeil{} \infty} y  \]
	Da
	\[ d(f_i x, hf_i x) = d(f_i x, f_i h x ) = d(x, hx) \]
	gilt
	\[  hy = y  \]
	D.\,h., $h$ ist die Identität auf $\Lambda(\Gamma)$.
	\item Sei $m$ minimal, sodass $\Lambda(\Gamma)$ in einer $m-1$-Sphäre in $\partial D^n$ enthalten ist. Ohne Beschränkung der Allgemeinheit gilt
	\[ \Lambda(\Gamma) \subset S^{m-1} \]
	Da $\Gamma$ die konvexe Hülle von $\Lambda(\Gamma)$ erhält, erhält $\Gamma$ auch die affine Hülle, die gerade $\overline{D^m}$ ist. Mit der Voraussetzung folgt nun $m \in \{n-1,n\}$.
	\item Sei $m = n$. Dann fixiert $h$ ganz $\partial S^{n-1}$, ergo $h = 1$.
	\item Sei $m = n-1$. $h$ muss $D^{n-1}$ fixieren, ergo ist $h$ entweder die Identität oder die Spiegelung $\rho$ entlang $D^{n-1}$. D.\,h.,
	\[ U \cap N(\Gamma) = \{1, \rho\} \]
	Ergo ist $N(\Gamma)$ diskret.
\end{itemize}
\end{Beweis}

\Kor{}
Es sei $M = \H^n / \Gamma$ eine vollständige hyperbolische Mannigfaltigkeit endlichen Volumens. Dann ist $Isom(M)$ endlich.
\begin{Beweis}{}
	\begin{itemize}
		\item $\Gamma$ ist endlich erzeugt, da $M$ diffeomorph zum Inneren einer kompakten Mannigfaltigkeit mit Rand ist und $\pi_1(M) \isom{} \Gamma$.
		\item Da $\Lambda(\Gamma) = \partial \H^n$ minimal ist, erhält $\Gamma$ keinen echten Unterraum von $\H^n$.
		\item Es ist ergo $N(\Gamma)$ diskret und deswegen verschwindet $Vol(\H^n / N(\Gamma))$ nicht. Es gilt nun
		\[ \# Isom(M) = [N(\Gamma) : \Gamma] = Vol(\H^n /\Gamma) / Vol(\H^n / N(\Gamma)) < \infty \]
	\end{itemize}
\end{Beweis}
 

\chapter{Aufgaben}
\Lem{}
Sei $M = \H^n / \Gamma$ eine Mannigfaltigkeit endlichen Volumens. Dann ist das Zentrum von $\Gamma$ in $Isom(\H^n)$ trivial.
\begin{Beweis}{}
	Sei $h \in Z(\Gamma)$ nichttrivial. Dann gilt für alle nichttriviale $\phi \in \Gamma$
	\[ Fix(\phi) = Fix(h) \]
	D.\,h., $\Gamma$ wird von einer Hyperbolischen erzeugt oder von mehreren Parabolischen, die denselben Punkt im Unendlichen fixieren. Im ersten Fall ist $M$ eine unendliche Tube, im zweiten Fall eine unbegrenzte Spitze. Beide Mannigfaltigkeiten haben kein endliches Volumen.
\end{Beweis}




\printindex
\end{document}