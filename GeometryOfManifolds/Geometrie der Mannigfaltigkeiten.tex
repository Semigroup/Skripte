\documentclass{book}

\usepackage{../Package/latexa}
\usepackage{../Package/algebra}
\usepackage{../Package/theorema}
\usepackage{../Package/diagramma}
\usepackage{../Package/categoria}

\newcommand{\qi}{\backsimeq_{\textsc{QI}}}
\newcommand{\ba}{\backsimeq_{\textsc{BA}}}
\newcommand{\normal}{\vartriangleleft}
\newcommand{\tm}{\subset}
\newcommand{\Stab}[2]{\textsf{Stab}_{#1}(#2)}
\newcommand{\Cay}[2]{\textsf{Cay}(#1,#2)}


\renewcommand{\A}{\mathbb{A}}
\newcommand{\Nc}{\mathcal{N}}


\renewcommand{\d}{\textsf{d}}
\renewcommand{\P}{\mathbb{P}}

\newcommand{\af}{\mathfrak{a}}
\newcommand{\Af}{\mathfrak{A}}
\renewcommand{\bf}{\mathfrak{b}}
\newcommand{\Bf}{\mathfrak{B}}
\newcommand{\cf}{\mathfrak{c}}
\newcommand{\Cf}{\mathfrak{C}}
\newcommand{\ff}{\mathfrak{f}}
\newcommand{\Ff}{\mathfrak{F}}
\newcommand{\pf}{\mathfrak{p}}
\newcommand{\Pf}{\mathfrak{P}}
\newcommand{\mf}{\mathfrak{m}}
\newcommand{\Mf}{\mathfrak{M}}
\newcommand{\qf}{\mathfrak{q}}
\newcommand{\Qf}{\mathfrak{Q}}

\renewcommand{\M}{\mathbb{M}}
\renewcommand{\l}[1]{\overline{#1}}

\newcommand{\Ac}{\mathcal{A}}
\newcommand{\Rc}{\mathcal{R}}
\renewcommand{\O}{\mathcal{O}}

\newcommand{\Frob}{\textsf{Frob}}

\newcommand{\Leg}[2]{\left(\frac{#1}{#2}\right)}

\newcommand{\ric}{\textsf{ric}}
\newcommand{\g}{\mathfrak{g}}
\newcommand{\kf}{\mathfrak{k}}
\newcommand{\p}{\mathfrak{p}}
\newcommand{\m}{\mathfrak{m}}

\newcommand{\GL}{\textsf{GL}(n,\R)}
\newcommand{\glf}{\mathfrak{gl}(n,\R)}
\newcommand{\SO}{\textsf{SO}(n,\R)}
\newcommand{\so}{\mathfrak{so}(n,\R)}
\newcommand{\SL}{\textsf{SL}(n,\R)}
\newcommand{\slf}{\mathfrak{sl}(n,\R)}
\newcommand{\POS}{\textsf{Pos}(n,\R)}
\newcommand{\symm}{\textsf{symm}(n,\R)}
\newcommand{\of}{\mathfrak{o}}

\usepackage{enumerate}

\makeindex

\begin{document}

\title{
\begin{huge}
Geometrie der Mannigfaltigkeiten\\
\end{huge}
\begin{large}
Kurzskript, SS 17
\end{large}}


%\author{Ak\i n Ünal}
\maketitle
\renewcommand{\i}{^{-1}}

%Das folgende Kurzskript orientiert sich an einer Vorlesung, die im Wintersemester 2016 / 2017 in Heidelberg gehalten wurde. Für alle Fehler im Text trägt ausschließlich der Autor die Verantwortung.

\setcounter{tocdepth}{1}
\tableofcontents

% % % Vorlesung 1
\newpage
\chapter{Hyperbolische Modelle}
\section{Das Hyperboloidenmodell}
\Def{}
Definiere die \df{Lorentzform} auf $\R^{n+1}$ durch
\[ \shrp{x,y} := x_1y_1 + \ldots + x_{n}y_n - x_{n+1} y_{n+1} \]
Ein Vektor$x \in \R^{n+1}$ heißt
\begin{align*}
\left\lbrace \begin{aligned}
\text{\df{zeitartig}, falls } &\shrp{x,x} < 0\\
\text{\df{lichtartig}, falls } &\shrp{x,x} = 0\\
\text{\df{raumartig}, falls } &\shrp{x,x} > 0
\end{aligned} \right.
\end{align*}
Definiere das \df{Hyperboloidenmodell} von $\H^n$ durch
\[ I^n = \set{p \in \R^{n+1}}{\shrp{p,p} = -1, p_{n+1 > 0}} \]
\Prop{}
$I^n$ ist eine Riemannsche Mannigfaltigkeit.
\begin{Beweis}{}
Definiere $f : \R^{n+1} \pfeil{} \R$ durch $x \mapsto \shrp{x,x}$. Dann ist $\d f_p(v) = 2\shrp{v,p}$. Ergo ist $\d f_p$ surjektiv für alle $p \in M :=f\i(-1)$. Ergo ist $M$ eine glatte Hyperfläche von $\R^{n+1}$.\\
Ferner ist
\[ T_pM = \Ker ~\d f_p  = \set{v \in \R^{n+1}}{\shrp{p,v} = 0} = p^\bot \]
Da $p$ zeitartig ist, ist $\shrp{\_, \_}$ auf $T_pM$ positiv definit. $I^n$ ist nun gerade die obere Zusammenhangskomponente von $M$.
\end{Beweis}

\Lem{}
Definiere
\[ O(n,1) = \set{A \in \R^{n+1\times n + 1}}{ \shrp{v,w} = \shrp{Av, Aw} }\]
und
\[ O(n,1)^+ = \set{A \in O(n,1)}{A (I^n) \subset I^n } \]
Dann ist $O(n,1)^+$ eine Index-2-Gruppe von $O(n,1)$ und
\[ Isom(I^n) = O(n,1)^+ \]
Ferner gilt
\[ Isom(S^n) = O(n) \text{ und } Isom(\R^n) = \set{x\mapsto Ax + b}{A \in O(n), b\in \R^n } \]


\Prop{}
Die $k$-dimensionalen, vollständigen, total geodätischen, zusammenhängenden Riemannschen Untermannigfaltigkeiten von $I^n$ sind genau die Schnitte
\[ I^n \cap W^{k+1}\]
wobei $W^{k+1}$ ein $k+1$-dimensionaler Untervektorraum von $\R^{n+1}$ ist, der keinen leeren Schnitt mit $I^n$ hat.\\
Folgende Aussagen sind für einen $k+1$-dimensionalen Untervektorraum von $\R^{n+1}$ äquivalent:
\begin{enumerate}[]
	\item $W^{k+1}\cap I^n \neq \emptyset$
	\item $W^{k+1}$ besitzt einen zeit-ähnlichen Vektor
	\item $\shrp{\_,\_}$ besitzt auf $W^{k+1}$ die Signatur $(k,1)$
\end{enumerate}

\Bem{}
Ein $k$-Unterraum von $I^n$ ist isometrisch zu $I^k$.

\Prop{}
Jede nach Bogenlänge parametrisierte Geodäte von $I^n$ ist von der Gestalt
\[ \gamma(t) = \cosh(t) \gamma(0) + \sinh(t)\dot{\gamma}(0) \]

\Kor{}
$H^n$ ist vollständig.

\section{Die Poincare Scheibe}
\Lem{Poincare-Scheiben-Modell}
Definiere die \df{Poincare-Scheibe} durch
\[D^n = \set{x \in \R^n}{\norm{x} < 1} \]
und folgenden Diffeomorphismus
\begin{align*}
p : I^n & \Pfeil{} D^n\\
(x_1, \ldots, x_n, x_{n+1}) & \longmapsto \frac{1}{x_{n+1} + 1} (x_1,\ldots, x_n) 
\end{align*}
Dann ist die Metrik auf $D^n$ gerade gegeben durch
\[ g^D_x = \klam{\frac{2}{1 - \norm{x}^2}}^2g^E_x \]
wobei $g^E$ die euklidische Metrik von $\R^n$ bezeichnet.
\begin{Beweis}{}
	Die Umkehrabbildung von $p$ ist gerade
	\[ p\i(x) = \frac{(2x, 1 + \norm{x}^2)}{1 - \norm{x}^2} \]
	Ihre Derivation ist
	\[ \d_xp\i(u) = \frac{ 2 (u (1 - \norm{x}^2) + 2 x(x|u), 2 (x|u) )}{ (1 - \norm{x}^2)^2 } \]
	Es gilt
	\[ \shrp{d_xp\i(u), d_xp\i(u)} = (\frac{2}{1 - \norm{x}})^2 \norm{u}^2 \]
	Da $p\i$ eine Isometrie sein soll und das Verhalten einer Metrik durch ihre Norm bestimmt ist, folgt nun
	\[ g^D_x = g^I_{p\i(x)} \circ \d_xp\i  = \klam{\frac{2}{1 - \norm{x}^2}}^2g^E_x \]
\end{Beweis}

\Def{}
Ein Diffeomorphismus
\[ f : (M,g) \Pfeil{} (N,h) \] 
heißt \df{konform}, falls eine glatte Funktion $f : M \pfeil{} \R_{>0}$ existiert, sodass
\[ f^*(h_{f(p)}) = \lambda(p)\cdot g_p  \]

\Bem{}
Die Poincare-Scheibe ist ein konformes Modell von $\H^n$, d.\,h., $(D^n, g^E)$ und $(D^n, g^D)$ sind zueinander konform.\\
Daraus folgt nun insbesondere, dass Winkel von sich schneidenden Geodäten in $(D^n, g^D)$ genauso wie in $(D^n, g^E)$ gemessen werden dürfen.

\Lem{}
Die $k$-dimensionalen, vollständigen, zusammenhängenden, total geodätischen Untermannigfaltigkeiten der Poincare-Scheibe sind ihre Schnitte mit $k$-Sphären und $k$-Ebenen von $\R^n$, die orthogonal zum Rand der Poincare-Scheibe liegen.

\Def{}
Sei $S_{p}(r) \subset \R^n$ eine Sphäre mit Radius $r$ um $p$. Definiere die \df{Inversion} an $S_p(r)$ durch
\begin{align*}
\phi : \R^n\setminus\{p\} & \Pfeil{} \R^n\setminus\{p\}\\
x & \longmapsto p + r^2\frac{x- p}{\norm{x-p}^2}
\end{align*}
\Prop{}
Jede Inversion ist \df{anti-konform}, d.\,h. konform und Orientierung umkehrend, und bildet Sphären und Ebenen auf Sphären und Ebenen ab.

\section{Das obere Halbraum-Modell}
\Def{}
Das \df{obere Halbraum-Modell}
\[ H^n = \set{x \in \R^{n}}{x_n > 0} \]
ergibt sich durch eine Inversion der Poincare-Scheibe an der Sphäre
\[ S = S_{(0,\ldots,0,-1)}(\sqrt{2}) \]
Insofern ist die obere Halbebene ein konformes Modell von $\H^n$.

\Prop{}
Die $k$-Ebenen von $H^n$ sind die $k$-Ebenen und $k$-Sphären von $\R^n$, die orthogonal zu $\partial H^n$ sind.

\Prop{}
Die Metrik auf $H^n$ ist gegeben durch
\[ g_x^H =  \frac{1}{x_n^2} g^E \]


\Prop{}
Folgende Abbildungen sind Isometrien von $H^n$:
\begin{enumerate}[1.)]
\item Horizontale Translationen:
\begin{align*}
x \longmapsto x + (b_1,\ldots, b_{n-1}, 0)
\end{align*}
\item Dilationen:
\[ x \longmapsto x \cdot \lambda \]
\item Inversionen an Sphären orthogonal zu $\partial H^n$
\end{enumerate}

\Prop{}
Die Isometrien der Poincare-Scheibe und der oberen Halbebene werden durch Inversionen an Sphären und Reflektion an Euklidischen Ebenen, die alle orthogonal zum Rand stehen, erzeugt.
\Prop{}
In den konformen Modellen sind Kugeln genau die euklidischen Kugeln mit exzentrischen Mittelpunkten.

\section{Die Kleinsche Scheibe}
\Def{}
Die \df{Kleinsche Ebene} besitzt dieselbe Trägermenge $K^n = D^n$ wie die Poincare-Scheibe. Allerdings entsteht die Kleinsche-Ebene durch einen Diffeomorphismus
\begin{align*}
I^n & \Pfeil{} K^n\\
x & \longmapsto \frac{(x_1,\ldots,x_n)}{x_n}
\end{align*}
Die Kleinsche Scheibe ist nicht konform, weswegen ihre Winkel nicht durch eine euklidische Einbettung gemessen werden können. Allerdings sind ihre Geodäten genau die Geraden des $\R^n$.

\section{Ränder}
\Def{}
Zwei nach Bogenlänge parametrisierte Geodäten $\alpha, \beta : [0, \infty) \pfeil{} M$ heißen \df{asymptotisch äquivalent}, falls die Funktion
\[ t \longmapsto d(\alpha(t), \beta(t)) \]
beschränkt ist.\\
Asymptotisch äquivalent Sein ist eine Äquivalenzrelation auf der Menge aller geodätischer Halbgeraden.\\
Teilt man diese Relation heraus, erhält man den \df{Rand} $\partial M$ einer Mannigfaltigkeit $M$. Insbesondere schreibt man
\[ \overline{M} = M \cup \partial M \]

\Prop{}
Es gibt eine Bijektion zwischen $\partial D^n$ als Rand einer Mannigfaltigkeit und $S^{n-1}$.\\
Da ferner $\overline{D^n}$ eine naheliegende Topologie besitzt, können wir diese auf $\overline{\H^n}$ zurückführen.
\begin{Beweis}{}
	Sei $\gamma : [0, \infty) \pfeil{} D^n$ ein geodätischer Strahl. Da $\gamma$ sich orthogonal mit $S^{n-1}$ im Unendlichen schneiden muss, folgt
	\[ \lim\limits_{t \pfeil{} \infty} \gamma(t) \in S^{n-1} \]
	Hierdurch erhalten wir eine surjektive Abbildung
	\[ R(\gamma) :=  \lim\limits_{t \pfeil{} \infty} \gamma(t) \]
	Wir müssen nun zeigen, dass zwei nach Bogenlänge parametrisierte Strahlen $\gamma$, $\beta$ unter $R$ genau dann dasselbe Bild haben, wenn sie asymptotisch äquivalent sind.\\
Wir transformieren das Problem zu einem Problem auf $H^n$ und rechnen die geforderte Eigenschaft dort konstruktiv nach.
\end{Beweis}

\Bem{}
Man kann auch alternativ wie folgt eine Basis der Topologie von $\overline{\H^n}$ definieren: Dazu nimmt man alle offenen Mengen von $\H^n$ und schmeißt alle Mengen der Gestalt
\[ \{ \alpha(t) \in \H^n~|~ \alpha(0) = \gamma(0), \dot{\alpha}(0)  \in V ,t > r\} \cup \set{ [\alpha] \in \partial \H^n }{\alpha(0) = \gamma(0),\dot{\alpha}(0)  \in V} \]
für alle $[\gamma] \in \partial \H^n, V \subseteq_o T_{\gamma(0)}M, r > 0$. Die hierdurch entstehende Topologie stimmt der durch obige Proposition überein.

\Prop{}
Seien $S,S'$ zwei geodätisch vollständige Teilräume von $\H^n$. Dann tretet genau einer der folgenden Fälle ein:
\begin{itemize}
	\item $S$ und $S'$ sind \df{inzident}, d.\,h., $S\cap S'\neq \emptyset$.
	\item $S$ und $S'$ sind \df{asymptotisch parallel}, d.\,h., $S\cap S' = \emptyset$ und $d(S, S') = 0$. Ferner ist dann $\overline{S} \cap \overline{S'}$ ein Punkt in $\H^n$ und existiert keine Geodäte, die zu beiden Räumen orthogonal ist.
	\item $S$ und $S'$ sind \df{ultra-parallel}, d.\,h., $\overline{S}\cap \overline{S'} = \emptyset$ und $d(S, S') > 0$. In diesem Fall existiert genau eine Geodäte, die orthogonal zu beiden Teilräumen steht und den Abstand zwischen beiden realisiert.
\end{itemize}
\begin{Beweis}{}
	Enthält $\overline{S} \cap \overline{S'}$ mindestens zwei Punkte, so enthält der Schnitt auch eine Geodäte zwischen beiden Punkten. Wir können also annehmen, dass sich $S$ und $S'$ wenn überhaupt nur im Unendlichen schneiden und dort höchstens einen Schnittpunkt haben.\\
	Besteht $\overline{S} \cap \overline{S'}$ aus genau einem Punkt, so können wir die Situation in den $H^n$ transformieren und fordern, dass der gemeinsame Schnittpunkt gerade $\infty$ ist. In diesem Fall stehen $S$ und $S'$ parallel zur imaginären Achse, weswegen die Eigenschaften des zweiten Falles folgen.\\
	Im zweiten Fall finden wir $x\in S, x' \in S'$ mit $d(x,x') = d(S, S')$, da $\overline{S}$ und $\overline{S'}$ kompakt sind. Die Geodäte, die $x$ und $x'$ verbindet, muss orthogonal sein, da wir sie sonst verschieben könnten, um den Abstand zwischen $S$ und $S'$ zu minimieren. Es kann keine weitere Geodäte zwischen $S$ und $S'$ mit Abstand $d(S, S')$ geben, da wir sonst einen flachen Bereich gefunden hätten.
\end{Beweis}

\Lem{}
Eine Isometrie $\phi : \H^n \pfeil{} \H^n$ lässt sich zu einem Homöomorphismus $\overline{\phi} : \overline{\H^n} \pfeil{} \overline{\H^n} $ fortsetzen. $\phi$ ist durch $\overline{\phi}_{\partial \H^n}$ eindeutig festgelegt.

\section{Isometrien}
\Prop{}
Sei $\phi : \H^n \pfeil{} \H^n$ eine nichttriviale Isometrie. Dann tritt genau einer der folgenden Fälle ein:
\begin{itemize}
	\item $\phi$ ist \df{elliptisch}, d.\,h., $\phi$ hat einen oder mehrere Fixpunkte in $\H^n$.
	\item $\phi$ ist \df{parabolisch}, d.\,h., $\phi$ hat keinen Fixpunkt in $\H^n$, aber genau einen in $\partial \H^n$.
	\item $\phi$ ist \df{hyperbolisch}, d.\,h., $\phi$ hat keinen Fixpunkt in $\H^n$, aber genau zwei in $\partial \H^n$.
\end{itemize}
\begin{Beweis}{}
Wir können $\overline{\phi}$ als einen Homöomorphismus von $\overline{D^n}$ auf sich selbst auffassen. Nach Brauers Fixpunktsatz muss $\phi$ dann mindestens einen Fixpunkt haben. Hat $\phi$ keinen Fixpunkt in $D^n$ aber mindestens drei auf dem Rand, so fixiert $\phi$ jeden Punkt in $\H^n$, da jeder Punkt in $D^n$ durch seine Winkel zu den drei verschiedenen Randpunkten eindeutig determiniert ist.
\end{Beweis}

\Def{}
Eine hyperbolische Isometrie fixiert zwei Randpunkte und damit auch die Geodäte, die zwischen beiden verläuft. Diese eindeutig bestimmte Geodäte nenne wir die \df{Achse} von $\phi$.

\Def{}
Eine \df{Horosphäre} $S \subset \overline{H^n}$ um den Punkt $p \in \partial H^n\setminus\{\infty\}$ ist eine $n-1$-dimensionale euklidische Sphäre, die $\partial H^n$ tangential in $p$ schneidet. Eine Horosphäre um $\infty$ ist eine $n-1$-dimensionale euklidische Hyperebene, die orthogonal zur imaginären Achse steht.\\
Beide Horosphären sind flache Untermannigfaltigkeiten mit der Eigenschaft, dass jede Geodäte, die ihr Zentrum verlässt, die Horosphäre orthogonal schneidet.

\Prop{}
Wir stellen Punkte aus $H^n$ in der Form $(x,t)$ dar. Sei $\phi$ eine nichttriviale Isometrie von $\H^n$.
\begin{itemize}
	\item Ist $\phi$ elliptisch mit Fixpunkt $0 \in D^n$, so gibt es ein $A \in O(n)$, sodass sich $\phi$ darstellen lässt durch
	\begin{align*}
\phi:	D^n & \Pfeil{} D^n\\
	x & \longmapsto Ax
	\end{align*}
	\item Ist $\phi$ parabolisch mit Fixpunkt $\infty \in \partial H^n$, so gibt es $A \in O(n-1)$ und $b\in \R^{n-1}$, sodass
	\begin{align*}
\phi:	H^n & \Pfeil{} H^n\\
	(x,t) & \longmapsto (Ax + b, t)
	\end{align*}
	\item Ist $\phi$ hyperbolisch mit Fixpunkten $0,\infty \in \partial H^n$, so gibt es ein $\lambda > 0, \neq 1$ und ein $A\in O(n-1)$, sodass
		\begin{align*}
	\phi:	H^n & \Pfeil{} H^n\\
	(x,t) & \longmapsto (\lambda Ax, \lambda t)
	\end{align*}
\end{itemize}
\begin{Beweis}{}
Der elliptische Fall ist klar.\\
Sei der zweite Fall gegeben. Ist $O$ eine Horosphäre um $\infty$, so muss $\phi(O)$ wieder eine Horosphäre um $\infty$ liefern. Dann gibt es ein $(x,t) \in O$, sodass $\phi(x,t) = (x, t')$. Ist $t' \neq t$, so erhält $\phi$ eine Geodäte durch $(x,t)$ und $\infty$ und ist nicht mehr parabolisch. Ergo ist $\phi(O) = O$. D.\,h., $\phi$ ist auf jeder Horosphäre um $\infty$ durch eine euklidische Isometrie gegeben.\\
Sei nun der dritte Fall gegeben. $\phi$ erhält die imaginäre Achse, ergo gibt es ein $\lambda$ mit $\phi(0,1) = (0, \lambda)$. Setzt man $\psi(x,t) = \lambda\i \phi(x,t)$, so gibt es ein $A\in O(n-1)$ mit
\[ \d_{(0,1)} \psi = \klam{
\begin{matrix}
A & 0\\
0 & 1
\end{matrix}
} \]
womit folgt
\[ \psi(x,t) = (Ax,t) \]
\end{Beweis}

\Def{}
Für eine Isometrie $\phi : M \pfeil{} M$ sei die \df{Versetzung} definiert durch
\begin{align*}
d(\phi) := \inf_{p\in M}d(p, \phi(p))
\end{align*}

\Kor{}
\begin{itemize}
	\item Eine elliptische Isometrie hat eine Versetzung von 0, die an ihren Fixpunkten verwirklicht wird.
	\item Eine parabolische Isometrie hat eine Versetzung von 0, die nirgendwo realisiert wird, und fixiert jede Horosphäre um ihren Fixpunkt.
	\item Eine hyperbolische Isometrie hat eine Versetzung von $d > 0$, die genau auf ihrer Achse realisiert wird.
\end{itemize}

\section{Möbiusgeschichten}
\Def{}
Es bezeichne $S = \C \cup \infty$ die \df{Riemannsche Zahlenkugel}. Die Gruppe $PSL_2(\C)$ agiert auf $S$ durch die \df{Möbiustransformation}
\[ \klam{
\begin{matrix}
a & b\\
c & d
\end{matrix}
}.z := \frac{az + b}{cz + d} \]
Die Möbiustransformation ist ein orientierungserhaltender Diffeomorphismus auf $S$.\\
Die \df{Anti-Möbiustransformation} gegeben durch
\[ \klam{
	\begin{matrix}
	a & b\\
	c & d
	\end{matrix}
}.z := \frac{a\overline{z} + b}{c\overline{z} + d} \]
ist ein orientierungsumkehrender Diffeomorphismus auf $S$.\\
Unter $Conf(S) \subset Diffeo(S)$ verstehen wir die Menge aller Möbius- und Anti-Möbiustransformationen, die durch Elemente aus $PSL_2(\C)$ induziert werden.

\Prop{}
Inversionen entlang Sphären und Spiegelungen entlang Geraden sind beides Anti-Möbiustransformationen und erzeugen $Conf(S)$.

\Lem{}
Betrachte
\[ H^2 = \set{ x + iy}{x \in \R, y > 0} \]
und setze
\[ Conf(H^2) = \set{\phi \in Conf(S)}{\phi(H^2) \subseteq H^2} \]
Dann ist jede Transformation aus $Conf(H^2)$ induziert durch eine Matrix mit reellen Einträgen, deren Determinante gleich 1 ist, falls die Transformation orientierungserhaltend ist, anderenfalls -1 ist.\\
Es gilt
\[ Conf^+(H^2) = PSL_2(\R) \]

\Prop{}
Inversionen entlang Kreisen und Reflexionen entlang Geraden, die beide orthogonal zu $\R$ sind, generieren $Conf(H^2)$.
\Kor{}
\[ Conf(H^2) = Isom(H^2) \]

\Prop{}
Eine nichttriviale Transformation $A \in PSL_2(\R)$ ist
\begin{itemize}
	\item elliptisch, falls $\bet{tr(A)} < 2$
	\item parabolisch, falls $\bet{tr(A)} = 2$
	\item hyperbolisch, falls $\bet{tr(A)} > 2$
\end{itemize}

\Prop{}
Da $\partial H^3 = S$, gilt
\[ Isom(H^2) = Conf(S) \]

\Prop{}
Eine nichttriviale Transformation $A \in PSL_2(\C)$ ist
\begin{itemize}
	\item elliptisch, falls $tr(A) \in (-2, 2)$
	\item parabolisch, falls $tr(A) = \pm 2$
	\item hyperbolisch, falls $tr(A) \in \C\setminus [-2, 2]$
\end{itemize}


\printindex
\end{document}