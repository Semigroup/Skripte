\chapter{Das Globale Reziprozitätsgesetz}
\section{Eigenschaften der Artin-Abbildung}
\Satz{Eigenschaften der Artin-Abbildung}
\label{Eigenschaften}
Sei $L|K$ eine endliche abelsche Erweiterung globaler Körper, $\delta = \delta_{L|K}$ die dazugehörige Diskriminante.
\begin{itemize}
\item[$(\Af 1)$] Ist $\sigma : L \pfeil{} \sigma L$ ein beliebiger Ringisomorphismus, so gilt für alle $\af \in \I(\delta)$
\[ (\sigma \af, \sigma L / \sigma K) = \sigma (\af , L/K)\sigma\i \]
\item[$(\Af 2)$] Ist $L'\supset L$ ein Oberkörper, der eine abelsche Erweiterung $L'|K$ bildet, so gilt für alle $\af \in \I(\delta_{L'|K})$
\[ (\af, L'/K)_{|L} = (\af, L/K) \]
\item[$(\Af 3)$] Ist $E|K$ eine weitere endliche Erweiterung, so gilt
\[ (\bf, LE /E)_{|L} = (N_{E|K}\bf, L/K) \]
für alle $\bf \in \I(\delta_{LE|E})$, die für alle Primideale $\qf $ über $ \pf$ in $E$ bzw. $K$ folgende Eigenschaft erfüllen
\[ \qf |\bf \Impl{} \pf \text{ ist unverzweigt in }L|K \]
\end{itemize}

\Lem{}
Sei folgender Körperturm gegeben
\[ K(\mu_n) ~|~ L ~|~ K ~|~ \Q \]
wobei $K|\Q$ endlich ist. Dann existiert ein $0\neq \cf \subset \O_K$, das nur durch die Primideale geteilt wird, durch die $n$ teilbar ist und für das gilt
\[ \P(\cf) \subset \Ker \omega_\cf \]

\Lem{}
Seien $a,r > 1$ natürliche Zahlen, $q > 0$ eine Primzahl. Dann existiert eine Primzahl $p > 0$, sodass gilt
\[ \textsf{ord}_{(\Z/p\Z)^\times}(a) = q^r \]

\Def{Unabhängigkeit Modulo einer Zahl}
Sei $m > 1$ natürlich.\\
$a,b \in \Z$ heißen \df{unabhängig mod $m$}, falls sie teilerfremd zu $m$ sind und
\[ \shrp{a} \cap \shrp{b} = \{1\}  \text{ in }(\Z /m\Z)^\times \]

\Lem{}
Seien $n,a\in \N$ und $a > 1$. Dann existiert eine quadratfreie Zahl $m \in \N$, sodass
\begin{itemize}
\item $n | \textsf{ord}_{(\Z/m\Z)^\times}(a)$
\item Es gibt ein $b \in \N$, sodass $n|\textsf{ord}_{(\Z/m\Z)^\times}(b)$ und $a,b$ sind unabhängig mod $m$.
\end{itemize} 
Ferner können die Primzahlen in der Faktorisierung von $m$ beliebig groß gewählt werden.

\Lem{}
Sei $L|K$ ein endliche und abelsche Erweiterung beliebiger Zahlkörper. $S$ sei eine endliche Menge von Primzahlen, $\pf \subset \O_K$ sei ein Primideal, das in $L$ nicht verzweigt.\\
Es existiert eine ganze Zahl $m$, die teilerfremd zu $\pf$ und allen Zahlen in $S$ ist und folgende Eigenschaften hat:
\begin{itemize}
\item $n | \textsf{ord}((\pf, K(\mu_m) / K))$
\item $L\cap K(\mu_m) = K$
\item Es existiert ein $\tau \in G(K(\mu_m) | K)$, sodass
\begin{itemize}
\item $n|\textsf{ord}(\tau)$
\item $\tau$ und $(\pf, K(\mu_m)/K) \in (\Z/m\Z)^\times$ sind unabhängig mod $m$
\end{itemize}
\end{itemize} 

\Satz{Lemma von Artin}
Sei $L|K$ eine endliche, zyklische Erweiterung beliebiger Zahlkörper. $S$ sei eine endliche Menge von Primzahlen und $\pf \subset \O_K$ sei ein Primideal, welches in $L$ nicht verzweigt.\\
Dann existiert eine beliebig große Zahl $m \in \N$, die von keiner Zahl aus $S$ geteilt wird, und eine endliche Erweiterung $E|K$, sodass gilt:
\begin{itemize}
\item $L\cap E = K$
\item $L(\mu_m) = E(\mu_m)$ und $L\cap K(\mu_m) = K$
\item $\pf$ zerlegt sich voll in $E$.
\end{itemize} 

\Lem{}
Sei $L|K$ eine endliche, zyklische Erweiterung beliebiger Zahlkörper. $\pf_1,\ldots, \pf_r$ seien Primideale in $K$, die über $L|K$ nicht verzweigen.\\
Zu jedem $\pf_i$ existiert laut Artins Lemma eine endliche Erweiterung $E_i|K$ und ein $m_i$, sodass die einzelnen $m_i$ paarweise teilerfremd sind.\\
Setzt man $E = E_1\cdots E_r$, so gilt $L\cap E = K$ und daher
\[G(L|K) = G(LE|E) \]

\Satz{}
Ist $L|K$ eine zyklische Erweiterung beliebiger Zahlkörper, so existiert ein zulässiges Ideal $0\neq \cf \subset \O_K$, das nur durch in $L$ verzweigende Primideale geteilt wird und für das gilt
\[ \Ker \omega_\cf = \P(\cf) \Nc(\cf) \]

\Satz{}
Ist $L|K$ eine endliche, abelsche Erweiterung beliebiger Zahlkörper, $\cf$ ein zulässiges Ideal, so hat $\omega_\cf$ als Kern $\P(\cf)\Nc(\cf)$ und induziert einen Isomorphismus
\[ \I(\cf) / (\P(\cf) \Nc(\cf)) \Pfeil{\isom{}} G(L|K) \]

\section{Ideltheoretische Formulierung}
\Def{Normrestsymbol}
Sei $L|K$ eine endliche, abelsche Erweiterung beliebiger Zahlkörper, $\cf$ ein zulässiges Ideal. Definiere das \df{Normrestsymbol} durch folgende Surjektion
\[ (\_, L/K) : \A_K^\times \surj{} \A_K^\times / K^\times N_{L|K}\A_L^\times \pfeil{} \I(\cf) /\P(\cf) \Nc(\cf) \pfeil{\isom{}} G(L|K) \]
Für ein $a \in \A_K^\times$ gilt
\[ (a,L/K) = (\af, L/K) \]
wobei $\af \in \I(\cf)$ assoziiert ist zu $\alpha a \in \A_{K,\cf}^\times = \prod_{v\in S(\cf)} U_v(\cf) \times \prod_{v\not \in S(\cf)}'K_v^\times $ für ein geeignetes $\alpha \in K^\times$, wobei
\[ S(\cf) = \set{v\text{ endliche Stelle von }K}{ v |\cf } \]

\Satz{}
Sei $L|K$ eine endliche, abelsche Erweiterung beliebiger Zahlkörper. Dann impliziert die Artin-Abbildung folgenden Isomorphismus
\[ C_K /N_{L|K}C_L \isom{} \A_K^\times / K^\times N_{L|K}\A_L^\times \Pfeil{\isom{}} G(L|K) \]
Für jedes $a \in \A_K^\times$ gilt
\[ (a,L/K) = \prod_v (a_v, L/K) \]
wobei wir unter $a_v$ das Idel verstehen, das an Stelle $v$ mit $a$ übereinstimmt und an allen anderen Stellen gleich Eins ist. Ferner erfüllt das Normrestsymbol dieselben Eigenschaften \ref{Eigenschaften} wie die ideltheoretische Formulierung.

\Bem{}
Ist $L|K$ eine endliche Erweiterung von Zahlkörpern, so sind $N_{E|K}\A_E^\times$ und $K^\times N_{E|K}\A_E^\times$ offen in $\A_K^\times$.\\
Insbesondere ist $\U(\af) \subset N_{E|K}\A_E^\times$ offen in $\A_K^\times$ für geeignete $\af \in \I$.\\
Es ergibt sich insofern folgende Korrespondenz
\[ \set{H \subset_o C_K}{} \overset{\text{1 : 1}}\longleftrightarrow \set{U \subset_o \A_K^\times}{K^\times \subset U} \]

\Def{Klassenkörper und Klassengruppen}
Sei $L|K$ eine endliche Erweiterung von Zahlkörpern.\\
$H \subset \A_K^\times$ bzw. $H' \subset C_K$ heißt die \df{Klassengruppe} von $L|K$, falls $H = K^\times N_{L|K}\A_L^\times$ bzw. $H' = N_{L|K}C_L$.\\
Umgekehrt heißt $L$ in diesem Fall der \df{Klassenkörper} von $H$ bzw. $H'$.

\Bem{}
Ist $L$ der Klassenkörper von $H \subset C_K$ und $\sigma : L \pfeil{} L$ ein beliebiger Isomorphismus, so ist $\sigma L$ der Klassenkörper von $\sigma H \subset C_{\sigma K}$.

\Def{Universelles Normrestsymbol}
Sei $K$ ein Zahlkörper. Definiere das \df{universelle Normrestymbol} durch
\begin{align*}
(\_, K^{ab} | K)  : C_K& \Pfeil{} G(K^{ab}|K) \\
a & \longmapsto  \lim\limits_{L|K \text{ endl, ab}}(a, L|K)
\end{align*}
Das universelle Normrestsymbol ist ein stetiger, surjektiver Gruppenhomomorphismus.
\paragraph{Bemerkung} Definiert man das universelle Normrestsymbol axiomatisch für beliebige Körper und Gruppen, die die Klassenkörperaxiome erfüllen, so ist das Normrestsymbol im Allgemeinem nicht surjektiv, besitzt aber immer ein dichtes Bild.

\Prop{}
Die kurze exakte Sequenz topologischer Gruppen
\[ 1 \Pfeil{} C_K^1 \Pfeil{} C_K \Pfeil{} \R_{> 0} \Pfeil{} 1 \]
spaltet.

\Lem{}
Bezeichne das universelle Normrestsymbol mit
\[ \phi : C_K \isom{} C_K^1 \times \R_{>0} \Pfeil{} G(K^{ab}|K) \]
Dann ist $\phi_{|\R_{>0}}$ trivial und $\phi_{|C_K^1}$ surjektiv.

\Bem{}
Es gilt
\[ \Ker \phi = \bigcap_{L|K \text{ endl, ab}}N_{L|K}C_L \]
und
\[ \Ker \phi_{|C_K^1} = \bigcap_{L|K \text{ endl, ab}}(N_{L|K}C_L)^1 \]
Dadurch induziert $\phi_{|C_K^1}$ folgenden Isomorphismus topologischer Gruppen
\[ C_K^1/\bigcap_{L|K}(N_{L|K}C_L)^1 \Pfeil{\isom{}} G(K^{ab}|K) \]
da $C_K^1$ kompakt ist.
