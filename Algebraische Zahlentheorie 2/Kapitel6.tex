\chapter{Differente und Diskriminante}
\Def{Ausartung}
Sei $R$ ein kommutativer Ring, $M,N$ seien $R$-Moduln.\\
Eine Bilinearform $\shrp{} : M\times N \pfeil{} R$ heißt \df{perfekt} oder \df{nicht ausgeartet}, falls die \df{Ausartungsräume}
\begin{align*}
M^\bot &:= \{n \in N ~|~ \forall m \in M : \shrp{m,n} = 0\}\\
N^\bot &:0 \{m \in M ~|~ \forall n \in N: \shrp{m,n} = 0\}
\end{align*}
verschwinden.
\Bem{}
Verschwinden die Ausartungsräume, so sind die natürlichen Abbildungen
\[ M \Pfeil{} \Hom{R}{N}{R} \text{ und } N \Pfeil{} \Hom{R}{M}{R} \]
injektiv. Ist ferner $R$ ein Körper, so sind jene Abbildungen sogar isomorph.\\
Sind die Ausartungsräume trivial, so existiert zu jeder Basis $b_1,\ldots, b_n$ von $M$ genau eine Basis $b_1^\vee,\ldots, b_n^\vee$ von $N$, sodass gilt
\[ \shrp{b_i,b_j^\vee} = \delta_{i,j} \]

\section{Komplementärmoduln}
\Bem{}
Sei $A$ ein Dedekindring, $K = Quot(A)$, $L|K$ eine endliche, separable Körpererweiterung. $B$ bezeichne den ganzen Abschluss von $A$ in $L$.\\
Die \df{Spurpaarung}
\begin{align*}
Tr : L\times L &\Pfeil{} K\\
(x,y) & \longmapsto Tr_{L|K}(xy)
\end{align*}
ist $K$-linear und nicht ausgeartet. Es folgt
\[ \Hom{K}{L}{K} \isom{} L \]
Aber im Allgemeinem ist es falsch, anzunehmen
\[ \Hom{A}{B}{A} \isom{} B \]

\Def{Komplementärmodul}
Sei $M\subset L$ ein $A$-Untermodul. Dann heißt
\[ D_A(M) := \set{x\in L}{Tr_{L|K}(xM) \subset A} \]
der \df{Komplementärmodul} von $M$.

\Satz{}
Seien $M \subseteq N \subseteq L$ $A$-Untermoduln, $\bf$ ein gebrochenes Ideal von $L$.
\begin{itemize}
	\item $D_A(M)$ ist ein $A$-Untermodul von $L$. Ist $M$ ein $B$-Modul, so auch $D_A(M)$.
	\item $D_A(N) \subset D_A(M)$
	\item $B \subset D_B(B)$
	\item Ist $w_1,\ldots,w_n$ eine $K$-Basis von $L$, so gilt
	\[ D_A(Aw_1 + \ldots + Aw_n) = Aw_1^\vee + \ldots + Aw_n^\vee \]
	\item $D_A(\bf)$ ist ein gebrochenes Ideal von $L$.
	\item $D_A(\bf) = D_A(B)\cdot \bf\i$
	\item $D_A(D_A(\bf)) = \bf$
\end{itemize}

\Def{}
Definiere die \df{Differente} von $B|A$ als das ganze Ideal
\[ D_{B/A} := D_A(B)\i \subset B \]

\Satz{}
Sei $L = K(\alpha), n = [L:K]$. $f \in K[X]$ sei das Mimimalpolynom von $\alpha$, es bezeichne
\[ \frac{f}{X-\alpha} = X^{n-1} + b_{n-2}X^{n-2} + \ldots + b_1X + b_0 \]
Dann ist
\[ \frac{1}{f'(\alpha)}, \frac{b_{n-2}}{f'(\alpha)}, \ldots , \frac{b_{0}}{f'(\alpha)} \]
die duale Basis zu
\[ 1, \alpha, \ldots, \alpha^{n-1} \]

\Kor{}
Sei $\alpha \in L$, $C = A[\alpha]$. Dann gilt
\[ D_A(C) = (f'(\alpha))\i C \]
Gilt $B = A[\alpha]$, so folgt insbesondere
\[ D_{B/A} = f'(\alpha)B \]

\Satz{}
\begin{itemize}
	\item Seien $K\subset L\subset E$ endliche, separable Erweiterungen. $C$ bezeichne den ganzen Abschluss von $A$ in $E$. Dann gilt
	\[ D_{C/A} = D_{B/A} \cdot D_{C/B} \]
	\item Ist $S \subset K^\times$ ein Untermonoid, so gilt
	\[ S\i D_{B/A} = D_{S\i B / S\i A} \]
	\item Sind $\Pf |\pf$ Primideale in $B$ bzw. $A$ und $\widehat{B}_\Pf$ bzw. $\widehat{A}_\pf$ diesbezügliche Komplettierungen, so gilt
	\[ D_{B/A} \widehat{A}_\Pf = D_{\widehat{B}_\Pf / \widehat{A}_\pf} \]
\end{itemize}

\Kor{}
Die Differente ist das formale Produkt
\[ D_{B/A} = \prod_{ \Pf \subset B \text{ prim}} D_\Pf \]
wobei $D_\Pf = D_{\widehat{B}_\Pf / \widehat{A}_\pf} \cap = D_{B_\Pf / A_\pf}\cap B $

\section{Differente und Verzweigungen}
\Lem{}
Sei $L|K$ eine endliche separable Körpererweiterung mit Ganzheitsringen $B|A$.\\
Zusätzlich sei $L = K[\alpha]$ für $\alpha \in B$ und
\[ F= \set{x \in A[\alpha]}{xB \subset A[\alpha]} \]
Dann gilt
\[ F = f'(\alpha)D\i_{B/A} \]
wobei $f\in K[X]$ das Minimalpolynom von $\alpha$ ist.

\Kor{}
Die Differente $D_{B/A}$ teilt $f'(\alpha) B$. Ferner gilt
\[ D_{B/A} = f'(\alpha) B \Gdw{} B = A[\alpha] \]

\Satz{}
Seien $A,B$ diskrete Bewertungsringe, deren Restklassenkörpererweiterung separabel ist. Dann existiert ein $\alpha \in B$, sodass $B = A[\alpha]$.

\Satz{}
Sei $\Pf \subset B$ ein Primideal über $\pf \subset A$ und sei $B/\Pf | A / \pf$ separabel. Es gilt:
\begin{itemize}
	\item $\pf$ verzweigt in $L$ genau dann, wenn $\pf$ die Differente $D_{B/A}$ teilt.
	\item Sei $s = v_\Pf(D_{B/A})$ und $e = e_\pf$.
	\begin{itemize}
		\item Ist $\pf$ zahm verzweigt, so gilt
		\[ s = e - 1\]
		\item Ist $\pf$ wild verzweigt, so gilt
		\[ e \leq s \leq e - 1 + v_\pf(e) \]
	\end{itemize}
\end{itemize}

\Satz{}
Seien alle Restklassenkörpererweiterungen separabel. Dann ist $D_{B/A}$ das Ideal, das von allen $f'_\alpha(\alpha)$ erzeugt wird, wobei $\alpha$ alle Elemente mit $L = K(\alpha)$ durchläuft und $f_\alpha$ sein Minimalpolynom bezeichnet.

\Def{Diskriminante}
Das Ideal $\delta_{B|A} := N_{L|K}(D_{B/A})$ heißt \df{Diskriminante} von $B|A$.

\Satz{}
Sei $K = \Q$, dann ist
\[ \delta_{\O_L/\Z}  = d(L)\cdot \Z \]
wobei $d(L)$ definiert ist durch
\[ d(L) := \det( \klam{Tr_{L|\Q} (w_i w_j)}_{i,j} ) \]
für eine $\Z$-Basis $w_1,\ldots, w_n$ von $\O_L$.

\Satz{}
\begin{itemize}
	\item Seien $K\subset L\subset E$ endliche, separable Erweiterungen. $C$ bezeichne den ganzen Abschluss von $A$ in $E$. Dann gilt
	\[ \delta_{C/A} = N_{L|K}(\delta_{C/B}) \cdot \delta_{B/A} \]
	\item Ist $S \subset K^\times$ ein Untermonoid, so gilt
	\[ S\i \delta_{B/A} = \delta_{S\i B / S\i A} \]
	\item Sind $\Pf |\pf$ Primideale in $B$ bzw. $A$ und $\widehat{B}_\Pf$ bzw. $\widehat{A}_\pf$ diesbezügliche Komplettierungen, so gilt
	\[ \delta_{B/A} \widehat{A}_\Pf = \delta_{\widehat{B}_\Pf / \widehat{A}_\pf} \]
\end{itemize}

\Satz{}
Sei $\pf \subset A$ prim, $\pf = \Pf_1^{e_1} \cdots \Pf_r^{e_r}$ die Zerlegung in $B$. $p = \textsf{char}(A/\pf)$. Dann gilt
\[ v_\pf (\delta_{B/A}) \left\lbrace
\begin{aligned}
= (e_1 - 1) f_1 + \ldots + (e_r - 1)f_r && \text{ falls } p \nmid e_i \forall i\\
<(e_1 - 1) f_1 + \ldots + (e_r - 1)f_r && \text{ sonst}
\end{aligned}
\right. \]
Insbesondere gilt
\[ \pf \text{ verzweigt in }L \Gdw{} \pf \mid \delta_{B/A} \]

\Satz{}
Sei $S$ eine endliche Menge von maximalen Idealen eines Zahlkörpers $K$, $n\in \N$.\\
Dann gibt es nur endlich viele Erweiterungen von $K$, die außerhalb von $S$ unverzweigt sind.

\Satz{}
Es gibt keine unverzweigten Erweiterungen von $\Q$.