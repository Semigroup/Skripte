\documentclass{book}

\usepackage{../Package/latexa}
\usepackage{../Package/algebra}
\usepackage{../Package/theorema}
\usepackage{../Package/diagramma}
\usepackage{../Package/categoria}

\newcommand{\qi}{\backsimeq_{\textsc{QI}}}
\newcommand{\ba}{\backsimeq_{\textsc{BA}}}
\newcommand{\normal}{\vartriangleleft}
\newcommand{\tm}{\subset}
\newcommand{\Stab}[2]{\textsf{Stab}_{#1}(#2)}
\newcommand{\Cay}[2]{\textsf{Cay}(#1,#2)}

\renewcommand{\C}{\mathcal{C}}
\newcommand{\Cz}{\mathbb{C}}

\renewcommand{\i}{^{-1}}

\begin{document}

\tableofcontents

\chapter{Topologische Gruppen}
\section{Topologische Gruppen}
\Def{Topologische Gruppen}
Ein Paar $(G,\T)$ einer Gruppe und einer Topologie auf $G$ heißt \df{topologische Gruppe}, wenn die Abbildungen
\begin{align*}
\_\cdot\_ : G \times G &\Pfeil{} G\\
\_\i : G &\Pfeil{} G
\end{align*}
stetig sind.\\
Unter einem \df{Homomorphismus topologischer Gruppen} verstehen wir einen stetigen Gruppenhomomorphismus.

\Bem{}
Seien $G,H$ topologische Gruppen.
\begin{itemize}
\item $U \subset G$ heißt \df{Umgebung} von $g\in G$, falls eine Teilmenge $V \subset_o G$ existiert, sodass $g \in V \subseteq U$.
\item $\phi: G \pfeil{} H$ ist genau ein Homomorphismus, wenn das Urbild jeder Umgebung der 1 in $H$ eine Umgebung der 1 in $G$ ist.
\end{itemize}

\Prop{}
\label{Prop4}
Sei $G$ eine topologische Gruppe und $U \subset G$ eine Umgebung der 1.
\begin{itemize}
\item[(i)] Es existiert eine offene Umgebung $V$ der 1, sodass $V\cdot V \subset U$ und $V = V\i$.
\item[(ii)] Es existiert eine Umgebung $V$ der 1, deren Abschluss $\overline{V}$ in $U$ enthalten ist.
\end{itemize}
Sei nun $H \leq G$ eine Untergruppe.
\begin{itemize}
\item[(iii)] Der Abschluss von $H$ ist ebenfalls eine Untergruppe. Dieser ist insbesondere normal, falls $H$ ebenfalls normal ist.
\item[(iv)] Ist $H \leq_o G$ offen, so auch abgeschlossen, also insbesondere eine Zusammenhangkomponente.
\end{itemize}
\begin{Beweis}{}
\begin{itemize}
\item[(i)] Definiere
\begin{align*}
f&: G \pfeil{} G, x \mapsto x^2\\
V' &:= f\i(U) \cap U\\
V &:= V' \cap {V'}\i
\end{align*}
\item[(ii)] Wir geben ohne Beweis einen Satz an, aus dem die Behauptung sofort folgt:
\paragraph{Satz von Weil}
Eine topologische Gruppe $G$ ist $\text{T}_{3\frac{1}{2}}$, d.\,h., ist $A\subseteq_aG$ eine Teilmenge, die die 1 nicht enthält, so existiert eine stetige Abbildung $f : G \pfeil{} [0,1] \subset \R$ mit folgenden Eigenschaften:
\begin{itemize}
\item $f(A) = \{1\}$
\item $f(1) = 0$
\end{itemize}
\item[(iii)] Seien $a,b \in \overline{H}$, dann existieren Folgen $a_n,b_n \in H$, die gegen $a,b$ konvergieren. Dann ist $(a_n,b_n\i)$ eine Folge in $G\times G$, die gegen $(a,b\i)$ konvergiert. Da Multiplikation stetig ist, konvergiert $a_nb_n\i \in H$ gegen $ab\i$, ergo liegt $ab\i$ in $\overline{H}$. Analog zeigt man, dass $\overline{H}$ normal ist, falls $H$ normal ist.
\item[(iv)] Sei $H \leq_o G$ offen und sei $a \in \overline{H}$. Dann existiert eine Folge $a_n \in H$, die gegen $a$ konvergiert. $aH$ ist eine Umgebung von $a$, ergo existiert ein $N \in \N$, sodass $a_n \in aH$. Daraus folgt $a \in a_nH\i = H$.
\end{itemize}
\end{Beweis}

\Prop{}
\label{Prop5}
Sei $G$ eine topologische Gruppe. Dann sind folgende Aussagen äquivalent:
\begin{itemize}
\item[(i)] $G$ ist hausdorffsch.
\item[(ii)] $\{1\}$ ist abgeschlossen in $G$.
\item[(iii)] $\{g\}$ ist abgeschlossen in $G$ für alle $g \in G$.
\end{itemize}
\begin{Beweis}{}
Es bleibt die Implikation (iii) $\Impl{}$ (i) zu zeigen. Seien $g,h \in G$ verschieden. Dann ist $U = G \setminus \{gh\i\}$ offen in $G$. Laut Proposition \ref{Prop4} (i) existiert eine offene Teilmenge $V$ von $U$ mit folgenden Eigenschaften:
\begin{itemize}
\item $1\in V$
\item $VV \subset U$
\item $V\i = V$
\end{itemize}
Dann sind $Vg, Vh$ disjunkte Umgebungen von $g,h$. Denn wäre ihr Schnitt nichtleer, so würden $v,w\in V$ existieren, sodass $vg  = wh$, woraus folgt dass $gh\i $ in $ U$ liegen würde.
\end{Beweis}

\Prop{}
\label{Prop6}
Sei $G$ eine topologische Gruppe und $H \leq G$ eine Untergruppe.
\begin{itemize}
\item[(i)] $H$ ist genau dann diskret, wenn $H$ einen isolierten Punkt besitzt.
\item[(ii)] Ist $G$ hausdorffsch und $H$ diskret, so ist $H$ abgeschlossen.
\end{itemize}
\begin{Beweis}{(ii)}
$H$ ist diskret, d.\,h., es existiert eine offene Teilmenge $V \subseteq_o G$, s.\,d. $V \cap H = \{1\}$. Ohne Einschränkung darf angenommen werden, dass $V = V\i$.\\
G ist hausdorffsch, ergo ist $\{1\}$ abgeschlossen in $V$. Sei $x \in \overline{H}$, dann existiert ein $y \in H$, das in $xV$ liegt. Man erhält durch Umformung
\[ x \in yV \cap \overline{H} = \bigcap_{H \subset A \subset_a G} A \cap yV = \bigcap_{\{y\} = H \cap yV \subset A \subset_a yV} A  = \{y\} \]
Ergo gilt $x = y \in H$.
\end{Beweis}

\Prop{}
\label{Prop1.1.8}
Sei $G$ eine topologische Gruppe mit Untergruppe $H$.
\begin{itemize}
\item $G$ operiert stetig auf $G/H$.
\item $\pi_H : G \pfeil{} G/H$ ist eine offene Abbildung.
\item $G/H$ ist genau dann hausdorffsch, wenn $H$ abgeschlossen ist.
\item $G/H$ ist genau dann diskret, wenn $H$ offen ist.
\item Ist $H$ normal, so ist $G/H$ eine topologische Gruppe und $\pi_H$ ein Morphismus topologischer Gruppen.
\end{itemize}
\begin{Beweis}{(iii)}
 $\Longrightarrow$: Sei $a \in \overline{H}$, dann existiert eine Folge $a_n \in H$, die gegen $a$ konvergiert. Da $\pi_H$ stetig ist, gilt
\[ \pi_H(a_n) \Pfeil{n\pfeil{} \infty} \pi_H(a) \]
Da alle $a_n$ in $H$ liegen, gilt aber $\pi_H(a_n) = \pi_H(1)$. Da $G/H$ hausdorffsch ist, besitzt diese Folge höchstens einen Grenzwert, ergo gilt
\[ \pi_H(a) = \pi_H(1) \Impl{} a \in H \]
$\Longleftarrow$: Seien $\pi_H(b),\pi_H(c) \in G/H$. Ohne Einschränkung nehmen wir an, dass $\pi_H(c) = \pi_H(1)$.\\
In jeder Umgebung $\widetilde{U}$ von $\pi_H(b)$ sei $\pi_H(1)$ enthalten. Dann ist $b$ im Abschluss von $H$ enthalten, denn ist $U$ eine Umgebung von $b$, so ist $\pi(U)_H$ eine Umgebung von $\pi_H(b)$. Ergo ist $\pi_H(1) \in \pi_H(U)$, ergo existiert ein $h \in H$, sodass $h \in U$.
\end{Beweis}

\Def{}
Ist $G$ eine topologische, so ist $\overline{\{1\}}$ normal. $G/\overline{\{1\}}$ wird als \df{Hausdorffquotient} von $G$ bezeichnet.

\Def{}
Ein Homomorphismus $\phi : G \pfeil{} G'$ topologischer Gruppen heißt \df{strikt}, falls er den Isomorphiesatz respektiert, d.\,h., die induzierte Abbildung
\[ \phi : G / \Ker \phi \Pfeil{} \Img \phi \]
ist homöomorph.

\Def{}
Eine kurze exakte Sequenz topologischer Gruppen heißt \df{topologisch exakt}, falls alle beteiligten Abbildungen strikt sind.

\section{Lokal-Kompakte Gruppen}
\Def{}
Sei $X$ ein topologischer Raum.
\begin{itemize}
\item Wir nennen $X$ \df{kompakt}, falls er \df{quasikompakt} ist, d.\,h., jede offene Überdeckung von $X$ besitzt eine offene Teilüberdeckung.
\item $X$ heißt \df{lokal kompakt}, falls jeder Punkt eine Umgebung enthält, deren Abschluss kompakt ist.
\item 
\end{itemize}

\end{document}