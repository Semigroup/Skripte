\chapter{Norm-Index-Berechnungen}
\section{Hilbert '90}
\Def{Gruppenkohomologiegruppen}
Sei $G$ eine endliche Gruppe, $A$ ein multiplikativ geschriebener $G$-Modul.\\
Definiere die Gruppe der 1-\df{Kozykel} durch
\[Z^1(G,A) := \{ f : G \pfeil{} A ~|~ \forall  \sigma, \tau \in G: f(\sigma \tau) = (f(\sigma))^\tau \cdot f(\tau) \} \]
und die Gruppe der 1-\df{Koränder} durch
\[ B^1(G,A) := \set{f \in Z^1(G,A)}{ \exists a \in A: f(\sigma) = a^{\sigma - 1} } \]
Definiere die erste \df{Gruppenkohomologiegruppe} von $G$ mit Koeffizienten in $A$ durch
\[ H^1(G,A) := Z^1(G,A) / B^1(G,A) \]

\Satz{Hilbert '90}
\begin{itemize}
\item Ist $L|K$ eine zyklische, galoissche Erweiterung beliebiger Körper, so gilt
\[ H\i(G(L|K), L^\times) = 1 \]
\item Ist $L|K$ eine endliche, galoissche Erweiterung beliebiger Körper, so gilt
\[ H^1(G(L|K), L^\times) = 1 \]
\end{itemize}
\begin{Beweis}{}
\begin{itemize}
\item Da
\[ H\i(G(L|K), L^\times) = \ _{N_{L|K}}L^\times / I_{G(L|K)}L^\times \]
ist die Behauptung äquivalent zur Aussage, dass zu jedem $\alpha \in L^\times$ mit
\[ N_{L|K}(\alpha) = 1 \]
ein $\beta \in L^\times$ existiert, so dass
\[ \alpha = \beta^{\sigma - 1} \]
wobei $\sigma$ einen Erzeuger der zyklischen Gruppe $G(L|K)$ bezeichnet.\\
Es bezeichne $n = [L:K]$ den Grad der zyklischen Erweiterung. Die Automorphismen
\[ \id{L^{\times}}, \sigma, \ldots, \sigma^{n-1} : L^\times \Pfeil{} L^\times \]
stellen $n$ verschiedene Charaktere da und sind deswegen linear unabhängig. Insbesondere verschwindet folgende Linearkombination nicht
\[ \sum_{i=0}^{n-1} \klam{\alpha^{\sum_{j = 0}^{i - 1} \sigma^j }} \sigma^i = \id{L^\times} + \alpha \sigma + \alpha^{1 + \sigma} \sigma^2 +  \cdots + \alpha^{1 + \sigma + \cdots + \sigma^{n-2}} \sigma^{n-1}  \]
 Ergo existiert ein $\gamma \in L^\times$, sodass
\[ \beta := \gamma + \alpha \gamma^\sigma + \alpha^{1 + \sigma} \gamma^{\sigma^2} + \cdots + \alpha^{1+\sigma + \cdots + \sigma^{n-2}}\gamma^{\sigma^{n-1}} \neq 0 \]
Es gilt nun
\[ \alpha \beta^\sigma = \alpha (\gamma^\sigma + \cdots + \alpha^{\sigma + \cdots + \sigma^{n-1}}\gamma^{\sigma^n} ) = \beta \]
da
\[ \alpha^{\sigma + \cdots + \sigma^{n-1}}\gamma^{\sigma^n} = \frac{N_{L|K}(\alpha)}{\alpha} \gamma = \alpha\i \gamma \]
\item Sei $f \in Z^1(G(L|K),A)$. Aufgrund der Unabhängigkeit der Charaktere existiert ein $\gamma \in L^\times$ mit
\[ \alpha := \sum_{\sigma \in G} f(\sigma) \gamma^\sigma \neq 0  \]
Für beliebige $\tau \in G$ gilt nun
\[ \alpha^\tau = \sum_{\sigma \in G(L|K)} f(\sigma)^\tau \gamma^{\sigma \tau} \sum_{\sigma \in G(L|K)} f(\sigma\tau)f(\tau\i) \gamma^{\sigma \tau} = f(\tau\i)\alpha \]
Daraus folgt nun
\[ f(\tau) = \alpha^{1 - \tau} = \beta^{\tau - 1} \]
für $\beta := \alpha\i$. Ergo $f \in B^1(G(L|K),A)$
\end{itemize}
\end{Beweis}

\Satz{Gruppenkohomologie ist auch wirklich eine Kohomologietheorie}
Zu jeder exakten Sequenz von $G$-Moduln
\[ 1 \Pfeil{} A \Pfeil{} B \Pfeil{} C \Pfeil{} 1\]
existiert folgende lange exakte Kohomologiesequenz
\begin{align*}
1 \Pfeil{} A^G \Pfeil{} B^G \Pfeil{} C^G \Pfeil{} H^1(G,A) \Pfeil{} H^1(G,B) \Pfeil{} H^1(G,C) \Pfeil{} \ldots
\end{align*}

\section{Herbrand-Quotient}
\Lem{}
Sei $f : A \pfeil{} C$ ein Homomorphismus abelscher Gruppen und $B \leq A$ eine Untergruppe. Setze
\[ \ _fA := A_f := \set{a\in A}{f(a) = 1} = \Ker f_{|A} \text{ und } A^f := f(A) = \Img f_{|A} \]
Es gilt
\[ (A:B) = (A^f:B^f) \cdot (A_f : B_f) \]

\Def{Herbrand-Quotient}
Für Morphismen $f, g : A\pfeil{} A $ einer Gruppe $A$ definiere den Herbrand-Quotienten durch
\[ Q(A) := Q_{f,g}(A) := \frac{(A_f : A^g)}{(A_g : A^f)} \]

\Lem{}
Seien $f, g : A\pfeil{} A $ Homomorphismen einer abelschen Gruppe. $B \leq A$ sei eine Untergruppe, sodass $f(B),g(B) \subset B$.\\
Dann gilt
\[ Q(A) = Q(B) \cdot Q(A/B) \]
Ist ferner $A$ endlich, so ist $Q(A) = 1$.

\Bem{}
Sei $G = \shrp{\sigma}$ zyklisch, $A$ ein multiplikativer $G$-Modul und $f = 1 - \sigma$ und $g =N = \prod_{\sigma \in G}\sigma$ zwei Selbstabbildungen von $A$. Es gilt dann
\[ Q(G,A):=Q_{f,g}(A) = \frac{(A_f : A^g)}{(A_g : A^f)} = \frac{(A^G : NA)}{(N_A : I_GA)} = \frac{\# H^0(G,A)}{\# H\i(G,A)} \]

\Def{Induzierte Moduln}
Seien $H\leq G$ Gruppen, $B$ ein multiplikativ geschriebenes $H$-Linksmodul.\\
Definiere den durch $B$ \df{induzierten $G$-Modul} durch
\[ \textsf{Ind}_G^H(B) := \set{f : G \pfeil{} B}{f(hg) = \ ^hf(g) \forall g \in G, h\in H }  = \Hom{\text{Links-}H-\Mod}{G}{B} \isom{} G \otimes_{H} B \]
$G$ operiert auf $\textsf{Ind}^H_G(B)$ von links durch
\[ \ ^\sigma f := [g \in G \mapsto f(g\sigma) \in B ] \]
für $\sigma \in G, f \in \textsf{Ind}^H_G(B)$.\\
Für $H = \{1\}$ schreiben wir auch $\textsf{Ind}_G(B)$.

\Bem{}
\begin{itemize}
\item Der kanonische $H$-Homomorphismus
\begin{align*}
\pi : \textsf{Ind}^H_G(B) & \Pfeil{} B\\
f & \longmapsto f(1)
\end{align*}
bildet den $H$-Untermodul
\[ B' := \set{f \in \textsf{Ind}_G^H(B)}{ f(g) = 1 \forall g \notin H } \]
isomorph auf $B$ ab. Insofern können $B'$ und $B$ identifiziert werden und $B$ als Untermodul von $\textsf{Ind}^H_G(B)$ aufgefasst werden.
\item Ist $(G:H)$ endlich, so erhalten wir folgende Isomorphie
\begin{align*}
\textsf{Ind}_G^H(B) &\Pfeil{} \prod_{\rho \in G/H} \ ^\rho B\\
f & \longmapsto \prod_{\rho \in G/H} \ ^\rho f_{\rho}
\end{align*}
wobei
\begin{align*}
\ ^\rho f_{\rho} (g) := \left\lbrace
\begin{aligned}
f(g) && g\rho \in H\\
1 && g\rho \notin H
\end{aligned}
\right.
\end{align*}
$G$ operiert auf $\prod_{\rho \in G/H} \ ^\rho B$ durch
\[ \ ^g(\ ^\rho f_{\rho}) := \ ^{g\rho}b_\rho = \ ^{\rho'}(\ ^{h}b_\rho) = (\ ^gb)_{\rho'} \]
wobei $g \in G$ und $g\rho = \rho' h$ für $\rho' \in G/H$ und $h \in H$.
\item Ist $(G:H)$ endlich, so liegt folgende universelle Abbildungseigenschaft für $G$-Moduln $C$ vor
\begin{align*}
\Hom{G}{\textsf{Ind}_G^H(B)}{C} &\Pfeil{\isom{}} \Hom{H}{B}{C}\\
f& \longmapsto f_{|B}
\end{align*}
\end{itemize}

\Satz{}
Seien $H\leq G$ endliche Gruppen, $H$ ein multiplikativ geschriebenes Links-$H$-Modul.\\
Es liegt folgender kanonischer Isomorphismus vor
\[ H^i(G, \textsf{Ind}_G^H(B)) = H^i(H,B) \]
für $i = 0,1$.

\section{Motivationen für die Norm-Index-Berechnung}
\Bem{}
Sei $L|K$ eine endliche Erweiterung globaler Körper.\\
Es bezeichne $\delta = \delta_{L|K}$ die \df{Diskriminante} von $L|K$, d.\,h., dasjenige Ideal, das von allen Determinanten der Galoispermutationen aller in $\O_L$ liegenden Basen des $K$-Vektorraums $L$ erzeugt wird. Es gilt
\[ \pf \text{ verzweigt in }L \Gdw{} \pf | \delta \]

\Def{Artin-Symbol}
Sei $L|K$ eine abelsche Erweiterung globaler Körper, $G = G(L|K)$, $\pf \subset \O_K$ ein Primideal, welches nicht in $L$ verzweigt.\\
Definiere das \df{Artin-Symbol} von $\pf$ in $G$ durch
\[ (\pf, L/K) := \Frob_\pf \in G_\pf \subset G \]
Definiere die \df{Artin-Abbildung} durch
\begin{align*}
\omega : \I(\delta) & \Pfeil{} G\\
\bf = \prod_{\pf \text{ prim zu }\delta} \pf^{v_{\pf}} & \longmapsto (\bf,L/K) := \prod_{\pf} (\pf, L/K)^{v_\pf}
\end{align*}
Ist $0\neq \cf \subset \O_K$ ein weiteres Ideal, das durch alle verzweigenden Primideale geteilt wird, so ergibt sich analog folgendes Diagramm
\begin{center}
\begin{tikzpicture}[scale =1]
\node (C) at (0,1)  {$\I(\cf)$};
\node (D) at (0,0)  {$\I(\delta)$};
\node (G) at (2,0)  {$G$};

\draw[right hook->] (C) -> (D) node[midway, left]{};
\draw[->] (C) -> (G) node[midway, above right]{$\omega_\cf$};
\draw[->] (D) -> (G) node[midway, below]{$\omega$};
\end{tikzpicture}
\end{center}

\Satz{}
$\omega_\cf$ ist surjektiv.
\begin{Beweis}{}
Es bezeichne $H := \Img \omega_\cf \leq G(L|K)$, $F = L^H$.\\
Jedes $\pf \in \I(\cf)$ zerlegt sich voll in $F$, da
\[ \Frob_\pf =  (\pf, F/K) =(\pf, L/K)_{|F} = \omega_\cf(\pf)_{|F} \in H/H = 1 \]
Sei $F'|K$ eine beliebige, zyklische Zwischenerweiterung. In $F'$ sind nun alle bis auf endliche viele Primideale von $\O_K$ voll zerlegt. Mit der folgenden Proposition folgt nun, dass $F' = K$. Ergo ist $F$ als Vereinigung aller zyklischen Zwischenerweiterungen gleich $K$. Ergo ist $H = G(L|K)$.
\end{Beweis}

\Prop{}
Ist $L|K$ eine echte, zyklische Erweiterung globaler Körper, so zerlegen sich unendlich viele Primideale nicht voll in $L$.

\begin{Beweis}{}
Angenommen es gäbe nur endliche viele Primstellen, die sich nicht voll zerlegen. $S$ bezeichne die Menge aller dieser Primstellen. Sei $a \in \A_K^\times$ ein Idel. Da $S$ endlich ist, erhalten wir durch den Approximationssatz erhalten ein Hauptideal $\alpha \in K^\times$, sodass
\[ (\alpha a)_v \in N_{L_w|K_v}L_w^\times\]
für alle $v \in S$. Da alle $v\notin S$ total zerlegt sind, erhalten wir für diese Stellen
\[ (\alpha a)_v \in K_v^\times = L_w^\times = N_{L_w|K_v}L_w^\times \]
Ergo ist das Idel $\alpha a$ eine Norm und es folgt
\[ a \in K^\times N_{L|K}\A_L^\times \]
Da $a \in \A_K^\times$ aber beliebig war, würde hieraus folgen
\[ (\A_K^\times :K^\times N_{L|K} \A_L^\times  ) = 1\]
was ein Widerspruch zum folgenden Satz ist.  
\end{Beweis}

\Satz{}
Ist $L|K$ eine zyklische Erweiterung globaler Körper von Grad $n$, so gilt
\[ (\A_K^\times : K^\times N_{L|K} \A_L^\times ) = (C_K : N_{L|K}C_L) = n \]

\Def{Führer der Artin-Abbildung}
Sei $L|K$ eine abelsche Erweiterung globaler Körper.\\
Unter einem \df{Führer der Artin-Abbildung} verstehen wir ein zulässiges Primideal $\ff$, welches von allen verzweigenden Primidealen geteilt wird und für das gilt
\[ \P(\ff) \subset \Ker \omega_\ff \]

\Bem{}
Für ein gebrochenes Ideal $\af \in \I(\ff)$ von $L$, welches teilerfremd zu den verzweigenden Primidealen ist, gilt
\[ (N_{L|K}\af, L|K) = (\af, L|L) = 1 \]
da $\N(\ff) = N_{L|K} \I(\ff)$, folgt
\[ \P(\ff) \N(\ff) \subset \Ker \omega_\ff  \]
d.\,h.
\[ C_K/N_{L|K}C_L \isom{} \I(\ff) / (\P(\ff) \N(\ff)) \surj{} G(L|K) \]
Mit dem folgenden Satz gilt nun, dass $\omega_\ff$ ein Isomorphismus ist.

\Satz{Universelle Normenungleichung}
Sei $L|K$ eine Galoiserweiterung von Grad $n$. Es gilt
\[ (C_K : N_{L|K}C_L ) \leq n \]

\section{Der Lokale Norm-Index}

\Bem{Unverzweigt bei Unendlich}
Ist $v$ eine archimedische Stelle einer Körpererweiterung $L|K$, so heißt $w|v$ \df{unverzweigt}, falls $L_w = K_v$.\\
D.\,h., $\C|\R$ ist verzweigt mit Verzweigungsgrad 2.

\Satz{}
Sei $L|K$ eine zyklische Erweiterung lokaler Körper, $G = G(L|K)$, $e$ der Verzweigungsindex. Dann gilt
\begin{itemize}
\item $\# H^0(G,L^\times) = (K^\times : N_{L|K}L^\times) = [L:K]$
\item $H\i(G,L^\times) = 0$
\item $(\O_K^\times : N_{L|K}\O_L^\times) = e$
\item $Q(G, \O_L^\times) = 1$, d.\,h. 
\[ \#H^0(G, \O_L^\times) = \#H\i(G, \O_L^\times) = e \]
\end{itemize}
\paragraph{Bemerkung}
Das Ergebnis der lokalen Klassenkörpertheorie wird am Ende sein, dass dieser Satz für abelsche Erweiterungen lokaler Körper gilt.\\
Ersetzt man $\O_K^\times$ durch $K_v^\times$ für $K = \R,\C$, so gilt der Satz auch für $\R$ und $\C$.

\Satz{Normalbasissatz}
Sei $L|K$ eine beliebige, endliche Galoiserweiterung. Dann gilt folgende Isomorphie von $G$-Moduln
\[ L \isom{} \textsf{Ind}_GK = K[G] \]
D.\,h., es existiert ein $b \in L$, sodass $\set{b^\sigma}{\sigma \in G(L|K)}$ eine Basis von $L$ ist.

\Lem{}
Sei $L|K$ eine endliche Galoiserweiterung mit Galoisgruppe $G(L|K) = \{\sigma_1, \ldots, \sigma_n\}$.\\
Es gilt für Elemente $a_1,\ldots, a_n \in L$
\[ a_1,\ldots, a_n \text{ sind eine }K\text{-Basis für }L \Gdw{} \det(a_j^{\sigma_i}) \neq 0 \]

\Kor{}
Ist $L|K$ eine unverzweigte Erweiterung lokaler Körper, so ist
\[ N_{L|K} : \O_L^\times \Pfeil{} \O_K^\times \]
unverzweigt.

\Kor{}
Sei $L|K$ eine beliebige, abelsche Erweiterung lokaler Körper so gilt
\begin{itemize}
\item $(K^\times : N_{L|K} L^\times ) ~|~ [L:K]$
\item $(\O_K^\times : N_{L|K} \O_L^\times ) ~|~ e$
\end{itemize}

\Bem{}
Sei $L|K$ eine endliche, abelsche Erweiterung von Zahlkörpern, und $\cf \subset \O_K$ ein zulässiges Ideal.\\
Ist $v$ eine endliche, verzweigende Stelle in $L$, so gilt $v | \cf$.

\section{Globale Norm-Index-Berechnungen}
\Def{}
Sei $L|K$ eine endliche Galoiserweiterung globaler Körper, $G = G(L|K)$, $v$ eine Stelle von $K$.\\
Definiere folgende $G$-Moduln
\[ L_v^\times := \prod_{w|v} L_w^\times \supseteq \O_{L,v}^\times := \prod_{w|v}\O_{L_w}^\times \]
Dann ist
\[ \A_L^\times = \prod'_vL_v^\times \text{ restringiert bzgl. } \O_{L,v}^\times  \]
Fixiert man ein $w_0|v$, so gilt
\[ L_v^\times = \prod_{\sigma \in G/G_{w_0}} L_{\sigma w_0}^\times = \prod_{\sigma \in G/G_{w_0}} \ ^\sigma L_{w_0}^\times \text{ und } \O_{L,v}^\times = \prod_{\sigma \in G/G_{w_0}} \ ^\sigma \O_{L_{w_0}}^\times \]
Es gilt ergo
\[ L^\times_v = \textsf{Ind}_G^{G_{w_0}} (L_{w_0}^\times) \text{ und } \O^\times_{L, v} = \textsf{Ind}_G^{G_{w_0}} (\O_{L_{w_0}}^\times) \]

\Satz{}
Sei $L|K$ eine zyklische Erweiterung und $S$ die Menge aller Stellen von $K$, die in $L$ verzweigen.\\
Setze
\[ \A_{L,S}^\times := \prod_{v\in S}L_v^\times \times \prod_{v\notin S} \O_{L,s}^\times \]
Für jede Stelle $v$ von $K$ bezeichne $w$ eine ausgewählte Stelle über $v$. Dann gilt für $i = 0,-1$
\begin{align*}
H^i(G, \A_{L,S}^\times ) &= \bigoplus_{v\in S}H^i(G_w, L_w^\times)\\
H^i(G, \A_{L}^\times ) &= \bigoplus_{v}H^i(G_w, L_w^\times)\\
\end{align*}
\Bem{}
Es gilt mit obigem Satz
\begin{align*}
H\i (G, \A_L^\times) &= 0\\
H^0(G,\A_L^\times) &= \A_K^\times/N_{L|K}\A_L^\times = \bigoplus_{v} K_v^\times / N_{L_w|K_v}L_w^\times
\end{align*}
Mit anderen Worten, ein Idel von $K$ ist genau dann Norm eines Ideles von $L$, wenn es überall lokal eine Norm ist.

\Satz{}
Sei $L|K$ eine zyklische Erweiterung globaler Körper, $S$ bezeichne die Menge aller verzweigenden Stellen. Es gilt
\[ Q(G, \A_{L,S}^\times) = \prod_{v\in S} [L_w:K_v] \]

\Satz{}
Sei $K$ ein Zahlkörper, $S \supseteq S_\infty$ eine endliche Menge von Stellen von $K$.\\
Definiere die Menge der \df{$S$-Einheiten} durch
\[ K_S := K^\times \cap \A_{K,S}^\times \]
Der Homomorphismus
\begin{align*}
\lambda_S : K_S &\Pfeil{} \prod_{v\in S} \R\\
a & \longmapsto (\log \bet{a}_v)_{v\in S} 
\end{align*}
hat als Kern $\mu(K)$, die Menge aller Einheitswurzeln in $K$. Sein Bild ist ein Gitter im $(\#S-1)$-dimensionalen \df{Spur-Null-Raum}, der definiert ist durch
\[ \H:= \set{(x_v)_v \in \prod_{v\in S}\R }{\sum_{v\in S}x_v = 0} \]

\Lem{}
Sei $V$ ein $n$-dimensionaler $\R$-Vektorraum, $G \leq \Aut{\R}{V}$ eine endliche Untergruppe, die eine gegebene Basis $v_1,\ldots,v_n$ von $V$ permutiert durch
\[ \sigma(v_i) =: v_{\sigma(i)} \]
für $\sigma \in G, i = 1,\ldots,n$.\\
Sei $\Gamma \leq V$ ein $G$-invariantes Gitter, d.\,h., $G\Gamma \subset \Gamma$. Dann gibt es ein Untergitter $\Gamma' = \Z w_1 + \ldots \Z w_n \leq \Gamma$, sodass
\[ \sigma w_i = w_{\sigma(i)} \]
für $\sigma \in G, i = 1,\ldots,n$.

\Satz{}
Sei $L|K$ eine zyklische Erweiterung globaler Körper, $S \supset S_\infty$ eine endliche Stellenmenge von $K$.\\
Es bezeichne 
\[\overline{S} = \set{w \text{ Stelle von }L}{w|v \text{ für ein } v \in S} \]
die über $S$ liegenden Stellen und setze ferner
\[ L_S := L_{\overline{S}} = L^\times \cap \A_{L,\overline{S}} \]
Dann gilt, wobei je $v \in S$ ein passendes $w\in \overline{S}$ gewählt wird,
\[ Q(G,L_S) = \frac{\prod_{v\in S} [L_w : K_v]}{[L:K]} \]

\Satz{}
Sei $L|K$ eine zyklische Erweiterung globaler Körper vom Grad $n$, $G = G(L|K)$ ihre Galoisgruppe. Es gilt
\[ Q(G, C_L) = \frac{\# H^0(G,C_L)}{\# H\i(G,C_L)} = [L:K]  \]
Zusammen mit der universellen Normenungleichung  folgt nun
\[ (C_K : N_{L|K} C_L ) = n \]
Insbesondere erhalten wir nun die Klassenkörperaxiome
\[ \# H^i(G,C_L) = \left\lbrace
\begin{aligned}
\ [L:K] && i = 0\\
1 && i = -1
\end{aligned}
\right. \]

\begin{Beweis}{}
Wähle eine endliche Stellenmenge $S \supset S_\infty$, so dass
\[ \A_L^\times = L^\times \A_{L,S}^\times \]
Dies ist laut dem nächsten Lemma möglich. Es ergibt sich nun folgende exakte Sequenz
\[ 1 \Pfeil{} L_S \Pfeil{} \A_{L,S}^\times \Pfeil{} \A_{L,S}^\times L^\times /L^\times = C_L \Pfeil{} 1 \]
Da der Herbrand-Quotient multiplikativ ist, ergibt sich
\[ Q(C_L) = Q(L_S)\i Q(\A_{L,S}^\times) = \frac{[L:K]}{\prod_{v\in S} [L_w : K_v]} \prod_{v\in S} [L_w : K_v] = [L:K] \]
\end{Beweis}

\Lem{}
Sei $K$ ein globaler Körper. Es seien folgende Repräsentanten $\{\af_1,\ldots , \af_n \} = Cl(K) = \I / \P$ der Idealklassengruppe gegeben.\\
$S \supset S_\infty$ sei eine endliche Stellenmenge, die alle endlichen Stellen $\pf$ enthalte, die im Träger eines $\af_i$
\[ \textsf{supp}(\af_i) = \set{\pf \in \textsf{spec}\O_K}{\pf | \af_i} \]
liegen. Dann gilt
\[ \A_K^\times = \A_{K,S}^\times K^\times \]

\Satz{Hasses Normensatz}
Sei $L|K$ eine zyklische Erweiterung, $x \in K^\times$. Es gilt
\[ x \text{ ist eine Norm, d.\,h. } x \in N_{L|K}L^\times \Gdw{} x \text{ ist überall lokal eine Norm} \]
\begin{Beweis}{}
Betrachte die exakte Sequenz
\[ 1 \Pfeil{} L^\times \Pfeil{} \A_K^\times \Pfeil{} C_L \Pfeil{} 1 \]
Sie induziert folgende exakte Sequenz in Kohomologie
\[ 1 = H\i(G,C_L) \Pfeil{} H^0(G,L^\times) \Pfeil{} H^0(G,\A_L^\times) = \bigoplus_v H^0(G_w, L_w^\times) \]
Wir lesen folgende Inklusion ab
\[ K^\times / N_{L|K}L^\times \Inj{} \bigoplus_v K_v^\times / N_{L_w|K_v}L_w^\times \]
\end{Beweis}