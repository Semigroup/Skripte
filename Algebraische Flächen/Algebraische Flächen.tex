\documentclass{article}

\usepackage{../Package/latexa}
\usepackage{../Package/algebra}
\usepackage{../Package/theorema}
\usepackage{../Package/diagramma}
\usepackage{../Package/categoria}

\usepackage{tikz}
\usepackage{tikz-cd}
\usetikzlibrary{arrows}

\newcommand{\qi}{\backsimeq_{\textsc{QI}}}
\newcommand{\ba}{\backsimeq_{\textsc{BA}}}
\newcommand{\normal}{\vartriangleleft}
\newcommand{\tm}{\subset}
\newcommand{\Stab}[2]{\textsf{Stab}_{#1}(#2)}
\newcommand{\Cay}[2]{\textsf{Cay}(#1,#2)}


\renewcommand{\A}{\mathbb{A}}
\newcommand{\Nc}{\mathcal{N}}

\renewcommand{\i}{^{-1}}
\renewcommand{\d}{\textsf{d}}
\renewcommand{\P}{\mathbb{P}}

\newcommand{\af}{\mathfrak{a}}
\newcommand{\Af}{\mathfrak{A}}
\renewcommand{\bf}{\mathfrak{b}}
\newcommand{\Bf}{\mathfrak{B}}
\newcommand{\cf}{\mathfrak{c}}
\newcommand{\Cf}{\mathfrak{C}}
\newcommand{\ff}{\mathfrak{f}}
\newcommand{\Ff}{\mathfrak{F}}
\newcommand{\pf}{\mathfrak{p}}
\newcommand{\Pf}{\mathfrak{P}}
\newcommand{\mf}{\mathfrak{m}}
\newcommand{\Mf}{\mathfrak{M}}
\newcommand{\qf}{\mathfrak{q}}
\newcommand{\Qf}{\mathfrak{Q}}

\newcommand{\Ac}{\mathcal{A}}

\newcommand{\Frob}{\textsf{Frob}}

\newcommand{\Leg}[2]{\left(\frac{#1}{#2}\right)}

\makeindex

\begin{document}

\section{Einführung}
\subsection{Algebraische Kurven}
Jede glatte, projektive Kurve $C$ über $\C$ ist eine eindimensionale, kompakte komplexwertige Riemannsche Fläche, die topologisch durch ihr Geschlecht klassifiziert werden kann.
\[ g = \frac{1}{2} \textsf{rk}H_1(C,\Z) = \dim_\C H^0(C,K_C) \]
$K_C= \Omega_C^1$ bezeichnet hierbei das \df{kanonische Bündel}, d.h., für jede offene Menge $U$ ergibt sich
\[ K_C(U) = \set{f(z)\d z}{f \text{ ist holomorph auf } U} \]
In dieser Hinsicht unterscheidet man drei Fälle:
\begin{itemize}
\item $g = 0$: Dann ist $C$ einfach zusammenhängend, ergo isomorph zu $\P^1 = \C\cup \{\infty\}$. Es gilt
\[ K_C \isom{} \O_{\P_1} (-2) \]
\item $g= 1$: Dann handelt es sich bei $C$ um eine elliptische Kurve. Es existiert in diesem Fall ein $\tau \in \H = \set{z \in \C}{\Im z > 0}$, sodass
\[ C \isom{} \C / (\Z \oplus \tau \Z) \isom{} E \subset \P^2 \]
Ferner gilt
\[ K_C = \O_C \]
Außerdem ist die universelle Überlagerung von $C$ ganz $\C$ in diesem Fall und es gilt
\[ \set{\text{Elliptische Kurven}}{} / \isom{} \Pfeil{\isom{}} \A^1 \]
\item $g > 1$: In diesem Fall ist $C$ isomorph zu $\H / \Gamma$, wobei $\Gamma$ eine Fuchssche Gruppe, d.h. diskrete Untergruppe von $\text{PSL}_2(\R)$, ist, die auf $\H$ operiert. Der Grad von $K_C$ ist $2g-2$ und für jedes $g> 0$ existiert eine quasi-projektive Varietät $M_g$ mit
\[ \set{\text{Glatte projektive Kurven von Geschlecht }g / \C}{} / \isom{} \Pfeil{\isom{}} M_g \]
\end{itemize}

\subsection{Projektive Varietäten}
\Def{Der projektive Raum} Definiere den \df{komplexwertigen projektiven Raum} der Dimension $n$ durch
\[ \P^n := (\C^{n+1} \setminus \{0\}) / \sim = \textsf{Proj}\C[X_0,\ldots, X_n]  \]
wobei $a \sim b :\Gdw{} \exists \lambda \in \C^\times : a = \lambda b$ für $a,b \in \C^{n+1} \setminus\{0\}$.\\
$\P^n$ wird von folgenden Karten überdeckt
\[ U_i = \set{ [z_0 : \ldots : z_n]}{z_i \neq 0} \isom{} \A^n \]

\Def{Zariski-Topologie} Eine Teilmenge $V \subset \P^n$ heißt \df{abgeschlossen} in der Zariski-Topologie, falls ein homogenes Ideal $I \subset \C[X_0,\ldots, X_n]$ existiert, sodass $V = V(I)$.\\
$V$ heißt \df{reduzibel}, falls $V = V_1 \cup V_2$ mit $\emptyset \neq V_1, V_2 \subsetneq_a V$

\Bem{}
Jede abgeschlossene Teilmenge von $\P^n$ ist eine Vereinigungen von irreduziblen, abgeschlossenen Komponenten.

\Def{}
Definiere die \df{Dimension} von $V \subseteq_a \P^n$ durch das maximale $d$, für welches irreduzible, abgeschlossene Mengen $X_i \subseteq \P^n$ existieren, sodass sich folgende, echt aufsteigende Kette ergibt
\[ \emptyset \neq X_0 \subsetneq X_1 \subsetneq \ldots \subsetneq X_n \subseteq V \]

\Def{}
Eine \df{projektive Varietät} ist ein geringter Raum isomorph zu $(V, \O_V)$ mit $V \subset_{a,i} \P^n$ und $\O_V = \O_{\P^n} / \I_V$.\\
Morphismen sind stetige Abbildungen $f : V \pfeil{} W$ mit
\[ f\i \O_W \inj{} \O_V \]
wobei $f\i O_W$ die Garbifizierung der Prägarbe
\[ U \longmapsto \lim_{f(U) \subseteq X \subseteq_o W } \O_W(X) \]

\Def{}
Setze für $p \in V$
\[ \O_{V,p} := \lim_{p \in U \subseteq_o V} \O_V(U) \triangleright \mf_{V,p} := \set{f \in \O_{V,p}}{f(p) = 0} \]
\[ T_pV := \Hom{\C}{\mf_{V,p} /\mf_{V,p}^2}{\C} \]
Es gilt
\[ \dim_\C(T_pV) \geq \dim V \]
Gilt hier Gleichheit, so heißt $V$ in $p$ \df{glatt}.

\subsection{Birationale Abbildungen}
\Lem{Aufblasung}
Sei $X$ eine glatte projektive Varietät, $n = \dim X$ und $p \in X$.\\
Dann existiert eine weitere glatte projektive Varietät $Y$ und eine Abbildung $f : Y \pfeil{} X$, sodass
\[ Y \setminus f\i(p) \Pfeil{\isom{}, f} X \setminus \{p\} \]
und
\[ f\i(p) \isom{} \P^{n-1} \]
\begin{Beweis}{}
Setze
\[ Y = \textsf{Proj}_{\O_X}( \bigoplus_{n\geq 0} I_p^n) \]
\end{Beweis}

\Def{}
Ein Morphismus $f : X \pfeil{} Y$ projektiver Varietäten heißt \df{birational}, falls offene Mengen $X_o \subset X, Y_o \subset Y$ existieren, sodass
\[ X_o \Pfeil{f, \isom{}} Y_o \]
In diesem Fall heißen $X$ und $Y$ \df{birational äquivalent}.

\subsection{Numerische Invarianten}
\Def{Plurigenera}
Definiere für eine glatte projektive Varietät $X$ seinen \df{Plurigenus} durch
\[ P_m(X) := \dim H^0(X, K_X^{\otimes m}) \]
Man kann zeigen, dass für jedes $X$ ein $\kappa = \kappa(X) \in \set{-\infty, 0, 1, \ldots, \dim X}{}$ und $a,b > 0$ existieren, sodass für fast alle $m$ gilt
\[ a m^\kappa \leq P_m(X) \leq bm^\kappa \]
Man nennt $\kappa(X)$ die \df{Kodaira-Dimension} von $X$.
\end{document}