\marginpar{Lecture from 01.04.2020}
\begin{proof}[Finishing Proof of Theorem.]
	Since $G' = \shrp{ghg\i h\i ~|~ g,h \in G}$, so $\det(r(G')) = 1$.
	
	On the other hand, for each $g \in G'$, we have
	\[
	r(g) = \mat{
		\chi_0(g) \\
		& \ddots \\
		& & \chi_0(g)
	}
	\]
	since $V = V_{\chi_0}$. This implies
	\[ 1 = \det(r(g)) = \chi_0(g)^d. \]
	Ergo, $\chi_0$ defines a morphism
	\[ \chi_0 : G' \Pfeil{} \mu_d \subseteq \G_m. \]
	
	But $G'$ is connected and $\mu_d$ is finite. Since $\chi_0$ is a morphism, $\chi_0$ must be constant, ergo the trivial character.
	
	As a consequence, we get $r(G') = 1$ on $V = V_{\chi_0}$.
	
	\begin{lemma}
		Let $G$ be an algebraic group, $r : G \pfeil{} \GL(V)$ a representation. $v \in V$ shall be a simultaneous non-zero eigenvector for $r(G)$.
		
		Then, for each $g \in G$, there is a value $\chi(g) \in k^\times$ s.t.
		\[ r(g) v =: \chi(g)v. \]
		Then, the mapping $\chi : G \pfeil{} \G_m$ is a morphism of algebraic groups.
	\end{lemma}
	
	Therefore, $r $ descends to a representation of the commutative group
	\[ \overline{r} : G / G' \Pfeil{} \GL(V). \]
	Ergo, $r(G / G') = r(G)$ is commutative and therefore trigonalizable (because of irreducibility).
	
\end{proof}


\begin{example}[Non-Example]
\begin{itemize}
	\item Take $ G = D_4 \inj{} \GL_2(\C)$ which is solvable and has an irreducible and faithful representation over $\C^2$. 
	\item Consider the solvable group
	\[ G = \shrp{\mat{\pm 1 \\ & \pm 1}, \mat{& 1 \\ 1}} \]
	which is a finite subgroup of $\GL_2(\C)$, s.t. $\C^2$ defines an irreducible representation of $G$.
\end{itemize}
\end{example}

\begin{lemma}[Form of Schur's Lemma]
	Let $S$ be any commutative subset of $\GL(V)$ for a finite-dimensional $0\neq V$ over an algebrically closed field $k$. Let $V$ be $S$-irreducible.
	
	Then, $\dim V = 1$.
\end{lemma}
\begin{proof}
	There is nothing to show if $S$ is empty.
	
	Let $s \in S$ and denote by $V_\lambda \subseteq V $ the $\lambda$-eigenspace for $s$. Then, since $S$ is commutative, $V_\lambda$ is $S$-invariant. Therefore, $V = V_\lambda$ for one $\lambda \in k^\times$.
	
	Thus, every $s \in S$ acts by scaling, therefore every subspace of $V$ is $S$-invariant. Since $V$ is invariant, we get $\dim V = 1$.
\end{proof}

\begin{corollary}
	Let $G$ be a connected algebraic group. Then, $G$ is solvable iff $G$ is trigonalizable.
\end{corollary}

\begin{proposition}
	If $G$ is trigonalizable, then $G_u$ is a normal algebraic subgroup.
\end{proposition}
\begin{proof}
	We have
	\[ G \inj{} B:= \curv{
\mat{* & \ldots & *\\  & \ddots & \vdots \\ & & *}	
}  \subseteq \GL_n(k). \]
$B$ has the normal subgroup $U := \curv{
	\mat{1 & \ldots & *\\  & \ddots & \vdots \\ & & 1}	
}$ and we have $G_u = G \cap U$. Now, $U$ is the kernel of the multiplicative morphism
\[ 
\mat{a_1 & \ldots & * \\ & \ddots & \vdots \\ & & a_n} \longmapsto \mat{a_1 &  &  \\ & \ldots &  \\ & & a_n}.
\]
\end{proof}
\begin{corollary}
	If $G$ is connected and solvable, then $G_u$ is a normal algebraic subgroup.
\end{corollary}

\newpage
\section{Semisimple Elements of nilpotent Groups}

\begin{theorem}
	Let $G$ be a connected nilpotent algebraic group. Then, we have
	\[ G_s \subseteq Z(G) \]
	where $Z(G)$ denotes the \df{center} of $G$, i.e.
	\[ Z(G) = \set{g \in G}{\forall h \in G:~gh = hg}. \]
\end{theorem}
\begin{theorem}[Lie-algebraic Analogue]
Let $V$ be a finite-dimensional vectorspace. Let $\mathfrak{g}$ be the Lie-Subalgebra of $\End(V)$, i.e. $\mathfrak{g}$ is a subspace s.t. we have for each $x,y \in \mathfrak{g}$
\[ [x,y] := xy - yx \in \mathfrak{g}. \]
Assume that $\mathfrak{g}$ is nilpotent, i.e. there is an $n \in \N_0$ s.t.
\[ [x_1, [x_2, [\ldots, [x_{n-1}, x_n]]]] = 0 \]
for all $x_1, \ldots, x_n \in \mathfrak{g}$.

Then, any semisimple (semisimple in $\End(V)$ that is) $x \in \mathfrak{g}$ is \df{central} in $\mathfrak{g}$, i.e. $[x,y] = 0$ for each $y \in \mathfrak{g}$.
\end{theorem}
\begin{remark}
	The Lie-algebraic Analogue implies the general theorem if -- for example -- $k = \C$.
\end{remark}
\begin{proof}
Let $g \in G_s$. We want to show $Z_G(g) = G$.

\newcommand{\Lie}{\mathrm{Lie}}
\paragraph{Fact from the theory of Lie-Algebras:}
For the Lie-Algebra $\Lie Z_G(g)$ we have
\[ \Lie Z_G(g) = \ker (\Ad(g)) \]
where $\Ad$ is the map
\begin{align*}
\Ad : G & \Pfeil{} \GL(\mathfrak{g})\\
x & \longmapsto gxg\i.
\end{align*}
Since $G$ is connected, it suffices to verify
\[ \ker (\Ad (g)) = \mathfrak{g} \]
i.e. $\Ad(g) = 1$.

Since $g$ is semisimple, we have for suitable basis
\[ g = \mat{a_1 \\ & \ddots \\ & & a_n} \]
with $a_j \in \C^\times$. This is $\exp(x)$ for a suitable diagonal matrix $x \mat{x_1 \\ & \ddots \\ & & x_n} \in \GL_n(\C)$.
\paragraph{Fact:}
We may assume that $x \in \mathfrak{g} := \Lie (G)$.

Since $G$ is nilpotent, it can be shown that $\mathfrak{g}$ is nilpotent.

By the theorem, $x$ is central in $\mathfrak{g}$. By the properties of $\exp$ we have
\[ \Ad(g) = \exp(\mathrm{ad}(g)) = 1 \]
ergo $\mathrm{ad}(x) = 0$ where $\mathrm{ad}: \mathfrak{g} \pfeil{} \mathfrak{g}$ is defined by 
\[ \mathrm{ad}(x) \cdot y := [x,y]. \]
\end{proof}

\begin{proof}
If $\mathfrak{g}$ is nilpotent, then $\mathrm{ad}(x) \in \End(\mathfrak{g})$ is nilpotent.

Since $x$ is semisimple, $\mathrm{ad}(x)$ is semisimple, because $\mathrm{ad}(x)$ is the restriction to $\mathfrak{g}$ of the map
\begin{align*}
\End(V) & \Pfeil{} \End(V)\\
y & \longmapsto [x,y]
\end{align*}
and, if $e_1, \ldots, e_n$ are a basis of eigenvectors for $x$, then $E_{i,j}$ is a basis of eigenvectors for $\ell$.

So, $\mathrm{ad}(x)$ is nilpotent and semisimple, therefore $\mathrm{ad}(x) = 0$.
\end{proof}

\begin{proof}[Proof Theorem]
	Let $G$ be a connected nilpotent algebraic group, $G \inj{} \GL(V)$.
	
	Let $g \in G_s$, we want to show that $g \in Z(G)$.
	
	Assume otherwise, then we have a $h \in G$ s.t. $(g,h) = ghg\i h\i \neq 1$.
	
	Since $G$ is connected and nilpotent (ergo solvable), we know by Lie-Kolchin that $G$ stabilizes some complete flag $V_0 \subset \ldots \subset V_n$.
	
	We have $g|_{V_i}, h|_{V_i} \in \GL(V_i)$. They commute, if $i = 0$, but not if $i = n$.
	
	So, there is an $i$ s.t. $g|_{V_i}, h|_{V_i}$ commute but  $g|_{V_{i+1}}, h|_{V_{i+1}}$ don't commute. W.l.o.g. $V =V_{i+1}, g = g|_{V_{i+1}}, h = h|_{V_{i+1}}$.
	Set $a:= g|_{V_i}, b:=  h|_{V_i} \in \GL(V_i)$. $a$ will be semisimple, since $g$ is.
	
	Since $g$ is semisimple, there is an eigenvector $v \in V_{i+1}$ for $g$ s.t.
	\[ V_{i+1} = V_i \oplus \shrp{v}. \]
	We have an isomorphism of vector spaces
	\[ \End(V_{i+1}) \isom{} \End(V_i) \oplus \Hom{}{\shrp{v}}{V_i} \oplus \Hom{}{V_i}{\shrp{v}} \oplus \End(\shrp{v}) \]
	with
	\[ \End(\shrp{v}) \isom{} k \text{  and  } \Hom{}{\shrp{v}}{V_i} \isom{} V_i. \]
	So, we can write $ g|_{V_{i+1}},   h|_{V_{i+1}}$ as
	\[
	g = \mat{a \\ & * \in k} \text{  and  } 	h = \mat{b & c \in V_i \\ & * }.
	\]
	We may replace $g,h$ with scalar multiples to reduce to the case that $* = 1$. Then,
 we can write $ g|_{V_{i+1}},   h|_{V_{i+1}}$  as
	\[
	g = \mat{a \\ & 1} \text{  and  } 	h = \mat{b & c  \\ & 1 }.
	\]
	Then,
	\[ h\neq ghg\i = \mat{b & ac \\ & 1}. \]
	Ergo, $c \neq ac$, i.e. $c \notin \ker(a -1)$. Define
	\[ h_1:= h\i gh g\i = \mat{
	1 & b\i(a-1)c\\ & 1}. \]
We claim that $h_1$ does not commute with $g$. This claim implies the theorem, since we can iterate the claim to obtain elements $h_i$ by $h_{i+1} := h_i\i g h_i g\i$. Then, $h_i$ does not commute with $g$. But $G$ is nilpotent, therefore $h_i = 1$ for some large enough $i$.

We can prove the claim as follows: By some calculation as for $h$ and $g$, we see, that $h_1$ and $g$ don't commute iff $b\i(a-1)c \notin \ker(a-1)$. This is equivalent to
\begin{align*}
\iff & (a-1)b\i(a-1)c \neq 0\\
\iff & b\i(a-1)^2c \neq 0\\
\iff & (a-1)^2c \neq 0\\
\iff & c \notin \ker((a-1)^2).
\end{align*}
But $a$ being semisimple implies $a-1$ being semisimple, therefore
\[ \ker((a-1)^2) = \ker(a-1).\]
So $h_1,g$ don't commute iff $c \in \ker(a-1) $ iff $h,g$ don't commute.
\end{proof}