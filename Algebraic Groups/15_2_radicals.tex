\subsection{Radicals}
Let $G$ be a connected algebraic group.

\begin{definition}
	The \df{radical} $R(G)$ of $G$ is defined as the intersection of all Borel subgroups of $G$ i.e.
	\[ R(G) := \bigcap_{B \subset G \text{ Borel}} B. \]
	The \df{unipotent radical} is defined by
	\[ R_u(G) := R(G)_u = \{ \text{unipotent elements of } R(G)\}. \]
\end{definition}

\begin{lemma}
	Let $G$ be a connected algebraic group.
	
	$R(G)$ is the largest connected solvable normal algebraic subgroup of $G$.
\end{lemma}
\begin{proof}
	It is clear that $R(G)$ is connected, solvable, normal and algebraic.
	
	We need to show that each connected solvable normal algebraic subgroup $H$ of $G$ is contained in $R(G)$.
	
	Clearly, $H$ is contained in one Borel group $B$. Since $H$ is normal, we have for each $g \in G$
	\[ H = gHg\i \subset gBg\i. \]
	Since $gBg\i$ is a Borel group and all Borel groups are conjugated, it follows $H$ is contained in each Borel group, ergo it is contained in $R(G)$.
\end{proof}

\begin{definition}
	We call $G$ \df{semisimple} iff $R(G) = 1$.
	
	We call $G$ \df{reductive} iff $R_u(G) = 1$ (iff $R(G)$ is a torus).
\end{definition}

\begin{example}
	\begin{itemize}
		\item Let $n \geq 1$ and $G = \GL_n(k)$. $G$ is reductive, but not semisimple:
		
		$G$ has two Borel groups:
		\begin{align*}
		B = \curv{\mat{* & *\\ & *}} && B' = \curv{\mat{* & \\ * & *}}.
		\end{align*}
		Ergo, we have for the radical
		\[ R(G) \subset B \cap B' = \curv{\mat{* & \\ & *}} =: T. \]
		But, now we have
		\[ \set{t \in T}{gtg\i \in T ~ \forall g \in G} = k^\times. \]
		Ergo,
		\[ R(G) =  k^\times. \]
	\end{itemize}
\item Let $G = \SL_n(k)$. $G$ is semisimple and reductive:

As above, one can compute
\[ Z = G \cap \curv{\mat{ \lambda & & \\
& \ddots & \\
& & \lambda}}. \]
However, $Z$ is not connected. In particular
\[ R(G) = Z^o = 1. \]
\item $G = \G^n_m$ is a torus: It is easy to see that $R(G) = G$ in this case.
\item $G$ is solvable (and connected): Trivially, we have then $R(G) = G$.
\item $G$ is unipotent: In this case, we know that $G$ is solvable. Further, we even have $R_u(G) = G$.
\item If $G$ is $\SO_n$ or $\SP_{2n}$, then $R(G) = R_u(G) = 1$.
\end{example}