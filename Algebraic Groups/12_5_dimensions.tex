\subsection{Dimensions}
\begin{definition}
	Let $X$ be a quasi-projective variety. We define its \df{dimension} as the transcendency degree of its function field, i.e.
	\[ \dim(X) := \text{tr.-deg}_k(k(X)). \]
\end{definition}
\begin{remark}
	If $X$ is affine, then
	\begin{align*}
	\dim (X) &= \text{tr.-deg}_k(k(X))\\
	&=\dim_{\mathrm{Krull}}(\O(X))\\
	&= \sup\set{n \in \N_0}{P_0\subsetneq P_1\subsetneq \ldots P_n, P_i \text{ prime in }\O(X) }\\
	&= \sup\set{n \in \N_0}{Z_n\subsetneq \ldots Z_0, Z_i \text{ closed, irreducible in }X }
	\end{align*}
\end{remark}
\begin{remark}
	If $U \subset X$ is open, then $k(U) = k(X)$ and $\dim(U) = \dim (X)$.
\end{remark}
\begin{lemma}
	Let $\phi : X \pfeil{} Y$ be a surjective morphism of quasi-projective varieties. Then,
	\[ \dim X \geq \dim Y. \]
\end{lemma}
\begin{proof}
	$\phi$ induces an inclusion
	\begin{align*}
	 \phi^* : k(Y) &\Inj{} k(X)\\
	 [(U_i, \alpha_i)_i] & \longmapsto [(\phi\i(U_i), \alpha_i\circ \phi)].
	\end{align*}
This map is indeed injective, since $\phi$ is surjective. Therefore, the claim follows.
\end{proof}
\begin{lemma}
	Let $X$ be a quasi-projective variety and $Y$ a proper, closed subvariety. Then,
	\[ \dim(Y) < \dim(X). \]
\end{lemma}
\begin{proof}
By going from $X$ to its closure $\widetilde{X}$ and from there to $\widetilde{X}^{(i)}$, we can assume that $X$ is affine.

Then, $I_X(Y)$ is a non-trivial prime ideal in $\O(X)$. Therefore, we have
\[ \dim_{\mathrm{Krull}}(\O(Y)) = \dim_{\mathrm{Krull}}(\O(X) / I_X(Y)) \leq \dim_{\mathrm{Krull}}(\O(X) ), \]
since $A$ is a finitely generated $k$-algebra and a domain.
\end{proof}

\begin{lemma}
	Let $X$ be an affine variety and $f\in \O(X)$ be non-zero.
	
	Then, the set
	\[ V_X(f) := \set{p \in X}{f(p) = 0} \]
	is a proper, closed subset of $X$ and we can decompose it into irreducible components
	\[ V_X(f) = Z_1\cup \ldots \cup Z_l. \]
	For each of those $Z_i$, we have
	\[ \dim(Z_i) = \dim(X) - 1. \]
\end{lemma}
\begin{proof}
	The $Z_i$ correspond to minimal prime ideals $P_i$ in $\O(X)$ which contain $(f)$. Since, they are minimal, we have
	\[ \mathrm{height}(P_i) = 1. \]
\end{proof}

\begin{lemma}
	Let $X$ be an quasi-projective algebraic set. Then -- as in the affine case -- we may write
	\[ X = Z_1\cup\ldots \cup Z_l \]
	where each $Z_i \subseteq X$ is an \df{irreducible component}, i.e. a maximal closed irreducible subset.
	
	We then define
	\[ \dim(X) := \max_i \dim(Z_j). \]
\end{lemma}
\begin{lemma}
	Let $\phi :X \pfeil Y$ be a morphism of quasi-projective varieties. Further, let $\phi$ be \df{dominant}, i.e., $\Img \phi$ is dense.
	
	Then, for all $p \in \Img(\phi)$, we have the following for the fiber of $\phi$ along $p$:
	\[ \dim(\phi\i(p)) \geq \dim(X) - \dim(Y). \]
\end{lemma}