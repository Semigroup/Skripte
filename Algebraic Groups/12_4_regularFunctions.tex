\subsection{Regular Functions and Morphisms}
\begin{definition}
	Let $X$ be quasi-projective variety. Let $U \subseteq X$ be open. Then, we define the \df{ring of regular functions} on $U$ by
	\[ \O(U) := \bigcap_{P \in U}\O_{X,P} \subseteq k(X). \]
\end{definition}

\begin{definition}
	Let $X,Y$ be two quasi-projective varieties. A map $f:X\pfeil{} Y$ is called a \df{morphism}, if $f$ is continuous and we have
	\[ f^*\O(U) := \set{h \circ f}{h \in \O(U)} \subseteq \O(f\i(U)). \]
\end{definition}
\begin{remark}
	Let $X,Y$ be affine varieties and $f : X \pfeil{} Y$ be a map. Then we have
	\[ f^*\O(U) \subseteq \O(f\i(U)) \]
	iff $f$ is given by polynomials.
\end{remark}

\begin{lemma}
	Let $X$ be a quasi-projective variety and let $p \in X$.
	
	Then there is an open neighborhood $U$ of $p$ in $X$ s.t. $U$ is isomorphic (as quasi-projetive varieties) to an affine variety $U' \subset k^n$.
\end{lemma}
\begin{proof}
	Let $Y$ be a projective variety s.t. $X$ lies open in $Y$. By replacing $X \inj{} Y$ with $X^{(i)}\inj{} Y^{(i)}$, we may assume that $X$ is an open subset of an affine variety $Y$ in $k^n$.
	
	Since the sets $D(f)$ give an open basis of $k^n$, there is a $f \in O(Y)$ s.t.
	\[ p \in D_Y(f) := \set{y \in Y}{f(y) \neq 0} \subset X. \]
	Now, $D_Y(f)$ is affine, because the map
	\begin{align*}
	D_Y(f) & \Pfeil{} \set{(q,r) \in k^{n+1}}{q \in Y, f(q) r = 1}\\
	q &\longmapsto (q, \frac{1}{f(q)})
	\end{align*}
is an isomorphism of quasi-projective varieties.
\end{proof}