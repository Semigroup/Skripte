\subsection{Borel's Fixed Point Theorem (General Case)}

\begin{theorem}
	Let $G$ be a connected solvable algebraic group which acts on a projective variety $X$.
	
	Then, there exists a $G$-fixed point in $X$.
\end{theorem}
\begin{proof}
	Since orbits of minimal dimensions are closed, we can replace $X$ by a $G$-orbit. That is, we can assume that $G$ acts transitively on $X$.
	
	For $p \in X$, the $G$-stabilizer set
	\[ \mathrm{Stab}_G(p) = \set{g \in G}{g.p = p} \]
	is a closed subgroup in $G$, since it is the preimage of $p$ under the continuous map $g\mapsto g.p$.
	
	We showed earlier, that there exist a finite-dimensional representation $\rho : G \pfeil{} \GL(V)$ with a one-dimensional subspace $L \subset V$ s.t.
	\[ G_p = \set{ g \in G}{gL = L}. \]
	
	Let $q = [L] \in \P(V)$. Then $G$ operates on $\P V$ and
	\[ G_q := \mathrm{Stab}_G(q) = \set{g \in G}{g.q = q} = G_p.\]
	
	Now, define
	\begin{align*}
	Y &:= G.q \subset \P V\\
	Z&:= G.(p,q) \subset X \times \P V.
	\end{align*}
	$Y$ and $Z$ are quasi-projective varieties, since $G$ is connected. We then have a $G$-equivariant diagram of quasi-projective varieties:
	\[ X \longleftarrow Z \Pfeil{\pi} Y \]
	via
	\[ X \longleftarrow X \times \P(V) \Pfeil{\pi} \P(V). \]
	Since $X$ is projective, $\pi$ is closed. Since $G_p = G_q$, the maps are bijective.
	Since all maps are bijective an $G$-equivariant, we need only to show that $Y$ has a fixed point.
	
	Since $\pi$ is closed, the existence of a $G$-fixed point $Y$ follows by the closedness of $Z$, because of Borel's special fixed point theorem.
	
	The closedness of $Z$ in $X \times \P(V)$ follows, if we can show that $Z$ is an orbit of minimal dimension in $X\times \P(V)$.
	Indeed, we have.
	\begin{itemize}
		\item If $O \subset X \times \P(V)$ is a $G$-orbit, the projection $O \pfeil{} X$  is $G$-equivariant and surjective, because $X$ is a $G$-orbit. Since $X,Y$ are quasi-procetive varieties, it then follows
		\[ \dim(X) \leq \dim(O). \]
		\item The map $Z \pfeil{} X$ is bijective, hence
		\[ \dim(*) \geq \dim(Z) - \dim (X). \]
		Ergo
		\[ \dim(Z) \leq \dim(X). \]
	\end{itemize}
\end{proof}