\marginpar{Lecture from 25.03.2020}
\paragraph{Recall:} We have seen that for diagonalizable algebraic groups $G,H$
\[ \Hom{}{G}{H} \isom{} \Hom{}{\Xi(H)}{\Xi(G)}. \]
If $G$ is diagonalizable, then
\[ \O(G) \isom{} k[\Xi(G)]. \]

\begin{theorem}
The functor
\begin{align*}
G & \Pfeil{} \Xi(G)\\
f &\longmapsto f^*
\end{align*}
defines an equivalence of categories:
\[ \{\text{diagonalizable alg. groups}\} \isom{} \{ \text{finite-dim. abelian groups with no }\mathrm{char}(k)\text{-torsion} \}. \]
\end{theorem}

This amounts to the bijection above between Hom-spaces and the following lemma.
\begin{lemma}
	\begin{enumerate}[(i)]
		\item Let $G$ be a diagonalizable alg. group. Then, $\Xi(G)$ is a finitely generated abelian group with no $\mathrm{char}(k)$-torsion.
		\item Let $\Gamma$ be a finitely generated abelian group with no $\mathrm{char}(k)$-torsion. Then, there is a diagonalizable algebraic group $G$ s.t. $\Xi(G) \isom{} \Gamma.$
	\end{enumerate}
\end{lemma}
\newcommand{\chr}{\mathrm{char}}
\begin{proof}
We will use the following facts:
\begin{itemize}
	\item Let $n \in \N$. Then, 
	$t^n - 1$ is square-free in $k[t]$ iff the ideal $(t^n - 1)$ is radical in $k[t]$ iff $t^n -1$ has not repetitive root iff either $\chr(k) = 0$ or $\chr(k) = p > 0$ and $p\not | n$.
	
	(Proof: Galois Theory, seperable/inseperable extensions.)
	
	\item Let $M:= \Z/n\Z$. Then, the $k$-group-algebra generated by $M$ 
	\[k[M] \isom{} k[t] / (t^n - 1)\]
	is reduced iff either $\chr(k) = 0$ or $\chr(k) = p > 0, p\not| n$.
	
	\item If $M_1, M_2$ are abelian groups, then we have the following isomorphism of Hopf algebras
	\begin{align*}
	k[M_1] \otimes_k k[M_2] &\Pfeil{\isom{}} k[M_1 \oplus M_2]\\
	m_1 \otimes m_2 & \longmapsto m_1m_2
	\end{align*}
	where $M_1 \oplus M_2 \isom{} M_1 \times M_2$.
\end{itemize}
\begin{enumerate}[(i)]
	\item Embed $G \inj{} T := \G^n_m$ for some $n$. Then, we have a surjection $\Z^n\isom{} \Xi(T)  \surj{} \Xi(G)$. Ergo, $\Xi(G)$ is finitely generated.
	
	Suppose $\chr(k) = p > 0$. Let $\chi \in \Xi(G)$ with $\chi^p = 1$. Then, for all $g \in G$, $\chi^p(g) = \chi(g^p) = 1$. The unit group $k^\times$ has not $p$-torsion, therefore $G \inj{} T = (k^\times)^n$ has also no $p$-torsion. Therefore, the frobenius $g \mapsto g^p$ is an isomorphism on $G$. Therefore, $\chi = 1$ is a trivial character. Ergo $\Xi(G)$ has no $p$-torsion.
	
	\item Let $\Gamma$ be a finitely generated abelian group with no $\mathrm{char}(k)$-torsion. Then,
	\[\Gamma \isom{} \Z^r \oplus \Z/n_1\Z \oplus \ldots \oplus \Z/n_l\Z\]
	where $\chr(k) \not| n_1 , \ldots , n_l$. We may reduce to the cases:
	\begin{enumerate}[(a)]
		\item $\Gamma = \Z$: take $G = \G_m$, then $\Xi(G) \isom{} \Z \isom{} \Gamma$.
		\item $\Gamma = \Z/n\Z$ with $\chr(k) =: p \not | n$:\\
		take $G:= \mu_n := \set{y \in k^\times}{y^n = 1}$. Then, since $p\not| n$, $(t^n - 1)$ is radical.
		So, 
		\[ \O(\mu_n) \overset{Nullstellensatz}{=} k[t] /(t^n - 1) \isom{as Hopf algebras} k[\Gamma] \]
		where $t$ gets mapped to the generator of $\Gamma$.
	\end{enumerate}
 \end{enumerate}
\end{proof}
\begin{corollary}
We have the bijection
\[
\{\text{tori}\}
\isom{}
\{ \text{ finitely generated free abelian groups} (\isom{} \Z^n) \}.
\]
\end{corollary}

\begin{remark}
\[
\{\text{algebraic group schemes} / k\}
\isom{\text{not necessarily natural}}
\{ \text{ f.g. Hopf algebras} \}.
\]
by
\[ G \mapsto \O(G) \]
and
\[
\{\text{diagonalizable algebraic group schemes}/k\}
\isom{}
\{ \text{ f.g. abelian groups} \}.
	\]
by
\[ G \mapsto \Xi(G). \]
Where $\mu_p$ in the left hand term gets mapped to $\O(\mu_p) = k[t] / (t^p - 1)$ with $p = \chr k$.
\end{remark}


\subsection{Trigonalization}
We say a representation $r : G \pfeil{} \GL(V)$ of a group $G$ on a finite-dimensional $k$-vectorspace $V$ is \df{trigonalizable} if it admits a basis with respect to which $r(V)$ is upper-triangular:
\[ r(G) \subseteq \curv{
\mat{* & \ldots & *\\ 0 & \ddots & \vdots \\ 0 & 0 & *}
} \]

\begin{definition}
We call a subgroup $G \subseteq{} \GL(V)$ \df{trigonalizable}, if the identity representation is.
\end{definition}
\begin{lemma}
	Let $G$ be an algebraic group. The following are equivalent:
	\begin{enumerate}[(i)]
		\item Every finite-dimensional representation $r : G \pfeil{} \GL(V)$ is trigonalizable.
		\item Every irreducible representation of $G$ is 1-dimensional.
		\item $G$ is isomorphic to an algebraic subgroup of
		\[ B_n := \curv{
	\mat{* & \ldots & *\\
	0 & \ddots & \vdots\\ 0 & 0& *}	
	} \subseteq \GL_n(k). \]
\item There is a normal unipotent algebraic subgroup $U$ of $G$ s.t. $G/U$ is diagonalizable.
	\end{enumerate}
\end{lemma}
\begin{proof}
We prove as follows:
\begin{enumerate}
	\item[(i) $\implies$ (ii):] Let $V$ be an irreducible representation. Then, $V \neq 0$. Choose a basis $e_1, \ldots, e_n$ of $V$ s.t.
	\[ r(G) \subseteq B_n. \]
	Then, $r(G)e_1 \subseteq ke_1$, so $V_0 := k e_1$ is $G$-invariant. Ergo $V = V_0$ is 1-dimensional.
	\item[(ii) $\implies$ (i):] Let $V$ be a f.d. representation. We show by induction on $\dim(V)$ that $r : G \pfeil{} \GL(V)$ is trigonalizable:
	
	In the cases $\dim(V) = 0,1$, there is nothing to show.
	
	In the case $\dim(V) \geq 2$, assume that $V$ is not irreducible. Then, there is a $G$-invariant $V_0 $ with $0 \neq V_0\neq V$.
	
	By the induction hypothesis, $V_0$ and $V/V_0$ are trigonalizable. Ergo, $V$ is trigonalizable. 
	
	(\textbf{Recall:} we used this criterion above in the proof that unipotent groups are trigonalizable by showing that every ??? of each $G$ is trivial.)
	\item[(i) $\implies$ (iii):] Choose a faithful representation $V$ of $G$. Then, $G \isom{} r(G)$. Since $r$ is trigonalizable, there is a basis of $V$ s.t.
	\[r(G) \subseteq B_n \subseteq \GL_n(k).\]
	\item[(iii) $\implies$ (ii):] Suppose $G \subseteq B_n \subseteq \GL_n(k)$. Set
	\begin{align*}
	A_n &:= \curv{\mat{* & 0 & 0\\ 0 & \ddots & 0\\ 0 &0 & *}} \subseteq \GL_n(k),\\
	U_n &:= \curv{\mat{1 & \ldots & *\\ 0 & \ddots & \vdots\\ 0 &0 & 1} }\subseteq \GL_n(k) \text{ normal algebraic subgroup of }B_n,\\
	U &:= G \cap U_n \text{ normal unipotent algebraic subgroup of }G.
	\end{align*}
	Let $V$ be an irreducible representation of $G$, then $V$ is not zero. Consider the subspace of $V$ fixed by $U$
	\[ V^U := \set{v \in V}{r(u)v = v \forall u \in U}. \]
	Then, we get a representation
	\[ r|_U : U \Pfeil{} \GL(V). \]
	Then, $r(U)$ is a unipotent algebraic group of $\GL(V)$. Ergo,
	\[ r(U) \subseteq \curv{\mat{1 & \ldots & *\\ 0 & \ddots & \vdots\\ 0 &0 & 1} }. \]
Ergo, $V^U \neq 0$. Since $U$ is normal in $G$, the subspace $V^U$ of $V$ is $G$-invariant: if $v \in V^U, g \in G$, then for all $u \in U$ we have
\[ r(u)r(g)v = r(g) r(g\i u g) v= r(g) v \]
since $v \in V^U$. Ergo $r(g)v \in V^U.$

Since $V$ is irreducible, $V = V^U$, i.e. $U$ acts trivially on $V$. Ergo, $r$ descends to a representation of the group $G/U$.

But $G/U \inj{} B_n / U_n \isom{} A_n$. Therefore, $G/U$ and $r(G)$ are commutative. Moreover, for all $g \in G$, $r(g) \in \GL(V)$ is semisimple:

if $g = g_sg_u$, then $g_u \in U$, because $U_n$ is the group of unipotent elements of $B_n$.

Hence, $r(g) = r(g_s) r(g_u) = r_(g_s)$ is semisimple.

It follows that $r(G)$ is commutative and consists of semisimple elements. By some HW: $r(G)$ is trigonalizable. It is easy to show now that $V$ is one-dimensional. (Since $V$ is irreducible and $ke_1$ is $G$-invariant.)
\end{enumerate}
\end{proof}
\begin{definition}
	$G$ is \df{trigonalizable}, if it satisfies one of the above equivalent conditions.
\end{definition}
	Later, we will see, that if $G$ is connected, then being trigonalizable implies being solvable.
	
	
\section{Commutative Groups}
Let $G$ be an algebraic group. Denote by $G_s$ resp. $G_u$ the subsets of semisimple resp. unipotent elements of $G$.

Then, $G_u$ is always algebraical i.e. closed: if $G \inj{} \GL_n(k)$, then $G_u = \set{g}{(g-1)^n = 0}$.
$G_u$ does not need to be closed under multiplication (for example, take $G = \SL_2(k)$, $\mat{1 & 1 \\ 0 & 1}$ and $\mat{1 & 0\\ 1 & 1}$).

$G_s$ needs not to be algebraic: for example, take $G = \SL_2(k)$ and if $G_s$ were algebraic, then
\[ \set{\lambda \in k^\times}{
\mat{\lambda & 1\\ 0 & \lambda\i} \in G_s
} = \set{ \lambda}{\lambda \neq \lambda\i} \]
but the last set is not algebraic. Also, $G_s$ does not need to be a subgroup.