\marginpar{Lecture from 16.03.2020 (Corona-Madness started here...)}
\paragraph{Recap:}
\begin{enumerate}[--]
	\item Basics: definitions, Hopf-algebras, ...	
	\item	Jordan decomposition
	\item	Primer on non-commutative algebra
	\begin{itemize}
		\item 	Jacobson density theorem
	\end{itemize}
	\item Unipotent groups
	\item Tori
\end{enumerate}

\subsection{Jacobson Density Theorem}
We had last week
\[ \End_D(M) := \set{\phi \in \End(M)}{\phi \circ d = d \circ \phi \forall d\in D}. \]

Let $k$ be an algebraically cloesd field, $V$ a non-trivial finite-dimensional $k$-vector space and let $G$ be a subgroup of $\GL(V)$ that acts \df{irreducibly} on $V$, i.e.,
$V$ is\df{ $G$-irreducible}, i.e., the only $G$-invariant subspaces of $V$ are ${0}$ and $V$.

Set
\[ D:= \set{d \in \End_k(V)}{ dg = gd \forall g \in G } = \Span(G) = \set{\sum_{i = 1}^nc_i g_i}{c_i \in k, g_i \in G, n \in \N_0}.\]
Then,
\[D = \End_R(V) \]
where $R$ is the $k$-subalgebra of $\End(V)$ that is generated by $G$.

\begin{lemma}[Schur's Lemma]
We understand $k \inj \End(V)$ as the inclusion of operations which operate by scalar multiplication
\[ k \Pfeil{\isom{}} \set{\phi : V \pfeil{} V}{\phi : v\mapsto t \cdot v ~~ \text{for some } t \in k}. \]
Then, we have
\[ D \isom{} k.\]
\end{lemma}
\begin{proof}
Let $d \in D$. Since $V \neq 0$, there is an eigenspace $V_\lambda\neq 0$ for $d$. Observe that $V_\lambda$ has to be $G$-invariant:\\
if $g \in G$ and $v \in V_\lambda$, then $gv \in V_\lambda$, since
\[ dgv = gdv = g(\lambda v) = \lambda gv. \]

Since $V_\lambda$ is a non-trivial $G$-invariant subspace and $V$ is irreducible under $G$, we have
\[ V_\lambda = V.\]
Ergo $d = \lambda$ in the sense of $k \inj{} \End(V).$
\end{proof}

\paragraph{Consequence of the Jacobson Density Theorem:} $R = \End_k(V)$, i.e., $G$ generates all linear operations on $V$, if $V$ is $G$-irreducible.

We will prove this after a lemma.
\begin{lemma}
	Let $n \in \N$. Set
	\[ V^n := V \oplus V \oplus \ldots \oplus V = V_1 \oplus \ldots \oplus V_n \]
	where each $V_i = V$.
	
	Let $v = (v_1,\ldots, v_n) \in V^n$ and set
	\[ Rv := \set{(rv_1, \ldots, rv_n)}{ r \in R } = \Span \{ (gv_1, \ldots, gv_n) ~|~ g \in G \}. \]
	
	Then, $Rv\neq V^n$ iff the $v_j$ are linearly dependent over $k$.
\end{lemma}
\paragraph{Consequence: } Take $n := \dim(V)$. Let $\{e_1, \ldots, e_n\}$ be a basisi of $V$ and set
\[ e := (e_1, \ldots, e_n) \in V^n. \]
Since the $(e_i)_i$ are linearly independent, the lemma states that $Re = V^n$.

Now, let $x \in \End_k(V)$. Choose $r \in R$ s.t.
\[re = (xe_1, \ldots, xe_n). \]
Then $re_i = xe_i$ for all $i$, thus $x = r$. Hence, $R = \End_k(V)$.
\begin{proof}
Choose $J \in \{1, \ldots, n\}$ as large as possible with
\[ Rv + V_1 + V_2 + \ldots + V_{J-1} =: U \neq V^n \].
Such an $J$ does exist, since we know that $Rv \neq V^n$.

Then, $V_J \not\subseteq U$, otherwise we may increase $J$. Also, $U$ is invariant by the diagonal action of $G$ on $V^n$.
Thus, $V_J \cap U \subseteq V_J$ is a proper $G$-invariant subspace of the $G$-irreducible $V_J \isom{} V$. Therefore, $V_J \cap U = 0$.

On the other hand, by maximality of $J$, we have
\[ U \oplus V_J = V^n. \]
Ergo, the map (composition)
\[ V\isom{} V_J \inj{} V^n \surj{} V^n/ U  \]
is a $G$-equivariant isomorphism, since $U$ is $G$-invariant.

Let $z : V^n / U \pfeil{\isom{}} V$ be the inverse isomorphism. Let $l$ be the $G$-equivariant map given by
\begin{center}
	\begin{tikzcd}
	V^n \arrow[r, "l"] \arrow[d] & V\\
	V^n / U \arrow[ur, "z" ] &
	\end{tikzcd}
\end{center}
and let $l_j$ be the $G$-equivariant maps by restricting $l$ on $V_j$. Then $l_j \in D \isom{} k$.

Say $l_j = t_j \in k$. Then, 
\[ l(w) = t_1 w_1 + \ldots t_nw_n. \]
Since $z$ is an isomorphism, $l$ is nonzero and $(t_1, \ldots, t_n) \neq (0,\ldots, 0)$.

Since $l|_U = 0$, we can deduce for all $u \in U$
\[ t_1u_1+ \ldots + t_nu_n = 0. \]
But $v \in Rv \subseteq U$, so we may conclude -- as required -- that the $(v_i)_i$ are linearly dependent ($l(v) = 0$).
\end{proof}

\newpage
\subsection{Unipotent Groups}
Let $G$ be a subgroup of $\GL(V)$ where $V$ is a finite-dimensional vector space and $k$ an algebrically closed field.
\begin{definition}
We say that $G$ is \df{unipotent} if one of the following equivalent conditions hold:
\begin{itemize}
	\item each $g \in G$ is unipotent (i.e. $(g-1)^n = 0$ for some $n \in \N$).
	\item all eigenvalues of $g$ are 1.
	\item $g$ is conjugate to $\mat{1 & \star & \star\\ 0 & \ddots & \star \\ 0 & 0 & 1}$.
\end{itemize}
\end{definition}
\begin{theorem}
	Any unipotent subgroup of $\GL_n(k)$ is conjugate to a subgroup of
	\[ U_n := \left\lbrace
	\mat{1 & \star & \star\\ 0 & \ddots & \star \\ 0 & 0 & 1}
	 \right\rbrace = \set{ U\in M_n(k) }{ U_{i,j} = \begin{aligned}
		0, && \text{if } i > j\\
		1, &&\text{if } i = j\\
		\text{arbitrary}, && \text{otherwise}.
		\end{aligned} }. \]
\end{theorem}
\begin{definition}
	For two subgroups $G,H$ of some common supergroup, define their \df{commutator} by
	\[ [G,H] := \shrp{ ghg\i h\i ~|~ g \in G, h \in H }. \]
	A group $G$ is called \df{nilpotent}, if one of its commutators is trivial, i.e. if we set
	\[ G_0 := G \text{  and  } G_{i+1} := [G_i, G], \]
	then $G$ is called nilpotent iff there is an $j\in \N$ with $G_j = 1$.
\end{definition}
\begin{corollary}
Any unipotent subgroup of $\GL(V)$ is nilpotent.
\end{corollary}
\begin{definition}
A group $G$ is called \df{solvable}, if $G^{(n)} = 1$ for some $n$ where
\begin{align*}
G^{(0)} &:= G,\\
G^{(i+1)} &:= [G^{(i)}, G^{(i)}].
\end{align*}
\end{definition}
\begin{notation}
In the following, we will write $G' := [G,G]$.
\end{notation}
\begin{definition}
Let $n := \dim (V)$. A \df{complete flag} is a maximal strictly increasing chain of subspaces
\[ 0 = V_0 \subsetneq V_1 \subsetneq \ldots \subsetneq V_n = V. \]
\end{definition}
Any complete flag is of the form
\[ V_j := \Span \{ e_1, \ldots, e_j\} \]
for some basis $e_1, \ldots, e_n$ of V.

Let $B$ be the basis of some flag $0 = V_0 \subsetneq V_1 \subsetneq \ldots \subsetneq V_n = V$.
For $x \in \End(V)$, we have that $x$ is upper-triangle with respect to $B$ iff $x$ leaves each member $V_i$ of the flag invariant, i.e. $xV_i \subseteq V_i.$

\begin{proposition}[Key Proposition]
	Let $G$ be a unipotent subgroup of $\GL(V)$. Then there is a complete flag $ V_0 \subsetneq V_1 \subsetneq \ldots \subsetneq V_n$ consisting of $G$-invariant subspaces, i.e., each $V_i$ is $G$-invariant.
\end{proposition}
\begin{proof}
Recall, that $G$ is a unipotent subgroup of $\GL_n(V)$. We will give an induction on $n = \dim V$.

If $n = 0$, there is nothing to show.

Let $n \geq 1$. We may assume that $V$ is $G$-irreducible. Because, if not, there is a $G$-invariant subspace $0\neq W \subset V$ s.t. $W$ and $V/W$ have dimension $< n$. Then there exist complete $G$-invariant flags in $W$ and $V/W$ and the claim -- that there is a complete $G$-invariant flag in $V$ -- follows by the induction hypothesis.

By Jacobson Density Theorem, we have
\[ R := \Span(G) = \End(V) := \End_k(V). \]
Since $G$ is unipotent, we have for each $g \in G$
\[ \trace(g) = n.\]
Ergo, for $g,h \in G$
\[ \trace(gh) = \trace(h) \]
and
\[ \trace((g-1)h) = \trace(gh) - \trace(h) = 0. \]
Since $\Span(G) = \End(V)$, it now in particularly follows for all $g \in G, \phi \in \End(V)$
\[ \trace((g-1) \phi) = 0. \]
Since the above holds for all $\phi \in \End(V)$, it must hold
\[ g-1 = 0 \]
for all $g \in G$ (take for example the elementary matrices $\phi = E_{i,j}$). Ergo, $G$ is trivial. Then, any complete flag is trivially $G$-invariant.
\end{proof}
\begin{remark}
	This gives the group analogue of Engel's Theorem.
\end{remark}
\begin{proof}[Proof Goal Theorem]
	Let $B$ be a basis of $V$ s.t. $G$ leaves each subspace in the corresponding flag invariant. Then, $G$ is upper-triangle with respect to this basis.
	
	On the other hand, each $g \in G$ us unipotent, hence its diagonal (i.e. eigenvalues) are all $1$. Thus, with respect to $B$
	\[ G \subseteq \left\lbrace
	\mat{1 & \star & \star\\ 0 & \ddots & \star \\ 0 & 0 & 1}\right\rbrace = U_n.\]
\end{proof}
\begin{remark}
	Tori are of the form $(k^\times)^n$. In the case $k = \C$, $(\C^\times)^n$ are the complexification of $U(1)^n$. This equals tori in top. sense.
	\[
	\mat{1 & \Z \\ 0 & 1} \subseteq \GL_2(\C)
	 \]
	 is a non-algebraic unipotent group.
\end{remark}
\paragraph{Exercise.} (to be discussed next time)

it would have sufficed to prove the Goal theorem in the special case that $G$ is algebraic.


\paragraph{Corollary of Proof:} If $G \subset \GL(V)$ (with $V \neq 0$) is unipotent and acts irreducibly (?), then $G = 1$, $\dim V = 1$.
