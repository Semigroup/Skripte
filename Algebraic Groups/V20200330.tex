\marginpar{Lecture from 30.03.2020}
We have the a surjective map of sets
\begin{align*}
G_s \times G_u & \Pfeil{} G\\
(g_1, g_2) & \longmapsto g_1g_2.
\end{align*}
\begin{example}[Non-Example]
Take generic $g \in G_s, h \in G_u$ for $G=\SL_2(k)$. Then, $g,h$ do not commute and we have
\[ ((gh)_s, (gh_u)) \neq (g, h) \]
because Jordan components don't commute.
\end{example}
\begin{theorem}
Let $G$ be a commutative algebraic group. Then:
\begin{enumerate}[(i)]
	\item $G_s, G_u$ are closed subgroups and the multiplicative map $G_s \times G_u \pfeil{} G$ is an isomorphism of algebraic groups.
	\item $G$ is trigonalizable. Moreover, for each finite dimensional representation $r : G \pfeil{} \GL(V)$ there is a basis s.t.
	\begin{align*}
	r(G_s) \subseteq \curv{\mat{* & 0& 0\\
	0 & \ddots & 0\\
0 & 0 & *}} && 
	r(G_u) \subseteq \curv{\mat{1 & * & *\\
		0 & \ddots & *\\
		0 & 0 & 1}}.
	\end{align*}
	\item $G_s$ is diagonalizable.
\end{enumerate}
\end{theorem}
\begin{proof}
\begin{enumerate}
	\item[(ii)] Let $V$ be any irreducible representation of $G$. We have seen that commuting semisimple operators may be simultanously diagonalizable, then
	\[ V = \bigoplus_{\chi : G_s \pfeil{} \G_m} V_\chi \]
	where
	\[ V_\chi = \set{v \in V}{r(h)v = \chi(h)v ~\forall h \in G_s}. \]
	Since $G$ is commutative, each subspace $V_\chi$ is $G$-invariant ($r(h)r(g)v = r(g)r(h)v= r(g)\chi(h)v = \chi(h) r(g)v$).
	
	Since $V$ is irreducible, we must have $V = V_\chi$ for some $\chi$.
	
	Recall that $G \isom{} G_s \times G_u$ as abstract groups. We have seen that $r(G_s) \subseteq \G_m^n$. We proved a while ago that any unipotent group, such as $G_u$, is trigonalizable. Ergo, $V$ is trigonalizable. Since $V$ is irreducible, we have $\dim V = 1$.
	
	If we apply the same argument without assuming that $V$ is irreducible, then we see that $V$ is the coproduct of $V_\chi$'s as above and that each $V_\chi$ admits a basis s.t.
	\begin{align*}
r(G_s)|_{V_\chi} \subseteq \curv{\mat{* & 0& 0\\
		0 & \ddots & 0\\
		0 & 0 & *}} && 
r(G_u)|_{V_\chi} \subseteq \curv{\mat{1 & * & *\\
		0 & \ddots & *\\
		0 & 0 & 1}}.
\end{align*}
This yields the same conclusion for $V$.
\item[(i)] We have to show that $G_s$ and $G_u$ are closed and $j : G_s \times G_u \pfeil{} G$ is an isomorphism of groups. Take any faithful representation
\[ G \Pfeil{\isom{}, r} r(G) \subseteq \GL(V) \]
and apply (ii). Then we have
	\begin{align*}
r(G) &\subseteq \curv{\mat{* & *& *\\
		0 & \ddots & *\\
		0 & 0 & *}} =: B \\
B_u &= \curv{\mat{1 & * & *\\
		0 & \ddots & *\\
		0 & 0 & 1}},\\
	r(G_s) &\subseteq \curv{\mat{* & 0& 0\\
			0 & \ddots & 0\\
			0 & 0 & *}} =: A. \\
\end{align*}
In fact, $r(G_s) = r(G) \cap A$, because if $g \in G$ with $r(g) \in A$, then $r(g)$ is semisimple, so $g \in G_s$.

Therefore, $G_s$ is closed in $G$. Ergo, $G_s$ and $G_u$ are closed subgroups.

Then, the map $j$ is a morphism of algebraic groups.

We need to show that $j\i$ is a morphism of algebraic groups. For this, it suffices to verify that the projection $G \pfeil{} G_s$ is a morphism. But this map is given under $r$ by the morphism:
\begin{align*}
t:={\mat{a_1 & *& *\\
		0 & \ddots & *\\
		0 & 0 & a_n}} \longmapsto{\mat{a_1 & 0& 0\\
		0 & \ddots & 0\\
		0 & 0 & a_n}} =: t_s.
\end{align*}
This suffices because if $g = g_s g_u$, then $g_u = g_s\i g$, so if the map $g \mapsto g_s$ is a morphism, so is $g\mapsto gg_s\i = g_u$, hence so is $g \mapsto (g_s, g_u)$.
\item[(iii)] We have seen that $G_s$ is a closed subgroup. Hence $G_s$ is a commutative algebraic group where elements are semisimple. Ergo, $G_s$ is diagonalizable.
\end{enumerate}
\end{proof}

\section{Connected Solvable Groups}

\begin{theorem}[Lie-Kolchin]
Let $G$ be a connected solvable algebraic group. Then $G$ is trigonalizable.
\end{theorem}
(By comparison, recall that we have seen so far that, if $G$ is commutative or unipotent, then $G$ is trigonalizable.)
We can reformulate this theorem as: Any connected solvable subgroup of $\GL(V)$ stabilizes some complete flag $\F = (V_0 \subsetneq \ldots \subsetneq  V_n)$.
\paragraph{Generalization (Borel's Fixed Point Theorem):} Any connected algebraic group $G$ acting on a projective variety $X$ has a fixed point in $X$.

We get a relation between complete flags and projective varieties.
\begin{proof}
	
	
Induct on the number $n$ s.t. $G^{(n)} = 1$.

For $n=0$, there is nothing to show.

If $n = 1$, $(G,G) = 1$, then $G$ is commutative, ergo trigonalizable.

Let $n \geq 2$. Then, we have $G' := (G,G) \neq 1$. We will show the following lemma:

\end{proof}
\begin{lemma}
	Let $G \subseteq \GL(V)$ be a subgroup.
	
	If $G$ is connected, then the group $G'$ with the induced topology is connected ($\iff$ the Zariski Closure of $G'$ is connected).
\end{lemma}
\begin{proof}
We have the following facts:
\begin{itemize}
	\item An increasing union of connected spaces is connected.
	\item A continuous image of a connected space is connected.
\end{itemize}
We have
\begin{align*}
G' =& \shrp{(g,h):= ghg\i h\i ~|~ g,h\in G }\\
=& \bigcup_{j \geq 0} \bigcup_{g_1,h_1, \ldots, g_j, h_j \in G} \{(g_1, h_1)\cdots (g_j, h_j)\}.
\end{align*}
Since
\[\bigcup_{g_1,h_1, \ldots, g_j, h_j \in G} \{(g_1, h_1)\cdots (g_j, h_j)\} = \Img \phi_j\]
for some continuous map $\phi_j : G^{2j} \pfeil{} G$, the claim follows.
Ergo, $G'$ is connected.
\end{proof}

\begin{remark}
	It is equivalent to show that $(*)$ any subgroup $G$ of $\GL(V)$ -- s.t. $G$ is connected and solvable -- is trigonalizable in $\GL(V)$.
	
	Indeed, the theorem implies $(*)$: the Zariski closure of $G$ is a connected algebraic group that is solvable (which extends by continuity).
	If $Zcl(G)$ is trigonalizable, then also $G$ is trigonalizable.
	
	On the other hand: $(*)$ implies the theorem, since if $G$ is given as in the theorem, apply $(*)$ to $r(G) \subseteq \GL(V)$.
\end{remark}

\begin{proof}[Proof of Theorem]
If $G^{(n)} = 1$, then $(G')^{(n-1)} = G^{(n)} = 1$. By induction, we may assume that $G'$ satisfies the following:

For all finite dimensional representations $r : G \pfeil{} \GL(V)$, $r(G')$ is trigonalizable.



Our aim is to show that any irreducible representation $V$ of $G$ has dimension $1$.

The induction hypothesis implies that $r(G')$ is trigonalizable. In particular, there exists an eigenspace $V_\chi \subseteq V$ for $G'$ for some character $\chi : G' \pfeil{} k^\times$. Since $G'$ is normal in $G$ we know that $G$ acts from the left on
\[\{ \text{eigenspaces }V_\chi \text{ in } V \text{ for } G' \}.\] 
Ergo, $\bigoplus_{\chi : G' \pfeil{} k^\times}V_\chi$ is $G$-invariant.
Since $V$ is $G$-irreducible, we have
\[ V = \bigoplus_{\chi : G' \pfeil{} k^\times}V_\chi = \bigoplus_{\chi \in \X'}V_\chi  \]
for some finite subset $\X' = \set{\chi}{V_\chi \neq 0}$ of $\Hom{}{G'}{\G_m}$, since $V$ is finite dimensional.

\paragraph{Claim:} Let $h \in G'$. Then, the map
\begin{align*}
G & \Pfeil{} \GL(V)\\
g & \longmapsto r(ghg\i)
\end{align*}
has a finite image.
\begin{proof}
	Denote by $\chi \mapsto \chi^g$ the action of $g \in G$ in $\Hom{}{G'}{\G_m}$ given by $\chi^g(h) := \chi(ghg\i)$. This is an action, since $G'$ is normal.
	
	Note, that $\X'\subseteq \Hom{}{G'}{\G_m}$ is a finite subset.
	
	Also note, that the action $\chi \mapsto \chi^g$ descends to an action $G\curvearrowright \X' $.
	
	
	Now, let $\X' = \{\chi_1,\ldots, \chi_r\}$. The matrix $r(h)$ is totally determined by the values $\chi_1(h), \ldots, \chi_r(h)$. Then, the element $r(ghg\i)$ is totally determined by the values $\chi_1^g(h), \ldots, \chi_r^g(h)$. It follows
	\[ \# \set{r(ghg\i)}{g \in G} \leq r!. \]
\end{proof}
The following lemma is easy to show:
\begin{lemma}
Let $G$ be an algebraic set. Then, $G$ is connected iff for each finite algebraic set $X$, and for each morphism $f : G \pfeil{} X$ of algebraic sets, we have that $f$ is constant.
\end{lemma}
Claim with the Lemma implies that the map $g \mapsto t(ghg\i)$ is constant. This implies that $r(ghg\i) = r(h)$ for all $g \in G, h \in G'$. Ergo, $G$ stabilizes each eigenspace $V_\chi$ for $G'$. Ergo, $V = V_{\chi_0}$, since $V$ is irreducible.
\end{proof}

\begin{lemma}
	Let $G$ be any group with a finite dimensional representation $r : G \pfeil{} \GL(V)$. Then, the subspaces $V_\chi$ for $\chi \in \Hom{}{G}{k^\times}$ are linearly independent, i.e., the map
	\begin{align*}
	\oplus V_\chi \Pfeil{} V
	\end{align*}
	is injective.
\end{lemma}
\begin{proof}
	The spaces $V_\chi$ are $G$-invariant. Suppose, there exist distinct $\chi_1, \ldots, \chi_n$ of non-zero $v_j \in V_{\chi_j}$ s.t. $\sum_j v_j = 0$.
	
	We may assume that $n$, the number of $v_j$, is minimal. W.l.o.g., $n \geq 2$.
	
	Choose $g \in G$ s.t. $\chi_1(g) \neq \chi_2(g)$. Use that $0 = g\sum_j v_j = \sum_j gv_j$ and take the linear combination as in the proof of linear indpendence of characters to contradict the minimality of $n$.
	
	($g- \chi_1(g)$ is not zero, but reduces $\sum_j v_j $ by one summand.)
\end{proof}



