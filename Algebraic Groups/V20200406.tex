\marginpar{Lecture from 06.04.2020}
\section{Projective Space}
Let $V$ be a finite-dimensional vector space. Then $\G_m = k^\times$ acts on $V$ by scalar multiplication. $\{0\}$ is a $\G_m$-invariant subspace of $V$. We are interested on the orbits of $\G_m$ on $V\setminus \{0\}$.

Define the \df{projective space} over $V$ by
\[ \P V := \G_m \backslash (V - 0) = (V- 0)/ \sim \isom{} \{ \text{lines in } V \} \]
where for $a, b \in V-0$ we set
\[ a\sim b : \Gdw{} \exists \lambda \in k^\times : \lambda a = b. \]
If $V = k^{n+1}$, we denote the $n$-dimensional projective space by $\P^n := \P V$.

Given $a = (a_0,a_1\ldots, a_{n}) \in k^{n+1} - 0$, we denote the $\sim$-class of $a$ by
\[ [a] = [a_0, \ldots a_{n}] \in \P^n. \]

Define $S$ to be the graded algebra of polynomials in $k$
\[S:= k[x_0, \ldots, x_n] = \bigoplus_{d \geq 0} S_d\]
where each $S_d$ is the space of homogenous polynomials of degree $d$, i.e.
\[ S_d = \bigoplus_{i_1, \ldots, i_d \in \{0,\ldots, n\}} k \cdot x_{i_1} \cdots x_{i_d}. \]
We identify $k$ with the space of constant polynomials $S_0 \subseteq S$.

We have
\[ S_d = \set{f \in S}{f(\lambda X) = \lambda^d f(X) ~\forall \lambda \in k^\times}. \]

Given $f \in S_d$, the set
\[ \set{a \in k^{n+1}}{f(a) = 0} \]
is $\G_m$-invariant. In other words, given $a \in \P^n$ and $f \in S^d$, it is well-defined to state $f(a) = 0$ and $f(a) \neq 0$.

\begin{definition}
	A \df{projective} algebraic subset $X \subseteq \P^n$ is a set of the form
	\[ X = V(\Sigma) := V_{\P^n}(\Sigma) \]
	where $\Sigma$ is a collection of homogenous elements of $S$, where
	\[ V_{\P^n}(\Sigma) := \set{a \in \P^n}{ f(a) = 0 ~ \forall f \in \Sigma }. \]
\end{definition}
\paragraph{Facts:}
\begin{itemize}
	\item Hilbert basis theorem states
	\[ V(\Sigma) = V(f_1, \ldots, f_m) \]
	for some finite collection $f_1,\ldots, f_m \in \Sigma$.
	\item It is useful to extend the meaning of "$f(a) = 0$" for $a \in \P^n$ to \emph{general} elements $f \in S$ by requiring that $f(a') = 0$ for each $a' \in [a]$.
	
	If we write $f = \sum_{d \geq 0} f_d$, $f_d \in S_d$, then we have
	\[ f(a) = 0 \iff f_d(a) = 0 ~ \forall d \geq 0. \]
	Therefore, we can extend the definition of $V(\Sigma)$ to any $\Sigma \subseteq S$.
	\item We have $V(\Sigma) = V((\Sigma))$ where $(\Sigma)$ is the ideal generated by some finite subset of $\Sigma$.
	\item We call an ideal $I \subseteq S$ \df{homogenous} if it is the direct sum of its $d$-homogeneous components, i.e.
	\[ I = \sum_{d \geq 0}I_d \]
	where $I_d = \set{f \in I}{f \text{ is homogenous of degree }d}$.
	
	$I$ is homogeneous iff it is generated by homogeneous elements.
	\item We have the following \emph{Nullstellensatz}:
	
	For any $X \subseteq \P^n$, set $I(X)$ to be the ideal generated by all homogeneous polynomials of $S$ vanishing on $X$. Then, we have
	\[ I(V_{\P^n}(I)) = I \]
	for each homogeneous ideal $I \subseteq S$ for which we have:
	\begin{enumerate}
		\item $I$ is radical.
		\item $I$ is not $(x_0, \ldots, x_n)$.
	\end{enumerate}
\begin{example}[Anti-example]
	The second property is necessary:
	
Set $I = (x_0, \ldots, x_n)$. Then $V_{k^{n+1}}(I) = 0$. Therefore, $V_{\P^n}(I) = \emptyset$. However,
\[ I(V_{\P^n}(I)) = S. \]
\end{example}
\item The above point induces a bijection between algebraic subsets of $\P^n$ and radical ideals $I \subset S$ which are not $(x_0,\ldots, x_n)$.
\end{itemize}

For $i = 0,\ldots, n$, set $D(x_i) := \set{a \in \P^n}{a_i \neq 0}$. $D(x_i)$ is an open set homeomorphic to $k^n$ by mapping
\[ \phi_i : a \longmapsto (\frac{a_0}{a_i}, \ldots,\frac{a_{i-1}}{a_i},\frac{a_{i+1}}{a_i},\ldots, \frac{a_{n}}{a_i} ). \]
The $D(x_i)$ cover $\P^n = \bigcup_i D(x_i)$.

Given a projective algebraic subset $X \subset \P^n$, define $X^{(i)} \subset k^n$ by
\[ X^{(i)} := \phi_i (X \cap D(x_i)) .\]
If $X = V_{\P^n}(I)$, then
\[ X^{(i)} = V_{k^n}(I^{(i)}) \]
where
\[ I^{(i)} := \set{f^{(i)}}{f \in I} \]
where $f^{(i)} (t_1,\ldots, t_n) := f(t_1,\ldots, t_{i-1}, 1, t_i, \ldots, t_n)$. Thus, $X^{(i)}$ is an algebraic subset of $k^n$.
\begin{definition}
The \df{Zariski topology} on $\P^n$ is defined by setting the set of closed sets to be the set of projective algebraic sets.
\end{definition}
\paragraph{Facts:}
\begin{itemize}
	\item $D(x_i)$ is open in $\P^n$, since $D(x_i) = \P^n - V(x_i)$.
	\item The bijections $D(x_i) \isom{} k^n$ are homeomorphims.
\end{itemize}
\begin{definition}
	A \df{quasi-projective} algebraic set $Y$ is an open subset of a projective algebraic set $X \subseteq \P^n$.
\end{definition}
\begin{example}
	Any algebraic set in $k^n$ is quasi-projective.
\end{example}
\begin{definition}
A quasi-projective variety is an irreducible quasi-projective algebraic set.
\end{definition}