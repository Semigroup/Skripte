\subsection{Semisimple Elements of Solvable Groups}
\begin{theorem}
	Let $B = U \rtimes T$ as before be a solvable connected algebraic group.
	
	Let $s \in B$ be semisimple. Then $s$ is conjugated to one element in $T$.
\end{theorem}
\begin{corollary}
	Let $G$ be a connected algebraic group. Then, every semisimple element of $G$ is contained in some torus.
\end{corollary}
\begin{proof}
	Let $s \in G$ be semisimple and choose a Borel group $B \subset G$ which contains $s$.
	
	$B$ is of the form $U \rtimes T$, ergo $s \in b\i T b$ for some $b \in B$.
\end{proof}

\begin{lemma}
	Suppose $\mathrm{char} k = 0$.
	\begin{enumerate}[(i)]
		\item Let $g \in \GL_n(k)$ be unipotent and set $G(g) := \overline{\shrp{g}}$. Then, we have the following isomorphism of algebraic groups
		\[ G(g) = \set{g^t}{t \in k} \isom{} k \]
		where
		\begin{align*}
		g^t &:= \exp( t \cdot \log(g))\\
		-\log(1 - X) &= \sum_{n = 1}^\infty \frac{X^n}{n}\\
		\exp(Y) &= \sum_{k = 0}^\infty \frac{Y^k}{k!}.
		\end{align*}
		
		\item Any unipotent algebraic group is connected. (This does not hold if $\mathrm{char} k > 0$.)
		\item Any unipotent commutative algebraic group is isomorphic to some vector space.
	\end{enumerate}
\end{lemma}
\begin{proof}
		\begin{enumerate}[(i)]
		\item We will not prove this, but the idea is that $\Z$ is dense in $k$.
		\item Let $g,h \in G$ be unipotent. Then the subgroups $G(g), G(h)$ are connected and share a common point ($e$), ergo $g,h$ are contained in the same component.
		\item Since all elements commute $\log$ gives an isomorphism into an additive group, on which $k$ acts.
	\end{enumerate}
\end{proof}

\begin{proof}[Proof of Theorem]
	We only prove the theorem in case $\mathrm{char} k = 0$.
	
	Let $s \in B = U\rtimes T$ be semisimple. Since $\mathrm{char} k = 0$, $U$ is connected.
	
	We induct on $\dim(U)$:
	\begin{itemize}
		\item $\dim(U) = 0$: In this case $U = 1$ and $s \in G= T$.
		\item $\dim(U) = 1$: This is the crucial case.
		
		Write
		\[ s=ut \]
		with $u \in U$ and $t \in T$.
		
		If $u$ and $t$ commute, then $ut$ is a Jordan decomposition and we have $u = 1$, ergo $s \in T$.
		
		Assume therefore, that $u,t$ don't commute. We claim:
		
		\paragraph{Claim:} For each $h \in sU = Us$, we have for the $B$-conjugacy class $C(h) = \set{ghg\i}{g \in B}$
		\[ C(h) = sU. \]
		The claim implies the theorem, because we then have
		\[ t = su \in sU = C(s) \]
		ergo $t = gsg\i$.
		
		\paragraph{Proof of Claim:}	
		\begin{itemize}
			\item First note, that $B$ acts by conjugation on $Us = sU$. This is because $G/U$ is commutative and $U$ is normal. In fact, we have for $g \in B, u \in U$
			\[ gsug\i = s \cdot (s\i g sug\i) = s \cdot (s\i g s g\i) \cdot u'. \]
			Now, $(s\i g s g\i)$ must lie in $U$ because $G/U$ is commutative.
			\item Since $\dim(U) = 1$, we have 
			\[ U = \set{v^k}{k \in K} \isom{} k. \]
			\item $h \in sU$ does not commute with $u$, since -- otherwise -- $s,t$ would commute with $u$.
			
			Ergo, $h \neq u\i h u$, which means $C(h) \supseteq \{h, u\i h u\}$ contains at least two different elements.
			\item Note, that $C(h)$ is a $B$-orbit and therefore connected and \df{locally closed} (that is a closed subset of an open subset of $G$). Since $G/U$ is commutative, we have
			\[ C(h) \subset sU = hU \isom{} k. \]
		\end{itemize}	
		Now, the only connected, locally closed subsets of $k$ are singletons and complements of finite sets.
		
		Since $C(h)$ is not a singleton, we have
		\[ C(h) = sU  - \Sigma \]
		for a finite set $\Sigma$.
		
		We claim that $\Sigma$ is empty. Note, that $B$ acts by conjugation on $sU$ and $C(h)$, ergo also on $\Sigma$. If we pick $h' \in \Sigma \subset sU$, then $C(h')$ must be finite, connected and contain two different elements. This is a contradiction.
		\item $\dim(U) \geq 2$:\\
		We want to reduce this case to the case $\dim (U) = 1$. We need therefore, to show a lemma:
		\begin{lemma}
			Let $B = U \rtimes T$ as above and suppose again $\mathrm{char} k = 0$.
			
			Then, there is an algebraic subgroup $V \subset U$ s.t. $V$ is normal in $B$ and
			\[ \dim(U/V) = 1. \]
		\end{lemma}
	\begin{proof}
		$U$ is nilpotent, since it is unipotent.
		Consider the chain
		\[ U = U_0 \supset U_1 \supset \ldots \supset U_n \supset 1 \]
		where
		\[ U_{i+1} := [U_i, U]. \]
		Note, that each $U_i$ must be invariant under each $\phi \in \Aut{}{U}$. Therefore, $U_i$ must be normal in $B$, since
		 $U$ is normal in $B$. In particular, $B$ acts on each $U_i$ by conjugation.
		
		Now, $U/U_1$ is unipotent and commutative, hence isomorphic to a vector space.
		
		Further, $T$ acts on $U/U_1$ by conjugation. Note, that $T$ is diagonalizable, ergo reductive. Therefore, $U/U_1$ must be completely reducible and we can decompose it
		\[ U/U_1 = \bigoplus_j V_j. \]
		Since $T$ is diagonalizable and each $V_j$ is $T$-invariant, each $V_j$ must be one-dimensional. Set
		\[ \overline{V} := \bigoplus_{j \geq 2} V_j. \]
		And now set
		\[ V := \pi\i(\overline{V}) = \set{u \in U}{uU_1 \in \overline{V}}. \]
		Then, we have
		\[ U/V = (U/U_1) / (V / U_1) = ( U / U_1) / \overline{V} \isom{} V_1 \isom{} k. \]
		$V$ is normal in $U$, since $U_1$ is normal in $U$ and $T$ acts on $V$ and $\overline{V}$ by conjugation.
	\end{proof}

Let $s \in B$ be semisimple and $\dim(U) \geq 2$. Choose $V \subset U$ s.t. $\dim(U / V) = 1$ and $V$ is normal in $B$. Set
\begin{align*}
B' &:= B/V\\
U' &:= U/V.
\end{align*}
Then, $B'$ is a connected algebraic group with
\begin{align*}
(B')_u &= U'\\
B' / U& \isom{} B/U = T\\
B' &= U' \rtimes T.
\end{align*}
Since $\dim(U') = 1$, we know that $\pi_V(s) \in B'$ is contained in a conjugacy class of $T$.
Let $s'\in B$ be the conjugate of $s \in B$ s.t. $\pi_V(s') \in T$. Then,
\[ s' \in TV. \]
But $TV$ is a connected solvable algebraic group and we have
\[ TV \isom{} V \rtimes T \subset U \times T. \]
Since $(TV)_u = V$ and $\dim(V) = \dim(U) - 1$, the induction hypothesis does also hold in $TV$. Ergo, $s'$ is conjugated to some element in $T$, as we wanted.
	\end{itemize}
\end{proof}

