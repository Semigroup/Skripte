\subsection{Quotients}
\begin{definition}
	A (left) \df{quotient} of an algebraic group $G$ by a closed group $H$ is a pair $(X, \rho)$ s.t.
	\begin{enumerate}[(1)]
		\item $X$ is a quasi-projective variety.
		\item $\rho : G \pfeil{} X$ is a morphism with
		\[ \rho(hg) = \rho(g) \]
		for all $h \in H, g \in G$.
	\end{enumerate}
Further, we demand that a quotient is \df{initial} in the category of all objects satisfying the above conditions. I.e. for each pair $(X', \rho')$ there must be a unique morphism $\phi$ s.t. the following diagram commutes:
\begin{center}
\begin{tikzcd}
G \arrow[d, "\rho"] \arrow[rd, "\rho'"] & \\
X \arrow[r, dashed, "\phi"] & X'
\end{tikzcd}	
\end{center}
\end{definition}
\begin{remark}
	Set theoretically, we just have $X = G/H$.
\end{remark}
\begin{lemma}
	Let $(X,\rho)$ satisfy conditions (1) and (2) from the above definition. Suppose further
	\begin{enumerate}[(i)]
		\item $\{ \text{fibers of } \rho \} = \{ \text{left }H\text{-cosets of }G \}$,
		\item $X$ is a $G$-homogenous space and $\rho$ is $G$-equivariant,
		\item for each open $U \subset X$ the pullback map
		\begin{align*}
		\rho^* : \O(U) & \Pfeil{} \O(\rho\i (U))\\
		f &\longmapsto f \circ \rho
		\end{align*}
		defines an isomorphism
		\[ \O(U) \isom{} \set{f \in \O(\rho\i U)}{f(Hg) = f(g)} =: \O(\rho\i(U))^H. \]
	\end{enumerate}
Then, $(X,\rho)$ is a quotient of $G$ by $H$.
\end{lemma}
\begin{proof}
	We have to show that $(X,\rho)$ is initial. Let $(X', \rho')$ be another object satisfying (1), (2).
	Because of (i), we have a unique settheoretic map $\phi : X \pfeil{} X'$ s.t. the diagram
	\begin{center}
		\begin{tikzcd}
		G \arrow[d, "\rho"] \arrow[rd, "\rho'"] & \\
		X \arrow[r, dashed, "\phi"] & X'
		\end{tikzcd}	
	\end{center}
commutes. We need to check that $\phi$ is a morphism:
\begin{itemize}
	\item $\phi$ is continuous, since $\rho'$ is continuous and $\rho$ is open (since $X$ is a $G$-homogenous space). Therefore, $\phi = \rho' \circ \rho\i$ is continuous.
	\item Let $U' \subset X'$ be open. We need to show
	\[ \phi^* \O(U') \subseteq \O(\phi\i U'). \]
	Let $f \in \O(U')$ and set $U := \phi\i U'$. Since $\rho'$ is a morphism, we have
	\[ {\rho'}^*(f) \in \O({\rho'}\i U'). \]
	Because of (iii), we have
	\[ \O(U) \isom{} \O(\rho\i U)^H. \]
	Therefore, it suffices to show
	\[ {\rho'}^*(f) \in \O(\rho\i U)^H. \]
	And, indeed
	\[ f \circ \rho'(hg) = f \circ \rho'(g) \]
	for $g \in G, h \in H$.
\end{itemize}
\end{proof}

\begin{lemma}
	Suppose $\mathrm{char} k = 0$. Any injective morphism of quasi-projective varieties with dense image is \df{birational}, i.e., induces, via pullback an isomorphism
	\[ k(X) \isom{} k(Y). \]
\end{lemma}

\begin{theorem}
	Let $G$ be an algebraic group with a closed subgroup $H$.
	\begin{itemize}
		\item A quotient $(X, \rho)$ exists and $X$ is a homogenous space for $G$ s.t. $H = \mathrm{Stab}_G(p)$ for some $p \in X$.
		\item If $\mathrm{char}(k)= 0$, then each $G$-homogenous space $X$ together with a point $p\in X$ s.t. $H = \mathrm{Stab}_G(p)$ gives a quotient of $G$ by $H$, where $\rho(g) = g.p$.
	\end{itemize}
\end{theorem}
\begin{proof}
	We only prove the theorem for the case $\mathrm{char} k = 0$. We construct $X$ as in a previous proposition, i.e. $X = G.p$ for a point $p \in \P(V)$ s.t. $H = \mathrm{Stab}_G(p)$.
	
	It is then clear, that conditions (i) and (ii) of the previous lemma are met. We only need to show
	\[ \rho^*\O(U) = \O(\rho\i U)^H. \]
	Naturally, $\rho^*\O(U)$ is contained in $\O(\rho\i U)^H$.
	
	Let $f \in \O(\rho\i U)^H$. W.l.o.g., we can assume that $U$ is affine. Consider the diagram
	\begin{center}
		\begin{tikzcd}
		\rho\i U \arrow[r, "f"] \arrow[d, "\rho"] & k \\
		U \arrow[ur, dashed, "g"] & 
		\end{tikzcd}	
	\end{center}
$g := f \circ \rho\i$ is well-defined, because $f$ is $H$-invariant. We need to show, that $g$ is regular, i.e. $g \in \O(\rho\i U)$.

We can blow up the diagram as follows:
	\begin{center}
	\begin{tikzcd}
	U \times k \arrow[rd, "\pi_2"] & &  V \supset {\Img (\rho \times f)} \arrow[ld, "\pi_2"] \arrow[dd, "p"] \arrow[ll,hook', "\mathrm{open}"] \\
	& k & \\
	\rho\i U \arrow[uu, "\rho \times f"] \arrow[ru, "f"] \arrow[rr, "\rho"] & & U \arrow[ul, "g"]
	\end{tikzcd}	
\end{center}
Then $V$ is a quasi-projective variety and $p$ is dominant and injective, hence birational. Therefore, we have
\[ k(U) \isom{} k(V). \]
Since $X$ is homogenous space for $G$, it is smooth. On a smooth quasi-projective variety, every rational function that fails to be regular must have a pole.

In particular, we do have $\pi_2 \in \O(V)$ and therefore
\[ g = p^*(\pi_2) \in \O(U). \]
\end{proof}

\begin{example}[Non-Example]
	The proof of the theorem does not hold, if $\mathrm{char} (k) = p > 0$.
	
	Consider,
	\begin{align*}
	 G &:= \G_a\\
	 H &:= 1\\
	 V &= k^2\\
	\end{align*}
	and
	\begin{align*}
	G & \Pfeil{} \GL(V)\\
	x &\longmapsto \mat{1 & x^{p^n}\\  & 1}
	\end{align*}
	for some $n \in \N_0$.
	
	For $q = [1,0] \in \P(V)$, we have
	\[ X:= G.q = \set{[1,x^{p^n}]}{x \in k} \isom{} k. \]
	Define $\rho$ by by
	\begin{align*}
	\rho : G & \Pfeil{} X\\
	g & \longmapsto g.q.
	\end{align*}
	Then, $(\rho, X)$ fulfills the conditions of the above theorem, but it is NOT a quotient for $n \geq 1$.
	
	Indeed, for $n_1 \geq n_2$, we have non-isomorphic maps
	\begin{align*}
	X_{n_2} & \Pfeil{} X_{n_1}\\
	x & \longmapsto x^{p^{n_1 - n_2}}.
	\end{align*}
\end{example}