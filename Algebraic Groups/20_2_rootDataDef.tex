\subsection{Root Data -- Definition}

\begin{definition}
	A \df{root datum} is a tuple
	\[ \Psi = (X, X^\vee, R, R^\vee) \]
	where $X,X^\vee$ are finitely generated free $\Z$-modules equipped with a pairing
	\begin{align*}
	X \times X^\vee & \Pfeil{} \Z\\
	(x, \xi) &\longmapsto \skp{x}{\xi}
	\end{align*}
	which is \df{perfect}, i.e. $\skp{\cdot}{\cdot}$ induces isomorphism $X \isom{} \Hom{\Z}{X^\vee}{\Z}$ and $X^\vee \isom{} \Hom{\Z}{X}{\Z}$.
	
	$R$ and $R^\vee$ are to be finite subsets $R \subset X, R^\vee \subset X^\subset$ with a bijective map
	\begin{align*}
	R & \Pfeil{} R^\vee\\
	\alpha & \longmapsto \alpha^\vee.
	\end{align*}
	The $(X, X^\vee, R, R^\vee)$ shall meet the following axioms:
	\begin{enumerate}[(i)]
		\item For each $\alpha \in R$, we have
		\[ \skp{\alpha}{\alpha^\vee} = 2. \]
		\item For each $\alpha \in R$, define the maps
		\begin{align*}
		S_\alpha : X & \Pfeil{} X\\
		x & \longmapsto x - \skp{x}{\alpha^\vee} \alpha\\
		S_{\alpha^\vee} : X^\vee & \Pfeil{} X^\vee\\
		\xi & \longmapsto \xi - \skp{\alpha}{\xi} \alpha^\vee.
		\end{align*}
		These maps satisfy for each $\alpha \in R$
		\[ S_\alpha(R) \subseteq R \text{  and  } S_{\alpha^\vee}(R^\vee) \subset R^\vee. \]
		\item We call $\Psi$ \df{reduced}, if additionally the following is fulfilled:
		
		If $\alpha, c\alpha \in R$, for some $c \in \Q$, then $c = \pm 1$.
	\end{enumerate}
Elements $\alpha \in R$ are called \df{roots}, while corresponding elements $\alpha^\vee \in R^\vee$ are called \df{coroots}.
\end{definition}

\begin{remark}
	Let $\Psi = (X, X^\vee, R, R^\vee)$ be a root datum:
	\begin{enumerate}
		\item For $\alpha \in R$, we have
		\[ S_\alpha(\alpha) = - \alpha. \]
		\item In particular, we have for $\alpha \in R$
		\[S_\alpha^2 = \id{R}. \]
		\item If $\Psi = (X, X^\vee, R, R^\vee)$ is a root datum, then so is $\Psi^\vee = (X^\vee, X, R^\vee, R)$.
		\item If we have for $\alpha, \beta \in R$
		\[ \skp{\_}{\alpha^\vee} \equiv \skp{\_}{\beta^\vee}, \]
		then $\alpha = \beta$.
	\end{enumerate}
\end{remark}

\begin{lemma}
	In the definition of a root datum, it would suffice to demand that
	\begin{align*}
	R & \Pfeil{} R^\vee\\
	\alpha & \longmapsto \alpha^\vee
	\end{align*}
	is only surjective.
\end{lemma}
\begin{proof}
	Let $\alpha, \beta \in R$ s.t.
	\[ \alpha^\vee = \beta^\vee. \]
	If $\alpha, \beta$ are linearly dependent, they must be equal, because of
	\[ \skp{\alpha}{\alpha^\vee} =2= \skp{\beta}{\alpha^\vee}. \]
	Assume, therefore, that they are linearly independent. Let $V$ be the $\Z$-module spanned by the basis $(\alpha, \beta)$. Regarding this basis, the action of $S_\alpha$ and $S_\beta$ can be represented by the following matrices:
	\begin{align*}
	S_\alpha &\hat{=} \mat{-1 & -2 \\ 0 & 1}\\
	S_\beta &\hat{=} \mat{1 & 0 \\ -2 & -1}.
	\end{align*}
	It is now easy to show that
		\begin{align*}
	(S_\alpha \circ S_\beta)^2 &\hat{=} \mat{3 & 2 \\ -2 & -1},\\
	(S_\alpha \circ S_\beta)^n &\hat{=} \mat{2n + 1 & 2n \\ -2n & 1 - 2n}.
	\end{align*}
	But $R$ must be closed under the action of $S_\alpha$ and $S_\beta$. Since $R$ must be finite, $\alpha, \beta$ cannot be linearly independent.
\end{proof}

\begin{definition}
	The \df{Weyl group} $W(\Psi)$ of a root datum $\Psi$ is the subgroup of $\Aut(X) \isom{} \GL_n(\Z)$ which is generated by
	\[ \set{S_\alpha}{\alpha \in R}. \]
\end{definition}


Our goal is to construct for each connected reductive group $G$ with a maximal torus $T$ a root datum
\[ \Phi(G,T) = \Phi(G) \]
s.t. the Weyl groups $W(G,T)$ and $W(\Phi(G))$ are isomorphic (in a canonical way).
\begin{theorem}[Facts]
	Suppose, we were given a notion of morphism of root data at this point.
\begin{enumerate}
\item $\Phi(G) \isom{} \Phi(G')$ iff $G \isom{} G'$.
\item Every root datum is isomorphic to some $\Phi(G)$ for a reductive connected group $G$.
\end{enumerate}
\end{theorem}
\begin{remark}
	\begin{itemize}
		\item Obviously, the notion of root data is independent of $k$.
		\item Root data also classify compact connected Lie groups.
		\item Root data refine the less precise notion of root systems (which classify semisimple Lie algebras and simply connected semisimple algebraic groups).
		\item Every root system is a finite direct sum of the simple root systems
		\begin{align*}
		(A_n)_{n\geq 1}, (B_n)_{n\geq 2}, (C_n), (D_n), E_6, E_7, E_8, F_4, G_2.
		\end{align*}
	\end{itemize}
\end{remark}