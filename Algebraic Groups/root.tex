\documentclass[12pt]{book}

\usepackage[T1]{fontenc}
\usepackage[utf8]{inputenc}
\usepackage[english]{babel}

\usepackage{tikz-cd}
\usetikzlibrary{babel}

\usepackage{amsfonts}
\usepackage{amssymb}
\usepackage{amsthm}
\usepackage{amsmath}
\usepackage{mathtools}
\usepackage{wasysym}
\usepackage{geometry}
\usepackage{makeidx}
\usepackage{booktabs}
\usepackage{hyperref}

\usepackage{enumerate}
\usepackage{todonotes}

\usepackage[strings]{underscore}

\theoremstyle{plain}
\newtheorem{theorem}{Theorem}
\newtheorem{lemma}{Lemma}
\newtheorem{corollary}{Corollary}
\newtheorem{proposition}{Proposition}


\theoremstyle{remark}
\newtheorem{motivation}{Motivation}
\newtheorem{remark}{Remark}
\newtheorem{conjecture}{Conjecture}

\theoremstyle{definition}
\newtheorem{definition}{Definition}
\newtheorem{example}{Example}
\newtheorem{notation}{Notation}
\newtheorem{convention}{Convention}

\newcommand{\df}[1]{\index{#1}\textbf{#1}}

\newcommand{\curv}[1]{\left\lbrace#1\right\rbrace}
\newcommand{\abs}[1]{\left|#1\right|}
\newcommand{\klam}[1]{\left(#1\right)}
\newcommand{\norm}[1]{\abs{\abs{#1}}}
\newcommand{\shrp}[1]{\left<#1\right>}
\newcommand{\brak}[1]{\left[#1\right]}
\newcommand{\set}[2]{\curv{#1 ~ | ~ #2}}
\newcommand{\skp}[2]{\shrp{#1 ~ | ~ #2}}
\newcommand{\ceil}[1]{\left\lceil#1\right\rceil}
\newcommand{\floor}[1]{\left\lfloor#1\right\rfloor}
\newcommand{\round}[1]{\left\lceil#1\right\rfloor}
\newcommand{\Span}{\mathrm{span}}
\newcommand{\trace}{\mathrm{trace}}

\newcommand{\A}{\mathcal{A}}
\newcommand{\B}{\mathcal{B}}
\newcommand{\C}{\mathbb{C}}
\newcommand{\D}{\mathcal{D}}
\newcommand{\E}{\mathcal{E}}
\newcommand{\F}{\mathcal{F}}
\newcommand{\G}{\mathcal{G}}
\renewcommand{\H}{\mathbb{H}}
\newcommand{\I}{\mathcal{I}}
\newcommand{\J}{\mathcal{J}}
\newcommand{\K}{\mathbb{K}}
\renewcommand{\L}{\mathcal{L}}
\newcommand{\M}{\mathcal{M}}
\newcommand{\N}{\mathbb{N}}
\renewcommand{\O}{\mathcal{O}}
\renewcommand{\P}{\mathbb{P}}
\newcommand{\Q}{\mathbb{Q}}
\newcommand{\R}{\mathbb{R}}
\renewcommand{\S}{\mathcal{S}}
\newcommand{\T}{\mathcal{T}}
\newcommand{\U}{\mathcal{U}}
\newcommand{\V}{\mathcal{V}}
\newcommand{\W}{\mathcal{W}}
\newcommand{\X}{\mathcal{X}}
\newcommand{\Y}{\mathcal{Y}}
\newcommand{\Z}{\mathbb{Z}}

\newcommand{\id}[1]{\text{Id}_{#1}}
\newcommand{\Ker}{\textsf{Kern}}
\newcommand{\Coker}{\textsf{Kokern}}
\newcommand{\Img}{\textsf{Img}}
\newcommand{\Coimg}{\textsf{CoImg}}
\newcommand{\Hom}[3]{\textsf{Hom}_{#1}\left(#2, #3\right)}
\newcommand{\Aut}[2]{\textsf{Aut}_{#1}\left(#2\right)}
\newcommand{\Sym}[1]{\textsf{Symm}_{#1}}

\newcommand{\e}{\varepsilon}

\newcommand{\isom}[1]{\overset{#1}{\cong}}
\newcommand{\draw}{\leftarrow}
\newcommand{\Pfeil}[1]{\overset{#1}{\longrightarrow}}
\newcommand{\pfeil}[1]{\overset{#1}{\rightarrow}}
\newcommand{\inj}[1]{\overset{#1}{\hookrightarrow}}
\newcommand{\Inj}[1]{\overset{#1}{\lhook\joinrel\longrightarrow}}
\newcommand{\surj}[1]{\overset{#1}{\twoheadrightarrow}}

\newcommand{\impl}[1]{\overset{#1}{\Rightarrow}}
\newcommand{\Impl}[1]{\overset{#1}{\Longrightarrow}}
\newcommand{\gdw}[1]{\overset{#1}{\Leftrightarrow}}
\newcommand{\Gdw}[1]{\overset{#1}{\Longleftrightarrow}}

\newcommand{\off}{\overset{o}{\subset}}
\newcommand{\abg}{\overset{c}{\subset}}

\newcommand{\GL}{\textsf{GL}}
\newcommand{\SL}{\textsf{SL}}
\newcommand{\SO}{\textsf{SO}}
\newcommand{\PGL}{\textsf{PGL}}
\newcommand{\Ad}{\textsf{Ad}}
\newcommand{\End}{\textsf{End}}


\newcommand{\supp}{\text{supp}}

\renewcommand{\i}{^{-1}}

\newcommand{\mat}[1]{\klam{\begin{matrix}
#1
\end{matrix}}}

\setlength{\marginparwidth}{20mm}

\makeindex
\date{\today}
\author{Akin}

\makeindex

\begin{document}
\title{Mitschrieb: Algebraic Groups\\
SS 20}
\maketitle
\section*{Vorwort}

\setcounter{tocdepth}{1}
\tableofcontents

\marginpar{Lecture from 03.03.2020}

\paragraph{Recall:} Last time we introduced the \df{Zariski-Topology} on $X$.

There, algebraic sets equal closed sets.

We called a set $X$ \df{irreducible} iff each open subset lies dense in $X$.


\begin{lemma}
	For an algebraic set $X$, the following are equivalent:
	\begin{enumerate}[(1)]
		\item $X$ is irreducible.
		\item $k[X] = k[x_1, \ldots, x_n] / I(X)$ is an (integral) domain.
		\item $I(X)$ is a prime ideal.
	\end{enumerate}
\end{lemma}
The proof of $(2) \iff (3)$ is a basic algebraic result.

\begin{lemma}
	An open base for the Zariski-Topology on an algebraic set $X$ is given by sets:
	\[ D(f) := \set{p \in X}{f(p) \neq 0} \]
	for each $f \in k[X]$. We call the $D(f)$ \df{basic open sets}.
\end{lemma}
\begin{proof}
	Suppose $U \subseteq X$ is nonempty and open. Set
	\[ Z:= X \setminus U\]
	then $Z$ is closed.
	Thus
	\[ Z = \set{x \in X}{f(x) = 0 ~\forall f \in  I} =V(I) \]
	for some ideal $I \subseteq k[X]$. Let $p \in U$, then there is an $f \in Z$ s.t.
	\[f (p) \neq 0. \]
	Also, $D(f) \cap Z = 0$, thus $p \in D(f) \subseteq U$.
\end{proof}

\begin{proof}[Proof: Lemma 1]
	It is clear that (2) is equivalent to (3).
	
	(1) is equivalent to
	\begin{align*}
	& \forall \text{ nonempty, open }U_1, U_2\subset X: U_1\cap U_2 \neq \emptyset\\
\overset{\text{Lemma 2}}{\iff}	& \forall \text{ nonempty, basic open }D(f_1), D(f_2)\subset X: D(f_1) \cap D(f_2) \neq \emptyset
	\end{align*}
	Since $D(f_1)\cap D_(f_2) = D(f_1f_2)$, this is equivalent to the statement
	\[\forall f_1, f_2 \in k[X]:~ f_1, f_2 \neq 0 \implies f_1f_2 \neq 0. \]
	Which states that $k[X]$ is a domain.
\end{proof}

\begin{lemma}
	Let $X$ be an algebraic set. We have bijections
	\[
	\{ \text{closed subsets } Z \subseteq X\} \leftrightarrow \{ \text{ radical ideals } I \subset k[X]\}
	\]
	and
	\[
	\{ \text{irreducible, closed subsets } Z \subseteq X\} \leftrightarrow \{ \text{ prime ideals } I \subset k[X]\}
	\]
		and
	\[
	\{ \text{points of } X\} \leftrightarrow \{ \text{ maximum ideals } I \subset k[X]\}.
	\]
\end{lemma}
\begin{lemma}[Primary Decompositions, Atiyah, Macdonald Ch. 4]
For an ideal $I$ we call $P \supseteq I$ a \df{minimal prime} of $I$ if $P$ is a prime ideal and we have for each prime ideal $Q$:
\[ P \supseteq Q \supseteq I \implies P = Q. \]

Any radical ideal $I $ of $k[x_1, \ldots, x_n]$ has only finitely many \textbf{minimal} primes $P_1,\ldots, P_r$.
Inparticular,
\[ I = \bigcap_{i=1}^r P_i \]
and for each $i$
\[ P_j \not \supseteq \bigcap_{j : j \neq i} P_j. \]
\end{lemma}

\begin{definition}
An \df{(irreducible) component} $Z$ of $X$ is a maximal irreducible closed subset, i.e., an irreducible closed $Z \subseteq X$ s.t. there does not exist an irreducible closed $Y \subset X$ s.t. $Y \supsetneq Z$.

Then, we have the bijection
	\[
\{ \text{irreducible components of } X\} \leftrightarrow \{ \text{ mimimal primes of } I(X)\}.
\]
\end{definition}

\begin{lemma}
Any algebraic set $X$ has finitely many irreducible components $Z_1, \ldots, Z_r$. We have
\[ X = Z_1 \cup \ldots \cup Z_r \]
and for each $i$
\[ Z_i \not \subset \bigcup_{j : j \neq i}Z_j. \]
\end{lemma}

\begin{example}
	\begin{enumerate}
		\item 	Let $X = V(x\cdot y) \subset k^2$. Then $X = Z_1\cup Z_2$ where $Z_1 = V(x), Z_2 = V(y)$.
		
		$X$ is connected, but not irreducible ($D(x)$ does not lie dense in $X$).
		\item Let $X$ be a \textbf{finite} algebraic set. It is easy to check that every subset of $X$ is closed:
		\[ \{p\} = V(x_1-p_1, \ldots, x_n-p_n) \]
		for each $p \in X$. Further
		\[ X = \{p_1\}\cup \ldots \cup \{p_r\}. \]
		Moreover: Any function $f : X \pfeil{} k$ is regular (i.e. given by polynomials).
	\end{enumerate}
\end{example}

\begin{lemma}
	We call an element $e \in k[X]$ \df{idempotent} iff $e^2 = e$.
	
	Let $X$ be an algebraic set. Then
	\begin{align*}
	X \text{ connected} &\iff \text{ the only idempotents }e \in k[X] \text{ are 0 and 1}\\
	&\iff k[X] \not\isom{} A \times B \text{ for any }k\text{-algebras }A,B.
	\end{align*}
\end{lemma}
\begin{lemma}
	Morphisms of algebraic sets are continuous.
\end{lemma}
\begin{proof}
	Let $\phi : X \pfeil{} Y$ be a morphism.
	It suffices to show that for all closed $Z \subset Y$ that $\phi\i(Z) \subset X$ is closed.
	
	But, if
	\[Z = V_Y(S) := \set{q \in Y}{f(q) = 0 \forall f \in S}\]
	 for some ideal $S \subset k[Y]$, then 
	 \[\phi\i(Z) = V_X(\phi^*(S)) = \set{\phi^*(f) = f \circ \phi}{f \in S}.\]
\end{proof}
\begin{lemma}
	Isomorphisms of algebraic sets are homeorphisms. In particular, any isomorphism of algebraic sets $\phi : X \pfeil{} X$ permutes the irreducible components $Z_1, \ldots, Z_r$ of $X$:
	\[ \forall i \exists j: \phi(Z_i) = Z_j. \]
\end{lemma}

\begin{theorem}
	Let $G$ be an algebraic group.
	\begin{enumerate}[(i)]
		\item There is a unique irreducible component $G^0$ of $G$ with $e \in G^0$.
		\item Every irreducible component $Z$ of $G$ is a coset $gG^0$ of $G$ for some $g\in Z$.
		\item $G^0$ is a normal algebraic subgroup of $G$.
		\item $G^0$ is of finite index, i.e.
		\[ [G: G^0] = \# \klam{G / G^0} < \infty. \]
		\item 	The irreducible components are also the connected components.
	\end{enumerate}
\end{theorem}
\begin{proof}
Let $G = Z_1 \cup \ldots \cup Z_r$ be the decomposition into components. 
We may assume that $e \in Z_1$.

Recall that $Z_1 \not \subset \bigcup_{j \geq 2} Z_j$.
Then, there is an $x \in Z_1 \setminus \bigcup_{j \geq 2} Z_j$.
Thus, for all algebraic set isomorphisms $\phi : G \pfeil{} G$, we have by some previous lemma
that $ \phi (x)$ is likewise contained in some unique component of $G$.
For example, we may take $\phi$ to be 
\begin{align*}
\phi_g : G& \pfeil{} G\\
y &\longmapsto gy
\end{align*}
for any $g \in G$.
Then, for all $g\in G$, the element $gx = \phi_g(x)$ is contained in only one component of $G$. Ergo, each $g \in G$ is contained in exactly one component.
	\begin{enumerate}
	\item[(i)] Take $g = e$.
	\item[(iii)] $G^0$ is an algebraic subset, by construction. Denote by $m : G \times G \pfeil{} G$ and $i : G \pfeil{} G$ the continuous multiplication and inversion map on $G$.
	\textbf{Why is $G^0$ a subgroup?} We need to show
	\begin{align*}
	 m(G^0 \times G^0) &\subseteq G^0.\\
	i(G^0) &\subseteq G^0. 
	\end{align*}
	We know that $i(G^0)$ is some component of $G$, since $i$ is an isomorphism. But it contains the identity $e$, since $e\i = e$. Therefore, $i(G^0) = G^0$.
	
	If $g \in G$, then $gG^0$ is some component of $G$. Suppose $g \in G^0$. Then $gG^0 \cap G^0 \supseteq \{g\}$, therefore $gG^0 = G^0$. Ergo, $G^0$ is closed under multiplication.
	
	\textbf{Why is $G^0$ a normal?} If $g \in G$, then $gG^0g\i$ is a component that contains $e$, therefore $G^0 = gG^0g\i$.
	
	
	(Alternative proof that $m(G^0 \times G^0) = G^0$: Consider
	\begin{itemize}
		\item any continuous image of an irreducible set is irreducible.
	\item	the closure of any irreducible set is irreducible.
	\end{itemize}
	Ergo $\overline{m(G^0\times G^0)}$ is a closed irreducible set containing $e$. Ergo, $\overline{m(G^0\times G^0)} = G^0$).
	
	\item[(ii)] Let $Z \subset G$ be a component. Let $g \in Z$. Then $g \in (gG^0 \cap Z) $, so $gG^0 = Z$.
	\item[(iv)] This follows from some previous lemma.
	\item[(v)] This is left as a topological exercise. It is true whenever the irreducible components do not intersect.
	\end{enumerate}
\end{proof}

It now follows:
\[ \{ \text{finite algebraic groups}\} \longleftrightarrow \{ \text{finite groups} \} \]
where the above arrow is an equivalence of categories.

\begin{example}
	\begin{itemize}
		\item Let $G = \{g_1, \ldots, g_r\}$ be a finite algebraic group. Then,
		\[ G^0 = \{e\}.\]
		\item Without proofs:
		\[ G \in \{\GL_n(k), \SO_n(k),\SL_n(k)\} \implies G^0 = G. \]
		Further,
		\[G = O_n(k) \implies G^0 = \SO_n(k).\]
		And  if $-1 = 1$ i.e. $\textsf{char} k = 2$, then $[G:G^0] = 1$. Otherwise $[G:G^0] = 2$.
		% Correct?
%		\item For $\Ad : \GL_n(k) \pfeil{} \GL(M_n(k))$
%		\[ G = \PGL_n(k) := \Img(\Ad) \]
%		turns out to be algebraic and $G/G^0 \isom{} M_n(k)$.
	\end{itemize}
\end{example}

\newpage
\section{Jordan Decomposition}
As usual, $k = \overline{k}$ is an algebraically closed field.
\begin{definition}
	Let $V$ be a finite-dimensional vector space.
	
	An element $x \in \End(V)$ is \df{semisimple}, if it is diagonalizable, i.e. it has a basis of eigenvectors, or equivalently, if the minimal polynomial of $x$ is square-free.
\end{definition}
	
Then, there is a decomposition $V = \bigoplus_{i=1}^r V_i$ and distinct elements $\lambda_1, \ldots, \lambda_n \in k$ s.t.
\[ x|_{V_i} = \lambda_i. \]
If $\dim(V_i)= n_i$, then
\[ \text{char polynomial of }x = \prod_{i=1}^r (T_i - \lambda_i)^{n_i} \in k[T] \]
and
\[ \text{minimal polynomial of }x = \prod_{i=1}^r (T_i - \lambda_i) \in k[T]. \]
(Where the minimal polynomial of $x$ is defined as the least degree monic polynomial $m \in k[T]$ s.t. $m(x) = 0$. )


\begin{remark}
	Let $m(T) \in k[T]$ be the minimal polynomial of $x \in k^{n\times n}$.
	
	The theorem of Cayley and Hamilton states that we have for each $p \in k[T]$:
	\[ p(x) = 0 \implies m | p. \]
\end{remark}

\begin{definition}
	$x \in End(V)$ is \df{nilpotent} if $x^n = 0$ for some $n$.

	$x$ is \df{unipotent}, if $x -1$ is nilpotent.
\end{definition}

\begin{lemma}
	$x$ is nilpotent iff the characteristic polynomial of $x$ is $T^{\dim(V)}$.
	(Use Cayley-Hamilton for one of the directions).
\end{lemma}

\begin{lemma}
	If $x$ is semisimple and nilpotent, then $x = 0$.
	
	If $x$ is semisimple and unipotent, then $x = 1$.
\end{lemma}
\begin{lemma}
	If $x,y$ are commuting elements, that are semisimple resp. unipotent resp. nilpotent, then so is $xy$.
\end{lemma}
\begin{proof}
	It is easy to see, that this is true for nilpotent $x,y$.
	
	Now, let $x,y$ be unipotent and commuting. Then, we have
	\[ xy - 1 = (x+1)(y-1) + (x-y). \]
	Since $x,y$ commute, $ (x+1)(y-1)$ must be nilpotent. $(x-y)$ must be nilpotent because the sum of commuting nilpotent elements must be nilpotent.
	Because everything commutes, also $xy-1$ as the sum of two commuting, nilpotent elements must be nilpotent.
	
	Now, let $A,B \in k^{n\times n}$ be two diagonalizable and commuting matrices. Let $\lambda_1,\ldots, \lambda_r$ be different eigenvectors of $A$ and let $E_i$ be the corresponding eigenspaces. We then have
	\[ A\cdot (B E_i) = BAE_i = \lambda_i \cdot BE_i. \]
	Ergo, each $E_i$ is invariant under $B$.
	Since $B_{|E_i}$ stays diagonalizable, we can simply choose a basis of eigenvectors $b_1,\ldots, b_n \in \bigcup_i E_i$ of $B$. Since each $b_i$ lies in a $E_j$, those vectors are also eigenvectors for $A$. Therefore, $b_1,\ldots,b_n$ is basis of eigenvectors for both matrices. 
\end{proof}

\begin{theorem}[Goal]
For all algebraic groups $G$ and for all $g \in G$, there exist unique group elements $g_s, g_u \in G$ s.t.
\[g = g_s g_u = g_ug_s\]
and
for all finite-dimensional representations  $\rho : G \pfeil{} \GL(V)$, $\rho(g_s)$ is semisimple and $\rho(g_u)$ is unipotent.
\end{theorem}

\begin{example}
If	$g = \klam{\begin{matrix}
	\lambda & 1 & 0 \\
	& \lambda & 1 \\
	& & \lambda
	\end{matrix}} \in G = \GL_3(k)$, then $g_s = \klam{\begin{matrix}
	\lambda & 0 & 0 \\
	& \lambda & 0 \\
	& & \lambda
	\end{matrix}}$, $g_u = \klam{\begin{matrix}
	1 & \lambda\i & 0 \\
	& 1 & \lambda\i \\
	& & 1
	\end{matrix}}. $
\end{example}
\marginpar{Vorlesung vom 09.02.2020}

\begin{center}
	\begin{tikzcd}
	\gamma \arrow[r, hook] \arrow[rd, swap, "s_i :="] & \pi^*E = \bigoplus L_i \arrow[d, "Proj."]\\
	& L_i
	\end{tikzcd}
\end{center}
\marginpar{Lecture from 11.03.2020}
Let $G$ be an algebraic group.
\paragraph{Easy Exercise}: If $V_1,V_2$ are representations $r_1,r_2$ of $G$, then $V_1\otimes V_2$ is also a representation with
\[ r = r_1 \otimes r_2: G \pfeil{} \GL(V_1 \otimes V_2) \]
given by
\[ r(g)(v_1 \otimes v_2) = (r_1(g)v_1) \otimes (r_2(g) v_2).  \]
\begin{proof}
	Given $\Delta_j : V_j \pfeil{} V_j \otimes k[G]$, define
	\[\Delta : V_1\otimes V_2 \Pfeil{} V_1 \otimes V_2 \otimes k[G]\]
	by:
	if
	\[ \Delta_1 u = \sum_i u_i \otimes f_i, ~~~~ \Delta_2 v = \sum_j v_j \otimes h_j, \]
	then
	\[ \Delta(u\otimes v) =\sum_i \sum_j u_i \otimes v_j \otimes f_i h_j. \]
	Set $A := k[G]$, then
	\[ r_A := \text{right regular representation with } r_A(g) f(x) = f(xg). \]
	
	The map
	\begin{align*}
	A \otimes A &\Pfeil{m} A\\
	f_1 \otimes f_2 &\longmapsto f_1f_2
	\end{align*}
	defines a morphism of representations
	\[ (A, r_A) \otimes (A, r_A) \pfeil{} (A, r_A). \]
	Indeed,
	\begin{align*}
	m((r_A \otimes r_A)(g)(f_1 \otimes f_2))(x) &= f_1(xg)f_2(xg),\\
	&= f_1f_2(xg) = r_A(g)(m(f_1 \otimes f_2))(x),
	\end{align*}
	since $f_1(\_ g) \otimes f_2(\_ g) = (r_A \otimes r_A)(g)(f_1 \otimes f_2)$.
	
	Ergo $m \circ (r_A \otimes r_A)(g) = r_A(g) \circ m$.
\end{proof}


Recall: We wanted to prove the following theorem
\begin{theorem}
	Let $\lambda_V \in  \End(V)$ be given s.t. for all finite-dim. rep.s $V$ of $G$ s.t.:
	\begin{itemize}
		\item[(i)] $\lambda_k = 1$
		\item[(ii)] $\lambda_{V\otimes W} = \lambda_V \otimes \lambda_W$
		\item[(iii)] for all morphisms of rep.s $\phi :V \pfeil{} W$ we have
		\[ \phi \circ \lambda_V = \lambda_W \circ \phi. \]
	\end{itemize}
Then, there is exactly one $g \in G$ s.t. $\lambda_V = r_V(g)$ for all $V$.
\end{theorem}
\begin{proof}
	Last time, we saw that any such family $V \mapsto \lambda_V$ extends to \textbf{all} rep.s $V$ of $G$.
	
	Let's note also that, if $(V_0, r_0)$ is any representation of $G$ with trivial action, i.e. $r(g) = 1$ for all $g$, then $\lambda_{V_0} = 1$.
	Indeed, let $v\in V_0$. We must check that $\lambda_{V_0} v = v$. Since the action is trivial, any subsapce of $V_0$ is $G$-invariant.
	
	Consider the map
	\begin{align*}
	\phi : k& \Pfeil{} V_0\\
	\alpha & \longmapsto \alpha v
	\end{align*}
	where $v = \phi(1)$. Then, $\phi$ is a morphism of rep.s because the action is trivial.
	
	Thus,
	\[ \lambda_Vv = (\lambda_V \circ \phi)(1) \overset{(iii)}{=} (\phi \circ \lambda_k) (1) \overset{(i)}{=} \phi(1) = v. \]
	
	Consider $\lambda_A \in \End(A)$. Then,
	\[ \lambda_{A\otimes A} = \lambda_A \otimes \lambda_A. \]
	It is an easy exercise to see that $m : (A, r_A) \otimes (A,r_A) \pfeil{} (A,r_A)$ is a morphism of rep.s.
	
	By (iii) it follows, $m \circ (\lambda_A\otimes \lambda_A) = \lambda_A \circ m$, i.e.
	\[ \lambda_A(f_1f_2) = \lambda_A(f_1) \lambda_A(f_2) \]
	for all $f_1, f_2 \in A$. Thus, $\lambda_A$ is an algebra morhism (check, using the morphism $k \inj{} A$, that $\lambda_A(1) = 1$).
	
	Thus, $\lambda_A = \phi^*$ for some unique morphism $\phi$ of algebraic sets $\phi : G \pfeil{} G$.
	
	We claim that $\phi$ commutes with left multiplication i.e.
	\[ \phi(hx) = h \phi(x) \]
	for all $h,x \in G$. Indeed, let's consider the map
	\begin{align*}
	A & \Pfeil{} A\\
	f &\longmapsto f(h\cdot\_).
	\end{align*}
	This induces a morphism
	\begin{align*}
	(A, r_A) \Pfeil{\psi} (A,r_A).
	\end{align*}
	By (ii), $\psi \circ \lambda_A = \lambda_A \circ \psi$.
	
	Since $\lambda_A = \phi^*$, this implies the claim.
	
	Now, set $g := \phi(e)$. Then for all $h \in G$,
	\[ \phi(h) = \phi(he) = hg. \]
	Thus, $\lambda_A = \phi^* = r_A(g)$.
	
	(It remains to show that
	\[ \lambda_V = r_V(g) \]
	for each finite-dim. rep. $V$.)
	
	Let $V = (V,r)$ be any rep. This induces a map
	\begin{align*}
\Delta: V & \Pfeil{}V\otimes  A.
\end{align*}
If $\Delta v = \sum{v_i} \otimes f_i$, then
\[ hv = \sum f_i(h_i) \otimes v_i.  \]
Let \begin{align*}
\e : V\otimes A & \Pfeil{} V\\
v\otimes f &\longmapsto f(1) v.
\end{align*}
It follows $\e \circ \Delta : V \pfeil{} V$ is the identity map.


Let $(V_0, r_0)$ be the representation of $G$ with $V_0 := V$ and $r_0$ the trivial action.
Then, $\Delta : V \pfeil{} V_0\otimes A$ is a morphism of representations.

(Indeed, if $\Delta v = \sum v_i \otimes f_i$, then 
\[ \Delta (r(h) v) \overset{?}{=} (r_0(h) \otimes r_A(h) ) \Delta v \]
since
\begin{align*}
\Delta v &= \sum v_i \otimes f_i\\
\iff xv &= \sum f_i(x_i)v_i ~ \forall x \in G\\
\iff xhv &= \sum f_i(xh) v_i ~\forall x,h \in G.\\
\end{align*}
Since $r(h) v = hv$, it follows
\[ \Delta(hv) = \sum v_i \otimes f_i(\cdot h) \implies (?). ) \]


We want to show
\[ \lambda_V = r_V(g). \]
We have
\begin{align*}
\Delta \circ \lambda_V &\overset{(iii)}{=} \lambda_{V_0 \otimes A} \circ \Delta\\
&\overset{(ii)}{=} \lambda_{V_0} \otimes \lambda_A\\
&= 1 \otimes  \lambda_A = 1 \otimes r_A(g).
\end{align*}
This implies
\[ \Delta \circ \lambda_V = (1 \otimes r_A(g)) \circ \Delta \]
but also
\[ \Delta \circ r_V(g) = (1 \otimes r_A(g)) \circ \Delta. \]
Because of the injectivity of $\Delta$ it now follows
\[ \lambda_V = r_V(g).\qedhere \]
\end{proof}

\begin{corollary}
Let $\phi : G \pfeil{} H$ be any morphism of algebraic groups. Then, for all $g \in G$
\begin{align*}
\phi(g)_s &= \phi(g_s)\\
\phi(g)_u &= \phi(g_u).
\end{align*}
\end{corollary}
\begin{proof}
Let $V$ be any \df{faithful} representation of $H$, i.e. $r_V : H \pfeil{} \GL(V)$ is injective, (for a finite-dim. $V$).

Then, $r_V \circ \phi$ is a rep. of $G$. To prove (i), it suffices to show
\[r_V(\phi(g)_s)  = r_V(\phi(g_s))\]
since $H$ operates faithfully on $V$.

We know that
\[r_V(\phi(g)_s) = r_V(\phi(g))_s \]
(characterizing property of $h_s$ for $h \in H$).
On the other hand,
\[ r_V(\phi(g_s)) = (r_V\circ \phi)(g_s) = r_V(\phi(g))_s.\]
Therefore, claim (i) follows. (ii) works analogously.
\end{proof}

\begin{definition}
	Let $g \in G$ where $G$ is an algebraic group.
	We call $g$ \df{semisimple}, if $g = g_s$.
	
	We call $g$ \df{unipotent}, if $g = g_u$.
\end{definition}

\begin{lemma}
	For $g \in G$, the following are equivalent:
	\begin{enumerate}[(i)]
		\item $g$ is semisimple.
		\item $r_V(g)$ is semisimple for all finite-dim. rep. $V$.
		\item $r_V(g)$ is semisimple for at least one faithful f.d. rep. $V$ of $G$.
	\end{enumerate}
\end{lemma}
We get an analogous lemma for unipotent group elements.
\begin{proof}
	We have
	\begin{align*}
	(i) &\iff g = g_s \\
	&\overset{\text{Def. of }g_s \text{ by goal thm.}}{\iff} r_V(g) = r_V(g)_s \forall \text{ f.d. } V \\
&\iff	r_V(g) \text{ is semisimple}\\
	&\iff(ii) \implies (iii).
	\end{align*}
	On the other hand,
	\begin{align*}
	(iii) & \implies \exists~~ \text{faithful f.d. }V \text{ s.t. } r_V(g) = r_V(g)_s = r_V(g_s) \implies g = g_s.\qedhere
	\end{align*}
\end{proof}


\newpage
\section{Non-Commutative Algebra}
\begin{definition}
A \df{ring} $R$ (for now) is unital, associative but not necessarily commutative.
\end{definition}
\begin{example}
The ring of matrices over some field or ring.
\end{example}

\begin{definition}
A \df{left ideal} $I \subset R$ is a subset that is an abelian subgroup of $(R,+)$ s.t. $ra \in I$ for all $r \in R, a \in I$.

A \df{right ideal} $I\subset R$ is a subset that is an abelian subgroup with
\[ IR \subset I. \]
A two-sided ideal $I$ is a subset that is a left and a right ideal of $R$.
\end{definition}

It is easy to check that for any homomorphism of rings $\phi : R\pfeil{} S$, $\Ker \phi$ is a two-sided ideal. Also, if $J \subset R$ is any two-sided ideal, then  there exists a unique ring structure on $R /J$ s.t. the projection $R \pfeil{} R/J$ is a ring homomorphism.

\begin{definition}
A \df{left module} $M$ for $R$ is an abelian group equipped with a ring homomorphism
\[ R \Pfeil{\alpha} \End(M)\]
where $\End(M)$ acts on the left of $M$. We write
\[ rm:= \alpha(r) m. \]
We have
\[ (r_1r_2)(m) = r_1(r_2(m)). \]
If $R$ \df{acts} on $M$ by the left, we write
\[ R \curvearrowright M. \]
\end{definition}
\begin{example}
	$M_n(k) \curvearrowright k^n$ where $k^n$ is the space of column vectors.
	
	If $k^n$ denotes the space of row vectors, we have $k^n \curvearrowleft M_n(k)$.
\end{example}

\begin{definition}
	A \df{(left) submodule} $N \subset M$ is an algebraic subgroup s.t.
	\[ RN \subset N. \]
	It follows that $N$ itself is a left module.
\end{definition}
\begin{definition}
	A (left) module $M$ of $R$ is \df{simple} (or irreducible) if it has exactly the two submodules: $0 = \{0\}$ and $M$.
\end{definition}
\begin{definition}
	A ring $R$ is a \df{division ring} (aka \df{skew field}) if it satisfies any of the following equivalent requirements:
	\begin{enumerate}[(i)]
		\item $R^\times = R \setminus \{0\} $
		where\footnote{If $ar = rb = 1$, then $a = arb = b$.} $R^\times = \set{r \in R}{\exists a,b \in R: ar = rb = 1.}$
		\item $R$ has no nontrivials left or right ideals.
	\end{enumerate}
\end{definition}
\begin{definition}
If $R \curvearrowright M$, then we can define
\[ \End_R(M) := \set{\phi \in \End(M)}{\phi(rm) = r\phi(m) ~\forall r\in R, m \in M}. \]
Note, that $\End_R(M)$ is a ring.
\end{definition}
\begin{lemma}[Schur's Lemma]
	If $M$ is simple, then $\End_R(M)$ is a division ring.
\end{lemma}
\begin{lemma}
	Let $k$ be a field. Then, $M_n(k)$ has no nontrivial twosided ideals.
\end{lemma}
\begin{theorem}[Jacobson Density Theorem (Double Commutant Theorem)]
Suppose $M$ is a simple left module which is finitely generated as a right $D$-module for $D = \End_R(M)$.

Assume that $R$ acts faithfully on $M$, i.e. $R \pfeil{} \End_R(M)$ is injective.

Then, the map $R \pfeil{} \End_D(M)$ is an isomorphism.
\end{theorem}
\marginpar{Lecture from 16.03.2020 (Corona-Madness started here...)}
\paragraph{Recap:}
\begin{enumerate}[--]
	\item Basics: definitions, Hopf-algebras, ...	
	\item	Jordan decomposition
	\item	Primer on non-commutative algebra
	\begin{itemize}
		\item 	Jacobson density theorem
	\end{itemize}
	\item Unipotent groups
	\item Tori
\end{enumerate}
%\subsection{Jacobson Density Theorem}
We had last week
\[ \End_D(M) := \set{\phi \in \End(M)}{\phi \circ d = d \circ \phi~ \forall d\in D}. \]

Let $k$ be an algebraically cloesd field, $V$ a non-trivial finite-dimensional $k$-vector space and let $G$ be a subgroup of $\GL(V)$ that acts \df{irreducibly} on $V$, i.e.,
$V$ is\df{ $G$-irreducible}, i.e., the only $G$-invariant subspaces of $V$ are ${0}$ and $V$.

Set
\[ D:= \set{d \in \End_k(V)}{ dg = gd~ \forall g \in G } \]
and
\[ R := \Span_k(G) = \set{\sum_{i = 1}^nc_i g_i}{c_i \in k, g_i \in G, n \in \N_0}.\]
Then,
\[D = \End_R(V) \]
where $R$ is the $k$-subalgebra of $\End(V)$ that is generated by $G$.

\begin{lemma}[Schur's Lemma]
We understand $k \inj{} \End(V)$ as the inclusion of operations which operate by scalar multiplication
\[ k \Pfeil{\isom{}} \set{\phi : V \pfeil{} V}{\phi : v\mapsto t \cdot v ~~ \text{for some } t \in k}. \]
Let $V$ be $G$-irreducible. Then, we have
\[ D \isom{} k.\]
\end{lemma}
\begin{proof}
Let $d \in D$. Since $V \neq 0$, there is an eigenspace $V_\lambda\neq 0$ for $d$. Observe that $V_\lambda$ has to be $G$-invariant:\\
if $g \in G$ and $v \in V_\lambda$, then $gv \in V_\lambda$, since
\[ dgv = gdv = g(\lambda v) = \lambda gv. \]

Since $V_\lambda$ is a non-trivial $G$-invariant subspace and $V$ is irreducible under $G$, we have
\[ V_\lambda = V.\]
Ergo $d = \lambda$ in the sense of $k \inj{} \End(V).$
\end{proof}

\paragraph{Consequence of the Jacobson Density Theorem:} $R = \End_k(V)$, i.e., $G$ generates all linear operations on $V$, if $V$ is $G$-irreducible.

We will prove this after a lemma.
\begin{lemma}
	Let $V$ be $G$-irreducible.
	
	Let $n \in \N$. Set
	\[ V^n := V \oplus V \oplus \ldots \oplus V = V_1 \oplus \ldots \oplus V_n \]
	where each $V_i = V$.
	
	Let $v = (v_1,\ldots, v_n) \in V^n$ and set
	\[ Rv := \set{(rv_1, \ldots, rv_n)}{ r \in R } = \Span \{ (gv_1, \ldots, gv_n) ~|~ g \in G \}. \]
	
	Then, $Rv\neq V^n$ iff the $v_j$ are linearly dependent over $k$.
\end{lemma}
\paragraph{Consequence: } Take $n := \dim(V)$. Let $\{e_1, \ldots, e_n\}$ be a basis of $V$ and set
\[ e := (e_1, \ldots, e_n) \in V^n. \]
Since the $(e_i)_i$ are linearly independent, the lemma states that $Re = V^n$.

Now, let $x \in \End_k(V)$. Choose $r \in R$ s.t.
\[re = (xe_1, \ldots, xe_n). \]
Then $re_i = xe_i$ for all $i$, thus $x = r$. Hence, $R = \End_k(V)$.
\begin{proof}
	For $v = (v_1,\ldots, v_n) \in V^n$
choose $J \in \{1, \ldots, n\}$ as large as possible with
\[ Rv + V_1 + V_2 + \ldots + V_{J-1} =: U \neq V^n. \]
Such an $J$ does exist, since we know that $Rv \neq V^n$.

Then, $V_J \not\subseteq U$, otherwise we may increase $J$. Also, $U$ is invariant by the diagonal action of $G$ on $V^n$.
Thus, $V_J \cap U \subseteq V_J$ is a proper $G$-invariant subspace of the $G$-irreducible $V_J \isom{} V$. Therefore, $V_J \cap U = 0$.

On the other hand, by maximality of $J$, we have
\[ U \oplus V_J = V^n. \]
Ergo, the map (composition)
\[ V\isom{} V_J \inj{} V^n \surj{} V^n/ U  \]
is a $G$-equivariant isomorphism, since $U$ is $G$-invariant.

Let $z : V^n / U \pfeil{\isom{}} V$ be the inverse isomorphism. Let $l$ be the $G$-equivariant map given by
\begin{center}
	\begin{tikzcd}
	V^n \arrow[r, "l"] \arrow[d] & V\\
	V^n / U \arrow[ur, "z" ] &
	\end{tikzcd}
\end{center}
and let $l_j$ be the $G$-equivariant maps by restricting $l$ on $V_j$. Then $l_j \in D \isom{} k$.

Say $l_j = t_j \in k$. Then, 
\[ l(w) = t_1 w_1 + \ldots t_nw_n. \]
Since $z$ is an isomorphism, $l$ is nonzero and $(t_1, \ldots, t_n) \neq (0,\ldots, 0)$.

Since $l|_U = 0$, we can deduce for all $u \in U$
\[ t_1u_1+ \ldots + t_nu_n = 0. \]
But $v \in Rv \subseteq U$, so we may conclude -- as required -- that the $(v_i)_i$ are linearly dependent ($l(v) = 0$).
\end{proof}

\newpage
\section{Unipotent Groups}
Let $G$ be a subgroup of $\GL(V)$ where $V$ is a finite-dimensional vector space and $k$ an algebrically closed field.
\begin{definition}
We say that $G$ is \df{unipotent} if one of the following equivalent conditions hold for each $g\in G$:
\begin{itemize}
	\item  $g $ is unipotent (i.e. $(g-1)^n = 0$ for some $n \in \N$).
	\item all eigenvalues of $g$ are 1.
	\item $g$ is conjugate to $\mat{1 & \star & \star\\ 0 & \ddots & \star \\ 0 & 0 & 1}$.
\end{itemize}
\end{definition}
\begin{theorem}
	Any unipotent subgroup of $\GL_n(k)$ is conjugate to a subgroup of
	\[ U_n := \left\lbrace
	\mat{1 & \star & \star\\ 0 & \ddots & \star \\ 0 & 0 & 1}
	 \right\rbrace = \set{ U\in M_n(k) }{ U_{i,j} = \left\lbrace\begin{aligned}
		0, && \text{if } i > j\\
		1, &&\text{if } i = j\\
		\text{arbitrary}, && \text{otherwise}.
		\end{aligned} \right.}. \]
\end{theorem}
\begin{definition}
	For two subgroups $G,H$ of some common supergroup, define their \df{commutator} by
	\[ [G,H] := \shrp{ ghg\i h\i ~|~ g \in G, h \in H }. \]
	A group $G$ is called \df{nilpotent}, if one of its commutators is trivial, i.e. if we set
	\[ G_0 := G \text{  and  } G_{i+1} := [G_i, G], \]
	then $G$ is called nilpotent iff there is an $j\in \N$ with $G_j = 1$.
\end{definition}
\begin{corollary}
Any unipotent subgroup of $\GL(V)$ is nilpotent.
\end{corollary}
\begin{definition}
A group $G$ is called \df{solvable}, if $G^{(n)} = 1$ for some $n$ where
\begin{align*}
G^{(0)} &:= G,\\
G^{(i+1)} &:= [G^{(i)}, G^{(i)}].
\end{align*}
\end{definition}
Note that nilpotent groups are solvable, since $G^{(i)} \subset G_i$.
\begin{notation}
In the following, we will write $G' := [G,G]$.
\end{notation}
\begin{definition}
Let $n := \dim (V)$. A \df{complete flag} is a maximal strictly increasing chain of subspaces
\[ 0 = V_0 \subsetneq V_1 \subsetneq \ldots \subsetneq V_n = V. \]
\end{definition}
Any complete flag is of the form
\[ V_j := \Span \{ e_1, \ldots, e_j\} \]
for some basis $e_1, \ldots, e_n$ of V.

Let $B$ be the basis of some flag $0 = V_0 \subsetneq V_1 \subsetneq \ldots \subsetneq V_n = V$.
For $x \in \End(V)$, we have that $x$ is upper-triangle with respect to $B$ iff $x$ leaves each member $V_i$ of the flag invariant, i.e. $xV_i \subseteq V_i.$

\begin{proposition}[Key Proposition]
	Let $G$ be a unipotent subgroup of $\GL(V)$. Then there is a complete flag $ V_0 \subsetneq V_1 \subsetneq \ldots \subsetneq V_n$ consisting of $G$-invariant subspaces, i.e., each $V_i$ is $G$-invariant.
\end{proposition}
\begin{proof}
Recall, that $G$ is a unipotent subgroup of $\GL_n(V)$. We will give an induction on $n = \dim V$.

If $n = 0$, there is nothing to show.

Let $n \geq 1$. We may assume that $V$ is $G$-irreducible. Because, if not, there is a $G$-invariant subspace $0\neq W \subset V$ s.t. $W$ and $V/W$ have dimension $< n$. Then there exist complete $G$-invariant flags in $W$ and $V/W$ and the claim -- that there is a complete $G$-invariant flag in $V$ -- follows by the induction hypothesis.

By Jacobson Density Theorem, we have
\[ R := \Span(G) = \End(V) := \End_k(V). \]
Since $G$ is unipotent, we have for each $g \in G$
\[ \trace(g) = n.\]
Ergo, for $g,h \in G$
\[ \trace(gh) = \trace(h) \]
and
\[ \trace((g-1)h) = \trace(gh) - \trace(h) = 0. \]
Since $\Span(G) = \End(V)$, it now in particularly follows for all $g \in G, \phi \in \End(V)$
\[ \trace((g-1) \phi) = 0. \]
Since the above holds for all $\phi \in \End(V)$, it must hold
\[ g-1 = 0 \]
for all $g \in G$ (take for example the elementary matrices $\phi = E_{i,j}$). Ergo, $G$ is trivial. Then, any complete flag is trivially $G$-invariant.
\end{proof}
\begin{remark}
	This gives the group analogue of Engel's Theorem.
\end{remark}
\begin{proof}[Proof Goal Theorem]
	Let $B$ be a basis of $V$ s.t. $G$ leaves each subspace in the corresponding flag invariant. Then, $G$ is upper-triangle with respect to this basis.
	
	On the other hand, each $g \in G$ is unipotent, hence its diagonal (i.e. eigenvalues) are all $1$. Thus, with respect to $B$
	\[ G \subseteq \left\lbrace
	\mat{1 & \star & \star\\ 0 & \ddots & \star \\ 0 & 0 & 1}\right\rbrace = U_n.\]
\end{proof}
\begin{remark}
	Tori are of the form $(k^\times)^n$. In the case $k = \C$, $(\C^\times)^n$ are the complexification of $U(1)^n$. This equals tori in top. sense.
	\[
	\mat{1 & \Z \\ 0 & 1} \subseteq \GL_2(\C)
	 \]
	 is a non-algebraic unipotent group.
\end{remark}
\paragraph{Exercise.} (to be discussed next time)

it would have sufficed to prove the Goal theorem in the special case that $G$ is algebraic.


\paragraph{Corollary of Proof:} If $G \subset \GL(V)$ (with $V \neq 0$) is unipotent and $V$ is $G$-irreducible, then $G = 1$, $\dim V = 1$.

\marginpar{Lecture from 18.03.2020}

\paragraph{Answer to last Exercise:} Recall that the main point was to show that any unipotent subgroup $G \subseteq \GL(V)$ leaves invariant some complete flag $\F = (V_0 \subset V_1 \ldots)$. But by some homework (problem 1), the group
\[ \GL(V)_\F := \set{g \in \GL(V)}{g\F = \F} \]
is algebraic.

\textbf{Proof:} If $\F$ is the standard flag with $V_i = \Span(e_1, \ldots, e_i)$ for the standard basis $\{e_1, \ldots, e_n\}$, then
\[ \GL(V)_\F = \{ A \in \GL(V) ~|~ A \text{ is upper-triangle} \}. \]
The condition that $A$ is upper triangle can be realized by polynomials. \qed


Thus,
\begin{align*}
& G \text{ fixes }\F\\
\iff & G \subseteq \GL(V)_\F\\
\overset{\GL(V)_\F \text{ is algebraic}}{\iff}& \overline{G} \subseteq\GL(V)_\F\\
\iff & \overline{G} \text{ fixes } \F.
\end{align*}
Now, the Zariski-Closure $\overline{G}$ of any group $G$ is an algebraic group (shown in some homework).

Further, if $G$ is unipotent, then $\overline{G}$ is unipotent.

\newpage
\section{Tori}
\begin{definition}
A \df{torus} is an algebraic group that is isomorphic to $\G_m^n$ for some $n \in \N_0$ where $\G_m = k^\times = \GL_1(k)$ is the unit group of $k$.

We think of $\G_m^n \subseteq \GL_n(k)$ as the subgroup of diagonal matrices.
\end{definition}

\begin{lemma}
	Let $G$ be a commutative algebraic group. Then the following are equivalent:
	\begin{enumerate}[(i)]
		\item each $g \in G$ is semisimple.
		\item for each finite-dimensional representation $V $ of $G$ and for each $g \in G$, the operator $r_V(g)$ is diagonalizable.
		\item for all finite-dimensional representations $V$ of $G$, there is a basis of common eigenvectors for $r_V(G)$, i.e. a basis s.t.
		\[ r_V(G) \subseteq \G_m^n. \]
		\item $G$ is isomorphic to an algebraic subgroup of a torus.
	\end{enumerate}
\end{lemma}
\begin{proof}
	We show:
	\begin{enumerate}
		\item[(i) $\iff$ (ii):] This follows from the Jordan decomposition and definition of semisimple.
		\item[(ii) $\implies$ (iii)]: This is homework. Note that any commutative subset $S$ of $\GL(V)$ consisting of semisimple operators may be diagonalized simultaneously.
		\item[(iii) $\implies$ (iv)]: Take any faithful representation $V$ of $G$ and diagonalize it simultaneously. Then, $G \isom{} r_V(G) \subseteq \G_m^n$.
		\item[(iv) $\implies$ (i)]: Any diagonal matrix is semisimple.
	\end{enumerate}
\end{proof}

\begin{definition}
	A commutative algebraic group $G$ is called \df{diagonalizable}, if it satisfies one of the above equivalent conditions.
\end{definition}
\begin{definition}
	A \df{character} $\chi$ of an algebraic group $G$ is an element $\chi \in \Hom{\mathrm{alg.grp.}}{G}{k^\times}$, i.e., a homomorphism $\chi : G \pfeil{} k^\times$ of algebraic groups.
\end{definition}
\begin{notation}
For an algebraic group $G$, set $\X(G) := \Hom{\mathrm{alg.grp.}}{G}{k^\times}$.

Also denote now by $\O(X) := k[T] / I(X)$ the coordinate ring of an algebraic set $X$ (rather than $k[X]$).
\end{notation}

\begin{lemma}
	There is a bijection
	\[ \X(G) = \{ \text{characters } \chi \text{ of } G \} \longleftrightarrow \{ x \in \O(G)^\times ~|~ \Delta(x) = x \otimes x \}. \]
\end{lemma}
\begin{proof}
Note, that any $x \in O(G)^\times$ can be thought of as a map $x : G \pfeil{} k^\times \subset k$.

We have
\begin{align*}
\Hom{\mathrm{alg.grp.}}{G}{\G_m} &= \set{\phi \in \Hom{\mathrm{alg.sets}}{G}{\G_m}}{ \phi(gh) = \phi(g) \phi(h)~\forall g,h }\\
&= \set{\phi \in \Hom{k\mathrm{-alg.}}{\O(\G_m)}{\O(G)}}{(\phi \otimes \phi) \circ \Delta = \Delta \circ \phi}.
\end{align*}
\textbf{Recall:} $\O(\G_m) \isom{} k[t, \frac{1}{t}]$ with $\Delta(t) = t \otimes t$.

Thus for any $k$-algebra $A$, $\Hom{k\mathrm{-alg.}}{\O(\G_m)}{A} \isom A^\times$ via
\[ [t \mapsto a, (t\i \mapsto a\i)] \longleftrightarrow a. \]
Thus,
\begin{align*}
\Hom{\mathrm{alg.grp.}}{G}{\G_m} \isom{} \set{a \in \O(G)^\times}{ a \otimes a = \Delta(a) }.
\end{align*}
Therefore, it suffices to test the condition $ (\phi \otimes \phi ) \circ \Delta = \Delta \circ \phi $ on the generators $t,t\i$ of $\O(\G_m)$.
Now, the above isomorphism is given by
\[ \phi \mapsto a = \phi(t) \]
which is equivalent or regarding $\chi : G \pfeil{} \G_m$ as a map $\chi : G \pfeil{} k$.
\end{proof}

\begin{example}
	Let $G = \G_m$, then $\O(G) = k[t, \frac{1}{t}]$.
	
	Which $x = \sum_{m \in \Z}c_mt^m \in \O(G)$ -- with almost all $c_m = 0$, but not all of them -- have the property
	\[ \Delta(x) = x\otimes x? \]
	We have
	\begin{align*}
	x \otimes x &= \sum_{m,n \in \Z} c_mc_n t^m \otimes t^n,\\
	\Delta(x) &= \sum_{m\in \Z} c_mt^m \otimes t^m.
	\end{align*}
	Those sums equal, if
	\begin{align*}
	c_mc_n = o & \text{ for all } m \neq n,\\
	c_m^2 = c_m & \text{ for all m}.
	\end{align*}
	By those conditions, it follows
	\[ x = t^m. \]
	Therefore
	\[ \X(G) = \set{\chi_m}{m \in \Z} \isom{} \Z\]
	with
	\[ \chi_m(y) = y^m. \]
\end{example}

\begin{example}
Let $T \isom{} \G_m^n$ be a torus. Then,
\[ \X(T) = \set{\chi_m}{m \in \Z^n} \isom{} \Z^n \]
where $\chi_m(y) = y^m = y_1^{m_1}\cdots y_n^{m_n}.$
\end{example}
\paragraph{Note:} For each algebraic group $G$, $\X(G)$ is naturally an abelian group:
\[ (\chi_1 \cdot \chi_2)(g) := \chi_1(g) \cdot \chi_2(g). \]


Given a morphism of algebraic groups $f : G \pfeil{} H$, we get a morphism of abelian groups
\begin{align*}
f^* : \X(H) & \Pfeil{} \X(G)\\
\chi & \longmapsto \chi \circ f =: f^*(\chi).
\end{align*}
This induces a contravariant functor from the category of algebraic groups to the category of abelian groups.

\begin{lemma}
Let $G$ be a diagonalizable algebraic group. Then, $\X(G)$ is a $k$-vector space basis for $\O(G)$.
\end{lemma}
\begin{example}
Let $G = \G_m^n$ be a torus. Then, we have the embedding
\begin{align*}
\X(G) &\Inj{} \O(G)\\
\chi_{(m_1,\ldots, m_n)} & \longmapsto t^{(m_1,\ldots, m_n)}.
\end{align*}
The lemma is obvious in this case: each elment of $\O(G) = k [t_1,\ldots, t_n, t_1\i, \ldots, t_n\i]$ can be written uniquely as a linear combination of monomials.
\end{example}
\begin{proof}
\begin{enumerate}[(i)]
	\item $\X(G)$ spans $\O(G)$:\\
	Choose an embedding $G \subset \G_m^n$ of algebraic groups. Then, by restriction, we get
	\[ \O(\G^n_m) \surj{} \O(G). \]
	Since the $\chi_m, m \in \Z^n, $ span $\O(\G_m^n)$, their images $\chi_m|_G \in \X(G)$ span $\O(G)$.
	\item $\X(G)$ is linearly independent:\\
	Suppose otherwise and let $\phi_1, \ldots, \phi_m$ be a linearly dependent subset of $\X(G)$ with $m \geq 1$ chosen minimally, with $c_1, \ldots, c_m \in k^\times$ s.t.
	\[ \sum_{i = 1}^m c_i \phi_i = 0. \]
	We distinguish the following cases:
	\begin{enumerate}
		\item[$m = 1$:] In this case, we have $\phi_1 = 0$, but $\phi_1(1) = 1$, a contradiction.
		\item[$m > 1$:] We can assume $\phi_1 \neq \phi_2$, so there is an $h \in G$ s.t. $\phi_1(h) \neq \phi_2(h)$. Then,
		\[ \phi_1(h) \sum_{i=1}^m c_i \phi_i = 0, \]
		but also for all $h,g \in G$
		\[ \sum_{i = 1}^mc_i \phi_i(hg) =\sum_{i = 1}^mc_i \phi_i(h)\phi_i(g) = 0. \]
		This implies
		\[ \sum_{i = 1}^m c_i\phi_i(h) \phi = 0. \]
		Ergo
		\[ \sum_{i= 1}^m c_j(\phi_i(h) - \phi_1(h)) \phi_i = \sum_{i= 2}^m c_j(\phi_i(h) - \phi_1(h)) \phi_i = 0. \]
		Now, $\phi_i(h) - \phi_1(h)$ is zero if $i = 1$ and non-zero, if $i = 2$. Therefore, this yields a shorter linear dependency for the elements
		\[ \phi_2, \ldots, \phi_m, \]
		which contradicts our requirement.
	\end{enumerate}
\end{enumerate}
\end{proof}

\begin{definition}
Let $M$ be an abelian group. The \df{group algebra} on $M$ is the $k$-algebra $k[M]$ (not a coordinate ring!) defined as follows:
\begin{align*}
k[M] := & \text{ the }k\text{-vectorspace with basis }M\\
:= & \set{ \sum_{m \in M} c_m \cdot m }{c_m \in k, \text{ almost all }c_m = 0},
\end{align*}
where the multiplication on $k[M]$ extends that on $M$:
\[ (\sum_{m\in M}c_m m) (\sum_{n\in M} d_n n) = \sum_{m,n \in M} c_md_n mn. \]
\end{definition}
\begin{corollary}
	For a diagonalizable $G$, we have
	\[ \O(G) \isom{} k[\X(G)]. \]
\end{corollary}

\paragraph{Fact:} For an abelian group $M$, there is exactly one Hopf algebra structure on $k[M]$ given by $\Delta(m) = m\otimes m$ for all $m \in M$.

With this definition, the above isomorphism is one of Hopf algebras.

\begin{lemma}
If $G,H$ are diagonalizable algebraic groups, then
\[ \Hom{\mathrm{alg.grp.s}}{G}{H} \Pfeil{f \mapsto f^*} \Hom{\mathrm{grp.}s}{\X(H)}{\X(G)} \]
is a bijection.
\end{lemma}
\begin{proof}
\begin{align*}
\Hom{}{G}{H} \isom{} &\Hom{\mathrm{Hopf-alg.}}{\O(H)}{\O(G)}\\
 \isom{} &\set{\phi \in \Hom{k\mathrm{-alg.}}{\O(H)}{\O(G)}}{(\phi \otimes \phi) \circ \Delta = \Delta \circ \phi}.
\end{align*}
Since $\Hom{k\mathrm{-alg.}}{\O(H)}{\O(G)} \isom{} \Hom{}{k[\X(H)]}{k[\X(G)]}$, this reduces to the following lemma:
\begin{lemma}
Let $M_1, M2$ be two abelian groups. Then
\begin{align*}
\Hom{}{M_1}{M_2} & \Pfeil{\isom{}} \Hom{\mathrm{Hopf-alg.}}{k[M_1]}{k[M_2]}\\
\phi & \longmapsto \brak{ \sum c_m m \mapsto \sum c_m \phi(m) } .
\end{align*}
\end{lemma}
\begin{proof}
We have to show that
\[ M = \set{x \in K[M]^\times}{\Delta(x) = x \otimes x}. \]
Then, by this, it follows for each $\phi \in \Hom{\mathrm{Hopf-alg.}}{k[M_1]}{k[M_2]}$,
\[ \phi(M_1) \subseteq M_2. \]
Ergo, $\phi|_{M_1} \in \Hom{}{M_1}{M_2}$. Therefore, the surjectivity of the claimed bijection is shown.
The injectivity is clear, since $M$ generates $k[M]$ as a $k$-algebra.

To show
\[ M = \set{x \in K[M]^\times}{\Delta(x) = x \otimes x}, \]
let
\begin{align*}
x &= \sum c_m m\in K[M]^\times\\
\Delta(x) &= \sum c_m m \otimes m\\
x \otimes x &= \sum c_m c_n m\otimes n.
\end{align*}
If $\Delta(x) = x \otimes x$, then it follows
\[ x = m \]
for some $m \in M$.

\end{proof}
\end{proof}
\marginpar{Lecture from 25.03.2020}
\paragraph{Recall:} We have seen that for diagonalizable algebraic groups $G,H$
\[ \Hom{}{G}{H} \isom{} \Hom{}{\X(H)}{\X(G)}. \]
If $G$ is diagonalizable, then
\[ \O(G) \isom{} k[\X(G)]. \]

\begin{theorem}
The functor
\begin{align*}
G & \Pfeil{} \X(G)\\
f &\longmapsto f^*
\end{align*}
defines an equivalence of categories:
\[ \{\text{diagonalizable alg. groups}\} \isom{} \{ \text{finite-dim. abelian groups with no }\mathrm{char}(k)\text{-torsion} \}. \]
\end{theorem}

This amounts to the bijection above between Hom-spaces and the following lemma.
\begin{lemma}
	\begin{enumerate}[(i)]
		\item Let $G$ be a diagonalizable alg. group. Then, $\X(G)$ is a finitely generated abelian group with no $\mathrm{char}(k)$-torsion.
		\item Let $\Gamma$ be a finitely generated abelian group with no $\mathrm{char}(k)$-torsion. Then, there is a diagonalizable algebraic group $G$ s.t. $\X(G) \isom{} \Gamma.$
	\end{enumerate}
\end{lemma}
\newcommand{\chr}{\mathrm{char}}
\begin{proof}
We will use the following facts:
\begin{itemize}
	\item Let $n \in \N$. Then, 
	$t^n - 1$ is square-free in $k[t]$ iff the ideal $(t^n - 1)$ is radical in $k[t]$ iff $t^n -1$ has not repetitive root iff either $\chr(k) = 0$ or $\chr(k) = p > 0$ and $p\not | n$.
	
	(Proof: Galois Theory, seperable/inseperable extensions.)
	
	\item Let $M:= \Z/n\Z$. Then, the $k$-group-algebra generated by $M$ 
	\[k[M] \isom{} k[t] / (t^n - 1)\]
	is reduced iff either $\chr(k) = 0$ or $\chr(k) = p > 0, p\not| n$.
	
	\item If $M_1, M_2$ are abelian groups, then we have the following isomorphism of Hopf algebras
	\begin{align*}
	k[M_1] \otimes_k k[M_2] &\Pfeil{\isom{}} k[M_1 \oplus M_2]\\
	m_1 \otimes m_2 & \longmapsto m_1m_2
	\end{align*}
	where $M_1 \oplus M_2 \isom{} M_1 \times M_2$.
\end{itemize}
\begin{enumerate}[(i)]
	\item Embed $G \inj{} T := \G^n_m$ for some $n$. Then, we have a surjection $\Z^n\isom{} \X(T)  \surj{} \Xi(G)$. Ergo, $\X(G)$ is finitely generated.
	
	Suppose $\chr(k) = p > 0$. Let $\chi \in \X(G)$ with $\chi^p = 1$. Then, for all $g \in G$, $\chi^p(g) = \chi(g^p) = 1$. The unit group $k^\times$ has not $p$-torsion, therefore $G \inj{} T = (k^\times)^n$ has also no $p$-torsion. Therefore, the frobenius $g \mapsto g^p$ is an isomorphism on $G$. Therefore, $\chi = 1$ is a trivial character. Ergo $\X(G)$ has no $p$-torsion.
	
	\item Let $\Gamma$ be a finitely generated abelian group with no $\mathrm{char}(k)$-torsion. Then,
	\[\Gamma \isom{} \Z^r \oplus \Z/n_1\Z \oplus \ldots \oplus \Z/n_l\Z\]
	where $\chr(k) \not| n_1 , \ldots , n_l$. We may reduce to the cases:
	\begin{enumerate}[(a)]
		\item $\Gamma = \Z$: take $G = \G_m$, then $\Xi(G) \isom{} \Z \isom{} \Gamma$.
		\item $\Gamma = \Z/n\Z$ with $\chr(k) =: p \not | n$:\\
		take $G:= \mu_n := \set{y \in k^\times}{y^n = 1}$. Then, since $p\not| n$, $(t^n - 1)$ is radical.
		So, 
		\[ \O(\mu_n) \overset{\text{Nullstellensatz}}{=} k[t] /(t^n - 1) \isom{\text{ as Hopf algebras}} k[\Gamma] \]
		where $t$ gets mapped to the generator of $\Gamma$.
	\end{enumerate}
 \end{enumerate}
\end{proof}
\begin{corollary}
We have the bijection
\[
\{\text{tori}\}
\isom{}
\{ \text{ finitely generated free abelian groups} (\isom{} \Z^n) \}.
\]
\end{corollary}

\begin{remark}
\[
\{\text{algebraic group schemes} / k\}
\isom{\text{not necessarily natural}}
\{ \text{ f.g. Hopf algebras} \}.
\]
by
\[ G \mapsto \O(G) \]
and
\[
\{\text{diagonalizable algebraic group schemes}/k\}
\isom{}
\{ \text{ f.g. abelian groups} \}.
	\]
by
\[ G \mapsto \X(G). \]
Where $\mu_p$ in the left hand term gets mapped to $\O(\mu_p) = k[t] / (t^p - 1)$ with $p = \chr k$.
\end{remark}

\newpage
\section{Trigonalization}
We say a representation $r : G \pfeil{} \GL(V)$ of a group $G$ on a finite-dimensional $k$-vectorspace $V$ is \df{trigonalizable} if it admits a basis with respect to which $r(V)$ is upper-triangular:
\[ r(G) \subseteq \curv{
\mat{* & \ldots & *\\ 0 & \ddots & \vdots \\ 0 & 0 & *}
} \]

\begin{definition}
We call a subgroup $G \subseteq{} \GL(V)$ \df{trigonalizable}, if the identity representation is.
\end{definition}
\begin{lemma}
	Let $G$ be an algebraic group. The following are equivalent:
	\begin{enumerate}[(i)]
		\item Every finite-dimensional representation $r : G \pfeil{} \GL(V)$ is trigonalizable.
		\item Every irreducible representation of $G$ is 1-dimensional.
		\item $G$ is isomorphic to an algebraic subgroup of
		\[ B_n := \curv{
	\mat{* & \ldots & *\\
	0 & \ddots & \vdots\\ 0 & 0& *}	
	} \subseteq \GL_n(k). \]
\item There is a normal unipotent algebraic subgroup $U$ of $G$ s.t. $G/U$ is diagonalizable.
	\end{enumerate}
\end{lemma}
\begin{proof}
We prove as follows:
\begin{enumerate}
	\item[(i) $\implies$ (ii):] Let $V$ be an irreducible representation. Then, $V \neq 0$. Choose a basis $e_1, \ldots, e_n$ of $V$ s.t.
	\[ r(G) \subseteq B_n. \]
	Then, $r(G)e_1 \subseteq ke_1$, so $V_0 := k e_1$ is $G$-invariant. Ergo $V = V_0$ is 1-dimensional.
	\item[(ii) $\implies$ (i):] Let $V$ be a f.d. representation. We show by induction on $\dim(V)$ that $r : G \pfeil{} \GL(V)$ is trigonalizable:
	
	In the cases $\dim(V) = 0,1$, there is nothing to show.
	
	In the case $\dim(V) \geq 2$, assume that $V$ is not irreducible. Then, there is a $G$-invariant $V_0 $ with $0 \neq V_0\neq V$.
	
	By the induction hypothesis, $V_0$ and $V/V_0$ are trigonalizable. Ergo, $V$ is trigonalizable. 
	
	(\textbf{Recall:} we used this criterion above in the proof that unipotent groups are trigonalizable by showing that every ??? of each $G$ is trivial.)
	\item[(i) $\implies$ (iii):] Choose a faithful representation $V$ of $G$. Then, $G \isom{} r(G)$. Since $r$ is trigonalizable, there is a basis of $V$ s.t.
	\[r(G) \subseteq B_n \subseteq \GL_n(k).\]
	\item[(iii) $\implies$ (ii):] Suppose $G \subseteq B_n \subseteq \GL_n(k)$. Set
	\begin{align*}
	A_n &:= \curv{\mat{* & 0 & 0\\ 0 & \ddots & 0\\ 0 &0 & *}} \subseteq \GL_n(k),\\
	U_n &:= \curv{\mat{1 & \ldots & *\\ 0 & \ddots & \vdots\\ 0 &0 & 1} }\subseteq \GL_n(k) \text{ normal algebraic subgroup of }B_n,\\
	U &:= G \cap U_n \text{ normal unipotent algebraic subgroup of }G.
	\end{align*}
	Let $V$ be an irreducible representation of $G$, then $V$ is not zero. Consider the subspace of $V$ fixed by $U$
	\[ V^U := \set{v \in V}{r(u)v = v \forall u \in U}. \]
	Then, we get a representation
	\[ r|_U : U \Pfeil{} \GL(V). \]
	Then, $r(U)$ is a unipotent algebraic group of $\GL(V)$. Ergo,
	\[ r(U) \subseteq \curv{\mat{1 & \ldots & *\\ 0 & \ddots & \vdots\\ 0 &0 & 1} }. \]
Ergo, $V^U \neq 0$. Since $U$ is normal in $G$, the subspace $V^U$ of $V$ is $G$-invariant: if $v \in V^U, g \in G$, then for all $u \in U$ we have
\[ r(u)r(g)v = r(g) r(g\i u g) v= r(g) v \]
since $v \in V^U$. Ergo $r(g)v \in V^U.$

Since $V$ is irreducible, $V = V^U$, i.e. $U$ acts trivially on $V$. Ergo, $r$ descends to a representation of the group $G/U$.

But $G/U \inj{} B_n / U_n \isom{} A_n$. Therefore, $G/U$ and $r(G)$ are commutative. Moreover, for all $g \in G$, $r(g) \in \GL(V)$ is semisimple:

if $g = g_sg_u$, then $g_u \in U$, because $U_n$ is the group of unipotent elements of $B_n$.

Hence, $r(g) = r(g_s) r(g_u) = r_(g_s)$ is semisimple.

It follows that $r(G)$ is commutative and consists of semisimple elements. By some HW: $r(G)$ is trigonalizable. It is easy to show now that $V$ is one-dimensional. (Since $V$ is irreducible and $ke_1$ is $G$-invariant.)
\end{enumerate}
\end{proof}
\begin{definition}
	$G$ is \df{trigonalizable}, if it satisfies one of the above equivalent conditions.
\end{definition}
	Later, we will see, that if $G$ is connected, then being trigonalizable implies being solvable.
	
\newpage	
\section{Commutative Groups}
Let $G$ be an algebraic group. Denote by $G_s$ resp. $G_u$ the subsets of semisimple resp. unipotent elements of $G$.

Then, $G_u$ is always algebraical i.e. closed: if $G \inj{} \GL_n(k)$, then $G_u = \set{g}{(g-1)^n = 0}$.
$G_u$ does not need to be closed under multiplication (for example, take $G = \SL_2(k)$, $\mat{1 & 1 \\ 0 & 1}$ and $\mat{1 & 0\\ 1 & 1}$).

$G_s$ needs not to be algebraic: for example, take $G = \SL_2(k)$ and if $G_s$ were algebraic, then
\[ \set{\lambda \in k^\times}{
\mat{\lambda & 1\\ 0 & \lambda\i} \in G_s
} = \set{ \lambda}{\lambda \neq \lambda\i} \]
but the last set is not algebraic. Also, $G_s$ does not need to be a group.
\marginpar{Lecture from 30.03.2020}
We have the surjective map of sets
\begin{align*}
G_s \times G_u & \Pfeil{} G\\
(g_1, g_2) & \longmapsto g_1g_2.
\end{align*}
\begin{example}[Non-Example]
Take generic $g \in G_s, h \in G_u$ for $G=\SL_2(k)$. Then, $g,h$ do not commute and we have
\[ ((gh)_s, (gh_u)) \neq (g, h) \]
because Jordan components don't commute.
\end{example}
\begin{theorem}
Let $G$ be a commutative algebraic group. Then:
\begin{enumerate}[(i)]
	\item $G_s, G_u$ are closed subgroups and the multiplicative map $G_s \times G_u \pfeil{} G$ is an isomorphism of algebraic groups.
	\item $G$ is trigonalizable. Moreover, for each finite dimensional representation $r : G \pfeil{} \GL(V)$ there is a basis s.t.
	\begin{align*}
	r(G_s) \subseteq \curv{\mat{* & 0& 0\\
	0 & \ddots & 0\\
0 & 0 & *}} && 
	r(G_u) \subseteq \curv{\mat{1 & * & *\\
		0 & \ddots & *\\
		0 & 0 & 1}}.
	\end{align*}
	\item $G_s$ is diagonalizable.
\end{enumerate}
\end{theorem}
\begin{proof}
\begin{enumerate}
	\item[(ii)] Let $V$ be any irreducible representation of $G$. We have seen that commuting semisimple operators may be simultanously diagonalizable, then
	\[ V = \bigoplus_{\chi : G_s \pfeil{} \G_m} V_\chi \]
	where
	\[ V_\chi = \set{v \in V}{r(h)v = \chi(h)v ~\forall h \in G_s}. \]
	Since $G$ is commutative, each subspace $V_\chi$ is $G$-invariant ($r(h)r(g)v = r(g)r(h)v= r(g)\chi(h)v = \chi(h) r(g)v$).
	
	Since $V$ is irreducible, we must have $V = V_\chi$ for some $\chi$.
	
	Recall that $G \isom{} G_s \times G_u$ as abstract groups. We have seen that $r(G_s) \subseteq \G_m^n$. We proved a while ago that any unipotent group, such as $G_u$, is trigonalizable. Ergo, $V$ is trigonalizable. Since $V$ is irreducible, we have $\dim V = 1$.
	
	If we apply the same argument without assuming that $V$ is irreducible, then we see that $V$ is the coproduct of $V_\chi$'s as above and that each $V_\chi$ admits a basis s.t.
	\begin{align*}
r(G_s)|_{V_\chi} \subseteq \curv{\mat{* & 0& 0\\
		0 & \ddots & 0\\
		0 & 0 & *}} && 
r(G_u)|_{V_\chi} \subseteq \curv{\mat{1 & * & *\\
		0 & \ddots & *\\
		0 & 0 & 1}}.
\end{align*}
This yields the same conclusion for $V$.
\item[(i)] We have to show that $G_s$ and $G_u$ are closed and $j : G_s \times G_u \pfeil{} G$ is an isomorphism of groups. Take any faithful representation
\[ G \Pfeil{\isom{}, r} r(G) \subseteq \GL(V) \]
and apply (ii). Then we have
	\begin{align*}
r(G) &\subseteq \curv{\mat{* & *& *\\
		0 & \ddots & *\\
		0 & 0 & *}} =: B \\
B_u &= \curv{\mat{1 & * & *\\
		0 & \ddots & *\\
		0 & 0 & 1}},\\
	r(G_s) &\subseteq \curv{\mat{* & 0& 0\\
			0 & \ddots & 0\\
			0 & 0 & *}} =: A. \\
\end{align*}
In fact, $r(G_s) = r(G) \cap A$, because if $g \in G$ with $r(g) \in A$, then $r(g)$ is semisimple, so $g \in G_s$.

Therefore, $G_s$ is closed in $G$. Ergo, $G_s$ and $G_u$ are closed subgroups.

Then, the map $j$ is a morphism of algebraic groups.

We need to show that $j\i$ is a morphism of algebraic groups. For this, it suffices to verify that the projection $G \pfeil{} G_s$ is a morphism. But this map is given under $r$ by the morphism:
\begin{align*}
t:={\mat{a_1 & *& *\\
		0 & \ddots & *\\
		0 & 0 & a_n}} \longmapsto{\mat{a_1 & 0& 0\\
		0 & \ddots & 0\\
		0 & 0 & a_n}} =: t_s.
\end{align*}
This suffices because if $g = g_s g_u$, then $g_u = g_s\i g$, so if the map $g \mapsto g_s$ is a morphism, so is $g\mapsto gg_s\i = g_u$, hence so is $g \mapsto (g_s, g_u)$.
\item[(iii)] We have seen that $G_s$ is a closed subgroup. Hence $G_s$ is a commutative algebraic group where elements are semisimple. Ergo, $G_s$ is diagonalizable.
\end{enumerate}
\end{proof}

\newpage
\section{Connected Solvable Groups}

\begin{theorem}[Lie-Kolchin]
Let $G$ be a connected solvable algebraic group. Then $G$ is trigonalizable.
\end{theorem}
(By comparison, recall that we have seen so far that, if $G$ is commutative or unipotent, then $G$ is trigonalizable.)
We can reformulate this theorem as: Any connected solvable subgroup of $\GL(V)$ stabilizes some complete flag $\F = (V_0 \subsetneq \ldots \subsetneq  V_n)$.
\paragraph{Generalization (Borel's Fixed Point Theorem):} Any connected algebraic group $G$ acting on a projective variety $X$ has a fixed point in $X$.

We get a relation between complete flags and projective varieties.
\begin{proof}
	
	
Induct on the number $n$ s.t. $G^{(n)} = 1$.

For $n=0$, there is nothing to show.

If $n = 1$, $(G,G) = 1$, then $G$ is commutative, ergo trigonalizable.

Let $n \geq 2$. Then, we have $G' := (G,G) \neq 1$. We will show the following lemma:

\end{proof}
\begin{lemma}
	Let $G \subseteq \GL(V)$ be a subgroup.
	
	If $G$ is connected, then the group $G'$ with the induced topology is connected ($\iff$ the Zariski Closure of $G'$ is connected).
\end{lemma}
\begin{proof}
We have the following facts:
\begin{itemize}
	\item An increasing union of connected spaces is connected.
	\item A continuous image of a connected space is connected.
\end{itemize}
We have
\begin{align*}
G' =& \shrp{(g,h):= ghg\i h\i ~|~ g,h\in G }\\
=& \bigcup_{j \geq 0} \bigcup_{g_1,h_1, \ldots, g_j, h_j \in G} \{(g_1, h_1)\cdots (g_j, h_j)\}.
\end{align*}
Since
\[\bigcup_{g_1,h_1, \ldots, g_j, h_j \in G} \{(g_1, h_1)\cdots (g_j, h_j)\} = \Img \phi_j\]
for some continuous map $\phi_j : G^{2j} \pfeil{} G$, the claim follows.
Ergo, $G'$ is connected.
\end{proof}

\begin{remark}
	It is equivalent to show that $(*)$ any subgroup $G$ of $\GL(V)$ -- s.t. $G$ is connected and solvable -- is trigonalizable in $\GL(V)$.
	
	Indeed, the theorem implies $(*)$: the Zariski closure of $G$ is a connected algebraic group that is solvable (which extends by continuity).
	If $Zcl(G)$ is trigonalizable, then also $G$ is trigonalizable.
	
	On the other hand: $(*)$ implies the theorem, since if $G$ is given as in the theorem, apply $(*)$ to $r(G) \subseteq \GL(V)$.
\end{remark}

\begin{proof}[Proof of Theorem]
If $G^{(n)} = 1$, then $(G')^{(n-1)} = G^{(n)} = 1$. By induction, we may assume that $G'$ satisfies the following:

For all finite dimensional representations $r : G \pfeil{} \GL(V)$, $r(G')$ is trigonalizable.



Our aim is to show that any irreducible representation $V$ of $G$ has dimension $1$.

The induction hypothesis implies that $r(G')$ is trigonalizable. In particular, there exists an eigenspace $V_\chi \subseteq V$ for $G'$ for some character $\chi : G' \pfeil{} k^\times$. Since $G'$ is normal in $G$ we know that $G$ acts from the left on
\[\{ \text{eigenspaces }V_\chi \text{ in } V \text{ for } G' \}.\] 
Ergo, $\bigoplus_{\chi : G' \pfeil{} k^\times}V_\chi$ is $G$-invariant.
Since $V$ is $G$-irreducible, we have
\[ V = \bigoplus_{\chi : G' \pfeil{} k^\times}V_\chi = \bigoplus_{\chi \in \X'}V_\chi  \]
for some finite subset $\X' = \set{\chi}{V_\chi \neq 0}$ of $\Hom{}{G'}{\G_m}$, since $V$ is finite dimensional.

\paragraph{Claim:} Let $h \in G'$. Then, the map
\begin{align*}
G & \Pfeil{} \GL(V)\\
g & \longmapsto r(ghg\i)
\end{align*}
has a finite image.
\begin{proof}
	Denote by $\chi \mapsto \chi^g$ the action of $g \in G$ in $\Hom{}{G'}{\G_m}$ given by $\chi^g(h) := \chi(ghg\i)$. This is an action, since $G'$ is normal.
	
	Note, that $\X'\subseteq \Hom{}{G'}{\G_m}$ is a finite subset.
	
	Also note, that the action $\chi \mapsto \chi^g$ descends to an action $G\curvearrowright \X' $.
	
	
	Now, let $\X' = \{\chi_1,\ldots, \chi_r\}$. The matrix $r(h)$ is totally determined by the values $\chi_1(h), \ldots, \chi_r(h)$. Then, the element $r(ghg\i)$ is totally determined by the values $\chi_1^g(h), \ldots, \chi_r^g(h)$. It follows
	\[ \# \set{r(ghg\i)}{g \in G} \leq r!. \]
\end{proof}
The following lemma is easy to show:
\begin{lemma}
Let $G$ be an algebraic set. Then, $G$ is connected iff for each finite algebraic set $X$, and for each morphism $f : G \pfeil{} X$ of algebraic sets, we have that $f$ is constant.
\end{lemma}
Claim with the Lemma implies that the map $g \mapsto t(ghg\i)$ is constant. This implies that $r(ghg\i) = r(h)$ for all $g \in G, h \in G'$. Ergo, $G$ stabilizes each eigenspace $V_\chi$ for $G'$. Ergo, $V = V_{\chi_0}$, since $V$ is irreducible.
\end{proof}

\begin{lemma}
	Let $G$ be any group with a finite dimensional representation $r : G \pfeil{} \GL(V)$. Then, the subspaces $V_\chi$ for $\chi \in \Hom{}{G}{k^\times}$ are linearly independent, i.e., the map
	\begin{align*}
	\oplus V_\chi \Pfeil{} V
	\end{align*}
	is injective.
\end{lemma}
\begin{proof}
	The spaces $V_\chi$ are $G$-invariant. Suppose, there exist distinct $\chi_1, \ldots, \chi_n$ of non-zero $v_j \in V_{\chi_j}$ s.t. $\sum_j v_j = 0$.
	
	We may assume that $n$, the number of $v_j$, is minimal. W.l.o.g., $n \geq 2$.
	
	Choose $g \in G$ s.t. $\chi_1(g) \neq \chi_2(g)$. Use that $0 = g\sum_j v_j = \sum_j gv_j$ and take the linear combination as in the proof of linear indpendence of characters to contradict the minimality of $n$.
	
	($g- \chi_1(g)$ is not zero, but reduces $\sum_j v_j $ by one summand.)
\end{proof}




\marginpar{Lecture from 01.04.2020}
Since $G' = \shrp{ghg\i h\i ~|~ g,h \in G}$, so $\det(r(G')) = 1$.

On the other hand, for each $g \in G'$, we have
\[
r(g) = \mat{
\chi_0(g) \\
& \ddots \\
 & & \chi_0(g)
}
\]
since $V = V_{\chi_0}$. This implies
\[ 1 = \det(r(g)) = \chi_0(g)^d. \]
Ergo, $\chi_0$ defines a morphism
\[ \chi_0 : G' \Pfeil{} \mu_d \subseteq \G_m. \]

But $G'$ is connected and $\mu_d$ is finite. Since $\chi_0$ is a morphism, $\chi_0$ must be constant, ergo the trivial character.

As a consequence, we get $r(G') = 1$ on $V = V_{\chi_0}$.

\begin{lemma}
	Let $G$ be an algebraic group, $r : G \pfeil{} \GL(V)$ a representation. $v \in V$ shall be a simultaneous non-zero eigenvector for $r(G)$.
	
	Then, for each $g \in G$, there is a value $\chi(g) \in k^\times$ s.t.
	\[ r(g) v =: \chi(g)v. \]
	Then, the mapping $\chi : G \pfeil{} \G_m$ is a morphism of algebraic groups.
\end{lemma}

Therefore, $r $ descends to a representation of the commutative group
\[ \overline{r} : G / G' \Pfeil{} \GL(V). \]
Ergo, $r(G / G') = r(G)$ is commutative and therefore trigonalizable (because of irreducibility). \qedsymbol


\begin{example}[Non-Example]
\begin{itemize}
	\item Take $ G = D_4 \inj{} \GL_2(\C)$ which is solvable and has an irreducible and faithful representation over $\C^2$. 
	\item Consider the solvable group
	\[ G = \shrp{\mat{\pm 1 \\ & \pm 1}, \mat{& 1 \\ 1}} \]
	which is a finite subgroup of $\GL_2(\C)$, s.t. $\C^2$ define an irreducible representation of $G$.
\end{itemize}
\end{example}

\begin{lemma}[Form of Schur's Lemma]
	If $S$ is any commutative subset of $\GL(V)$ for a finite-dimensional $0\neq V$ over an algebrically closed field $k$. Let $V$ be $S$-irreducible.
	
	Then, $\dim V = 1$.
\end{lemma}
\begin{proof}
	There is notthing to show if $S$ is empty.
	
	Let $s \in S$ and denote by $V_\lambda \subseteq V $ the $\lambda$-eigenspace for $s$. Then, since $S$ is commutative, $V_\lambda$ is $S$-invariant. Therefore, $V = V_\lambda$ for one $\lambda \in k^\times$.
	
	Thus, every $s \in S$ acts by scaling, therefore every subspace of $V$ is $S$-invariant. Since $V$ is invariant, we get $\dim V = 1$.
\end{proof}

\begin{corollary}
	Let $G$ be a connected algebraic group. Then, $G$ is solvable iff $G$ is trigonalizable.
\end{corollary}

\begin{proposition}
	If $G$ is trigonalizable, then $G_u$ is a normal algebraic subgroup.
\end{proposition}
\begin{proof}
	We have
	\[ G \inj{} B:= \curv{
\mat{* & \ldots & *\\  & \ddots & \vdots \\ & & *}	
}  \subseteq \GL_n(k). \]
$B$ has the normal subgroup $U := \curv{
	\mat{1 & \ldots & *\\  & \ddots & \vdots \\ & & 1}	
}$ and we have $G_u = G \cap U$. Now, $U$ is the kernel of the multiplicative morphism
\[ 
\mat{a_1 & \ldots & * \\ & \ddots & \vdots \\ & & a_n} \longmapsto \mat{a_1 &  &  \\ &  &  \\ & & a_n}.
\]
\end{proof}
\begin{corollary}
	If $G$ is connected and solvable, then $G_u$ is a normal algebraic subgroup.
\end{corollary}

\subsection{Semisimple Elements of nilpotent Groups}

\begin{theorem}
	Let $G$ be a connected nilpotent algebraic group. Then, we have
	\[ G_s \subseteq Z(G) \]
	where $Z(G)$ denotes the center of $G$.
\end{theorem}
\begin{theorem}[Lie-algebraic Analogue]
Let $V$ be a finite-dimensional vectorspace. Let $\mathfrak{g}$ be the Lie-Subalgebra of $\End(V)$, i.e. $\mathfrak{g}$ is a subspace s.t. we have for each $x,y \in \mathfrak{g}$
\[ [x,y] := xy - yx \in \mathfrak{g}. \]
Assume that $\mathfrak{g}$ is nilpotent, i.e. there is an $n \in \N_0$ s.t.
\[ [x_1, [x_2, [\ldots, [x_{n-1}, x_n]]]] = 0 \]
for all $x_1, \ldots, x_n \in \mathfrak{g}$.

Then, any semisimple (semisimple in $\End(V)$ that is) $x \in \mathfrak{g}$ is \df{central} in $\mathfrak{g}$, i.e. $[x,y] = 0$ for each $y \in \mathfrak{g}$.
\end{theorem}
\begin{remark}
	The Lie-algebraic Analogue implies the general theorem if -- for example -- $k = \C$.
\end{remark}
\begin{proof}
Let $g \in G_s$. We want to show $Z_G(g) = G$.

\newcommand{\Lie}{\mathrm{Lie}}
\paragraph{Fact from the theory of Lie-Algebras:}
For the Lie-Algebra $\Lie Z_G(g)$ we have
\[ \Lie Z_G(g) = \ker (\Ad(g)) \]
where $\Ad$ is the map
\begin{align*}
\Ad : G & \Pfeil{} \GL(\mathfrak{g})\\
x & \longmapsto gxg\i.
\end{align*}
Since $G$ is connected, it suffices to verify
\[ \ker (\Ad (g)) = \mathfrak{g} \]
i.e. $\Ad(g) = 1$.

Since $g$ is semisimple, we have for suitable basis
\[ g = \mat{a_1 \\ & \ddots \\ & & a_n} \]
with $a_j \in \C^\times$. This is $\exp(x)$ for a suitable diagonal matrix $x \mat{x_1 \\ & \ddots \\ & & x_n} \in \GL_n(\C)$.
\paragraph{Fact:}
We may assume that $x \in \mathfrak{g} := \Lie (G)$.

Since $G$ is nilpotent, it can be shown that $\mathfrak{g}$ is nilpotent.

By the theorem, $x$ is central in $\mathfrak{g}$. By the properties of $\exp$ we have
\[ \Ad(g) = \exp(\mathrm{ad}(g)) = 1 \]
ergo $\mathrm{ad}(x) = 0$ where $\mathrm{ad}: \mathfrak{g} \pfeil{} \mathfrak{g}$ is defined by 
\[ \mathrm{ad}(x) \cdot y := [x,y]. \]
\end{proof}

\begin{proof}
If $\mathfrak{g}$ is nilpotent, then $\mathrm{ad}(x) \in \End(\mathfrak{g})$ is nilpotent.

Since $x$ is semisimple, $\mathrm{ad}(x)$ is semisimple, because $\mathrm{ad}(x)$ is the restriction to $\mathfrak{g}$ of the map
\begin{align*}
\End(V) & \Pfeil{} \End(V)\\
y & \longmapsto [x,y]
\end{align*}
and, if $e_1, \ldots, e_n$ are a basis of eigenvectors for $x$, then $E_{i,j}$ is a basis of eigenvectors for $\ell$.

So, $\mathrm{ad}(x)$ is nilpotent and semisimple, therefore $\mathrm{ad}(x) = 0$.
\end{proof}

\begin{proof}[Proof Theorem]
	Let $G$ be a connected nilpotent algebraic group, $G \inj \GL(V)$.
	
	Let $g \in G_s$, we want to show that $g \in Z(G)$.
	
	Assume otherwise, then we have a $h \in G$ s.t. $(g,h) = ghg\i h\i \neq 1$.
	
	Since $G$ is connected and nilpotent (ergo solvable), we know by Lie-Kolchin that $G$ stabilizes some complete flag $V_0 \subset \ldots \subset V_n$.
	
	We have $g|_{V_i}, h|_{V_i} \in \GL(V_i)$. They commute, if $i = 0$, but not if $i = n$.
	
	So, there is an $i$ s.t. $g|_{V_i}, h|_{V_i}$ commute but  $g|_{V_{i+1}}, h|_{V_{i+1}}$ don't commute. W.l.o.g. $V =V_{i+1}, g = g|_{V_{i+1}}, h = h|_{V_{i+1}}$.
	Set $a:= g|_{V_i}, b:=  h|_{V_i} \in \GL(V_i)$. $a$ will be semisimple, since $g$ is.
	
	Since $g$ is semisimple, there is an eigenvector $v \in V_{i+1}$ for $g$ s.t.
	\[ V_{i+1} = V_i \oplus \shrp{v}. \]
	We have an isomorphism of vector spaces
	\[ \End(V_{i+1}) \isom{} \End(V_i) \oplus \Hom{}{\shrp{v}}{V_i} \oplus \Hom{}{V_i}{\shrp{v}} \oplus \End(\shrp{v}) \]
	with
	\[ \End(\shrp{v}) \isom{} k \text{  and  } \Hom{}{\shrp{v}}{V_i} \isom{} V_i. \]
	So, we can write $ g|_{V_{i+1}},   h|_{V_{i+1}}$ write as
	\[
	g = \mat{a \\ & * \in k} \text{  and  } 	h = \mat{b & c \in V_i \\ & * }.
	\]
	We may replace $g,h$ with scalar multiples to reduce to the case that $* = 1$. Then,
		So, we can write $ g|_{V_{i+1}},   h|_{V_{i+1}}$ write as
	\[
	g = \mat{a \\ & 1} \text{  and  } 	h = \mat{b & c  \\ & 1 }.
	\]
	Then,
	\[ h\neq ghg\i = \mat{b & ac \\ & 1}. \]
	Ergo, $c \neq ac$, i.e. $c \notin \ker(a -1)$. Let $h_1 := h\i gh g\i$. Check
	\[ h_1 = \mat{
	1 & b\i(a-1)c\\ & 1}. \]
We claim that $h_1$ does not commute with $g$. This claim implies the theorem, since we can iterate the claim to obtain elements $h_i$ by $h_{i+1} := h_i\i g h_i g\i$. Then, $h_i$ does not commute with $g$. But $G$ is nilpotent, therefore $h_i = 1$ for some large enough $i$.

We can prove the claim as follows: By some calculation as for $h$ and $g$, we see, that $h_1$ and $g$ don't commute iff $b\i(a-1)c \notin \ker(a-1)$. This is equivalent to
\begin{align*}
\iff & (a-1)b\i(a-1)c \neq 0\\
\iff & b\i(a-1)^2c \neq 0\\
\iff & (a-1)^2c \neq 0\\
\iff & c \in \ker((a-1)^2).
\end{align*}
But $a$ being semisimple implies $a-1$ being semisimple, therefore $\ker((a-1)^2) = \ker(a-1)$. So $h_1,g$ don't commute iff $c \in \ker(a-1) $ iff $h,g$ don't commute.
\end{proof}
\marginpar{Lecture from 06.04.2020}
\section{Algebraic Geometry}
\subsection{Projective Algebraic Sets}
Let $V$ be a finite-dimensional vector space. Then $\G_m = k^\times$ acts on $V$ by scalar multiplication. $\{0\}$ is a $\G_m$-invariant subspace of $V$. We are interested on the orbits of $\G_m$ on $V\setminus \{0\}$.

Define the \df{projective space} over $V$ by
\[ \P V := \G_m \backslash (V - 0) = (V- 0)/ \sim \isom{} \{ \text{lines in } V \} \]
where for $a, b \in V-0$ we set
\[ a\sim b : \Gdw{} \exists \lambda \in k^\times : \lambda a = b. \]
If $V = k^{n+1}$, we denote the $n$-dimensional projective space by $\P^n := \P V$.

Given $a = (a_0,a_1\ldots, a_{n}) \in k^{n+1} - 0$, we denote the $\sim$-class of $a$ by
\[ [a] = [a_0, \ldots a_{n}] \in \P^n. \]

Define $S$ to be the graded algebra of polynomials in $k$
\[S:= k[x_0, \ldots, x_n] = \bigoplus_{d \geq 0} S_d\]
where each $S_d$ is the space of homogenous polynomials of degree $d$, i.e.
\[ S_d = \bigoplus_{i_1, \ldots, i_d \in \{0,\ldots, n\}} k \cdot x_{i_1} \cdots x_{i_d}. \]
We identify $k$ with the space of constant polynomials $S_0 \subseteq S$.

We have
\[ S_d = \set{f \in S}{f(\lambda X) = \lambda^d f(X) ~\forall \lambda \in k^\times}. \]

Given $f \in S_d$, the set
\[ \set{a \in k^{n+1}}{f(a) = 0} \]
is $\G_m$-invariant. In other words, given $a \in \P^n$ and $f \in S^d$, it is well-defined to state $f(a) = 0$ and $f(a) \neq 0$.

\begin{definition}
	A \df{projective} algebraic subset $X \subseteq \P^n$ is a set of the form
	\[ X = V(\Sigma) := V_{\P^n}(\Sigma) \]
	where $\Sigma$ is a collection of homogenous elements of $S$, where
	\[ V_{\P^n}(\Sigma) := \set{a \in \P^n}{ f(a) = 0 ~ \forall f \in \Sigma }. \]
\end{definition}
\paragraph{Facts:}
\begin{itemize}
	\item Hilbert's basis theorem states
	\[ V(\Sigma) = V(f_1, \ldots, f_m) \]
	for some finite collection $f_1,\ldots, f_m \in \Sigma$.
	\item It is useful to extend the meaning of "$f(a) = 0$" for $a \in \P^n$ to \emph{general} elements $f \in S$ by requiring that $f(a') = 0$ for each $a' \in [a]$.
	
	If we write $f = \sum_{d \geq 0} f_d$, $f_d \in S_d$, then we have
	\[ f(a) = 0 \iff f_d(a) = 0 ~ \forall d \geq 0. \]
	Therefore, we can extend the definition of $V(\Sigma)$ to any $\Sigma \subseteq S$.
	\item We have $V(\Sigma) = V((\Sigma))$ where $(\Sigma)$ is the ideal generated by some finite subset of $\Sigma$.
	\item We call an ideal $I \subseteq S$ \df{homogenous} if it is the direct sum of its $d$-homogeneous components, i.e.
	\[ I = \sum_{d \geq 0}I_d \]
	where $I_d = \set{f \in I}{f \text{ is homogenous of degree }d}$.
	
	$I$ is homogeneous iff it is generated by homogeneous elements.
	\item We have the following \emph{Nullstellensatz}:
	
	For any $X \subseteq \P^n$, set $I(X)$ to be the ideal generated by all homogeneous polynomials of $S$ vanishing on $X$.
	
	Let $I \subseteq S$ be a \emph{homogeneous} ideal which is \emph{not equal} to $(x_0, \ldots, x_n)$. Then, we have
	\[ I(V_{\P^n}(I)) = \sqrt{I}. \]
\begin{example}[Anti-example]
	The second property is necessary:
	
Set $I = (x_0, \ldots, x_n)$. Then $V_{k^{n+1}}(I) = 0$. Therefore, $V_{\P^n}(I) = \emptyset$. However,
\[ I(V_{\P^n}(I)) = S. \]
\end{example}
\item The above point induces a bijection between algebraic subsets of $\P^n$ and radical ideals $I \subset S$ which are not $(x_0,\ldots, x_n)$.
\end{itemize}

For $i = 0,\ldots, n$, set $D(x_i) := \set{a \in \P^n}{a_i \neq 0}$. $D(x_i)$ is an open set homeomorphic to $k^n$ by mapping
\[ \phi_i : a \longmapsto (\frac{a_0}{a_i}, \ldots,\frac{a_{i-1}}{a_i},\frac{a_{i+1}}{a_i},\ldots, \frac{a_{n}}{a_i} ). \]
The $D(x_i)$ cover $\P^n = \bigcup_i D(x_i)$.

Given a projective algebraic subset $X \subset \P^n$, define $X^{(i)} \subset k^n$ by
\[ X^{(i)} := \phi_i (X \cap D(x_i)) .\]
If $X = V_{\P^n}(I)$, then
\[ X^{(i)} = V_{k^n}(I^{(i)}) \]
where
\[ I^{(i)} := \set{f^{(i)}}{f \in I} \]
where $f^{(i)} (t_1,\ldots, t_n) := f(t_1,\ldots, t_{i-1}, 1, t_i, \ldots, t_n)$. Thus, $X^{(i)}$ is an algebraic subset of $k^n$.
\begin{definition}
The \df{Zariski topology} on $\P^n$ is defined by setting the set of closed sets to be the set of projective algebraic sets.
\end{definition}
\paragraph{Facts:}
\begin{itemize}
	\item $D(x_i)$ is open in $\P^n$, since $D(x_i) = \P^n - V(x_i)$.
	\item The bijections $D(x_i) \isom{} k^n$ are homeomorphims.
\end{itemize}
\begin{definition}
	A \df{quasi-projective} algebraic set $Y$ is an open subset of a projective algebraic set $X \subseteq \P^n$.
\end{definition}
\begin{example}
	Any algebraic set in $k^n$ is quasi-projective.
\end{example}
\begin{definition}
A \df{quasi-projective variety} is defined as an irreducible quasi-projective algebraic set.
\end{definition}

\begin{lemma}[Products]
Define the \df{Segre-embedding} by
\begin{align*}
S^{n,m} : \P^n \times \P^m & \Inj{} \P^{nm+n+m}\\
(a,b) & \longmapsto [(a_ib_j)_{i,j=0,\ldots, n}].
\end{align*}
We have:
\begin{enumerate}
	\item  $S^{n,m}$ is injective.
	\item $S^{n,m}$ has a closed image.
	\item $k^n \times k^m \isom{} D(z_{00}) \cap S^{n,m}(\P^n \times \P^m) = S^{n,m}(D(x_0)\times D(y_0))$.
\end{enumerate}
\end{lemma}

\begin{definition}
	For quasi-projective algebraic sets $X \subset \P^n, Y \subset \P^m$, we define their product by
	\[ X\times Y := S^{n,m}(X,Y) \subseteq \P^{nm+n+m}. \]
	Then, $X\times Y$ is a quasi-projective algebraic subset of $\P^{nm+n+m}$.
\end{definition}

\marginpar{Lecture from 08.04.2020}

\printindex
\end{document}