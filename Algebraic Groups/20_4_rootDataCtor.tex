\subsection{Root Data -- Construction}
Let $G$ be a connected reductive group with a torus $T$. We want to construct a corresponding root datum
\[ \Psi = (X, X^\vee, R, R^\vee). \]
\begin{itemize}
	\item Set $X$ to be the \df{character lattice}
	\[ X = \X(T) = \Hom{}{T}{\G_m}. \]
	\item Set $X^\vee$ to be the \df{cocharacter lattice}
	\[ X^\vee =  \Hom{}{\G_m}{T}. \]
	\item If we are given $x \in X$ and $\xi \in X^\vee$, we define
	\[ \skp{x}{\xi} := m \in \Z \]
	s.t.
	\begin{align*}
	x \circ \xi : \G_m & \Pfeil{} \G_m\\
	t & \longmapsto t^m.
	\end{align*}
	(Recall, $\X(\G_m) = \Z$.)
	\item
	\[ R:= \set{0\neq \alpha \in \X(T)}{\mathfrak{g}^\alpha \neq 0}. \]
	\item Let $g \in \G_m$ be s.t. $\G_m = \overline{\shrp{g}}$. For $x \in R$, we need to choose $t \in T$ s.t.
	\[ x(t) = g^2 \]
	and, if we set
	\begin{align*}
	x^\vee = [g^n \mapsto t^n]
	\end{align*}
	the other axioms of a root datum are fulfilled.
\end{itemize}

\begin{example}
	\begin{itemize}
		\item $G = \SL_2(k)$:\\
		In $G$ we have the maximal torus
		\[ T= \curv{\mat{t & \\ & t\i}}. \]
		Consider the character
		\begin{align*}
		\lambda : T & \Pfeil{} \G_m\\
		\mat{t & \\ & t\i} & \longmapsto t.
		\end{align*}
		Set
		\[ R = \curv{\alpha, - \alpha} \]
		with
		\begin{align*}
		\alpha = 2\lambda : \mat{t & \\ & t\i} & \longmapsto t^2\\
		-\alpha : \mat{t & \\ & t\i} & \longmapsto t^{-2}.
		\end{align*}
		Then, we have
		\begin{align*}
		\mathfrak{g}^\alpha &= \curv{\mat{0 & * \\ 0 & 0}}\\
		\mathfrak{g}^{-\alpha} &= \curv{\mat{0 & 0 \\ * & 0}},
		\end{align*}
		since
		\[ 
		\mat{t & \\ & t\i} \mat{0 & 1 \\ 0 & 0} \mat{t & \\ & t\i}\i = \mat{0 & t^2 \\ 0 & 0}.
		 \]
		 Define
		 \begin{align*}
		 \alpha^\vee &:= \brak{
	t\mapsto \mat{t & \\ & t\i}	 
	 }\\
 (-\alpha)^\vee &:= - \alpha^\vee \brak{
 	t\mapsto \mat{t\i & \\ & t}	 
 }.
		 \end{align*}
		 Then, we have
		 \[ \skp{\alpha}{ \alpha^\vee} = 2. \]
		\item $G = P \GL_2(k)$:\\
		We have the maximal torus
		\[ T = \curv{\mat{t & \\ & 1}}. \]
		Set
		\[ R = \curv{\alpha, -\alpha} \]
		with
		\[ \alpha\mat{t & \\ & 1} := t. \]
		Define $\alpha^\vee$ by
		\[ \alpha^\vee(t) := \mat{t^2 & \\ & ^1} = \mat{t & \\ & t\i}. \]
		We have
		\[ \mat{t & \\ & 1} \mat{0 & 1\\ 0 & 0} \mat{t & \\ & 1}\i = \mat{0 & t \\ 0 & 0}. \]
	\end{itemize}
\end{example}
Now, let $G$ be reductive of semisimple rank one. Then:
\begin{itemize}
	\item $G /R(G) \isom{} P \GL_2(k)$ with $R(G) = 1$ or $R(G) = \mu_n$.
	\item $[G,G] \isom{} \SL_2(k)$ is semisimple of semisimple rank 1.
	\item $\{ \text{roots for }G/R(G) \} = \{ \text{roots for } G\}$.
	\item If $\alpha$ is a root of $G$, then it is also a root of $[G,G]$. Hence, define
	\[ \alpha^\vee : \G_m \pfeil{} [G,G] \inj{} G. \]
\end{itemize}
Let $G$ be reductive. Take a root $\alpha \in \X(T)$ and define the torus of codimension $1$ in $T$
\[ S_\alpha := \Ker (\alpha)^o. \]
Then,
\[ G_\alpha := Z_G(S_\alpha) \]
is connected and reductive.
Then,
\[ S_\alpha \subset Z_G(G_\alpha) \]
and
\[ S_\alpha \subset R(G_\alpha). \]
Ergo, $T / S_\alpha$ is a maximal torus of rank 1 in $G_\alpha / R(G_\alpha)$, while $G_\alpha / R(G_\alpha)$ is semisimple.

Ergo, $G_\alpha$ is reductive of semisimple rank 1.

\begin{example}
	Let $G = \GL_3$ with the maximal torus $ T = \curv{\mat{* & & \\ & * & \\ & & *}}$.
	
	Let $\alpha = \chi_1 / \chi_2$. Then,
	\[ S_\alpha = \curv{\mat{x & & \\ & x & \\ & & y}}. \]
\end{example}