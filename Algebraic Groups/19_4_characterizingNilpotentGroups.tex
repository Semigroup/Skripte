\subsection{Characterizing Nilpotent Groups via maximal Tori}

\begin{lemma}
	Let $T$ be a diagonaliable algebraic group. Then,
	\[ T = \overline{\bigcup_{n \geq 1} T[n]} \]
	where
	\[ T[n] := \set{t \in T}{t^n = 1}. \]
\end{lemma}
\begin{proof}
	If the claim is true for any $T_1, T_2$, then it is also true for $T_1\times T_2$.
	
	Therefore, we can reduce the claim to the cases where $T$ is finite or $T = \G_m$.
	
	A finite $T$ is contained in some $T[n]$.
	
	Let $T = \G_m$. Then, we have
	\[ \O(T) = k[x, \frac{1}{x}]. \]
	We need to show for each $f \in \O(T)$:
	\[ f(\alpha)  = 0 ~\forall \alpha \in k, \in \N_0 \text{ s.t. } \alpha^n = 1 \Impl{} f = 0. \]
	So, let $f \in \O(T)$. By multiplying with a large enough $x^r$, we can assume $f \in k[x]$. If $f\neq 0$, then $f$ has finitely many roots. However the set
	\[ \set{\alpha \in k}{\exists n \in \N_0:~ \alpha^n = 1} \]
	has infinitely many elements. Indeed we have
	\[\# \set{\alpha \in k}{ \alpha^n = 1} = n, \]
	if $\mathrm{char} k = 0$ or if $\gcd(\mathrm{char} k , n) = 1$.
\end{proof}
\begin{remark}
	If $\mathrm{char} k = 0$ and if $U$ is unipotent, then
	\[U[n] = 1  \]
	for all $n$.
\end{remark}

\begin{theorem}
	Let $G$ be a connected algebraic group. Then, the following are equivalent:
	\begin{enumerate}[(i)]
		\item $G$ is nilpotent.
		\item Each maximal torus $T$ of $G$ satisfies $ T \subset Z(G)$.
		\item $G$ has a unique maximal torus.
	\end{enumerate}
\end{theorem}
\begin{proof}
	We show:
	\begin{enumerate}
		\item[(i) $\implies$ (ii):] We have shown, if $G$ is nilpotent and connected, then each semisimple element is central.
		\item[(ii) $\implies$ (iii):] Any two maximal tori are conjugated.
		\item[(iii) + (ii) $\implies$ (i):] Since each semisimple element is contained in some torus, we have that each semisimple element is central.
		
		Let $B = U \rtimes T' \subset G$ be a Borel subgroup. Since $T' \subset T$ is central in $G$, we have $B = U \times T'$. So, $B$ is nilpotent. We showed in this case that
		\[ G = B. \]
		\item[(iii) $\implies$ (ii):] Let $T$ be the maximal torus of $G$. We have to show that $T$ is central.
		
		Since $T$ is unique, it must be normal in $G$. Then, $G$ acts via conjugation on $T$ and each $T[n]$. Therefore, we get a morphism for each $t \in T[n]$
		\begin{align*}
		G &\Pfeil{} T[n]\\
		g & \longmapsto gtg\i.
		\end{align*}
		Since $G$ is connected, this morphism must be constant, ergo trivial. Ergo, each $T[n]$ is central in $G$. Since
		\[ T = \overline{\bigcup_{n \in \N}{T[n]}}, \]
		$T$ must be central in $G$.
	\end{enumerate}
\end{proof}