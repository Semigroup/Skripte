\section{Reductivity}
Let $G$ be a connected algebraic group which acts on an affine variety $X$.
\begin{definition}
	A \df{quotient} of $X$ by $G$ is a pair $(Y, \rho)$ s.t.
	\begin{enumerate}
		\item $Y$ is an affine variety
		\item and $\rho : X \pfeil{} Y$ is a morphism which is constant on $G$-orbits.
	\end{enumerate}
Further, we demand that a quotient is initial in the category of all objects which fulfill the above conditions. I.e.
\begin{center}
	\begin{tikzcd}
	X \arrow[d, "\rho"] \arrow[rd, "\rho'"] & \\
	Y \arrow[r, dashed, "\exists_1 \phi"] & Y'
	\end{tikzcd}	
\end{center}
\end{definition}

\begin{remark}
	\begin{itemize}
		\item Such quotients need not to exist.
		\item Even when such quotients exist, they don't need to describe orbits. I.e., $G\backslash X$ must not be related to $Y$.
	\end{itemize}
\end{remark}

\begin{example}
	Consider the action of $G = \G_m$ on $X = k^1$. This action has two orbits:
	
	the open orbit $k \setminus \{0\}$ and the closed orbit $\{0\}$.
	
	Then the quotient of $X$ by $G$ is given by $(Y,\rho) = (\{0\}, x \mapsto 0)$.
	
	Note, if $f : X \pfeil{} k$ is regular and constant on $G$-orbits, then $f$ is constant on $X$, because $k \setminus \{0\}$ lies dense in $k$.
\end{example}
\begin{definition}
Let $G$ be a connected algebraic group.

We call $G$ \df{geometrically reductive} if we have for each finite-dimensional representation $V$ of $G$:
\begin{align*}
\forall v \in V^G ~ \exists f : V \pfeil{} k :~ f \text{ is a homogenous }G\text{-invariant polynomial s.t. }f(v) \neq 0
\end{align*}
where
\[V^G = \set{v\in V}{g.v = v ~\forall g \in G}. \]
\end{definition}
\begin{remark}
	$G$ is geometrically reductive iff for each affine $X$ on which $G$ operates and for each pair of closed $G$-invariant disjoint subsets $W_1, W_2 \subset X$ there is an $f \in \O(X)^G$ s.t.
	\begin{align*}
	f_{|W_1} &\equiv 1,\\
	f_{|W_2} &\equiv 0.\\
	\end{align*}
\end{remark}
\begin{proof}
	\begin{enumerate}
		\item[$\Rightarrow$:] Let $G$ be geomtrically reductive and connected. W.l.o.g., we can assume $X = W_1 \cup W_2$. According to the next theorem, we then have a quotient $(Y,\rho)$ of $X$ by $G$.
		
		$Y$ only contains points associated with the closed orbits of $X$. Each point $Y$ comes either from $W_1$ or $W_2$.

		 Therefore, we can define a function
		\[f' : Y \pfeil{} k \]
		s.t.
		\[ f'(\rho(W_1)) = 1 \text{  and  } f'(\rho(W_2)) = 0. \]
		Now, (for some reason?) $f'$ lies in $\O(Y)$. For $f = f'\circ \rho$, we now have
		\[ f \in \rho^*\O(Y) = \O(X)^G.  \]
		\item[$\Leftarrow$:] Let $V$ be a f.d. vector space, on which $G$ acts. Let $v \in V^G$. We need to show, that there is a homogenous, $G$-invariant $f : V \pfeil{} k$ s.t.
		\[ f(v) \neq 0. \]
		Assume $v \neq 0$ and take $X = V, W_1 = 0$ and $W_2 = v^\bot$. Then, there is an $f \in O(V)^G$ s.t.
		\begin{align*}
		f(v) = 1\text{  and  } f(v^\bot) = 0
		\end{align*}
		Now, because $f(v^\bot) = 0$, $f$ must be linear (or of high degree and homogenous in $\mathrm{char} k \neq 0$), ergo homogenous.
	\end{enumerate}
\end{proof}

\begin{theorem}
	Let $G$ be a connected algebraic group. 
	
	Then, $G$ is reductive iff $G$ is geometrically reductive.
\end{theorem}

\begin{theorem}
	Let $G$ be a connected algebraic group which is geometrically reductive and acts on an affine set $X$.
	
	Then, there is a quotient $(Y, \rho)$ of $X$ by $G$.
	
	Moreover, $\rho$ induces a bijection
	\[ \curv{ \text{closed }G\text{-orbits in }X } \longleftrightarrow Y. \]
\end{theorem}

\begin{definition}
	Let $G$ be a connected algebraic group.
	
	We call $G$ \df{linearly reductive} if we have for each finite-dimensional representation $V$ of $G$:
	\begin{align*}
	\forall v \in V^G\setminus \{0\} ~ \exists f : V \pfeil{} k :~ f \text{ is a linear }G\text{-invariant polynomial s.t. }f(v) \neq 0.
	\end{align*}
\end{definition}

\begin{remark}
	Naturally, linear reductivity implies geometrical reductivity.
	The converse does hold iff $\mathrm{char} k = 0$.
\end{remark}

\begin{remark}
	$\GL_n(k)$ is linear reductive.
\end{remark}
\begin{remark}
	$G$ is linear reductive iff every finite-dimensional representation $V$ of $G$ is completely \df{reducible}, i.e.
	\[ V = \bigoplus_i V_i \]
	where each $V_i$ is irreducible.
\end{remark}