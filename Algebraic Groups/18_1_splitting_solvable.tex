\section{Splitting Solvable Groups}
Let $B$ be a connected solvable algebraic group. (Then, $B$ is trigonalizable.)

Then, $U := B_u$ is a unipotent normal algebraic subgroup (since $U = R_u(B)$, since $B = R(B)$).

\begin{lemma}
	The group $B / U$ is a torus.
\end{lemma}
\begin{proof}
	We have an injective morphism
	\[ B \inj{} \curv{\mat{* & \ldots & * \\  & \ddots & \vdots \\ & & *}}\]
	with
	\[ U = B \cap \curv{\mat{1 & \ldots & * \\  & \ddots & \vdots \\ & & 1}}. \]
	Therefore, we get an injection
	\[B / U \inj{} \curv{\mat{* & \ldots & * \\  & \ddots & \vdots \\ & & *}} / \curv{\mat{1 & \ldots & * \\  & \ddots & \vdots \\ & & 1}} = \curv{\mat{* &  &  \\  & \ddots & \\ & & *}}. \]
	Ergo, $B/U$ is diagonalizable. Since $B$ is connected, $B/U$ is connected, too.
	It follows that $B/U$ is a torus.
\end{proof}

\begin{theorem}
	Let $B$ be a connected solvable algebraic group.
	
	Then, there is a torus $T \subset B$ s.t. the composition
	\[ T \inj{} B \surj{} B/U \]
	is an isomorphism.
\end{theorem}
Before, we can prove the therenm we need some lemmata:

\begin{lemma}
	Suppose $\mathrm{char} k = 0$. Let $T$ be a torus.
	
	Then, there is an $s \in T$ s.t.
	\[ \overline{\shrp{s}} = T. \]
	$s$ is called the \df{generator} of $T$.
\end{lemma}
\begin{remark}
	The lemma does not hold, if $\mathrm{char} k > 0$.
\end{remark}
\begin{proof}
	Recall that we have the following correspondence:
	\[ \curv{\text{tori}} \overset{T \mapsto \X(T)}{\longleftrightarrow} \curv{ \text{f.g. free }\Z\text{-modules} }. \]
	and in particular for each torus $T$:
	\begin{align*}
	\curv{\text{alg. subgroups }H\text{ of }T} &{\longleftrightarrow} \curv{ \text{submodules }\Gamma\text{ of }\X(T)  }\\
	H & \longmapsto \set{\chi \in \X(T)}{ \chi \in \X(T) : ~\chi_{|H} \equiv 1 }\\
	\set{t \in T}{\chi(t) = 1 ~\forall \chi \in \Gamma} &\longleftarrow\!\shortmid \Gamma.
	\end{align*}
	So, we have $\overline{\shrp{s}} = T$ iff
	\[ \chi(s) \neq 1 \]
	for each $1\neq \chi \in \X(T)$.
	
	W.l.o.g. $T = (k^\times)^n$. Then
	\[ \X(T) = \set{\chi_m}{m \in \Z^n} \]
	with
	\[ \chi_m(t_1,\ldots, t_n) = t_1^{m_1} \ldots t_n^{m_n}. \]
	We can then pick
	\[ s = (2,3,5,7, \ldots). \]
\end{proof}

\begin{lemma}
	If $\mathrm{char} k = 0$, then any bijective morphism of algebraic groups is an isomorphism of algebraic groups.
\end{lemma}
\begin{remark}
	This does not need to hold for non-zero charateristic. If $\mathrm{char} k = p$, then
	\begin{align*}
	k & \Pfeil{} k\\
	x & \longmapsto x^p
	\end{align*}
	is bijective without being isomorphic.
\end{remark}

\begin{proof}[Proof of Theorem]
	We only show the theorem in case $\mathrm{char} k = 0$.
	
	Let $B$ be a connected solvable algebraic group.
	
	We need to show, that there is a torus $T \subset B$ s.t. the composition
	\[ T \inj{} B \surj{} B/U \]
	is an isomorphism where
	\[ U = B_u. \]
	
	We know, that $B/U$ is a torus. Take $s' \in B/U$ s.t.
	\[ \overline{\shrp{s'}} = B / U. \]
	Take a preimage $g \in B$ s.t. $\pi(g) = s'$.
	
	We can decompose $g=su$ into a semisimple and a unipotent element. We then have
	\[ \phi(g) = \phi(s) \cdot \phi(u) = \phi(s), \]
	since $\phi(u)$ must be unipotent, ergo trivial.
	
	Set
	\[ T= \overline{\shrp{s}}. \]
	Since $s$ is semisimple, $T$ must be diagonalizable. Ergo
	\[ T\cap U = 1. \]
	Ergo, the chain
	\[ T \inj{} B \surj{} B/U \]
	must be bijective, hence an isomorphism, since $\mathrm{char} k = 0$.
\end{proof}
The theorem gives the structure of a semidirect product of algebraic groups:
\[ B = U \rtimes T \]
(where $T \curvearrowright U$ by conjugation.)

\begin{definition}
	Let $G_1, G_2$ be algebraic groups. Let $G_2$ act algebraically on $G_1$ via $b : G_2 \pfeil{} \Aut{}{G_1}$ s.t. the map
	\begin{align*}
	G_2 \times G_1 &\Pfeil{} G_1\\
	(g_2,g_1) & \longmapsto b(g_2)(g_1)
	\end{align*}
	is a morphism.
	
	Their semidirect group $G_1 \rtimes_b G_2$ is an algebraic group which is:
	\begin{itemize}
		\item set-theoretically $G_1 \times G_2$,
		\item group-theoretically the semidirect product $G_1 \rtimes_b G_2$. I.e. multiplication works by
		\[ (g_1,g_2) \cdot (h_1, h_2) = (g_1 \cdot b(g_2)(h_1), g_2 h_2 ). \]
	\end{itemize}
\end{definition}
\begin{remark}
	Even if we are given an algebraic group $G$ with closed subgroups $G_1, G_2$ s.t.
	\[ G = G_1 \rtimes G_2 \]
	 as abstract groups, it does not need to be the case that
	 \[ G_1 \rtimes G_2 \Pfeil{} G \]
	 is an isomorphism.
	 (However, it is the case, if $\mathrm{char} k = 0$.)
\end{remark}

\subsection{An Aside}
Let $G$ be an algebraic group and $H$ a normal algebraic subgroup.

Then, $G/H$ is a quasi-projective variety equipped with a $G$-action. Ergo, we have an algebraic group structure on $G/H$.

\begin{theorem}
	$G/H$ is an affine algebraic group.
\end{theorem}
\begin{proof}
	We need to show that $G/H$ is affine.
	
	We showed in a lemma long before, that there is a finite-dimensional representation $V,\rho$ s.t.
	\[ H = \ker \rho. \]
	
	Therefore, we can simply set
	\[ G/H := \Img (\rho) \subset \GL(V) \]
	which is closed as $\rho$ is a morphism of algebraic groups.
\end{proof}

