\subsection{Lie Algebras}
\begin{lemma}[Fact]
	Let $G$ be an algebraic group. Take a faithful representation
	\[ G \Inj{} \GL(V). \]
	Set
	\[ I:= I(G) \subset \O(\GL(V)). \]
	Consider the nilpotent element $\e$ in $\O(\GL(V))[\e] / (\e^2)$ and
	define the \df{Lie algebra} of $G$ by
	\begin{align*}
	\mathfrak{g} := \mathrm{Lie}(G) := &\set{x \in \End(V)}{\forall f \in I:~ f(1+\e x) = 0 \mod (\e^2)}\\
	= &\set{x \in \End(V)}{\forall f \in I:~ f(1+\e x) \in (\e^2)}\\
	= &\set{x \in \End(V)}{\forall f \in I:~ \frac{\mathrm{d}}{\mathrm{d} t}\klam{\frac{f(1+t x)}{t}}_{|t = 0} = 0. }.
	\end{align*}
	Then, $\mathfrak{g}$ is a $k$-vector space of dimension
	\[ \dim_k(\mathfrak{g}) = \dim(G). \]
\end{lemma}
\begin{proof}[Idea of Proof.]
	The proof boils down to show that $G$ is smooth at 1.
	
	However, each variety is generically smooth. Therefore, $G$ is smooth at some points. Since $G$ acts on itself via isomorphisms, 1 must look locally identically to one of $G$'s smooth points. Ergo, $G$ is smooth everywhere.
\end{proof}
\begin{example}
	\begin{itemize}
		\item $\mathrm{Lie}(\GL(V)) = \End(V)$ because $I(G) = 0$.
		\item 
		\begin{align*}
		\mathrm{Lie}(\O_n(k)) &= \set{ x \in M_n(k)}{ (1+\e x)(1+\e x^T) = 1 + (\e^2) }\\
		&= \set{ x \in M_n(k)}{ x^T = -x }.
		\end{align*}
	\end{itemize}
\end{example}
\begin{definition}
	To each algebraic group we can attach the \df{adjoint representation}
	\begin{align*}
	\Ad : G &\Pfeil{} \GL(\mathfrak{g})\\
	g &\longmapsto [x \mapsto gxg\i].
	\end{align*}
	If $T$ is a maximal torus in $G$, we can restrict the adjoint representation
	\begin{align*}
	\Ad : T &\Pfeil{} \GL(\mathfrak{g}).
	\end{align*}
\end{definition}
Given any representation $\rho : T \pfeil{} \GL(V)$, we may decompose $V$
\[ V = \bigoplus_{\chi \in \X(T) } V^\chi \]
where
\[ V^\chi = \set{v\in V}{\forall t \in T: ~t.v = \chi(t) v }. \]
Therefore,
\[ \mathfrak{g} = \bigoplus_{\chi \in \X(T)} \mathfrak{g}^\chi = g^o \oplus \bigoplus_{0\neq \alpha \in \X(T)} \mathfrak{g}^\alpha \]
where
\[ \mathfrak{g}^\alpha = \set{x \in \mathfrak{g}}{\forall t \in T: ~txt\i = \alpha(t) x } \]
and
\[ \mathfrak{g}^o = \set{x \in \mathfrak{g}}{\forall t \in T: ~txt\i =  x }. \]

\begin{example}
	Let $G \subset \GL_n(k)$ with the torus $T = \G_m^n$. Then,
	\[ \mathfrak{g} = M_n(k) = \mathfrak{g}^o \oplus \klam{\bigoplus_{0\neq \alpha} \mathfrak{g}^\alpha } \]
	where
	\[ \mathfrak{g}^o = \curv{\mat{ * & & \\ & \ddots & \\ & & *}}. \]
	For $i = 1,\ldots, n$ we define $\chi_i \in \X(T)$ as follows:
	\begin{align*}
	\chi_i( \mat{t_1 & & \\ & \ddots & \\ & & t_n} ) = t_i.
	\end{align*}
	Then, we have for each $i\neq j$
	\[ \mathfrak{g}^{\chi_i / \chi_j} = k E_{i,j} \]
	and for each other $\chi \in \X(T)$
	\[ \mathfrak{g}^{\chi} = 0.  \]
	Therefore
	\[ \mathfrak{g} = \mathfrak{t} \oplus \klam{
\bigoplus_{i\neq j} k E_{i,j}	
} \]
where
\[ \mathfrak{t} = \mathrm{Lie}(T) = \mathfrak{g}^o. \]
\end{example}