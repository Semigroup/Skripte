\section{Root Data}
\subsection{More on Reductive Groups}

\begin{proposition}
Let $G$ be a connected, reductive algebraic group. Then,
\[ R(G) = Z(G)^o. \]	
\end{proposition}
\begin{proof}
	$Z(G)^o$ is connected, normal and commutative, ergo solvable. Ergo $Z(G)^o$ is contained in $R(G)$.
	
	Since $R(G)$ is a normal subgroup of the connected group $G$, we have
	\[  G = N_G(R(G)) =N_G(R(G))^o. \]
	Since $R(G)$ is a torus in a connected group $G$, we have
	\[ N_G(R(G))^o = Z_G(R(G))^o. \]
	$G = Z_G(R(G))^o$ implies $R(G) \subset Z(G)$.
\end{proof}

\begin{proposition}
	Let $G$ be a connected, reductive algebraic group. Then,
	\[ R(G) \cap [G,G] \]
	is finite.	
\end{proposition}
\begin{proof}
	Take a faithful representation $G \inj{} \GL(V)$. $R(G)$ is a torus in $G$, therefore we can decompose $V$ into eigenspaces of $R(G)$:
	\[ V= \bigoplus_{\chi} V_\chi \]
	where $\chi \in \X(R(G))$ and
	\[ V_\chi = \set{ v \in V}{h.v = \chi(h) v ~\forall h \in R(G)}. \]
	Since $R(G)$ is normal in $G$, $G$ acts on each $V_\chi$. Consider the representations
	\[ \rho_\chi : G \Pfeil{} \GL(V_\chi). \]
	It is easy to see that
	\[ \rho_\chi ([G,G]) \subseteq \SL(V_\chi) \]
	and
	\[ \rho_\chi(R(G)) \subseteq Z_{\GL(V_\chi)}(\GL(V_\chi)) = k^\times. \]
	Ergo,
	\[ \rho_\chi(R(G)\cap [G,G]) \subseteq \mu_{\dim(V_\chi)}. \]
	And, therefore,
	\[ \# ([G,G] \cap R(G)) \leq \prod_{\chi : V_\chi \neq 0} \dim(V_\chi). \qedhere \]
\end{proof}

\begin{example}
	Let $G = \GL_n(k)$. Then
	\[ R(G) = k^\times \cdot 1_n. \]
	\[ [G,G] = \SL_n. \]
	Ergo,
	\[ R(G) \cap [G,G] \isom{} \mu_n. \]
\end{example}

\begin{proposition}
		Let $G$ be a connected, reductive algebraic group. 
		Then, $ [G,G] $ is semisimple, i.e. $R([G,G]) = 1$.
\end{proposition}
\begin{proof}
If $B'$ is Borel group in $[G,G]$, then $gBg\i$ stays a Borel group in $[G,G]$ for each $g \in G$. Therefore, $R([G,G])$ is normal in $G$.

	Now, take a Borel subgroup $B$ of $G$ s.t.
	\[ R([G,G]) \subset B. \]
Then, we have
\[ R([G,G]) \subset \bigcap_{g\in G}gBg\i = R(G). \]
So,
\[ R([G,G]) \subset R(G) \cap [G,G]. \]
Since $R([G,G])$ is finite and connected, it is trivial. Ergo, $[G,G]$ is semisimple.
\end{proof}

\begin{proposition}
	Let $G$ be a connected, reductive algebraic group. 
	
	Let $S\subset G$ be a torus. Then, $Z_G(S)$ is reductive.
	
	If $T$ is a maximal torus, then $Z_G(T) = T$.
\end{proposition}
\begin{proof}
	Since $S$ is a torus, $Z_G(S)$ is connected. Note that:
	\begin{enumerate}[(i)]
		\item every Borel subgroup of $Z_G(S)$ is contained in some Borel subgroup of $G$.
		\item for each Borel $B \subset G$ which contains $S$, the intersection
		\[ Z_B(S) = Z_G(S) \cap B \]
		is a Borel subgroup of $Z_G(S)$.
		\item From the above, it follows
		\[R(Z_G(S)) = \bigcap_{\substack{S \subset B \subset G \text{ Borel} }} Z_B(S)  \subset \bigcap_{\substack{S \subset B \subset G \text{ Borel} }} B . \]
		\item Since $R_u(G)$ is connected, we have
		\[ R_u(G) = \klam{ \bigcap_{\substack{B \subset G \text{ Borel} }} B}_u^o. \]
		\item One can show for any maximal torus $T$
		\[ R_u(G) = \klam{ \bigcap_{\substack{T \subset B \subset G \text{ Borel} }} B}_u^o. \]
	\end{enumerate}
Now, it follows
\[ R_u(Z_G(S)) \subset \klam{ \bigcap_{\substack{S \subset B \subset G \text{ Borel} }} B}_u^o \subset R_u(G) = 1. \]

For the second part:
$T$ is central in $Z_G(T)$ and a maximal torus in $Z_G(T)$. Therefore, $Z_G(T)$ must be nilpotent. Now, $W(Z_G(T), T) = 1$, ergo $Z_G(T)$ only has one Borel subgroup $R(Z_G(T))$. But $R(Z_G(T))$ must be a torus, hence
\[ R(Z_G(T)) = T.\qedhere \]
\end{proof}