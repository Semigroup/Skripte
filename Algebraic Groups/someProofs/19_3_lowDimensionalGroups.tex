\begin{lemma}[To be Proved]
	Let $G$ be a connected algebraic group with a Borel subgroup $B$.
	
If $B$ is nilpotent, then $G$ is solvable i.e. $B = G$.
\end{lemma}
\begin{theorem}[Low Dimensional Groups]
	Let $G$ be connected with $\dim(G) \leq 2$.
	
	Then, $G$ is solvable.
\end{theorem}
\begin{example}[Non-Example]
	The condition $\dim(G) \leq 2$ is necessary. Consider e.g. $G= \SL_2(k)$ which has a dimension of $3$.
\end{example}

\begin{corollary}
	Let $G$ be connected with $\dim(G) = 1$. Then, $G$ is commutative.
\end{corollary}
\begin{proof}
	Because of the theorem, $G$ is solvable. Therefore,
	$[G,G]$ is a closed proper subgroup of $G$. Hence, $\dim([G,G]) = 0$. Since $[G,G]$ is connected, it follows $[G,G] = 1$.
\end{proof}
\begin{remark}
	If $G$ is commutative, it decomposes nicely into semisimple and unipotent elements
	\[ G = G_s \times G_u. \]
	So, if $\dim(G) = 1$ and if $G$ is connected, then $G = G_s\isom{} \G_m$ is a torus, or $G = G_u \isom{} \G_a$ is unipotent.
	
	Further, we can consider
	\[ G = \curv{\mat{ * & * \\  & 1}}. \]
	$G$ is connected and of dimension 2. It decomposes
	\[ G = G_s \times G_u \]
	into two groups of dimension 1.
\end{remark}


\newpage
We induct on $\dim(B)$:

$\dim(B) = 0$: In this case, we have $B = 1$.
$G = \bigcup_{g\in G} gBg\i$, since $G$ is connected. Since $B = 1$, it follows $G = 1$.

$\dim(B) \geq 1$:
Since $B$ is nilpotent, we have a descending chain
\[ B = B_0 \supsetneq \ldots \supsetneq B_n \supsetneq 1 \]
where
\[ B_{i+1} = [B, B_i]. \]
Note, that each $B_i$ is connected, since $B$ is connected.
Let $Z(B) = \set{b \in B}{\forall g \in B:~ gb=bg}$ be the center of $B$ and let $Z:= Z(B)^o$ be the component of the neutral element. Then, we have
\[ B_{n} \subset Z.  \]
Ergo, $Z$ is not the trivial subgroup.

We want to show
\[ Z \subset Z(G). \]

Let $z \in Z$ and consider the morphism
\begin{align*}
\phi : G/B & \Pfeil{} G\\
gB & \longmapsto gzg\i.
\end{align*}
$\phi$ is well-defined, because $z \in Z(B)$. Since $\phi$ is a morphism from a projective variety to an affine variety, $\phi$ must be constant. Thus,
\[ Z \subset Z(G). \]
In particular, $Z$ is normal in $G$. We now get an inclusion of quotient groups
\[ B/Z \Inj{} G/Z. \]
It is clear that
\[ \dim(B/Z) < \dim(B). \]
Further, $B/Z$ is parabolic, since
\[ (G/Z) / (B/Z) = G/B \]
is projective. Ergo, $B/Z$ is Borel. By the induction hypothesis, we get
\[ G/Z = B/Z. \]
Ergo, $B = G$.
