\begin{lemma}
	Let $G$ be a connected algebraic group, $B \subset G$ a Borel subgroup and $S \subset B$ any torus.
	
	Then, $Z_B(S)$ is a Borel subgroup of $Z_G(S)$.
\end{lemma}
\paragraph{Claim in Proof:}
\[ Z_G(S) B = \set{g \in G}{\forall s \in S:~g\i s g \in sU} =:A \]
where $U := B_u$.
\begin{example}
	\begin{align*}
	G &= \GL_5(k)\\
	B &= \curv{\mat{ * & * & * & * & *\\  & * & * & * & *\\  &  & * & * & *\\  &  &  & * & *\\  &  &  &  & * }}\\
	S &= \set{\mat{t_1 & & & & \\ & t_1 & & & \\ & & t_1 & & \\ & & & t_2 & \\ & & & & t_2}}{t_1, t_2 \in k^\times}\\
	Z_G(S) &= \curv{\mat{ * & * & * &  & \\ * & * & * &  & \\ * & * & * &  & \\  &  &  & * & *\\  &  &  & * & *}} = \GL_3(k) \times \GL_2(k)\\
	Z_G(S) &= \curv{\mat{ * & * & * &  & \\  & * & * &  & \\  &  & * &  & \\  &  &  & * & *\\  &  &  &  & *}}.
	\end{align*}
\end{example}
\newpage
We showed before, that $Z_G(S)$ is connected, if $G$ is connected and $S$ a torus.

\paragraph{Proof of Claim:} It is easy to see, that
\[ Z_G(S) \subset A. \]
For $b \in B$, we have
\[ b\i sb\in sU, \]
since $B/U$ is commutative.

Now, let $g \in A$. Then,
\[ g\i S g \subset SU \subset B. \]
One can extend $S$ to a maximal torus $T$ of $B$. Then,
\[ B = U \rtimes T \supset SU = U \rtimes S.\]
Since $S$ is closed in $T$, $SU$ is closed in $B$. Further, $g\i S g$ and $S$ are maximal tori in $SU$. Then, there is a $b \in B$ s.t.
\[ b(g\i S g)b\i = S. \]
Set
\[ z:= gb\i. \]
We need to show, that $z$ lies in $Z_G(S)$.

Since $B/U$ is commutative, we have for each $s \in S$
\[ z\i s z = b(g\i s g)b\i \in g\i s g U= sU, \]
since $g \in A$.
Now, we have for each $s \in S$
\[ z\i s z \in sU \cap S = \{s\}.  \]
Ergo, $z \in Z_G(S)$.


\paragraph{Proof of Lemma:} We showed that
\[ Z_G(S) B = \set{g \in G}{\forall s \in S:~g\i s g \in sU} = \set{g \in G}{\forall s \in S:[g,s] \in U}. \]
Then, $Z_G(S)B$ is closed. Since
\[ \pi : G \surj{} G/B \]
is an open and surjective map, it is easy to see that
\[ Z_G(S) / Z_B(S) \isom{} \pi(Z_G(S)B) \]
is closed. Since $Z_B(S)$ is closed, $Z_B(S)$ is a parabolic subgroup of $Z_G(S)$. Since $Z_B(S)$ is contained in $B$, it is solvable, hence a Borel subgroup.


\newpage

\begin{theorem}[Normalizers of Borel Subgroups]
	Let $G$ be a connected algebraic group with a Borel subgroup $B \subset G$. Then,
	\[ N_G(B) = B. \]
\end{theorem}
\begin{corollary}
	We have a bijection:
	\begin{align*}
	G/B & \Pfeil{} \curv{\text{Borel Subgroups of }G}\\
	gB & \longmapsto gBg\i.
	\end{align*}
\end{corollary}

\newpage
	We induct on $\dim(G)$:\\
	$\dim(G) \leq 1$: $B$ is nilpotent, ergo $G = B$.
	
	$\dim(G) \geq 2$:
	Let $T$ be a maximal torus in $B$. Let $x \in N_G(B)$.
	Then, $xTx\i$ is again a maximal torus in $B$. Since all maximal tori in $B$ are related via $B$-conjugation, there is $b \in B$ s.t.
	$xTx\i = bTb\i$.
	We therefore replace $x$ by $b\i x$ to achieve
	$ xTx\i = T$.
	
	Now, consider the map
$\rho : T  \pfeil{} T, t \mapsto txt\i x\i$.

If $\rho$ is \textbf{not surjective}, then we have, since all tori are irreducible, $\dim(\Img(\rho)) < \dim(T)$ and $\dim(\Ker(\rho)^o)  >0$.
		If we set $S:=\Ker(\rho)^o$, then $S$ is a non-trivial torus in $T$.
		Since $S \subset \Ker (\rho)$, $x$ centralizes $S$ and normalizes $B$. Hence, $x$ normalizes $Z_B(S)$.
		Because of the previous lemma, $Z_B(S)$ is Borel subgroup of $Z_G(S)$. If $Z_G(S) \neq G$, then the induction hypothesis implies
		$ x \in N_{Z_G(S)}(Z_B(S)) = Z_B(S) \subset B$.
		
		Otherwise, if $Z_G(S) = G$, then $B/S$ is a Borel subgroup of $G/S$. So, the induction hypothesis implies
		$ xS \in N_{G/S} (B/S) = B/S$,
		ergo $x \in B$.
		
$\rho$ is \textbf{surjective}:
		Then,
		$ T = \Img \rho \subset [N_G(B), N_G(B)]$.
		Set $H := N_G(B)$ and
		choose a finite-dimensional representation
		$ G \Pfeil{} \GL(V)$
		and a line $L \subset V$ s.t.
		$H = \set{g \in G}{gL = L}$.
		
		Then, we have a morphism of algebraic groups
		$ \gamma : H \Pfeil{} \GL(L) = \G_m(k)$.
		Since the right side is a torus, we have
		$\gamma_{| H_u} \equiv 1$ and $\gamma_{|[H,H]}  \equiv 1$.
		Ergo, $\gamma(T) = 1$ and, since $B = B_u \rtimes T$, $\gamma(B) = 1$.
		
		Choose a non-zero element $v \in L$ and consider $\phi : G/B \pfeil{} V, gB \mapsto gv.$.
		Since $G/B$ is a projective variety, while $V$ is an affine variety, $\phi$ must be constant. Therefore, we have $gv \in L$ for each $g \in G$.
		Ergo, $ G = H$ and $B$ is normal in $G$. But, now
		$ G = \bigcup_{g\in G} gBg\i = B$.
		Ergo $H = B$.
