\section{Borel and Parabolic Groups}
Let $G$ be a connected algebraic group.
\begin{definition}
	A subgroup $B \subset G$ is called \df{Borel}, if $B$ is maximal among all connected solvable closed subgroups.
	
	Since $\dim(G) < \infty$, Borel subgroups exist.
\end{definition}
\begin{definition}
	A subgroup $P \subset G$ is called \df{parabolic}, if the quasi-projective variety $G / P$ is \df{projective}, i.e. closed in $\P^n$.
\end{definition}
\begin{lemma}
	Let $G$ connected, $P$ parabolic, $B$ Borel. Then, $P$ contains some conjugate of $B$.
\end{lemma}
\begin{proof}
	$B$ acts on the projective variety $G/P$. According to Borel's fixed point theorem, there is a fixed point $gP \in G/P$ s.t.
	\[ bgP = gP \]
	for each $b \in B$. Ergo
	\[ g\i b g \in P \]
	for each $b \in B$.
\end{proof}

\begin{theorem}
	Let $G$ be connected.
	 
	Any two Borel subgroups are conjugate.
\end{theorem}
\begin{proof}
Take a faithful representation $G \inj{} \GL(V)$ with a finite-dimensional $V$. Let $\F = \Flag(V)$ be the flag variety of $V$.

Choose $F \in \F$ s.t. the orbit $G.F$ has a minimal dimension. Then, $G.F$ is closed, hence projective. If we set
\[ H:= \Stab_G(F), \]
then $H$ is parabolic. Therefore, each Borel group $B$ has a conjugate in $H$. Since $B$ is connected, its conjugate is contained in an irreducible component $H^o$ of the neutral element.

We claim that $H$ is solvable. Indeed, $H$ stabilizes a complete flag. Since the embedding $G \inj{} \GL(V)$ is faithful, $H$ must be isomorphic to a trigonalizable subgroup of $\GL(V)$. Since those are solvable, $H$ must be, too.

Since $H$ is solvable, $H^o$ is a connected, solvable, closed subrgoup. Ergo $H^o$ is the conjugate of $B$.
\end{proof}
\begin{proposition}
Let $G$ be connected. Then, each Borel group is parabolic.
\end{proposition}
\begin{proof}
	Let $B$ be a Borel subgroup of $G$.
	
	Take a representation $G \pfeil{} \GL(V)$ with a finite-dimensional $V$ s.t. there is a one-dimensional $L \subseteq V$ s.t.
	\[ B = \set{g \in G}{gL = L}.\]
	B acts on $V/L$. Since $B$ is connected and solvable there must be a complete $B$-invariant flag $\overline{F}$ in $V/L$. We can lift $\overline{F}$ to a complete flag $F = (L=V_1 \subset \ldots \subset V_n)$ of $V$. Then, it is easy to see
	\[  B= \Stab_G(F). \]
	
	Choose $F' \in \Flag(V)$ s.t. the orbit $G.F'$ has a minimal dimension. Then, $G.F'$ is closed, hence projective. If we set
	\[ H:= \Stab_G(F'), \]
	we have (by conjugating)
	\[ B = H^o.\]
	Consider the map
	\[ G/B = G/H^o \surj{} G/H. \]
	This map has finite fibers, because $[H: B] < \infty$. Ergo
	\[ \dim(G / B) \leq \dim(G/H). \]
	Ergo, $G/B$ is of minimal dimension, hence closed. Hence, $B$ is parabolic.
\end{proof}
\begin{corollary}
	Let $P$ be an algebraic subgroup of a connected algebraic group $G$.
	
	Then, $P$ is parabolic iff it contains a Borel group.
\end{corollary}
\begin{proof}
	The direction to the right is known.
	
	Let $P$ contain a Borel group $B$. Consider the maps
	\[ G / B \surj{} G /P \inj{} \P^n. \]
	Since $B$ is parabolic, $G/B$ is closed. Therefore, the morphism $G / B \pfeil{} \P^n$ has a closed image. But its image is exactly $G/P$. Ergo, $P$ is parabolic.
\end{proof}
\begin{corollary}
	Let $B$ be an algebraic subgroup of a connected algebraic group $G$.
	
	Then, $B$ is Borel iff it is a minimal parabolic subgroup.
\end{corollary}

\begin{example}
	If $G  = \GL_n(k)$, then
	\[ B = \curv{\mat{* & \ldots & * \\
	0 & \ddots & \vdots\\
 0  & 0 & * }} \]
is a Borel group.


Let $n = n1 + \ldots + n_r$ and set
\[ P_{(n_1,\ldots, n_r)} := 
\curv{
\mat{\GL_{n_1}(k) & * & * \\
	0 & \ddots & *\\
	0  & 0 & \GL_{n_r}(k) }
}.
 \]
 Each $P_{(n_1,\ldots, n_r)}$ is closed, since it is the stabilizer of an incomplete flag.
 
 In fact, each parabolic group is conjugate to one of those $P_{(n_1,\ldots, n_r)}$.
 
 If $P \neq G$ is parabolic, $P$ is called a \df{proper} parabolic subgroup.
\end{example}

\begin{example}
	\begin{itemize}
		\item $G = \SL_n(k)$: In this case parabolic groups are like in the above case, but inside of $\SL_n(k)$.
		\item $G = \SO_n(k)$: Then, we can embed $G$ in $\GL(V)$. Let $\skp{\cdot}{\cdot}$ be (any?) symmetric bilinear form.
		
		A subspace $W \subset V$ is called \df{isotopic} iff $\skp{\cdot}{\cdot}_{|W\times W} \equiv 0$.
		
		Then, we have the equivalence
		\[ \curv{ \text{Borel Group } B \subset G } \Leftrightarrow \{ \text{maximal isotropic flags }\F \text{ in }V \}. \]
		\item $G = \SP_{2n}$: The symplectic group is defined by
		\[ \SP_{2n} := \set{ A \in \GL_{2n}(k) }{ A^T \cdot \mat{ & 1\\ - 1 & } \cdot A = \mat{ & 1\\ - 1 & } }. \]
		Embed again $G$ in $\GL(V)$.
		
		Let $\skp{\cdot}{\cdot}$ be a \df{symplectic} form on $V$, i.e., $\skp{\cdot}{\cdot}$ is bilinear, alternating ($\skp{v}{v}= 0$) and nonsingular, i.e. $\skp{v}{\_} \equiv 0 \iff v = 0$.
		
		Then, again, we have the equivalence
		\[ \curv{ \text{Borel Group } B \subset G } \Leftrightarrow \{ \text{maximal isotropic flags }\F \text{ in }V \}. \]
		Further, we can take a basis $e_1,\ldots, e_n, f_1,\ldots, f_n$ of $V$ with
		\begin{align*}
		\skp{e_i}{e_j} &= \skp{f_i}{f_j} = 0\\
		\skp{e_i}{f_j} &= \delta_{i,j}.
		\end{align*}
		Then, one can for example set
		\[V_j = \mathrm{span}\{e_1,\ldots, e_j\} \]
		to get a flag $V_0 \subset V_1 \subset ...$.
		
		Vice versa, one can convert each maximal isotropic flag to such a symplectic basis.
	\end{itemize}
\end{example}