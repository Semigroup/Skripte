\subsection{Groups of Semisimple Rank One}

\begin{definition}
	Let $G$ be an algebraic groug. The \df{rank} of $G$ is defined by
	\[ \rank (G) := \dim(T) \]
	for any maximal torus $T$.
\end{definition}
\begin{remark}
	If $G$ is connected and of rank zero, then $G$ is unipotent.
\end{remark}
\begin{definition}
	The \df{semisimple rank} of $G$ is defined by
	\[ \ssr (G ):= \rank ( G / R(G)). \]
	(Note, that $G/R(G)$ is semisimple.)
\end{definition}
\begin{example}
	\begin{itemize}
		\item If $T$ is a torus, then $\ssr(T) = 0$.
		\item Let $Z = Z_{\GL_n(k)}(\GL_n(k)) $ be the centralizer of $\GL_n(k)$. Then, $Z \isom{} k$. For a matrix group $G \subset \GL_n(k)$, set
		\[ PG := G / (G \cap Z). \]
		Then, $PG$ acts on $\P^n$.
		
		We now have
		\begin{center}
			\begin{tabular}{l|c|c}
				$G$ & $\ssr(G)$ & $ \rank(G)$\\\hline
				$\SL_2$ & 1 & 1\\
				$P\GL_2$ & 1 & 1\\
				$\GL_2$ & $1$ & 2\\
				$\GL_n$ & $n-1$ & n
			\end{tabular}
		\end{center}
	\item Consider
	\[ G = \curv{\mat{* & * & * \\* & * & * \\ & & * }} \subset \GL_3(k). \]
	Then,
	\begin{align*}
	\rank(G) &= 3,\\
	\ssr (G) &= 1.
	\end{align*}
	\end{itemize}
\end{example}

\begin{remark}
	Let $G$ be a connected algebraic group. Then, $G$ is of semisimple rank zero iff $G$ is Borel (because $R(G)$ is the connected component of $1$ in the intersection of all Borel sungroups in $G$).
\end{remark}

\begin{lemma}[Fact]
	We call $X$ a \df{curve}, if $X$ is a one-dimensional variety.
	
	Let $X$ be a smooth projective {curve} $X$ that admits a nontrivial action by a nontrivial connected algebraic group $H$.
	Then, we have
	\[ X \isom{} \P^1. \]
\end{lemma}
\begin{proof}[Idea of Proof.]
		Reduce to the case $H = \G_m$ or $H= \G_a$. They give a commutative diagramm:
	\begin{center}
		\begin{tikzcd}
		\G_m \arrow[d, hook] \arrow[r] & X \\
		\P^1 \arrow[ur, "\exists \phi"] &
		\end{tikzcd}
	\end{center}
	Now, $\phi$ is a non-constant orbit map, therefore
	\[ k(X) \inj{} k(\P^1) \isom{} k(T). \]
	By Lüroth's Theorem\footnote{Lüroth's Theorem states that each intermediate field $L \supset P \supset K$ of a purely transcendental extension $L\supset K$ of degree 1 is either $K$ or purely transcendental over $K$.}, there is a transcendent $T' \in k(T)$ s.t. $k(X) = k(T')$.
	Ergo
	\[ X = \P^1.\qedhere \]
\end{proof}

\begin{lemma}[Fact]
	We have
	\[ \Aut{}{\P^1} \isom{} P\GL_2(k). \]
\end{lemma}

\begin{proposition}
	Let $G$ be of semisimple rank one. Then, there is a surjective morphism
	\[ \rho : G \pfeil{} P\GL_2(k) \]
	s.t. $\Ker \rho^o = R(G)$.
	
	In particular, if $G$ is semisimple, then $\Ker \rho$ is finite, since $R(G)$ is trivial in this case.
\end{proposition}
\begin{proof}
	By dividing out $R(G)$, we can reduce the proof to the case, in which $G$ is semisimple and of rank $1$. In particular, $G$ cannot be solvable, ergo $\dim(G) \geq 3$.
	
	We have seen for unsolvable groups, that $\# W(G,T) \geq 2$. But, since $T \isom{} \G_m$
	\[ W(G,T) = N_G(T) / Z_G(T) \Inj{} \Aut{}{T} = \{ \id, t\mapsto t\i \} \isom{} \Z/2\Z. \]
	Ergo, $\# W(G,T) = 2$. We have seen, that we have in this case
	\[ \dim(G/B) = 1. \]
	Ergo, $G/B$ is a projective one-dimensional variety. Ergo, $G/B \isom{} \P^1$.
	Now, define
	\begin{align*}
	\rho : G & \Pfeil{} \Aut{}{G/B} \isom{} \Aut{}{\P^1} \isom{} P\GL_2(k)\\
	g & \longmapsto [xB \mapsto gxB].
	\end{align*}
	Clearly,
	\begin{align*}
	\Ker \rho = \set{g\in G}{gxB = xB \forall x \in G} = \bigcap_{x\in G}xBx\i = \bigcap\{ B \subset G \text{ Borel}\}.
	\end{align*}
	Ergo $(\Ker \rho)^0 = R(G) = 1$, since $G$ is semisimple.
	
	It remains to show, that $\rho$ is surjective. Indeed, we have
	\[ \dim(\rho(G)) \geq \dim(G) - \dim(\Ker \rho) \geq 3. \]
	Since $P\GL_2 (k)$ is 3-dimensional and connected, we have
	\[ \rho(G) = P\GL_2(k). \qedhere \]
\end{proof}

\begin{proposition}
Let $G$ be a reductive algebraic group of semisimple rank one.

Let $\rho : G \surj{} \Aut{}{G/B} \isom{} P\GL_2(k)$ be as in the previous proposition. Then, $\ker \rho$ is diagonalizable.
\end{proposition}
\begin{proof}
	Let $T$ be a maximal torus in $G$. It suffices to show that
	\[ \Ker \rho \subset T. \]
	By the above proof
	\[ 2 = \# W(G,T) = \# \B^T. \]
	Therefore,
	\[ \B^T = \curv{ B^+, B^- }. \]
	Since $\Ker \rho \subset \bigcap_{B \subset G \text{ Borel}} B = B^+ \cap B^-$ it suffices to show
	\[ B^+ \cap B^- = T. \]
	Which is equivalent to
	\[ B_u^+ \cap B_u^- = 1. \]
	Since $G$ has a semisimple rank of one, we have $G \neq B^{\pm}$. Ergo, $B^\pm$ is not nilpotent (otherwise $G = B^\pm$).
	Thus, $B^\pm_u$ is connected and non-trivial. Thus,
	\[ \dim(B^\pm_u) \geq 1. \]
	Also,
	\[ (\Ker \rho)^o \cap B_u^\pm = R(G) \cap B_u^\pm \subset R_u(G) = 1 \]
	since $G$ is reductive. So, $\Ker (\rho_{|B_u^\pm} )$ is finite. And therefore
	\[ \dim(\rho(B_u^\pm)) \geq 1. \]
	But $\rho(B_u^\pm)$ is a unipotent subgroup of $P \GL_2(k)$. Therefore
	\[ \dim(\rho(B_u^\pm)) = 1. \]
	Since $\Ker (\rho_{|B_u^\pm} )$ is finite, we have
	\[ \dim B_u^\pm = 1. \]
	Ergo, $B_u^\pm \isom{} \G_a$.
	
	In characteristic zero, we proved that $T$ acts on $B_u^{\pm}$ by conjugation. We have that the composition
	\[ T \Pfeil{} \GL(\G_a) \isom{} \G_m \]
	is nontrivial, since $B^\pm$ is not nilpotent.
	
	We still want to show
	\[ B_u^+ \cap B_u^- = 1. \]
	Assume, for the sake of contradiction, that we have a non-trivial $x \in B_u^+ \cap B_u^-$. Then, $T.x = \set{txt\i}{t \in T}$ lies dense in $B_u^+ \cap B_u^-$, in fact
	\[ T.x \isom{} \G_a \setminus \{0\}. \]
	Since $B_u^\pm$ is one-dimensional, it follows $B^+_u = B^-_u$. Therefore,
	\[ B^+ = B^-. \]
	This is a contradiction to $\# W(G,T) = 2$.
\end{proof}
\begin{corollary}[Eventual Corollary]
	Any connected reductive group $G$ of semisimple rank one is isomorphic to one of the following:
	\begin{align*}
	\SL_2(k) \times T\\
	P\GL_2(k) \times T\\
	\GL_2(k) \times T
	\end{align*}
	for some torus $T$.
\end{corollary}