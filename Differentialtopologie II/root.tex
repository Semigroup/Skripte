%
% -------------------------------------------------------------------------------------------
% "THE BEER-WARE LICENSE"
% Jo, Ihr könnt mit diesem Code machen, was Ihr wollt, solange keinerlei Verantwortlichkeiten auf mich zurückfallen...
% Ansonsten kann ich Euch nur anregen, Wissen frei zu gestalten und allen Menschen zugänglich zu machen.
% Ja, und trinkt ein Bier oder eine Chai Latte oder einen Grünen Tee auf das Skript hier (oder ladet mich ein, wenn Ihr wollt).
% LG Akin
% -------------------------------------------------------------------------------------------
%

\documentclass[12pt]{book}

\usepackage[T1]{fontenc}
\usepackage[utf8]{inputenc}
\usepackage[ngerman]{babel}

\usepackage{tikz-cd}
\usetikzlibrary{babel}

\usepackage{amsfonts}
\usepackage{amssymb}
\usepackage{amsmath}
\usepackage{mathtools}
\usepackage{wasysym}
\usepackage{dsfont}
\usepackage{geometry}
\usepackage{makeidx}
\usepackage{booktabs}
\usepackage{hyperref}

\usepackage{enumerate}
\usepackage{adjustbox}

\newcommand{\ifLeer}[3]{\ifx&#1&\relax#2\relax\else\relax#3\relax\fi\relax}

\newcommand{\Def}[1]{\subsection{Definition\ifLeer{#1}{}{: #1}}}
\newcommand{\Bsp}[1]{\subsection{Beispiel\ifLeer{#1}{}{: #1}}}
\newcommand{\Lem}[1]{\subsection{Lemma\ifLeer{#1}{}{: #1}}}
\newcommand{\Bem}[1]{\subsection{Bemerkung\ifLeer{#1}{}{: #1}}}
\newcommand{\Kor}[1]{\subsection{Korollar\ifLeer{#1}{}{: #1}}}
\newcommand{\Satz}[1]{\subsection{Satz\ifLeer{#1}{}{: #1}}}
\newcommand{\Prop}[1]{\subsection{Proposition\ifLeer{#1}{}{: #1}}}

\newcommand{\QED}{\hfill $\square$}
\newcommand{\qed}{\hfill $\blacksquare$}

\newenvironment{Beweis}[1]{\paragraph{Beweis\ifLeer{#1}{}{: #1}\\}}{\QED}
\newenvironment{Beweisskizze}[1]{\paragraph{Beweisskizze\ifLeer{#1}{}{: #1}\\}}{\qed}

\newcommand{\df}[1]{\index{#1}\textbf{#1}}

\newcommand{\klam}[1]{\left(#1\right)}
\newcommand{\bet}[1]{\left|#1\right|}
\newcommand{\norm}[1]{\bet{\bet{#1}}}
\newcommand{\brak}[1]{\left[#1\right]}
\newcommand{\curv}[1]{\left\lbrace#1\right\rbrace}
\newcommand{\shrp}[1]{\left<#1\right>}
\newcommand{\quot}[1]{\glqq #1 \grqq\relax}
\newcommand{\set}[2]{\curv{\ifLeer{#2}{#1}{#1 ~ | ~ #2}}}
\newcommand{\grp}[2]{\shrp{\ifLeer{#2}{#1}{#1 ~ | ~ #2}}}

\newcommand{\A}{\mathcal{A}}
\newcommand{\B}{\mathcal{B}}
\newcommand{\C}{\mathbb{C}}
\newcommand{\D}{\mathcal{D}}
\newcommand{\E}{\mathcal{E}}
\newcommand{\F}{\mathcal{F}}
\newcommand{\G}{\mathcal{G}}
\renewcommand{\H}{\mathbb{H}}
\newcommand{\I}{\mathcal{I}}
\newcommand{\J}{\mathcal{J}}
\newcommand{\K}{\mathbb{K}}
\renewcommand{\L}{\mathcal{L}}
\newcommand{\M}{\mathcal{M}}
\newcommand{\N}{\mathbb{N}}
\renewcommand{\O}{\mathcal{O}}
\renewcommand{\P}{\mathcal{P}}
\newcommand{\Q}{\mathbb{Q}}
\newcommand{\R}{\mathbb{R}}
\renewcommand{\S}{\mathcal{S}}
\newcommand{\T}{\mathcal{T}}
\newcommand{\U}{\mathcal{U}}
\newcommand{\V}{\mathcal{V}}
\newcommand{\W}{\mathcal{W}}
\newcommand{\X}{\mathcal{X}}
\newcommand{\Y}{\mathcal{Y}}
\newcommand{\Z}{\mathbb{Z}}

\newcommand{\id}[1]{\text{Id}_{#1}}
\newcommand{\Ker}{\textsf{Kern}}
\newcommand{\Coker}{\textsf{Kokern}}
\newcommand{\Img}{\textsf{Bild}}
\newcommand{\Coimg}{\textsf{Kobild}}
\newcommand{\Hom}[3]{\textsf{Hom}_{#1}\left(#2, #3\right)}
\newcommand{\Aut}[2]{\textsf{Aut}_{#1}\left(#2\right)}
\newcommand{\Sym}[1]{\textsf{Symm}_{#1}}

\newcommand{\e}{\varepsilon}

\newcommand{\Pfeil}[1]{\overset{#1}{\longrightarrow}}
\newcommand{\pfeil}[1]{\overset{#1}{\rightarrow}}
\newcommand{\inj}[1]{\overset{#1}{\hookrightarrow}}
\newcommand{\Inj}[1]{\overset{#1}{\lhook\joinrel\longrightarrow}}
\newcommand{\surj}[1]{\overset{#1}{\twoheadrightarrow}}

\newcommand{\impl}[1]{\overset{#1}{\Rightarrow}}
\newcommand{\Impl}[1]{\overset{#1}{\Longrightarrow}}
\newcommand{\gdw}[1]{\overset{#1}{\Leftrightarrow}}
\newcommand{\Gdw}[1]{\overset{#1}{\Longleftrightarrow}}

\newcommand{\off}{\overset{o}{\subset}}
\newcommand{\abg}{\overset{c}{\subset}}

\newcommand{\gl}[1]{\overset{#1}{=}}
\newcommand{\grgl}[1]{\overset{#1}{\geq}}
\newcommand{\klgl}[1]{\overset{#1}{\leq}}
\newcommand{\gr}[1]{\overset{#1}{>}}
\newcommand{\kl}[1]{\overset{#1}{<}}
\newcommand{\isom}[1]{\overset{#1}{\cong}}

\newcommand{\supp}{\text{supp}}

\renewcommand{\i}{^{-1}}
\renewcommand{\phi}{\varphi}
\renewcommand{\d}{\text{d}}

\newcommand{\rot}{\text{rot}}

\renewcommand{\epsilon}{\varepsilon}
\newcommand{\sgn}{\text{sign}}
\newcommand{\Dd}[1]{\frac{\text{D}}{\d #1}}

\newcommand{\vol}{\mathrm{vol}}
\newcommand{\Ric}{\mathrm{Ric}}

\setlength{\marginparwidth}{20mm}

\makeindex
\date{\today}
\author{\href{mailto:tensor.produkt@gmx.de}{tensor.produkt@gmx.de}}

\makeindex

\begin{document}
\title{Mitschrieb: Differentialtopologie II\\
SS 18}
\maketitle
\section*{Vorwort}
Dies ist ein Mitschrieb der Vorlesungen vom 16.04.18 bis zum ... des Kurses \textsc{Differentialtopologie II} an der Universität Heidelberg.\\
Dieses Dokument wurde \glqq{live}\grqq\ in der Vorlesung getext. Sämtliche Verantwortung für Fehler übernimmt alleine der Autor dieses Dokumentes.\\
Auf Fehler kann gerne hingewiesen werden bei folgende E-Mail-Adresse
\begin{center}
	\href{mailto:tensor.produkt@gmx.de}{tensor.produkt@gmx.de}
\end{center}
Ferner kann bei dieser E-Mail-Adresse auch der Tex-Code für dieses Dokument erfragt werden.

\setcounter{tocdepth}{1}
\tableofcontents

%Prüfungstag: Mittwoch, 18. Juli

\chapter{Einführung in die Riemannsche Geometrie}
\section{Überblick und Ideen}
\marginpar{Vorlesung vom 16.04.18}

Bisher können wir durch die äußere Ableitung
\[ \d : \Omega^p(M) \pfeil{} \Omega^{p+1}(M) \]
nur Differentialformen auf glatten Mannigfaltigkeiten ableiten, aber keine anderen Objekte wie zum Beispiel Vektorfelder. Wir können also auch nicht über Phänomene aus der Physik wie Beschleunigung zum Beispiel sprechen.

\paragraph{Ziel}
Wir wollen einen Rahmen finden, in dem Objekte wie zum Beispiel Vektorfelder abgeleitet werden können.

\Bsp{}
Sei $f:M \pfeil{} \R$ eine glatte Funktion. Gilt $\d f = 0$ und ist $M$ zusammenhängend, so ist $f$ konstant.\\
Hätten wir für ein Vektorfeld $\xi$ eine Ableitung $\d \xi$, dann sollte die Gleichung $\d \xi = 0$ implizieren, dass $\xi$ \textsl{konstant} ist.\\
Ist zum Beispiel $\xi$ auf $M = \R^n$ konstant, so ist $\xi$ parallel, im Sinne von, die einzelnen Tangentialvektoren, die im Bild von $\xi$ liegen, sind parallel.\\
Somit impliziert eine Ableitung für Vektorfelder ein Konzept von \textsl{Parallelismus}.

\paragraph{Problem}
Ein Konzept von Parallelismus kann nicht über Karten erklärt werden, weil Kartenwechsel im Allgemeinen nicht winkeltreu sind.

\Bsp{}
Sei $M = S^2 \subset \R^3$ die zweidimensionale Einheitssphäre. Sei $p \in S^2$ und $\xi(p) \in T_pS^2$.\\
$\gamma$ sei ein Großkreis, der durch $p$ in Richtung $\xi(p)$ geht. Ist $p_1$ ein weiterer Punkt auf $\gamma$, so lässt sich $\xi(p)$ \textsl{naiv} wie gewohnt in $\R^3$ von $p$ auf $p_1$ verschieben. Dies hat das offensichtliche Problem, das der so parallel verschobene Vektor im Allgemeinem nicht tangential an $S^2$ anliegt.\\
Diesen kann man nun orthogonal auf den Tangentialraum $T_{p_1}S^2$ projizieren. Dadurch erhält man einen Tangentialvektor $\xi(p_1) \in T_{p_1}S^2$. Durch dieses Prozedere lässt sich $\xi$ glatt auf $S^2$ fortsetzen. Wählt man weitere Punkte $p_i$ auf $\gamma$, die gegen einen Punkt $q$ am Äquator konvergieren und für die gilt
\[ d(p_i, p_{i+1}) \Pfeil{} 0 \]
dann erhalten wir einen Vektor $\xi(q) \in T_qS^2$. Dies nennt man den \df{Paralleltransport} von $\xi(p)$ entlang $\gamma$ zu $\xi(q)$.\\
Allerdings kann man $\xi(p)$ auch entlang eines weiteren Großkreises $\gamma_1$ verschieben. Verschiebt man entlang $\gamma_1$ wieder auf den Äquator und von dort wieder auf $q$, so erhält man einen anderen Tangentialvektor auf $q$.

\paragraph{Neues Phänomen}
Für allgemeine Mannigfaltigkeiten hängt der Paralleltransport vom Weg $\gamma$ ab; im Gegensatz zum Euklidischen Raum.

\subsection{Zurück zu Ableitungen von Vektorfeldern $\xi$}
Auf $M$ sei Parallelismus gegeben (zum Beispiel ist $M$ eingebettet im $\R^n$). $p\in M$ sei ein Punkt und $v \in T_pM$ sei ein Tangentialvektor. $\xi$ sei ein Vektorfeld auf $M$.\\
Sei $\gamma$ eine glatte Kurve mit $\gamma(0) = p$ und $\dot{\gamma} (0) = v$. $q$ sei ein Punkt auf $\gamma$. Durch den vorgegebenen Parallelismus lässt sich $\xi(p)$ entlang $\gamma$ verschieben. D.\,h., im Punkt $q$ haben wir die Vektoren $\xi(q)$ und $\tau^q_p\xi(p)$, wobei $\tau^q_p\xi(p)$ der Paralleltransport von $\xi(p)$ nach $q$ entlang $\gamma$ ist.

\paragraph{Idee}
Betrachte
\[ \xi(q) - \tau^q_p\xi(p) \in T_pM  \]
für $\d(p,q) \pfeil{} 0$. Dies bezeichnet man dann auch als die \df{kovariante Ableitung} von $\xi$ in Richtung $v$
\[ \nabla_v\xi \in T_qM \]
$\nabla_v$ nennt man dabei einen \df{Zusammenhang.} Diese hat folgende Eigenschaften:
\begin{itemize}
	\item $\nabla_v$ ist $\Omega^0(M)$-linear in $v$, d.\,h.
	\[ \nabla_{\lambda v + w}(\xi) = \lambda \nabla_{v} (\xi) + \nabla_{w} (\xi) \]
	für glatte Funktionen $\lambda : M \pfeil{} \R$.
	\item Sie ist $\R$-linear im zweiten Argument
	\[ \nabla_{v}(\xi + \eta) = \nabla_{v}(\xi) + \nabla_{v}(\eta) \]
	\item Ist $f : M \pfeil{} \R$ linear, so liegt folgende Produktregel vor
	\[ \nabla_{v}(f\cdot \xi) = f \cdot \nabla_{v}(\xi) + \nabla_{v}(f) \cdot \xi \]
	wobei
	\[ \nabla_{v} f := v(f) \]
\end{itemize}


\subsection{Geodätische}
Sei $\gamma$ eine (glatte) Kurve auf $M$. $\gamma$ heißt eine \df{Geodätische}, falls gilt
\[ \nabla_{\dot{\gamma}}\dot{\gamma} = 0 \]
Obige Bedingung ist in lokalen Koordinaten eine Differentialgleichung zweiter Ordnung.\\
Physikalisch gesprochen verschwindet die Beschleunigung. Geometrisch gesprochen ist $\gamma$ parallel entlang $\gamma$.

\Bsp{}
Sei $M$ eine Riemannsche Fläche im $\R^3$. Die Gleichung
\[ \nabla_{\dot{\gamma}}\dot{\gamma} = 0 \]
bedeutet
\[ \ddot{\gamma} \bot M \]
D.\,h., die Euklidische zweite Ableitung steht orthogonal auf der Fläche $M$.

\Bsp{}
\begin{itemize}
	\item Geraden sind Geodätische im Euklidischen Raum.
	\item Großkreise sind Geodätische auf Sphären.
	\item Allgemein sind Geodätische lokal kürzeste Kurven.
\end{itemize}

\newpage
\subsection{Parallelogramme}
Sei $p\in M$. $\mu, \lambda$ seien zwei Geodätische, die sich im Punkt $p$ schneiden mit $\mu(0) = \lambda(0) = p$.\\
$\mu, \lambda$ seien parametrisiert durch die Bogenlänge, d.\,h.,
\[ \norm{\dot{\lambda}(t)} = \norm{\dot{\mu}(t)} = 1 \]
für alle $t$. Setze $v := \dot{\mu}(0)$ und $w := \dot{\lambda}(0)$. Sei $\e > 0$.\\
Indem wir $w$ entlang $\mu$ verschieben, erhalten wir einen Vektor $\overline{w}$ auf $\mu(\e)$ und analog einen Vektor $\overline{v}$ auf $\lambda(\e)$.\\
Es gilt
\[ \norm{\overline{v}} = \norm{\overline{w}} = 1 \]
da der Paralleltransport eine Isometrie ist, wenn die Riemannsche Metrik kompatibel ist zum Zusammenhang $\nabla$.\\
Indem man $\overline{v}$ und $\overline{w}$ durch durch Bogenlänge parametrisierte Geodätische fortsetzt, erhält man Geodätische $\overline{\mu}$ und $\overline{\lambda}$. Dadurch erhält man dann Punkte $\overline{\lambda}(\e) $ und $\overline{\mu}(\e)$. Im Euklidischen würden die beiden Punkte zusammen fallen und das Parallelogramm schließen. Für allgemeine Riemannsche Mannigfaltigkeiten muss dies nicht der Fall sein, aber es gilt
\[ d(\overline{\mu}(\e), \overline{\lambda}(\e)) \in O(\e^2) \]
\marginpar{Vorlesung vom 18.04.18}
\Def{}
Definiere die \df{Torsion} des Zusammenhangs durch
\[ T(\xi, \eta) := \nabla_\xi \eta - \nabla_\eta \xi - [\xi, \eta] \]
wobei $[\xi, \eta]$ die \df{Lie-Klammer}\footnote{Lassen sich die beiden Vektorfelder als Koordinatenrichtungen schreiben, so gilt zum Beispiel $[\frac{\partial}{\partial x_i}, \frac{\partial}{\partial x_j}] = 0$} der beiden Vektorfelder $\xi$ und $\eta$ bezeichnet.\\
$T$ ist ein \df{Tensor}, d.\,h., $\mathcal{C}^{\infty}(M)$-linear.\\
$\nabla$ heißt \df{symmetrisch} bzw. \df{torsionsfrei}, falls $T = 0$.

\newcommand{\crv}{\text{R}}

\Lem{}
Ist $\nabla$ symmetrisch, dann gilt sogar
\[ d(\overline{\mu}(\e), \overline{\lambda}(\e)) \in O(\e^3) \]

Sei $u \in T_p(M)$ ein weiterer Tangentialvektor. $u_1$ sei der Paralleltransport von $u$ entlang $\lambda\overline{\mu}$. $u_2$ sei der Paralleltransport entlang $\mu \overline{\lambda}$.\\
Es liegt dann folgende asymptotische Gleichheit vor
\[ \norm{u_1 - u_2} \sim \epsilon^2 \crv(v,w)u  \]
$\crv(v,w)u$ heißt \df{Riemannscher Krümmungstensor}. Er ist definiert durch
\[ \crv(v,w)u := \nabla_v \nabla_wu - \nabla_w \nabla_v u - \nabla_{[v,w]} u \]

Wir werden nun im Folgenden mit den Formalen Definitionen beginnen.

\newpage
\section{Die Lie-Klammer}
Sei $M$ im Folgenden eine glatte $n$-dimensionale Mannigfaltigkeit und $X,Y : M \pfeil{} \T M$ glatte Vektorfelder auf $M$.

\newcommand{\CC}[1]{\mathcal{C}^{#1}}

\Lem{}
Es existiert genau ein glattes Vektorfeld $Z$ auf $M$, sodass gilt
\[Z(f) = X(Y(f)) - Y(X(f)) \]
für alle $f \in \CC{\infty}(M)$. Beachte, $X(f)$ bezeichnet die glatte Funktion, die sich ergibt durch
\[ X(f)(p) := X(p)(f) \]

\begin{Beweis}{}
\begin{itemize}
	\item Eindeutigkeit:\\
	Sei $p \in M$. $\{x_i\}$ seien lokale Koordinaten bei $p$. $X, Y$ lassen sich dann schreiben durch
	\begin{align*}
	X = \sum_i a_i \frac{\partial}{\partial x_i} && \text{ und } && Y = \sum_j b_j \frac{\partial}{\partial x_j}
	\end{align*}
	und es gilt
	\begin{align*}
	X(Yf) &= X\klam{\sum_j b_j \frac{\partial f}{\partial x_j}}\\
	&= \sum_i a_i \sum_j \frac{\partial }{\partial x_i} \klam{b_j \frac{\partial f}{\partial x_j}}\\
	&= \sum_{i,j} a_i \frac{\partial b_j}{\partial x_i} \frac{\partial f}{\partial x_j} + 
	\sum_{i,j} a_i b_j \frac{\partial^2 f}{\partial x_j\partial x_i} 
	\end{align*}
	bzw.
	\begin{align*}
		Y(Xf) =  \sum_{i,j} b_j \frac{\partial a_i}{\partial x_j} \frac{\partial f}{\partial x_i} + 
	\sum_{i,j} b_j a_i \frac{\partial^2 f}{\partial x_i\partial x_j} \\
	\end{align*}
	In der Differenz ergibt sich
	\begin{align*}
	X(Yf) - Y(Xf) &= \sum_{i,j} a_i \frac{\partial b_j}{\partial x_i} \frac{\partial f}{\partial x_j}
	- \sum_{i,j} b_i \frac{\partial a_j}{\partial x_i} \frac{\partial f}{\partial x_j}\\
	&= \sum_{i,j} \klam{a_i \frac{\partial b_j}{\partial x_i} - b_i \frac{\partial a_j}{\partial x_i} } \frac{\partial f}{\partial x_i}
	\end{align*}
	Lokal ist $Z$ also bestimmt durch
	\[ Z =  \sum_{i,j} \klam{a_i \frac{\partial b_j}{\partial x_i} - b_j \frac{\partial a_j}{\partial x_i} } \frac{\partial }{\partial x_j} \]
	\item Existenz:\\
	Durch obige Formel ist für jedes lokale Koordinatensystem ein $Z$ gegeben. Diese lassen sich global zu einem glatten Vektorfeld auf ganz $M$ zusammen setzen.
\end{itemize}
\end{Beweis}

\Def{}
Definiere nun die \df{Lie-Klammer} von $X$ und $Y$ durch
\[ Z:= [X,Y] = XY - YX\]

\Bem{}
Die Lie-Klammer hat folgende Eigenschaften
\begin{itemize}
	\item $[X,Y] = - [Y,X]$
	\item Für $a,b \in \R$ gilt
	\[ [aX_1 + bX_2, Y] = a[X_1, Y] + b[X_2, Y] \]
	\item Iteration: Für beliebige Vektorfelder $X,Y,Z$ gilt
	\[ [[X,Y], Z] = [ XY - YX, Z ] = XYZ- YXZ - ZXY + ZYX \]
	und
	\[ [[Y,Z], X] = [ YZ - ZY, X ] = YZX- ZYX - XYZ + XZY \]
	und
	\[ [[Z,X], Y] = [ ZX - XZ, Y ] = ZXY - XZY - YZX + YXZ \]
	Durch Aufsummieren ergibt sich
	\[ [[X,Y], Z] + [[Y,Z], X]  + [[Z,X], Y] = 0\]
	Dies nennt sich die \df{Jacobi-Identität}.
	\item Seien $f,g \in \CC{\infty}(M)$. Es gilt
	\[ [fX, gY] = fX(gY) - gY(fX) = f(X(g)Y - gXY) - g(Y(f)X - fYX) = fg[X,Y] + fX(g) Y - gY(f)X \]
\end{itemize}

\newcommand{\pf}[2]{\frac{\partial #1}{\partial #2}}
%\vspace{12mm}
Da eine Mannigfaltigkeit lokal wie $\R^n$ aussieht, lassen sich die bekannten Sätze zu Existenz, Eindeutigkeit und Abhängigkeit von Anfangsbedingungen von gewöhnlichen Differentialgleichungen von $\R^n$ auf $M$ verallgemeinern.

\Satz{}
Sei $M$ eine glatte Mannigfaltigkeit, $X$ ein glattes Vektorfeld auf $M$, $p \in M$ ein Punkt.\\
Dann existiert eine offene Umgebung $U \subset M$ von $p$ und ein $\delta > 0$ zusammen mit einer Abbildung
\[ \phi : (-\delta, \delta) \times U \Pfeil{} M \]
sodass $t \mapsto \phi(t,p)$ die eindeutige Lösung von
\begin{align*}
\pf{}{t} \phi(t,q) &= X(\phi(t,q)) && \forall q \in U\\
\phi(0,q) &= q
\end{align*}
ist.\\
Schreibweise:
\[ \phi_t(p) := \phi(t,p) \]
Die glatte Abbildung
\[ \phi_t : U\pfeil{} M \]
heißt \df{Fluss} von $X$ (in der Umgebung von $p$).

\Bem{}
Sei $\bet{s}, \bet{t}, \bet{s+t} < \delta$. Betrachte
\[ \gamma_1(t) := \phi(t, \phi(s,p)) \]
Das impliziert
\begin{align*}
\dot{\gamma_1} = X(\gamma_1) && \gamma_1(0) = \phi(s,p)
\end{align*}
und
\[ \gamma_1(t) := \phi(t+s, p) \]
impliziert
\begin{align*}
\dot{\gamma_2} = X(\gamma_2) && \gamma_2(0) = \phi(s,p)
\end{align*}
Aus der Eindeutigkeit folgt nun
\[ \gamma_1 = \gamma_2 \]
D.\,h.,
\[ \phi_{s+t} = \phi_s \circ \phi_t \]
Insbesondere gilt
\[ \id{M} = \phi_t\circ \phi_{-t} \]
Daraus folgt, dass jedes $\phi_t$ ein Diffeomorphismus ist. Die Menge aller $\{\phi_t\}_{t}$ nennt man eine \df{Einparameter-Untergruppe} von Diffeomorphismen.

\newpage
\section{Die Lie-Ableitung}
Seien $X,Y$ zwei Vektorfelder auf $M$, $p \in M$ ein Punkt.\\
Sei $\phi_t$ der Fluss auf $X$ mit
\begin{align*}
\pf{}{t} \phi(t,p) = X(\phi_t(p)) && \text{ und } && \phi_0(p) = p
\end{align*}
Definiere nun die \df{Lie-Ableitung durch}
\begin{align*}
(L_XY)(p):=\lim\limits_{h\pfeil{} 0} \frac{1}{h} \klam{
Y_p 
- (\d \phi_h)(Y_{\phi_{-h}(p)})
}\in T_p(M)
\end{align*}
wobei $Y_p = Y(p), \d \phi_h = \phi_{h,*}$. Die Lie-Ableitung leitet das Vektorfeld $Y$ bzgl. dem Fluss von $X$ im Punkt $p$ ab.

\Prop{}
\label{LieKlammerProp}
Es gilt
\[ L_XY = [X,Y] \]
Für den Beweis dieser Proposition benötigen wir ein Lemma:
\paragraph{Idee}
Sei $f: \R \pfeil{} \R$ glatt mit $f(0) = 0$. $f$ hat die Taylor-Entwicklung
\[ f(t) = t f'(0) + \frac{t^2}{2} f''(0) + \ldots =: t \cdot g(t) \]
Es gilt
\[ f(t) = tg(t) \]
und $f'(0) = g(0)$.\\
Wir brauchen nun folgende Verallgemeinerung dieser Beobachtung:

\Lem{}
Sei $M$ eine Mannigfaltigkeit, $f : (-\e, \e) \times M \pfeil{} \R$ glatt, $f(0,p) = 0$ für alle $p \in M$.\\
Dann existiert eine glatte Funktion $g: (-\e, \e) \times M \pfeil{} \R$ mit
\begin{align*}
f(t,p) = t\cdot g(t,p) && \text{ und } && \pf{f}{t}(0,p) = g(0,p)
\end{align*}
\begin{Beweis}{}
Wir definieren $g$ durch
\begin{align*}
g(t,p) := \int_{0}^{1} \klam{\pf{f}{s}} (s\cdot t,p) \d s
\end{align*}
Der Rest ist nachrechnen.
\end{Beweis}

\begin{Beweis}{\ref{LieKlammerProp}}
Sei $f \in \CC{\infty}(M)$. Definiere die Hilfsfunktion
\begin{align*}
h(t,p) := f(\phi_t(p)) - f(p)
\end{align*}
Aufgrund des Lemmas existiert ein $g$ mit
\begin{align*}
h(t,p) = t \cdot g(t,p) && \pf{h}{t}(0,p) = g(0,p)
\end{align*}
Es gilt
\begin{align*}
f \circ \phi_t = f + t g_t
\end{align*}
\end{Beweis}
\marginpar{Vorlesung vom 23.04.18}
\begin{Beweis}{\ref{LieKlammerProp}}
	Sei $f \in \CC{\infty}(M)$. Wir wollen Folgendes zeigen.
	\[ (L_XY)(f) = [X,Y](f) = XYf - YXf \]
	Definiere die Hilfsfunktion
	\begin{align*}
	h(t,p) := f(\phi_t(p)) - f(p).
	\end{align*}
	Da $h(0,p) = 0$, existiert aufgrund des Lemmas ein $g$ mit
	\begin{align*}
	h(t,p) = t \cdot g(t,p) &&\text{ und }&& \pf{h}{t}(0,p) = g(0,p).
	\end{align*}
	Es gilt
	\begin{align*}
	f \circ \phi_t = f + t g_t
	\end{align*}
	und
	\begin{align*}
	X_p(f) = \klam{\pf{}{t}_{t= 0}\phi_t(p)}(f) = \pf{}{t}_{t = 0} f(\phi_t(p)) = \pf{h}{t} (0,p) = g(0,p).
	\end{align*}
	Durch die erste der beiden obigen Gleichung erhalten wir
	\begin{align*}
	(\d \phi_h)(Y_{\phi_{-h}(p)})(f) &= Y_{\phi_{-h}(p)}(f\circ \phi_h)\\
	&= Y_{\phi_{-h}(p)} (f+tg_t).
	\end{align*}
	Setzt man dies in die Lie-Ableitung ein, so erhält man
	\begin{align*}
	(L_XY)(f) &= \lim\limits_{h\pfeil{} 0}\frac{1}{h}\klam{ Y_p - (\d \phi_h)(Y_{\phi_{-h}(p)})(f) }\\
	&= \lim\limits_{h\pfeil{} 0}\frac{1}{h}\klam{ Y_p - (Y_{\phi_{-h}(p)})(f) } - \lim\limits_{h\pfeil{} 0}\frac{1}{h}\klam{h (Y_{\phi_{-h}(p)})(g_h) }.
	\end{align*}
	Da gilt
	\[ \lim\limits_{h\pfeil{} 0}\frac{1}{h}\klam{h (Y_{\phi_{-h}(p)})(g_h) } =
	\lim\limits_{h\pfeil{} 0} (Y_{\phi_{-h}(p)})(g_h)  = Y_p(g_0) =YXf,  \]
	folgt
		\begin{align*}
	(L_XY)(f)&= \lim\limits_{h\pfeil{} 0}\frac{1}{h}\klam{ (Yf)_p - (Yf)_{\phi_{-h}(p)} } - Y_pXf\\
	&= X_pYf - Y_pXf.
	\end{align*}
\end{Beweis}
\paragraph{Folgerungen}
\begin{align*}
L_YX = -L_XY, && L_XX = 0
\end{align*}\\\\
Seien Vektorfelder $X,Y$ gegeben. Man kann zeigen, dass lokale Koordinaten $x_1,\ldots, x_n$ existieren mit
\[ X = \pf{}{x_1}. \]
Gilt ferner
\[ Y = \pf{}{x_2}, \]
so folgt
\[ [X,Y] = \pf{\partial}{x_1\partial x_2} - \pf{\partial}{x_2\partial x_1} = 0.  \]
Insofern ist das Verschwinden von $[X,Y]$ eine notwendige Bedingung für die Existenz von lokalen Koordinaten $x_1,\ldots, x_n$ mit
\begin{align*}
X = \pf{}{x_1} && \text{ und } && Y = \pf{}{x_2}.
\end{align*}

\subsection{Geometrische Interpretation der Lie-Klammer}
Seien $X,Y$ Vektorfelder. $\phi$ und $\psi$ seien korrespondierende Flüsse, $p\in M$ sei ein Punkt. Setze
\[ c(h) := \psi_{-h}\phi_{-h}\psi_h\phi_h(p). \]
Die Zuordnung $h \mapsto c(h)$ definiert eine glatte Kurve. Man kann zeigen
\[ \dot{c}(h) = 0. \]
Für Kurven $\gamma(t)$ mit $\dot{\gamma}(0) = 0$ lässt sich die zweite Ableitung definieren durch
\[ \ddot{\gamma}(t)(0):= \frac{\d^2}{\d t^2}_{t = 0} f(\gamma(t)) .\]
Dann ist $\ddot{\gamma}(0)$ eine Derivation.\\
Daraus folgt, dass $\ddot{c}(0)$ definiert ist, und es gilt
\begin{align*}
\ddot{c}(0) = 2[X,Y]_p.
\end{align*}

\newpage
\section{Riemannsche Mannigfaltigkeiten}
Sei $M$ eine glatte, $n$-dimensionale Mannigfaltigkeit.
\Def{}
Eine \df{Riemannsche Metrik} auf $M$ ist eine Zuordnung
\begin{align*}
p \longmapsto \shrp{\cdot ~|~\cdot}_p
\end{align*}
für $p\in M$, wobei $\shrp{\cdot ~|~\cdot}_p$ jeweils ein inneres Produkt\footnote{Inneres Produkt heißt hier eine symmetrische, positiv definite Bilinearform.} auf $T_pM$ ist. Ferner soll diese Zuordnung\df{glatt} sein in dem Sinne, dass für lokale Koordinaten $(U,x)$ die Funktionen
\begin{align*}
g_{i,j}(p) := \shrp{\pf{}{x_i}(p) ~|~ \pf{}{x_j}(p) }_p
\end{align*} 
für alle $i,j$ glatt sind auf $U$.\\
Wir werden manchmal $g(p)$ anstatt $\shrp{\cdot ~|~ \cdot}_p$ schreiben.\\
Das Paar $(M, \shrp{\cdot~|~ \cdot})$ heißt \df{Riemannsche Mannigfaltigkeit}.

\Def{}
Ein Diffeomorphismus $\phi: (M, \shrp{\cdot~|~\cdot}_M) \pfeil{} (N, \shrp{\cdot~|~\cdot}_N)$ heißt \df{Isometrie}, falls für alle $p \in M$ und $u,v \in T_pM$ gilt
\[ \shrp{u,v}_{M,p} = \shrp{ \d \phi_pu, \d \phi_pv}_{N,\phi(p)}. \]


\begin{enumerate}[(1)]
	\item \Bsp{} Sei $M = \R^n$. $x$ seien die Standardkoordinaten auf $\R^n$. Setzt man
	\[ \shrp{\pf{}{x_i}, \pf{}{x_j}}_p = \delta_{i,j} \]
	so erhält man die euklidische Metrik auf $\R^n$.
	\item Sei $f:M\pfeil{} N$ eine glatte Immersion. $(N,\shrp{\cdot~|~\cdot}_N)$ sei eine Riemannsche Mannigfaltigkeit, $M$ eine glatte Mannigfaltigkeit. Dann induziert $f$ eine Riemannsche Metrik $\shrp{\cdot ~|~\cdot}_M$ auf $M$ durch
	\[ \shrp{u|v}_M:= \shrp{\d f(u), \d f(v)}_N. \]
	Da $\d f$ injektiv ist, ist $ \shrp{u|v}_{M,p}$ positiv definit.
	\item \Bsp{} Es bezeichne $S^n = \set{(x_1,\ldots,x_{n+1}) \in \R^{n+1}}{x_1^2 + \ldots + x_{n+1}^2 = 1}$ die Einheitssphäre. Durch die Einebettung $S^n \subset \R^{n+1}$ erhalten wir eine Riemannsche Metrik auf $S^n$. $S^n$ zusammen mit dieser Metrik nennt man \df{Standardsphäre}.
	\item \textbf{Produktmetrik}: Seien $(M, g_m), (N,g_N)$ zwei Riemannsche Mannigfaltigkeiten. $\pi_1, \pi_2$ seien die korrespondierenden Projektionen von $M\times N$ auf $M$ bzw. $N$. Seien $u,v \in T_{(p,q)}(M\times N)$, setze
	\begin{align*}
	\shrp{u,v}_{p,q} := \shrp{\d \pi_1(u), \d \pi_1(v)}_{M,p} + \shrp{\d \pi_2(u), \d \pi_2(v)}_{N,q}. 
	\end{align*}
	$\shrp{u,v}_{p,q}$ ist eine Riemannsche Metrik auf $M \times N$, die sogenannte \df{Produktmetrik}.
	\item \Bsp{} Betrachte $T^n := S^1 \times \ldots \times S^1$. Ist $S^1$ mit der Standardmetrik versehen, so induziert uns dies eine Produktmetrik auf $T^n$. In diesem Fall spricht man vom \df{flachen Torus}.\\
	Für $n=2$ kann man $T^2$ in den $\R^3$ einbetten. Dadurch erhält man eine andere induzierte Metrik auf $T^2$, die nicht äquivalent zu obiger Produktmetrik ist. Diese beiden Tori sind nicht isometrisch.
\end{enumerate}

\Prop{}
Jede glatte Mannigfaltigkeit besitzt eine Riemannsche Metrik.

\begin{Beweis}{}
Sei $\set{(U_\alpha,x_\alpha)}{}$ eine offene Überdeckung von $M$ durch Karten und $\{f_\alpha\}$ eine glatte Partition der Eins bzgl. dieser Überdeckung.\\
Über $U_\alpha$ betrachte man die eindeutige Riemannsche Metrik $g^\alpha$, sodass
\[ (U_\alpha,g^\alpha) \Pfeil{x_\alpha} (\R^n, g_{eukl}) \]
eine Isometrie ist. Auf $M$ erhält man nun eine Riemannsche Metrik durch
\begin{align*}
g_p := \sum_{p\in U_\alpha} f_\alpha(p) g_p^\alpha.
\end{align*}
\end{Beweis}

\Def{}
Sei $c : \R \pfeil{} M$ eine glatte Kurve. Ein \df{Vektorfeld entlang einer Kurve} $c$ ist eine glatte Zuordnung
\[ t \longmapsto V(t) \in T_{c(t)}M \]

\Bem{}
Ein Vektorfeld entlang einer Kurve lässt sich im Allgemeinem nicht auf ein Vektorfeld einer offenen Umgebung der Kurve fortsetzen. Zum Beispiel könnte sich die Kurve selbst schneiden und $V$ die Ableitung der Kurve sein.

\paragraph{Notation}
Wir schreiben auch für $v \in T_pM$
\[ \norm{v} := \sqrt{ \shrp{v|v}_p } \]

\Def{}
Für eine Kurve $c$ definiere wir die \df{Länge} durch
\[ L^b_a(c) := \int_{a}^{b} \norm{\dot{c}(t)} \d t \]
\marginpar{Vorlesung vom 25.04.18}

Sei $(M,g)$ eine Riemannsche Mannigfaltigkeit, die obendrein orientiert ist. $(U,x)$ und $(V,y)$ seien orientierte Karten auf $M$, die sich schneiden.\\
Wir erinnern an folgendes Lemma aus Differentialtopologie I.
\Lem{}
Auf $U\cap V$ gilt
\[ f\d x_1 \wedge \ldots \wedge \d x_n = g \d y_1 \wedge \ldots \wedge \d y_n \]
genau dann, wenn
\[ f = \det\klam{ \pf{y_i}{x_j} } g \]
gilt.\\\\
Auf einer orientierten Karte $U$ sind Funktionen $g_{i,j} : U \pfeil{} \R$ gegeben durch
\[ g_{i,j} = \shrp{\pf{}{x_i}|\pf{}{x_j} } \]
für $p \in U$. Setze ferner $X_i := \pf{}{x_i}$.\\
Sei $e_1,\ldots, e_n$ eine {Orthonormalbasis} (ONB) für $T_pM$ bzgl. $g_p$. Dann lässt sich $X_i$ darstellen durch
\[ X_i = \sum_j a_{i,j}e_j. \]
Wir erhalten so eine $n\times n$-Matrix $A := (a_{i,j})_{i,j}$. Es gilt
\begin{align*}
g_{i,j} :=& \shrp{X_i|X_j}\\
=& \shrp{ \sum_k a_{i,k}e_k| \sum_l a_{j,l}e_l  }\\
=& \sum_{k,l} a_{i,k}a_{j,l} \shrp{e_k|e_l}\\
=& \sum_k a_{i,k}a_{j,k}.
\end{align*}
Daraus folgt
\[ (g_{i,j})_{i,j} = AA^T. \]
Dies impliziert insbesondere
\[ \det(g_{i,j}) = \det(A)^2 > 0. \]
Insbesondere ist $\sqrt{\det(g_{i,j})} = \bet{\det A}$ wohldefiniert. Durch den Transformationssatz folgt nun im Punkt $p$
\begin{align*}
vol(X_1, \ldots, X_n) = \bet{\det A} \cdot vol(e_1, \ldots, e_n) = \bet{\det A},
\end{align*}
da $vol(e_1, \ldots, e_n) = 1$. Daraus folgt insbesondere
\[ vol(\pf{}{x_1}, \ldots, \pf{}{x_n}) = \sqrt{\det (g_{i,j})}. \]
Auf $(V,y)$ erhält man analog
\[ vol(Y_1, \ldots, Y_n) = \sqrt{\det (h_{i,j})}. \]
für
\[ Y_i = \pf{}{y_i} \]
und
\[ h_{i,j} = \shrp{Y_i|Y_j}. \]
Man erhält hierdurch
\begin{align*}
\sqrt{\det (h_{i,j})} &= vol(Y_1, \ldots, Y_n)\\
&= \det \klam{ \pf{x_i}{y_j} } vol(X_1, \ldots, X_n)\\
&=  \det \klam{ \pf{x_i}{y_j} } \sqrt{\det (g_{i,j})}.
\end{align*}
Mit dem obigen Lemma folgt nun auf $U\cap V$
\[ \sqrt{\det (g_{i,j})} \d x_1\wedge \ldots \wedge \d x_n = \sqrt{\det (h_{i,j})} \d y_1\wedge \ldots \wedge \d y_n.  \]
Durch Verkleben erhalten wir eine glatte $n$-Form $\nu \in \Omega^n(M)$.

\Def{}
$\nu$ heißt \df{Riemannsche Volumenform} von $M$. $\nu$ ist durch die Riemannsche Metrik eindeutig festgelegt.

\Def{}
Wenn $M$ kompakt ist, setzen wir
\[ vol(M) := \int_M \nu < \infty. \]
$vol(M)$ heißt das \df{Riemannsche Volumen}.\\
Wenn $vol(K)$ unbeschränkt ist über kompakte Untermannigfaltigkeiten (mit Rand) $K \subset M$, dann sagen wir, dass $M$ unendliches Volumen habe.

\Bem{}
Oft sieht man in der Literatur $\nu = \d V = \d vol$, obwohl $\nu$ im Allgemeinem nicht im Bild des Randhomomorphismus
\[ \d : \Omega^{n-1}(M) \Pfeil{} \Omega^n(M) \]
liegt.

\newpage
\section{Zusammenhänge}
Sei $\Gamma(\T M)$ der Vektorraum der glatten Schnitte von $\T M$, d.\,h., $\Gamma(\T M)$ ist der Vektorraum der glatten Tangentialvektorfelder auf $M$.

\Def{}
Ein \df{Zusammenhang} auf $M$ ist eine Abbildung
\begin{align*}
\nabla : \Gamma(\T M) \times \Gamma(\T M ) &\Pfeil{} \Gamma(\T M)\\
(X,Y) & \longmapsto \nabla_XY ,
\end{align*}
sodass:
\begin{enumerate}[(1)]
	\item Für $f,g \in \CC{\infty}(M)$ gilt
	\[\nabla_{fX_1 + gX_2}Y = f\nabla_{X_1}Y + g \nabla_{X_2}Y.\]
	\item Ferner gilt
	\[ \nabla_X(Y_1 + Y_2) = \nabla_X Y_1 + \nabla_X Y_2. \]
	\item Zuletzt wird folgende Produktregel gefordert
	\[ \nabla_X(f\cdot Y) = f\nabla_XY + X(f)Y. \]
\end{enumerate}

\paragraph{In Lokalen Koordinaten} Sei $(U,x)$ eine Karte. $X,Y$ seien Vektorfelder der Gestalt
\begin{align*}
X = \sum_ia_i \pf{}{x_i}, && Y = \sum_j b_j\pf{}{y_j}.
\end{align*}
Es gilt
\begin{align*}
\nabla_XY &= \sum_ia_i \nabla_{\pf{}{x_i}} (\sum_j \pf{}{y_j})\\
&= \sum_ia_i\sum_j \nabla_{\pf{}{x_i}} (b_j \pf{}{y_j})\\
&= \sum_{i,j} a_i 
\klam{
\pf{b_j}{x_i} \pf{}{x_j} + b_j \nabla_{\pf{}{x_i}} \pf{}{y_j} 
}.
\end{align*}
Wir dröseln die Terme $\nabla_{\pf{}{x_i}} \pf{}{y_j}$ weiter auf und erhalten
\begin{align*}
\nabla_{\pf{}{x_i}} \pf{}{y_j}  = \sum_k \Gamma_{i,j}^k \pf{}{x_k}.
\end{align*}
Die Funktionen $\Gamma_{i,j}^k$ nennt man \df{Christoffel-Symbole} des Zusammenhangs.\\
Wir erhalten final
\begin{align*}
\nabla_XY &= \sum_{i,k} a_i 
\klam{
\pf{b_k}{x_i} \pf{}{x_k} +\sum_{i,j} a_i b_j \sum_k \Gamma_{i,j}^k \pf{}{x_k} 
}\\
&= \sum_k \klam{
\sum_i  a_i \pf{b_k}{x_i}
+
\sum_{i,j} a_i b_j \Gamma_{i,j}^k
}
\pf{}{x_k}.
\end{align*}

\Bem{}
Die Gleichung
\begin{align*}
\nabla_XY &= \sum_k \klam{
	\sum_i  a_i \pf{b_k}{x_i}
	+
	\sum_{i,j} a_i b_j \Gamma_{i,j}^k
}
\pf{}{x_k}
\end{align*}
impliziert, dass $\nabla_X Y$ eine lokale Operation ist. Denn für $p \in M$ gilt
\begin{align*}
(\nabla_XY)(p) &= \sum_k \klam{
\sum_i a_i(p) \pf{b_k}{x_i}(p)
+ \sum_{i,j} a_i(p) b_j(p) \Gamma_{i,j}^k(p)
}\pf{}{x_k}_{|p}.
\end{align*}
D.\,h., $(\nabla_XY)(p)$ hängt nur von $X(p), Y(p)$ und $\pf{b_k}{x_i}(p)$ ab.

\Def{}
Sei $V = V(t)$ ein Vektorfeld entlang einer Kurve $c(t)$ in $M$.\\
Eine \df{kovariante Ableitung} ist eine Zuordnung
\[ \Dd{t} : \V_c \Pfeil{} \V_c, \]
wobei $\V_c$ den Raum aller Vektorfelder entlang $c$ bezeichnet, sodass
\begin{enumerate}[(1)]
	\item $\Dd{t}(V+W) = \Dd{t}V + \Dd{t} W$
	\item und $\Dd{t}(fV) = f \Dd{t} V + \pf{f}{t} V$ für $f : \R \pfeil{}\R$ glatt gelten.
	\item Wenn ferner ein Vektorfeld $X$ auf $M$ existiert mit $X(c(t)) = V(t)$, dann soll gelten
	\[ \nabla_{\dot{c}}X = \Dd{t}(V) .\]
\end{enumerate}

\Prop{}
Sei $M$ eine glatte Mannigfaltigkeit mit Zusammenhang $\nabla$. Sei $c$ eine Kurve auf $M$. Dann existiert eindeutig eine kovariante Ableitung $\Dd{t}$ mit obigen Eigenschaften.
\begin{Beweis}{}
\begin{itemize}
	\item Eindeutigkeit: Sei $V(t)$ ein Vektorfeld entlang $c(t)$. In lokalen Koordinaten $x_1, \ldots, x_n$:
	\begin{align*}
	V(t) = \sum_i v_i(t) \pf{}{x_i}, && c(t) = (x_1(t), \ldots, x_n(t))
	\end{align*}
	Es gilt dann
	\begin{align*}
	\Dd{t} V &= \sum_i \klam{
v_i \Dd{t} \klam{
\pf{}{x_i}_{|c(t)}
}	+ v_i' \pf{}{x_i}
}\\
&= \sum_i \klam{
v_i \nabla_{\dot{c}(t)} \pf{}{x_i} 
+ v_i'\pf{}{x_i}
}.
	\end{align*}
	
	\item Existenz: Sei $(U_\alpha, x^\alpha)$ eine offene Überdeckung von $M$ durch Karten. Definiere $\Dd{t}$ auf $U_\alpha$ durch
	\begin{align*}
	\Dd{t} V := \sum_i \klam{
		v_i \nabla_{\dot{c}(t)} \pf{}{x_i} 
		+ v_i'\pf{}{x_i}
	}.
	\end{align*}
	Auf $U_\alpha \cap U_\beta$ stimmen diese $\Dd{t}$ überein wegen Eindeutigkeit und definieren somit $\Dd{t}$ überall.
\end{itemize}
\end{Beweis}

\Prop{}
Sei $c$ eine Kurve in $M$, $p = c(0)$. Sei ferner $V^0 \in T_pM$ ein Tangentialvektor. Dann existiert genau ein Vektorfeld $V$ entlang $c$ mit
\[ \Dd{t}V = 0 \]
und
\[ V(0) = V^0. \]

\Def{}
Sei $V$ ein Vektorfeld entlang einer Kurve $c$. $V$ heißt \df{parallel} entlang $c$, falls
\[ \Dd{t}V =0. \]


\marginpar{Vorlesung vom 30.04.18}

\Prop{}
Sei $c$ eine Kurve in $M$ und $V^0\in T_{c(t_0)}M$ ein Vektor bei $c(t_0)$.
Dann existiert genau ein Vektorfeld $V(t)$ entlang $c(t)$, das die Eigenschaften
\begin{align*}
V(t_0) &= V^0\\
\Dd{t} V &= 0
\end{align*}
erfüllt.
\begin{Beweis}{}
\begin{itemize}
	\item Existenz und Eindeutigkeit in lokalen Koordinaten:\\
	Existiert so ein $V$, so gilt
	\begin{align*}
	0 = \Dd{t} V=\sum_k (v'_k + \sum_{i,j} x_i'v_j \Gamma_{i,j}^k) \pf{}{x_k}.
	\end{align*}
	Daraus folgt
	\[ v'_k = - \sum_{i,j} (x_i' \Gamma_{i,j}^k) v_j \]
	für alle $k = 1,\ldots, n$. Dadurch ergibt sich ein System von linearen gewöhnlichen Differentialgleichungen. Aus der Theorie der gewöhnlichen Differentialgleichungen wissen wir, dass es in kleinen Umgebungen von $t$ eindeutige Lösungen für $v_k(t)$ gibt für alle $t$. Da obiges DGL linear ist, sind die $v_k(t)$ für alle $t\in \R$ definiert.
	\item Globale Existenz:\\
	Sei $t_1 > t_0$ beliebig. Der Kurvenabschnitt $c[t_0, t_1]$ ist kompakt und wird folglich überdeckt durch endlich viele Karten. Man kann nun eine lokale Lösung von Karte zu Karte fortsetzen. Die lokalen Lösungen stimmen auf den Durchschnitten der Karten überein wegen ihrer Eindeutigkeit.
\end{itemize}
\end{Beweis}

\Bem{}
\begin{enumerate}[1.)]
	\item Wir erhalten folgende Abbildung
	\begin{align*}
	\tau : T_{c(t_0)}M & \Pfeil{} T_{c(t_1)}M\\
	V^0 & \longmapsto V(t_1).
	\end{align*}
	Diese Abbildung nennt man den \df{Paralleltransport} von $c(t_0)$ nach $c(t_1)$ entlang $c$.\\
	Die Linearität des vorangegangenen Differentialgleichungssystems stellt die Linearität von $\tau$ sicher. Durch Umkehren der Zeit erhält man eine lineare Abbildung
	\begin{align*}
	\tau' : T_{c(t_1)}M & \Pfeil{} T_{c(t_0)}M.
	\end{align*}
	Naheliegenderweise gilt
	\[ \tau' = \tau\i .\]
	Hierdurch folgt insbesondere, dass $\tau$ ein Isomorphismus ist. D.\,h., wir können Tangentialräume an verschiedenen Punkten mittels Paralleltransporte vergleichen.\\
	Daher die Terminologie \textsl{Zusammenhang}.
	\item $\Dd{t}V$ ordnet auch Vektoren an Punkten mit $\dot{c}(t) = 0$ zu. Diese Vektoren müssen nicht Null sein!
	\Bsp{}
	Wenn $c(t) = p$ konstant ist, dann ist $V(t)$ eine Kurve in $T_pM$. $\Dd{t}V$ ist dann einfach die Ableitung von $V(t)$ nach $t$, also $V'(t)$ im euklidischen Sinne.
\end{enumerate}


\newpage
\section{Der Levi-Civita-Zusammenhang}
Sei $(M, g)$ eine Riemannsche Mannigfaltigkeit.

\Def{}
Ein Zusammenhang $\nabla$ auf $M$ heißt \df{kompatibel} mit der Riemannschen Metrik $g$, falls für jede Kurve $c$ und für alle parallele Vektorfelder $V,W$ entlang $c$ gilt:
\[ \shrp{V,W} = \text{konst.} \]
d.\,h., der Paralleltransport ist in diesem Fall sogar eine Isometrie.

\Prop{}
$g$ und $\nabla$ sind genau dann kompatibel, wenn für alle Vektorfelder $V,W$ entlang einer beliebigen Kurve $c$ gilt
\[ \pf{}{t}\shrp{V,W} = \shrp{ \Dd{t}V, W } +\shrp{V, \Dd{t} W}. \]

\begin{Beweis}{}
\begin{itemize}
	\item[$\Leftarrow )$] Seien $V,W$ parallele Vektorfelder entlang $c$. Dann gilt
	\[\pf{}{t}\shrp{V,W} = \shrp{ \Dd{t}V, W } +\shrp{V, \Dd{t} W} = \shrp{ 0, W } +\shrp{V, 0} = 0. \]
	$\shrp{V,W}$ ist als Funktion in $t$ konstant.
	\item[$\Rightarrow )$] $\shrp{,}$ und $\nabla$ seien kompatibel. Sei $\{ P_1(t_0), \ldots, P_n(t_0) \} \subset T_{c(t_0)}M$ eine Orthonormalbasis. Durch den Paralleltransport erhalten wir die parallelen Vektorfelder $P_1, \ldots, P_n$ entlang $c$.\\
	Durch die Kompatibilität bleiben die $P_1, \ldots, P_n$ an jeder Stelle auf $c$ eine Orthonormalbasis. Seien $V,W$ nun beliebige Vektorfelder entlang $c$. Wir können dann schreiben
	\begin{align*}
	V = \sum_i v_i P_i && \text{ und } && W = \sum_j w_j P_j.
	\end{align*}
	Es gilt dann
	\[ \Dd{t}V = \sum_i (v_i' P_i + v_i \Dd{t}P_i) = \sum_i v_i' P_i. \]
	Und somit
	\[ \shrp{\Dd{t}V, W} = \shrp{ \sum_i v_i' P_i , \sum_j w_j P_j  } = \sum_{i,j} v_i'w_j\shrp{P_i,P_j} = \sum_i v_i'w_i. \]
	Und analog
	\[ \shrp{V, \Dd{t}W} =  \sum_i v_iw_i'. \]
	Zusammen also
	\[\shrp{\Dd{t}V, W} + \shrp{V, \Dd{t}W} = \sum_i (v_i'w_i + v_i w_i'). \]
	Ferner gilt
	\[ \shrp{V,W} = \ldots = \sum_i v_i w_i. \]
	Mit der Produktregel folgt nun
	\[ \pf{}{t} \shrp{V,W} = \sum_i (v_i'w_i + v_i w_i'). \]
\end{itemize}
\end{Beweis}

\Kor{}
$g$ und $\nabla$ sind genau dann kompatibel, wenn gilt
\[ X\shrp{Y,Z} = \shrp{\nabla_x Y, Z} + \shrp{Y, \nabla_X Z} \]
für beliebige Tangentialvektorfelder $X,Y,Z$ auf $M$.
\begin{Beweis}{}
Für einen Punkt $p \in M$ wähle eine Kurve $c$ mit $c(0) = p$ und $\dot{c}(0) = X(p)$. Es gilt dann
\[ X(p)\shrp{Y,Z} = \pf{}{t}_{|t = 0} \shrp{Y_{c(t)}, Z_{c(t)}}. \]
\end{Beweis}

\Def{Symmetrie von Zusammenhängen}
Ein Zusammenhang $\nabla$ heißt \df{symmetrisch}, wenn gilt
\[ \nabla_{X} Y - \nabla_YX = [X,Y]. \]
In lokalen Koordinaten für $X = \pf{}{x_i}$ und $Y = \pf{}{y_j}$ gilt dann
\[ \nabla_{\pf{}{x_i}}\pf{}{x_j} - \nabla_{\pf{}{x_j}}\pf{}{x_i} = [\pf{}{x_i}, \pf{}{x_j}] = 0. \]
Daraus folgt dann
\[ \nabla_{\pf{}{x_i}}\pf{}{x_j} = \nabla_{\pf{}{x_j}}\pf{}{x_i}. \]
Für die Christoffel-Symbole bedeutet dies
\[ \Gamma_{i,j}^k = \Gamma_{j,i}^k. \]

\Bem{}
Definiere die \df{Torsion} durch
\[ T(X,Y) := \nabla_{X}Y - \nabla_{Y}X - [X,Y]. \]
$T$ ist linear über $\CC{\infty}(M)$. D.\,h., $T$ ist ein Tensor.\\
Ferner ist ein Zusammenhang genau dann symmetrisch, wenn er torsionsfrei ist.

\Satz{Levi-Civita}
Sei $(M,g)$ eine Riemannsche Mannigfaltigkeit. Dann existiert genau ein Zusammehang $\nabla$ auf $M$, sodass gilt:
\begin{enumerate}[1.)]
	\item $\nabla$ und $g$ sind kompatibel.
	\item $\nabla$ ist symmetrisch.
\end{enumerate}
Diesen Zusammenhang nennen wir den \df{Levi-Civita-Zusammenhang} bzw. den \df{Riemannschen Zusammenhang}.

\begin{Beweis}{}
\begin{enumerate}[option]
	\item[Eindeutigkeit] Seien $X,Y,Z$ beliebige Tangentialvektorfelder auf $M$. Es gilt dann
	\[ X\shrp{Y,Z} = \shrp{\nabla_{X}Y, Z} + \shrp{Y, \nabla_{X} Z} \]
	und
	\[ Y\shrp{Z,X} = \shrp{ \nabla_{Y} Z, X } + \shrp{Z, \nabla_{Y} X} \]
	und
	\[ Z\shrp{X,Y} = \shrp{\nabla_{Z}X, Y} + \shrp{X, \nabla_Z Y}. \]
	Wir addieren die ersten beiden Zeilen und subtrahieren die dritte. Dadurch erhalten wir
	\begin{align*}
	&X\shrp{Y,Z} + Y\shrp{Z,X} - Z \shrp{X,Y} \\
	=&
	\shrp{Y, \nabla_XZ - \nabla_ZX} + \shrp{ X, \nabla_YZ - \nabla_YZ }\\
	+& \shrp{Z, \nabla_XY + \nabla_YX}\\
	\gl{\nabla \text{ symm}}& \shrp{Y, [X,Z]} + \shrp{X, [Y,Z]} + \shrp{Z, [X,Y] + 2 \nabla_YX}\\
	=&  \shrp{Y, [X,Z]} + \shrp{X, [Y,Z]} + \shrp{Z, [X,Y] }+ 2 \shrp{Z,\nabla_YX}\
	\end{align*}
	Daraus erhalten wir für $\nabla$
	\begin{align*}
	&\shrp{Z,\nabla_YX} =\\
	 &\frac{1}{2} \klam{
		X\shrp{Y,Z} + Y\shrp{Z,X} - Z\shrp{X,Y} - \shrp{Y,[X,Z]} - \shrp{X, [Y,Z]} - \shrp{Z, [X,Y]}
	}
	\end{align*}
Daraus folgt die Eindeutigkeit von $\nabla$.
\item[Existenz] Definiere $\nabla_YX$ durch obige Gleichung. Dann bleibt nachzurechnen, dass $\nabla$ ein symmetrischer und kompatibler Zusammenhang ist.
\end{enumerate}

\end{Beweis}
\marginpar{Vorlesung vom 02.05.18}

\section{Geodätische Kurven}
\Def{}
Sei $(M,g)$ eine Riemannsche Mannigfaltigkeit. Sei $\nabla$ der Levi-Civita-Zusammenhang auf $M$.\\
{Geodätische} sind Kurven auf $M$ mit Beschleunigung Null, d.\,h., eine glatte Kurve $c : I \pfeil{} M $ heißt \df{geodätisch}, falls
\[ \Dd{t}\dot{c} = 0 \]
gilt.

\Bsp{}
Betrachte $\R^n$ mit der Euklidischen Metrik. Durch den Levi-Civita-Zusammenhang werden alle Christoffel-Symbole Null. Gilt
\[ 0 = \Dd{t}\dot{\gamma} = \ddot{\gamma}, \]
so muss $\dot{\gamma}$ konstant gleich $a$ sein. Ergo ist $\gamma(t) = at +b$ eine Gerade.
\vspace{6 mm}\\

Sei $\gamma$ eine Geodätische. Betrachte
\[ \pf{}{t} \shrp{\dot{\gamma}, \dot{\gamma}} = 2 \shrp{\Dd{t}\dot{\gamma}, \dot{\gamma}} =2 \shrp{0, \dot{\gamma}} = 0, \]
da $\gamma$ geodätisch ist. Somit ist $\norm{\gamma'(t)}$ konstant gleich $c \in \R_{\geq 0}$.\\
Sei $c\neq 0$. $0,t$ seien in $I$. Dann
\begin{align*}
L_0^t(\gamma) &= \int_{0}^t \norm{\dot{\gamma}(\tau)} \d\tau = \int_{0}^t c \d\tau = ct.
\end{align*}
D.\,h., die Bogenlänge ist proportional zum Parameter $t$. Ist insbesondere $c = 1$, dann sagen wir, dass $\gamma$ durch die Bogenlänge parametrisiert sei.
\paragraph{In lokalen Koordinaten $x$} lässt sich $\gamma$ darstellen durch
\[ \gamma(t) = (x_1(t), \ldots, x_n (t)) .\]
Sei $V(t)$ ein Vektorfeld entlang $\gamma$. $V$ hat die Gestalt
\[ V(t) = \sum_i v_i(t) \pf{}{x_i}_{|\gamma(t)}. \]
Es gilt allgemein
\begin{align*}
\Dd{t}V = \sum_k \klam{
v_k' + \sum_{i,j} x_i' v_j \Gamma_{i,j}^k
}
\pf{}{x_k}.
\end{align*}
Für $V(t) = \dot{\gamma(t)}$ gilt $v_k(t) = x_k'(t)$. Dann gilt
\begin{align*}
0 = \Dd{t}\dot{\gamma} = \sum_k \klam{
	x_k'' + \sum_{i,j} x_i' x_j' \Gamma_{i,j}^k
}
\pf{}{x_k}.
\end{align*}
Daraus folgt für alle $k$
\begin{align*}
	x_k'' = - \sum_{i,j} x_i' x_j' \Gamma_{i,j}^k.
\end{align*}
Dadurch erhalten wir ein System von gewöhnlichen Differentialgleichungen 2. Ordnung. Auf dem Tangentialbündel $\T M$ kann dieses System umgeschrieben werden in ein System 1. Ordnung. Seien die Koordinaten $x$ definiert auf $U \subset M$. Ein Tangentialvektor kann geschrieben werden als eine Linearkombindation
\[ \sum_i y_i \pf{}{x_i}. \]
Dann sind $(x_1, \ldots, x_n, y_1, \ldots, y_n)$ lokale Koordinaten auf $\T M$, definiert in $\T U$.\\
Die Abbildung
\[ t \longmapsto (\gamma(t), \dot{\gamma}(t)) \]
definiert eine glatte Kurve in $\T M$. Hierfür gilt
\begin{align*}
y_k &= x_k\\
y_k' &= - \sum_{i,j} \Gamma_{i,j}^ky_iy_j.
\end{align*}
Dies ist ein System von Differentialgleichungen 1. Ordnung auf $\T M$. Wir wenden den Satz über Existenz, Eindeutigkeit und Abhängigkeit von Anfangsbedingungen an auf dieses System. Es folgt dann:

\Prop{}
Für alle $p \in M$ existieren $\delta, \e_1 > 0$, eine offene Umgebung $V \subset M$ von $p$ und eine glatte Abbildung
\[ \gamma : (-\delta, \delta) \times U \Pfeil{} M, \]
wobei
\[ U = \set{(q,v) \in V \times T_qM}{\norm{v} < \e_1}, \] 
sodass
\[ t \longmapsto \gamma(t,q,v) \]
die eindeutige Geodätische in $M$ ist mit
\begin{align*}
\gamma(0,q,v) = q && \text{ und } && \dot{\gamma(0,q,v)} = v.
\end{align*}

\Lem{Homogenität von Geodätischen}
Ist die Geodäte $\gamma(t,q, v)$ definiert für $\bet{t} < \delta$, so ist die Geodäte $\gamma(at,q,v)$ definiert für $a > 0$ und $\bet{t} < \frac{q}{a}$, und es gilt
\[ \gamma(at, q, v) = \gamma(t,q,av). \]
\begin{Beweis}{}
Setze $c(t) := \gamma(at, q,v)$. Dann ist $c(0) = q$ und $\dot{c}(0) = a \dot{\gamma}(0,q,v) = av$. Damit erfüllt $c$ dieselben Anfangsbedingungen wie $\gamma(t,q, av)$. Es bleibt zu zeigen, dass $c$ tatsächlich eine Geodätische ist. Es gilt
\[ \Dd{t}\dot{c} = \nabla_{\dot{c}} \dot{c} = \nabla_{a\dot{\gamma}} (a\dot{\gamma}) = a^2 \nabla_\{\dot{\gamma}\} \dot{\gamma} = a^2 \cdot 0 = 0. \]
Aus der Eindeutigkeit folgt nun
\[ c(t) = \gamma(t,q,av). \]
\end{Beweis}\\
Betrachte insbesondere $\bet{t} < 2 = \frac{\delta}{\delta / 2}$ und $a = \frac{\delta}{2}$. Setze $\e = \frac{\delta \e_1}{2}$. Dann ist $\gamma(t,q,v)$ definiert für $\bet{t}<2$ und $\norm{v} < \epsilon$.

\Def{Die Exponentialabbildung}
Sei $q \in V$, $v \in T_qM$ mit $\norm{v} < \e$. Definiere die Abbildung
\begin{align*}
 \exp_q(v) = \gamma(1, q, v).
\end{align*}
Für $v \neq 0$ gilt
\[ \exp_q(v) = \gamma(1, q, v) = \gamma(\norm{v}, q, \frac{v}{\norm{v}}) \].
Bezeichnet $B_0(\e)$ den $\e$-Ball in $T_qM$, so ist $\exp_q$ eine Abbildung vom Typ
\[ \exp_q : B_0(\e) \subset T_qM \Pfeil{} M. \]
Wir schreiben allgemein auch $\exp$ statt $\exp_q$.

\Bem{}
Die Bezeichnung obiger Abbildung als Exponentialabbildung kommt aus der Theorie der Lie-Gruppen. Ist $G$ eine Lie-Gruppe, so erhält man eine Abbildung
\[ \exp : \mathfrak{g} := T_1G \Pfeil{} G, \]
wobei $\mathfrak{g}$ die Lie-Algebra von $G$ bezeichnet. D.\,h., in diesem Fall gilt
\[ \exp(\mathfrak{a} + \mathfrak{b}) = \exp(\mathfrak{a}) \cdot \exp(\mathfrak{b}). \]

\Prop{}
Es existiert ein $\e > 0$, sodass
\[ \exp : B_0(\e) \]
ein Diffeomorphismus auf sein Bild ist.
\begin{Beweis}{}
Betrachte
\[ (\d \exp)_0(v) = \pf{}{t}_{| t = 0} \exp(tv) = \pf{}{t}_{| t = 0} \gamma(1, q, tv) =  \pf{}{t}_{| t = 0} \gamma(t, q, v) = v . \]
D.\,h., $\d \exp_0$ ist die Identität auf $B_0(\epsilon)$. Der Satz über umkehrbare Funktionen impliziert, dass $\exp$ ein lokaler Diffeomorphismus in der Nähe von $0$ ist.
\end{Beweis}

\Bsp{}
\begin{enumerate}[1)]
	\item Sei $M = \R^n$. Betrachte
	\[ \exp_0 : T_0\R^n\isom{} \R^n \Pfeil{\id{\R^n}} \R^n \]
	\item Sei $S^n \subset \R^{n+1}$ die Einheitssphäre. Betrachte
	\[ \exp_q  : B_0(\pi) \Pfeil{} S^n - \{-q\} \]
	wobei $q$ den Nordpol bezeichnet. $\exp_q$ ist dann tatsächlich surjektiv auf $S^n - \{-q\}$. Allerdings gilt
	\[ \exp_q(\partial B_0(\pi)) = \{-q\}. \]
\end{enumerate}

\Satz{Gauss-Lemma}
Es gilt
\[ \shrp{\d \exp_v(v), \d \exp_v(w)} = \shrp{v,w} \]
für $v,w \in T_qM$. Dabei wurde stillschweigend die Identifikation
\[ T_v(T_qM) \isom{} T_qM \]
angenommen.
\begin{Beweis}{}
Wir schreiben $w = w_{||} + w_\bot$ mit $w_{||} \in \R\cdot {v}$ und $w_\bot \in v^\bot$. Die Linearität impliziert, dass es genügt die Aussage für $w_{||}$ und für $w_\bot$ jeweils zu beweisen.
\begin{enumerate}[1)]
	\item Für $w_{||} = \lambda v$:\\
	Es gilt
	\begin{align*}
	\shrp{\d \exp_v(v), \d \exp_v(\lambda v)} = {\lambda} \norm{\d \exp_v(v)}^2
	\end{align*} 
	und
	\[ \shrp{v,\lambda v} = \lambda \norm{v}^2. \]
	Zu zeigen bleibt
	\[ \norm{\d \exp_v(v)} = \norm{v}. \]
	Es gilt nun
	\begin{align*}
	\norm{\d \exp_v(v)} &= \norm{\pf{}{t}_{|t = 0} \gamma(1,q, v+tv)} =\norm{ \pf{}{t}_{|t = 0} \gamma(1+t, q, v)} = \norm{v}
	\end{align*}
	\item Für $w_\bot $:\\
	Wir schreiben $w = w_\bot$ und es gilt $\shrp{v,w} = 0$. Zu zeigen ist
	\[ \shrp{\d \exp_v(v), \d \exp_v(w)} = 0. \]
	Sei $v(s)$ eine Kurve in $T_qM$ mit $v(0) = v, \dot{v} = w$ und $\norm{v(s)}$ konstant. Setze
	\[ f(t,s) := \exp(tv(s)). \]
	$f$ ist eine parametrisierte Fläche. Es gilt dann
	\[ \shrp{\d \exp_v(v), \d \exp_v(w)} = \shrp{\pf{f}{t}, \pf{f}{s}}(t = 1, s = 0). \]
	Wir behaupten, dass $\shrp{\pf{f}{t}, \pf{f}{s}}$ unabhängig von $t$ ist, denn:
	\begin{align*}
	\pf{}{t} \shrp{\pf{f}{t}, \pf{f}{s}} = \shrp{\Dd{t} \pf{f}{t}, \pf{f}{s}} +  \shrp{ \pf{f}{t}, \Dd{t}\pf{f}{s}}
	\end{align*}
	Nun ist $\pf{}{t} \pf{f}{t}$ gleich Null, da $\gamma$ eine Geodätische ist. Es gilt nun
	\begin{align*}
	\shrp{\Dd{t} \pf{f}{s}, \pf{f}{t}} \gl{Symmetrie} \shrp{\Dd{s}\pf{f}{t}, \pf{f}{t} } = \frac{1}{2} \pf{}{s} \norm{\pf{f}{t}}^2 = 0,
	\end{align*}
	da $\norm{v(s)}$ konstant ist.\\
	
	Betrachte wieder
	\[ \shrp{\pf{f}{t}, \pf{f}{s}}(1, s) = \shrp{\pf{f}{t}, \pf{f}{s}} (0,s). \]
	Nun gilt
	\[ \pf{f}{s}(0,s) = 0, \]
	da $f(0,s) = \exp(0\cdot v(s)) = \exp(0) = q$ konstant in $s$ ist.
\end{enumerate}
\end{Beweis}

\marginpar{Vorlesung vom 07.05.18}
\Def{}
Sei $\exp_p : T_pM \pfeil{} M$ die Exponentialabbildung und $\e > 0$ so, dass $\exp_p$ auf $B_0(\e)$ injektiv ist.\\
Für $0<r<\e$ nennen wir dann
\[ B_p(r) := \exp_p(B_0(r)) \]
den \df{geodätischen Ball} und
\[ S_p(r) := \exp_p(\partial B_0(r)) \]
die \df{geodätische Sphäre} um $p$ von Radius $r$.

\Bem{Interpretation: Gauss-Lemma}
Wir können nun das Gauss-Lemma wie folgt ausdrücken:
\begin{center}
\emph{
	Geodätische Kurven durch $p$ stehen senkrecht auf geodätischen Sphären.
}	
\end{center}

\Prop{Geodätische minimieren lokal die Länge von Kurven.}
Sei $p \in M$ und $\e > 0$ so klein, dass $\exp_p : B_0(\e) \pfeil{} M$ injektiv ist. Sei $\gamma: [0,1] \pfeil{} B := B_p(r)$ für $r<\e$ mit $\gamma(0) = p$ eine Geodätische.\\
Sei $c : [0,1] \pfeil{} M$ eine stückweise glatte Kurve mit $c(0) = p$ und $c(1) = q:= \gamma(1)$.\\
Dann gilt
\[ L(c) \geq L(\gamma). \]
Ferner gilt Gleichheit genau dann, wenn $c$ und $\gamma$ dasselbe Bild haben.
\begin{Beweis}{}
\paragraph{Idee:} Wir schreiben $c = c(s)$ in Polarkoordinaten:
\[ c(s) = \exp ( r(s) \cdot v(s) ) \]
für $r > 0, s > 0$ und $\norm{v(s)} = 1$. Wir nehmen dabei zunächst an, dass $c[0,1] \subset B$. Ferner nehmen wir ohne Einschränkung an, dass $c(s)\neq p$ für $s > 0$. Setze
\[ f(r,s) := \exp (r \cdot v(s)). \] 
Dann gilt
\[ c(s) = f(r(s), s). \]
Daraus folgt
\[ \dot{c}(s) = \pf{f}{r} \cdot r' + \pf{f}{s}. \]
Und hieraus
\begin{align*}
\norm{\dot{c}(s)}^2 &= 
\norm{\pf{f}{r} \cdot r' }^2 
+ 2 \shrp{\pf{f}{r} r, \pf{f}{s}}
+ \norm{\pf{f}{s}}^2\\
&= \bet{r'}^2 \cdot \norm{\pf{f}{r}}^2
+ 2r' \shrp{ \pf{f}{r}, \pf{f}{s} }
+ \norm{\pf{f}{s}}^2\\
&= \bet{r'}^2 \cdot 1 + 2r' \cdot 0 + \norm{\pf{f}{s}}^2,
\end{align*}
denn $\shrp{\pf{f}{r} r, \pf{f}{s}} = 0$ nach Gauss-Lemma und $\norm{\pf{f}{r}} = \norm{v(s)} = 1$.\\
Es folgt also
\[ \norm{\dot{c}(s)}^2 = \bet{r'}^2 + \norm{\pf{f}{s}}^2 \geq \bet{r'}^2. \]
Wähle nun $\delta > 0$ klein, und betrachte
\[ \int_{\delta}^{1} \norm{\dot{c}(s)} \d s \geq_{\delta}^1 \bet{r'(s)} \d s \geq \int_{\delta}^1r'(s)\d s = r(1) - r(\delta) \pfeil{\delta \pfeil{} 0} r(1) = L(\gamma).  \]
Ferner gilt
\[ \int_{\delta}^{1} \norm{\dot{c}(s)} \d s \pfeil{\delta \pfeil{} 0}  L(c).  \]
Gilt Gleichheit, so muss
\[ \norm{\pf{f}{s}} = 0 \]
gelten. Daraus folgt aber, dass $f(r,s)$ konstant in $s$ ist. Ergo
\[ f(r,s) = \exp (r \cdot v(0)). \]
Insofern haben in diesem Fall $c$ und $\gamma$ tatsächlich dasselbe Bild.\\

Wenn nun $c[0,1]$ nicht in $B$ enthalten ist, dann sei $s_0$ der kleinste Wert $s$, sodass $c(s_0) \in \partial B$. Es gilt
\[ L_0^1(c) \geq L_0^{s_0}(c) \geq L(\gamma_1) = r \geq L(\gamma).\]
$\gamma_1 : p \mapsto c(s_0)$ ist eine Geodätische.
\end{Beweis}

\Bem{}
\begin{enumerate}[1.)]
	\item Man kann auch zeigen:\\
	Ist $\gamma$ eine Kurve parametrisiert proportional zur Bogenlänge, sodass
	\[ L(\gamma) \leq L(c) \]
	für alle Kurven $c$ mit denselben Randpunkten gilt, so muss $\gamma$ eine Geodätische sein.
	\item Isometrien erhalten Geodätische.
\end{enumerate}

\newpage
\section{Krümmung}
\Bsp{}
\begin{itemize}
	\item Die Krümmung eines Kreises von Radius $r$ definieren wir durch $\frac{1}{r}$.
	\item Wir betrachten nun Kurven in $\R^2$, die durch die Bogenlänge parametrisiert sind.\\
	Sei dazu $c$ eine solche Kurve mit $\ddot{c(s)} \neq 0$ für ein $s$. Betrachte $s_1, s_2, s_3$ nahe bei $s$. Da die zweite Ableitung nicht verschwindet, sind $c(s_1), c(s_2)$ und $c(s_3)$ nicht kolinear.\\
	Daraus folgt, dass $c(s_1), c(s_2)$ und $c(s_3)$ auf einem eindeutig bestimmten Kreis mit Radius $R$ liegen. Für $s_1, s_2, s_3 \pfeil{} s$ erhält man einen Grenzkreis, den sogenannten oskulierenden Kreis in $c(s)$.\\
	Die Krümmung von $c$ im Punkt $c(s)$ definiert man nun als $\frac{1}{R}$, die Krümmung dieses oskulierenden Kreises.\\
	Es gilt nun ferner
	\[ \frac{1}{R} = \bet{\ddot{c}(s)}. \]
	\item Kurven in $\R^3$:\\
	Wir fixieren wieder $s$. Sei $\ddot{c}(s) \neq 0$. $c(s_1), c(s_2)$ und $c(s_3)$ definieren dann eine Ebene in $\R^3$. Laufen $s_1, s_2, s_3$ nach $s$, so definieren sie eine Grenzebene, die oskulierende Ebene.\\
	Ferner erhält man in dieser oskulierenden Ebene den oskulierenden Kreis mit Radius $R$. Die Krümmung bei $c(s)$ definieren wir dann wieder als die Krümmung $\frac{1}{R}$ des oskulierenden Kreises. Es gilt nun
	\[ 0 = \pf{}{s} \norm{\dot{c}(s)}^2 = \pf{}{s}\shrp{\dot{c}, \dot{c}} = 2\shrp{\ddot{c}, \dot{c}} \]
	ergo steht $\ddot{c}(s)$ orthogonal auf $\dot{c}(s)$. Beide liegen in der oskulierenden Ebene und spannen diese auf.
	\item Flächen und Euler:\\
	Sei $M$ eine zweidimensionale Mannigfaltigkeit im $\R^3$. Sei $p \in M$ und $\nu_p$ ein Einheitsnormalenvektor, d.\,h.,
	\[ \nu_p \bot T_pM \text{ und } \norm{\nu_p} = 1. \]
	Sei ferner $v \in T_pM$ mit $\norm{v} = 1$. $\nu_p$ und $v$ spannen eine Ebene $E_v$ auf. Schneidet man diese mit $M$, so erhält man eine Kurve
	\[ E_v\cap M = \text{ Kurve }c_v. \]
	$c_v$ sei hierbei durch Bogenlänge parametrisiert mit $c_v(0) = p$ und $\dot{c_v}(0) = v$. Es gilt nun
	\[ \ddot{c_v}(0) \bot T_pM. \]
	Dann existiert genau ein $\kappa_v \in \R$, sodass
	\[ \ddot{c_v}(0) = \kappa_v \nu_p. \]
	Es gilt
	\[ \kappa_{-v} = \kappa_v, \]
	insofern erhalten wir eine Funktion
	\[ \kappa : \R P^1 \Pfeil{} \Pfeil{} \R. \]
	
	\Satz{Satz von Euler}
	Es existieren eindeutige Richtungen $v_1,v_2 \in \R P^1$, sodass
	\[ k_1 := \kappa_{-v_1} = \min_v \kappa_v \]
	und
	\[ k_2 := \kappa_{v_2} = \max_v\kappa_v. \]
	Es gilt ferner
	\[v_1 \bot v_2\]
	und
	\[ \kappa_v = k_1 \cos^2\theta + k_2 \sin^2\theta \]
	wobei $\theta = \angle (v, v_1)$.	 
	
\end{itemize}



\marginpar{Vorlesung vom 09.05.18}

\paragraph{Krümmung von Flächen nach Gauss}
Sei $M^2 \subset \R^3$ eine orientierte Fläche, $p\in M$. Sei ferner $\nu_p \in T_pM^\bot$ ein Einheitsnormalenvektor orthogonal auf $M$ am Punkt $p$, sodass $(\nu_p, v, w)$ positiv orientiert ist, wobei $(v,w)$ positiv orientiert in $T_pM$ sei.\\
Dies induziert die \df{Gauss-Abbildung}:
\begin{align*}
\nu: M &\Pfeil{} S^2\\
p &\longmapsto \nu_p
\end{align*}
Ist $A \subset M$ eine Umgebung um $p$, so kann man die \df{Gauss-Krümmung} definieren durch
\begin{align*}
\kappa(p) := \lim\limits_{A\pfeil{} p} \frac{\vol(\nu(A))}{\vol(A)}.
\end{align*}

\Bsp{}
\begin{enumerate}[1.)]
	\item Sei $M = S^2 = S^2_1$ die Einheitssphäre. Dann ist $\nu = \id{S^2}$. Daraus folgt $\kappa(p) = 1$ für alle $p \in M$.
	\item Sei $M = S^2_r$ die Sphäre von Radius $r$. Dann gilt
	\[ \vol(\nu(A)) = \frac{1}{r^2}\vol(A). \]
	Daraus folgt
	\[ \kappa(p) = \lim\limits_{A\pfeil{} p} \frac{\vol(\nu(A))}{\vol(A)} = \frac{1}{r^2}. \]
	\item Ist $M$ eine Ebene, so sind alle $\nu_p$ parallel zueinander. Daraus folgt, dass $\nu$ konstant ein Punkt ist. Und somit gilt $\kappa(p) = 0$ für alle $p\in M$.
	\item Sei $M$ ein Zylinder, $p \in M$. Ist $A$ eine kleine Umgebung um $p$, so induziert die Nabe bei $b$ eine Strecke auf dem Äquator von $S^2$. Die Längsachse des Zylinders induziert nur einen Punkt in $S^2$. Insofern ist $\nu(A)$ eine Strecke in $S^2$. Es folgt $\vol(\nu(A)) = 0$ und $\kappa(p) = 0$.\\
	Daraus folgt, der Zylinder ist \textbf{nicht} gekrümmt!
\end{enumerate}

\Satz{Beziehung Gauss-Euler}
Es gilt
\[ \kappa(p) = \kappa_1(p) \cdot \kappa_2(p),\]
wobei $\kappa(p)$ die Gauss-Krümmung und $\kappa_1(p), \kappa_2(p)$ die Eulerschen Minimal- und Maximal-Krümmungen bezeichnet.

\Bsp{}
\begin{enumerate}[1)]
	\item Betrachte $S^2_r \subset \R^3$. Dann ist $\kappa_1 = \kappa_2 = \frac{1}{r}$. Insbesondere gilt
	\[ \kappa_1 \kappa_2 = \frac{1}{r^2} = \kappa(p). \]
	\item Betrachte den Zylinder. Dann ist $\kappa_2 = \frac{1}{r}$ bei einem Radius von $r$ und $\kappa_1 = 0$. Es folgt
	\[ \kappa(p) = 0 = \kappa_1 \kappa_2. \]
	\item Betrachte die Fläche $z = \frac{a}{2}(x^2 - y^2)$ für $a > 0$. Dann gilt
	\[ \pf{}{x} \pf{}{x}(\frac{ax^2}{2}) = a = \kappa_2 > 0 \]
	und
	\[ \pf{}{y} \pf{}{y}(-\frac{ay^2}{2}) = -a = \kappa_1 < 0 \]
	bei $p= (0,0,0)$. Folglich gilt
	\[ \kappa_1 \kappa_2 = -a^2 < 0. \]
	Insofern handelt es sich hierbei um eine Fläche negativer Krümmung.
\end{enumerate}

\paragraph{Krümmung nach Riemann}
Idee: Sei $M$ eine $n$-dimensionale Mannigfaltigkeit und $p \in M$. Sei $\sigma \subset T_pM$ ein zweidimensionaler Untervektorraum. Betrachte
\begin{align*}
\exp_p : B_0(\e) \subset T_pM \Pfeil{\isom{}} U,
\end{align*}
wobei $U$ den geodätischen Ball um $p$ bezeichnet. $F^2 := \exp_p(\sigma \cap B_\e(0))$ ist dann eine Fläche in $U$. $F$ erhalte die induzierte Metrik von $M$.\\
Dann sei $\kappa(p, \sigma)$ definiert als die Krümmung von $F$ im Punkt $p$ nach Euler-Gauss.\\
Formell: Sei $\nabla$ der Levi-Civita-Zusammenhang auf der Riemannschen Mannigfaltigkeit $(M, \shrp{,})$. Wir definieren eine Abbildung
\begin{align*}
R : \Gamma(\T M )^3 & \Pfeil{} \Gamma(\T M)\\
(X,Y,Z) & \longmapsto R(X,Y)Z
\end{align*}
wobei
\[ R(X,Y) Z := \nabla_{Y} \nabla_{X} Z - \nabla_X \nabla_{Y} Z + \nabla_{[X,Y]}Z. \]
In lokalen Koordinaten $\{x_i\}$ mit $X = \pf{}{x_i}$ und $Y = \pf{}{x_j}$ gilt
\[ [X,Y] = 0 \]
und insbesondere
\[ R(X,Y) = \nabla_{Y}\nabla_{X} - \nabla_{X} \nabla_{Y}. \]
Ferner
\[ R(X,Y) \pf{}{x_k} =: \sum_l R^l_{i,j,k} \pf{}{x_l}. \]

\paragraph{Eigenschaften}
Sind $f,g : M \pfeil {} \R$ und $X,Y \in \Gamma(\T M)$ glatt, so gilt:
\begin{itemize}
	\item $R(fX_1+gX_2,Y)Z = fR(X_1,Y)Z + gR(X_2,Y)Z$.
	\item $R(X, fY_1 + gY_2)Z = fR(X,Y_1)Z + gR(X,Y_2)Z$.
\end{itemize}
Ferner gilt
\begin{align*}
R(X,Y)(fZ) &= \nabla_Y \nabla_{X} (fZ) - \nabla_{X} \nabla_{Y} (fZ) + \nabla_{[X,Y]} (fZ)\\
&= \nabla_{Y}( f\nabla_{X} Z + X(f)Z) - \nabla_{X}(f\nabla_{Y} Z + Y(f)Z) + f\nabla_{[X,Y]} Z + [X,Y](f) Z\\
&=f \nabla_{Y} \nabla_{X} Z + Y(f) \nabla_{Y} Z + YX(f) Z\\
&- f\nabla_{X} \nabla_{Y} Z - X(f) \nabla_{Y} Z - Y(f) \nabla_{X} Z - XY(f)Z \\
&+ f\nabla_{[X,Y]} Z + XY(f) Z -YX(f)Z\\
&= f R(X,Y)Z
\end{align*}
und insbesondere
\[ R(X,Y)(fZ_1 + gZ_2) = f R(X,Y) Z_1 + g R(X,Y)Z_2.  \]
Daraus folgt, dass $R$ ein Tensor ist, der sogenannte \df{Riemannsche Krümmungstensor}. (Dies erklärt den Term $\nabla_{[X,Y]}Z$.)\\
Es folgt auch, dass $(R(X,Y)Z)_p$ am Punkt $p \in M$ nur von den Vektoren $X(p), Y(p)$ und $Z(p)$ abhängt.\\
Ferner gilt:
\label{SymmetrienRiemannscherKrümmungstensor}
\begin{enumerate}[1)]
	\item $R(X,Y)Z + R(Y,X) Z = 0$ (offensichtlich).
	\item Symmetrie von $\nabla$ + Jacobi-Identität für $[,]$ impliziert die \df{Bianchi-Identität}
	\[ R(X,Y)Z + R(Y,Z)X  + R(Z,X)Y = 0. \]
	\item Es gilt $\shrp{R(X,Y)Z, W} + \shrp{R(X,Y)W, Z} = 0 $, denn
	\begin{align*}
	\shrp{R(X,Y)Z,Z} &= \shrp{ \nabla_{Y} \nabla_{X} Z - \nabla_{X} \nabla_{Y} Z + \nabla_{[X,Y]} Z, Z }\\
	&= \shrp{\nabla_{Y} \nabla_{X} Z, Z}  - \shrp{\nabla_{X} \nabla_{Y}Z, Z } + \shrp{\nabla_{[X,Y]} Z, Z}\\
	&= Y\shrp{\nabla_{X} Z, Z}
	- \shrp{\nabla_{X} Z, \nabla_{Y} Z}
	- X\shrp{\nabla_{Y} Z, Z }
	+ \shrp{\nabla_Y Z, \nabla_{X} Z}
	+ \frac{1}{2} [X,Y] \shrp{Z,Z}\\
	&= 0,
	\end{align*}
	wobei
	\begin{align*}
	Y\shrp{\nabla_XZ,Z} &= \shrp{ \nabla_{Y} \nabla_{X} Z, Z } + \shrp{\nabla_{X} Z, \nabla_{Y} Z}\\
	X\shrp{\nabla_YZ,Z} &= \shrp{ \nabla_{X} \nabla_{Y} Z, Z } + \shrp{\nabla_{Y} Z, \nabla_{X} Z}\\
	[X,Y] \shrp{Z,Z} &= 2 \shrp{ \nabla_{[X,Y]} Z,Z }.
	\end{align*}
	\item Ferner gilt
	\begin{align*}
	\shrp{R(X,Y)Z, W} + \shrp{ R(Y,Z)X ,W} + \shrp{ R(Z,X)Y , W} &= 0\\
	\shrp{R(Y,Z)W, X} + \shrp{ R(Z,W)Y ,X} + \shrp{ R(W,Y)Z , X} &= 0\\
	\shrp{R(Z,W)X, Y} + \shrp{ R(W,X)Z ,Y} + \shrp{ R(X,Z)W , Y} &= 0\\
	\shrp{R(W,X)Y, Z} + \shrp{ R(X,Y)W ,Z} + \shrp{ R(Y,W)X , Z} &= 0.
	\end{align*}
	Indem man alle Zeilen aufaddiert, erhält man
	\begin{align*}
	0 &= \shrp{ R(Z,X)Y , W} + \shrp{ R(W,Y)Z , X} + \shrp{ R(X,Z)W , Y} +\shrp{ R(Y,W)X , Z}  \\
	&= 2 \shrp{R(Z,X)Y,W} - 2 \shrp{ R(Y,W)Z,X }.
	\end{align*}
	Ergo gilt auch folgende Symmetrie
	\[ \shrp{R(X,Y)Z, W} = \shrp{R(Z,W)X,Y}.  \]
\end{enumerate}

In lokalen Koordinaten $(x_1, \ldots, x_n)$ setzen wir
\[ X_i := \pf{}{x_i}. \]
$X,Y,Z \in \Gamma(\T M)$ schreiben wir als
\[ X = \sum_i x_i X_i,~~ Y = \sum_i y_i X_i~ \text{ und }~ Z = \sum_i z_i X_i. \]
Dann gilt
\[ R(X,Y)Z = \sum_{i,j,k} x_i y_j z_k R(X,i,X_j)X_k = \sum_{i,j,k} x_i y_j z_k R^l_{ijk} X_l \]
wobei
\[ R(X_i,X_j)X_k = \sum_l R_{jjk}^l X_l. \]
Insbesondere gilt
\begin{align*}
R(X_i, X_j)X_k &= \nabla_{X_j} \nabla_{X_i} X_k - \nabla_{X_i} \nabla_{X_j} X_k\\
&= \nabla_{X_j} \sum_l \Gamma_{ik}^l X_l - \nabla_{X_i} \sum_l \Gamma_{jk}^l X_l\\
&= \sum_l \sum_a (\Gamma_{ik}^l \Gamma_{jl}^a  - \Gamma_{jk}^l \Gamma_{il}^a)X_a
\end{align*}
Ergo
\[ R_{ijk}^a = \sum_l (\Gamma_{ik}^l \Gamma_{jl}^a  - \Gamma_{jk}^l \Gamma_{il}^a) \]
\marginpar{Vorlesung vom 14.05.18}
\paragraph{Schnittkrümmung}
Sei $p \in M$ ein Punkt und $\sigma \subset T_pM$ ein zweidimensionaler Untervektorraum. Sei $\{x,y\}$ eine Basis für $\sigma$. Die Fläche des von $x$ und $y$ aufgespannten Parallelogramms ist
\[ A(x,y) := \sqrt{\norm{x}^2 \norm{y}^2 - \shrp{x,y}}. \]
Wir betrachten
\[ \kappa(x,y) := \frac{\shrp{R(x,y)x,y}}{ A(x,y)^2 }. \]
\Lem{}
$\kappa(x,y)$ hängt nicht von der Wahl der Basisvektoren $x,y$ für $\sigma$ ab.
\begin{Beweis}{}
Jede andere Basis von $\sigma$ erhält man aus $\{x,y\}$ durch Anwendung der folgenden drei elementaren Transformationen:
\begin{align*}
\{x,y\} &\Impl{} \{y,x\}\\
\{x,y\} &\Impl{} \{\lambda x, y\} \text{ für }\lambda \neq 0\\
\{x,y\} &\Impl{} \{x + \lambda y, y\}
\end{align*}
Überprüfe dann, dass $\kappa(x,y)$ invariant bleibt unter diesen drei Transformationen.
\end{Beweis}

Aufgrund obigen Lemmas dürfen wir die \df{Schnittkrümmung} von $M$ entlang $\sigma$ in $p$ definieren:
\[ \kappa_p(\sigma) := \kappa(x,y) \]

Die Familie aller $\set{\kappa_p(\sigma)}{}_{\sigma \subset T_pM}$ bestimmt $R$ im Punkt $p$ eindeutig. Dies folgt aus einem Resultat der linearen Algebra, nämlich:
\Prop{}
\label{PropLAREindeutig}
Sei $(V,\shrp{\cdot, \cdot})$ ein Euklidischer Vektorraum und seien $R,R' : V\times V \times V \pfeil{} V$ trilineare Abbildungen, die beide die Symmetrien aus 1) bis 4) aus \ref{SymmetrienRiemannscherKrümmungstensor} erfüllen. Wenn ferner folgende Gleichheit vorliegt
\begin{align*}
\shrp{R(x,y)x,y} = \shrp{R'(x,y)x,y}
\end{align*}
für alle $x,y \in V$, dann gilt
\[ R = R'.\]

\paragraph{Ricci-Krümmung}
Sei $p \in M$ und $x \in T_pM$ mit $\norm{x} = 1$. Wir ergänzen $x$ zu einer Orthonormalbasis $\{x, z_1, \ldots, z_{n-1}\}$ von $T_pM$. Definiere die \df{Ricci-Krümmung} durch
\[ \Ric_p(x) := \frac{1}{n-1} \sum_{i = 1}^{n-1} \shrp{R(x,z_i)x,z_i}. \]
$\Ric_p(x)$ ist unabhängig von der Wahl von $\{z_i\}_{i = 1}^{n-1}:$
\[ Q(x,y) := \mathrm{Spur}(z \mapsto R(x,z)y) .\]
$Q$ ist eine Bilinearform und es gilt
\[ Q(x,x) = (n-1)\Ric_p(x). \]

\newpage
\section{Jacobi-Felder}
Wir stellen uns die Frage:
\begin{center}
	Wie Schnell Entfernen sich Geodäten Voneinander?
\end{center}
Sei dazu $(M,g)$ eine Riemannsche Mannigfaltigkeit zusammen mit dem Levi-Civita-Zusammenhang $\nabla$. Sei $p \in M$. Ferner sei die Abbildung $\exp_p : B_0(\e) \pfeil{} M$ gegeben. Sei $v \in T_pM$, dann ist
\[ \gamma(t) = \exp_p(tv) \]
die eindeutig bestimmte Geodätische in $p$ mit $\dot{\gamma}(0) = v$. Wir betrachten Vektorfelder entlang von Geodätischen. Sei $w \in T_v(T_pM)$. Wie im Gauss-Lemma sei $v(s)$ eine Kurve in $T_pM$ mit $v(0) = v$ und $\dot{v} (0) = w$. Setze nun
\[ f(t,s) := \exp_p(tv(s)). \]
Sei
\[ J(t) = (\d \exp_p)_{tv}(tw) = \pf{f}{s} (t, s = 0). \]
$J$ ist ein Vektorfeld entlang $\gamma$.\\
$\gamma$ ist eine Geodäte, ergo gilt
\[ \Dd{t} \pf{f}{t} = \Dd{t}\dot{\gamma} = 0. \]
Daraus folgt
\[ \Dd{s} (\Dd{t} \pf{f}{t}) = 0. \]
Ist $V$ ein Vektorfeld entlang einer parametrisierten Fläche, so gilt
\[ \Dd{s}\Dd{t} V - \Dd{t} \Dd{s} V = R(\pf{f}{t}, \pf{f}{s}) V. \]
Das kann man durch Nachrechnen in lokalen Koordinaten überprüfen.
\begin{align*}
\Dd{s} \Dd{t} \pf{f}{t} &= \Dd{t} \Dd{s} (\pf{f}{t}) + R(\pf{f}{t}, \pf{f}{s})\pf{f}{t}\\
= \Dd{t} \Dd{t} (\pf{f}{s})  + R(\dot{\gamma}, \pf{f}{s}) \dot{\gamma}.
\end{align*}
Daraus folgt, dass $J = \pf{f}{s}$ folgende Gleichung erfüllt
\[ \Dd{t}\Dd{t}J + R(\dot{\gamma}, J) \dot{\gamma} = 0. \]
Diese Gleichung nennt man \df{Jacobi-Gleichung}.

\paragraph{In Lokalen Koordinaten:}
Seien $\{e_1(t), e_2(t), \ldots, e_n(t)\}$ parallele Vektorfelder entlang $\gamma$, die an jedem Punkt $\gamma(t)$ eine Orthonormalbasis von $T_{\gamma(t)}M$ bilden. Betrachte
\[ J(t) = \sum_i f_i(t) e_i(t). \]
Es gilt
\begin{align*}
\Dd{t}\Dd{t} J(t) = \sum_i f_i''(t) e_i(t).
\end{align*}
Insbesondere folgt
\begin{align*}
 R(\dot{\gamma}, J) \dot{\gamma} &= \sum_i \shrp{ R(\dot{\gamma}, J)  \dot{\gamma}, e_i}e_i\\
 &\gl{\text{Fourier-Entwicklung}} \sum_{i,j} f_j \shrp{ R(\dot{\gamma}, e_j) \dot{\gamma}, e_i }e_i.
\end{align*}
Setzt man $a_{i,j} := \shrp{ R(\dot{\gamma}, e_j) \dot{\gamma}, e_i }$, so gilt
\[ f''_i(t) +\sum_j a_{i,j} f_j(t) = 0 \]
Dies ist eine \emph{lineare} Differentialgleichung zweiter Ordnung.

\Def{}
Ein Vektorfeld $J(t)$ entlang einer Geodätischen $\gamma(t)$ heißt \df{Jacobi-Feld}, wenn $J(t)$ die Jacobi-Gleichung erfüllt.\\

Die Tatsache, dass eine Differentialgleichung zweiter Ordnung vorliegt, impliziert nun, dass man nach Wahl von $J(0)$ und $\Dd{t}J(0)$ ein eindeutiges Jacobi-Feld durch Lösen von 
\[ f''_i(t) +\sum_j a_{i,j} f_j(t) = 0 \]
erhält.

\Bsp{}
$\dot{\gamma}(t)$ und $t\dot{\gamma(t)}$ sind Jacobi-Felder für eine Geodäte $\gamma$.

\Bsp{Jacobi-Felder auf Mannigaltigkeiten Konstanter Schnittkrümmung}
Sei $M$ eine Mannigfaltigkeit der konstanten Schnittkrümmung $\kappa$. Definiere $R'$ durch
\[ \shrp{R'(X,Y)Z, W} := \shrp{X,Z} \shrp{Y,W} - \shrp{X,W} \shrp{Y,Z}. \]
$R'$ ist trilinear und erfüllt die Symmetrien 1) - 4) des echten Krümmungstensors aus \ref{SymmetrienRiemannscherKrümmungstensor}. Betrachte
\[ \shrp{R'(X,Y)X,Y} = \norm{X}^2\norm{Y}^2 - \shrp{X,Y} = A(X,Y)^2. \]
Ferner gilt
\[ \frac{\kappa R'(X,Y)X,Y}{A(X,Y)^2} = \kappa = \frac{\shrp{ R(X,Y)X,Y }}{A(X,Y)^2}. \]
Aus \ref{PropLAREindeutig} folgt nun
\[ R = KR'. \]
Setzt man dies in die Jacobi-Gleichung ein, so erhält man
\begin{align*}
\shrp{R(\dot{\gamma}, J)\dot{\gamma}, T} &= \shrp{\kappa R'(\dot{\gamma}, J)\dot{\gamma}, T}\\
&= \kappa ( \shrp{\dot{\gamma}, \dot{\gamma}} \shrp{J,T} - \shrp{\dot{\gamma}, T} \shrp{\dot{\gamma}, J} ).
\end{align*}
Sei $\gamma$ parametrisiert durch die Bogenlänge und $J$ orthogonal zu $\gamma$. Es gilt
\begin{align*}
\shrp{ R(\dot{\gamma}, J) \dot{\gamma}, T } = \kappa \shrp{J,T}.
\end{align*}
Daraus vereinfacht sich die Jacobi-Gleichung zu
\[ \Dd{t} \Dd{t} J+  \kappa J = 0. \]
Sei $W(t)$ ein Vektorfeld entlang $\gamma$, $\norm{W(t)} = 1$, $\shrp{W, \dot{\gamma}} = 0$, $W$ parallel. Die vereinfachte Jacobi-Gleichung impliziert
\begin{align*}
J(t) = \left\lbrace
\begin{aligned}
\frac{\sin(\sqrt{\kappa} t)}{\sqrt{\kappa}} W(t) &&\text{, falls }\kappa > 0,\\
t W(t) && \text{, falls }\kappa = 0,\\
\frac{\sinh(\sqrt{-\kappa}(t))}{\sqrt{-\kappa}} W(t) && \text{, falls }\kappa < 0.
\end{aligned}
\right.
\end{align*}
\marginpar{Vorlesung vom 16.05.18}

\Bem{}
Ist $J(0) = 0$ und $ v := \dot{\gamma}(0),$, $w := \Dd{t}J(0)$, dann gilt
\[ J(t) = (\d \exp)_{tv}(tw) = \pf{f}{s}(t,0) \]
für $f(t,s) = \exp_{p}(tv(s))$ für $v(0) = v, \dot{v}(0) = w$. Dies folgt aus der Eindeutigkeit von Differentialgleichungen zweiter Ordnung.

\paragraph{Notation:}
Wir schreiben in Zukunft
\[J'(t) := \Dd{t}J \]
und allgemeiner
\[ J^{(k)}(t) := \klam{\Dd{t}}^k J. \]
\vspace{12mm}\\
Sei $J(t)$ ein Jacobi-Feld mit $J(0) = 0, v = \dot{\gamma}(0), w = J'(0), \norm{w} = 1$.
Wir interessieren uns für $\norm{J(t)}^2$ für kleine $t$ und taylorn es deswegen im Folgenden.
\begin{itemize}
	\item $\shrp{J,J}(0) = 0$,
	\item $\shrp{J,J}' = 2 \shrp{J',J}$, und insbesondere
	\[ \shrp{J,J}'(0) = 2\shrp{J'(0),J(0)} = 0, \]
	da $J(0) = 0$.
	\item $\shrp{J,J}'' = 2 \klam{ \shrp{J',J'} + \shrp{J,J''} }$, und ferner
	\[ \shrp{J,J}''(0) = 2\shrp{J'(0), J'(0)} = 2\norm{w}^2 = 2. \]
	\item $\shrp{J,J}''' = 2\klam{ 
2\shrp{J'',J'} + \shrp{J',J''} + \shrp{J,J'''}	
 } = 6 \shrp{J',J''} + 2 \shrp{J,J'''}$. Indem man $0$ einsetzt, erhält man
\[ \shrp{J,J}'''(0) = 6 \shrp{w, J''(0)} \gl{\mathrm{Jacobi}} 6 \shrp{w, - R(\dot{\gamma},J)\dot{\gamma}(0)} = 
 6 \shrp{w, - R(v,J(0))v} = 0
 \]
	\item $\shrp{J,J}'''' = 6 \shrp{J'',J''} + 8\shrp{J',J'''} + 2 \shrp{J,J''''}$.
	\[ \shrp{J,J}''''(0) = 6\shrp{J''(0), J''(0)} + 8\shrp{w, J'''(0)} = 8\shrp{w, J'''(0)}  \] 
	$J'' = - R(\dot{\gamma}, J) \dot{\gamma}$. Es gilt
	\begin{align*}
	J''' = - \Dd{t}_{t= 0} R(\dot{\gamma}, J)\dot{\gamma} \gl{(1)}  R(\dot{\gamma}, J'(0))\dot{\gamma} = -R(v,w)v
	\end{align*}
	Somit ergibt sich
		\[ \shrp{J,J}''''(0) = 8\shrp{ -R(v,w)v, w} = -8\kappa(v,w).  \] 
	Die Gleichheit bei (1) gilt, da
	\begin{align*}
	&\pf{}{t} \shrp{R(\dot{\gamma}, J)\dot{\gamma}, W} \klam{= \shrp{ \Dd{t}R(\dot{\gamma}, J)\dot{\gamma}, W } + \shrp{R(\dot{\gamma},J)\dot{\gamma}, \Dd{t}W}}\\
=	&\pf{}{t} \shrp{R(\dot{\gamma}, W)\dot{\gamma}, J} = \shrp{ \Dd{t}R(\dot{\gamma}, W)\dot{\gamma}, J } + \shrp{R(\dot{\gamma},W)\dot{\gamma}, J'}\\
	&=\shrp{ \Dd{t}R(\dot{\gamma}, W)\dot{\gamma}, J } + \shrp{R(\dot{\gamma},J')\dot{\gamma}, W}
	\end{align*}
\end{itemize}
\paragraph{Zusammenfassung:} Wir haben gezeigt:
\[ \norm{J(t)}^2 = t^2 - \frac{1}{3} \shrp{ 
R(v,w)v, w
 } t^4
+ o(t^4) \]
für $t \pfeil{} 0$.\\

Gilt $\norm{v} = \norm{w} = 1$, $v\bot w$, dann $A(v,w) = 1$ und somit
\[ \shrp{R(v,w)v, w} = \kappa_p(v,w). \]
Daraus folgt
\[ \norm{J(t)}^2 = t^2 - \frac{1}{3} \kappa_p(v,w)t^4 + o(t^4) \]
und somit
\[ \norm{J(t)}= t - \frac{1}{6} \kappa_p(v,w)t^3 + o(t^3). \]


Wir wollen die Abweichungsgeschwindigkeit von geodätischen Kurven in $M$ mit der Abweichungsgeschwindigkeit solcher Kurven in $T_pM$ vergleichen.\\
Die Abweichungsgeschwindigkeit in $T_pM$ der Szrahlen $t \mapsto tv(s)$ und $t\mapsto tv(0)$ ist gerade
\[ \norm{ \pf{}{s}_{s=0 }tv(s)} = t \norm{ \pf{}{s}_{s = 0 } v(s) } = t\norm{w} = t. \]
Daraus folgt, dass die Differenz der Abweichungsgeschwindigkeit in $M$ und der in $T_pM$ gegeben ist durch
\[ - \frac{1}{6}\kappa_p(v,w)t^3  + o(t^3) \]
Daraus folgt, ist $\kappa > 0$, so ist die Abweichungsgeschwindigkeit in $M$ langsamer als in $T_pM$. Ist die Schnittkrümmung bei $p$ negativ, so ist die Abweichungsgeschwindigkeit in $M$ schneller als in $T_pM$.

\newpage
\section{Konjugationspunkte}
\Def{}
Sei $p \in M$, $\gamma$ eine Geodätische in $M$ mit $\gamma(0) = p$. Ein Punkt $\gamma(t_0)\neq p$ heißt \df{konjugiert} zu $p$ entlang $\gamma$, falls ein Jacobi-Feld $J\neq 0$ entlang $\gamma$ existiert, sodass 
\[J(0) = J(t_0) = 0.\]
Die \df{Vielfachheit} von $\gamma(t_0)$ ist dann die maximale Anzahl von linear unabhängigen Jacobi-Feldern mit dieser Eigenschaft.

\Bsp{}
Sei $ M = S^n \subset \R^{n+1}$ die Einheitssphäre. Dann ist $\kappa_p(\sigma) = 1$ konstant positiv.\\
Wir haben gezeigt
\[ J(t) = \sin(t) W(t) \]
mit $\norm{W(t)} = 1$ und $W\bot \dot{\gamma}.$\\
Ist $p \in S^n$, $p = \gamma(0)$, dann ist $-p = \gamma(\pi)$ konjugiert zu $p$. Daraus folgt für alle $p$, dass $-p$ konjugiert zu $p$ ist. Die Vielfachheit dieser Konjugation ist $n-1$.

\Bem{}
$J(t) = t\dot{\gamma}(t)$ ist ein Jacobi-Feld mit $J(0) = 0$, und $J(t) \neq 0$ für alle $t\neq 0$, falls $\dot{\gamma} (0) \neq 0$.\\
Daraus folgt, dass die Vielfachheit einer Konjugation immer höchstens $n-1$ ist für zwei verschiedene Konjugationspunkte.

\Prop{}
$q = \gamma(t_0)$ ist genau dann konjugiert zu $p = \gamma(0)$ entlang $\gamma$, wenn $t_0v$ ein kritischer Punkt von $\exp_{p}$ ist für $v = \dot{\gamma}(0)$.
\begin{Beweis}{}
Ist $J(0) = 0$, so folgt $J(t) = (\d \exp_p)_{tv}(tw)$ und somit
\[ 
0 = J(t_0) = (\d \exp_p)_{t_0v}(t_0w),
 \]
wobei $t_0w \neq 0$.
\end{Beweis}

\Def{}
Wir definieren den \df{Konjugationslokus} durch
\[ C(p) = \set{q \in M}{p, q \text{ sind konjugiert}} \]
\Bsp{}
Ist $M= S^n$, so gilt
\[ C(p) = \{-p\}. \]

\newpage
\section{Vollständige Mannigfaltigkeiten}
Sei $(M,g)$ eine Riemannsche Mannigfaltigkeiten.
\Def{}
$(M,g)$ heißt \df{geodätisch vollständig} bzw. \df{vollständig}, falls für alle $p \in M$ die Abbildung $\exp_p$ auf ganz $T_pM$ definiert ist.

\Bsp{}
\begin{itemize}
	\item $M = \R^n$ ist geodätisch vollständig.
	\item Die eingebettete Untermannigfaltigkeit $B := \set{x \in \R^n}{\norm{x} < 1} \subset \R^n$ mit der induzierten Metrik ist nicht vollständig.\\
	Allerdings kann man die Mannigfaltigkeit $B$ mit $\R^n$ identifizieren und dementsprechend eine Metrik auf $B$ einführen, sodass $B $ und $ \R^n$ isometrisch sind. Dadurch wird $B$ zu einer vollständigen Mannigfaltigkeit.
\end{itemize}

\Def{}
Seien $p,q \in M.$ Definiere die Distanz zwischen $p$ und $q$ durch
\[ d(p,q) := \inf \set{L(c)}{ c:p\mapsto q \text{ ist eine stückw. glatte Kurve} }. \]
Gilt $d(p,q) = 0$, so folgt $p = q$, da Geodätische lokal die Länge minimieren. Die anderen Axiome eines metrischen Raumes werden durch $d(p,q)$ ebenfalls erfüllt.\\
Dadurch wird $(M,d)$ zu einem metrischen Raum.


\printindex
\end{document}