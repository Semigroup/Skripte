%
% -------------------------------------------------------------------------------------------
% "THE BEER-WARE LICENSE"
% Jo, Ihr könnt mit diesem Code machen, was Ihr wollt, solange keinerlei Verantwortlichkeiten auf mich zurückfallen...
% Ansonsten kann ich Euch nur anregen, Wissen frei zu gestalten und allen Menschen zugänglich zu machen.
% Ja, und trinkt ein Bier oder eine Chai Latte oder einen Grünen Tee auf das Skript hier (oder ladet mich ein, wenn Ihr wollt).
% LG Akin
% -------------------------------------------------------------------------------------------
%

\documentclass[12pt]{book}

\usepackage[T1]{fontenc}
\usepackage[utf8]{inputenc}
\usepackage[ngerman]{babel}

\usepackage{tikz-cd}
\usetikzlibrary{babel}

\usepackage{amsfonts}
\usepackage{amssymb}
\usepackage{amsmath}
\usepackage{mathtools}
\usepackage{wasysym}
\usepackage{dsfont}
\usepackage{geometry}
\usepackage{makeidx}
\usepackage{booktabs}
\usepackage{hyperref}

\usepackage{enumerate}
\usepackage{adjustbox}

\newcommand{\ifLeer}[3]{\ifx&#1&\relax#2\relax\else\relax#3\relax\fi\relax}

\newcommand{\Def}[1]{\subsection{Definition\ifLeer{#1}{}{: #1}}}
\newcommand{\Bsp}[1]{\subsection{Beispiel\ifLeer{#1}{}{: #1}}}
\newcommand{\Lem}[1]{\subsection{Lemma\ifLeer{#1}{}{: #1}}}
\newcommand{\Bem}[1]{\subsection{Bemerkung\ifLeer{#1}{}{: #1}}}
\newcommand{\Kor}[1]{\subsection{Korollar\ifLeer{#1}{}{: #1}}}
\newcommand{\Satz}[1]{\subsection{Satz\ifLeer{#1}{}{: #1}}}
\newcommand{\Prop}[1]{\subsection{Proposition\ifLeer{#1}{}{: #1}}}

\newcommand{\QED}{\hfill $\square$}
\newcommand{\qed}{\hfill $\blacksquare$}

\newenvironment{Beweis}[1]{\paragraph{Beweis\ifLeer{#1}{}{: #1}\\}}{\QED}
\newenvironment{Beweisskizze}[1]{\paragraph{Beweisskizze\ifLeer{#1}{}{: #1}\\}}{\qed}

\newcommand{\df}[1]{\index{#1}\textbf{#1}}

\newcommand{\klam}[1]{\left(#1\right)}
\newcommand{\bet}[1]{\left|#1\right|}
\newcommand{\norm}[1]{\bet{\bet{#1}}}
\newcommand{\brak}[1]{\left[#1\right]}
\newcommand{\curv}[1]{\left\lbrace#1\right\rbrace}
\newcommand{\shrp}[1]{\left<#1\right>}
\newcommand{\quot}[1]{\glqq #1 \grqq\relax}
\newcommand{\set}[2]{\curv{\ifLeer{#2}{#1}{#1 ~ | ~ #2}}}
\newcommand{\grp}[2]{\shrp{\ifLeer{#2}{#1}{#1 ~ | ~ #2}}}

\newcommand{\A}{\mathcal{A}}
\newcommand{\B}{\mathcal{B}}
\newcommand{\C}{\mathbb{C}}
\newcommand{\D}{\mathcal{D}}
\newcommand{\E}{\mathcal{E}}
\newcommand{\F}{\mathcal{F}}
\newcommand{\G}{\mathcal{G}}
\renewcommand{\H}{\mathbb{H}}
\newcommand{\I}{\mathcal{I}}
\newcommand{\J}{\mathcal{J}}
\newcommand{\K}{\mathbb{K}}
\renewcommand{\L}{\mathcal{L}}
\newcommand{\M}{\mathcal{M}}
\newcommand{\N}{\mathbb{N}}
\renewcommand{\O}{\mathcal{O}}
\renewcommand{\P}{\mathbb{P}}
\newcommand{\Q}{\mathbb{Q}}
\newcommand{\R}{\mathbb{R}}
\renewcommand{\S}{\mathcal{S}}
\newcommand{\T}{\mathcal{T}}
\newcommand{\U}{\mathcal{U}}
\newcommand{\V}{\mathcal{V}}
\newcommand{\W}{\mathcal{W}}
\newcommand{\X}{\mathcal{X}}
\newcommand{\Y}{\mathcal{Y}}
\newcommand{\Z}{\mathbb{Z}}

\newcommand{\id}[1]{\text{Id}_{#1}}
\newcommand{\Ker}{\textsf{Kern}}
\newcommand{\Coker}{\textsf{Kokern}}
\newcommand{\Img}{\textsf{Bild}}
\newcommand{\Coimg}{\textsf{Kobild}}
\newcommand{\Hom}[3]{\textsf{Hom}_{#1}\left(#2, #3\right)}
\newcommand{\Aut}[2]{\textsf{Aut}_{#1}\left(#2\right)}
\newcommand{\Sym}[1]{\textsf{Symm}_{#1}}

\newcommand{\e}{\varepsilon}

\newcommand{\Pfeil}[1]{\overset{#1}{\longrightarrow}}
\newcommand{\pfeil}[1]{\overset{#1}{\rightarrow}}
\newcommand{\inj}[1]{\overset{#1}{\hookrightarrow}}
\newcommand{\Inj}[1]{\overset{#1}{\lhook\joinrel\longrightarrow}}
\newcommand{\surj}[1]{\overset{#1}{\twoheadrightarrow}}

\newcommand{\impl}[1]{\overset{#1}{\Rightarrow}}
\newcommand{\Impl}[1]{\overset{#1}{\Longrightarrow}}
\newcommand{\gdw}[1]{\overset{#1}{\Leftrightarrow}}
\newcommand{\Gdw}[1]{\overset{#1}{\Longleftrightarrow}}

\newcommand{\off}{\overset{o}{\subset}}
\newcommand{\abg}{\overset{c}{\subset}}

\newcommand{\gl}[1]{\overset{#1}{=}}
\newcommand{\grgl}[1]{\overset{#1}{\geq}}
\newcommand{\klgl}[1]{\overset{#1}{\leq}}
\newcommand{\gr}[1]{\overset{#1}{>}}
\newcommand{\kl}[1]{\overset{#1}{<}}
\newcommand{\isom}[1]{\overset{#1}{\cong}}

\newcommand{\supp}{\text{supp}}

\renewcommand{\i}{^{-1}}
\renewcommand{\phi}{\varphi}
\renewcommand{\d}{\text{d}}

\newcommand{\rot}{\text{rot}}

\renewcommand{\epsilon}{\varepsilon}
\newcommand{\sgn}{\text{sign}}
\newcommand{\Dd}[1]{\frac{\text{D}}{\d #1}}

\newcommand{\vol}{\mathrm{vol}}
\newcommand{\Ric}{\mathrm{Ric}}

\newcommand{\cln}[3]{#1 \colon #2 \pfeil{} #3}

\newcommand{\rott}[1]{%
\mathrel{\reflectbox{\rotatebox[origin=c]{270}{$#1$}}}}

\renewcommand{\Im}{\text{Im}}

\setlength{\marginparwidth}{20mm}

\makeindex
\date{\today}
\author{\href{mailto:tensor.produkt@gmx.de}{tensor.produkt@gmx.de}}

\makeindex

\newcommand{\eu}{\mathrm{e}}

\begin{document}
\title{Mitschrieb: Differentialtopologie II\\
SS 18}
\maketitle
\section*{Vorwort}
Dies ist ein Mitschrieb der Vorlesungen vom 16.04.18 bis zum ... des Kurses \textsc{Differentialtopologie II} an der Universität Heidelberg.\\
Dieses Dokument wurde \glqq{live}\grqq\ in der Vorlesung getext. Sämtliche Verantwortung für Fehler übernimmt alleine der Autor dieses Dokumentes.\\
Auf Fehler kann gerne hingewiesen werden bei folgende E-Mail-Adresse
\begin{center}
	\href{mailto:tensor.produkt@gmx.de}{tensor.produkt@gmx.de}
\end{center}
Ferner kann bei dieser E-Mail-Adresse auch der Tex-Code für dieses Dokument erfragt werden.

\setcounter{tocdepth}{1}
\tableofcontents

%Prüfungstag: Mittwoch, 18. Juli

\chapter{Einführung in die Riemannsche Geometrie}
\section{Überblick und Ideen}
\marginpar{Vorlesung vom 16.04.18}

Bisher können wir durch die äußere Ableitung
\[ \d : \Omega^p(M) \pfeil{} \Omega^{p+1}(M) \]
nur Differentialformen auf glatten Mannigfaltigkeiten ableiten, aber keine anderen Objekte wie zum Beispiel Vektorfelder. Wir können also auch nicht über Phänomene aus der Physik wie Beschleunigung zum Beispiel sprechen.

\paragraph{Ziel}
Wir wollen einen Rahmen finden, in dem Objekte wie zum Beispiel Vektorfelder abgeleitet werden können.

\Bsp{}
Sei $f:M \pfeil{} \R$ eine glatte Funktion. Gilt $\d f = 0$ und ist $M$ zusammenhängend, so ist $f$ konstant.\\
Hätten wir für ein Vektorfeld $\xi$ eine Ableitung $\d \xi$, dann sollte die Gleichung $\d \xi = 0$ implizieren, dass $\xi$ \textsl{konstant} ist.\\
Ist zum Beispiel $\xi$ auf $M = \R^n$ konstant, so ist $\xi$ parallel, im Sinne von, die einzelnen Tangentialvektoren, die im Bild von $\xi$ liegen, sind parallel.\\
Somit impliziert eine Ableitung für Vektorfelder ein Konzept von \textsl{Parallelismus}.

\paragraph{Problem}
Ein Konzept von Parallelismus kann nicht über Karten erklärt werden, weil Kartenwechsel im Allgemeinen nicht winkeltreu sind.

\Bsp{}
Sei $M = S^2 \subset \R^3$ die zweidimensionale Einheitssphäre. Sei $p \in S^2$ und $\xi(p) \in T_pS^2$.\\
$\gamma$ sei ein Großkreis, der durch $p$ in Richtung $\xi(p)$ geht. Ist $p_1$ ein weiterer Punkt auf $\gamma$, so lässt sich $\xi(p)$ \textsl{naiv} wie gewohnt in $\R^3$ von $p$ auf $p_1$ verschieben. Dies hat das offensichtliche Problem, das der so parallel verschobene Vektor im Allgemeinem nicht tangential an $S^2$ anliegt.\\
Diesen kann man nun orthogonal auf den Tangentialraum $T_{p_1}S^2$ projizieren. Dadurch erhält man einen Tangentialvektor $\xi(p_1) \in T_{p_1}S^2$. Durch dieses Prozedere lässt sich $\xi$ glatt auf $S^2$ fortsetzen. Wählt man weitere Punkte $p_i$ auf $\gamma$, die gegen einen Punkt $q$ am Äquator konvergieren und für die gilt
\[ d(p_i, p_{i+1}) \Pfeil{} 0 \]
dann erhalten wir einen Vektor $\xi(q) \in T_qS^2$. Dies nennt man den \df{Paralleltransport} von $\xi(p)$ entlang $\gamma$ zu $\xi(q)$.\\
Allerdings kann man $\xi(p)$ auch entlang eines weiteren Großkreises $\gamma_1$ verschieben. Verschiebt man entlang $\gamma_1$ wieder auf den Äquator und von dort wieder auf $q$, so erhält man einen anderen Tangentialvektor auf $q$.

\paragraph{Neues Phänomen}
Für allgemeine Mannigfaltigkeiten hängt der Paralleltransport vom Weg $\gamma$ ab; im Gegensatz zum Euklidischen Raum.

\subsection{Zurück zu Ableitungen von Vektorfeldern $\xi$}
Auf $M$ sei Parallelismus gegeben (zum Beispiel ist $M$ eingebettet im $\R^n$). $p\in M$ sei ein Punkt und $v \in T_pM$ sei ein Tangentialvektor. $\xi$ sei ein Vektorfeld auf $M$.\\
Sei $\gamma$ eine glatte Kurve mit $\gamma(0) = p$ und $\dot{\gamma} (0) = v$. $q$ sei ein Punkt auf $\gamma$. Durch den vorgegebenen Parallelismus lässt sich $\xi(p)$ entlang $\gamma$ verschieben. D.\,h., im Punkt $q$ haben wir die Vektoren $\xi(q)$ und $\tau^q_p\xi(p)$, wobei $\tau^q_p\xi(p)$ der Paralleltransport von $\xi(p)$ nach $q$ entlang $\gamma$ ist.

\paragraph{Idee}
Betrachte
\[ \xi(q) - \tau^q_p\xi(p) \in T_pM  \]
für $\d(p,q) \pfeil{} 0$. Dies bezeichnet man dann auch als die \df{kovariante Ableitung} von $\xi$ in Richtung $v$
\[ \nabla_v\xi \in T_qM \]
$\nabla_v$ nennt man dabei einen \df{Zusammenhang.} Diese hat folgende Eigenschaften:
\begin{itemize}
	\item $\nabla_v$ ist $\Omega^0(M)$-linear in $v$, d.\,h.
	\[ \nabla_{\lambda v + w}(\xi) = \lambda \nabla_{v} (\xi) + \nabla_{w} (\xi) \]
	für glatte Funktionen $\lambda : M \pfeil{} \R$.
	\item Sie ist $\R$-linear im zweiten Argument
	\[ \nabla_{v}(\xi + \eta) = \nabla_{v}(\xi) + \nabla_{v}(\eta) \]
	\item Ist $f : M \pfeil{} \R$ linear, so liegt folgende Produktregel vor
	\[ \nabla_{v}(f\cdot \xi) = f \cdot \nabla_{v}(\xi) + \nabla_{v}(f) \cdot \xi \]
	wobei
	\[ \nabla_{v} f := v(f) \]
\end{itemize}


\subsection{Geodätische}
Sei $\gamma$ eine (glatte) Kurve auf $M$. $\gamma$ heißt eine \df{Geodätische}, falls gilt
\[ \nabla_{\dot{\gamma}}\dot{\gamma} = 0 \]
Obige Bedingung ist in lokalen Koordinaten eine Differentialgleichung zweiter Ordnung.\\
Physikalisch gesprochen verschwindet die Beschleunigung. Geometrisch gesprochen ist $\gamma$ parallel entlang $\gamma$.

\Bsp{}
Sei $M$ eine Riemannsche Fläche im $\R^3$. Die Gleichung
\[ \nabla_{\dot{\gamma}}\dot{\gamma} = 0 \]
bedeutet
\[ \ddot{\gamma} \bot M \]
D.\,h., die Euklidische zweite Ableitung steht orthogonal auf der Fläche $M$.

\Bsp{}
\begin{itemize}
	\item Geraden sind Geodätische im Euklidischen Raum.
	\item Großkreise sind Geodätische auf Sphären.
	\item Allgemein sind Geodätische lokal kürzeste Kurven.
\end{itemize}

\newpage
\subsection{Parallelogramme}
Sei $p\in M$. $\mu, \lambda$ seien zwei Geodätische, die sich im Punkt $p$ schneiden mit $\mu(0) = \lambda(0) = p$.\\
$\mu, \lambda$ seien parametrisiert durch die Bogenlänge, d.\,h.,
\[ \norm{\dot{\lambda}(t)} = \norm{\dot{\mu}(t)} = 1 \]
für alle $t$. Setze $v := \dot{\mu}(0)$ und $w := \dot{\lambda}(0)$. Sei $\e > 0$.\\
Indem wir $w$ entlang $\mu$ verschieben, erhalten wir einen Vektor $\overline{w}$ auf $\mu(\e)$ und analog einen Vektor $\overline{v}$ auf $\lambda(\e)$.\\
Es gilt
\[ \norm{\overline{v}} = \norm{\overline{w}} = 1 \]
da der Paralleltransport eine Isometrie ist, wenn die Riemannsche Metrik kompatibel ist zum Zusammenhang $\nabla$.\\
Indem man $\overline{v}$ und $\overline{w}$ durch durch Bogenlänge parametrisierte Geodätische fortsetzt, erhält man Geodätische $\overline{\mu}$ und $\overline{\lambda}$. Dadurch erhält man dann Punkte $\overline{\lambda}(\e) $ und $\overline{\mu}(\e)$. Im Euklidischen würden die beiden Punkte zusammen fallen und das Parallelogramm schließen. Für allgemeine Riemannsche Mannigfaltigkeiten muss dies nicht der Fall sein, aber es gilt
\[ d(\overline{\mu}(\e), \overline{\lambda}(\e)) \in O(\e^2) \]
\marginpar{Vorlesung vom 18.04.18}
\Def{}
Definiere die \df{Torsion} des Zusammenhangs durch
\[ T(\xi, \eta) := \nabla_\xi \eta - \nabla_\eta \xi - [\xi, \eta] \]
wobei $[\xi, \eta]$ die \df{Lie-Klammer}\footnote{Lassen sich die beiden Vektorfelder als Koordinatenrichtungen schreiben, so gilt zum Beispiel $[\frac{\partial}{\partial x_i}, \frac{\partial}{\partial x_j}] = 0$} der beiden Vektorfelder $\xi$ und $\eta$ bezeichnet.\\
$T$ ist ein \df{Tensor}, d.\,h., $\mathcal{C}^{\infty}(M)$-linear.\\
$\nabla$ heißt \df{symmetrisch} bzw. \df{torsionsfrei}, falls $T = 0$.

\newcommand{\crv}{\text{R}}

\Lem{}
Ist $\nabla$ symmetrisch, dann gilt sogar
\[ d(\overline{\mu}(\e), \overline{\lambda}(\e)) \in O(\e^3) \]

Sei $u \in T_p(M)$ ein weiterer Tangentialvektor. $u_1$ sei der Paralleltransport von $u$ entlang $\lambda\overline{\mu}$. $u_2$ sei der Paralleltransport entlang $\mu \overline{\lambda}$.\\
Es liegt dann folgende asymptotische Gleichheit vor
\[ \norm{u_1 - u_2} \sim \epsilon^2 \crv(v,w)u  \]
$\crv(v,w)u$ heißt \df{Riemannscher Krümmungstensor}. Er ist definiert durch
\[ \crv(v,w)u := \nabla_v \nabla_wu - \nabla_w \nabla_v u - \nabla_{[v,w]} u \]

Wir werden nun im Folgenden mit den Formalen Definitionen beginnen.

\newpage
\section{Die Lie-Klammer}
Sei $M$ im Folgenden eine glatte $n$-dimensionale Mannigfaltigkeit und $X,Y : M \pfeil{} \T M$ glatte Vektorfelder auf $M$.

\newcommand{\CC}[1]{\mathcal{C}^{#1}}

\Lem{}
Es existiert genau ein glattes Vektorfeld $Z$ auf $M$, sodass gilt
\[Z(f) = X(Y(f)) - Y(X(f)) \]
für alle $f \in \CC{\infty}(M)$. Beachte, $X(f)$ bezeichnet die glatte Funktion, die sich ergibt durch
\[ X(f)(p) := X(p)(f) \]

\begin{Beweis}{}
\begin{itemize}
	\item Eindeutigkeit:\\
	Sei $p \in M$. $\{x_i\}$ seien lokale Koordinaten bei $p$. $X, Y$ lassen sich dann schreiben durch
	\begin{align*}
	X = \sum_i a_i \frac{\partial}{\partial x_i} && \text{ und } && Y = \sum_j b_j \frac{\partial}{\partial x_j}
	\end{align*}
	und es gilt
	\begin{align*}
	X(Yf) &= X\klam{\sum_j b_j \frac{\partial f}{\partial x_j}}\\
	&= \sum_i a_i \sum_j \frac{\partial }{\partial x_i} \klam{b_j \frac{\partial f}{\partial x_j}}\\
	&= \sum_{i,j} a_i \frac{\partial b_j}{\partial x_i} \frac{\partial f}{\partial x_j} + 
	\sum_{i,j} a_i b_j \frac{\partial^2 f}{\partial x_j\partial x_i} 
	\end{align*}
	bzw.
	\begin{align*}
		Y(Xf) =  \sum_{i,j} b_j \frac{\partial a_i}{\partial x_j} \frac{\partial f}{\partial x_i} + 
	\sum_{i,j} b_j a_i \frac{\partial^2 f}{\partial x_i\partial x_j} \\
	\end{align*}
	In der Differenz ergibt sich
	\begin{align*}
	X(Yf) - Y(Xf) &= \sum_{i,j} a_i \frac{\partial b_j}{\partial x_i} \frac{\partial f}{\partial x_j}
	- \sum_{i,j} b_i \frac{\partial a_j}{\partial x_i} \frac{\partial f}{\partial x_j}\\
	&= \sum_{i,j} \klam{a_i \frac{\partial b_j}{\partial x_i} - b_i \frac{\partial a_j}{\partial x_i} } \frac{\partial f}{\partial x_i}
	\end{align*}
	Lokal ist $Z$ also bestimmt durch
	\[ Z =  \sum_{i,j} \klam{a_i \frac{\partial b_j}{\partial x_i} - b_j \frac{\partial a_j}{\partial x_i} } \frac{\partial }{\partial x_j} \]
	\item Existenz:\\
	Durch obige Formel ist für jedes lokale Koordinatensystem ein $Z$ gegeben. Diese lassen sich global zu einem glatten Vektorfeld auf ganz $M$ zusammen setzen.
\end{itemize}
\end{Beweis}

\Def{}
Definiere nun die \df{Lie-Klammer} von $X$ und $Y$ durch
\[ Z:= [X,Y] = XY - YX\]

\Bem{}
Die Lie-Klammer hat folgende Eigenschaften
\begin{itemize}
	\item $[X,Y] = - [Y,X]$
	\item Für $a,b \in \R$ gilt
	\[ [aX_1 + bX_2, Y] = a[X_1, Y] + b[X_2, Y] \]
	\item Iteration: Für beliebige Vektorfelder $X,Y,Z$ gilt
	\[ [[X,Y], Z] = [ XY - YX, Z ] = XYZ- YXZ - ZXY + ZYX \]
	und
	\[ [[Y,Z], X] = [ YZ - ZY, X ] = YZX- ZYX - XYZ + XZY \]
	und
	\[ [[Z,X], Y] = [ ZX - XZ, Y ] = ZXY - XZY - YZX + YXZ \]
	Durch Aufsummieren ergibt sich
	\[ [[X,Y], Z] + [[Y,Z], X]  + [[Z,X], Y] = 0\]
	Dies nennt sich die \df{Jacobi-Identität}.
	\item Seien $f,g \in \CC{\infty}(M)$. Es gilt
	\[ [fX, gY] = fX(gY) - gY(fX) = f(X(g)Y - gXY) - g(Y(f)X - fYX) = fg[X,Y] + fX(g) Y - gY(f)X \]
\end{itemize}

\newcommand{\pf}[2]{\frac{\partial #1}{\partial #2}}
%\vspace{12mm}
Da eine Mannigfaltigkeit lokal wie $\R^n$ aussieht, lassen sich die bekannten Sätze zu Existenz, Eindeutigkeit und Abhängigkeit von Anfangsbedingungen von gewöhnlichen Differentialgleichungen von $\R^n$ auf $M$ verallgemeinern.

\Satz{}
Sei $M$ eine glatte Mannigfaltigkeit, $X$ ein glattes Vektorfeld auf $M$, $p \in M$ ein Punkt.\\
Dann existiert eine offene Umgebung $U \subset M$ von $p$ und ein $\delta > 0$ zusammen mit einer Abbildung
\[ \phi : (-\delta, \delta) \times U \Pfeil{} M \]
sodass $t \mapsto \phi(t,p)$ die eindeutige Lösung von
\begin{align*}
\pf{}{t} \phi(t,q) &= X(\phi(t,q)) && \forall q \in U\\
\phi(0,q) &= q
\end{align*}
ist.\\
Schreibweise:
\[ \phi_t(p) := \phi(t,p) \]
Die glatte Abbildung
\[ \phi_t : U\pfeil{} M \]
heißt \df{Fluss} von $X$ (in der Umgebung von $p$).

\Bem{}
Sei $\bet{s}, \bet{t}, \bet{s+t} < \delta$. Betrachte
\[ \gamma_1(t) := \phi(t, \phi(s,p)) \]
Das impliziert
\begin{align*}
\dot{\gamma_1} = X(\gamma_1) && \gamma_1(0) = \phi(s,p)
\end{align*}
und
\[ \gamma_1(t) := \phi(t+s, p) \]
impliziert
\begin{align*}
\dot{\gamma_2} = X(\gamma_2) && \gamma_2(0) = \phi(s,p)
\end{align*}
Aus der Eindeutigkeit folgt nun
\[ \gamma_1 = \gamma_2 \]
D.\,h.,
\[ \phi_{s+t} = \phi_s \circ \phi_t \]
Insbesondere gilt
\[ \id{M} = \phi_t\circ \phi_{-t} \]
Daraus folgt, dass jedes $\phi_t$ ein Diffeomorphismus ist. Die Menge aller $\{\phi_t\}_{t}$ nennt man eine \df{Einparameter-Untergruppe} von Diffeomorphismen.

\newpage
\section{Die Lie-Ableitung}
Seien $X,Y$ zwei Vektorfelder auf $M$, $p \in M$ ein Punkt.\\
Sei $\phi_t$ der Fluss auf $X$ mit
\begin{align*}
\pf{}{t} \phi(t,p) = X(\phi_t(p)) && \text{ und } && \phi_0(p) = p
\end{align*}
Definiere nun die \df{Lie-Ableitung durch}
\begin{align*}
(L_XY)(p):=\lim\limits_{h\pfeil{} 0} \frac{1}{h} \klam{
Y_p 
- (\d \phi_h)(Y_{\phi_{-h}(p)})
}\in T_p(M)
\end{align*}
wobei $Y_p = Y(p), \d \phi_h = \phi_{h,*}$. Die Lie-Ableitung leitet das Vektorfeld $Y$ bzgl. dem Fluss von $X$ im Punkt $p$ ab.

\Prop{}
\label{LieKlammerProp}
Es gilt
\[ L_XY = [X,Y] \]
Für den Beweis dieser Proposition benötigen wir ein Lemma:
\paragraph{Idee}
Sei $f: \R \pfeil{} \R$ glatt mit $f(0) = 0$. $f$ hat die Taylor-Entwicklung
\[ f(t) = t f'(0) + \frac{t^2}{2} f''(0) + \ldots =: t \cdot g(t) \]
Es gilt
\[ f(t) = tg(t) \]
und $f'(0) = g(0)$.\\
Wir brauchen nun folgende Verallgemeinerung dieser Beobachtung:

\Lem{}
Sei $M$ eine Mannigfaltigkeit, $f : (-\e, \e) \times M \pfeil{} \R$ glatt, $f(0,p) = 0$ für alle $p \in M$.\\
Dann existiert eine glatte Funktion $g: (-\e, \e) \times M \pfeil{} \R$ mit
\begin{align*}
f(t,p) = t\cdot g(t,p) && \text{ und } && \pf{f}{t}(0,p) = g(0,p)
\end{align*}
\begin{Beweis}{}
Wir definieren $g$ durch
\begin{align*}
g(t,p) := \int_{0}^{1} \klam{\pf{f}{s}} (s\cdot t,p) \d s
\end{align*}
Der Rest ist nachrechnen.
\end{Beweis}

\begin{Beweis}{\ref{LieKlammerProp}}
Sei $f \in \CC{\infty}(M)$. Definiere die Hilfsfunktion
\begin{align*}
h(t,p) := f(\phi_t(p)) - f(p)
\end{align*}
Aufgrund des Lemmas existiert ein $g$ mit
\begin{align*}
h(t,p) = t \cdot g(t,p) && \pf{h}{t}(0,p) = g(0,p)
\end{align*}
Es gilt
\begin{align*}
f \circ \phi_t = f + t g_t
\end{align*}
\end{Beweis}
\marginpar{Vorlesung vom 23.04.18}
\begin{Beweis}{\ref{LieKlammerProp}}
	Sei $f \in \CC{\infty}(M)$. Wir wollen Folgendes zeigen.
	\[ (L_XY)(f) = [X,Y](f) = XYf - YXf \]
	Definiere die Hilfsfunktion
	\begin{align*}
	h(t,p) := f(\phi_t(p)) - f(p).
	\end{align*}
	Da $h(0,p) = 0$, existiert aufgrund des Lemmas ein $g$ mit
	\begin{align*}
	h(t,p) = t \cdot g(t,p) &&\text{ und }&& \pf{h}{t}(0,p) = g(0,p).
	\end{align*}
	Es gilt
	\begin{align*}
	f \circ \phi_t = f + t g_t
	\end{align*}
	und
	\begin{align*}
	X_p(f) = \klam{\pf{}{t}_{t= 0}\phi_t(p)}(f) = \pf{}{t}_{t = 0} f(\phi_t(p)) = \pf{h}{t} (0,p) = g(0,p).
	\end{align*}
	Durch die erste der beiden obigen Gleichung erhalten wir
	\begin{align*}
	(\d \phi_h)(Y_{\phi_{-h}(p)})(f) &= Y_{\phi_{-h}(p)}(f\circ \phi_h)\\
	&= Y_{\phi_{-h}(p)} (f+tg_t).
	\end{align*}
	Setzt man dies in die Lie-Ableitung ein, so erhält man
	\begin{align*}
	(L_XY)(f) &= \lim\limits_{h\pfeil{} 0}\frac{1}{h}\klam{ Y_p - (\d \phi_h)(Y_{\phi_{-h}(p)})(f) }\\
	&= \lim\limits_{h\pfeil{} 0}\frac{1}{h}\klam{ Y_p - (Y_{\phi_{-h}(p)})(f) } - \lim\limits_{h\pfeil{} 0}\frac{1}{h}\klam{h (Y_{\phi_{-h}(p)})(g_h) }.
	\end{align*}
	Da gilt
	\[ \lim\limits_{h\pfeil{} 0}\frac{1}{h}\klam{h (Y_{\phi_{-h}(p)})(g_h) } =
	\lim\limits_{h\pfeil{} 0} (Y_{\phi_{-h}(p)})(g_h)  = Y_p(g_0) =YXf,  \]
	folgt
		\begin{align*}
	(L_XY)(f)&= \lim\limits_{h\pfeil{} 0}\frac{1}{h}\klam{ (Yf)_p - (Yf)_{\phi_{-h}(p)} } - Y_pXf\\
	&= X_pYf - Y_pXf.
	\end{align*}
\end{Beweis}
\paragraph{Folgerungen}
\begin{align*}
L_YX = -L_XY, && L_XX = 0
\end{align*}\\\\
Seien Vektorfelder $X,Y$ gegeben. Man kann zeigen, dass lokale Koordinaten $x_1,\ldots, x_n$ existieren mit
\[ X = \pf{}{x_1}. \]
Gilt ferner
\[ Y = \pf{}{x_2}, \]
so folgt
\[ [X,Y] = \pf{\partial}{x_1\partial x_2} - \pf{\partial}{x_2\partial x_1} = 0.  \]
Insofern ist das Verschwinden von $[X,Y]$ eine notwendige Bedingung für die Existenz von lokalen Koordinaten $x_1,\ldots, x_n$ mit
\begin{align*}
X = \pf{}{x_1} && \text{ und } && Y = \pf{}{x_2}.
\end{align*}

\subsection{Geometrische Interpretation der Lie-Klammer}
Seien $X,Y$ Vektorfelder. $\phi$ und $\psi$ seien korrespondierende Flüsse, $p\in M$ sei ein Punkt. Setze
\[ c(h) := \psi_{-h}\phi_{-h}\psi_h\phi_h(p). \]
Die Zuordnung $h \mapsto c(h)$ definiert eine glatte Kurve. Man kann zeigen
\[ \dot{c}(h) = 0. \]
Für Kurven $\gamma(t)$ mit $\dot{\gamma}(0) = 0$ lässt sich die zweite Ableitung definieren durch
\[ \ddot{\gamma}(t)(0):= \frac{\d^2}{\d t^2}_{t = 0} f(\gamma(t)) .\]
Dann ist $\ddot{\gamma}(0)$ eine Derivation.\\
Daraus folgt, dass $\ddot{c}(0)$ definiert ist, und es gilt
\begin{align*}
\ddot{c}(0) = 2[X,Y]_p.
\end{align*}

\newpage
\section{Riemannsche Mannigfaltigkeiten}
Sei $M$ eine glatte, $n$-dimensionale Mannigfaltigkeit.
\Def{}
Eine \df{Riemannsche Metrik} auf $M$ ist eine Zuordnung
\begin{align*}
p \longmapsto \shrp{\cdot ~|~\cdot}_p
\end{align*}
für $p\in M$, wobei $\shrp{\cdot ~|~\cdot}_p$ jeweils ein inneres Produkt\footnote{Inneres Produkt heißt hier eine symmetrische, positiv definite Bilinearform.} auf $T_pM$ ist. Ferner soll diese Zuordnung\df{glatt} sein in dem Sinne, dass für lokale Koordinaten $(U,x)$ die Funktionen
\begin{align*}
g_{i,j}(p) := \shrp{\pf{}{x_i}(p) ~|~ \pf{}{x_j}(p) }_p
\end{align*} 
für alle $i,j$ glatt sind auf $U$.\\
Wir werden manchmal $g(p)$ anstatt $\shrp{\cdot ~|~ \cdot}_p$ schreiben.\\
Das Paar $(M, \shrp{\cdot~|~ \cdot})$ heißt \df{Riemannsche Mannigfaltigkeit}.

\Def{}
Ein Diffeomorphismus $\phi: (M, \shrp{\cdot~|~\cdot}_M) \pfeil{} (N, \shrp{\cdot~|~\cdot}_N)$ heißt \df{Isometrie}, falls für alle $p \in M$ und $u,v \in T_pM$ gilt
\[ \shrp{u,v}_{M,p} = \shrp{ \d \phi_pu, \d \phi_pv}_{N,\phi(p)}. \]


\begin{enumerate}[(1)]
	\item \Bsp{} Sei $M = \R^n$. $x$ seien die Standardkoordinaten auf $\R^n$. Setzt man
	\[ \shrp{\pf{}{x_i}, \pf{}{x_j}}_p = \delta_{i,j} \]
	so erhält man die euklidische Metrik auf $\R^n$.
	\item Sei $f:M\pfeil{} N$ eine glatte Immersion. $(N,\shrp{\cdot~|~\cdot}_N)$ sei eine Riemannsche Mannigfaltigkeit, $M$ eine glatte Mannigfaltigkeit. Dann induziert $f$ eine Riemannsche Metrik $\shrp{\cdot ~|~\cdot}_M$ auf $M$ durch
	\[ \shrp{u|v}_M:= \shrp{\d f(u), \d f(v)}_N. \]
	Da $\d f$ injektiv ist, ist $ \shrp{u|v}_{M,p}$ positiv definit.
	\item \Bsp{} Es bezeichne $S^n = \set{(x_1,\ldots,x_{n+1}) \in \R^{n+1}}{x_1^2 + \ldots + x_{n+1}^2 = 1}$ die Einheitssphäre. Durch die Einebettung $S^n \subset \R^{n+1}$ erhalten wir eine Riemannsche Metrik auf $S^n$. $S^n$ zusammen mit dieser Metrik nennt man \df{Standardsphäre}.
	\item \textbf{Produktmetrik}: Seien $(M, g_m), (N,g_N)$ zwei Riemannsche Mannigfaltigkeiten. $\pi_1, \pi_2$ seien die korrespondierenden Projektionen von $M\times N$ auf $M$ bzw. $N$. Seien $u,v \in T_{(p,q)}(M\times N)$, setze
	\begin{align*}
	\shrp{u,v}_{p,q} := \shrp{\d \pi_1(u), \d \pi_1(v)}_{M,p} + \shrp{\d \pi_2(u), \d \pi_2(v)}_{N,q}. 
	\end{align*}
	$\shrp{u,v}_{p,q}$ ist eine Riemannsche Metrik auf $M \times N$, die sogenannte \df{Produktmetrik}.
	\item \Bsp{} Betrachte $T^n := S^1 \times \ldots \times S^1$. Ist $S^1$ mit der Standardmetrik versehen, so induziert uns dies eine Produktmetrik auf $T^n$. In diesem Fall spricht man vom \df{flachen Torus}.\\
	Für $n=2$ kann man $T^2$ in den $\R^3$ einbetten. Dadurch erhält man eine andere induzierte Metrik auf $T^2$, die nicht äquivalent zu obiger Produktmetrik ist. Diese beiden Tori sind nicht isometrisch.
\end{enumerate}

\Prop{}
Jede glatte Mannigfaltigkeit besitzt eine Riemannsche Metrik.

\begin{Beweis}{}
Sei $\set{(U_\alpha,x_\alpha)}{}$ eine offene Überdeckung von $M$ durch Karten und $\{f_\alpha\}$ eine glatte Partition der Eins bzgl. dieser Überdeckung.\\
Über $U_\alpha$ betrachte man die eindeutige Riemannsche Metrik $g^\alpha$, sodass
\[ (U_\alpha,g^\alpha) \Pfeil{x_\alpha} (\R^n, g_{eukl}) \]
eine Isometrie ist. Auf $M$ erhält man nun eine Riemannsche Metrik durch
\begin{align*}
g_p := \sum_{p\in U_\alpha} f_\alpha(p) g_p^\alpha.
\end{align*}
\end{Beweis}

\Def{}
Sei $c : \R \pfeil{} M$ eine glatte Kurve. Ein \df{Vektorfeld entlang einer Kurve} $c$ ist eine glatte Zuordnung
\[ t \longmapsto V(t) \in T_{c(t)}M \]

\Bem{}
Ein Vektorfeld entlang einer Kurve lässt sich im Allgemeinem nicht auf ein Vektorfeld einer offenen Umgebung der Kurve fortsetzen. Zum Beispiel könnte sich die Kurve selbst schneiden und $V$ die Ableitung der Kurve sein.

\paragraph{Notation}
Wir schreiben auch für $v \in T_pM$
\[ \norm{v} := \sqrt{ \shrp{v|v}_p } \]

\Def{}
Für eine Kurve $c$ definiere wir die \df{Länge} durch
\[ L^b_a(c) := \int_{a}^{b} \norm{\dot{c}(t)} \d t \]
\marginpar{Vorlesung vom 25.04.18}

Sei $(M,g)$ eine Riemannsche Mannigfaltigkeit, die obendrein orientiert ist. $(U,x)$ und $(V,y)$ seien orientierte Karten auf $M$, die sich schneiden.\\
Wir erinnern an folgendes Lemma aus Differentialtopologie I.
\Lem{}
Auf $U\cap V$ gilt
\[ f\d x_1 \wedge \ldots \wedge \d x_n = g \d y_1 \wedge \ldots \wedge \d y_n \]
genau dann, wenn
\[ f = \det\klam{ \pf{y_i}{x_j} } g \]
gilt.\\\\
Auf einer orientierten Karte $U$ sind Funktionen $g_{i,j} : U \pfeil{} \R$ gegeben durch
\[ g_{i,j} = \shrp{\pf{}{x_i}|\pf{}{x_j} } \]
für $p \in U$. Setze ferner $X_i := \pf{}{x_i}$.\\
Sei $e_1,\ldots, e_n$ eine {Orthonormalbasis} (ONB) für $T_pM$ bzgl. $g_p$. Dann lässt sich $X_i$ darstellen durch
\[ X_i = \sum_j a_{i,j}e_j. \]
Wir erhalten so eine $n\times n$-Matrix $A := (a_{i,j})_{i,j}$. Es gilt
\begin{align*}
g_{i,j} :=& \shrp{X_i|X_j}\\
=& \shrp{ \sum_k a_{i,k}e_k| \sum_l a_{j,l}e_l  }\\
=& \sum_{k,l} a_{i,k}a_{j,l} \shrp{e_k|e_l}\\
=& \sum_k a_{i,k}a_{j,k}.
\end{align*}
Daraus folgt
\[ (g_{i,j})_{i,j} = AA^T. \]
Dies impliziert insbesondere
\[ \det(g_{i,j}) = \det(A)^2 > 0. \]
Insbesondere ist $\sqrt{\det(g_{i,j})} = \bet{\det A}$ wohldefiniert. Durch den Transformationssatz folgt nun im Punkt $p$
\begin{align*}
vol(X_1, \ldots, X_n) = \bet{\det A} \cdot vol(e_1, \ldots, e_n) = \bet{\det A},
\end{align*}
da $vol(e_1, \ldots, e_n) = 1$. Daraus folgt insbesondere
\[ vol(\pf{}{x_1}, \ldots, \pf{}{x_n}) = \sqrt{\det (g_{i,j})}. \]
Auf $(V,y)$ erhält man analog
\[ vol(Y_1, \ldots, Y_n) = \sqrt{\det (h_{i,j})}. \]
für
\[ Y_i = \pf{}{y_i} \]
und
\[ h_{i,j} = \shrp{Y_i|Y_j}. \]
Man erhält hierdurch
\begin{align*}
\sqrt{\det (h_{i,j})} &= vol(Y_1, \ldots, Y_n)\\
&= \det \klam{ \pf{x_i}{y_j} } vol(X_1, \ldots, X_n)\\
&=  \det \klam{ \pf{x_i}{y_j} } \sqrt{\det (g_{i,j})}.
\end{align*}
Mit dem obigen Lemma folgt nun auf $U\cap V$
\[ \sqrt{\det (g_{i,j})} \d x_1\wedge \ldots \wedge \d x_n = \sqrt{\det (h_{i,j})} \d y_1\wedge \ldots \wedge \d y_n.  \]
Durch Verkleben erhalten wir eine glatte $n$-Form $\nu \in \Omega^n(M)$.

\Def{}
$\nu$ heißt \df{Riemannsche Volumenform} von $M$. $\nu$ ist durch die Riemannsche Metrik eindeutig festgelegt.

\Def{}
Wenn $M$ kompakt ist, setzen wir
\[ vol(M) := \int_M \nu < \infty. \]
$vol(M)$ heißt das \df{Riemannsche Volumen}.\\
Wenn $vol(K)$ unbeschränkt ist über kompakte Untermannigfaltigkeiten (mit Rand) $K \subset M$, dann sagen wir, dass $M$ unendliches Volumen habe.

\Bem{}
Oft sieht man in der Literatur $\nu = \d V = \d vol$, obwohl $\nu$ im Allgemeinem nicht im Bild des Randhomomorphismus
\[ \d : \Omega^{n-1}(M) \Pfeil{} \Omega^n(M) \]
liegt.

\newpage
\section{Zusammenhänge}
Sei $\Gamma(\T M)$ der Vektorraum der glatten Schnitte von $\T M$, d.\,h., $\Gamma(\T M)$ ist der Vektorraum der glatten Tangentialvektorfelder auf $M$.

\Def{}
Ein \df{Zusammenhang} auf $M$ ist eine Abbildung
\begin{align*}
\nabla : \Gamma(\T M) \times \Gamma(\T M ) &\Pfeil{} \Gamma(\T M)\\
(X,Y) & \longmapsto \nabla_XY ,
\end{align*}
sodass:
\begin{enumerate}[(1)]
	\item Für $f,g \in \CC{\infty}(M)$ gilt
	\[\nabla_{fX_1 + gX_2}Y = f\nabla_{X_1}Y + g \nabla_{X_2}Y.\]
	\item Ferner gilt
	\[ \nabla_X(Y_1 + Y_2) = \nabla_X Y_1 + \nabla_X Y_2. \]
	\item Zuletzt wird folgende Produktregel gefordert
	\[ \nabla_X(f\cdot Y) = f\nabla_XY + X(f)Y. \]
\end{enumerate}

\paragraph{In Lokalen Koordinaten} Sei $(U,x)$ eine Karte. $X,Y$ seien Vektorfelder der Gestalt
\begin{align*}
X = \sum_ia_i \pf{}{x_i}, && Y = \sum_j b_j\pf{}{y_j}.
\end{align*}
Es gilt
\begin{align*}
\nabla_XY &= \sum_ia_i \nabla_{\pf{}{x_i}} (\sum_j \pf{}{y_j})\\
&= \sum_ia_i\sum_j \nabla_{\pf{}{x_i}} (b_j \pf{}{y_j})\\
&= \sum_{i,j} a_i 
\klam{
\pf{b_j}{x_i} \pf{}{x_j} + b_j \nabla_{\pf{}{x_i}} \pf{}{y_j} 
}.
\end{align*}
Wir dröseln die Terme $\nabla_{\pf{}{x_i}} \pf{}{y_j}$ weiter auf und erhalten
\begin{align*}
\nabla_{\pf{}{x_i}} \pf{}{y_j}  = \sum_k \Gamma_{i,j}^k \pf{}{x_k}.
\end{align*}
Die Funktionen $\Gamma_{i,j}^k$ nennt man \df{Christoffel-Symbole} des Zusammenhangs.\\
Wir erhalten final
\begin{align*}
\nabla_XY &= \sum_{i,k} a_i 
\klam{
\pf{b_k}{x_i} \pf{}{x_k} +\sum_{i,j} a_i b_j \sum_k \Gamma_{i,j}^k \pf{}{x_k} 
}\\
&= \sum_k \klam{
\sum_i  a_i \pf{b_k}{x_i}
+
\sum_{i,j} a_i b_j \Gamma_{i,j}^k
}
\pf{}{x_k}.
\end{align*}

\Bem{}
Die Gleichung
\begin{align*}
\nabla_XY &= \sum_k \klam{
	\sum_i  a_i \pf{b_k}{x_i}
	+
	\sum_{i,j} a_i b_j \Gamma_{i,j}^k
}
\pf{}{x_k}
\end{align*}
impliziert, dass $\nabla_X Y$ eine lokale Operation ist. Denn für $p \in M$ gilt
\begin{align*}
(\nabla_XY)(p) &= \sum_k \klam{
\sum_i a_i(p) \pf{b_k}{x_i}(p)
+ \sum_{i,j} a_i(p) b_j(p) \Gamma_{i,j}^k(p)
}\pf{}{x_k}_{|p}.
\end{align*}
D.\,h., $(\nabla_XY)(p)$ hängt nur von $X(p), Y(p)$ und $\pf{b_k}{x_i}(p)$ ab.

\Def{}
Sei $V = V(t)$ ein Vektorfeld entlang einer Kurve $c(t)$ in $M$.\\
Eine \df{kovariante Ableitung} ist eine Zuordnung
\[ \Dd{t} : \V_c \Pfeil{} \V_c, \]
wobei $\V_c$ den Raum aller Vektorfelder entlang $c$ bezeichnet, sodass
\begin{enumerate}[(1)]
	\item $\Dd{t}(V+W) = \Dd{t}V + \Dd{t} W$
	\item und $\Dd{t}(fV) = f \Dd{t} V + \pf{f}{t} V$ für $f : \R \pfeil{}\R$ glatt gelten.
	\item Wenn ferner ein Vektorfeld $X$ auf $M$ existiert mit $X(c(t)) = V(t)$, dann soll gelten
	\[ \nabla_{\dot{c}}X = \Dd{t}(V) .\]
\end{enumerate}

\Prop{}
Sei $M$ eine glatte Mannigfaltigkeit mit Zusammenhang $\nabla$. Sei $c$ eine Kurve auf $M$. Dann existiert eindeutig eine kovariante Ableitung $\Dd{t}$ mit obigen Eigenschaften.
\begin{Beweis}{}
\begin{itemize}
	\item Eindeutigkeit: Sei $V(t)$ ein Vektorfeld entlang $c(t)$. In lokalen Koordinaten $x_1, \ldots, x_n$:
	\begin{align*}
	V(t) = \sum_i v_i(t) \pf{}{x_i}, && c(t) = (x_1(t), \ldots, x_n(t))
	\end{align*}
	Es gilt dann
	\begin{align*}
	\Dd{t} V &= \sum_i \klam{
v_i \Dd{t} \klam{
\pf{}{x_i}_{|c(t)}
}	+ v_i' \pf{}{x_i}
}\\
&= \sum_i \klam{
v_i \nabla_{\dot{c}(t)} \pf{}{x_i} 
+ v_i'\pf{}{x_i}
}.
	\end{align*}
	
	\item Existenz: Sei $(U_\alpha, x^\alpha)$ eine offene Überdeckung von $M$ durch Karten. Definiere $\Dd{t}$ auf $U_\alpha$ durch
	\begin{align*}
	\Dd{t} V := \sum_i \klam{
		v_i \nabla_{\dot{c}(t)} \pf{}{x_i} 
		+ v_i'\pf{}{x_i}
	}.
	\end{align*}
	Auf $U_\alpha \cap U_\beta$ stimmen diese $\Dd{t}$ überein wegen Eindeutigkeit und definieren somit $\Dd{t}$ überall.
\end{itemize}
\end{Beweis}

\Prop{}
Sei $c$ eine Kurve in $M$, $p = c(0)$. Sei ferner $V^0 \in T_pM$ ein Tangentialvektor. Dann existiert genau ein Vektorfeld $V$ entlang $c$ mit
\[ \Dd{t}V = 0 \]
und
\[ V(0) = V^0. \]

\Def{}
Sei $V$ ein Vektorfeld entlang einer Kurve $c$. $V$ heißt \df{parallel} entlang $c$, falls
\[ \Dd{t}V =0. \]


\marginpar{Vorlesung vom 30.04.18}

\Prop{}
Sei $c$ eine Kurve in $M$ und $V^0\in T_{c(t_0)}M$ ein Vektor bei $c(t_0)$.
Dann existiert genau ein Vektorfeld $V(t)$ entlang $c(t)$, das die Eigenschaften
\begin{align*}
V(t_0) &= V^0\\
\Dd{t} V &= 0
\end{align*}
erfüllt.
\begin{Beweis}{}
\begin{itemize}
	\item Existenz und Eindeutigkeit in lokalen Koordinaten:\\
	Existiert so ein $V$, so gilt
	\begin{align*}
	0 = \Dd{t} V=\sum_k (v'_k + \sum_{i,j} x_i'v_j \Gamma_{i,j}^k) \pf{}{x_k}.
	\end{align*}
	Daraus folgt
	\[ v'_k = - \sum_{i,j} (x_i' \Gamma_{i,j}^k) v_j \]
	für alle $k = 1,\ldots, n$. Dadurch ergibt sich ein System von linearen gewöhnlichen Differentialgleichungen. Aus der Theorie der gewöhnlichen Differentialgleichungen wissen wir, dass es in kleinen Umgebungen von $t$ eindeutige Lösungen für $v_k(t)$ gibt für alle $t$. Da obiges DGL linear ist, sind die $v_k(t)$ für alle $t\in \R$ definiert.
	\item Globale Existenz:\\
	Sei $t_1 > t_0$ beliebig. Der Kurvenabschnitt $c[t_0, t_1]$ ist kompakt und wird folglich überdeckt durch endlich viele Karten. Man kann nun eine lokale Lösung von Karte zu Karte fortsetzen. Die lokalen Lösungen stimmen auf den Durchschnitten der Karten überein wegen ihrer Eindeutigkeit.
\end{itemize}
\end{Beweis}

\Bem{}
\begin{enumerate}[1.)]
	\item Wir erhalten folgende Abbildung
	\begin{align*}
	\tau : T_{c(t_0)}M & \Pfeil{} T_{c(t_1)}M\\
	V^0 & \longmapsto V(t_1).
	\end{align*}
	Diese Abbildung nennt man den \df{Paralleltransport} von $c(t_0)$ nach $c(t_1)$ entlang $c$.\\
	Die Linearität des vorangegangenen Differentialgleichungssystems stellt die Linearität von $\tau$ sicher. Durch Umkehren der Zeit erhält man eine lineare Abbildung
	\begin{align*}
	\tau' : T_{c(t_1)}M & \Pfeil{} T_{c(t_0)}M.
	\end{align*}
	Naheliegenderweise gilt
	\[ \tau' = \tau\i .\]
	Hierdurch folgt insbesondere, dass $\tau$ ein Isomorphismus ist. D.\,h., wir können Tangentialräume an verschiedenen Punkten mittels Paralleltransporte vergleichen.\\
	Daher die Terminologie \textsl{Zusammenhang}.
	\item $\Dd{t}V$ ordnet auch Vektoren an Punkten mit $\dot{c}(t) = 0$ zu. Diese Vektoren müssen nicht Null sein!
	\Bsp{}
	Wenn $c(t) = p$ konstant ist, dann ist $V(t)$ eine Kurve in $T_pM$. $\Dd{t}V$ ist dann einfach die Ableitung von $V(t)$ nach $t$, also $V'(t)$ im euklidischen Sinne.
\end{enumerate}


\newpage
\section{Der Levi-Civita-Zusammenhang}
Sei $(M, g)$ eine Riemannsche Mannigfaltigkeit.

\Def{}
Ein Zusammenhang $\nabla$ auf $M$ heißt \df{kompatibel} mit der Riemannschen Metrik $g$, falls für jede Kurve $c$ und für alle parallele Vektorfelder $V,W$ entlang $c$ gilt:
\[ \shrp{V,W} = \text{konst.} \]
d.\,h., der Paralleltransport ist in diesem Fall sogar eine Isometrie.

\Prop{}
$g$ und $\nabla$ sind genau dann kompatibel, wenn für alle Vektorfelder $V,W$ entlang einer beliebigen Kurve $c$ gilt
\[ \pf{}{t}\shrp{V,W} = \shrp{ \Dd{t}V, W } +\shrp{V, \Dd{t} W}. \]

\begin{Beweis}{}
\begin{itemize}
	\item[$\Leftarrow )$] Seien $V,W$ parallele Vektorfelder entlang $c$. Dann gilt
	\[\pf{}{t}\shrp{V,W} = \shrp{ \Dd{t}V, W } +\shrp{V, \Dd{t} W} = \shrp{ 0, W } +\shrp{V, 0} = 0. \]
	$\shrp{V,W}$ ist als Funktion in $t$ konstant.
	\item[$\Rightarrow )$] $\shrp{,}$ und $\nabla$ seien kompatibel. Sei $\{ P_1(t_0), \ldots, P_n(t_0) \} \subset T_{c(t_0)}M$ eine Orthonormalbasis. Durch den Paralleltransport erhalten wir die parallelen Vektorfelder $P_1, \ldots, P_n$ entlang $c$.\\
	Durch die Kompatibilität bleiben die $P_1, \ldots, P_n$ an jeder Stelle auf $c$ eine Orthonormalbasis. Seien $V,W$ nun beliebige Vektorfelder entlang $c$. Wir können dann schreiben
	\begin{align*}
	V = \sum_i v_i P_i && \text{ und } && W = \sum_j w_j P_j.
	\end{align*}
	Es gilt dann
	\[ \Dd{t}V = \sum_i (v_i' P_i + v_i \Dd{t}P_i) = \sum_i v_i' P_i. \]
	Und somit
	\[ \shrp{\Dd{t}V, W} = \shrp{ \sum_i v_i' P_i , \sum_j w_j P_j  } = \sum_{i,j} v_i'w_j\shrp{P_i,P_j} = \sum_i v_i'w_i. \]
	Und analog
	\[ \shrp{V, \Dd{t}W} =  \sum_i v_iw_i'. \]
	Zusammen also
	\[\shrp{\Dd{t}V, W} + \shrp{V, \Dd{t}W} = \sum_i (v_i'w_i + v_i w_i'). \]
	Ferner gilt
	\[ \shrp{V,W} = \ldots = \sum_i v_i w_i. \]
	Mit der Produktregel folgt nun
	\[ \pf{}{t} \shrp{V,W} = \sum_i (v_i'w_i + v_i w_i'). \]
\end{itemize}
\end{Beweis}

\Kor{}
$g$ und $\nabla$ sind genau dann kompatibel, wenn gilt
\[ X\shrp{Y,Z} = \shrp{\nabla_x Y, Z} + \shrp{Y, \nabla_X Z} \]
für beliebige Tangentialvektorfelder $X,Y,Z$ auf $M$.
\begin{Beweis}{}
Für einen Punkt $p \in M$ wähle eine Kurve $c$ mit $c(0) = p$ und $\dot{c}(0) = X(p)$. Es gilt dann
\[ X(p)\shrp{Y,Z} = \pf{}{t}_{|t = 0} \shrp{Y_{c(t)}, Z_{c(t)}}. \]
\end{Beweis}

\Def{Symmetrie von Zusammenhängen}
Ein Zusammenhang $\nabla$ heißt \df{symmetrisch}, wenn gilt
\[ \nabla_{X} Y - \nabla_YX = [X,Y]. \]
In lokalen Koordinaten für $X = \pf{}{x_i}$ und $Y = \pf{}{y_j}$ gilt dann
\[ \nabla_{\pf{}{x_i}}\pf{}{x_j} - \nabla_{\pf{}{x_j}}\pf{}{x_i} = [\pf{}{x_i}, \pf{}{x_j}] = 0. \]
Daraus folgt dann
\[ \nabla_{\pf{}{x_i}}\pf{}{x_j} = \nabla_{\pf{}{x_j}}\pf{}{x_i}. \]
Für die Christoffel-Symbole bedeutet dies
\[ \Gamma_{i,j}^k = \Gamma_{j,i}^k. \]

\Bem{}
Definiere die \df{Torsion} durch
\[ T(X,Y) := \nabla_{X}Y - \nabla_{Y}X - [X,Y]. \]
$T$ ist linear über $\CC{\infty}(M)$. D.\,h., $T$ ist ein Tensor.\\
Ferner ist ein Zusammenhang genau dann symmetrisch, wenn er torsionsfrei ist.

\Satz{Levi-Civita}
Sei $(M,g)$ eine Riemannsche Mannigfaltigkeit. Dann existiert genau ein Zusammehang $\nabla$ auf $M$, sodass gilt:
\begin{enumerate}[1.)]
	\item $\nabla$ und $g$ sind kompatibel.
	\item $\nabla$ ist symmetrisch.
\end{enumerate}
Diesen Zusammenhang nennen wir den \df{Levi-Civita-Zusammenhang} bzw. den \df{Riemannschen Zusammenhang}.

\begin{Beweis}{}
\begin{enumerate}[option]
	\item[Eindeutigkeit] Seien $X,Y,Z$ beliebige Tangentialvektorfelder auf $M$. Es gilt dann
	\[ X\shrp{Y,Z} = \shrp{\nabla_{X}Y, Z} + \shrp{Y, \nabla_{X} Z} \]
	und
	\[ Y\shrp{Z,X} = \shrp{ \nabla_{Y} Z, X } + \shrp{Z, \nabla_{Y} X} \]
	und
	\[ Z\shrp{X,Y} = \shrp{\nabla_{Z}X, Y} + \shrp{X, \nabla_Z Y}. \]
	Wir addieren die ersten beiden Zeilen und subtrahieren die dritte. Dadurch erhalten wir
	\begin{align*}
	&X\shrp{Y,Z} + Y\shrp{Z,X} - Z \shrp{X,Y} \\
	=&
	\shrp{Y, \nabla_XZ - \nabla_ZX} + \shrp{ X, \nabla_YZ - \nabla_YZ }\\
	+& \shrp{Z, \nabla_XY + \nabla_YX}\\
	\gl{\nabla \text{ symm}}& \shrp{Y, [X,Z]} + \shrp{X, [Y,Z]} + \shrp{Z, [X,Y] + 2 \nabla_YX}\\
	=&  \shrp{Y, [X,Z]} + \shrp{X, [Y,Z]} + \shrp{Z, [X,Y] }+ 2 \shrp{Z,\nabla_YX}\
	\end{align*}
	Daraus erhalten wir für $\nabla$
	\begin{align*}
	&\shrp{Z,\nabla_YX} =\\
	 &\frac{1}{2} \klam{
		X\shrp{Y,Z} + Y\shrp{Z,X} - Z\shrp{X,Y} - \shrp{Y,[X,Z]} - \shrp{X, [Y,Z]} - \shrp{Z, [X,Y]}
	}
	\end{align*}
Daraus folgt die Eindeutigkeit von $\nabla$.
\item[Existenz] Definiere $\nabla_YX$ durch obige Gleichung. Dann bleibt nachzurechnen, dass $\nabla$ ein symmetrischer und kompatibler Zusammenhang ist.
\end{enumerate}

\end{Beweis}
\marginpar{Vorlesung vom 02.05.18}

\section{Geodätische Kurven}
\Def{}
Sei $(M,g)$ eine Riemannsche Mannigfaltigkeit. Sei $\nabla$ der Levi-Civita-Zusammenhang auf $M$.\\
{Geodätische} sind Kurven auf $M$ mit Beschleunigung Null, d.\,h., eine glatte Kurve $c : I \pfeil{} M $ heißt \df{geodätisch}, falls
\[ \Dd{t}\dot{c} = 0 \]
gilt.

\Bsp{}
Betrachte $\R^n$ mit der Euklidischen Metrik. Durch den Levi-Civita-Zusammenhang werden alle Christoffel-Symbole Null. Gilt
\[ 0 = \Dd{t}\dot{\gamma} = \ddot{\gamma}, \]
so muss $\dot{\gamma}$ konstant gleich $a$ sein. Ergo ist $\gamma(t) = at +b$ eine Gerade.
\vspace{6 mm}\\

Sei $\gamma$ eine Geodätische. Betrachte
\[ \pf{}{t} \shrp{\dot{\gamma}, \dot{\gamma}} = 2 \shrp{\Dd{t}\dot{\gamma}, \dot{\gamma}} =2 \shrp{0, \dot{\gamma}} = 0, \]
da $\gamma$ geodätisch ist. Somit ist $\norm{\gamma'(t)}$ konstant gleich $c \in \R_{\geq 0}$.\\
Sei $c\neq 0$. $0,t$ seien in $I$. Dann
\begin{align*}
L_0^t(\gamma) &= \int_{0}^t \norm{\dot{\gamma}(\tau)} \d\tau = \int_{0}^t c \d\tau = ct.
\end{align*}
D.\,h., die Bogenlänge ist proportional zum Parameter $t$. Ist insbesondere $c = 1$, dann sagen wir, dass $\gamma$ durch die Bogenlänge parametrisiert sei.
\paragraph{In lokalen Koordinaten $x$} lässt sich $\gamma$ darstellen durch
\[ \gamma(t) = (x_1(t), \ldots, x_n (t)) .\]
Sei $V(t)$ ein Vektorfeld entlang $\gamma$. $V$ hat die Gestalt
\[ V(t) = \sum_i v_i(t) \pf{}{x_i}_{|\gamma(t)}. \]
Es gilt allgemein
\begin{align*}
\Dd{t}V = \sum_k \klam{
v_k' + \sum_{i,j} x_i' v_j \Gamma_{i,j}^k
}
\pf{}{x_k}.
\end{align*}
Für $V(t) = \dot{\gamma(t)}$ gilt $v_k(t) = x_k'(t)$. Dann gilt
\begin{align*}
0 = \Dd{t}\dot{\gamma} = \sum_k \klam{
	x_k'' + \sum_{i,j} x_i' x_j' \Gamma_{i,j}^k
}
\pf{}{x_k}.
\end{align*}
Daraus folgt für alle $k$
\begin{align*}
	x_k'' = - \sum_{i,j} x_i' x_j' \Gamma_{i,j}^k.
\end{align*}
Dadurch erhalten wir ein System von gewöhnlichen Differentialgleichungen 2. Ordnung. Auf dem Tangentialbündel $\T M$ kann dieses System umgeschrieben werden in ein System 1. Ordnung. Seien die Koordinaten $x$ definiert auf $U \subset M$. Ein Tangentialvektor kann geschrieben werden als eine Linearkombindation
\[ \sum_i y_i \pf{}{x_i}. \]
Dann sind $(x_1, \ldots, x_n, y_1, \ldots, y_n)$ lokale Koordinaten auf $\T M$, definiert in $\T U$.\\
Die Abbildung
\[ t \longmapsto (\gamma(t), \dot{\gamma}(t)) \]
definiert eine glatte Kurve in $\T M$. Hierfür gilt
\begin{align*}
y_k &= x_k\\
y_k' &= - \sum_{i,j} \Gamma_{i,j}^ky_iy_j.
\end{align*}
Dies ist ein System von Differentialgleichungen 1. Ordnung auf $\T M$. Wir wenden den Satz über Existenz, Eindeutigkeit und Abhängigkeit von Anfangsbedingungen an auf dieses System. Es folgt dann:

\Prop{}
Für alle $p \in M$ existieren $\delta, \e_1 > 0$, eine offene Umgebung $V \subset M$ von $p$ und eine glatte Abbildung
\[ \gamma : (-\delta, \delta) \times U \Pfeil{} M, \]
wobei
\[ U = \set{(q,v) \in V \times T_qM}{\norm{v} < \e_1}, \] 
sodass
\[ t \longmapsto \gamma(t,q,v) \]
die eindeutige Geodätische in $M$ ist mit
\begin{align*}
\gamma(0,q,v) = q && \text{ und } && \dot{\gamma(0,q,v)} = v.
\end{align*}

\Lem{Homogenität von Geodätischen}
Ist die Geodäte $\gamma(t,q, v)$ definiert für $\bet{t} < \delta$, so ist die Geodäte $\gamma(at,q,v)$ definiert für $a > 0$ und $\bet{t} < \frac{q}{a}$, und es gilt
\[ \gamma(at, q, v) = \gamma(t,q,av). \]
\begin{Beweis}{}
Setze $c(t) := \gamma(at, q,v)$. Dann ist $c(0) = q$ und $\dot{c}(0) = a \dot{\gamma}(0,q,v) = av$. Damit erfüllt $c$ dieselben Anfangsbedingungen wie $\gamma(t,q, av)$. Es bleibt zu zeigen, dass $c$ tatsächlich eine Geodätische ist. Es gilt
\[ \Dd{t}\dot{c} = \nabla_{\dot{c}} \dot{c} = \nabla_{a\dot{\gamma}} (a\dot{\gamma}) = a^2 \nabla_\{\dot{\gamma}\} \dot{\gamma} = a^2 \cdot 0 = 0. \]
Aus der Eindeutigkeit folgt nun
\[ c(t) = \gamma(t,q,av). \]
\end{Beweis}\\
Betrachte insbesondere $\bet{t} < 2 = \frac{\delta}{\delta / 2}$ und $a = \frac{\delta}{2}$. Setze $\e = \frac{\delta \e_1}{2}$. Dann ist $\gamma(t,q,v)$ definiert für $\bet{t}<2$ und $\norm{v} < \epsilon$.

\Def{Die Exponentialabbildung}
Sei $q \in V$, $v \in T_qM$ mit $\norm{v} < \e$. Definiere die Abbildung
\begin{align*}
 \exp_q(v) = \gamma(1, q, v).
\end{align*}
Für $v \neq 0$ gilt
\[ \exp_q(v) = \gamma(1, q, v) = \gamma(\norm{v}, q, \frac{v}{\norm{v}}) \].
Bezeichnet $B_0(\e)$ den $\e$-Ball in $T_qM$, so ist $\exp_q$ eine Abbildung vom Typ
\[ \exp_q : B_0(\e) \subset T_qM \Pfeil{} M. \]
Wir schreiben allgemein auch $\exp$ statt $\exp_q$.

\Bem{}
Die Bezeichnung obiger Abbildung als Exponentialabbildung kommt aus der Theorie der Lie-Gruppen. Ist $G$ eine Lie-Gruppe, so erhält man eine Abbildung
\[ \exp : \mathfrak{g} := T_1G \Pfeil{} G, \]
wobei $\mathfrak{g}$ die Lie-Algebra von $G$ bezeichnet. D.\,h., in diesem Fall gilt
\[ \exp(\mathfrak{a} + \mathfrak{b}) = \exp(\mathfrak{a}) \cdot \exp(\mathfrak{b}). \]

\Prop{}
Es existiert ein $\e > 0$, sodass
\[ \exp : B_0(\e) \]
ein Diffeomorphismus auf sein Bild ist.
\begin{Beweis}{}
Betrachte
\[ (\d \exp)_0(v) = \pf{}{t}_{| t = 0} \exp(tv) = \pf{}{t}_{| t = 0} \gamma(1, q, tv) =  \pf{}{t}_{| t = 0} \gamma(t, q, v) = v . \]
D.\,h., $\d \exp_0$ ist die Identität auf $B_0(\epsilon)$. Der Satz über umkehrbare Funktionen impliziert, dass $\exp$ ein lokaler Diffeomorphismus in der Nähe von $0$ ist.
\end{Beweis}

\Bsp{}
\begin{enumerate}[1)]
	\item Sei $M = \R^n$. Betrachte
	\[ \exp_0 : T_0\R^n\isom{} \R^n \Pfeil{\id{\R^n}} \R^n \]
	\item Sei $S^n \subset \R^{n+1}$ die Einheitssphäre. Betrachte
	\[ \exp_q  : B_0(\pi) \Pfeil{} S^n - \{-q\} \]
	wobei $q$ den Nordpol bezeichnet. $\exp_q$ ist dann tatsächlich surjektiv auf $S^n - \{-q\}$. Allerdings gilt
	\[ \exp_q(\partial B_0(\pi)) = \{-q\}. \]
\end{enumerate}

\Satz{Gauss-Lemma}
Es gilt
\[ \shrp{\d \exp_v(v), \d \exp_v(w)} = \shrp{v,w} \]
für $v,w \in T_qM$. Dabei wurde stillschweigend die Identifikation
\[ T_v(T_qM) \isom{} T_qM \]
angenommen.
\begin{Beweis}{}
Wir schreiben $w = w_{||} + w_\bot$ mit $w_{||} \in \R\cdot {v}$ und $w_\bot \in v^\bot$. Die Linearität impliziert, dass es genügt die Aussage für $w_{||}$ und für $w_\bot$ jeweils zu beweisen.
\begin{enumerate}[1)]
	\item Für $w_{||} = \lambda v$:\\
	Es gilt
	\begin{align*}
	\shrp{\d \exp_v(v), \d \exp_v(\lambda v)} = {\lambda} \norm{\d \exp_v(v)}^2
	\end{align*} 
	und
	\[ \shrp{v,\lambda v} = \lambda \norm{v}^2. \]
	Zu zeigen bleibt
	\[ \norm{\d \exp_v(v)} = \norm{v}. \]
	Es gilt nun
	\begin{align*}
	\norm{\d \exp_v(v)} &= \norm{\pf{}{t}_{|t = 0} \gamma(1,q, v+tv)} =\norm{ \pf{}{t}_{|t = 0} \gamma(1+t, q, v)} = \norm{v}
	\end{align*}
	\item Für $w_\bot $:\\
	Wir schreiben $w = w_\bot$ und es gilt $\shrp{v,w} = 0$. Zu zeigen ist
	\[ \shrp{\d \exp_v(v), \d \exp_v(w)} = 0. \]
	Sei $v(s)$ eine Kurve in $T_qM$ mit $v(0) = v, \dot{v} = w$ und $\norm{v(s)}$ konstant. Setze
	\[ f(t,s) := \exp(tv(s)). \]
	$f$ ist eine parametrisierte Fläche. Es gilt dann
	\[ \shrp{\d \exp_v(v), \d \exp_v(w)} = \shrp{\pf{f}{t}, \pf{f}{s}}(t = 1, s = 0). \]
	Wir behaupten, dass $\shrp{\pf{f}{t}, \pf{f}{s}}$ unabhängig von $t$ ist, denn:
	\begin{align*}
	\pf{}{t} \shrp{\pf{f}{t}, \pf{f}{s}} = \shrp{\Dd{t} \pf{f}{t}, \pf{f}{s}} +  \shrp{ \pf{f}{t}, \Dd{t}\pf{f}{s}}
	\end{align*}
	Nun ist $\pf{}{t} \pf{f}{t}$ gleich Null, da $\gamma$ eine Geodätische ist. Es gilt nun
	\begin{align*}
	\shrp{\Dd{t} \pf{f}{s}, \pf{f}{t}} \gl{Symmetrie} \shrp{\Dd{s}\pf{f}{t}, \pf{f}{t} } = \frac{1}{2} \pf{}{s} \norm{\pf{f}{t}}^2 = 0,
	\end{align*}
	da $\norm{v(s)}$ konstant ist.\\
	
	Betrachte wieder
	\[ \shrp{\pf{f}{t}, \pf{f}{s}}(1, s) = \shrp{\pf{f}{t}, \pf{f}{s}} (0,s). \]
	Nun gilt
	\[ \pf{f}{s}(0,s) = 0, \]
	da $f(0,s) = \exp(0\cdot v(s)) = \exp(0) = q$ konstant in $s$ ist.
\end{enumerate}
\end{Beweis}

\marginpar{Vorlesung vom 07.05.18}
\Def{}
Sei $\exp_p : T_pM \pfeil{} M$ die Exponentialabbildung und $\e > 0$ so, dass $\exp_p$ auf $B_0(\e)$ injektiv ist.\\
Für $0<r<\e$ nennen wir dann
\[ B_p(r) := \exp_p(B_0(r)) \]
den \df{geodätischen Ball} und
\[ S_p(r) := \exp_p(\partial B_0(r)) \]
die \df{geodätische Sphäre} um $p$ von Radius $r$.

\Bem{Interpretation: Gauss-Lemma}
Wir können nun das Gauss-Lemma wie folgt ausdrücken:
\begin{center}
\emph{
	Geodätische Kurven durch $p$ stehen senkrecht auf geodätischen Sphären.
}	
\end{center}

\Prop{Geodätische minimieren lokal die Länge von Kurven.}
Sei $p \in M$ und $\e > 0$ so klein, dass $\exp_p : B_0(\e) \pfeil{} M$ injektiv ist. Sei $\gamma: [0,1] \pfeil{} B := B_p(r)$ für $r<\e$ mit $\gamma(0) = p$ eine Geodätische.\\
Sei $c : [0,1] \pfeil{} M$ eine stückweise glatte Kurve mit $c(0) = p$ und $c(1) = q:= \gamma(1)$.\\
Dann gilt
\[ L(c) \geq L(\gamma). \]
Ferner gilt Gleichheit genau dann, wenn $c$ und $\gamma$ dasselbe Bild haben.
\begin{Beweis}{}
\paragraph{Idee:} Wir schreiben $c = c(s)$ in Polarkoordinaten:
\[ c(s) = \exp ( r(s) \cdot v(s) ) \]
für $r > 0, s > 0$ und $\norm{v(s)} = 1$. Wir nehmen dabei zunächst an, dass $c[0,1] \subset B$. Ferner nehmen wir ohne Einschränkung an, dass $c(s)\neq p$ für $s > 0$. Setze
\[ f(r,s) := \exp (r \cdot v(s)). \] 
Dann gilt
\[ c(s) = f(r(s), s). \]
Daraus folgt
\[ \dot{c}(s) = \pf{f}{r} \cdot r' + \pf{f}{s}. \]
Und hieraus
\begin{align*}
\norm{\dot{c}(s)}^2 &= 
\norm{\pf{f}{r} \cdot r' }^2 
+ 2 \shrp{\pf{f}{r} r, \pf{f}{s}}
+ \norm{\pf{f}{s}}^2\\
&= \bet{r'}^2 \cdot \norm{\pf{f}{r}}^2
+ 2r' \shrp{ \pf{f}{r}, \pf{f}{s} }
+ \norm{\pf{f}{s}}^2\\
&= \bet{r'}^2 \cdot 1 + 2r' \cdot 0 + \norm{\pf{f}{s}}^2,
\end{align*}
denn $\shrp{\pf{f}{r} r, \pf{f}{s}} = 0$ nach Gauss-Lemma und $\norm{\pf{f}{r}} = \norm{v(s)} = 1$.\\
Es folgt also
\[ \norm{\dot{c}(s)}^2 = \bet{r'}^2 + \norm{\pf{f}{s}}^2 \geq \bet{r'}^2. \]
Wähle nun $\delta > 0$ klein, und betrachte
\[ \int_{\delta}^{1} \norm{\dot{c}(s)} \d s \geq_{\delta}^1 \bet{r'(s)} \d s \geq \int_{\delta}^1r'(s)\d s = r(1) - r(\delta) \pfeil{\delta \pfeil{} 0} r(1) = L(\gamma).  \]
Ferner gilt
\[ \int_{\delta}^{1} \norm{\dot{c}(s)} \d s \pfeil{\delta \pfeil{} 0}  L(c).  \]
Gilt Gleichheit, so muss
\[ \norm{\pf{f}{s}} = 0 \]
gelten. Daraus folgt aber, dass $f(r,s)$ konstant in $s$ ist. Ergo
\[ f(r,s) = \exp (r \cdot v(0)). \]
Insofern haben in diesem Fall $c$ und $\gamma$ tatsächlich dasselbe Bild.\\

Wenn nun $c[0,1]$ nicht in $B$ enthalten ist, dann sei $s_0$ der kleinste Wert $s$, sodass $c(s_0) \in \partial B$. Es gilt
\[ L_0^1(c) \geq L_0^{s_0}(c) \geq L(\gamma_1) = r \geq L(\gamma).\]
$\gamma_1 : p \mapsto c(s_0)$ ist eine Geodätische.
\end{Beweis}

\Bem{}
\begin{enumerate}[1.)]
	\item Man kann auch zeigen:\\
	Ist $\gamma$ eine Kurve parametrisiert proportional zur Bogenlänge, sodass
	\[ L(\gamma) \leq L(c) \]
	für alle Kurven $c$ mit denselben Randpunkten gilt, so muss $\gamma$ eine Geodätische sein.
	\item Isometrien erhalten Geodätische.
\end{enumerate}

\newpage
\section{Krümmung}
\Bsp{}
\begin{itemize}
	\item Die Krümmung eines Kreises von Radius $r$ definieren wir durch $\frac{1}{r}$.
	\item Wir betrachten nun Kurven in $\R^2$, die durch die Bogenlänge parametrisiert sind.\\
	Sei dazu $c$ eine solche Kurve mit $\ddot{c(s)} \neq 0$ für ein $s$. Betrachte $s_1, s_2, s_3$ nahe bei $s$. Da die zweite Ableitung nicht verschwindet, sind $c(s_1), c(s_2)$ und $c(s_3)$ nicht kolinear.\\
	Daraus folgt, dass $c(s_1), c(s_2)$ und $c(s_3)$ auf einem eindeutig bestimmten Kreis mit Radius $R$ liegen. Für $s_1, s_2, s_3 \pfeil{} s$ erhält man einen Grenzkreis, den sogenannten oskulierenden Kreis in $c(s)$.\\
	Die Krümmung von $c$ im Punkt $c(s)$ definiert man nun als $\frac{1}{R}$, die Krümmung dieses oskulierenden Kreises.\\
	Es gilt nun ferner
	\[ \frac{1}{R} = \bet{\ddot{c}(s)}. \]
	\item Kurven in $\R^3$:\\
	Wir fixieren wieder $s$. Sei $\ddot{c}(s) \neq 0$. $c(s_1), c(s_2)$ und $c(s_3)$ definieren dann eine Ebene in $\R^3$. Laufen $s_1, s_2, s_3$ nach $s$, so definieren sie eine Grenzebene, die oskulierende Ebene.\\
	Ferner erhält man in dieser oskulierenden Ebene den oskulierenden Kreis mit Radius $R$. Die Krümmung bei $c(s)$ definieren wir dann wieder als die Krümmung $\frac{1}{R}$ des oskulierenden Kreises. Es gilt nun
	\[ 0 = \pf{}{s} \norm{\dot{c}(s)}^2 = \pf{}{s}\shrp{\dot{c}, \dot{c}} = 2\shrp{\ddot{c}, \dot{c}} \]
	ergo steht $\ddot{c}(s)$ orthogonal auf $\dot{c}(s)$. Beide liegen in der oskulierenden Ebene und spannen diese auf.
	\item Flächen und Euler:\\
	Sei $M$ eine zweidimensionale Mannigfaltigkeit im $\R^3$. Sei $p \in M$ und $\nu_p$ ein Einheitsnormalenvektor, d.\,h.,
	\[ \nu_p \bot T_pM \text{ und } \norm{\nu_p} = 1. \]
	Sei ferner $v \in T_pM$ mit $\norm{v} = 1$. $\nu_p$ und $v$ spannen eine Ebene $E_v$ auf. Schneidet man diese mit $M$, so erhält man eine Kurve
	\[ E_v\cap M = \text{ Kurve }c_v. \]
	$c_v$ sei hierbei durch Bogenlänge parametrisiert mit $c_v(0) = p$ und $\dot{c_v}(0) = v$. Es gilt nun
	\[ \ddot{c_v}(0) \bot T_pM. \]
	Dann existiert genau ein $\kappa_v \in \R$, sodass
	\[ \ddot{c_v}(0) = \kappa_v \nu_p. \]
	Es gilt
	\[ \kappa_{-v} = \kappa_v, \]
	insofern erhalten wir eine Funktion
	\[ \kappa : \R P^1 \Pfeil{} \Pfeil{} \R. \]
	
	\Satz{Satz von Euler}
	Es existieren eindeutige Richtungen $v_1,v_2 \in \R P^1$, sodass
	\[ k_1 := \kappa_{-v_1} = \min_v \kappa_v \]
	und
	\[ k_2 := \kappa_{v_2} = \max_v\kappa_v. \]
	Es gilt ferner
	\[v_1 \bot v_2\]
	und
	\[ \kappa_v = k_1 \cos^2\theta + k_2 \sin^2\theta \]
	wobei $\theta = \angle (v, v_1)$.	 
	
\end{itemize}



\marginpar{Vorlesung vom 09.05.18}

\paragraph{Krümmung von Flächen nach Gauss}
Sei $M^2 \subset \R^3$ eine orientierte Fläche, $p\in M$. Sei ferner $\nu_p \in T_pM^\bot$ ein Einheitsnormalenvektor orthogonal auf $M$ am Punkt $p$, sodass $(\nu_p, v, w)$ positiv orientiert ist, wobei $(v,w)$ positiv orientiert in $T_pM$ sei.\\
Dies induziert die \df{Gauss-Abbildung}:
\begin{align*}
\nu: M &\Pfeil{} S^2\\
p &\longmapsto \nu_p
\end{align*}
Ist $A \subset M$ eine Umgebung um $p$, so kann man die \df{Gauss-Krümmung} definieren durch
\begin{align*}
\kappa(p) := \lim\limits_{A\pfeil{} p} \frac{\vol(\nu(A))}{\vol(A)}.
\end{align*}

\Bsp{}
\begin{enumerate}[1.)]
	\item Sei $M = S^2 = S^2_1$ die Einheitssphäre. Dann ist $\nu = \id{S^2}$. Daraus folgt $\kappa(p) = 1$ für alle $p \in M$.
	\item Sei $M = S^2_r$ die Sphäre von Radius $r$. Dann gilt
	\[ \vol(\nu(A)) = \frac{1}{r^2}\vol(A). \]
	Daraus folgt
	\[ \kappa(p) = \lim\limits_{A\pfeil{} p} \frac{\vol(\nu(A))}{\vol(A)} = \frac{1}{r^2}. \]
	\item Ist $M$ eine Ebene, so sind alle $\nu_p$ parallel zueinander. Daraus folgt, dass $\nu$ konstant ein Punkt ist. Und somit gilt $\kappa(p) = 0$ für alle $p\in M$.
	\item Sei $M$ ein Zylinder, $p \in M$. Ist $A$ eine kleine Umgebung um $p$, so induziert die Nabe bei $b$ eine Strecke auf dem Äquator von $S^2$. Die Längsachse des Zylinders induziert nur einen Punkt in $S^2$. Insofern ist $\nu(A)$ eine Strecke in $S^2$. Es folgt $\vol(\nu(A)) = 0$ und $\kappa(p) = 0$.\\
	Daraus folgt, der Zylinder ist \textbf{nicht} gekrümmt!
\end{enumerate}

\Satz{Beziehung Gauss-Euler}
Es gilt
\[ \kappa(p) = \kappa_1(p) \cdot \kappa_2(p),\]
wobei $\kappa(p)$ die Gauss-Krümmung und $\kappa_1(p), \kappa_2(p)$ die Eulerschen Minimal- und Maximal-Krümmungen bezeichnet.

\Bsp{}
\begin{enumerate}[1)]
	\item Betrachte $S^2_r \subset \R^3$. Dann ist $\kappa_1 = \kappa_2 = \frac{1}{r}$. Insbesondere gilt
	\[ \kappa_1 \kappa_2 = \frac{1}{r^2} = \kappa(p). \]
	\item Betrachte den Zylinder. Dann ist $\kappa_2 = \frac{1}{r}$ bei einem Radius von $r$ und $\kappa_1 = 0$. Es folgt
	\[ \kappa(p) = 0 = \kappa_1 \kappa_2. \]
	\item Betrachte die Fläche $z = \frac{a}{2}(x^2 - y^2)$ für $a > 0$. Dann gilt
	\[ \pf{}{x} \pf{}{x}(\frac{ax^2}{2}) = a = \kappa_2 > 0 \]
	und
	\[ \pf{}{y} \pf{}{y}(-\frac{ay^2}{2}) = -a = \kappa_1 < 0 \]
	bei $p= (0,0,0)$. Folglich gilt
	\[ \kappa_1 \kappa_2 = -a^2 < 0. \]
	Insofern handelt es sich hierbei um eine Fläche negativer Krümmung.
\end{enumerate}

\paragraph{Krümmung nach Riemann}
Idee: Sei $M$ eine $n$-dimensionale Mannigfaltigkeit und $p \in M$. Sei $\sigma \subset T_pM$ ein zweidimensionaler Untervektorraum. Betrachte
\begin{align*}
\exp_p : B_0(\e) \subset T_pM \Pfeil{\isom{}} U,
\end{align*}
wobei $U$ den geodätischen Ball um $p$ bezeichnet. $F^2 := \exp_p(\sigma \cap B_\e(0))$ ist dann eine Fläche in $U$. $F$ erhalte die induzierte Metrik von $M$.\\
Dann sei $\kappa(p, \sigma)$ definiert als die Krümmung von $F$ im Punkt $p$ nach Euler-Gauss.\\
Formell: Sei $\nabla$ der Levi-Civita-Zusammenhang auf der Riemannschen Mannigfaltigkeit $(M, \shrp{,})$. Wir definieren eine Abbildung
\begin{align*}
R : \Gamma(\T M )^3 & \Pfeil{} \Gamma(\T M)\\
(X,Y,Z) & \longmapsto R(X,Y)Z
\end{align*}
wobei
\[ R(X,Y) Z := \nabla_{Y} \nabla_{X} Z - \nabla_X \nabla_{Y} Z + \nabla_{[X,Y]}Z. \]
In lokalen Koordinaten $\{x_i\}$ mit $X = \pf{}{x_i}$ und $Y = \pf{}{x_j}$ gilt
\[ [X,Y] = 0 \]
und insbesondere
\[ R(X,Y) = \nabla_{Y}\nabla_{X} - \nabla_{X} \nabla_{Y}. \]
Ferner
\[ R(X,Y) \pf{}{x_k} =: \sum_l R^l_{i,j,k} \pf{}{x_l}. \]

\paragraph{Eigenschaften}
Sind $f,g : M \pfeil {} \R$ und $X,Y \in \Gamma(\T M)$ glatt, so gilt:
\begin{itemize}
	\item $R(fX_1+gX_2,Y)Z = fR(X_1,Y)Z + gR(X_2,Y)Z$.
	\item $R(X, fY_1 + gY_2)Z = fR(X,Y_1)Z + gR(X,Y_2)Z$.
\end{itemize}
Ferner gilt
\begin{align*}
R(X,Y)(fZ) &= \nabla_Y \nabla_{X} (fZ) - \nabla_{X} \nabla_{Y} (fZ) + \nabla_{[X,Y]} (fZ)\\
&= \nabla_{Y}( f\nabla_{X} Z + X(f)Z) - \nabla_{X}(f\nabla_{Y} Z + Y(f)Z) + f\nabla_{[X,Y]} Z + [X,Y](f) Z\\
&=f \nabla_{Y} \nabla_{X} Z + Y(f) \nabla_{Y} Z + YX(f) Z\\
&- f\nabla_{X} \nabla_{Y} Z - X(f) \nabla_{Y} Z - Y(f) \nabla_{X} Z - XY(f)Z \\
&+ f\nabla_{[X,Y]} Z + XY(f) Z -YX(f)Z\\
&= f R(X,Y)Z
\end{align*}
und insbesondere
\[ R(X,Y)(fZ_1 + gZ_2) = f R(X,Y) Z_1 + g R(X,Y)Z_2.  \]
Daraus folgt, dass $R$ ein Tensor ist, der sogenannte \df{Riemannsche Krümmungstensor}. (Dies erklärt den Term $\nabla_{[X,Y]}Z$.)\\
Es folgt auch, dass $(R(X,Y)Z)_p$ am Punkt $p \in M$ nur von den Vektoren $X(p), Y(p)$ und $Z(p)$ abhängt.\\
Ferner gilt:
\label{SymmetrienRiemannscherKrümmungstensor}
\begin{enumerate}[1)]
	\item $R(X,Y)Z + R(Y,X) Z = 0$ (offensichtlich).
	\item Symmetrie von $\nabla$ + Jacobi-Identität für $[,]$ impliziert die \df{Bianchi-Identität}
	\[ R(X,Y)Z + R(Y,Z)X  + R(Z,X)Y = 0. \]
	\item Es gilt $\shrp{R(X,Y)Z, W} + \shrp{R(X,Y)W, Z} = 0 $, denn
	\begin{align*}
	\shrp{R(X,Y)Z,Z} &= \shrp{ \nabla_{Y} \nabla_{X} Z - \nabla_{X} \nabla_{Y} Z + \nabla_{[X,Y]} Z, Z }\\
	&= \shrp{\nabla_{Y} \nabla_{X} Z, Z}  - \shrp{\nabla_{X} \nabla_{Y}Z, Z } + \shrp{\nabla_{[X,Y]} Z, Z}\\
	&= Y\shrp{\nabla_{X} Z, Z}
	- \shrp{\nabla_{X} Z, \nabla_{Y} Z}
	- X\shrp{\nabla_{Y} Z, Z }
	+ \shrp{\nabla_Y Z, \nabla_{X} Z}
	+ \frac{1}{2} [X,Y] \shrp{Z,Z}\\
	&= 0,
	\end{align*}
	wobei
	\begin{align*}
	Y\shrp{\nabla_XZ,Z} &= \shrp{ \nabla_{Y} \nabla_{X} Z, Z } + \shrp{\nabla_{X} Z, \nabla_{Y} Z}\\
	X\shrp{\nabla_YZ,Z} &= \shrp{ \nabla_{X} \nabla_{Y} Z, Z } + \shrp{\nabla_{Y} Z, \nabla_{X} Z}\\
	[X,Y] \shrp{Z,Z} &= 2 \shrp{ \nabla_{[X,Y]} Z,Z }.
	\end{align*}
	\item Ferner gilt
	\begin{align*}
	\shrp{R(X,Y)Z, W} + \shrp{ R(Y,Z)X ,W} + \shrp{ R(Z,X)Y , W} &= 0\\
	\shrp{R(Y,Z)W, X} + \shrp{ R(Z,W)Y ,X} + \shrp{ R(W,Y)Z , X} &= 0\\
	\shrp{R(Z,W)X, Y} + \shrp{ R(W,X)Z ,Y} + \shrp{ R(X,Z)W , Y} &= 0\\
	\shrp{R(W,X)Y, Z} + \shrp{ R(X,Y)W ,Z} + \shrp{ R(Y,W)X , Z} &= 0.
	\end{align*}
	Indem man alle Zeilen aufaddiert, erhält man
	\begin{align*}
	0 &= \shrp{ R(Z,X)Y , W} + \shrp{ R(W,Y)Z , X} + \shrp{ R(X,Z)W , Y} +\shrp{ R(Y,W)X , Z}  \\
	&= 2 \shrp{R(Z,X)Y,W} - 2 \shrp{ R(Y,W)Z,X }.
	\end{align*}
	Ergo gilt auch folgende Symmetrie
	\[ \shrp{R(X,Y)Z, W} = \shrp{R(Z,W)X,Y}.  \]
\end{enumerate}

In lokalen Koordinaten $(x_1, \ldots, x_n)$ setzen wir
\[ X_i := \pf{}{x_i}. \]
$X,Y,Z \in \Gamma(\T M)$ schreiben wir als
\[ X = \sum_i x_i X_i,~~ Y = \sum_i y_i X_i~ \text{ und }~ Z = \sum_i z_i X_i. \]
Dann gilt
\[ R(X,Y)Z = \sum_{i,j,k} x_i y_j z_k R(X,i,X_j)X_k = \sum_{i,j,k} x_i y_j z_k R^l_{ijk} X_l \]
wobei
\[ R(X_i,X_j)X_k = \sum_l R_{jjk}^l X_l. \]
Insbesondere gilt
\begin{align*}
R(X_i, X_j)X_k &= \nabla_{X_j} \nabla_{X_i} X_k - \nabla_{X_i} \nabla_{X_j} X_k\\
&= \nabla_{X_j} \sum_l \Gamma_{ik}^l X_l - \nabla_{X_i} \sum_l \Gamma_{jk}^l X_l\\
&= \sum_l \sum_a (\Gamma_{ik}^l \Gamma_{jl}^a  - \Gamma_{jk}^l \Gamma_{il}^a)X_a
\end{align*}
Ergo
\[ R_{ijk}^a = \sum_l (\Gamma_{ik}^l \Gamma_{jl}^a  - \Gamma_{jk}^l \Gamma_{il}^a) \]
\marginpar{Vorlesung vom 14.05.18}
\paragraph{Schnittkrümmung}
Sei $p \in M$ ein Punkt und $\sigma \subset T_pM$ ein zweidimensionaler Untervektorraum. Sei $\{x,y\}$ eine Basis für $\sigma$. Die Fläche des von $x$ und $y$ aufgespannten Parallelogramms ist
\[ A(x,y) := \sqrt{\norm{x}^2 \norm{y}^2 - \shrp{x,y}}. \]
Wir betrachten
\[ \kappa(x,y) := \frac{\shrp{R(x,y)x,y}}{ A(x,y)^2 }. \]
\Lem{}
$\kappa(x,y)$ hängt nicht von der Wahl der Basisvektoren $x,y$ für $\sigma$ ab.
\begin{Beweis}{}
Jede andere Basis von $\sigma$ erhält man aus $\{x,y\}$ durch Anwendung der folgenden drei elementaren Transformationen:
\begin{align*}
\{x,y\} &\Impl{} \{y,x\}\\
\{x,y\} &\Impl{} \{\lambda x, y\} \text{ für }\lambda \neq 0\\
\{x,y\} &\Impl{} \{x + \lambda y, y\}
\end{align*}
Überprüfe dann, dass $\kappa(x,y)$ invariant bleibt unter diesen drei Transformationen.
\end{Beweis}

Aufgrund obigen Lemmas dürfen wir die \df{Schnittkrümmung} von $M$ entlang $\sigma$ in $p$ definieren:
\[ \kappa_p(\sigma) := \kappa(x,y) \]

Die Familie aller $\set{\kappa_p(\sigma)}{}_{\sigma \subset T_pM}$ bestimmt $R$ im Punkt $p$ eindeutig. Dies folgt aus einem Resultat der linearen Algebra, nämlich:
\Prop{}
\label{PropLAREindeutig}
Sei $(V,\shrp{\cdot, \cdot})$ ein Euklidischer Vektorraum und seien $R,R' : V\times V \times V \pfeil{} V$ trilineare Abbildungen, die beide die Symmetrien aus 1) bis 4) aus \ref{SymmetrienRiemannscherKrümmungstensor} erfüllen. Wenn ferner folgende Gleichheit vorliegt
\begin{align*}
\shrp{R(x,y)x,y} = \shrp{R'(x,y)x,y}
\end{align*}
für alle $x,y \in V$, dann gilt
\[ R = R'.\]

\paragraph{Ricci-Krümmung}
Sei $p \in M$ und $x \in T_pM$ mit $\norm{x} = 1$. Wir ergänzen $x$ zu einer Orthonormalbasis $\{x, z_1, \ldots, z_{n-1}\}$ von $T_pM$. Definiere die \df{Ricci-Krümmung} durch
\[ \Ric_p(x) := \frac{1}{n-1} \sum_{i = 1}^{n-1} \shrp{R(x,z_i)x,z_i}. \]
$\Ric_p(x)$ ist unabhängig von der Wahl von $\{z_i\}_{i = 1}^{n-1}:$
\[ Q(x,y) := \mathrm{Spur}(z \mapsto R(x,z)y) .\]
$Q$ ist eine Bilinearform und es gilt
\[ Q(x,x) = (n-1)\Ric_p(x). \]

\newpage
\section{Jacobi-Felder}
Wir stellen uns die Frage:
\begin{center}
	Wie Schnell Entfernen sich Geodäten Voneinander?
\end{center}
Sei dazu $(M,g)$ eine Riemannsche Mannigfaltigkeit zusammen mit dem Levi-Civita-Zusammenhang $\nabla$. Sei $p \in M$. Ferner sei die Abbildung $\exp_p : B_0(\e) \pfeil{} M$ gegeben. Sei $v \in T_pM$, dann ist
\[ \gamma(t) = \exp_p(tv) \]
die eindeutig bestimmte Geodätische in $p$ mit $\dot{\gamma}(0) = v$. Wir betrachten Vektorfelder entlang von Geodätischen. Sei $w \in T_v(T_pM)$. Wie im Gauss-Lemma sei $v(s)$ eine Kurve in $T_pM$ mit $v(0) = v$ und $\dot{v} (0) = w$. Setze nun
\[ f(t,s) := \exp_p(tv(s)). \]
Sei
\[ J(t) = (\d \exp_p)_{tv}(tw) = \pf{f}{s} (t, s = 0). \]
$J$ ist ein Vektorfeld entlang $\gamma$.\\
$\gamma$ ist eine Geodäte, ergo gilt
\[ \Dd{t} \pf{f}{t} = \Dd{t}\dot{\gamma} = 0. \]
Daraus folgt
\[ \Dd{s} (\Dd{t} \pf{f}{t}) = 0. \]
Ist $V$ ein Vektorfeld entlang einer parametrisierten Fläche, so gilt
\[ \Dd{s}\Dd{t} V - \Dd{t} \Dd{s} V = R(\pf{f}{t}, \pf{f}{s}) V. \]
Das kann man durch Nachrechnen in lokalen Koordinaten überprüfen.
\begin{align*}
\Dd{s} \Dd{t} \pf{f}{t} &= \Dd{t} \Dd{s} (\pf{f}{t}) + R(\pf{f}{t}, \pf{f}{s})\pf{f}{t}\\
= \Dd{t} \Dd{t} (\pf{f}{s})  + R(\dot{\gamma}, \pf{f}{s}) \dot{\gamma}.
\end{align*}
Daraus folgt, dass $J = \pf{f}{s}$ folgende Gleichung erfüllt
\[ \Dd{t}\Dd{t}J + R(\dot{\gamma}, J) \dot{\gamma} = 0. \]
Diese Gleichung nennt man \df{Jacobi-Gleichung}.

\paragraph{In Lokalen Koordinaten:}
Seien $\{e_1(t), e_2(t), \ldots, e_n(t)\}$ parallele Vektorfelder entlang $\gamma$, die an jedem Punkt $\gamma(t)$ eine Orthonormalbasis von $T_{\gamma(t)}M$ bilden. Betrachte
\[ J(t) = \sum_i f_i(t) e_i(t). \]
Es gilt
\begin{align*}
\Dd{t}\Dd{t} J(t) = \sum_i f_i''(t) e_i(t).
\end{align*}
Insbesondere folgt
\begin{align*}
 R(\dot{\gamma}, J) \dot{\gamma} &= \sum_i \shrp{ R(\dot{\gamma}, J)  \dot{\gamma}, e_i}e_i\\
 &\gl{\text{Fourier-Entwicklung}} \sum_{i,j} f_j \shrp{ R(\dot{\gamma}, e_j) \dot{\gamma}, e_i }e_i.
\end{align*}
Setzt man $a_{i,j} := \shrp{ R(\dot{\gamma}, e_j) \dot{\gamma}, e_i }$, so gilt
\[ f''_i(t) +\sum_j a_{i,j} f_j(t) = 0 \]
Dies ist eine \emph{lineare} Differentialgleichung zweiter Ordnung.

\Def{}
Ein Vektorfeld $J(t)$ entlang einer Geodätischen $\gamma(t)$ heißt \df{Jacobi-Feld}, wenn $J(t)$ die Jacobi-Gleichung erfüllt.\\

Die Tatsache, dass eine Differentialgleichung zweiter Ordnung vorliegt, impliziert nun, dass man nach Wahl von $J(0)$ und $\Dd{t}J(0)$ ein eindeutiges Jacobi-Feld durch Lösen von 
\[ f''_i(t) +\sum_j a_{i,j} f_j(t) = 0 \]
erhält.

\Bsp{}
$\dot{\gamma}(t)$ und $t\dot{\gamma(t)}$ sind Jacobi-Felder für eine Geodäte $\gamma$.

\Bsp{Jacobi-Felder auf Mannigaltigkeiten Konstanter Schnittkrümmung}
Sei $M$ eine Mannigfaltigkeit der konstanten Schnittkrümmung $\kappa$. Definiere $R'$ durch
\[ \shrp{R'(X,Y)Z, W} := \shrp{X,Z} \shrp{Y,W} - \shrp{X,W} \shrp{Y,Z}. \]
$R'$ ist trilinear und erfüllt die Symmetrien 1) - 4) des echten Krümmungstensors aus \ref{SymmetrienRiemannscherKrümmungstensor}. Betrachte
\[ \shrp{R'(X,Y)X,Y} = \norm{X}^2\norm{Y}^2 - \shrp{X,Y} = A(X,Y)^2. \]
Ferner gilt
\[ \frac{\kappa R'(X,Y)X,Y}{A(X,Y)^2} = \kappa = \frac{\shrp{ R(X,Y)X,Y }}{A(X,Y)^2}. \]
Aus \ref{PropLAREindeutig} folgt nun
\[ R = KR'. \]
Setzt man dies in die Jacobi-Gleichung ein, so erhält man
\begin{align*}
\shrp{R(\dot{\gamma}, J)\dot{\gamma}, T} &= \shrp{\kappa R'(\dot{\gamma}, J)\dot{\gamma}, T}\\
&= \kappa ( \shrp{\dot{\gamma}, \dot{\gamma}} \shrp{J,T} - \shrp{\dot{\gamma}, T} \shrp{\dot{\gamma}, J} ).
\end{align*}
Sei $\gamma$ parametrisiert durch die Bogenlänge und $J$ orthogonal zu $\gamma$. Es gilt
\begin{align*}
\shrp{ R(\dot{\gamma}, J) \dot{\gamma}, T } = \kappa \shrp{J,T}.
\end{align*}
Daraus vereinfacht sich die Jacobi-Gleichung zu
\[ \Dd{t} \Dd{t} J+  \kappa J = 0. \]
Sei $W(t)$ ein Vektorfeld entlang $\gamma$, $\norm{W(t)} = 1$, $\shrp{W, \dot{\gamma}} = 0$, $W$ parallel. Die vereinfachte Jacobi-Gleichung impliziert
\begin{align*}
J(t) = \left\lbrace
\begin{aligned}
\frac{\sin(\sqrt{\kappa} t)}{\sqrt{\kappa}} W(t) &&\text{, falls }\kappa > 0,\\
t W(t) && \text{, falls }\kappa = 0,\\
\frac{\sinh(\sqrt{-\kappa}(t))}{\sqrt{-\kappa}} W(t) && \text{, falls }\kappa < 0.
\end{aligned}
\right.
\end{align*}
\marginpar{Vorlesung vom 16.05.18}

\Bem{}
Ist $J(0) = 0$ und $ v := \dot{\gamma}(0),$, $w := \Dd{t}J(0)$, dann gilt
\[ J(t) = (\d \exp)_{tv}(tw) = \pf{f}{s}(t,0) \]
für $f(t,s) = \exp_{p}(tv(s))$ für $v(0) = v, \dot{v}(0) = w$. Dies folgt aus der Eindeutigkeit von Differentialgleichungen zweiter Ordnung.

\paragraph{Notation:}
Wir schreiben in Zukunft
\[J'(t) := \Dd{t}J \]
und allgemeiner
\[ J^{(k)}(t) := \klam{\Dd{t}}^k J. \]
\vspace{12mm}\\
Sei $J(t)$ ein Jacobi-Feld mit $J(0) = 0, v = \dot{\gamma}(0), w = J'(0), \norm{w} = 1$.
Wir interessieren uns für $\norm{J(t)}^2$ für kleine $t$ und taylorn es deswegen im Folgenden.
\begin{itemize}
	\item $\shrp{J,J}(0) = 0$,
	\item $\shrp{J,J}' = 2 \shrp{J',J}$, und insbesondere
	\[ \shrp{J,J}'(0) = 2\shrp{J'(0),J(0)} = 0, \]
	da $J(0) = 0$.
	\item $\shrp{J,J}'' = 2 \klam{ \shrp{J',J'} + \shrp{J,J''} }$, und ferner
	\[ \shrp{J,J}''(0) = 2\shrp{J'(0), J'(0)} = 2\norm{w}^2 = 2. \]
	\item $\shrp{J,J}''' = 2\klam{ 
2\shrp{J'',J'} + \shrp{J',J''} + \shrp{J,J'''}	
 } = 6 \shrp{J',J''} + 2 \shrp{J,J'''}$. Indem man $0$ einsetzt, erhält man
\[ \shrp{J,J}'''(0) = 6 \shrp{w, J''(0)} \gl{\mathrm{Jacobi}} 6 \shrp{w, - R(\dot{\gamma},J)\dot{\gamma}(0)} = 
 6 \shrp{w, - R(v,J(0))v} = 0
 \]
	\item $\shrp{J,J}'''' = 6 \shrp{J'',J''} + 8\shrp{J',J'''} + 2 \shrp{J,J''''}$.
	\[ \shrp{J,J}''''(0) = 6\shrp{J''(0), J''(0)} + 8\shrp{w, J'''(0)} = 8\shrp{w, J'''(0)}  \] 
	$J'' = - R(\dot{\gamma}, J) \dot{\gamma}$. Es gilt
	\begin{align*}
	J''' = - \Dd{t}_{t= 0} R(\dot{\gamma}, J)\dot{\gamma} \gl{(1)}  R(\dot{\gamma}, J'(0))\dot{\gamma} = -R(v,w)v
	\end{align*}
	Somit ergibt sich
		\[ \shrp{J,J}''''(0) = 8\shrp{ -R(v,w)v, w} = -8\kappa(v,w).  \] 
	Die Gleichheit bei (1) gilt, da
	\begin{align*}
	&\pf{}{t} \shrp{R(\dot{\gamma}, J)\dot{\gamma}, W} \klam{= \shrp{ \Dd{t}R(\dot{\gamma}, J)\dot{\gamma}, W } + \shrp{R(\dot{\gamma},J)\dot{\gamma}, \Dd{t}W}}\\
=	&\pf{}{t} \shrp{R(\dot{\gamma}, W)\dot{\gamma}, J} = \shrp{ \Dd{t}R(\dot{\gamma}, W)\dot{\gamma}, J } + \shrp{R(\dot{\gamma},W)\dot{\gamma}, J'}\\
	&=\shrp{ \Dd{t}R(\dot{\gamma}, W)\dot{\gamma}, J } + \shrp{R(\dot{\gamma},J')\dot{\gamma}, W}
	\end{align*}
\end{itemize}
\paragraph{Zusammenfassung:} Wir haben gezeigt:
\[ \norm{J(t)}^2 = t^2 - \frac{1}{3} \shrp{ 
R(v,w)v, w
 } t^4
+ o(t^4) \]
für $t \pfeil{} 0$.\\

Gilt $\norm{v} = \norm{w} = 1$, $v\bot w$, dann $A(v,w) = 1$ und somit
\[ \shrp{R(v,w)v, w} = \kappa_p(v,w). \]
Daraus folgt
\[ \norm{J(t)}^2 = t^2 - \frac{1}{3} \kappa_p(v,w)t^4 + o(t^4) \]
und somit
\[ \norm{J(t)}= t - \frac{1}{6} \kappa_p(v,w)t^3 + o(t^3). \]


Wir wollen die Abweichungsgeschwindigkeit von geodätischen Kurven in $M$ mit der Abweichungsgeschwindigkeit solcher Kurven in $T_pM$ vergleichen.\\
Die Abweichungsgeschwindigkeit in $T_pM$ der Szrahlen $t \mapsto tv(s)$ und $t\mapsto tv(0)$ ist gerade
\[ \norm{ \pf{}{s}_{s=0 }tv(s)} = t \norm{ \pf{}{s}_{s = 0 } v(s) } = t\norm{w} = t. \]
Daraus folgt, dass die Differenz der Abweichungsgeschwindigkeit in $M$ und der in $T_pM$ gegeben ist durch
\[ - \frac{1}{6}\kappa_p(v,w)t^3  + o(t^3) \]
Daraus folgt, ist $\kappa > 0$, so ist die Abweichungsgeschwindigkeit in $M$ langsamer als in $T_pM$. Ist die Schnittkrümmung bei $p$ negativ, so ist die Abweichungsgeschwindigkeit in $M$ schneller als in $T_pM$.

\newpage
\section{Konjugationspunkte}
\Def{}
Sei $p \in M$, $\gamma$ eine Geodätische in $M$ mit $\gamma(0) = p$. Ein Punkt $\gamma(t_0)\neq p$ heißt \df{konjugiert} zu $p$ entlang $\gamma$, falls ein Jacobi-Feld $J\neq 0$ entlang $\gamma$ existiert, sodass 
\[J(0) = J(t_0) = 0.\]
Die \df{Vielfachheit} von $\gamma(t_0)$ ist dann die maximale Anzahl von linear unabhängigen Jacobi-Feldern mit dieser Eigenschaft.

\Bsp{}
Sei $ M = S^n \subset \R^{n+1}$ die Einheitssphäre. Dann ist $\kappa_p(\sigma) = 1$ konstant positiv.\\
Wir haben gezeigt
\[ J(t) = \sin(t) W(t) \]
mit $\norm{W(t)} = 1$ und $W\bot \dot{\gamma}.$\\
Ist $p \in S^n$, $p = \gamma(0)$, dann ist $-p = \gamma(\pi)$ konjugiert zu $p$. Daraus folgt für alle $p$, dass $-p$ konjugiert zu $p$ ist. Die Vielfachheit dieser Konjugation ist $n-1$.

\Bem{}
$J(t) = t\dot{\gamma}(t)$ ist ein Jacobi-Feld mit $J(0) = 0$, und $J(t) \neq 0$ für alle $t\neq 0$, falls $\dot{\gamma} (0) \neq 0$.\\
Daraus folgt, dass die Vielfachheit einer Konjugation immer höchstens $n-1$ ist für zwei verschiedene Konjugationspunkte.

\Prop{}
$q = \gamma(t_0)$ ist genau dann konjugiert zu $p = \gamma(0)$ entlang $\gamma$, wenn $t_0v$ ein kritischer Punkt von $\exp_{p}$ ist für $v = \dot{\gamma}(0)$.
\begin{Beweis}{}
Ist $J(0) = 0$, so folgt $J(t) = (\d \exp_p)_{tv}(tw)$ und somit
\[ 
0 = J(t_0) = (\d \exp_p)_{t_0v}(t_0w),
 \]
wobei $t_0w \neq 0$.
\end{Beweis}

\Def{}
Wir definieren den \df{Konjugationslokus} durch
\[ C(p) = \set{q \in M}{p, q \text{ sind konjugiert}} \]
\Bsp{}
Ist $M= S^n$, so gilt
\[ C(p) = \{-p\}. \]

\newpage
\section{Vollständige Mannigfaltigkeiten}
Sei $(M,g)$ eine Riemannsche Mannigfaltigkeiten.
\Def{}
$(M,g)$ heißt \df{geodätisch vollständig} bzw. \df{vollständig}, falls für alle $p \in M$ die Abbildung $\exp_p$ auf ganz $T_pM$ definiert ist.

\Bsp{}
\begin{itemize}
	\item $M = \R^n$ ist geodätisch vollständig.
	\item Die eingebettete Untermannigfaltigkeit $B := \set{x \in \R^n}{\norm{x} < 1} \subset \R^n$ mit der induzierten Metrik ist nicht vollständig.\\
	Allerdings kann man die Mannigfaltigkeit $B$ mit $\R^n$ identifizieren und dementsprechend eine Metrik auf $B$ einführen, sodass $B $ und $ \R^n$ isometrisch sind. Dadurch wird $B$ zu einer vollständigen Mannigfaltigkeit.
\end{itemize}

\Def{}
Seien $p,q \in M.$ Definiere die Distanz zwischen $p$ und $q$ durch
\[ d(p,q) := \inf \set{L(c)}{ c:p\mapsto q \text{ ist eine stückw. glatte Kurve} }. \]
Gilt $d(p,q) = 0$, so folgt $p = q$, da Geodätische lokal die Länge minimieren. Die anderen Axiome eines metrischen Raumes werden durch $d(p,q)$ ebenfalls erfüllt.\\
Dadurch wird $(M,d)$ zu einem metrischen Raum.

\marginpar{Vorlesung vom 23.05.18}
Für hinreichend kleinen Raidus sind geodätische Bälle metrische Bälle. Daraus folgt, dass die Topologie, die durch die Metrik induziert wird, mit der Topologie der Riemannschen Mannigfaltigkeit $M$ übereinstimmt.

\Satz{Hopf-Rinow}
Sei $M$ eine zusammenhängende Riemannsche Mannigfaltigkeit und $p\in M$ ein Punkt. Dann sind folgende Aussagen äquivalent:
\begin{enumerate}[(1)]
	\item $\exp_p$ ist auf ganz $T_pM$ definiert.
	\item Abgeschlossene beschränkte Mengen in $M$ sind kompakt.
	\item $(M,\d)$ ist metrisch vollständig.
	\item $(M,g)$ ist geodätisch vollständig.
	\item Für jede Folge von kompakten Teilmengen $K_j \subset M$ mit
	\begin{align*}
	K_j \subseteq K_{j+1} && \text{ und } && \bigcup_j K_j = M
	\end{align*}
	und jede Folge $x_j \in M \setminus K_j$ gilt
	\[ d(p,x_j) \Pfeil{j\pfeil{}\infty} \infty. \]
\end{enumerate}
Jeder der Aussagen (1) - (5) impliziert:
\begin{enumerate}[(6)]
	\item Für jedes $q \in M$ existiert eine Geodätische $\gamma$, die $p$ und $q$ verbindet, und für die gilt
	\[ L(\gamma) = d(p,q). \]
\end{enumerate}
\begin{Beweis}{}
\begin{enumerate}
	\item[(1) $\impl{}$ (6):] Setze $r := d(p,q)$. Sei $\overline{B}_\delta(p)$ ein abgeschlossener geodätischer Ball um $p, \delta > 0$. sei $S := \partial \overline{B}_\delta(p)$ die korrespondierende geodätische Sphäre. Dann existiert ein $x_0 \in S$ sodass gilt
	\[ d(x_0, q) = \min_{x \in S} d(x,q). \]
	Dann exisiert ein $v \in T_pM$ mit $\norm{v} = 1$ und $x_0 = \exp_p(\delta v)$.\\
	Mit (1) folgt jetzt, dass
	\[ \gamma(s):= \exp_p(sv) \]
	eine Geodätische ist.
	\paragraph{Behauptung:} $\gamma(r) = q$.\\
	Setze, um dies zu zeigen, 
	\[ A:= \set{s \in [0,r]}{ d(\gamma(s),q) = r-s }. \]
	0 liegt in $A$, somit ist $A$ nicht leer.
	\paragraph{Behauptung:} Ist $s_0\in A$ mit $s_0<r$, dann existiert ein $\epsilon > 0$ mit $s_0 + \epsilon \in A$.\\
	Sei $\overline{B}_\e(\gamma(s_0))$ ein geodätischer Ball um $\gamma(s_0)$ mit Radius $\epsilon > 0$. sei
	$S':= \partial \overline{B}_\e (\gamma(s_0)) $ die korrespondierende geodätische Sphäre.\\
	Dann existiert ein $y_0 \in S'$ mit $d(y_0, q) = \min_{y\in S'} d(y,q)$. Es gilt nun
	\begin{align*}
	d(\gamma(s_0), q) &= \e + \min_{y\in S'} d(y,q) = \e + d(y_0, q)\\
	d(\gamma(s_0), q) &\gl{s_0 \in A} r - s_0.
	\end{align*}
	Daraus folgt
	\[ d(y_0, q) = r - s_0 - \e. \]
	Da gilt
	\[ d(p,q) \leq d(p,y_0) + d(y_0,q), \]
	 folgt
	 \begin{align*}
	 d(p,y_0) &\geq d(p,q) - d(y_0, q)\\
	 &= r - (r - s_0 - \e)\\
	 &= s_0+\e.
	 \end{align*}
	 Andererseits gibt es eine stückweise glatte Kurve $c$, die $p$ und $y_0$ verbindet und Länge $s_0 + \e$ hat. Somit folgt
	 \[ d(p, y_0) = s_0 + \e. \]
	 Damit folgt insbesondere, dass $c$ eine Geodätische, also durchgehend glatte Kurve ist. Damit folgt
	 \[ \gamma(s_0 + \e) = y_0. \]
	 Es gilt nun
	 \[ d(\gamma(s_0+\e), q) = d(y_0, q) = r - (s_0 + \e). \]
	 Ergo liegt $s_0 + \e$ in $A$.\\
	 
	 Da $A$ abgeschlossen ist, gilt nun $r \in A$. Damit gilt
	 \[ d(\gamma(r), q) = r- r = 0, \]
	 ergo
	 \[ \gamma(r) = q. \]
	 \item[(1) $\impl{}$ (2):] Sei $C \subset M$ abgeschlossen und beschränkt. Wegen der Beschränktheit existiert ein metrischer Ball $B$, sodass $C$ in $B$ enthalten ist. Da $\exp_p$ laut (6) surjektiv ist, existiert somit ein $\overline{B}_r(0) \subset T_pM$ mit $B \subset \exp \overline{B}_r(0)$. Da $\overline{B}_r(0)$ kompakt ist, ist $\exp \overline{B}_r(0)$ ebenfalls kompakt. Ergo ist $C \subset \exp \overline{B}_r(0)$ eine abgeschlossene Teilmenge eines kompakten Raumes und dadurch selbst kompakt.
	 \item[(2) $\impl{}$ (3):] Wir müssen zeigen: Jede Cauchy-Folge $(x_n)$ in $M$ konvergiert.\\
	 $X := \set{x_n}{}\subset M$ ist beschränkt. Ergo ist $\overline{X}$ beschränkt und abgeschlossen und somit kompakt. Daraus folgt, dass $(x_n)$ eine konvergente Teilfolge hat. Da $(x_n)$ Cauchy-konvergent ist, folgt, dass $(x_n)$ konvergiert.
	 \item[(3) $\impl{}$ (4):] Wir führen einen Widerspruchsbeweis: Angenommen, $M$ wäre nicht geodätisch vollständig. Dann gibt es eine nach Bogenlänge parametrisierte Geodätische $\gamma$ und es existiert ein $s_0 \in \R$, sodass $\gamma(s)$ für alle $s_0 > s$ definiert ist, aber $\gamma$ sich auf $s$ nicht fortsetzen lässt.\\
	 Betrachte eine Folge $(s_n)$ mit $s_n \pfeil{} s$ für $n \pfeil{} \infty $. Wir behaupten, dass die Folge $(\gamma(s_n))_n$ dann Cauchy-konvergent ist. Sei $\e > 0$. Es existiert ein $N$, sodass für alle $n,m \geq N$ gilt
	 \[ \bet{s_n - s_m} < \e. \]
	 Dann gilt
	 \[ d(\gamma(s_n), \gamma(s_m)) \leq L_{s_n}^{s_m}(\gamma) = \bet{s_n - s_m} < \e. \]
	 (3) impliziert nun, dass $\gamma(s_n)$ gegen einen Punkt $q \in M$ konvergiert. Ergo liegen ab einem bestimmten Index alle Folgenglieder in einer geodätischen Umgebung von $q$. Nun kann man durch $q$ eine Geodäte wählen, die $\gamma$ fortsetzt.
	 \item[(4) $\impl{}$ (1):] trivial.
	 \item[(2) $\gdw{}$ (5):] Dies zeigt man durch Punktmengen-Topologie.
\end{enumerate}
\end{Beweis}

\Kor{}
Kompakte Mannigfaltigkeiten sind vollständig.

\newpage
\section{Überlagerungen}
Seien $E,B$ topologische Räume. Sei $p : E \pfeil{} B$ eine stetige surjektive Abbildung.
\Def{}
$p$ heißt \df{Überlagerung} von $B$, wenn für jedes $b \in B$ eine offene Umgebung $U \off B, b \in U,$ existiert mit
\[ p\i(U) = \bigsqcup_i V_i, \]
wobei die $V_i$ topologisch disjunkt und jeweils offen sind. Ferner soll für jedes $i$ die Einschränkung
\[ p_{|V_i} : V_i \Pfeil{} U_i \]
ein Homöomorphismus sein.\\
Man nennt in diesem Zusammenhang $B$ den \df{Basisraum} und $E$ den Totalraum.

\Bsp{}
$\R$ lässt sich wie eine Spirale über $S^1$ aufdrehen. Betrachte dazu
\begin{align*}
p : \R^1 & \Pfeil{} S^1\\
t & \longmapsto e^{2\pi i t}.
\end{align*}
\marginpar{Vorlesung vom 28.05.18}
%12.te Vorlesung
\Bem{}
Nicht jeder surjektiver lokaler Homöomorphismus ist eine Überlagerung.

\subsection{Hochhebungsproblem}
In voller Allgemeinheit gestaltet sich das Problem wie folgt. Es sind stetige Abbildungen $f:X \pfeil{} Y$ und $g:Z \pfeil{} Y$ gegeben und man fragt, ob eine stetige Abbildung $\widetilde{f} : X \pfeil{} Z$ existiert mit $g\circ \widetilde{f} = f$.
\begin{center}
	\begin{tikzcd}
		& Z \arrow[d, "g"] \\
	X  \arrow[r, "f"] \arrow[ru, "\exists_? \widetilde{f}", dashed]	& Y
	\end{tikzcd}
\end{center}
Im allgemeinem Fall ist die Antwort natürlich Nein. Betrachte als Gegenbeispiel die Identität $f = \id{S^1}$ und die Überlagerung $g : \R^1 \pfeil{} S^1$. Würde $\widetilde{f}$ existieren, so würde sich folgendes kommutierende Diagramm von Kohomologiegruppen existieren:
\begin{center}
	\begin{tikzcd}
	& H^1(\R) = 0 \arrow[dl, "\widetilde{f}^*"] \\
	H^1(S^1) = \R   	& H^1(S^1) = \R \arrow[l, "f^* = \id{}"] \arrow[u, "g^*"]
	\end{tikzcd}
\end{center}
Sei $g = p$ nun eine Überlagerung.
\paragraph{Fakt:}
Wege lassen sich bzgl. Überlagerungen hochheben, und zwar \emph{eindeutig}, wenn de Anfangspunkt der Hochhebung fixiert wurde.
\begin{Beweis}{}
Ist $\gamma : [0,1] \pfeil{} B$ ein Weg im Basisraum und $x \in p\i(\gamma(0))$, so besitzt $\gamma\cap U$ in einer offenen Umgebung um $U$ eine eindeutige Fortsetzung von $x$ aus. Da $\gamma([0,1])$ kompakt ist, können wir $\gamma([0,1])$ mit endlich vielen offenen Mengen überdecken, deren Urbilder unter $p$ sich in homöomorphe Komponenten zerlegen lassen. Auf jede dieser Komponente ist die Hochliftung aufgrund der Homöomorphie eindeutig.
\end{Beweis}

\Def{}
Zwei Überlagerungen $p_i : E_i \pfeil{} B, i = 1,2,$ heißen \df{äquivalent}, wenn es einen Homöomorphismus $\phi : E_1 \pfeil{} E_2$ zwischen den Totalräumen, sodass sich folgendes Diagramm ergibt:
\begin{center}
	\begin{tikzcd}
	E_1 \arrow[rr, "\phi"] \arrow[rd, "p_1"]	& 	&	E_2 \arrow[ld, "p_2"]	 \\
		& B	&	
	\end{tikzcd}
\end{center}

\Def{}
Ein topologischer Raum $X$ heißt \df{einfach zusammenhängend}, wenn $X$ wegzusammenhängend ist und jede stetige punktierte Abbildung $\gamma : (S,1) \pfeil{} (X,x_0)$ homotop relativ 1 zur konstanten Abbildung $z\mapsto x_0$ ist.

\paragraph{Fakt:}
Sei $p : E\pfeil{} B$ eine Überlagerung. Seien $\gamma, \gamma' : I \pfeil{} B$ Wege, die homotop relativ Endpunkte sind. Seien $\widetilde{\gamma}, \widetilde{\gamma}' : I \pfeil{} E$ Hochhebungen von $\gamma$ und $\gamma'$, sodass gilt
\[ \widetilde{\gamma}(0) = \widetilde{\gamma}'(0). \]
Dann gilt
\[ \widetilde{\gamma}(1) = \widetilde{\gamma}'(1). \]

\paragraph{Fakt:}
Abbildungen, die auf einem einfach zusammenhängenden Raum definiert sind, lassen sich immer bzgl. Überlagerungen hochheben.
\begin{Beweis}{}
Sei $X$ einfach zusammenhängend und es seien eine Abbildung $f : X \pfeil{} B$ und eine Überlagerung $g : E \pfeil{} B$ gegeben. Wir wollen ein Hochhebung $\widetilde{f} : X \pfeil{} E$ konstruieren.\\
Sei $x_0 \in X$ ein beliebiger Basispunkt und $y_0 \in p\i(x_0)$ fixiert. Ist $x\in X$ ein anderer Punkt, so sei $\gamma : x_0 \mapsto x$ eine Strecke zwischen beiden Punkte. Der Weg $f\gamma$ besitzt eine Hochhebung $\widetilde{\gamma}$ mit $\widetilde{\gamma}(0) = y_0$. Wir setzen
\[ \widetilde{f}(x):= \widetilde{\gamma}(1). \]
$\widetilde{f}$ ist dadurch wohldefiniert. Denn sind $\gamma, \gamma': x_0 \pfeil{} x$ zwei verschiedene Wege. Dann ist $f\circ \overline{\gamma}*\gamma'$ ein nullhomotoper Weg in $B$. Dieser wird zu einem nullhomotopen Weg in $E$ hochgeliftet. Es gilt ergo
\[ f(x) = \widetilde{\gamma}(1) = \widetilde{\gamma}'. \] 
Es bleibt zu zeigen, dass $\widetilde{f}$ stetig ist.
\end{Beweis}

\Def{}
Eine Überlagerung $p : E \pfeil{} B$ heißt \df{universell}, wenn $E$ einfach zusammenhängend.\\\\



Wir nehmen nun an, dass unsere Räume wegzusammenhängend sind.\\
Seien $E_1, E_2$ einfach zusammenhängend mit Überlagerungen $p_i : E_i \pfeil{} B$ für $i = 1,2$.
Es ergibt sich folgendes Diagramm
\begin{center}
	\begin{tikzcd}
	 	&	E_2 \arrow[d, "p_2"]	 \\
E_1 \arrow[ru, "\exists_1 \widetilde{p_1}", dashed] \arrow[r, "p_1"]	& B	
	\end{tikzcd}
\end{center}
da $E_1$ einfach zusammenhängend ist. Da $E_2$ ebenfalls einfach zusammenhängend ist, ergibt sich ferner
\begin{center}
	\begin{tikzcd}
	&	E_1 \arrow[d, "p_1"]	 \\
	E_2 \arrow[ru, "\exists_1 \widetilde{p_2}", dashed] \arrow[r, "p_2"]	& B	
	\end{tikzcd}
\end{center}
Da die Hochhebungen $\widetilde{\id{E_1}} = \widetilde{p_2} \widetilde{p_1}$ und $\widetilde{\id{E_2}}= \widetilde{p_1} \widetilde{p_2}$ eindeutig sind, folgt, dass $\widetilde{p_1}$ und $\widetilde{p_2}$ zueinander inverse Homöomorphismen sind. Daraus folgt, dass $E_1$ und $E_2$ äquivalent sind.\\
Wir haben gezeigt: Die \emph{universelle} Überlagerung eines Basisraumes ist eindeutig bis auf Äquivalenz von Überlagerungen.

\Kor{}
Ist $B$ einfach zusammenhängend, so ist
\[ p : E \Pfeil{} B \]
ein Homöomorphismus.
\begin{Beweis}{}
$E$ muss einfach zusammenhängend sein.
Ferner liegt folgendes Diagramm von Überlagerungen vor:
\begin{center}
	\begin{tikzcd}
	E \arrow[rr, "\exists_1\phi", dashed] \arrow[rd, "p"]	& 	&	B \arrow[ld, "\id{B}"]	 \\
	& B	&	
	\end{tikzcd}
\end{center}
Da $B$ einfach zusammenhängend ist, folgt, dass $p = \phi$ homöomorph ist.
\end{Beweis}

\Bem{}
Im glatten Kontext ersetzen wir \emph{Homöomorphismus} durch \emph{Diffeomorphismus}, etc.\,.\\\\

Sei $M^n$ eine Riemannsche Mannigfaltigkeit.
\Prop{}
Ist $M$ zusammenhängend und vollständig mit Schnittkrümmung $\kappa \leq 0$ überall, so ist $\exp_{p} : T_pM \pfeil{} M$ die universelle Überlagerung von $M$.\\\\
Daraus folgt:
\Satz{Hadamard}
Sei $M$ einfach zusammenhängend und vollständig mit $\kappa \leq 0$.\\
Dann ist $M$ diffeomorph zu $\R^n$.
\begin{Beweis}{}
$p : T_pM \pfeil{} M$ ist universell laut der Proposition. Mit dem voran gegangenem Korollar folgt nun, dass $\exp_{p}$ ein Diffeomorphismus ist.
\end{Beweis}

Wir wollen nun die Proposition beweisen.

\subsection{Lemma 1}
Seien $T,M$ Riemannsche Mannigfaltigkeiten der Dimension $n$ und $T$ vollständig. Ist $p : T \pfeil{} M$ eine isometrische Immersion, so ist $p$ eine Überlagerung.
\begin{Beweis}{}
\begin{enumerate}[(1)]
	\item $M$ ist vollständig:\\
	Sei $x^* \in T$. Setze $x:= p(x^*)$. Sei $v \in T_xM$ mit $\norm{v} = 1$. Dann ist $\d p_{x^*} : T_{x^*}T \pfeil{} T_xM$ eine Isometrie von Vektorräumen. Setze
	\[ v^* := (\d p_{x^*})\i (v). \]
	Definiere eine Geodäte durch
	\[ \gamma^*(s) := \exp_{x^*}(sv^*) \]
	für alle $s \in \R$, da $T$ vollständig. Da gilt
	\[ p(\gamma^*(s)) = \exp_{x}(sv) \]
	ist auch $\exp_{x}(sv)$ für alle $s \in \R$ definiert. Ergo ist auch $M$ vollständig.
	\item $p$ ist surjektiv:\\
	Sei $x_0^* \in T$, setze $x_0:= p(x_0^*)$. Sei $x \in M$. Wegen Hopf-Rinow (6) gibt es eine Geodätische $\gamma$, die $x_0$ mit $x$ verbindet, wobei $v := \dot{\gamma}(0)$ Länge $1$ hat. Dann gilt $\gamma(l) = x$ für $l = d(x_0,x)$.\\
	Wir können $v$ zu $v^* \in T_{x_0^*}T$ hochliften. Es gilt dann für $x^* := \exp_{x_0^*}(lv*)$
	\[ p(x^*) = x. \]
	\item $p$ ist eine Überlagerung:\\
	
\end{enumerate}
\end{Beweis}
\marginpar{Vorlesung vom 30.05.18}
%13.te Vorlesung

\subsection{Lemma 2}
Ist $M$ vollständig mit $\kappa \leq 0$, so ist $\exp_p : T_pM \pfeil{} M$ ein lokaler Diffeomorphismus.
\begin{Beweis}{}
Sei $J$ ein Jacobi-Feld entlang $\gamma(t) = \exp_p(tv)$ für $\norm{v} = 1$, mit $J(0) = 0$ und $J\neq 0$.\\
Es genügt zu zeigen, dass $J(t) \neq 0$ für alle $t > 0$. Es gilt
\begin{align*}
\shrp{J,J}'' &= 2 \shrp{J', J'} + 2 \shrp{J, J''}\\
&= 2 \norm{J'}^2 + 2 \shrp{-R(\dot{\gamma}, J')\dot{\gamma}, J}\\
&= 2\norm{J'}^2 - 2 \kappa(\dot{\gamma}, J) \cdot A(\dot{\gamma}, J)^2.
\end{align*}
Da $A(\dot{\gamma}, J)^2 \geq 0$ und $\kappa(\dot{\gamma}, J) \leq 0$, folgt
\[ \shrp{J,J}'' \geq 0. \]
D.\,h., $\norm{J}^2$ ist eine konvexe Funktion von $t$. Es gilt ferner $J'(0) \neq 0$ und $\shrp{J,J}'(0) = 2 \shrp{J,J'}(0) = 0$.\\
Daraus folgt $J(t) \neq 0$ für alle $t > 0$.
\end{Beweis}

\begin{Beweis}{Proposition \ref{PropHadamard}}
Mit Lemma 2 folgt, dass $\exp_{p}$ ein lokaler Diffeomorphismus ist. Daraus folgt, dass die Riemannsche Metrik auf $M$ eine eindeutige Riemannsche Metrik auf $T_pM$ induziert, durch die $\exp_{p}$ zu einer isometrischen Isomorphismus wird.\\
Die Geodätischen in $T_pM$ durch den Ursprung 0 sind die Geraden durch 0. Mit Hopf-Rinow folgt nun, dass $T_pM$ mit der gegebenen Metrik vollständig ist. Mit Lemma 1 folgt nun, dass $\exp_p$ eine Überlagerung ist.
\end{Beweis}




\chapter{Morse-Theorie}
\section{Crash-Kurs: Zellkomplexe und Homologie}
\begin{itemize}
	\item \textbf{Zellkomplexe}:\\
	Idee: Zerlege einen Raum in Teile (\emph{Zellen}), die selbst \emph{keine Topologie} besitzen. Dann kann die Homologie/Kohomologie aus der Kombinatorik dieser Teile abgelesen werden.
\end{itemize}
\Def{Zelle}
Wir definieren \df{Zelle} als alles, was homöomorph zu $D^n = \set{x \in \R^n}{\norm{x} \leq 1}$ ist. Es gilt $\partial D^n = S^{n-1}$.
\paragraph{Schreibweise:}
\[ e^n \isom{}D^n. \]
Wir beschränken uns auf \textbf{endliche} Zellkomplexe.
\begin{itemize}
	\item \textbf{CW-Komplexe}\footnote{Das \emph{C} steht für \emph{closure finite} und das \emph{W} für \emph{weak topology}.}:\\
	CW-Komplexe sind induktiv definiert:
	\begin{align*}
	X^0 &= e_1^0 \sqcup e_2^0 \cup \ldots \sqcup e_{k_0}^0 
	\end{align*}
	ist eine disjunkte Vereinigung von Punkten.
	$X^1$ ergibt sich, indem 1-Zellen dazu nimmt und ihre Randpunkte mit Punkten in $X^0$ identifiziert:
	\[ X^0 \cup_f e^1 := (X^0 \sqcup e^1)/( x\sim f(x)~\forall x \in \partial e^1) \text{ mit }\partial e^1 = S^0 \pfeil{f} X^0. \]
	Dann gilt
	\begin{align*}
	X^1 &= X^0 \cup_{f_1} e_1^1
	\cup_{f_2} e_2^1 \ldots
	\cup_{f_{k_1}} e_{k_1}^1.
	\end{align*}
	Die $f_i : e_i^1 \pfeil{} X^0$ nennt man \df{anheftende Abbildungen}. $X^1$ nennt man auch \df{Graph}.
	
	\[ X^2 : e^2, \partial e^2 = S^1 \pfeil{f} X^1 \]
	\[ X^1 \cup_f e^2 := (X^1 \sqcup e^2) / (\forall x \in \partial e^2 = S^1: ~x\sim f(x)) \]
	\[ X^2 = X^1 \cup_{f_1} e^2_1 \ldots \cup_{f_{k_2}}e_{k_2}^2. \]
	 \textbf{Allgemein}:
	\[ X^n = X^{n-1} \cup_{f_1} e^n_1 \ldots \cup_{f_{k_n}} e_{k_n}^n \]
	mit $f_i : \partial e_i^n \pfeil{} X^{n-1}$ stetig. $X^n$ nennt man das \df{$n$-Skelett}.\\
	Es ergibt sich ferner folgendes Diagramm
	\begin{center}
		\begin{tikzcd}
		e_i^k \arrow[r, "\chi_i"] & X^k \subset X \\
		\partial e_i^k \arrow[u, hook]  \arrow[r, "f"] & X^{k-1} \arrow[u, hook] 
		\end{tikzcd}
	\end{center}
wobei $\chi_i$ ein Homöomorphismus vom Inneren von $e_i^k$ auf sein Bild ist. $\chi_i$ nennt man auch die \df{charakteristische Abbildung}.
\end{itemize}

\Def{}
Ein topologischer Raum von der Form $X^n$ heißt \df{CW-Komplex}.

\Bsp{}
\begin{align*}
S^0 &= e^0 \sqcup e^0\\
S^1 &= e^0\sqcup e^1\\
S^n &= e^0 \sqcup e^n
\end{align*}
und
\begin{align*}
\R P^2 &= D^2 /(x \sim -x, x \in \partial D^2)\\
\R P^2 &= S^1\cup_f e^2
\end{align*}
mit $f$ antipodal. Daraus folgt
\[ \R P^2 = e^0\cup_{\mathrm{konst.}} e^1 \cup_f e^2. \]
\textbf{Allgemein}:
\[ \R P^n = \R P^{n-1} \cup_{\mathrm{Quot} = f}e^n \]
mit $f : \partial e^n = S^{n-1} \Pfeil{\mathrm{Quot}} \R P^{n-1} = S^{n-1}/(x\sim - x)$.
Daraus folgt
\[ \R P^n = e^0 \cup e^1 \cup \ldots \cup_{\mathrm{Quot}} e^n. \]
\textbf{Torus}:
\[ T^2 = e^0 \cup_{\mathrm{konst.}} e^1_a \cup_{\mathrm{konst.}} e^1_b \cup_f e^2 \]
mit $f : \partial e^2 = S^1 \pfeil{} (T^2)^1$. $(T^2)^1$ soll das 1-Skelett des Tori beschreiben. $f = aba\i b\i$.\\
\textbf{Kleinsche Flasche}:
\[ K^2 = e^0 \cup e^1_a \cup e^1_b \cup_f e^2  \]
mit $f : aba\i b$, sonst gilt $(K^2)^1 = (T^2)^1$.

\Def{}
Sei $X$ ein CW-Komplex. Setze
\[ C_k(X) := \Z[X^k]. \]
D.\,h., $C_k(X)$ ist die frei abelsche Gruppe, die von den $k$-dimensionalen Zellen in $X$ frei erzeugt wird. $C_k(X)$ nennen wir die $k$-te \df{zelluläre Kettengruppe} von $X$.

\Bsp{}
Für $X = T^2 = e^0 \cup e_a^1\cup e_b^1 \cup_f e^2$ ist
\begin{align*}
C_{-1} &= 0,
C_0(T^2) &= \Z\shrp{e^0},\\
C_1(T^2) &= \Z\shrp{e^1_a, e^1_b} = \Z\shrp{e^1_a} \oplus \Z\shrp{e^1_b},\\
C_2(T^2) &= \Z \shrp{e^2}.
\end{align*}
Es gibt ferner Randabbildungen
\begin{align*}
\Pfeil{\partial_3} C_2(T^2) \Pfeil{\partial_2}
C_1(T^2) \Pfeil{\partial_1}
C_0(T^2) \Pfeil{\partial_0}
C_{-1}(T^2) \Pfeil{\partial_{-1}}.
\end{align*}
Diese sind gegeben durch
\begin{align*}
\partial_1(e^1_a) &= e^0 - e^0 = 0,\\
\partial_1(e^1_b) &= e^0 - e^0 = 0,\\
\partial_2(e^2) &= e^1_a + e^1_b - e^1_a - e^1_b = 0.\\
\end{align*}
Für die Homologie
\[ H_k(X) 
:=
\frac{ \ker (\partial_k : C_k(X) \pfeil{} C_{k-1}(X)}{ \mathrm{im}(\partial_{k+1} : C_{k+1}(X) \pfeil{} C_k(X) }
\]
ergibt sich nun
\begin{align*}
H_2(T^2) &= C_2(T^2) = \Z,\\
H_1(T^2) &= C_1(T^2) = \Z\oplus \Z,\\
H_0(T^2) &= C_0(T^2) = \Z.
\end{align*}
Für $K^2$ kann man analog nachrechnen
\begin{align*}
H_2(K^2) &= 0,\\
H_1(K^2) &= \Z \oplus \Z /2\Z ,\\
H_0(K^2) &= \Z.
\end{align*}


\Def{Randoperatoren}
Wir definieren \df{Randoperatoren}
\[ \partial_{k} : C_k(X) \Pfeil{} C_{k-1}(X).  \]
Sei $e^k_i$ eine $k$-Zelle von $X$.
\[ \partial e_i^k = S^{k-1}_i \Pfeil{f_i} X^{k-1} \Pfeil{\mathrm{Quot}} \frac{X^{k-1}}{X^{k-2}} \isom{} \bigvee_j S_j^{k-1} \Pfeil{\mathrm{Proj}}S_j^{k-1}. \]
$\partial e_i^k = S^{k-1}_i \pfeil{} S^{k-1}_i$ habe den Abbildungsgrad $d_{i,j} \in \Z$. $\partial_k$ ist dann gerade die Matrix $(d_{i,j})_{i,j}$. Es gilt
\[ \partial_{k-1} \circ \partial_k = 0. \]
Wir definieren ferner die \df{zelluläre Homologie} von $X$ durch
\[ H_k(X) 
:=
\frac{ \ker (\partial_k : C_k(X) \pfeil{} C_{k-1}(X)}{ \mathrm{im}(\partial_{k+1} : C_{k+1}(X) \pfeil{} C_k(X) }.
\]
\marginpar{Vorlesung vom 04.06.18}
%14.te Vorlesung

\Bem{}
Man kann auch Kettengruppen $C_k(X;R) = C_k(X) \otimes_\Z R $ über beliebige \df{Koeffizientenringe} einführen. Diese besitzen Randabbildungen $\partial_k : C_k(X;R) \pfeil{} C_{k-1}(X;R)$ und führen zu $R$-wertigen Homologierguppen $H_k(X;R) = \ker \partial_k / \Img \partial_{k+1}$.\\
Die Definition von $H_k(X;R)$ ist unabhängig von der Wahl der CW-Struktur auf $X$.
\paragraph{Achtung: }Im Allgemeinem gilt \textbf{nicht}
\[ H_k(X;R) = H_k(X)\otimes_\Z R. \]

\Bsp{}
Sei $X = \R P^2 = e^0 \cup e^1 \cup_f e^2$, wobei $f : D^2 = e^2 \pfeil{} S^1 = e^1 \cup e^0$ die Abbildung von Grad 2 ist. Es ergeben sich folgende Abbildungen
\[ \Pfeil{0} C_2 = \Z\shrp{e^2} \Pfeil{\partial_2 = 2} C_1 = \Z\shrp{e^1} \Pfeil{\partial_1 = 0} C_0 = \Z \shrp{e^0} \Pfeil{0}.  \]
Somit folgt
\begin{align*}
H_i(\R P^2) = \left\lbrace
\begin{aligned}
&0,&&i=2,\\
&\Z/2\Z && i= 1,\\
&\Z && i = 0.
\end{aligned} 
\right.
\end{align*}

\Def{Induzierte Abbildungen}
Eine stetige Abbildung $f :X \pfeil{} Y$ zwischen CW-Komplexen heißt \df{zellulär}, wenn $f(X^k) \subset Y^k$ für alle $k$ gilt.\\\\

Sei $f : X \pfeil{} Y$ zellulär und sei $e_i^k$ eine $k$-Zelle von $X$. Betrachte die charakteristische Abbildung
	\begin{center}
	\begin{tikzcd}
	e_i^k \arrow[r, "\chi_i"] & X^k \subset X \\
	\partial e_i^k \arrow[u, hook]  \arrow[r, "f"] & X^{k-1} \arrow[u, hook] 
	\end{tikzcd}
\end{center}
Dies induziert Abbildungen
\[ S_i^k := \frac{e_i^k}{\partial e_i^k} \Pfeil{\overline{\chi}} \frac{X^k}{X^{k-1}} \Pfeil{f} \frac{Y^k}{Y^{k-1}} = \bigvee_j S_j^k \Pfeil{\text{Proj}} S^k_j. \]
Den Grad der Abbildung $S_i^k \pfeil{} S_j^k$ nennen wir $f_{ij} \in \Z$. Diese ergeben die Matrix
\[ f_* := (f_{ij}) : C_k(X) \Pfeil{} C_k(Y). \]
Es gilt
\[ \partial_* \circ f_* = f_* \circ \partial_*. \]
Daraus folgt, dass $f_*$ Homomorphismen auf der Homologie
\[ f_* : H_k(X) \Pfeil{} H_k(Y) \]
induziert.

\Satz{Zellulärer Approximationssatz}
Sei $f : X \pfeil{} Y$ eine stetige Abbildung zwischen CW-Komplexen. Dann ist $f$ homotop zu einer zellulären Abbildung.\\\\

Beachte, dass die auf den Homologiegruppen induzierte Abbildung $f_*$ nur von der Homotopieklasse von $f$ abhängt.

\Def{Relative Homologie}
Sei $X$ ein CW-Komplex und $A \subset X$ ein \df{Unterkomplex}, d.\,h., eine Vereinigung von abgeschlossenen Zellen. Dann ergibt sich eine Inklusion
\[ C_k(A) \subset C_k(X). \]
Wir setzen
\[ C_k(X, A) := \frac{C_k(X)}{C_k(A)}. \]
Betrachte nun das kommutierende Diagramm
\begin{center}
\begin{tikzcd}
C_k(A) \arrow[r, hook] \arrow[d, "\partial^A_k"] & C_k(X) \arrow[d, "\partial^X_k"] \\
C_{k-1}(A) \arrow[r, hook] & C_{k-1}(X) 
\end{tikzcd}
\end{center}
Wir definieren ergo die \df{relative Homologie} des Paares $(X,A)$ durch
\[ H^k(X,A) := \frac{\ker \partial_{k} : C_k(X,A) \pfeil{} C_{k-1}(X,A)}{\Img \partial_{k+1} : C_{k+1}(X,A) \pfeil{} C_{k}(X,A)} \]

Nach Konstruktion ist
\[ 0 \Pfeil{} C_*(A) \Pfeil{} C_*(X) \Pfeil{} C_*(X,A) \Pfeil{} 0 \]
exakt. Dies induziert uns eine lange exakte Sequenz
\[ \ldots \pfeil{} H_k(A) \pfeil{} H_k(X) \pfeil{} H_k(X,A) \pfeil{} H_{k-1}(A) \pfeil{}\ldots \]

\Def{Tripel}
Betrachte das \df{Tripel} $A \subset Y \subset X$ von Unterkomplexen. Wir schreiben hierfür auch $(X,Y,A)$.\\
Dann ergibt sich folgendes kommutierende Zopf-Diagramm
\begin{center}
	\begin{tikzcd}[scale = 0.5]
					&			& H_k(Y,A)\arrow[rr]\arrow[du]	&			& H_{k-1}(A)\arrow[rr]\arrow[du]	& 				& H_{k-1}(A) 	& \\
					& H_k(Y) \arrow[ru] \arrow[rd]	&						& H_k(X,A) \arrow[ru] \arrow[rd]	&				& H_{k-1}(Y) \arrow[ru] \arrow[rd]	&				& H_{k-1}(X,A)\\
H_k(A) \arrow[rr] \arrow[ru]	&			& H_k(X) \arrow[rr]	\arrow[ru]	&			& H_k(X,Y)	\arrow[rr]\arrow[ru]	& 				&H_{k-1}(Y,A)	

	\end{tikzcd}
\end{center}


\newpage
\section{Morse-Theorie}
Die Morse-Theorie soll eine Beziehung zwischen auf der einen Seite den kritischen Punkten und den Indizes einer glatten Funktion $f: M \pfeil{} \R$ und auf der anderen Seite der Zellstruktur bzw. der Homologie von $M$.

\Bsp{}
Es sei $f : M = T^2 \Pfeil{} \R^1$ die glatte Abbildung, die jedem Punkt des auf der Seite stehenden Tori seine Höhe zuordnet. Dies Abbildung hat vier kritische Punkte: Der höchste Punkt, der tiefste Punkt und der maximal bzw. minimale Innenpunkt. Wir nennen diese kritischen Punkte in absteigender Höhe $s,r,p,q$.\\
Für ein $a \in \R$ setze $M^a := \set{x \in M}{f(x) \subseteq a}$.
\begin{itemize}
	\item Ist $a< f(p)$, so gilt $M^a = \emptyset$.
	\item Ist $a \in (f(p), f(q))$, so ist $M^a$ eine Kreisscheibe.
	\item Ist $a \in (f(q), f(r))$, so ist $M^a$ ein Zylinder.
	\item Ist $a \in (f(r), f(s))$, so ist $M^a$ der Torus minus eine obere Kreisscheibe.
	\item Ist $a > f(s)$, so ist $M^a = M$.
\end{itemize}

\paragraph{Homotopietheoretisch:}
Ist $a \in (f(p), f(q))$, so ist $M^a$ homotop zum Punkt, ergo zu einer 0-Zelle $e^0$.\\
Ist $a \in (f(q), f(r))$, so ist $M^a$ homotop zu einer Sphäre, ergo erhält man eine 1-Zelle $e^1$, die man an $e^0$ anklebt.\\
Ist $a \in (f(r), f(s))$, so ist $M^a$ homotop zu $S^1\wedge S^1$, ergo erhält man eine weiteren 1-Zelle $e^1$, die man an den oberen Punkt von $e^1$ klebt.\\
Ist $a > f(s)$, so ist $M^a = M$ homotop zum Torus, ergo erhält man eine 2-Zelle, die man entlang den beiden $e^1$s verklebt.
\marginpar{Vorlesung vom 06.06.18}
%15.te Vorlesung
\Def{Die Hesse-Form}
Sei $M$ eine glatte Mannigfaltigkeit, $f : M \pfeil{} \R$ glatt, $p \in M$ ein kritischer Punkt von $f$.\\
Seien $v,w \in T_pM$. Wähle glatte Fortsetzungen von $v,w$ als glatte Vektorfelder $\widetilde{v}, \widetilde{w}$ auf $M$. Wir definieren folgenden Ausdruck
\[ H(f)(v,w) := v_p(\widetilde{w}(f)). \]
Um dies zu untersuchen, betrachten wir folgende Lie-Klammer für $\widetilde{v} = \sum_i a_i \pf{}{x_i}, \widetilde{w} = \sum_i b_i \pf{}{x_i}$ in lokalen Koordinaten
\[ [\widetilde{v}, \widetilde{w}](f)_p = \sum_{i,j} \klam{
a_i \pf{b_j}{x_i} - b_i \pf{a_j}{x_i}
} \pf{f}{x_j}|_{p}. \]
Aber $\pf{f}{x_j}|_{p}$ muss Null sein, da $p$ ein kritischer Punkt ist. Es folgt
\[ v_p(\widetilde{w}(f)) = w_p(\widetilde{v}(f)) \]
bzw.
\[ H(f)(v,w) = H(f)(w,v). \]
Dies zeigt, dass $H(f)$ symmetrisch und wohldefiniert, d.\,h. unabhängig von der Wahl von $\widetilde{v}$ und $\widetilde{w}$, ist.\\
Wir erhalten so eine symmetrische Bilinearform
\[ H(f) : T_pM \times T_pM \Pfeil{} \R, \]
die sogenannte \df{Hesse-Form}.\\\\

In der Basis $\{\pf{}{x_1}, \ldots, \pf{}{x_n}\}$ von $T_pM$ ist $H(f)$ durch die \df{Hesse-Matrix} $\klam{\pf{}{x_i}\pf{}{x_j}f(0)}_{i,j}$ gegeben.

\Def{}
\begin{itemize}
	\item Ein kritischer Punkt $p \in M$ von $f\colon M \pfeil{} \R$ heißt \df{nicht ausgeartet}, wenn $H(f)_p$ als Bilinearform nicht ausgeartet ist. D.\,h., die Hesse-Matrix bei $p$ ist nicht singulär.
	\item Der \df{Index} eines nicht ausgearteten kritischen Punktes $p$ von $f$ ist die maximale Dimension von Untervektorräumen von $T_pM$, auf denen die Hesse-Form negativ definit ist.
\end{itemize}

\Lem{}
Sei $f\colon U \pfeil{} \R$ glatt, $U \subset \R^n$ offen und konvex. Ferne soll $f(0) = 0$ gelten. Dann existieren glatte Funktionen $g_i : U \pfeil{} \R$ für $i = 1, \ldots, n$ mit:
\begin{enumerate}[1.]
	\item $f(x) = \sum_{i = 1}^n x_i g_i(x)$.
	\item $g_i(0) = \pf{f}{x_i}(0)$.
\end{enumerate}
\begin{Beweis}{}
Es gilt
\[ f(x) = \int_0^1 \pf{}{t}f(tx) \d x = \int_0^1 \sum_{i = 1}^n \pf{f}{x_i} (tx) \cdot x_i \d t.   \]
Setze ergo
\[ g_i(x) := \int_{0}^1 \pf{f}{x_i} (tx) \d t. \]
\end{Beweis}

\Lem{Morse Lemma}
Sei $\cln{f}{M}{\R}$ glatt und $p \in M$ ein nicht ausgearteter kritischer Punkt von $f$. Dann gibt es lokale Koordinaten $y_1,\ldots, y_n$ bei $p$, sodass
\[ f(y) = f(p) - y_1^2 - \ldots - y_\iota^2 + y_{\iota+1}^2 + \ldots + y_n^2, \]
wobei $\iota$ gleich dem Index von $f$ bei $p$ ist.
\begin{Beweis}{}
Seien $x_1, \ldots, x_n$ lokale Koordinaten. Wir dürfen ohne Einschränkung annehmen, dass $f(0) = 0$ gilt.\\
Das vorangegangene Lemma impliziert, dass $g_i$ existieren mit
\begin{align*}
f(x) &= \sum_{i = 1}^n x_i g_i(x),
g_i(0) &= \pf{f}{x_i}(0).
\end{align*}
Da $p$ ein kritischer Punkt ist, ist $g_i(0) = \pf{f}{x_i}(0) = 0$ für alle $i$. Wir wenden das Lemma nochmal für alle $g_i$ an und erhalten Funktionen $h_{ij}(x)$ mit
\[ g_i(x) = \sum_j x_j h_{ij}(x). \]
Daraus folgt
\[ f(x) = \sum_{i,j} x_i x_j h_{ij}(x). \]
Man kann ferner ohne Einschränkung annehmen, dass $h_{ij} = h_{ji}$ gilt (anderenfalls kann man stattdessen $\frac{h_{ij} + h_{ji}}{2}$ betrachten).
\paragraph{Induktion:}
Annahme: Wir haben lokale Koordinaten $u$ bei $p$ mit
\[ f(u) = \pm u_1^2 \pm \ldots \pm u_{r-1}^2 + \sum_{i,j\geq r} u_i u_j H_{ij}(u) \]
für glatte Funktionen $H_{ij}$ mit $H_{ij} = H_{ji}$. Aufgrund der Symmetrie können wir die Matrix $H_{ij}(0)$ diagonalisieren, d.\,h.
\[ A\i H_{ij}(0)A = \left(
\begin{matrix}
\lambda_1 \\
& \ddots \\
& & \lambda_n
\end{matrix}
\right). \]
Da $H_{ij}(0)$ nicht singulär ist, ist $\lambda_1 \neq 0$. Es sei $r$ maximal mit $\bet{H_{rr}(u)} > 0$. Dann ist $g(u) := \sqrt{\bet{H_{rr}(u)}}$ glatt in der Nähe von $u = 0$.\\
\paragraph{Transformation}
\begin{align*}
v_i &:= u_i \text{ für } i \neq r\\
v_r &:= g(u)\klam{
u_r + \sum_{i > r}u_i \frac{H_{ir}}{H_{rr}}.
}
\end{align*}
Dann gilt
\[ f(v) = \pm v_1^2 \pm \ldots \pm v_r^2 + \sum_{i,j \geq r+ 1}v_i v_j \widetilde{H}_{ij}. \]
\end{Beweis}
Wenn $f$ die Form in der Behauptung hat, dann
\[ (H(f))_{ij} = \left(
\begin{matrix}
-2 \\
& -2\\
& & \ddots \\
& & & -2\\
& & & & 2 \\
& & & & & \ddots\\
& & & & & & 2
\end{matrix}
\right), \]
wobei die Anzahl der negativen Diagonaleinträge dem Index von $f$ bei $p$ entspricht.

\Kor{}
Nichtausgeartete kritische Punkte sind isoliert.

\Bsp{}
Betrachte einen auf der Seite liegenden Zylinder. Alle Punkte auf der oberen Seitenlinie, sind kritisch und liegen kontinuierlich, ergo sind sie alle ausgeartet.\\\\


Sei $M$ eine glatte Mannigfaltigkeit und $X$ ein Vektorfeld auf $M$, sodass $X_{|M-K} = 0$ für eine kompakte Menge $K \subset M$.\\
Sei $U_1\cup \ldots \cup U_m$ eine Überdeckung von $K$ durch offene Mengen auf denen lokale Flüsse $\phi_{i,t} : U_i \pfeil{} M$ existieren mit
\[ \pf{}{t} \phi_{i,t} = X(\phi_{i,t}) \]
für $\bet{t}<\e_i$. Wegen der Eindeutigkeit von Lösungen von Differentialgleichungssystemen gilt $\phi_{i,t} = \phi_{j,t}$ auf $U_{i,t} \cap U_{j,t}$. Setze nun
\begin{align*}
\phi_t(q):=
\left\lbrace
\begin{aligned}
&q && q \in M \setminus K,\\
& \phi_{i,t}(q), && q \in U_i
\end{aligned}
\right.
\end{align*}
für $\bet{t}< \e := \min_i \e_i$.
$\phi_t$ ist ein globaler Fluss für $X$.\\
Ist $\bet{t} \geq \e$, dann schreibe
\[ t = k \cdot \frac{\e}{2} + r \]
mit $k \in \Z, \bet{r} < \frac{\e}{2}$. Es gilt dann konsequenterweise
\[ \phi_t = (\phi_{\frac{\e}{2}})^k \circ \phi_r. \]
Ergo ist $\phi_t(p)$ für alle $p \in M$ und $t\in \R$ definiert.

\Prop{}
Sei $M$ eine glatte kompakte Mannigfaltigkeit, $\cln{f}{M}{\R}$ glatt ohne kritische Punkte in $f\i([a,b])$ für ein Paar $a,b \in \R, a < b$. Dann ist $M^a$ diffeomorph zu $M^b$, die Inklusion $M^a \inj{} M^b$ ist eine Homotopieäquivalenz und $M^a$ ist ein Deformationsretrakt von $M^b$.
\begin{Beweis}{}
Sei $\shrp{\cdot, \cdot}$ eine Riemannsche Metrik auf $M$.
\Def{}
Wir definieren einen \df{Gradienten} $\nabla f$ durch
\[ \shrp{X, \nabla f} := X(f) \]
für alle Vektorfelder $X$.\\

$p$ ist genau dann ein kritischer Punkt von $f$, wenn $\nabla f$ bei $p$ Null ist.\\
Nun folgt, dass $\norm{\nabla f} > 0$ auf $f\i([a,b])$. Setze
\[ \rho(q) :=
\left\lbrace
\begin{aligned}
&\frac{1}{\norm{\nabla f}^2}(q),&& q \in f\i([a,b])\\
&0, && q \in M \setminus K
\end{aligned}
\right.\]
für eine kompakte Umgebung $K \supset f\i([a,b])$. Auf $K \setminus f\i([a,b])$ sei $\rho$ irgendwie definiert, sodass $\cln{\rho}{M}{\R}$ glatt ist.\\
Definiere folgendes Vektorfeld auf $M$
\[ X:= \rho \cdot \nabla f. \]
$X$ verschwindet außerhalb von $K$,
ergo existiert ein globaler Fluss $\phi_t \colon M \pfeil{} M$ mit $\pf{}{t} \phi_t = X(\phi_t)$ für $ t \in \R$.\\
Setze $g(t) := f(\phi_t(q))$. Es gilt
\begin{align*}
g'(t) &= \shrp{ \pf{}{t} \phi_t , \nabla f }\\
&= \shrp{ X , \nabla f }\\
&= \shrp{ \rho \nabla f , \nabla f }\\
&= \rho \norm{\nabla f}^2 = +1,\\
\end{align*}
solange sich $q$ auf $f\i([a,b])$ befindet.
Sei $c = g(0) = f(\phi_0(q)) = f(q)$. Wenn $q \in M^a$, dann $f(q) \leq a$. Daraus folgt $c \leq a$ und $g(t) = t + c$. Es gilt
\[ f(\phi_{b-a}(q)) = g(b-a) = b-a + c \leq b -a \leq b \]
d.\,h., $\phi_{b-a} (q) \in M^b$. Es ist nun $\phi_{b-a}$ der gesuchte Diffeomorphismus.\\
Der Deformationsretrakt ist nun gegeben durch $r_1$ für 
\[ r_t := 
\left\lbrace
\begin{aligned}
&q && q \in M^a\\
&\phi_{t(a- f(q))} && q \in f\i[a,b]
\end{aligned}
\right.
 \]
\end{Beweis}
\marginpar{Vorlesung vom 11.06.18}
%16.te Vorlesung
\Def{}
Eine glatte Funktion $f : M \pfeil{} \R$, deren kritischen Punkte alle nicht ausgeartet sind, heißt \df{Morse-Funktion}.

\Satz{}
Sei $M$ eine glatte Mannigfaltigkeit, $f : M \pfeil{{}} \R$ glatt, $\e > 0$, sodass $f\i[c-\e, c +\e]$ kompakt ist. Ferner sei $p$ der einzige kritische Punkt von $f$ in $[c - \e, c + \e]$ für $c = f(p)$.\\
Ist $p$ nicht ausgeartet mit Index $i$, so ist $M^{c+\e}$ homotopieäquivalent zu $M^{c-\e} \cup e^i$.

\begin{Beweis}{}
Wegen dem Morse-Lemma existieren lokale Koordinaten $u$ um $p$, sodass sich $f$ lokal darstellen lässt durch
\[ f(u) = c - u_1^2 - \ldots - u_i^2 + u_{i+1}^2 + \ldots u_n^2. \]
Setze
\[ e^i := \set{u}{ u_1^2 + \ldots + u_i^2 \leq \e, u_{i+1} = \ldots = u_n = 0 }, \]
dann ist der Rand gegeben durch
\[ \partial e^i := \set{u}{ u_1^2 + \ldots + u_i^2 = \e, u_{i+1} = \ldots = u_n = 0 }\subset M^{c- \e}. \]
Die anheftende Abbildung für diese $i$-Zelle an $ M^{c- \e}$ ist dann die Inklusion.\\
Wir modifizieren $f$ und definieren mithilfe der modifizierten Funktion $F$ einen sogenannten Henkel, der $e^i$ als Deformationsretrakt enthält.
\begin{itemize}
	\item \textbf{Hilfsfunktion $\mu$:} Sei $\mu : \R \pfeil{{}} \R$ glatt, sodass gilt
	\begin{align*}
	\mu(t) \left\lbrace
	\begin{aligned}
	&> \e,&& \text{ falls } t = 0\\
	& \in [0, 2\e],&& \text{ falls } t  \in [0, 2\e]\\
	&= 0,&& \text{ falls } t \geq 2 \e.
	\end{aligned}
	\right.
	\end{align*}
	und $\mu' \in [-1,0]$.
	\item Setze
	\begin{align*}
	\xi(u) :=&u_1^2 + \ldots + u_i^2\\
	\eta(u) :=&u_{i+1}^2 + \ldots + u_n^2\\
	F(u) := & f(u) - \mu( \xi(u) + 2 \eta(u) ) = c - \xi + \eta - \mu(\xi  - 2 \eta)
	\end{align*}
	\item Im Ellipsoid $\xi + 2 \eta \leq 2\e$ gilt nun
	\begin{align*}
	F \leq f = c - \xi + \eta \leq c + \frac{1}{2} \xi + \eta \leq c + \e 
	\end{align*}
	Daraus folgt
	\[ F\i(-\infty, c+ \e] = M^{c+\e} \]
	\item Die Bedingung $\mu' \in (-1,0]$ stellt sicher, dass $F$ und $f$ in $F\i(-\infty, c+ \e]$ dieselben kritischen Punkte haben. Denn es gilt
	\begin{align*}
	\pf{}{u_k}F(u) &= \pf{}{u_k}(f(u) - \mu( \xi(u) + 2 \eta(u) )) \\
	&= \pf{}{u_k}f(u)-\pf{}{u_k}\mu( \xi(u) + 2 \eta(u) ) \\
	&= \pm2u_k - c u_k\mu'(\xi(u) + 2 \eta(u)) \\
	&= (2 - c\mu'(\xi(u) + 2 \eta(u)))u_k \\
	&\gdw{} u_k = 0.
	\end{align*}
	für ein passendes $c \in \{1,2\}$.
	\item Der einzige kritische Punkt von $F$ in $F\i(-\infty, c+ \e]$ könnte somit nur $p$ sein. Es gilt aber
	\[ F(p) = f(p) - \mu(0) < c - \e. \]
	Daraus folgt, dass $p$ nicht in $F\i(-\infty, c+ \e]$ liegt, also hat $F$ keine kritischen Punkte auf $F\i(-\infty, c+ \e]$.
\end{itemize}
Mit der letzten Proposition folgt, dass $F\i(-\inf,c - \e] \subset F\i(-\inf,c + \e]$ ein Deformationsretrakt ist. Daraus folgt, dass $F\i(-\inf,c - \e] \subset M^{c+\e}$ ein Deformationsrektrakt ist.\\
Wir definieren den Henkel durch den topologischen Abschluss
\[ H:= \mathrm{closure}(F\i(-\inf,c + \e] - M^{c- \e}). \]
Dann wird $M^{c- \e} \cup e^i$ von $M^{c - \e}\cup H$ als Deformationsretrakt enthalten. Insgesamt ergibt sich folgendes Diagramm:
 \begin{center}
 	\begin{tikzcd}
M^{c- \e} \cup e^i\arrow[r,hook, "\mathrm{Def-retr.}"]\arrow[rd,hook, "\mathrm{Def-retr.}"]  &M^{c-\e} \cup H\arrow[d,hook, "\mathrm{Def-retr.}"]\\
&M^{c+\e}
 	\end{tikzcd}
 \end{center}
\end{Beweis}

\Satz{Hauptsatz}
Sei $M$ eine kompakte Mannigfaltigkeit und $f : M \pfeil{{}} \R$ eine Morse-Funktion. Dann hat $M$ den Homotopie-Typ eines CW-Komplexes mit genau einer Zelle der Dimension $i$ für jeden kritischen Punkt mit Index $i$.
\begin{Beweis}{}
$C$ sei die Menge der kritischen Punkte von $f$. Jeder kritische Punkt ist isoliert, da er nicht ausgeartet ist. Da $M$ kompakt ist, ist $C$ somit endlich. Insofern besteht $f(C)$ aus den Elementen $c_1< c_2 < \ldots < c_k$.

Wir führen nun eine vollständige Induktion nach $k$. In der Induktionsvoraussetzung nehmen wir an, dass wir eine Homotopieäquivalenz $h$ zwischen $M^a$ und einem CW-Komplex $K$ haben für einen Wert $a$, der nicht in $f(C)$ liegt.

Im Induktionsschritt sei $c \in f(C)$ minimal mit der Eigenschaft $c > a$. Dann existiert für ein hinreichend kleines $\e > 0$ eine Homotopie
\[ h' : M^a \simeq M^{c-\e}. \]
Aus dem vorhergegangenen Satz folgt nun
\[ M^{c+\e} \simeq M^{c-\e} \cup_{\phi_1}e_1^{i_1}\cup \ldots \cup_{\phi_l}e_l^{i_l} \]
für anheftende Abbildungen
\[ \phi_j : \partial e_j^{i_j} \Pfeil{} M^{c-\e} \simeq^{h'} M^a \simeq^h K. \]
Durch den zellulären Approximationssatz erhalten wir homotope Abbildungen $\psi_i$:
\begin{align*}
hh'\phi_j & : \partial e_j^{i_j} \Pfeil{}  K\\
\rott{\simeq}& ~~~~~~~~~~~~~~\rott{\supseteq}\\
\psi_i & : \partial e_j^{i_j} \Pfeil{}  K^{i_j -1}.
\end{align*}
Es ergibt sich nun folgend Homotopie
\begin{align*}
K\cup_{\psi_1} e_1^{i_1} \cup \ldots \cup_{\psi_l} e_l^{i_l} 
\simeq
K\cup_{hh'\phi_1} e_1^{i_1} \cup \ldots \cup_{hh'\phi_l} e_l^{i_l}
\simeq
M^{c-\e}\cup_{\phi_1} e_1^{i_1} \cup \ldots \cup_{\phi_l} e_l^{i_l}
\simeq M^{c+\e}.
\end{align*}
Rechterseits steht der CW-Komplex, den wir haben wollen. Ergo folgt durch Induktion die Aussage.
\end{Beweis}

\marginpar{Vorlesung vom 13.06.18}
%17.te Vorlesung

\subsection{Anwendungen}
Sei $M^n$ glatt, geschlossen. Es habe $f : M \pfeil{} \R$ genau 2 kritische Punkte.\\
Mit ein paar Homotopie-Tricks folgt, dass $M$ homöomorph\footnote{$M$ und $S^n$ müssen nicht diffeomorph sein, da es in höheren Dimensionen \textsl{exotische Sphären} gibt.} zu $S^n$ ist (Satz von Reeb).

\section{Die Morse Ungleichungen}

\Def{}
Sei $S: \{ \text{top. Paaren} (X,Y) \} \Pfeil{} \Z$ eine Funktion. $S$ heißt \df{subadditiv}, wenn für Tripel $X \supseteq Y \supseteq Z$ gilt
\[ S(X,Z) \leq S(X,Y) + S(Y,Z). \]
Besteht Gleichheit, so heißt $S$ \df{additiv}.\\
Wir legen ferner folgende Konvention für Funktionen wie $S$ fest:
\[ S(X) := S(X,\emptyset). \]

\Bsp{}
Setzt man
\[  R_i(X,Y) := \dim_\Q H_i(X,Y; \Q), \]
so kann man die \df{Euler-Charakteristik} definieren durch
\[ \chi(X,Y) := \sum_{i} (-1)^i R_i(X,Y). \]
Für  $X \supseteq Y \supseteq Z$ ergibt sich nun die Lange exakte Sequenz des Tripels:
\begin{center}
	\begin{tikzcd}
	\ldots \pfeil{} H_i(Y,Z) \pfeil{} H_i(X,Z) \pfeil{} H_i(X,Y) \pfeil{} H_{i+1}(Y,Z) \pfeil{} \ldots
	\end{tikzcd}
\end{center}
Für eine lange exakte Sequenz
\begin{center}
	\begin{tikzcd}
	\ldots \pfeil{} A_i \pfeil{} B_i \pfeil{} C_i \pfeil{} A_i \pfeil{} \ldots
	\end{tikzcd}
\end{center}
gilt
\[ \chi(B_*) = \chi(A_*) + \chi(C_*). \]
Daraus folgt
\[ \chi(X,Z) = \chi(Y,Z) + \chi(X,Y). \]
Also ist $\chi$ ein Beispiel für eine additive Funktion.\\
Der Rang $R_i$ selbst ist nur subadditiv (dies folgt aus der langen exakten Sequenz).

\Lem{}
Sei $X$ ein Raum mit einer Filtrierung durch Teilräume
\[ \emptyset = X_0 \subset X_1 \subset \ldots \subset X_{n-1} \subset X_n = X \]
Ist $S$ subadditv, so gilt
\[ S(X) \leq \sum_{i = 1}^nS(X_i, X_{i-1}) \]
Ist $S$ additiv, so gilt sogar Gleichheit.
\begin{Beweis}{}
Durch Induktion folgt
\[ S(X) \leq S(X, X_{n-1}) + S(X_{n-1}) \leq S(X,X_n) + \sum_{i = 1}^{n-1} S(X_i, X_{i-1}). \]
\end{Beweis}


Sei $M$ eine glatte, kompakte Mannigfaltigkeit mit Morse-Funktion $f : M \pfeil{} \R$ und $a_1 < \ldots < a_k$ in $\R$ so, dass $M^{a_i}$ genau $i$ kritische Punkte von $f$ enthält. Es soll außerdem gelten
\[ M^{a_k} = M. \]
Wir erhalten hierdurch eine Filtrierung
\[ \emptyset = M^{a_0} \subset M^{a_1} \subset \ldots \subset M^{a_k} = M. \]

\subsection{Ausschneidung in Zellulärer Homologie:}
Sei ein CW-Paar $X \supset Y$ gegeben, $Y \supseteq U$. $U$ sei gerade $U = X -Z$ für einen weiteren Unterkomplexen $Z \subset X$.\\
Es gilt nun
\[ C_*(X,Y) = \frac{C_*(X)}{C_*(Y)} 
\isom{} \frac{C_*(X-U)}{C_*(Y-U)}
= C_*(X- U, Y - U). \]
Insbesondere gilt für die Homologiegruppen
\[ H_*(X,Y) \isom{} H_*(X- U, Y-U). \]



Wir betrachten nun weiter
\[ H_j(M^{a_i}, M^{a_{i-1}})\]
Der Hauptsatz impliziert folgende Homotopie
\[ M^{a_i} \simeq M^{a_{i-1}} \cup e^{\iota_i}. \]
Dadurch folgt
\[  H_j(M^{a_i}, M^{a_{i-1}}) \isom{} H_j(M^{a_{i-1}} \cup e^{\iota_i}, M^{a_{i-1}} ) \isom{} H_j(e^{\iota_i}, \partial e^{\iota_i}). \]
Mit Koeffizienten über $\Q$ folgt nun
\[ H_j(e^{\iota_i}, \partial e^{\iota_i}; \Q) =
\left\lbrace
\begin{aligned}
&\Q && j = \iota_i,\\
&0 && j \neq \iota_i.
\end{aligned}
\right.
   \]
   
Mit dem Lemma und dem vorangegangenem Beispiel folgt nun
\[ R_\iota(M) \leq \sum_{i} R_\iota(M^{a_i}, M^{a_{i-1}}) = \# \text{krit. Punkte von Index} \iota =: c_\iota. \]
Ferner folgt
\[ \chi(M) = \sum_{i} \chi(M^{a_i}, M^{a_{i-1}}) = c_0 - c_1 + c_2 - \ldots = \sum_{i} c_i. \]
Zusammenfassend haben wir nun folgende \df{schwache Morse-Ungleichungen}
\begin{align*}
R_\iota (M) &\leq c_\iota\\
\chi(M) &= c_0 - c_1 + c_2 - \ldots.
\end{align*}

Definiert man
\[ S_\iota(X,Y) := R_{j}(X,Y) - R_{\iota-1}(X,Y) + \ldots \pm R_0(X,Y), \]
so kann man zeigen:
\Lem{}
$S_i$ ist subadditiv.
\begin{Beweis}{}
Sei eine exakte Sequenz wie folgt gegeben
\[ A_{i+1} \Pfeil{\psi}A_{i} \Pfeil{\psi_{i}}A_{i} \Pfeil{\psi_{i-1}} \ldots  \Pfeil{\psi_{1}}A_{0} \Pfeil{\psi_{0}} 0.  \]
Da gilt
\[ \dim A_i = \text{Rg} \psi + \text{Rg} \psi_i, \]
folgt
\[ 0 \leq \text{Rg} \psi = \dim A_i - \text{Rg} \psi_i = \ldots = \dim A_i - \dim A_{i-1} + \ldots \pm \dim A_0.  \]
Für die Tripel-Sequenz ergibt sich nun eine exakte Sequenz
\[ \pfeil{\psi = \partial} H_i(Y,Z)
\pfeil{}
H_i(X,Z)
\pfeil{}
H_i(X,Y)
\pfeil{\partial}
\ldots
\pfeil{}
0. \]
Durch obige Beobachtung folgt nun
\[ 0\leq R_i(Y,Z) - R_i(X,Z) + R_i(X,Y) - R_{i-1} (Y,Z) + \ldots \pm R_0(X,Y) 
= S_i(Y,Z) - S_i(X,Z) + S_i(X,Z) \]
\end{Beweis}



Mit dem Lemma folgt nun
\[ S_i(M) \leq \sum_i S_i(M^{a_i}, M^{a_{i-1}}) = c_\iota - c_{\iota -1} + \ldots \pm c_0. \]
Hieraus folgen nun die \df{Starken Morse-Ungleichungen}:
\[ R_\iota(M) - R_{\iota-1}(M) + \ldots \pm R_0(M) \leq c_\iota - c_{\iota - 1} + \ldots \pm c_0 \]
für alle $\iota$.

\subsection{Anwendungen}
Sei $M^2\subset \R^3$ seine glatt eingebettete geschlossene Fläche (daraus folgt, dass $M$ orientierbar ist).

Sei $N_p$ für $p \in M$ ein Normalenfeld auf $M$ mit
\[ \norm{N_p} = 1. \]
Wir betrachten die \df{Gauss-Abbildung}
\begin{align*}
G : M & \Pfeil{} S^2\\
p & \longmapsto N_p.
\end{align*}
Wir haben einen Abbildungsgrad $\deg G \in \Z$, den wir uns näher anschauen wollen. Sei dazu $N_0 \in S^2$ ein regulärer Wert für $G$, sodass auch $-N_0$ ein regulärer Wert ist (so einen Wert findet man durch den Satz von Sard).

Es gilt nun
\[ \deg G = \sum_{p \in G\i(N_0)} \e_p \]
für
\[ \e_p = \left\lbrace
\begin{aligned}
&+1, && \d G \text{ ist orientierungserhaltend},\\
&-1, && \text{ sonst.}
\end{aligned}
\right. \]
Ferner gilt
\[ 2 \deg G= \sum_{p \in G\i(N_0)} \e_p + \sum_{p \in G\i(-N_0)} \e_p. \]
Wir wenden nun das Morse-Lemma\footnote{Wir nehmen hierbei an, dass $f$ eine Morse-Funktion ist. Dies muss im Allgemeinem nicht der Fall sein, allerdings kann man in diesen Fällen durch leichtes Variieren von $f$ eine sehr nahe Morse-Funktion finden.} für $f : M \pfeil{} \R$ mit
\[f(x) = \shrp{x, N_0} \]
an.
Die kritischen Punkte von $f$ sind dann gerade
\[  C(f) = G\i(\pm N_0).\]
Ist $p \in C(f)$, so muss $f$ in einer Umgebung von $p$ eine der folgenden Gestalten haben
\begin{center}
	\begin{tabular}{c|c|c}
		$f$ & Index& Orientierung\\\hline
		$x^2 + y^2$ & 0 & bleibt erhalten\\
		$-x^2 - y^2$ & 2 & bleibt erhalten\\
		$x^2 - y^2$ & 1 & wird umgekehrt\\
		$-x^2 + y^2$ & 1 & wird umgekehrt
	\end{tabular}
\end{center}
Somit gilt
\[ \sum_{p \in G\i(N_0)} \e_p + \sum_{p \in G\i(-N_0)} \e_p = c_0 - c_1 + c_2 = \chi(M). \]
Es folgt
\[ \deg(G) = \frac{1}{2} \chi(M). \]

\marginpar{Vorlesung vom 18.06.18}
%18.te Vorlesung

Sei $\omega \in \Omega(S^2), \int_{S^2}\omega \neq 0$. Dann gilt
\[ \deg G = \frac{\int_M G^* \omega}{\int_{S^2} \omega }.  \]
Es bezeichne $g$ die durch die euklidische Metrik induzierte Metrik auf $S^2 \subset \R^3$. Sei $\omega$ die Riemannsche Volumenform auf $S^2 \subset \R^3$. Dann folgt
\[ \int_{S^2}\omega = 4\pi \]
und somit
\[ 4\pi \deg G = \int_M G^* \omega. \]
Bezeichnet $\kappa$ die Gauß-Krümmung auf $M$, so gilt für $p \in M$
\[ \kappa(p) = \lim\limits_{\theta \pfeil{} p} \frac{\vol (G(\theta))}{\vol(\theta )}
= \lim\limits_{\theta \pfeil{} p} \frac{\int_{G(\theta)}\omega}{\int_\theta\d A} = \lim\limits_{\theta \pfeil{} p} \frac{\int_{M}G^*\omega}{\int_\theta\d A}  \]
wobei $\d A$ die Riemannsche Volumenform auf $M$ ist. Daraus folgt
\[ \kappa(p) \d A = G^* \omega. \]
Und ferner
\[ 2\pi \chi(M) = 2\pi (2 \deg G ) = 4\pi \deg G = \int_M \kappa \d A. \]
Wir haben nun folgenden Satz gezeigt.
\Satz{Gauss-Bonnet}
\[ \int_M \kappa \d A = 2\pi \chi(M). \]
\Bem{}
Man kann Gauss-Bonnet auch alternativ unabhängig von der Einbettung beweisen. Sei dazu $(M^2, g)$ eine (abstrakte) Riemannsche Mannigfaltigkeit der Dimension 2, geschlossen und orientiert.

Zunächst nehmen wir an, dass $M$ gerade $S^2$ ist. Für ein geodätisches Dreieck $\Delta$ mit Innenwinkeln $\alpha, \beta, \gamma$ und einer Fläche $A$ auf $S^2$ kann man folgendes nachweisen
\[ A= \alpha + \beta + \gamma - \pi. \]
Da $\kappa = 1$ konstant auf $S^2$ ist, folgt
\[ \int_{\Delta}\kappa \d A = \alpha + \beta + \gamma - \pi.  \]
Dies gilt auch für beliebige $M$.


Ein allgemeines $M$ kann man nun durch geodätische Dreiecke triangulieren. Bezeichnet $f$ die Anzahl der Dreiecke, $e$ die Anzahl der Kanten und $v$ die Anzahl der Eckpunkte, so gilt
\begin{align*}
\int_M \kappa \d A &= \sum_{\Delta} \int_{\Delta}\kappa \d A\\
&= \sum_{\Delta} (\alpha_\Delta + \beta_\Delta + \gamma_\Delta - \pi)\\
&= \sum_{\Delta} (\alpha_\Delta + \beta_\Delta + \gamma_\Delta) - \pi f\\
&= 2\pi v- \pi f = 2\pi (v-e+f) = 2\pi \chi(M).
\end{align*}

\Bem{Folgerung}
\begin{itemize}
	\item Wenn $\kappa>0$ überall auf $M$ gilt, dann folgt aus der gezeigten Gleichung
	\[ 2-2g = \chi(M) > 0  \]
	Daraus folgt $g = 0$, ergo ist $M$ homöomorph zu $S^2$.
	\item Ist $\kappa = 0$ überall auf $M$, so muss das Geschlecht von $M$ Eins sein. In diesem Fall ist $M$ sogar diffeomorph zu $T^2$.
	\item Wenn $\kappa < 0$ überall auf $M$ ist, so folgt
	\[ g \geq 2. \]
	Daraus folgt, dass $M$ weder homöomorph zu $S^2$ noch $T^2$ ist.
\end{itemize}

\paragraph{Frage}
Gibt es auf Flächen mit Geschlecht $g\geq 2$ eine Metrik mit $\kappa<0$ überall.

\Def{I. Konstruktion}
Wir definieren die \df{Hyperbolische Ebene} durch den Raum
\[ \H = \set{z \in \C}{\Im z > 0} \]
und der Metrik
\[ \frac{1}{y^2(\d x^2 + \d y^2)}. \]
$\H$ ist dann vollständig als Riemannsche Mannigfaltigkeit und hat eine konstante Schnittkrümmung von $-1$.

Für hyperbolische Dreiecke gilt
\[ \alpha + \beta + \gamma \leq \pi \]
und
\[ \vol(\Delta) = \pi - (\alpha + \beta + \gamma). \]
D.\,h., kleine hyperbolische Dreiecke sind nahezu hyperbolisch. Wir betrachten ein gleichmäßiges $n$-Eck. Dann existiert ein $r> 0$, sodass jeder Innenwinkel die Größe $\frac{1}{n}2\pi$ hat. In diesem Fall kann gegenüberliegende Seiten isometrisch verkleben und erhält so eine geschlossene Fläche mit konstanter Schnittkrümmung $-1$.

\Bem{}
Die Polygone können verwendet werden, um $\H$ zu partitionieren und eine universelle Überlagerung zu erhalten.

\marginpar{Vorlesung vom 20.06.18}
%19.te Vorlesung

\Def{II. Konstruktion}
Sei $z$ eine komplexe Variable. Wir betrachten die Funktion
\begin{align*}
f(z) := \sqrt{(z-1)(z-2)(z-3) \cdots (z- (2g+2))}.
\end{align*}
Wir betrachten zwei Kopien $C_1, C_2$ von $\C\cup \{\infty\} \isom{} S^2$ und markieren die Stellen 1, 2, 3, \ldots, $2g+2$ auf der reellen Achse. Wir reißen das Intervall $[1,2]$ auf in zwei Seiten $a$ und $a'$ und verkleben $a$ von $C_1$ mit $a'$ von $C_2$ und $a'$ von $C_1$ mit $a$ von $C_2$. Analog verfahren wir mit $[3,4], [5,6], \ldots, [2g+1, 2g+2]$. $f$ wird dann zu einer wohldefinierten stetigen Funktion
\[ f : \C \Pfeil{} C_1 \cup C_2 / \sim. \]

Die Riemannsche Fläche von $f$ ist entspricht dann zwei Sphären, in die man $g+1$ Löcher jeweils reingerissen hat und die Sphären an den entsprechenden Löchern verklebt hat. Die Naben der sich so ergebenden verbindenden Zylinder sind $a$ und $a'$. Wir nennen diese Mannigfaltigkeit $\Sigma$.

Sei $\pi \colon \widetilde{\Sigma} \pfeil{} \Sigma$ die universelle Überlagerung. $\widetilde{\Sigma}$ trägt eine komplexe Struktur, weswegen der Uniformisierungssatz liefert, dass $\widetilde{\Sigma}$ entweder $\H$ oder $\C$ ist.

Angenommen $\widetilde{\Sigma}$ wäre $\C$. Die analytischen Automorphismen von $\C$ sind von der Form
\[ z \longmapsto a z +b. \]
Bezeichnet $\pi_1 \Sigma$ die Fundamentalgruppe von $\Sigma$, so wirkt sie eigentlich diskontinuierlich und analytisch auf $\widetilde{\Sigma}$. Die Elemente von $\pi_1\Sigma$ haben keine Fixpunkte, da $\pi_1 \Sigma$ die Decktransformationsgruppe der Überlagerung ist. Daraus folgt für alle $z$ und eine Wirkung
\begin{align*}
az+b &\neq z\\
(a-1)z+b &\neq 0
\end{align*}
Daraus folgt $a = 1$. Also haben die Elemente von $\pi_1 \Sigma$ die Gestalt $z\mapsto z + b$. Daraus würde folgen, dass $\pi_1 \Sigma$ abelsch wäre. Dies widerspricht aber der Tatsache, dass $\pi_1 \Sigma$ nicht abelsch ist, da $\Sigma$ ein Geschlecht größer gleich 2 hat. Dies widerspricht unserer Annahme, ergo gilt
\[ \widetilde{\Sigma} =  \H. \]

Es ergibt sich ergo eine analytische eigentlich diskontinuierliche Wirkung
\[ \pi_1 \Sigma \curvearrowright \H. \]
Die analytischen Transformationen von $\H$ sind \df{Möbiustransformationen}, d.\,h., sie sind von der Gestalt
\[ z \longmapsto \frac{az+b}{cz + d} \]
mit
\[ ad - bc > 0.\]
Diese Transformationen sind aber Isometrien für die Metrik $\frac{1}{y^2}(\d x^2 + \d y^2)$ auf $\H$.

Daraus folgt, dass eine eindeutig bestimmte Metrik $\shrp{\cdot, \cdot}$ auf $\Sigma$ existiert, sodass gilt
\[ \pi^*\shrp{\_,\_} = \frac{1}{y^2} (\d x^2 + \d y^2). \]
$\pi$ wird dadurch zu einer Isometrie, ergo hat $\Sigma$ ebenfalls eine konstante Schnittkrümmung von $-1$ mit der gegebenen Metrik.


\section{Rahmenfelder (E. Carton)}
Sei $M$ eine Riemannsche Mannigfaltigkeit, $U \subset M$ offen.
\Def{}
Ein \df{Rahmenfeld} auf $U$ ist ein Tupel
\[ \{e_1, \ldots, e_n\} \]
für $n = \dim M$, wobei $e_i\in \Gamma(\T U)$, sodass
\[ \{e_1(p), \ldots, e_n(p)\} \]
eine ONB für $T_pM$ ist für alle $p \in U$.

\Bsp{}
$M = S^2$ hat kein globales Rahmenfeld.
\begin{itemize}
	\item \textbf{Duale 1-Formen:} $\theta^1, \ldots, \theta^n \in \Omega^1(U),$ mit
	\[ \theta^i(e_j) = \delta_{i,j}. \]
	Man setzt nun
	\[ \theta := 
	\left(
	\begin{matrix}
	\theta^1\\
	\vdots\\
	\theta^n
	\end{matrix}
	\right).
	 \]
	 \item \textbf{Zusammenhangs-1-Formen:} Es existiert genau eine 1-Form $\omega^j_i$ auf $U$, sodass gilt:
	 \begin{enumerate}[(1)]
	 	\item $\omega^j_i = -\omega^i_j$,
	 	\item \df{die Erste Strukturgleichung:}
	 	\[\d \theta = - \omega \wedge \theta\] 
	 \end{enumerate}
 	\[ \omega^i_j (v) = \shrp{\nabla_v e_i, e_j}. \]
 	 \item \textbf{Krümmungs-2-Formen:}
 	 \[ \Omega := \d \omega + \omega \wedge \omega \]
 	 Dies Gleichung nennt man die \df{Zweite Strukturgleichung}.\\
 	 Es gilt dann
 	 \[ \Omega^i_j(v,w) = \shrp{ R(v,w)e_i, e_j}. \]
\end{itemize}

\Bsp{}
\begin{itemize}
	\item Auf $M = \R^n$ mit der euklidischen Metrik:
	\begin{align*}
	&e_i = \pf{}{x_i}\\
	\Impl{} & \theta^i = \d x_i\\
	\Impl{} & \d \theta = 0 = -\omega \wedge \theta\\
	\Impl{} & \omega = 0\\
	\Impl{} & \Omega= 0
	\end{align*}
	\item Auf einer Fläche $M^2$:
	\begin{align*}
	\omega = \left(
	\begin{matrix}
	0 & \omega^1_2\\
	- \omega^1_2 & 0
	\end{matrix}
	\right)
	\end{align*}
	und
	\begin{align*}
	\Omega &= \left(
	\begin{matrix}
	0 & \d \omega^1_2\\
	- \d \omega^1_2 & 0
	\end{matrix}
	\right)
	+
	\left(
	\begin{matrix}
	0 & \d \omega^1_2\\
	- \d \omega^1_2 & 0
	\end{matrix}
	\right)
	\wedge
	\left(
	\begin{matrix}
	0 & \d \omega^1_2\\
	- \d \omega^1_2 & 0
	\end{matrix}
	\right)\\
	&= \left(
	\begin{matrix}
	0 & \d \omega^1_2\\
	- \d \omega^1_2 & 0
	\end{matrix}
	\right)
	+
	\left(
	\begin{matrix}
	- \omega^1_2\wedge \omega^1_2 & 0\\
	0 & -\omega^1_2 \wedge \omega^1_2
	\end{matrix}
	\right) = 
		\left(
	\begin{matrix}
	- \omega^1_2\wedge \omega^1_2 & \d \omega_2^1\\
	-\d \omega_2^1 & -\omega^1_2 \wedge \omega^1_2
	\end{matrix}
	\right) 
	\end{align*}
	Daraus folgt
	\[ \Omega^1_2 = \d \omega_2^1 \]
	und
	\[ \Omega_2^1(e_1, e_2) = \shrp{R(e_1,e_2)e_1, e_2} = \kappa. \]
	Daraus folgt
	\[ \kappa = \d \omega^1_2. \]
\item \textbf{Basiswechsel:} Seien $\{ e_1, e_2 \}$, $\{e_1', e_2'\}$ zwei Rahmenfelder.
Dann gilt
\[ {\omega'}_2^1 = \omega^1_2 + \d \alpha, \]
wobei $\alpha$ $\{e_1, e_2\}$ zu $e_1', e_2'$ transformiert.
\end{itemize}


\subsection{Anwendung:}
Sei $M$ eine geschlossene, orientierte Fläche und $X$ ein Vektorfeld auf $M$ mit endlich vielen Singularitäten $p_1,\ldots, p_k \in M$.

Sei $\e > 0$. Wir betrachten um jedes $p_i$ eine Kreisscheiben $D_i(\e)$ mit Radius $\e$. Wir entfernen ihr Inneres und erhalten eine Mannigfaltigkeit mit Rand. Wir setzen
\[ M_\e := M - \bigcup_i D^o_i(\e). \]
Wir haben dann folgendes Rahmenfeld auf $M_\e$
\[ \{ e_i := \frac{X}{\norm{X}}, e_2 \}. \]

Es gilt
\[ \int_{M_\e} \kappa \d A = \int_{M_\e} \d \omega^1_2 \gl{\text{Stokes}} \int_{\partial M_\e} \omega^1_2
= \sum_i \int_{\partial D_i(\e)} \omega_2' = \sum_i \int_{\partial D_i(\e)} ({\omega'}_2^1 + \d \alpha_i) 
  \]
\[ = \sum_{i} \int_{\partial D_i(\e)} (\omega')^1_2 +\sum_{i} \int_{\partial D_i(\e)} \d \alpha_i \]
Mit
\[ \int_{\partial D_i(\e)} (\omega')^1_2 \gl{\text{Stokes}} \int_{D_i(\e)} \d {\omega'}^1_2 \]
und
\[ \int_{\partial D_i(\e)} \d \alpha_i = \text{Index}_{p_i}X \]
folgt
\[ \int_{M_\e} \kappa \d A \Pfeil{\e \pfeil{} 0} \sum_i \text{Index}_{p_i}X  = \int_{M} \kappa \d A \]
Auf $D_i(\e)$ betrachten wir dabei ein Rahmenfeld $\{ e_1', e_2' \}$.

Es folgt nun der \df{Poincaresche Indexsatz}
\[ 2\pi \chi(M) = \int_M \kappa \d A = \sum_i \text{Index}_{p_i}(X). \]


\chapter{Faserbündel}
\paragraph{Prototyp:} Lokal sieht ein Faserbündel $E \pfeil{} B$ aus wie $E = B \times F$.

\Def{}
Ein \df{Faserbündel} mit \df{Strukturgruppe} $G$ und Faser $F$ ist ein Quadrupel $(E,p,B, G)$ mit folgenden Eigenschaften:
\begin{enumerate}[1)]
	\item $E,B,F$ sind topologische Räume.
	\item $G$ ist eine topologische Gruppe, die effektiv auf $F$ wirkt\footnote{D.\,h., $G \pfeil{} \text{Homöo}(F)$ ist injektiv.}.
	\item $p: E \pfeil{} B$ ist eine stetige Abbildung.
	\item Es gilt \df{lokale Trivialität}:\\
	$B$ hat eine Überdeckung durch offene Mengen
	\[ B = \bigcup_\alpha U_\alpha, \]
	sodass für jedes $\alpha$ folgendes kommutierende Diagramm vorliegt:
	\begin{center}
		\begin{tikzcd}
		p\i(U_\alpha) \arrow[rd, "p|"] \arrow[rr, "\exists \phi_\alpha~~\text{homeom.}"] & & U_\alpha \times F \arrow[dl, "pr_1"] \\
		& U_\alpha&
		\end{tikzcd}
	\end{center}
Für $x \in U_\alpha \cap U_\beta$ soll der Homöomorphismus
\[ \phi_\beta \circ \phi_\alpha\i |_{\{x\}} : \{x\} \times F \Pfeil{}  \{x\} \times F   \]
in $G$ liegen.
	
\end{enumerate}
\marginpar{Vorlesung vom 25.06.18}
%20.te Vorlesung

%\begin{center}
%\begin{tikzcd}
%	p\i(U_\alpha) \arrow[rd, "p|"] \arrow[rr, "\exists \phi_\alpha~~\text{homeom.}"] & & U_\alpha \times F \arrow[dl, "pr_1"] \\
%	& U_\alpha&
%\end{tikzcd}
%\end{center}

\section{Der Satz von Leray-Hirsch}

\Bsp{}
Sei $p:E \pfeil{} B$ eine Projektion, $F$ eine Faser, auf die eine topologische Gruppe $G$ von links effektiv wirkt.
\begin{enumerate}[(1)]
	\item Vektorraumbündel sind Faserbündel $\R^n$ bzw. $\C^n$ mit Strukturgruppe $\mathrm{GL}_n$.
	\item Sei $F$ ein topologischer Raum und $\mu : F \pfeil{} F$ ein Homöomorphismus. Es bezeichne $I$ das Einheitsintervall.
	
	Betrachte das Faserbündel
	\[ p : E:= (I \times F)/((0,x) \sim (1, \mu(x))) \Pfeil{(x,y) \mapsto x} S^1. \]
	$p$ nennt man auch den \df{Abbildungstorus} zu $\mu$.
	\item Sei $ F = S^1 \subset \C$, $\mu (z) = \overline{z}$. Durch obiger Konstruktion ist $E$ die Kleinsche Flasche.
	\item Betrachte
	\[ S^{2n+1} \subset \R^{2n+2} = \C^{n+1}. \]
	Wir betrachten den komplexen projektiven Raum
	\begin{align*}
	\C P^n :=&	(\C^{n+1} - \{0\}) / (z\sim \lambda z,\lambda \in \C^\times = \C - \{0\})\\
	=& S^{2n+1} / (z \sim \lambda z,
	\bet{\lambda} = 1, \lambda \in \C
	 ) 
	\end{align*}
	Hierdurch ergibt sich ein Faserbündel $S^{2n+1} \Pfeil{} \C P^n$ mit Faser $S^1$. Diese Faserung nennt man auch die \df{Hopf-Faserung}.
	
	
 	Für $n = 1$ ergibt sich folgendes Diagramm.
 	\begin{center}
 	\begin{tikzcd}
 	S^1  \arrow[r]& S^3 \arrow[d, "p"] \\
 	& \C P^1 \arrow[r, "\isom{}"] & S^2
 	\end{tikzcd}
 	\end{center}
\end{enumerate}

\paragraph{Frage:} Inwieweit lässt sich der Künneth-Satz für die deRham-Kohomologie verallgemeinern auf Faserprodukte?


Seien $E \isom{} B \times F$ alles Mannigfaltigkeiten. Wir erhalten dann eine Abbildung
\begin{align*}
H^p(B) \otimes H^q(F) &\Pfeil{} H^{p+q}(E)\\
[\omega] \otimes [\eta] & \longmapsto [\pi_1^*\omega \wedge \pi_2^*\eta].
\end{align*}
Wir fassen diese Abbildungen zu einem Isomorphismus zusammen
\begin{align*}
\bigotimes_{p+q = k}H^p(B) \otimes H^q(F) &\Pfeil{} H^{k}(E).
\end{align*}

%\Satz{Leray-Hirsch}
Sei $F \inj{j} E \pfeil{p} B$ ein Faserbündel.
\begin{center}
	\begin{tikzcd}
	H^*(F) & H^*(E) \arrow[l, "j^*"]\\
	& H^*(B) \arrow[u, "p^*"]
	\end{tikzcd}
\end{center}
$H^*(E)$ ist ein $H^*(B)$-Modul durch
\[ \alpha \cdot x := p^*(\alpha) \wedge x \]
für $\alpha \in H^*(B), x \in H^*(E)$.

\Def{}
Das Faserbündel $E \pfeil{p} B$ heißt \df{kohomologisch gespalten}, wenn es eine additive Abbildung
\[ \beta : H^*(F) \Pfeil{} H^*(E) \]
existiert mit
\[ j^* \beta = \id{H^*(F)}. \]
\paragraph{Bemerkung:} $\beta$ muss nur linear sein, aber kein Ring-Homomorphismus.
\paragraph{Bemerkung:} Da wir mit reellen Koeffizienten arbeiten, genügt es für die Existenz einer Spaltung $\beta$ anzunehmen, dass $j^*$ surjektiv ist.
\paragraph{Bemerkung:} Wenn $p: E \pfeil{} B$ kohomologisch gespalten ist mit Spaltung $\beta$, dann ist $j^*$ surektiv und $\beta$ injektiv.

\Def{}
Sei nun $p : E \pfeil{} B$ ein kohomologisch gespaltenes Bündel. Wir definieren dann
\begin{align*}
\phi : H^*(B) \otimes H^*(F) &\Pfeil{} H^*(E)\\
\alpha \otimes y & \longmapsto p^*(\alpha) \wedge \beta(y)
\end{align*}
\paragraph{Bemerkung:}
$\phi$ ist linear, aber im Allgemeinem \textbf{nicht} multiplikativ.

\Satz{}
Ist $E \pfeil{p} B$ ein kohomologisch gespaltenes Faserbündel, dann ist $\Phi$ ein Isomorphismus.
\begin{Beweis}{}
Wie beim Satz von Künneth über Mayer-Vietoris-Sequenzen und dem Fünfer-Lemma.
\end{Beweis}


\Bsp{}
\begin{itemize}
	\item Hopf-Faserung $S^1 \pfeil{} S^3 \pfeil{p} S^2$. Es gilt
	\[ H^*(S^3) \not\isom{} H^*(S^2) \otimes H^*(S^1). \]
	Insoweit ist die Forderung der kohomologischen Spaltung zwingend notwendig. Tatsächlich ist $p : S^3 \pfeil{} S^2$ nicht kohomologisch gespalten, denn
	\[ j^* : H^1(S^3) = 0 \Pfeil{} H^1(S^1) \isom{} \R \]
	ist trivial auf Grad 1.
	\item Sei $E \pfeil{} B$ ein Vektorraumbündel mit Faser $\R^n$ bzw. $\C^n$. Bezeichnet $\P$ die Projektivizierung eines Raumes, so kann man sich die projektive Versionen der Fasern anschauen und diese zusammensetzen zu einem Faserbündel
	\[ \P (\R^n) \Pfeil{} \P(E) \Pfeil{} B \]
	bzw.
	\[ \P (\C^n) \Pfeil{} \P(E) \Pfeil{} B. \]
	Dies nennt man die \df{Projektivisierung} von E.
\end{itemize}

\paragraph{Fakt:}
Projektivisierungen von Vektorraumbündeln sind immer kohomologisch gespalten.

\chapter{Charakteristische Klassen}
\section{Eulerklasse}
Sei $\R^2 \pfeil{} E \pfeil{} M = B$ ein orientiertes Vektorraumbündel über einer glatten Mannigfaltigkeit $M$.

Sei $\sigma \in \Omega^1(S^1)$ eine Form, die den Erzeuger $[\sigma] \in H^1(S^1)$ repräsentiert. Z.Bsp. $\sigma = \d \theta$, wobei $\theta = \tan(x,y) : S^1 \pfeil{} \R$ den Winkel auf $S^1$ darstellt.

Sei $\rho : \R^2 \setminus \{0\} \pfeil{} S^1$ die übliche Retraktion. Setze $\Psi := \rho^*\sigma \in \Omega^1(\R^2 - \{0\})$.

Se ferner $f : \R \pfeil{} [0,1] $ glatt und so, dass $f'$ eine Bump-Function ist.

$f\Psi$ liegt in $\Omega^1(\R^2 - 0)$ und es gilt
\[ \d(f\Psi) = \d f \wedge \Psi. \]
$\d(f \Psi)$ ist eine geschlossene Form. Betrachte
\[ [\d (f\Psi) ] \in H_c^2(\R^2 - 0) \isom{} \R. \]
Da $\int \d f \wedge \Psi$ nicht Null ist, ist $[\d (f\Psi) ]$ ein Erzeuger von $H_c^2(\R^2 - 0) $.\\
$[\d (f\Psi) ]$  nennt man auch die \df{Thom-Klasse} für das Bündel $\R^2 \pfeil{} \mathrm{Pkt.}$.


Wähle eine Metrik auf $E$. Diese induziert einen Radius $r = \norm{v}$ für $v \in E$.\\
Wähle eine offene Überdeckung $\{U_\alpha\}_\alpha$ von $M$, sodass 
\begin{center}
	\begin{tikzcd}
p\i (U_\alpha) \arrow[rd, "p|"] \arrow[rr, "\isom{}"] & & U_\alpha \times \R^2 \arrow[ld, "\mathrm{pr}_1"]\\
& U_\alpha & 	
	\end{tikzcd}
\end{center}
Wähle Rahmenfelder auf $p\i(U_\alpha)$ für alle $\alpha$. Diese ergeben für $\alpha$ Polarkoordinaten  $(r_\alpha, \theta_\alpha)$ gegen den Uhrzeigersinn. Dies ist möglich, da $E$ orientiert ist.
\marginpar{Vorlesung vom 27.06.18}
%21.te Vorlesung
\begin{itemize}
	\item Es gilt $r_\alpha = r_\beta$ auf
	\[ U_{\alpha \beta} := U_\alpha \cap U_\beta. \]
	\item Setze
	\[ \phi_{\alpha \beta} := \theta_\alpha - \theta_\beta. \]
	\item Auf $U_\alpha \cap U_\beta \cap U_\gamma$ gilt im Allgemeinem
	\[  \phi_{\alpha \beta} + \phi_{\beta \gamma} - \phi_{\alpha \gamma} \neq 0, \]
	sondern es gilt lediglich
	\[ \phi_{\alpha \beta} + \phi_{\beta \gamma} - \phi_{\alpha \gamma} \in 2\pi \Z. \]
	\item Sei $\{g_\alpha\}_\alpha$ eine Partition der Eins bzgl. $\{U_\alpha\}_\alpha$. Setze
	\[ \xi_\alpha := \frac{1}{2\pi} \sum_\gamma g_\gamma \d \phi_{\gamma\alpha} \in \Omega^1(U_\alpha). \]
	Es gilt
	\[ \xi_\beta - \xi_\alpha = \frac{1}{2\pi} \sum_\gamma g_\gamma (\d \phi_{\gamma\alpha} - \d \phi_{\gamma\beta}). \]
	Nun gilt aber auch
	\[ \d \phi_{\alpha \beta} = \d (2\pi n) + \d \phi_{\alpha \beta} = \d \phi_{\gamma\beta} - \d \phi_{\gamma\alpha} \]
	für ein $n \in \Z$. Somit gilt
	\[ \xi_\beta - \xi_\alpha = \frac{1}{2\pi} \sum_\gamma g_\gamma \d \phi_{\alpha\beta} = \frac{\d \phi_{\alpha\beta}}{2\pi} \sum_\gamma g_\gamma  = \frac{\d \phi_{\alpha\beta}}{2\pi} . \]
	Insbesondere folgt nun
	\[ \d\xi_\beta - \d \xi_\alpha = 0. \]
	D.\,h., die $\{ \d \xi_\alpha \}$ definieren eine globale 2-Form auf $M$. Diese 2-Form nennen wir $\eu \in \Omega^2(M)$. $\eu$ ist geschlossen, denn
	\[ \d \eu_{|U_\alpha} = \d\d \xi_\alpha = 0. \]
	Daraus folgt, dass $\eu$ eine Kohomologieklasse
	\[ \eu(E) := [\eu] \in H^2(M) .\]
	$\eu(E)$ nennt man die \df{Eulerklasse} von $E$ (unabhängig von Wahlen).
\end{itemize}

\Bem{}
Es gibt im Allgemeinem keine globale 1-Form $\xi$ auf $M$, sodass $\d \xi = \eu$. Die $\{\xi_\alpha\}_{\alpha}$ lassen sich im Allgemeinem nicht verkleben zu einer globalen Form.

\subsection{Eigenschaften der Eulerklasse}
\begin{itemize}
	\item Eulerklasse des trivialen Bündels:
	\[ E = M\times \R^2 \Impl{} \phi_{\alpha\beta} = 0 \Impl{} \xi_\alpha = 0 \Impl{} \eu = 0. \]
	Daraus folgt
	\[ \eu(E) = 0 \in H^2(M). \]
	\item \textbf{Formal mit Übergangsfunktionen:}
	\[ g_{\alpha \beta} :U_{\alpha \beta} \Pfeil{} \mathrm{GL}(2, \R) ~~\text{a priori.} \]
	Die Wahl der Riemannschen Metrik und der Orientierung des Bündels führen zu einer Reduktion der Strukturgruppe auf
	\[ g_{\alpha \beta} :U_{\alpha \beta} \Pfeil{} \mathrm{SO}(2) = \set{
\left(\begin{matrix}
\cos \theta & -\sin \theta\\
\sin \theta & \cos \theta
\end{matrix}\right)	= e^{i\theta}
}{\theta \in [0,2\pi]} \isom{} S^1. \]
Es gilt
\[ \phi_{\alpha \beta} = \theta_\beta - \theta_\alpha = \frac{1}{i} \log e^{i(\theta_\beta - \theta_\alpha)} = \frac{1}{i} \log g_{\alpha \beta}. \]
Und somit
\begin{align*}
\xi_\alpha &= \frac{1}{2\pi} \sum_\gamma g_\gamma \d (\frac{1}{i} \log g_{\gamma \alpha})\\
 &= \frac{1}{i2\pi} \sum_\gamma g_\gamma \d \log g_{\gamma \alpha}.\\
\end{align*}
Daraus folgt diese Darstellung der Eulerform durch Übergangsfunktionen des Bündels
\[ \eu_{|U_\alpha} = \d \xi_\alpha =
\frac{1}{i2\pi} \sum_\gamma \d (g_\gamma \d \log  g_{\gamma \alpha}).
 \]
	\item \textbf{Pullbacks von Vektorraumbündeln:}\\
	\begin{itemize}
		\item $X,Y$ topologische Räume.
		\item Sei $\R^n \pfeil{} E \pfeil{p} Y$ ein Vektorraumbündel über $Y$.
		\begin{center}
			\begin{tikzcd}
			f^*E \arrow[r] \arrow[d, "q"] & E \arrow[d, "p"]\\
			X \arrow[r, "f~ \text{stetig}"] & Y
			\end{tikzcd}
		\end{center}
	Mit
	\[ f^*E = \set{(x,v)}{f(x) = p(v)} \subset X \times E \]
	ist $q$ die Projektion auf $X$.
	\end{itemize}
\end{itemize}
\Bsp{}
\begin{enumerate}[1)]
	\item Ist $Y$ ein Punkt und $f : X \pfeil{} \text{Pkt.}$ mit dem Bündel $E = \R^n \pfeil{} \text{Pkt.}$, so ist $f^* E = X \times \R^n$ das triviale Vektorbündel.
	\item Ist $Y$ allgemein und $E$ trivial, so ist auch $f^* E$ trivial.
	\item Ist $f$ eine Inklusion, so ist $f^*E$ die Einschränkung von $E$ auf $X \subset Y$. 
\end{enumerate}

\begin{itemize}
	\item Sei $U_\alpha$ eine offene Überdeckung von $Y$, sodass $p\i(U_\alpha) = U_\alpha \times \R^n$ jeweils trivial ist. Setze
	\[ V_\alpha := f\i(U_\alpha) \subset_{\text{offen}} X. \]
	Dann ist
	\[f^*E|_{V_\alpha}\isom{} f^*(U_\alpha \times \R^n) = V_\alpha \times \R^n\]
	ebenfalls trivial.
	Die Übergangsfunktionen von $f^*E \pfeil{} X$ sind dann
	\begin{center}
		\begin{tikzcd}
		V_{\alpha \beta} = V_\alpha \cap V_\beta \arrow[r, "f"] \arrow[rd, "g_{\alpha \beta}\circ f"] & U_{\alpha \beta}\arrow[d, "g_{\alpha \beta}\circ f"]\\
		& \text{GL}(n,\R)
		\end{tikzcd}
	\end{center}
$g_{\alpha \beta}\circ f$ ist dann die Übergangsfunktion für $f^*E \Pfeil{} X$.
\end{itemize}

\subsection{Eulerklasse eines Pullbacks}
Sei $f : M \pfeil{} N$ eine glatte Abbildung von Mannigfaltigkeiten. $\R^2 \pfeil{} E \pfeil{} N$ sei eine Vektorraumbündel. $f^*E \pfeil{} M$  bezeichne den Pullback von $E$. Es gilt
\begin{align*}
\eu(f^*E) 
= &
\frac{1}{2\pi i}
\sum_\gamma \d (g_\alpha \circ f \cdot \d \log (g_{\gamma\alpha} \circ f))\\
= & \frac{1}{2\pi i}
\sum_\gamma f^*\d (g_\alpha \cdot \d \log (g_{\gamma\alpha} ))\\
= & f^*\eu (E)
\end{align*}
Wir haben also gezeigt
\[ \eu (f^* E) = f^*\eu(E) \in H^2(M), \]
wobei $f^*\colon H^2(N) \pfeil{} H^2 M$.


\section{Chernklassen}
Betachte ein Bündel $\C^n \pfeil{} E \pfeil{} B$. Die Strukturgruppe ist $\mathrm{GL}(n,\C)$.

Sei $V$ ein Vektorraum über $\C$. Durch Vergessen der komplexen Struktur erhalten wir den unterliegenden reellen Vektorraum $V_\R$. Es gilt
\[ \dim_\C V = n  \Impl{} \dim_\R V_\R = 2n. \]
Führt man dies faserweise durch, so erhält man ein unterliegendes reelles Vektorraumbündel $\R^{2n} \pfeil{} E_\R \pfeil{} B$.

$E_\R$ ist kanonisch orientiert, denn sei $\{v_1, \ldots, v_n\}$ eine Basis von $V$ als komplexer Vektorraum, dann ist $\{v_1, iv_1, \ldots, v_n, iv_n\}$ eine Basis für den unterliegenden reellen Vektorraum. Diese Basis definiert die Orientierung von $V_\R$. Da $\mathrm{GL}(n,\C)$ im Gegensatz zu $\mathrm{GL}(n,\R)$ zusammenhängend ist, sind alle Basen für $V$ äquivalent.

Ein $\C^1$-Bündel heißt auch \df{Geradenbündel}.

Sei $E \pfeil{} M$ ein komplexes Geradenbündel. Dann hat $E$ ein unterliegendes reelles Vektorraumbündel $E_\R$ mit Faser $\R^2$ und dieses Bündel ist kanonisch orientiert.

Die Eulerklasse von $E_\R$ ist daher wohldefiniert und wir setzen
\[ c_1(E) := \eu (E_\R) \in H^2(M). \]
$c_1$ nennt man die erste \df{Chernklasse} von $E$.
\marginpar{Vorlesung vom 02.07.18}
%22.te Vorlesung

\section{Die Thom-Klasse}
\Def{}
Sei $\R^2 \pfeil{} E \pfeil{\pi} M$ ein orientiertes Vektorraumbündel über einer kompakten Mannigfaltigkeit $M$. Wir erinnern uns daran, dass für die Karten aus dem vorhergehenden Abschnitt gilt
\[ \d \theta_\beta - \d \theta_\alpha = \d \phi_{\alpha\beta} = 2\pi (\xi_\beta - \xi_\alpha) \]
für
\[ \xi_\alpha = \frac{1}{2 \pi} \sum_\gamma g_\gamma \d \phi_{\gamma\alpha}. \]
Ferner gilt
\[ \d \theta_\beta - 2\pi \xi_\beta = \d \theta_\alpha - 2\pi \xi_\alpha \]
und somit
\[ \frac{\d \theta_\beta}{2 \pi} - \xi_\beta = \frac{\d \theta_\alpha}{2\pi} - \xi_\alpha \]
auf $U_{\alpha \beta} = U_\alpha \cap U_\beta$.

Die $\{\frac{\d \theta_\alpha}{2 \pi} - \xi_\alpha\}_\alpha$ definieren eine globale 1-Form $\Psi \in \Omega^1(E_0)$, die sogenannte \df{globale Winkelform}, für
\[ E_0 = \set{v\in E}{v\neq 0} = E - \mathrm{0-Schnitt}. \]

Wähle eine glatte monotone Funktion $f(r)$ mit folgenden Eigenschaften
\begin{align*}
f(r) = \left\lbrace
\begin{aligned}
&-1, && r \leq 0\\
&\in [0,1], && r \in [0, 1]\\
&0, && r \geq 1.
\end{aligned}
\right.
\end{align*}
Die Ableitung von $f$ ist dann eine bump function.

Wir setzen
\[ \Phi := \d (f \cdot \Psi). \]
Es gilt
\[ \d \Psi = \d (\frac{\d \theta_\alpha}{2 \pi} - \xi_\alpha) = \d \xi_\alpha = -\pi^* \eu, \]
d.\,h.,
\[ \d \Psi = -\pi^*\eu. \]
Somit gilt
\[ \Phi = \d (f \cdot \Psi) = \d f \wedge \Psi + f \d \Psi = \d f \wedge \Psi - f \pi^*\eu.  \]
Ergo hat $\Phi$ einen kompakten Träger, d.\,h.
\[ \Phi \in \Omega^2_c (E). \]
Ferner ist $\Phi$ offensichtlich geschlossen und repräsentiert somit eine Kohomologieklasse
\[ \Phi := [\Phi] \in H^2_c(E). \]
$\Phi$ nennt man die \df{Thom-Klasse} von $E \pfeil{} M$.



Sei $s : M \pfeil{} E$ der Nullschnitt, d.\,h.
\[ \pi \circ s = \id{M}, \]
woraus folgt
\[ s^* \pi^* = \id{H^*(M)}. \]
Betrachte nun
\[ s^*\Phi = s^* \d (f\Psi) = \d (s^* (f\Psi)) = \d ( (f\circ s)(r) \cdot s^*\Psi) = \d (f(0) \cdot s^*\Psi ) = - \d (s^* \Psi) = - s^*(\d \Psi) = s^*\pi^* \eu = \eu. \]
Somit folgt
\[ s^*(\text{Thom-Klasse}) = \text{Eulerklasse von E}. \]

\Satz{Isomorphiesatz von Thom}
Sei $M$ kompakt. Die Abbildung
\begin{align*}
H^*(M) & \Pfeil{} H^{*+2}_c(E)\\
[\omega] & \longmapsto [(\pi^*\omega) \wedge \Phi ]
\end{align*}
ist ein Isomorphismus. Er heißt auch \df{Thom-Isomorphismus}.
\begin{Beweis}{}
Wir besprechen zwei mögliche Beweise.
\begin{enumerate}[1.]
	\item Man benutzt wie gewohnt die Mayer-Vietoris-Sequenz, das Fünfer-Lemma, etc..
	\item Wir verwenden die Intergation entlang der Faser
	\begin{align*}
\pi_* : H^{* + 2}_c (E )  \Pfeil{} H^*(M).
\end{align*}
	Es gilt dann die sogenannte \df{Projektionsformel}:
	
	Sind $\omega \in H^*(M)$ und $\eta \in H^*_c(E)$, so gilt
	\[ \pi_*(\pi^* \omega \wedge \eta) = \omega \wedge \pi_*\eta.\]
	Für $\eta = \Psi$ gilt
	\[ \pi_*\Phi = 1. \]
	Und somit
	\[ 
	\pi_*\Phi_{|\text{Faser}} = \int_{0}^\infty \int_{0}^{2\pi} \d f \frac{\d \theta}{2\pi} = f(\infty) - f(0) = 0 - (-1) = 1.
	 \]
	 Mit der Projektionsformel gilt
	 \[ \pi_*(\pi^* \omega \wedge \Phi) = (\omega\wedge \pi_* \Phi) = \omega \wedge 1 = \omega. \]
	Daraus folgt, dass der Thom-Isomorphismus ist invers zu $\pi_*$, der Integration der Faser.
\end{enumerate}
\end{Beweis}

\section{Gysin-Sequenz}
Betrachte wieder das Bündel $E \pfeil{\pi} M$. Es trägt eine Riemannsche Metrik, insofern können wir das \df{Disk-Bündel} definieren durch
\[ D(E) := \set{v \in E}{\norm{v} \leq 1}. \]
Das \df{Sphären-Bündel} definiert man analog durch
\[ S(E) := \set{v \in E}{\norm{v} = 1}. \]
Wir erhalten dann Faserbündel.
\begin{center}
	\begin{tikzcd}
	D^2  \arrow[r]& D(E)\arrow[d, "\pi|"] & & S^1 \arrow[r] & S(E)\arrow[d, "\pi|"]\\
	& M & & & M  
	\end{tikzcd}
\end{center}
$D(E)$ ist dann eine glatte Mannigfaltigkeit mit $S(E)$. Ferner gilt
\[ D(E) \simeq M \]
durch eine von $\pi$ induziert Homotopie. Und
\[ S(E) \simeq E_0 \]
durch die Homotopie $\R^2 - 0 \simeq S^1$.

Wir betrachten
\[ H^*(D(E), S(E))  \isom{} H^*_c(D(E) - S(E)). \]
Da folgende Diffeomorphie vorliegt
\[ D(E) - S(E) \isom{} E, \]
gilt nun 
\[ H^*(D(E), S(E))  \isom{} H^*_c(D(E) - S(E)) \isom{} H^*_c(E). \]
Wir betrachten die lange exakte Sequenz
\begin{center}
\begin{tikzcd}
H^{j-1}(SE) \arrow[r, "\delta^*"] & H^j(DE, SE) \arrow[r] \arrow[d, "\isom{}"] & H^j(DE) \arrow[r]\arrow[d, "\pi^*; \isom{}"]  & H^j (SE) \arrow[r, "\delta^*"] \arrow[d, "\isom{}"]& H^{j+1}(DE, SE)\arrow[d, "\isom{}"]\\
& H^j_c(E) \arrow[d, "\isom{}"] & H^j(M) & H^j(E_0) & H^{j-1}(M)\\
& H^{j-2}(M) \arrow[ur, "\_ \wedge \eu"]
\end{tikzcd}	
\end{center}
Wir erhalten folgende exakte Sequenz, die sogenannte \df{Gysin-Sequenz}
\[ H^{j-1}(E_0) \Pfeil{\delta^*} H^{j-2} (M) \Pfeil{\_ \wedge \eu} H^j(M) \Pfeil{} H^j(E_0) \Pfeil{\delta^*} H^{j-1}(M). \]

\section{Euler-/Chernklasse des Tautologischen Geradenbündels über $\C\P^{n-1}$}
Wir schreiben in diesem Abschnitt kurz
\[ \P^{n-1} := \C P^{n-1}.\]
Es gilt
\[ \P^{n-1} = \set{l \subset \C^n \text{ UVR}}{
\dim_\C l = 1
} \]
Das \df{Tautologische Geradenbündel} ergibt sich durch
\[ \C \Pfeil{} \gamma := \set{
(l,v)
}{
v \in l
} \subset \P^{n-1} \times \C^n \Pfeil{} \P^{n-1}. \]

Betrachte
\begin{center}
	\begin{tikzcd}
	\gamma \arrow[r, hook] \arrow[rd, swap, "\sigma := "] & \P^{n-1} \times \C^n \arrow[d, "\text{pr}_2"]\\
	& \C^n
	\end{tikzcd}
\end{center}
Sei $v \in \C^n$.
\begin{itemize}
	\item Ist $v\neq 0$, so liegt $v$ in einem $l = \shrp{v}\in \P^{n-1}$. Es ist
	\[ \sigma\i(v) = \{l\} \]
	ein Punkt.
	\item Ist $v = 0$, so gilt
	\[ \sigma\i(\{0\})  = \P^{n-1}.\]
\end{itemize}
$\sigma : \gamma \pfeil{} \C^n$ nennt man einen \df{Blow-Up} von $\C^n$ entlang $\{0\} \subset \C^n$. Die Einschränkung
\[ \sigma| : \gamma - \text{0-Schnitt} \Pfeil{\isom{}} \C^n - \{0\} \]
ist ein Diffeomorphismus. Es folgt
\[ E_0 \isom{} S^{2n-1}. \]
Ferner wissen wir
\[ H^*(\P^{n-1}) = \R[x] /(x^n = 0), \]
wobei $x \in H^2(\P^{n-1})\isom{} \R$ ein Erzeuger von Rang 2 ist. Setzt man das in die Gysin-Sequenz ein, so erhält man
\[ H^{1}(S^{2n-1}) = 0 \Pfeil{\delta^*} H^0(\P^{n-1}) \Pfeil{\_ \wedge \eu} H^2(\P^{n-1}) \Pfeil{} H^2(S^{2n-1}) = 0\Pfeil{\delta^*}H^1(\P^{n-1}). \]
Dadurch erhalten wir eine Isomorphie
\begin{align*}
H^0(\P^{n-1}) & \Pfeil{} H^2(\P^{n-1})\\
1 \longmapsto \eu.
\end{align*}
Dadurch ist $\eu(\gamma_\R) = c_1(\gamma)$ der Erzeuger von $H^*(\P^{n-1})$.
\marginpar{Vorlesung vom 04.07.18}
%23.te Vorlesung

Sei $\C^n \pfeil{} E \pfeil{} B$ ein komplexes Vektorraumbündel.
\paragraph{Projektivisierung:}
$\P^{n-1} = P\C^n \pfeil{} \P(E) \pfeil{\pi} B$.

Ist $A \in \mathrm{GL}_n(\C)$, so induziert $A$ einen Homöomorphismus
\begin{align*}
A : P\C^{n} & \Pfeil{} P\C^n\\
l & \longmapsto A(l).
\end{align*}
Betrachte den Pullback
\begin{center}
	\begin{tikzcd}
\gamma_{\P E} \arrow[r, hook] \arrow[rd]&	\pi^* E \arrow[r] \arrow[d] & E \arrow[d, "\pi"]\\
	&\P(E) \arrow[r, "\pi"] & B
	\end{tikzcd}
\end{center}
und das Geradenbündel über $\P E$
\[ \gamma_{\P E}E := \set{(l,v) }{v \in l} \subset \pi^* E. \]
$\gamma_{\P E}$ nennt man das \df{tautologische Geradenbündel}. In der algebraischen Geometrie ist $\gamma_{\P E}$ unter der Bezeichnung $\O(-1)$ geläufig.

\Prop{}
$\P E \pfeil{\pi} B$ ist kohomologisch gespalten.
\begin{Beweis}{}
Setze
\[ y:= c_1(\gamma_{\P E}) \in H^2(\P E) .\]
Die Inklusion
\[ j : \P^{n-1} \Inj{} \P E \]
induziert einen Homomorphismus
\[ j^* : H^*(\P E) \Pfeil{} H^*(\P^{n-1}) = \R[\mathrm{Erz.}] / (\mathrm{Erz.}^n = 0). \]
Nun gilt
\[ j^*(y) = j^*c_1(\gamma_{\P E}) = c_1(j^*\gamma_{\P E}) = c_1(\gamma_{\P E}|_{\mathrm{Faser}})
= c_1(\gamma_{\P^{n-1}}) = \mathrm{Erz.} \]
Da $j^*$ ein Ringhomomorphismus ist, gilt nun
\[ j^*(y^k) = \text{Erz.}^k. \]
Ergo muss $j^*$ surjektiv sein. Da es sich bei den Kohomologiegruppen um reelle Vektorräume handelt, können wir eine Spaltung $\beta$ wie folgt konstruieren.
\begin{align*}
\beta : H^*(\P^{n-1}) & \Pfeil{} H^*(\P E)\\
\mathrm{Erz.}^k & \longmapsto y^k.
\end{align*}
\paragraph{Bemerkung:} $\beta$ ist eine lineare Spaltung, aber im Allgemeinem kein Ringhomomorphismus. Zum Beispiel gilt
\[ \mathrm{Erz.}^n = 0, \]
aber
\[ y^n \neq 0 \]
im Allgemeinem.\\



Es gilt nun
\[ j^*\beta (\mathrm{Erz.}^k) = j^*(y^k) = \mathrm{Erz.}^k\]
und somit
\[ j^*\beta = \id{}. \]
\end{Beweis}

Aufgrund dieser Proposition können wir den Satz von Leray-Hirsch auf $\P E$ anwenden und erhalten folgenden Isomorphismus von linearen Vektorräumen
\[ H^*(\P E) \isom{}H^*(B) \otimes (\R[\mathrm{Erz.}]/ (\mathrm{Erz.}^n = 0)).  \]
Es bezeichne $\Phi$ den Isomorphismus $H^*(B) \otimes (\R[\mathrm{Erz.}]/ (\mathrm{Erz.}^n = 0)) \pfeil{} H^*(\P E)$. Es sei $n$ der Rang von $E$. Es gilt nun
\[ \Phi\i(y^n) = a_0\otimes 1  
+ a_1\otimes \mathrm{Erz.}
+ a_2\otimes \mathrm{Erz.}^2
+ a_3\otimes \mathrm{Erz.}^3
+ \ldots
+ a_{n-1}\otimes \mathrm{Erz.}^{n-1} \]
für $a_i \in H^{2(n-i)}(B)$. Wendet man nun $\Phi$ an, so gilt
\[ y^n = \pi^*(a_0) + \pi^*(a_1) \wedge y + \pi^*(a_2) \wedge y^2 + \ldots +\pi^*(a_{n-1}) \wedge y^{n-1}.  \]

\Def{}
Wir definieren die $i$-te \df{Chernklasse} von $E$ durch
\[ c_i(E) := a_{n-i} \in H^{2i}(B). \]
Insbesondere setzen wir
\[ c_0(E) := 1 \in H^0(B) \]
und
\[ c_i(E) := 0 \]
für $i > \text{Rang } E.$

\Bem{Chernklassen des trivialen Bündels}
Betrachte die Bündel
\begin{center}
\begin{tikzcd}
E = B\times \C^n \arrow[d, "\pi = \pi_1"] & \Impl{} & \P E = B\times \P^{n-1} \arrow[d, "\pi_1"] \arrow[r, "\pi_2"] & \P^{n-1}\\
B & & B 
\end{tikzcd}
\end{center}
Es gilt dann
\[ \pi_2^*\gamma_{\P^{n-1}} = \gamma_{\P E} \]
und
\[ y = c_1(\gamma_{\P E}) = c_1(\pi_2^*\gamma_{\P^{n-1}}) = \pi_2^* c_1(\gamma_{\P^{n-1}}) = \pi_2^*\mathrm{Erz.} \]
Somit
\[ y^n = (\pi_2^*\mathrm{Erz.})^n = \pi_2^*(\mathrm{Erz.}^n) = \pi_2^* 0 = 0. \]
Somit sind alle Chernklassen $c_i(E)$ gleich Null im trivialen Fall für alle $i> 0$.
\Bem{Natürlichkeit von $c_i$}
Betrachte das Diagramm
\begin{center}
\begin{tikzcd}
f^*E \arrow[r, "F"] \arrow[d] & E \arrow[d] & \rightsquigarrow & \P(f^*E)  \arrow[r] \arrow[d, "\overline{\pi}"]& \P(E) \arrow[d, "\pi"]\\
 X \arrow[r, "f"] & Y& & X \arrow[r, "f"] & Y
\end{tikzcd}
\end{center}
und
\begin{center}
\begin{tikzcd}
\C  \arrow[r] & \gamma_{\P f^*E} \arrow[d] \arrow[r, "\mathrm{kart.}"] & \gamma_{\P E} \arrow[d] & \arrow[l] \C\\
&\P(f^*E)  \arrow[r, "F"] & \P E  
\end{tikzcd}
\end{center}
D.\,h.
\[ \gamma_{\P f^*E} = F^*\gamma_{\P E}. \]
Setzt man
\[ x:= c_1(\gamma_{\P f^*E}) \in H^2(\P f^* E) .\]
Somit gilt
\[ F^*(y) = F^*c_1(\gamma_{\P E}) = c_1 (F^*\gamma_{\P E}) = c_1(\gamma_{\P f^*E}) = x. \]
Somit gilt
\[ F^*(y) = x. \]

Ferner gilt nun
\[ y^n = \pi^* c_n(E) + \pi^*c_{n-1}(E) y + \ldots \]
und somit
\[ F^*(y^n) = F^*\pi^*c_n(E) + F^*\pi^*c_{n-1}(E)F^*(y) + \ldots
= \overline{\pi}^* f^*c_n(E) + \overline{\pi}^* f^*c_{n-1}(E) F^*(y) + \ldots \]
Substituiert man $x = F^*(y)$, so ist dies gleich zu
\[ x^n = \overline{\pi}^*c_n(f^* E) + 
\overline{\pi}^*c_{n-1}(f^* E) x +
\overline{\pi}^*c_{n-2}(f^* E) x^2 +\ldots 
 \]
Somit gilt
\[ c_i(f^* E) = f^*c_i(E) \in H^{2i}(X) \]
für alle $i$.

\Prop{}
Hat $\pi : E \pfeil{} B$ einen nirgendwo verschwindenden Schnitt, so gilt $c_n(E) = 0$, wobei $n = \text{Rang }E$.
\begin{Beweis}{}
Sei $s : B \pfeil{} E$ ein Schnitt, der nirgendwo verschwindet. $s$ induziert einen nichtverschwindenden Schnitt in der Projektivisierung
\begin{align*}
\widetilde{s} : B & \Pfeil{} \P E\\
b & \longmapsto \text{Gerade erzeugt durch }s(b).
\end{align*}
Nun ist $\widetilde{s}^*(\gamma_{\P E})$ das triviale Geradenbündel, denn es hat einen nirgendwo verschwindenden Schnitt, nämlich $s$. Ferner gilt für $y = c_1(\gamma_{\P E})$
\[ \widetilde{s}^*(y) = c_1(\widetilde{s}^*\gamma_{\P E}) = c_1(\mathrm{triv.}) = 0. \]
Somit folgt
\[ \widetilde{s}^*(y^n) = 0. \]
Da
\[ y^n = \pi^* c_n(E) +
\pi^*c_{n-1}(E) y
+\pi^* c_{n-2}(E) y^2 + \ldots,
 \]
folgt
\[ 0 = \widetilde{s}^*(y^n)
= \widetilde{s}^*\pi^*c_n(E)
+ \widetilde{s}^*\pi^*c_{n-1}(E)\widetilde{s}^*(y) + \ldots
 \]
 Da gilt
 \[ \widetilde{s}^* \pi^* = \id{} \]
 und
 \[ \widetilde{s}^*(y) = 0, \]
 folgt
 \[ 0 = c_n(E) + ( \ldots) \cdot \widetilde{s}^*(y) = c_n(E). \]
\end{Beweis}


\Bem{Weitere Konstruktionen mit Vektorraumbündeln}
\begin{itemize}
	\item \textbf{Tensorprodukt: } Es seien $E$ und $E'$ zwei Vektorraumbündel über demselben Basisraum $B$.
	Wir definieren das Tensorprodukt $E \otimes E'$, indem wir faserweise Tensorprodukte bilden.	
	Um die Übergangsabbildungen zu erklären, führen wir das \df{Kroneckerprodukt} von Matrizen ein:
	
	Sind $A \in \mathrm{GL}_n(\C)$ und $B \in \mathrm{GL}_m(\C)$ gegeben, so definieren wir eine $(nm)\times (nm)$-Matrix durch
	\[A \otimes B := 
	\left(
	\begin{matrix}
	a_{1,1} B & \ldots & a_{1,n} B\\
	\vdots & & \vdots\\
	a_{n,1} B & \ldots & a_{n,n} B
	\end{matrix}
	\right). \]
	Sind nun Übergangsfunktionen $g_{\alpha \beta} : U_{\alpha \beta} \pfeil{} \mathrm{GL}_n(\C)$ für $E$ und
	$g_{\alpha \beta}' : U_{\alpha \beta} \pfeil{} \mathrm{GL}_n(\C)$ für $E'$. Die Übergangsfunktionen für $E\otimes E'$ sind dann die Kroneckerprodukte $g_{\alpha \beta} \otimes g_{\alpha \beta}' : U_{\alpha \beta} \pfeil{} \mathrm{GL}_{nm}(\C).$
	\paragraph{Beispiel:}
	Sei $n = m = 1$. Dann sind $E,E'$ Geradenbündel mit Übergangsfunktionen
	\begin{align*}
	g_{\alpha\beta} : U_{\alpha \beta} & \Pfeil{} \C^\times = \C - 0\\
	g_{\alpha\beta}' : U_{\alpha \beta} & \Pfeil{} \C^\times.\\
	\end{align*}
	Die Übergangsfunktionen von $E\otimes E'$ sind dann $g_{\alpha\beta}\cdot g_{\alpha\beta}'$. Es gilt dann
	
	\[c_1(E \otimes E') = \eu((E \otimes E')_{\R}) = 
	\frac{1}{2\pi i} \sum_\gamma \d (g_\gamma \cdot \d \log (g_{\alpha \beta}\cdot g_{\alpha \beta}')).
	\]
	Da
	\[ \log (g_{\alpha \beta}\cdot g_{\alpha \beta}') = \log (g_{\alpha \beta}) + \log (g_{\alpha \beta}'),\]
	folgt
	\[ c_1(E\otimes E') = c_1(E) + c_1(E'). \]
\end{itemize}

\printindex
\end{document}