%
% -------------------------------------------------------------------------------------------
% "THE BEER-WARE LICENSE"
% Jo, Ihr könnt mit diesem Code machen, was Ihr wollt, solange keinerlei Verantwortlichkeiten auf mich zurückfallen...
% Ansonsten kann ich Euch nur anregen, Wissen frei zu gestalten und allen Menschen zugänglich zu machen.
% Ja, und trinkt ein Bier oder eine Chai Latte oder einen Grünen Tee auf das Skript hier (oder ladet mich ein, wenn Ihr wollt).
% LG Akin
% -------------------------------------------------------------------------------------------
%

\documentclass[12pt]{book}

\usepackage[T1]{fontenc}
\usepackage[utf8]{inputenc}
\usepackage[ngerman]{babel}

\usepackage{tikz-cd}
\usetikzlibrary{babel}

\usepackage{amsfonts}
\usepackage{amssymb}
\usepackage{amsmath}
\usepackage{mathtools}
\usepackage{wasysym}
\usepackage{dsfont}
\usepackage{geometry}
\usepackage{makeidx}
\usepackage{booktabs}
\usepackage{hyperref}

\usepackage{enumerate}
\usepackage{adjustbox}

\newcommand{\ifLeer}[3]{\ifx&#1&\relax#2\relax\else\relax#3\relax\fi\relax}

\newcommand{\Def}[1]{\subsection{Definition\ifLeer{#1}{}{: #1}}}
\newcommand{\Bsp}[1]{\subsection{Beispiel\ifLeer{#1}{}{: #1}}}
\newcommand{\Lem}[1]{\subsection{Lemma\ifLeer{#1}{}{: #1}}}
\newcommand{\Bem}[1]{\subsection{Bemerkung\ifLeer{#1}{}{: #1}}}
\newcommand{\Kor}[1]{\subsection{Korollar\ifLeer{#1}{}{: #1}}}
\newcommand{\Satz}[1]{\subsection{Satz\ifLeer{#1}{}{: #1}}}
\newcommand{\Prop}[1]{\subsection{Proposition\ifLeer{#1}{}{: #1}}}

\newcommand{\QED}{\hfill $\square$}
\newcommand{\qed}{\hfill $\blacksquare$}

\newenvironment{Beweis}[1]{\paragraph{Beweis\ifLeer{#1}{}{: #1}\\}}{\QED}
\newenvironment{Beweisskizze}[1]{\paragraph{Beweisskizze\ifLeer{#1}{}{: #1}\\}}{\qed}

\newcommand{\df}[1]{\index{#1}\textbf{#1}}

\newcommand{\klam}[1]{\left(#1\right)}
\newcommand{\bet}[1]{\left|#1\right|}
\newcommand{\norm}[1]{\bet{\bet{#1}}}
\newcommand{\brak}[1]{\left[#1\right]}
\newcommand{\curv}[1]{\left\lbrace#1\right\rbrace}
\newcommand{\shrp}[1]{\left<#1\right>}
\newcommand{\quot}[1]{\glqq #1 \grqq\relax}
\newcommand{\set}[2]{\curv{\ifLeer{#2}{#1}{#1 ~ | ~ #2}}}
\newcommand{\grp}[2]{\shrp{\ifLeer{#2}{#1}{#1 ~ | ~ #2}}}

\newcommand{\A}{\mathcal{A}}
\newcommand{\B}{\mathcal{B}}
\newcommand{\C}{\mathbb{C}}
\newcommand{\D}{\mathcal{D}}
\newcommand{\E}{\mathcal{E}}
\newcommand{\F}{\mathcal{F}}
\newcommand{\G}{\mathcal{G}}
\renewcommand{\H}{\mathbb{H}}
\newcommand{\I}{\mathcal{I}}
\newcommand{\J}{\mathcal{J}}
\newcommand{\K}{\mathbb{K}}
\renewcommand{\L}{\mathcal{L}}
\newcommand{\M}{\mathcal{M}}
\newcommand{\N}{\mathbb{N}}
\renewcommand{\O}{\mathcal{O}}
\renewcommand{\P}{\mathcal{P}}
\newcommand{\Q}{\mathbb{Q}}
\newcommand{\R}{\mathbb{R}}
\renewcommand{\S}{\mathcal{S}}
\newcommand{\T}{\mathcal{T}}
\newcommand{\U}{\mathcal{U}}
\newcommand{\V}{\mathcal{V}}
\newcommand{\W}{\mathcal{W}}
\newcommand{\X}{\mathcal{X}}
\newcommand{\Y}{\mathcal{Y}}
\newcommand{\Z}{\mathbb{Z}}

\newcommand{\id}[1]{\text{Id}_{#1}}
\newcommand{\Ker}{\textsf{Kern}}
\newcommand{\Coker}{\textsf{Kokern}}
\newcommand{\Img}{\textsf{Bild}}
\newcommand{\Coimg}{\textsf{Kobild}}
\newcommand{\Hom}[3]{\textsf{Hom}_{#1}\left(#2, #3\right)}
\newcommand{\Aut}[2]{\textsf{Aut}_{#1}\left(#2\right)}
\newcommand{\Sym}[1]{\textsf{Symm}_{#1}}

\newcommand{\e}{\varepsilon}

\newcommand{\Pfeil}[1]{\overset{#1}{\longrightarrow}}
\newcommand{\pfeil}[1]{\overset{#1}{\rightarrow}}
\newcommand{\inj}[1]{\overset{#1}{\hookrightarrow}}
\newcommand{\Inj}[1]{\overset{#1}{\lhook\joinrel\longrightarrow}}
\newcommand{\surj}[1]{\overset{#1}{\twoheadrightarrow}}

\newcommand{\impl}[1]{\overset{#1}{\Rightarrow}}
\newcommand{\Impl}[1]{\overset{#1}{\Longrightarrow}}
\newcommand{\gdw}[1]{\overset{#1}{\Leftrightarrow}}
\newcommand{\Gdw}[1]{\overset{#1}{\Longleftrightarrow}}

\newcommand{\off}{\overset{o}{\subset}}
\newcommand{\abg}{\overset{c}{\subset}}

\newcommand{\gl}[1]{\overset{#1}{=}}
\newcommand{\grgl}[1]{\overset{#1}{\geq}}
\newcommand{\klgl}[1]{\overset{#1}{\leq}}
\newcommand{\gr}[1]{\overset{#1}{>}}
\newcommand{\kl}[1]{\overset{#1}{<}}
\newcommand{\isom}[1]{\overset{#1}{\cong}}

\newcommand{\supp}{\text{supp}}

\renewcommand{\i}{^{-1}}
\renewcommand{\phi}{\varphi}
\renewcommand{\d}{\text{d}}

\newcommand{\rot}{\text{rot}}

\renewcommand{\epsilon}{\varepsilon}
\newcommand{\sgn}{\text{sign}}
\newcommand{\Dd}[1]{\frac{\text{D}}{\d #1}}

\setlength{\marginparwidth}{20mm}

\makeindex
\date{\today}
\author{\href{mailto:tensor.produkt@gmx.de}{tensor.produkt@gmx.de}}

\makeindex

\begin{document}
\title{Mitschrieb: Differentialtopologie II\\
SS 18}
\maketitle
\section*{Vorwort}
Dies ist ein Mitschrieb der Vorlesungen vom 16.04.18 bis zum ... des Kurses \textsc{Differentialtopologie II} an der Universität Heidelberg.\\
Dieses Dokument wurde \glqq{live}\grqq\ in der Vorlesung getext. Sämtliche Verantwortung für Fehler übernimmt alleine der Autor dieses Dokumentes.\\
Auf Fehler kann gerne hingewiesen werden bei folgende E-Mail-Adresse
\begin{center}
	\href{mailto:tensor.produkt@gmx.de}{tensor.produkt@gmx.de}
\end{center}
Ferner kann bei dieser E-Mail-Adresse auch der Tex-Code für dieses Dokument erfragt werden.

\setcounter{tocdepth}{1}
\tableofcontents

\marginpar{Vorlesung vom 16.04.18}

\section{Einführung}

\Bsp{Halb-Duplex-Kanal}
Es seien $A,B$ zwei Knoten, die durch einen Halb-Duplex-Kanal verbunden sind, d.\,h., es kann in beide Richtungen gesendet werden, aber nur in eine Richtung zu einem bestimmten Zeitpunkt. Senden $A$ und $B$ gleichzeitig, so erfahren sie weder was noch, ob die andere Seite sendet.\\
Dieses Problem lässt sich durch folgendes Protokoll lösen:
\begin{itemize}
	\item Sendet eine Seite bereits, so schweigt die andere Seite.
	\item Senden beide Seiten gleichzeitig, so warten beide ein jeweils zufälliges Zeitintervall und fangen danach an zu senden; wobei diejenige Seite Priorität erhält, die als erste sendet.
\end{itemize}

\Bsp{Randomisiertes Quicksort}
Bei Quicksort wird in jeder Iteration ein Element der Liste als Pivot-Element gewählt. Betrachte zwei Herangehensweisen:
\begin{itemize}
	\item Im deterministischen Quicksort wird das Pivotelement entsprechend einem deterministischen Verfahren gewählt, z.\,Bsp. das erste Element oder der mittlere.
	\item Im randomisierten Quicksort wird das Pivotelement zufällig gleich verteilt aus allen Elementen der Liste gewählt.
\end{itemize}
Die randomisierte Variante hat für jede Liste eine Worst-Case Laufzeit von $O(n^2)$. Allerdings kann man zeigen, dass man eine erwartete Laufzeit von $O(n\log n)$ erhält.\\
Die deterministische Variante realisiert für einige wenige Listen ihre Worst-Case Laufzeit und performt für viele Listen in $O(n\log n)$.

\Bsp{Das Copy Game}
Zu Beginn jeder Runde wählen zwei Spieler $A,B$ zufällig und geheim einen Wert $x_A,x_B \in \{0,1\}$ und committen sich auf den.\\
Spieler $A$ gewinnt, falls $x_A\neq x_b$. Ansonsten gewinnt $B$.\\\\
Hier ist es essentiell, dass $A$ einer randomisierten Strategie folgt. Denn ist die Strategie von $A$ deterministisch, so kann $B$ diese erfahren oder lernen und sie in Zukunft simulieren (und dadurch immer gewinnen).\\
Folgt aber $A$ einer zufälligen Strategie mit echten Zufall, so gewinnen beide Spieler mit einer Chance von $50\%$. Dies ist unabhängig von den Rechenkraft, die $B$ zur Verfügung steht.

\Def{}
Für $a,b\in \{0,1\}$ definieren wir die \df{Paritätssumme} durch
\begin{align*}
a \oplus b := \left\lbrace
\begin{aligned}
1 && a\neq b\\
0 && a = b
\end{aligned}
\right.
\end{align*}
Die Menge $\{0,1\}$ zusammen mit der Operation $\oplus$ ist eine abelsche Gruppe.

\Bsp{One-Time Pad}
$A$ will $B$ eine Nachricht $w = w_1\ldots w_n \in \{0,1\}^n$ senden, wobei kein Dritter in der Lage sein soll diese Nachricht $w$ durch Abhören des Kommunikationskanals zu lernen.\\
Kennen $A$ und $B$ beide ein gemeinsames Geheimnis
\[r = r_1\ldots r_n \in \{0,1\}^n \]
das zufällig gewählt wurde, so kann $A$ die bitweise Paritätssumme übermitteln. D.\,h., $A$ sendet
\[ w \oplus r := (w_1\oplus r_1)\ldots (w_n\oplus r_n) \in \{0,1\}^n \]
$B$ erhält $w\oplus r$ und kann durch Addieren von $r$ die Nachricht $w = (w\oplus r) \oplus r$ erlernen. Ein Dritter, der $w\oplus r$ abhört, hat keine Chance $w$ zu erraten, weil $r$ zufällig gleichverteilt gewählt wurde.\\
Beachte, die Sicherheit des One-Time Pads ist kompromittiert, wenn derselbe Zufall $r$ für die Verschlüsselung verschiedener Nachrichten $w,v,x, \ldots$ benutzt wird.

\Bsp{Das Dining Cryptographers Problem}
Drei Kryptographen $A,B,C$ essen in einem Restaurant und erfahren am Ende, dass ihre Rechnung bereits bezahlt wurde.\\
Sie wollen ein Protokoll etablieren, durch das sie am Ende wissen, ob einer von ihnen für die Rechnung bezahlt hat:
\begin{itemize}
	\item Jedes Paar von Kryptographen wirft eine faire Münze, dessen Ergebnis gegenüber dem Dritten verborgen wird. Dadurch erhalten wir Zufallbits $r_{A,B}, r_{B,C}$ und $r_{C,A}$.
	\item Setze dann
	\begin{align*}
	u_A := r_{A,B} \oplus r_{C,A} && u_B := r_{A,B} \oplus r_{B,C} && u_C := r_{C,A} \oplus r_{B,C}
	\end{align*}
	\item Danach macht jeder Kryptograph $X$ sein Bit $v_X := u_X$ öffentlich, falls er nicht die Rechnung bezahlt hat. Hat er bezahlt, so veröffentlicht er $v_X:=1 \oplus u_X$.
	\item Es gilt nun
	\[ v_A \oplus v_B \oplus v_C = r_{A,B} \oplus \ldots \oplus r_{C,A}= 0 \]
	falls keiner der Kryptographen die Rechnung bezahlt hat (oder falls zwei der Kryptographen die Rechnung bezahlt haben), und
	\[ v_A \oplus v_B \oplus v_C = 1\oplus r_{A,B} \oplus \ldots \oplus r_{C,A}= 1  \]
	falls genau einer (oder drei) der Kryptographen die Rechnung bezahlt hat.
	\item Keiner der Kryptographen kann erfahren, wer bezahlt von den anderen bezahlt hat, da er eine der Zufallsvariablen $r_{A,B}, r_{B,C}, r_{C,A}$ nicht kennen kann.
\end{itemize}

\renewcommand{\P}{\text{Prob}}
\section{Diskrete Wahrscheinlichkeitsmaße}
Betrachte ein Zufallsexperiment mit Ergebnissen aus einer Menge $\Omega$, die entweder endlich oder abzählbar unendlich ist.\\
Die Wahrscheinlichkeiten werden durch ein \df{diskretes Wahrscheinlichkeitsmaß}
\[ \P : \Omega \Pfeil{} [0,1] \]
so dass gilt
\[ \sum_{\omega \in \Omega} \P(\omega) = 1 \]
Bei $(\Omega, \P)$ handelt es sich um einen \df{diskreten Wahrscheinlichkeitsraum}.\\
Eine Teilmenge von $E\subset \Omega$ heißt \df{Ereignis}. Ereignisse erhalten eine Wahrscheinlichkeit durch
\[ \P(E) := \sum_{\omega \in E} \P(\omega) \]
Auf einer endlichen Menge $\Omega$ ist das gleich verteilte Wahrscheinlichkeitsmaß definiert durch
\[ \P(\omega) := \frac{1}{\bet{\Omega}} \]

\Def{}
Eine \df{Zufallsvariable} ist eine Abbildung
\[ X : \Omega \Pfeil{} \R \]
Jede Zufallsvariable definiert ein Wahrscheinlichkeitsmaß $\P_X$ auf $\R$ durch
\[ \P_X(x) := \sum_{\omega \in X\i(x)} \P(\omega) \]
$\P_X$ heißt die \df{Verteilung} von $X$.\\
Wir legen folgende Notationen fest
\begin{align*}
\P(X = x) &:= \P_X(x)\\
\P(X \geq x) &:= \sum_{t \geq x}\P_X(t)\\
\text{etc.}
\end{align*}

\Def{}
Die \df{Indikator-Variable} für eine Menge $A \subset \Omega$ ist die Zufallsvariable
\begin{align*}
1_A (\omega) := \left\lbrace
\begin{aligned}
1 && \omega \in A\\
0 && \text{ sonst}
\end{aligned}
\right.
\end{align*}

\Def{Multivariate Verteilungen}
Seien $X_1,\ldots, X_m$ Zufallsvariablen auf $\Omega$. Wir definieren die \df{Multivariate Verteilung} $\P_{X_1,\ldots, X_m}$ durch
\begin{align*}
\P_{X_1,\ldots, X_m}(r_1,\ldots, r_m) := \sum_{\omega \in \Omega: X_1(\omega) = r_1, \ldots, X_m(\omega) = r_m}\P(\omega)
\end{align*}
Wir schreiben in Zukunft:
\[ \P(X_1 = r_1, \ldots, X_m = r_m) := \P_{X_1,\ldots, X_m}(r_1,\ldots, r_m) \]

\Def{}
$X_1,\ldots, X_m$ heißen \df{gemeinsam unabhängig}, falls für alle $r_1,\ldots, r_m$ gilt
\[ \P(X_1 = r_1, \ldots, X_m = r_m) = \P(X_1 = r_1) \cdots \P(X_m = r_m) \]
$X_1,\ldots, X_m$ heißen \df{paarweise (stochastisch) unabhängig}, falls für alle $r_1,\ldots, r_m$ und $i\neq j$ gilt
\[ \P(X_i = r_i,  X_j = r_j) = \P(X_i = r_i) \cdot \P(X_j = r_j) \]
Sie heißen \df{$k$-weise unabhängig}, falls jedes $k$-Tupel von Zufallsvariablen gemeinsam unabhängig ist.

\Bem{}
Gemeinsame oder $k+1$-weise Unabhängigkeit impliziert immer $k$-weise Unabhängigkeit. Allerdings ist die Umkehrung im Allgemeinem falsch.

\Bsp{}
$X_1,X_2,X_3\in \{0,1\}$ seien zufällig gezogen. Setze
\begin{align*}
Z_1 := X_2 \oplus X_3 && Z_2 := X_1 \oplus X_3 && Z_3 := X_1 \oplus X_2
\end{align*}
Dann sind die $Z_i$ paarweise unabhängig, aber nicht gemeinsam unabhängig.

\renewcommand{\E}{\mathbb{E}}

\Def{}
Für eine Zufallsvariable $X$ ist der Erwartungswert definiert durch
\[ \E(X) := \sum_{\omega \in \Omega} X(\omega) \cdot \P(\omega) \]
Wir sagen, dass $\E(X)$ existiert, falls obige Reihe absolut konvergiert. D.\,h.
\[ \sum_{\omega \in \{\omega_1,\ldots, \omega_n\}} \bet{X(\omega) \cdot \P(\omega)}  \]
konvergiert für $n \pfeil{} \bet{\Omega}$.

\Bem{}
\begin{itemize}
	\item $\E$ ist $\R$-linear.
	\item Sind $X_1,\ldots, X_m$ gemeinsam unabhängig, so gilt
	\[ \E(X_1 \cdots X_m) = \E(X_1) \cdots \E(X_m) \]
\end{itemize}

\Def{}
Die \df{bedingte Wahrscheinlichkeit} eines Ereignisses $A$ unter $B$ ist definiert durch
\[ \P(A|B) := \frac{\P(A\cap B)}{\P(B)} \]
für $\P(B) > 0$.\\
Die \df{bedingte Verteilung} einer Zufallsvariable $X$ unter $B$ ist definiert durch
\[ \P(X = x|B) := \frac{\sum_{\omega \in \Omega \cap B} \P(\omega)}{\P(B)} \]
Definiere den \df{bedingten Erwartungswert} durch
\[ \E(X | B) := \sum_{r \in \R} r\cdot \P(X = r | B) \]
\marginpar{Vorlesung vom 18.04.18}
\Def{}
Definiere die \df{Torsion} des Zusammenhangs durch
\[ T(\xi, \eta) := \nabla_\xi \eta - \nabla_\eta \xi - [\xi, \eta] \]
wobei $[\xi, \eta]$ die \df{Lie-Klammer}\footnote{Lassen sich die beiden Vektorfelder als Koordinatenrichtungen schreiben, so gilt zum Beispiel $[\frac{\partial}{\partial x_i}, \frac{\partial}{\partial x_j}] = 0$} der beiden Vektorfelder $\xi$ und $\eta$ bezeichnet.\\
$T$ ist ein \df{Tensor}, d.\,h., $\mathcal{C}^{\infty}(M)$-linear.\\
$\nabla$ heißt \df{symmetrisch} bzw. \df{torsionsfrei}, falls $T = 0$.

\newcommand{\crv}{\text{R}}

\Lem{}
Ist $\nabla$ symmetrisch, dann gilt sogar
\[ d(\overline{\mu}(\e), \overline{\lambda}(\e)) \in O(\e^3) \]

Sei $u \in T_p(M)$ ein weiterer Tangentialvektor. $u_1$ sei der Paralleltransport von $u$ entlang $\lambda\overline{\mu}$. $u_2$ sei der Paralleltransport entlang $\mu \overline{\lambda}$.\\
Es liegt dann folgende asymptotische Gleichheit vor
\[ \norm{u_1 - u_2} \sim \epsilon^2 \crv(v,w)u  \]
$\crv(v,w)u$ heißt \df{Riemannscher Krümmungstensor}. Er ist definiert durch
\[ \crv(v,w)u := \nabla_v \nabla_wu - \nabla_w \nabla_v u - \nabla_{[v,w]} u \]

Wir werden nun im Folgenden mit den Formalen Definitionen beginnen.

\newpage
\section{Die Lie-Klammer}
Sei $M$ im Folgenden eine glatte $n$-dimensionale Mannigfaltigkeit und $X,Y : M \pfeil{} \T M$ glatte Vektorfelder auf $M$.

\newcommand{\CC}[1]{\mathcal{C}^{#1}}

\Lem{}
Es existiert genau ein glattes Vektorfeld $Z$ auf $M$, sodass gilt
\[Z(f) = X(Y(f)) - Y(X(f)) \]
für alle $f \in \CC{\infty}(M)$. Beachte, $X(f)$ bezeichnet die glatte Funktion, die sich ergibt durch
\[ X(f)(p) := X(p)(f) \]

\begin{Beweis}{}
\begin{itemize}
	\item Eindeutigkeit:\\
	Sei $p \in M$. $\{x_i\}$ seien lokale Koordinaten bei $p$. $X, Y$ lassen sich dann schreiben durch
	\begin{align*}
	X = \sum_i a_i \frac{\partial}{\partial x_i} && \text{ und } && Y = \sum_j b_j \frac{\partial}{\partial x_j}
	\end{align*}
	und es gilt
	\begin{align*}
	X(Yf) &= X\klam{\sum_j b_j \frac{\partial f}{\partial x_j}}\\
	&= \sum_i a_i \sum_j \frac{\partial }{\partial x_i} \klam{b_j \frac{\partial f}{\partial x_j}}\\
	&= \sum_{i,j} a_i \frac{\partial b_j}{\partial x_i} \frac{\partial f}{\partial x_j} + 
	\sum_{i,j} a_i b_j \frac{\partial^2 f}{\partial x_j\partial x_i} 
	\end{align*}
	bzw.
	\begin{align*}
		Y(Xf) =  \sum_{i,j} b_j \frac{\partial a_i}{\partial x_j} \frac{\partial f}{\partial x_i} + 
	\sum_{i,j} b_j a_i \frac{\partial^2 f}{\partial x_i\partial x_j} \\
	\end{align*}
	In der Differenz ergibt sich
	\begin{align*}
	X(Yf) - Y(Xf) &= \sum_{i,j} a_i \frac{\partial b_j}{\partial x_i} \frac{\partial f}{\partial x_j}
	- \sum_{i,j} b_i \frac{\partial a_j}{\partial x_i} \frac{\partial f}{\partial x_j}\\
	&= \sum_{i,j} \klam{a_i \frac{\partial b_j}{\partial x_i} - b_i \frac{\partial a_j}{\partial x_i} } \frac{\partial f}{\partial x_i}
	\end{align*}
	Lokal ist $Z$ also bestimmt durch
	\[ Z =  \sum_{i,j} \klam{a_i \frac{\partial b_j}{\partial x_i} - b_j \frac{\partial a_j}{\partial x_i} } \frac{\partial }{\partial x_j} \]
	\item Existenz:\\
	Durch obige Formel ist für jedes lokale Koordinatensystem ein $Z$ gegeben. Diese lassen sich global zu einem glatten Vektorfeld auf ganz $M$ zusammen setzen.
\end{itemize}
\end{Beweis}

\Def{}
Definiere nun die \df{Lie-Klammer} von $X$ und $Y$ durch
\[ Z:= [X,Y] = XY - YX\]

\Bem{}
Die Lie-Klammer hat folgende Eigenschaften
\begin{itemize}
	\item $[X,Y] = - [Y,X]$
	\item Für $a,b \in \R$ gilt
	\[ [aX_1 + bX_2, Y] = a[X_1, Y] + b[X_2, Y] \]
	\item Iteration: Für beliebige Vektorfelder $X,Y,Z$ gilt
	\[ [[X,Y], Z] = [ XY - YX, Z ] = XYZ- YXZ - ZXY + ZYX \]
	und
	\[ [[Y,Z], X] = [ YZ - ZY, X ] = YZX- ZYX - XYZ + XZY \]
	und
	\[ [[Z,X], Y] = [ ZX - XZ, Y ] = ZXY - XZY - YZX + YXZ \]
	Durch Aufsummieren ergibt sich
	\[ [[X,Y], Z] + [[Y,Z], X]  + [[Z,X], Y] = 0\]
	Dies nennt sich die \df{Jacobi-Identität}.
	\item Seien $f,g \in \CC{\infty}(M)$. Es gilt
	\[ [fX, gY] = fX(gY) - gY(fX) = f(X(g)Y - gXY) - g(Y(f)X - fYX) = fg[X,Y] + fX(g) Y - gY(f)X \]
\end{itemize}

\newcommand{\pf}[2]{\frac{\partial #1}{\partial #2}}
%\vspace{12mm}
Da eine Mannigfaltigkeit lokal wie $\R^n$ aussieht, lassen sich die bekannten Sätze zu Existenz, Eindeutigkeit und Abhängigkeit von Anfangsbedingungen von gewöhnlichen Differentialgleichungen von $\R^n$ auf $M$ verallgemeinern.

\Satz{}
Sei $M$ eine glatte Mannigfaltigkeit, $X$ ein glattes Vektorfeld auf $M$, $p \in M$ ein Punkt.\\
Dann existiert eine offene Umgebung $U \subset M$ von $p$ und ein $\delta > 0$ zusammen mit einer Abbildung
\[ \phi : (-\delta, \delta) \times U \Pfeil{} M \]
sodass $t \mapsto \phi(t,p)$ die eindeutige Lösung von
\begin{align*}
\pf{}{t} \phi(t,q) &= X(\phi(t,q)) && \forall q \in U\\
\phi(0,q) &= q
\end{align*}
ist.\\
Schreibweise:
\[ \phi_t(p) := \phi(t,p) \]
Die glatte Abbildung
\[ \phi_t : U\pfeil{} M \]
heißt \df{Fluss} von $X$ (in der Umgebung von $p$).

\Bem{}
Sei $\bet{s}, \bet{t}, \bet{s+t} < \delta$. Betrachte
\[ \gamma_1(t) := \phi(t, \phi(s,p)) \]
Das impliziert
\begin{align*}
\dot{\gamma_1} = X(\gamma_1) && \gamma_1(0) = \phi(s,p)
\end{align*}
und
\[ \gamma_1(t) := \phi(t+s, p) \]
impliziert
\begin{align*}
\dot{\gamma_2} = X(\gamma_2) && \gamma_2(0) = \phi(s,p)
\end{align*}
Aus der Eindeutigkeit folgt nun
\[ \gamma_1 = \gamma_2 \]
D.\,h.,
\[ \phi_{s+t} = \phi_s \circ \phi_t \]
Insbesondere gilt
\[ \id{M} = \phi_t\circ \phi_{-t} \]
Daraus folgt, dass jedes $\phi_t$ ein Diffeomorphismus ist. Die Menge aller $\{\phi_t\}_{t}$ nennt man eine \df{Einparameter-Untergruppe} von Diffeomorphismen.

\newpage
\section{Die Lie-Ableitung}
Seien $X,Y$ zwei Vektorfelder auf $M$, $p \in M$ ein Punkt.\\
Sei $\phi_t$ der Fluss auf $X$ mit
\begin{align*}
\pf{}{t} \phi(t,p) = X(\phi_t(p)) && \text{ und } && \phi_0(p) = p
\end{align*}
Definiere nun die \df{Lie-Ableitung durch}
\begin{align*}
(L_XY)(p):=\lim\limits_{h\pfeil{} 0} \frac{1}{h} \klam{
Y_p 
- (\d \phi_h)(Y_{\phi_{-h}(p)})
}\in T_p(M)
\end{align*}
wobei $Y_p = Y(p), \d \phi_h = \phi_{h,*}$. Die Lie-Ableitung leitet das Vektorfeld $Y$ bzgl. dem Fluss von $X$ im Punkt $p$ ab.

\Prop{}
\label{LieKlammerProp}
Es gilt
\[ L_XY = [X,Y] \]
Für den Beweis dieser Proposition benötigen wir ein Lemma:
\paragraph{Idee}
Sei $f: \R \pfeil{} \R$ glatt mit $f(0) = 0$. $f$ hat die Taylor-Entwicklung
\[ f(t) = t f'(0) + \frac{t^2}{2} f''(0) + \ldots =: t \cdot g(t) \]
Es gilt
\[ f(t) = tg(t) \]
und $f'(0) = g(0)$.\\
Wir brauchen nun folgende Verallgemeinerung dieser Beobachtung:

\Lem{}
Sei $M$ eine Mannigfaltigkeit, $f : (-\e, \e) \times M \pfeil{} \R$ glatt, $f(0,p) = 0$ für alle $p \in M$.\\
Dann existiert eine glatte Funktion $g: (-\e, \e) \times M \pfeil{} \R$ mit
\begin{align*}
f(t,p) = t\cdot g(t,p) && \text{ und } && \pf{f}{t}(0,p) = g(0,p)
\end{align*}
\begin{Beweis}{}
Wir definieren $g$ durch
\begin{align*}
g(t,p) := \int_{0}^{1} \klam{\pf{f}{s}} (s\cdot t,p) \d s
\end{align*}
Der Rest ist nachrechnen.
\end{Beweis}

\begin{Beweis}{\ref{LieKlammerProp}}
Sei $f \in \CC{\infty}(M)$. Definiere die Hilfsfunktion
\begin{align*}
h(t,p) := f(\phi_t(p)) - f(p)
\end{align*}
Aufgrund des Lemmas existiert ein $g$ mit
\begin{align*}
h(t,p) = t \cdot g(t,p) && \pf{h}{t}(0,p) = g(0,p)
\end{align*}
Es gilt
\begin{align*}
f \circ \phi_t = f + t g_t
\end{align*}
\end{Beweis}
\marginpar{Vorlesung vom 23.04.18}
\begin{Beweis}{\ref{LieKlammerProp}}
	Sei $f \in \CC{\infty}(M)$. Wir wollen Folgendes zeigen.
	\[ (L_XY)(f) = [X,Y](f) = XYf - YXf \]
	Definiere die Hilfsfunktion
	\begin{align*}
	h(t,p) := f(\phi_t(p)) - f(p).
	\end{align*}
	Da $h(0,p) = 0$, existiert aufgrund des Lemmas ein $g$ mit
	\begin{align*}
	h(t,p) = t \cdot g(t,p) &&\text{ und }&& \pf{h}{t}(0,p) = g(0,p).
	\end{align*}
	Es gilt
	\begin{align*}
	f \circ \phi_t = f + t g_t
	\end{align*}
	und
	\begin{align*}
	X_p(f) = \klam{\pf{}{t}_{t= 0}\phi_t(p)}(f) = \pf{}{t}_{t = 0} f(\phi_t(p)) = \pf{h}{t} (0,p) = g(0,p).
	\end{align*}
	Durch die erste der beiden obigen Gleichung erhalten wir
	\begin{align*}
	(\d \phi_h)(Y_{\phi_{-h}(p)})(f) &= Y_{\phi_{-h}(p)}(f\circ \phi_h)\\
	&= Y_{\phi_{-h}(p)} (f+tg_t).
	\end{align*}
	Setzt man dies in die Lie-Ableitung ein, so erhält man
	\begin{align*}
	(L_XY)(f) &= \lim\limits_{h\pfeil{} 0}\frac{1}{h}\klam{ Y_p - (\d \phi_h)(Y_{\phi_{-h}(p)})(f) }\\
	&= \lim\limits_{h\pfeil{} 0}\frac{1}{h}\klam{ Y_p - (Y_{\phi_{-h}(p)})(f) } - \lim\limits_{h\pfeil{} 0}\frac{1}{h}\klam{h (Y_{\phi_{-h}(p)})(g_h) }.
	\end{align*}
	Da gilt
	\[ \lim\limits_{h\pfeil{} 0}\frac{1}{h}\klam{h (Y_{\phi_{-h}(p)})(g_h) } =
	\lim\limits_{h\pfeil{} 0} (Y_{\phi_{-h}(p)})(g_h)  = Y_p(g_0) =YXf,  \]
	folgt
		\begin{align*}
	(L_XY)(f)&= \lim\limits_{h\pfeil{} 0}\frac{1}{h}\klam{ (Yf)_p - (Yf)_{\phi_{-h}(p)} } - Y_pXf\\
	&= X_pYf - Y_pXf.
	\end{align*}
\end{Beweis}
\paragraph{Folgerungen}
\begin{align*}
L_YX = -L_XY, && L_XX = 0
\end{align*}\\\\
Seien Vektorfelder $X,Y$ gegeben. Man kann zeigen, dass lokale Koordinaten $x_1,\ldots, x_n$ existieren mit
\[ X = \pf{}{x_1}. \]
Gilt ferner
\[ Y = \pf{}{x_2}, \]
so folgt
\[ [X,Y] = \pf{\partial}{x_1\partial x_2} - \pf{\partial}{x_2\partial x_1} = 0.  \]
Insofern ist das Verschwinden von $[X,Y]$ eine notwendige Bedingung für die Existenz von lokalen Koordinaten $x_1,\ldots, x_n$ mit
\begin{align*}
X = \pf{}{x_1} && \text{ und } && Y = \pf{}{x_2}.
\end{align*}

\subsection{Geometrische Interpretation der Lie-Klammer}
Seien $X,Y$ Vektorfelder. $\phi$ und $\psi$ seien korrespondierende Flüsse, $p\in M$ sei ein Punkt. Setze
\[ c(h) := \psi_{-h}\phi_{-h}\psi_h\phi_h(p). \]
Die Zuordnung $h \mapsto c(h)$ definiert eine glatte Kurve. Man kann zeigen
\[ \dot{c}(h) = 0. \]
Für Kurven $\gamma(t)$ mit $\dot{\gamma}(0) = 0$ lässt sich die zweite Ableitung definieren durch
\[ \ddot{\gamma}(t)(0):= \frac{\d^2}{\d t^2}_{t = 0} f(\gamma(t)) .\]
Dann ist $\ddot{\gamma}(0)$ eine Derivation.\\
Daraus folgt, dass $\ddot{c}(0)$ definiert ist, und es gilt
\begin{align*}
\ddot{c}(0) = 2[X,Y]_p.
\end{align*}

\newpage
\section{Riemannsche Mannigfaltigkeiten}
Sei $M$ eine glatte, $n$-dimensionale Mannigfaltigkeit.
\Def{}
Eine \df{Riemannsche Metrik} auf $M$ ist eine Zuordnung
\begin{align*}
p \longmapsto \shrp{\cdot ~|~\cdot}_p
\end{align*}
für $p\in M$, wobei $\shrp{\cdot ~|~\cdot}_p$ jeweils ein inneres Produkt\footnote{Inneres Produkt heißt hier eine symmetrische, positiv definite Bilinearform.} auf $T_pM$ ist. Ferner soll diese Zuordnung\df{glatt} sein in dem Sinne, dass für lokale Koordinaten $(U,x)$ die Funktionen
\begin{align*}
g_{i,j}(p) := \shrp{\pf{}{x_i}(p) ~|~ \pf{}{x_j}(p) }_p
\end{align*} 
für alle $i,j$ glatt sind auf $U$.\\
Wir werden manchmal $g(p)$ anstatt $\shrp{\cdot ~|~ \cdot}_p$ schreiben.\\
Das Paar $(M, \shrp{\cdot~|~ \cdot})$ heißt \df{Riemannsche Mannigfaltigkeit}.

\Def{}
Ein Diffeomorphismus $\phi: (M, \shrp{\cdot~|~\cdot}_M) \pfeil{} (N, \shrp{\cdot~|~\cdot}_N)$ heißt \df{Isometrie}, falls für alle $p \in M$ und $u,v \in T_pM$ gilt
\[ \shrp{u,v}_{M,p} = \shrp{ \d \phi_pu, \d \phi_pv}_{N,\phi(p)}. \]


\begin{enumerate}[(1)]
	\item \Bsp{} Sei $M = \R^n$. $x$ seien die Standardkoordinaten auf $\R^n$. Setzt man
	\[ \shrp{\pf{}{x_i}, \pf{}{x_j}}_p = \delta_{i,j} \]
	so erhält man die euklidische Metrik auf $\R^n$.
	\item Sei $f:M\pfeil{} N$ eine glatte Immersion. $(N,\shrp{\cdot~|~\cdot}_N)$ sei eine Riemannsche Mannigfaltigkeit, $M$ eine glatte Mannigfaltigkeit. Dann induziert $f$ eine Riemannsche Metrik $\shrp{\cdot ~|~\cdot}_M$ auf $M$ durch
	\[ \shrp{u|v}_M:= \shrp{\d f(u), \d f(v)}_N. \]
	Da $\d f$ injektiv ist, ist $ \shrp{u|v}_{M,p}$ positiv definit.
	\item \Bsp{} Es bezeichne $S^n = \set{(x_1,\ldots,x_{n+1}) \in \R^{n+1}}{x_1^2 + \ldots + x_{n+1}^2 = 1}$ die Einheitssphäre. Durch die Einebettung $S^n \subset \R^{n+1}$ erhalten wir eine Riemannsche Metrik auf $S^n$. $S^n$ zusammen mit dieser Metrik nennt man \df{Standardsphäre}.
	\item \textbf{Produktmetrik}: Seien $(M, g_m), (N,g_N)$ zwei Riemannsche Mannigfaltigkeiten. $\pi_1, \pi_2$ seien die korrespondierenden Projektionen von $M\times N$ auf $M$ bzw. $N$. Seien $u,v \in T_{(p,q)}(M\times N)$, setze
	\begin{align*}
	\shrp{u,v}_{p,q} := \shrp{\d \pi_1(u), \d \pi_1(v)}_{M,p} + \shrp{\d \pi_2(u), \d \pi_2(v)}_{N,q}. 
	\end{align*}
	$\shrp{u,v}_{p,q}$ ist eine Riemannsche Metrik auf $M \times N$, die sogenannte \df{Produktmetrik}.
	\item \Bsp{} Betrachte $T^n := S^1 \times \ldots \times S^1$. Ist $S^1$ mit der Standardmetrik versehen, so induziert uns dies eine Produktmetrik auf $T^n$. In diesem Fall spricht man vom \df{flachen Torus}.\\
	Für $n=2$ kann man $T^2$ in den $\R^3$ einbetten. Dadurch erhält man eine andere induzierte Metrik auf $T^2$, die nicht äquivalent zu obiger Produktmetrik ist. Diese beiden Tori sind nicht isometrisch.
\end{enumerate}

\Prop{}
Jede glatte Mannigfaltigkeit besitzt eine Riemannsche Metrik.

\begin{Beweis}{}
Sei $\set{(U_\alpha,x_\alpha)}{}$ eine offene Überdeckung von $M$ durch Karten und $\{f_\alpha\}$ eine glatte Partition der Eins bzgl. dieser Überdeckung.\\
Über $U_\alpha$ betrachte man die eindeutige Riemannsche Metrik $g^\alpha$, sodass
\[ (U_\alpha,g^\alpha) \Pfeil{x_\alpha} (\R^n, g_{eukl}) \]
eine Isometrie ist. Auf $M$ erhält man nun eine Riemannsche Metrik durch
\begin{align*}
g_p := \sum_{p\in U_\alpha} f_\alpha(p) g_p^\alpha.
\end{align*}
\end{Beweis}

\Def{}
Sei $c : \R \pfeil{} M$ eine glatte Kurve. Ein \df{Vektorfeld entlang einer Kurve} $c$ ist eine glatte Zuordnung
\[ t \longmapsto V(t) \in T_{c(t)}M \]

\Bem{}
Ein Vektorfeld entlang einer Kurve lässt sich im Allgemeinem nicht auf ein Vektorfeld einer offenen Umgebung der Kurve fortsetzen. Zum Beispiel könnte sich die Kurve selbst schneiden und $V$ die Ableitung der Kurve sein.

\paragraph{Notation}
Wir schreiben auch für $v \in T_pM$
\[ \norm{v} := \sqrt{ \shrp{v|v}_p } \]

\Def{}
Für eine Kurve $c$ definiere wir die \df{Länge} durch
\[ L^b_a(c) := \int_{a}^{b} \norm{\dot{c}(t)} \d t \]
\marginpar{Vorlesung vom 25.04.18}

Sei $(M,g)$ eine Riemannsche Mannigfaltigkeit, die obendrein orientiert ist. $(U,x)$ und $(V,y)$ seien orientierte Karten auf $M$, die sich schneiden.\\
Wir erinnern an folgendes Lemma aus Differentialtopologie I.
\Lem{}
Auf $U\cap V$ gilt
\[ f\d x_1 \wedge \ldots \wedge \d x_n = g \d y_1 \wedge \ldots \wedge \d y_n \]
genau dann, wenn
\[ f = \det\klam{ \pf{y_i}{x_j} } g \]
gilt.\\\\
Auf einer orientierten Karte $U$ sind Funktionen $g_{i,j} : U \pfeil{} \R$ gegeben durch
\[ g_{i,j} = \shrp{\pf{}{x_i}|\pf{}{x_j} } \]
für $p \in U$. Setze ferner $X_i := \pf{}{x_i}$.\\
Sei $e_1,\ldots, e_n$ eine {Orthonormalbasis} (ONB) für $T_pM$ bzgl. $g_p$. Dann lässt sich $X_i$ darstellen durch
\[ X_i = \sum_j a_{i,j}e_j. \]
Wir erhalten so eine $n\times n$-Matrix $A := (a_{i,j})_{i,j}$. Es gilt
\begin{align*}
g_{i,j} :=& \shrp{X_i|X_j}\\
=& \shrp{ \sum_k a_{i,k}e_k| \sum_l a_{j,l}e_l  }\\
=& \sum_{k,l} a_{i,k}a_{j,l} \shrp{e_k|e_l}\\
=& \sum_k a_{i,k}a_{j,k}.
\end{align*}
Daraus folgt
\[ (g_{i,j})_{i,j} = AA^T. \]
Dies impliziert insbesondere
\[ \det(g_{i,j}) = \det(A)^2 > 0. \]
Insbesondere ist $\sqrt{\det(g_{i,j})} = \bet{\det A}$ wohldefiniert. Durch den Transformationssatz folgt nun im Punkt $p$
\begin{align*}
vol(X_1, \ldots, X_n) = \bet{\det A} \cdot vol(e_1, \ldots, e_n) = \bet{\det A},
\end{align*}
da $vol(e_1, \ldots, e_n) = 1$. Daraus folgt insbesondere
\[ vol(\pf{}{x_1}, \ldots, \pf{}{x_n}) = \sqrt{\det (g_{i,j})}. \]
Auf $(V,y)$ erhält man analog
\[ vol(Y_1, \ldots, Y_n) = \sqrt{\det (h_{i,j})}. \]
für
\[ Y_i = \pf{}{y_i} \]
und
\[ h_{i,j} = \shrp{Y_i|Y_j}. \]
Man erhält hierdurch
\begin{align*}
\sqrt{\det (h_{i,j})} &= vol(Y_1, \ldots, Y_n)\\
&= \det \klam{ \pf{x_i}{y_j} } vol(X_1, \ldots, X_n)\\
&=  \det \klam{ \pf{x_i}{y_j} } \sqrt{\det (g_{i,j})}.
\end{align*}
Mit dem obigen Lemma folgt nun auf $U\cap V$
\[ \sqrt{\det (g_{i,j})} \d x_1\wedge \ldots \wedge \d x_n = \sqrt{\det (h_{i,j})} \d y_1\wedge \ldots \wedge \d y_n.  \]
Durch Verkleben erhalten wir eine glatte $n$-Form $\nu \in \Omega^n(M)$.

\Def{}
$\nu$ heißt \df{Riemannsche Volumenform} von $M$. $\nu$ ist durch die Riemannsche Metrik eindeutig festgelegt.

\Def{}
Wenn $M$ kompakt ist, setzen wir
\[ vol(M) := \int_M \nu < \infty. \]
$vol(M)$ heißt das \df{Riemannsche Volumen}.\\
Wenn $vol(K)$ unbeschränkt ist über kompakte Untermannigfaltigkeiten (mit Rand) $K \subset M$, dann sagen wir, dass $M$ unendliches Volumen habe.

\Bem{}
Oft sieht man in der Literatur $\nu = \d V = \d vol$, obwohl $\nu$ im Allgemeinem nicht im Bild des Randhomomorphismus
\[ \d : \Omega^{n-1}(M) \Pfeil{} \Omega^n(M) \]
liegt.

\newpage
\section{Zusammenhänge}
Sei $\Gamma(\T M)$ der Vektorraum der glatten Schnitte von $\T M$, d.\,h., $\Gamma(\T M)$ ist der Vektorraum der glatten Tangentialvektorfelder auf $M$.

\Def{}
Ein \df{Zusammenhang} auf $M$ ist eine Abbildung
\begin{align*}
\nabla : \Gamma(\T M) \times \Gamma(\T M ) &\Pfeil{} \Gamma(\T M)\\
(X,Y) & \longmapsto \nabla_XY ,
\end{align*}
sodass:
\begin{enumerate}[(1)]
	\item Für $f,g \in \CC{\infty}(M)$ gilt
	\[\nabla_{fX_1 + gX_2}Y = f\nabla_{X_1}Y + g \nabla_{X_2}Y.\]
	\item Ferner gilt
	\[ \nabla_X(Y_1 + Y_2) = \nabla_X Y_1 + \nabla_X Y_2. \]
	\item Zuletzt wird folgende Produktregel gefordert
	\[ \nabla_X(f\cdot Y) = f\nabla_XY + X(f)Y. \]
\end{enumerate}

\paragraph{In Lokalen Koordinaten} Sei $(U,x)$ eine Karte. $X,Y$ seien Vektorfelder der Gestalt
\begin{align*}
X = \sum_ia_i \pf{}{x_i}, && Y = \sum_j b_j\pf{}{y_j}.
\end{align*}
Es gilt
\begin{align*}
\nabla_XY &= \sum_ia_i \nabla_{\pf{}{x_i}} (\sum_j \pf{}{y_j})\\
&= \sum_ia_i\sum_j \nabla_{\pf{}{x_i}} (b_j \pf{}{y_j})\\
&= \sum_{i,j} a_i 
\klam{
\pf{b_j}{x_i} \pf{}{x_j} + b_j \nabla_{\pf{}{x_i}} \pf{}{y_j} 
}.
\end{align*}
Wir dröseln die Terme $\nabla_{\pf{}{x_i}} \pf{}{y_j}$ weiter auf und erhalten
\begin{align*}
\nabla_{\pf{}{x_i}} \pf{}{y_j}  = \sum_k \Gamma_{i,j}^k \pf{}{x_k}.
\end{align*}
Die Funktionen $\Gamma_{i,j}^k$ nennt man \df{Christoffel-Symbole} des Zusammenhangs.\\
Wir erhalten final
\begin{align*}
\nabla_XY &= \sum_{i,k} a_i 
\klam{
\pf{b_k}{x_i} \pf{}{x_k} +\sum_{i,j} a_i b_j \sum_k \Gamma_{i,j}^k \pf{}{x_k} 
}\\
&= \sum_k \klam{
\sum_i  a_i \pf{b_k}{x_i}
+
\sum_{i,j} a_i b_j \Gamma_{i,j}^k
}
\pf{}{x_k}.
\end{align*}

\Bem{}
Die Gleichung
\begin{align*}
\nabla_XY &= \sum_k \klam{
	\sum_i  a_i \pf{b_k}{x_i}
	+
	\sum_{i,j} a_i b_j \Gamma_{i,j}^k
}
\pf{}{x_k}
\end{align*}
impliziert, dass $\nabla_X Y$ eine lokale Operation ist. Denn für $p \in M$ gilt
\begin{align*}
(\nabla_XY)(p) &= \sum_k \klam{
\sum_i a_i(p) \pf{b_k}{x_i}(p)
+ \sum_{i,j} a_i(p) b_j(p) \Gamma_{i,j}^k(p)
}\pf{}{x_k}_{|p}.
\end{align*}
D.\,h., $(\nabla_XY)(p)$ hängt nur von $X(p), Y(p)$ und $\pf{b_k}{x_i}(p)$ ab.

\Def{}
Sei $V = V(t)$ ein Vektorfeld entlang einer Kurve $c(t)$ in $M$.\\
Eine \df{kovariante Ableitung} ist eine Zuordnung
\[ \Dd{t} : \V_c \Pfeil{} \V_c, \]
wobei $\V_c$ den Raum aller Vektorfelder entlang $c$ bezeichnet, sodass
\begin{enumerate}[(1)]
	\item $\Dd{t}(V+W) = \Dd{t}V + \Dd{t} W$
	\item und $\Dd{t}(fV) = f \Dd{t} V + \pf{f}{t} V$ für $f : \R \pfeil{}\R$ glatt gelten.
	\item Wenn ferner ein Vektorfeld $X$ auf $M$ existiert mit $X(c(t)) = V(t)$, dann soll gelten
	\[ \nabla_{\dot{c}}X = \Dd{t}(V) .\]
\end{enumerate}

\Prop{}
Sei $M$ eine glatte Mannigfaltigkeit mit Zusammenhang $\nabla$. Sei $c$ eine Kurve auf $M$. Dann existiert eindeutig eine kovariante Ableitung $\Dd{t}$ mit obigen Eigenschaften.
\begin{Beweis}{}
\begin{itemize}
	\item Eindeutigkeit: Sei $V(t)$ ein Vektorfeld entlang $c(t)$. In lokalen Koordinaten $x_1, \ldots, x_n$:
	\begin{align*}
	V(t) = \sum_i v_i(t) \pf{}{x_i}, && c(t) = (x_1(t), \ldots, x_n(t))
	\end{align*}
	Es gilt dann
	\begin{align*}
	\Dd{t} V &= \sum_i \klam{
v_i \Dd{t} \klam{
\pf{}{x_i}_{|c(t)}
}	+ v_i' \pf{}{x_i}
}\\
&= \sum_i \klam{
v_i \nabla_{\dot{c}(t)} \pf{}{x_i} 
+ v_i'\pf{}{x_i}
}.
	\end{align*}
	
	\item Existenz: Sei $(U_\alpha, x^\alpha)$ eine offene Überdeckung von $M$ durch Karten. Definiere $\Dd{t}$ auf $U_\alpha$ durch
	\begin{align*}
	\Dd{t} V := \sum_i \klam{
		v_i \nabla_{\dot{c}(t)} \pf{}{x_i} 
		+ v_i'\pf{}{x_i}
	}.
	\end{align*}
	Auf $U_\alpha \cap U_\beta$ stimmen diese $\Dd{t}$ überein wegen Eindeutigkeit und definieren somit $\Dd{t}$ überall.
\end{itemize}
\end{Beweis}

\Prop{}
Sei $c$ eine Kurve in $M$, $p = c(0)$. Sei ferner $V^0 \in T_pM$ ein Tangentialvektor. Dann existiert genau ein Vektorfeld $V$ entlang $c$ mit
\[ \Dd{t}V = 0 \]
und
\[ V(0) = V^0. \]

\Def{}
Sei $V$ ein Vektorfeld entlang einer Kurve $c$. $V$ heißt \df{parallel} entlang $c$, falls
\[ \Dd{t}V =0. \]


\marginpar{Vorlesung vom 30.04.18}

\Prop{}
Sei $c$ eine Kurve in $M$ und $V^0\in T_{c(t_0)}M$ ein Vektor bei $c(t_0)$.
Dann existiert genau ein Vektorfeld $V(t)$ entlang $c(t)$, das die Eigenschaften
\begin{align*}
V(t_0) &= V^0\\
\Dd{t} V &= 0
\end{align*}
erfüllt.
\begin{Beweis}{}
\begin{itemize}
	\item Existenz und Eindeutigkeit in lokalen Koordinaten:\\
	Existiert so ein $V$, so gilt
	\begin{align*}
	0 = \Dd{t} V=\sum_k (v'_k + \sum_{i,j} x_i'v_j \Gamma_{i,j}^k) \pf{}{x_k}.
	\end{align*}
	Daraus folgt
	\[ v'_k = - \sum_{i,j} (x_i' \Gamma_{i,j}^k) v_j \]
	für alle $k = 1,\ldots, n$. Dadurch ergibt sich ein System von linearen gewöhnlichen Differentialgleichungen. Aus der Theorie der gewöhnlichen Differentialgleichungen wissen wir, dass es in kleinen Umgebungen von $t$ eindeutige Lösungen für $v_k(t)$ gibt für alle $t$. Da obiges DGL linear ist, sind die $v_k(t)$ für alle $t\in \R$ definiert.
	\item Globale Existenz:\\
	Sei $t_1 > t_0$ beliebig. Der Kurvenabschnitt $c[t_0, t_1]$ ist kompakt und wird folglich überdeckt durch endlich viele Karten. Man kann nun eine lokale Lösung von Karte zu Karte fortsetzen. Die lokalen Lösungen stimmen auf den Durchschnitten der Karten überein wegen ihrer Eindeutigkeit.
\end{itemize}
\end{Beweis}

\Bem{}
\begin{enumerate}[1.)]
	\item Wir erhalten folgende Abbildung
	\begin{align*}
	\tau : T_{c(t_0)}M & \Pfeil{} T_{c(t_1)}M\\
	V^0 & \longmapsto V(t_1).
	\end{align*}
	Diese Abbildung nennt man den \df{Paralleltransport} von $c(t_0)$ nach $c(t_1)$ entlang $c$.\\
	Die Linearität des vorangegangenen Differentialgleichungssystems stellt die Linearität von $\tau$ sicher. Durch Umkehren der Zeit erhält man eine lineare Abbildung
	\begin{align*}
	\tau' : T_{c(t_1)}M & \Pfeil{} T_{c(t_0)}M.
	\end{align*}
	Naheliegenderweise gilt
	\[ \tau' = \tau\i .\]
	Hierdurch folgt insbesondere, dass $\tau$ ein Isomorphismus ist. D.\,h., wir können Tangentialräume an verschiedenen Punkten mittels Paralleltransporte vergleichen.\\
	Daher die Terminologie \textsl{Zusammenhang}.
	\item $\Dd{t}V$ ordnet auch Vektoren an Punkten mit $\dot{c}(t) = 0$ zu. Diese Vektoren müssen nicht Null sein!
	\Bsp{}
	Wenn $c(t) = p$ konstant ist, dann ist $V(t)$ eine Kurve in $T_pM$. $\Dd{t}V$ ist dann einfach die Ableitung von $V(t)$ nach $t$, also $V'(t)$ im euklidischen Sinne.
\end{enumerate}


\newpage
\section{Der Levi-Civita-Zusammenhang}
Sei $(M, g)$ eine Riemannsche Mannigfaltigkeit.

\Def{}
Ein Zusammenhang $\nabla$ auf $M$ heißt \df{kompatibel} mit der Riemannschen Metrik $g$, falls für jede Kurve $c$ und für alle parallele Vektorfelder $V,W$ entlang $c$ gilt:
\[ \shrp{V,W} = \text{konst.} \]
d.\,h., der Paralleltransport ist in diesem Fall sogar eine Isometrie.

\Prop{}
$g$ und $\nabla$ sind genau dann kompatibel, wenn für alle Vektorfelder $V,W$ entlang einer beliebigen Kurve $c$ gilt
\[ \pf{}{t}\shrp{V,W} = \shrp{ \Dd{t}V, W } +\shrp{V, \Dd{t} W}. \]

\begin{Beweis}{}
\begin{itemize}
	\item[$\Leftarrow )$] Seien $V,W$ parallele Vektorfelder entlang $c$. Dann gilt
	\[\pf{}{t}\shrp{V,W} = \shrp{ \Dd{t}V, W } +\shrp{V, \Dd{t} W} = \shrp{ 0, W } +\shrp{V, 0} = 0. \]
	$\shrp{V,W}$ ist als Funktion in $t$ konstant.
	\item[$\Rightarrow )$] $\shrp{,}$ und $\nabla$ seien kompatibel. Sei $\{ P_1(t_0), \ldots, P_n(t_0) \} \subset T_{c(t_0)}M$ eine Orthonormalbasis. Durch den Paralleltransport erhalten wir die parallelen Vektorfelder $P_1, \ldots, P_n$ entlang $c$.\\
	Durch die Kompatibilität bleiben die $P_1, \ldots, P_n$ an jeder Stelle auf $c$ eine Orthonormalbasis. Seien $V,W$ nun beliebige Vektorfelder entlang $c$. Wir können dann schreiben
	\begin{align*}
	V = \sum_i v_i P_i && \text{ und } && W = \sum_j w_j P_j.
	\end{align*}
	Es gilt dann
	\[ \Dd{t}V = \sum_i (v_i' P_i + v_i \Dd{t}P_i) = \sum_i v_i' P_i. \]
	Und somit
	\[ \shrp{\Dd{t}V, W} = \shrp{ \sum_i v_i' P_i , \sum_j w_j P_j  } = \sum_{i,j} v_i'w_j\shrp{P_i,P_j} = \sum_i v_i'w_i. \]
	Und analog
	\[ \shrp{V, \Dd{t}W} =  \sum_i v_iw_i'. \]
	Zusammen also
	\[\shrp{\Dd{t}V, W} + \shrp{V, \Dd{t}W} = \sum_i (v_i'w_i + v_i w_i'). \]
	Ferner gilt
	\[ \shrp{V,W} = \ldots = \sum_i v_i w_i. \]
	Mit der Produktregel folgt nun
	\[ \pf{}{t} \shrp{V,W} = \sum_i (v_i'w_i + v_i w_i'). \]
\end{itemize}
\end{Beweis}

\Kor{}
$g$ und $\nabla$ sind genau dann kompatibel, wenn gilt
\[ X\shrp{Y,Z} = \shrp{\nabla_x Y, Z} + \shrp{Y, \nabla_X Z} \]
für beliebige Tangentialvektorfelder $X,Y,Z$ auf $M$.
\begin{Beweis}{}
Für einen Punkt $p \in M$ wähle eine Kurve $c$ mit $c(0) = p$ und $\dot{c}(0) = X(p)$. Es gilt dann
\[ X(p)\shrp{Y,Z} = \pf{}{t}_{|t = 0} \shrp{Y_{c(t)}, Z_{c(t)}}. \]
\end{Beweis}

\Def{Symmetrie von Zusammenhängen}
Ein Zusammenhang $\nabla$ heißt \df{symmetrisch}, wenn gilt
\[ \nabla_{X} Y - \nabla_YX = [X,Y]. \]
In lokalen Koordinaten für $X = \pf{}{x_i}$ und $Y = \pf{}{y_j}$ gilt dann
\[ \nabla_{\pf{}{x_i}}\pf{}{x_j} - \nabla_{\pf{}{x_j}}\pf{}{x_i} = [\pf{}{x_i}, \pf{}{x_j}] = 0. \]
Daraus folgt dann
\[ \nabla_{\pf{}{x_i}}\pf{}{x_j} = \nabla_{\pf{}{x_j}}\pf{}{x_i}. \]
Für die Christoffel-Symbole bedeutet dies
\[ \Gamma_{i,j}^k = \Gamma_{j,i}^k. \]

\Bem{}
Definiere die \df{Torsion} durch
\[ T(X,Y) := \nabla_{X}Y - \nabla_{Y}X - [X,Y]. \]
$T$ ist linear über $\CC{\infty}(M)$. D.\,h., $T$ ist ein Tensor.\\
Ferner ist ein Zusammenhang genau dann symmetrisch, wenn er torsionsfrei ist.

\Satz{Levi-Civita}
Sei $(M,g)$ eine Riemannsche Mannigfaltigkeit. Dann existiert genau ein Zusammehang $\nabla$ auf $M$, sodass gilt:
\begin{enumerate}[1.)]
	\item $\nabla$ und $g$ sind kompatibel.
	\item $\nabla$ ist symmetrisch.
\end{enumerate}
Diesen Zusammenhang nennen wir den \df{Levi-Civita-Zusammenhang} bzw. den \df{Riemannschen Zusammenhang}.

\begin{Beweis}{}
\begin{enumerate}[option]
	\item[Eindeutigkeit] Seien $X,Y,Z$ beliebige Tangentialvektorfelder auf $M$. Es gilt dann
	\[ X\shrp{Y,Z} = \shrp{\nabla_{X}Y, Z} + \shrp{Y, \nabla_{X} Z} \]
	und
	\[ Y\shrp{Z,X} = \shrp{ \nabla_{Y} Z, X } + \shrp{Z, \nabla_{Y} X} \]
	und
	\[ Z\shrp{X,Y} = \shrp{\nabla_{Z}X, Y} + \shrp{X, \nabla_Z Y}. \]
	Wir addieren die ersten beiden Zeilen und subtrahieren die dritte. Dadurch erhalten wir
	\begin{align*}
	&X\shrp{Y,Z} + Y\shrp{Z,X} - Z \shrp{X,Y} \\
	=&
	\shrp{Y, \nabla_XZ - \nabla_ZX} + \shrp{ X, \nabla_YZ - \nabla_YZ }\\
	+& \shrp{Z, \nabla_XY + \nabla_YX}\\
	\gl{\nabla \text{ symm}}& \shrp{Y, [X,Z]} + \shrp{X, [Y,Z]} + \shrp{Z, [X,Y] + 2 \nabla_YX}\\
	=&  \shrp{Y, [X,Z]} + \shrp{X, [Y,Z]} + \shrp{Z, [X,Y] }+ 2 \shrp{Z,\nabla_YX}\
	\end{align*}
	Daraus erhalten wir für $\nabla$
	\begin{align*}
	&\shrp{Z,\nabla_YX} =\\
	 &\frac{1}{2} \klam{
		X\shrp{Y,Z} + Y\shrp{Z,X} - Z\shrp{X,Y} - \shrp{Y,[X,Z]} - \shrp{X, [Y,Z]} - \shrp{Z, [X,Y]}
	}
	\end{align*}
Daraus folgt die Eindeutigkeit von $\nabla$.
\item[Existenz] Definiere $\nabla_YX$ durch obige Gleichung. Dann bleibt nachzurechnen, dass $\nabla$ ein symmetrischer und kompatibler Zusammenhang ist.
\end{enumerate}
\end{Beweis}

\printindex
\end{document}