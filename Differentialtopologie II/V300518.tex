\marginpar{Vorlesung vom 30.05.18}
%13.te Vorlesung

\subsection{Lemma 2}
Ist $M$ vollständig mit $\kappa \leq 0$, so ist $\exp_p : T_pM \pfeil{} M$ ein lokaler Diffeomorphismus.
\begin{Beweis}{}
Sei $J$ ein Jacobi-Feld entlang $\gamma(t) = \exp_p(tv)$ für $\norm{v} = 1$, mit $J(0) = 0$ und $J\neq 0$.\\
Es genügt zu zeigen, dass $J(t) \neq 0$ für alle $t > 0$. Es gilt
\begin{align*}
\shrp{J,J}'' &= 2 \shrp{J', J'} + 2 \shrp{J, J''}\\
&= 2 \norm{J'}^2 + 2 \shrp{-R(\dot{\gamma}, J')\dot{\gamma}, J}\\
&= 2\norm{J'}^2 - 2 \kappa(\dot{\gamma}, J) \cdot A(\dot{\gamma}, J)^2.
\end{align*}
Da $A(\dot{\gamma}, J)^2 \geq 0$ und $\kappa(\dot{\gamma}, J) \leq 0$, folgt
\[ \shrp{J,J}'' \geq 0. \]
D.\,h., $\norm{J}^2$ ist eine konvexe Funktion von $t$. Es gilt ferner $J'(0) \neq 0$ und $\shrp{J,J}'(0) = 2 \shrp{J,J'}(0) = 0$.\\
Daraus folgt $J(t) \neq 0$ für alle $t > 0$.
\end{Beweis}

\begin{Beweis}{Proposition \ref{PropHadamard}}
Mit Lemma 2 folgt, dass $\exp_{p}$ ein lokaler Diffeomorphismus ist. Daraus folgt, dass die Riemannsche Metrik auf $M$ eine eindeutige Riemannsche Metrik auf $T_pM$ induziert, durch die $\exp_{p}$ zu einer isometrischen Isomorphismus wird.\\
Die Geodätischen in $T_pM$ durch den Ursprung 0 sind die Geraden durch 0. Mit Hopf-Rinow folgt nun, dass $T_pM$ mit der gegebenen Metrik vollständig ist. Mit Lemma 1 folgt nun, dass $\exp_p$ eine Überlagerung ist.
\end{Beweis}




\chapter{Morse-Theorie}
\section{Crash-Kurs: Zellkomplexe und Homologie}
\begin{itemize}
	\item \textbf{Zellkomplexe}:\\
	Idee: Zerlege einen Raum in Teile (\emph{Zellen}), die selbst \emph{keine Topologie} besitzen. Dann kann die Homologie/Kohomologie aus der Kombinatorik dieser Teile abgelesen werden.
\end{itemize}
\Def{Zelle}
Wir definieren \df{Zelle} als alles, was homöomorph zu $D^n = \set{x \in \R^n}{\norm{x} \leq 1}$ ist. Es gilt $\partial D^n = S^{n-1}$.
\paragraph{Schreibweise:}
\[ e^n \isom{}D^n. \]
Wir beschränken uns auf \textbf{endliche} Zellkomplexe.
\begin{itemize}
	\item \textbf{CW-Komplexe}\footnote{Das \emph{C} steht für \emph{closure finite} und das \emph{W} für \emph{weak topology}.}:\\
	CW-Komplexe sind induktiv definiert:
	\begin{align*}
	X^0 &= e_1^0 \sqcup e_2^0 \cup \ldots \sqcup e_{k_0}^0 
	\end{align*}
	ist eine disjunkte Vereinigung von Punkten.
	$X^1$ ergibt sich, indem 1-Zellen dazu nimmt und ihre Randpunkte mit Punkten in $X^0$ identifiziert:
	\[ X^0 \cup_f e^1 := (X^0 \sqcup e^1)/( x\sim f(x)~\forall x \in \partial e^1) \text{ mit }\partial e^1 = S^0 \pfeil{f} X^0. \]
	Dann gilt
	\begin{align*}
	X^1 &= X^0 \cup_{f_1} e_1^1
	\cup_{f_2} e_2^1 \ldots
	\cup_{f_{k_1}} e_{k_1}^1.
	\end{align*}
	Die $f_i : e_i^1 \pfeil{} X^0$ nennt man \df{anheftende Abbildungen}. $X^1$ nennt man auch \df{Graph}.
	
	\[ X^2 : e^2, \partial e^2 = S^1 \pfeil{f} X^1 \]
	\[ X^1 \cup_f e^2 := (X^1 \sqcup e^2) / (\forall x \in \partial e^2 = S^1: ~x\sim f(x)) \]
	\[ X^2 = X^1 \cup_{f_1} e^2_1 \ldots \cup_{f_{k_2}}e_{k_2}^2. \]
	 \textbf{Allgemein}:
	\[ X^n = X^{n-1} \cup_{f_1} e^n_1 \ldots \cup_{f_{k_n}} e_{k_n}^n \]
	mit $f_i : \partial e_i^n \pfeil{} X^{n-1}$ stetig. $X^n$ nennt man das \df{$n$-Skelett}.\\
	Es ergibt sich ferner folgendes Diagramm
	\begin{center}
		\begin{tikzcd}
		e_i^k \arrow[r, "\chi_i"] & X^k \subset X \\
		\partial e_i^k \arrow[u, hook]  \arrow[r, "f"] & X^{k-1} \arrow[u, hook] 
		\end{tikzcd}
	\end{center}
wobei $\chi_i$ ein Homöomorphismus vom Inneren von $e_i^k$ auf sein Bild ist. $\chi_i$ nennt man auch die \df{charakteristische Abbildung}.
\end{itemize}

\Def{}
Ein topologischer Raum von der Form $X^n$ heißt \df{CW-Komplex}.

\Bsp{}
\begin{align*}
S^0 &= e^0 \sqcup e^0\\
S^1 &= e^0\sqcup e^1\\
S^n &= e^0 \sqcup e^n
\end{align*}
und
\begin{align*}
\R P^2 &= D^2 /(x \sim -x, x \in \partial D^2)\\
\R P^2 &= S^1\cup_f e^2
\end{align*}
mit $f$ antipodal. Daraus folgt
\[ \R P^2 = e^0\cup_{\mathrm{konst.}} e^1 \cup_f e^2. \]
\textbf{Allgemein}:
\[ \R P^n = \R P^{n-1} \cup_{\mathrm{Quot} = f}e^n \]
mit $f : \partial e^n = S^{n-1} \Pfeil{\mathrm{Quot}} \R P^{n-1} = S^{n-1}/(x\sim - x)$.
Daraus folgt
\[ \R P^n = e^0 \cup e^1 \cup \ldots \cup_{\mathrm{Quot}} e^n. \]
\textbf{Torus}:
\[ T^2 = e^0 \cup_{\mathrm{konst.}} e^1_a \cup_{\mathrm{konst.}} e^1_b \cup_f e^2 \]
mit $f : \partial e^2 = S^1 \pfeil{} (T^2)^1$. $(T^2)^1$ soll das 1-Skelett des Tori beschreiben. $f = aba\i b\i$.\\
\textbf{Kleinsche Flasche}:
\[ K^2 = e^0 \cup e^1_a \cup e^1_b \cup_f e^2  \]
mit $f : aba\i b$, sonst gilt $(K^2)^1 = (T^2)^1$.

\Def{}
Sei $X$ ein CW-Komplex. Setze
\[ C_k(X) := \Z[X^k]. \]
D.\,h., $C_k(X)$ ist die frei abelsche Gruppe, die von den $k$-dimensionalen Zellen in $X$ frei erzeugt wird. $C_k(X)$ nennen wir die $k$-te \df{zelluläre Kettengruppe} von $X$.

\Bsp{}
Für $X = T^2 = e^0 \cup e_a^1\cup e_b^1 \cup_f e^2$ ist
\begin{align*}
C_{-1} &= 0,
C_0(T^2) &= \Z\shrp{e^0},\\
C_1(T^2) &= \Z\shrp{e^1_a, e^1_b} = \Z\shrp{e^1_a} \oplus \Z\shrp{e^1_b},\\
C_2(T^2) &= \Z \shrp{e^2}.
\end{align*}
Es gibt ferner Randabbildungen
\begin{align*}
\Pfeil{\partial_3} C_2(T^2) \Pfeil{\partial_2}
C_1(T^2) \Pfeil{\partial_1}
C_0(T^2) \Pfeil{\partial_0}
C_{-1}(T^2) \Pfeil{\partial_{-1}}.
\end{align*}
Diese sind gegeben durch
\begin{align*}
\partial_1(e^1_a) &= e^0 - e^0 = 0,\\
\partial_1(e^1_b) &= e^0 - e^0 = 0,\\
\partial_2(e^2) &= e^1_a + e^1_b - e^1_a - e^1_b = 0.\\
\end{align*}
Für die Homologie
\[ H_k(X) 
:=
\frac{ \ker (\partial_k : C_k(X) \pfeil{} C_{k-1}(X)}{ \mathrm{im}(\partial_{k+1} : C_{k+1}(X) \pfeil{} C_k(X) }
\]
ergibt sich nun
\begin{align*}
H_2(T^2) &= C_2(T^2) = \Z,\\
H_1(T^2) &= C_1(T^2) = \Z\oplus \Z,\\
H_0(T^2) &= C_0(T^2) = \Z.
\end{align*}
Für $K^2$ kann man analog nachrechnen
\begin{align*}
H_2(K^2) &= 0,\\
H_1(K^2) &= \Z \oplus \Z /2\Z ,\\
H_0(K^2) &= \Z.
\end{align*}


\Def{Randoperatoren}
Wir definieren \df{Randoperatoren}
\[ \partial_{k} : C_k(X) \Pfeil{} C_{k-1}(X).  \]
Sei $e^k_i$ eine $k$-Zelle von $X$.
\[ \partial e_i^k = S^{k-1}_i \Pfeil{f_i} X^{k-1} \Pfeil{\mathrm{Quot}} \frac{X^{k-1}}{X^{k-2}} \isom{} \bigvee_j S_j^{k-1} \Pfeil{\mathrm{Proj}}S_j^{k-1}. \]
$\partial e_i^k = S^{k-1}_i \pfeil{} S^{k-1}_i$ habe den Abbildungsgrad $d_{i,j} \in \Z$. $\partial_k$ ist dann gerade die Matrix $(d_{i,j})_{i,j}$. Es gilt
\[ \partial_{k-1} \circ \partial_k = 0. \]
Wir definieren ferner die \df{zelluläre Homologie} von $X$ durch
\[ H_k(X) 
:=
\frac{ \ker (\partial_k : C_k(X) \pfeil{} C_{k-1}(X)}{ \mathrm{im}(\partial_{k+1} : C_{k+1}(X) \pfeil{} C_k(X) }.
\]