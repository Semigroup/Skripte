\marginpar{Vorlesung vom 09.07.18}
%24.te Vorlesung

\Satz{Das Spaltungsprinzip}
Sei $E \pfeil{} B$ ein Vektorraumbündel.

Dann existiert eine stetige (bzw. glatte) Abbildung
\[ f: B' \Pfeil{} B \]
mit den Eigenschaften:
\begin{enumerate}[(1)]
	\item $f^*E $ ist die direkte Summe von Geradenbündeln.
	\item Die Abbildung
	\[ f^* : H^*(B) \Pfeil{} H^*(B') \]
	ist injektiv.
\end{enumerate}

\begin{Beweis}{}
Wir führen eine vollständige Induktion nach $n = \text{Rang } E$.
\begin{enumerate}
	\item[$n = 1$:] In diesem Fall ist $E$ bereits ein Geradenbündel, ergo ist in diesem Fall nichts zu zeigen
	\[ f:= \id{B}. \]
	\item[$n- 1 \pfeil{} n$:] Sei $\pi : \P E \pfeil{} B$ die Projektivisierung von $E$. Dann betrachten wir den Pullback
	\begin{center}
		\begin{tikzcd}
		\pi^*E \arrow[r] \arrow[d] & E \arrow[d, "p"]\\
		\P E \arrow[r, "\pi"] & B
		\end{tikzcd}
	\end{center}
$\pi^*E$ enthält das tautologische Geradenbündel $\gamma$ als Unterbündel, wobei $\gamma$ definiert war durch
\[ \gamma := \set{(l_b, v)}{ v \in p\i(b), v \in l_b } \subset \pi^*E. \]
Bilde faserweise den Quotienten
\[ Q := \pi^*E/ \gamma, \]
dadurch wird $Q$ zu einem Vektorraumbündel über $\P E$. Wir erhalten nun eine kurze exakte Sequenz
\[ 0 \Pfeil{} \gamma \Pfeil{} \pi^*E \Pfeil{} Q \Pfeil{} 0. \]
Diese Sequenz spaltet aufgrund geometrischer Möglichkeiten\footnote{Zu jedem Vektor aus $Q$ wählt man das orthogonale Komplement bei passender Metrik.}, wodurch folgt
\[ \pi^* E = \gamma \oplus Q. \]
Der Rang von $Q$ ist kleiner gleich $n-1$, ergo können wir das Prozedere induktiv fortsetzen.
\end{enumerate}
Es bleibt zu zeigen, dass Eigenschaft (2) erfüllt ist. Allerdings ist die Abbildung
\[ \pi^* H^*(B) \Pfeil{} H^*(\P E) \]
wegen dem Diagramm
\begin{center}
	\begin{tikzcd}
	H^*(B) \arrow[r, "\pi^*"] & H^*(\P E)\\
	H^*(B) \otimes H^0(F) \arrow[ur, hook] \arrow[r, hook] & H^*(B) \otimes H^*(F) \arrow[u,swap, "\isom{}\mathrm{LerayHirsch}"]
	\end{tikzcd}
\end{center}
auf die naheliegende Weise injektiv.
\end{Beweis}

\paragraph{Bedeutung des Spaltungsprinzips:} Wir wollen eine polynomiale Identität zwischen Chernklasssen zeigen.
\Def{}
Wir definieren die \df{totale Chernklasse} eines Vektorbündels $E$ durch
\[ c(E) := c_0(E) + c_1(E) + \ldots \in H^*(B). \]


Wir wollen zeigen, dass ein Polynom $p$ existiert, sodass für alle Paare $E, E'$ von Vektorbündeln über $B$ gilt
\[ p(c(E), c(E'), c(E \oplus E')) = 0. \]
Angenommen, diese Identität gilt nur für Geradenbündel, dann können wir das Spaltungsprinzip anwenden und erhalten eine Abbildung $f : B' \pfeil{} B$, sodass gilt
\begin{align*}
f^*p(cE, cE', c(E\oplus E')) &= p(f^*cE, f^*cE', f^*c(E\oplus E'))\\
&= p(c(f^*E), c(f^*E'), c(f^*E \oplus f^*E'))
\end{align*}
Nun sind die Pullbacks der Vektorraumbündeln alles Summen von Geradenbündel. Da wir angenommen haben, dass Geradenbündel unsere Identität erfüllen und $f^*$ injektiv ist, folgt
\[ p(cE, cE', c(E\oplus E')) = 0. \]
D.\,h., es genügt die Identität für Geradenbündel nachzuweisen.
\paragraph{Bemerkung:} Man kann mehrere Bündel gleichzeitig spalten (das nehmen wir auch an, um $E$ und $E'$ gleichzeitig über $B$ zu spalten).

\Bem{Konstruktionen}
\begin{itemize}
	\item \textbf{Tensorprodukt:} Sind $E,E'$ Bündel über dem selben Basisraum, so ist $E \otimes E'$ lokal ihr Tensorprodukt über dem gegebenen Basisraum. Die Übergangsabbildungen sind gegeben durch das Kroneckerprodukt.
	\item \textbf{Duales Bündel:} Sei ein Bündel $\C^n \pfeil{} E \pfeil{} B$ gegeben.
	
	Zu einem komplexen Vektorraum $V$ kann man den dualen Vektorraum $V^* := \Hom{\C}{V}{\C}$ bilden. Führt man dies faserweise durch, erhält man ein Bündel $(\C^n)^* \pfeil{} E^* \pfeil{} B$, dessen Fasern die korrespondierenden dualen Vektorräume sind.
	
	Seien $g_{\alpha \beta} : U_{\alpha \beta} \pfeil{} \mathrm{GL}_n(\C)$ die Übergangsabbildungen von $E$, dann sind $(g_{\alpha \beta}^T)\i$ die Übergangsabbildungen von $E^*$ (d.\,h., die Pfeile drehen sich um).

	Für Geradenbündel $E = l$ gilt insbesondere
	\[ (g_{\alpha \beta}^T)\i = \frac{1}{g_{\alpha \beta}}. \]
	Damit folgt
	\[ c_1(l^*)|_{U_\alpha} = 
	\frac{1}{2\pi i}
	\sum_\gamma \d (g_\gamma \cdot \d \log (g\i_{\gamma \alpha}))
	= -c_1(l)|_{U_\alpha},
	 \]
	 also global
	 \[ c_1(l^*) = -c_1(l). \]
	 Wir merken ferner noch an, dass $l^*$ das multiplikative Inverse von $l$ für Geradenbündel ist, d.\,h.,
	 \[ l\otimes l^* = \text{triviales Geradenbündel} \isom{} \e^1 := B\times \C, \]
	 da die Übergangsabbildungen alle die Identität sind. Beachte, dass $\e^1$ ein Neutralelement ist, d.\,h.
	 \[ l \otimes \e^1 \isom{} l. \]
	 Zusammenfassend folgt, dass die Isomorphieklassen von Geradenbündeln eine Gruppe bilden. Diese Gruppe nennt man die \df{Picard-Gruppe}, geschrieben
	 \[ \mathrm{Pic}^{(\infty)}(B). \]
	 \item \textbf{Homomorphismenbündel}: Sind zwei Vektorbündel $E, E'$ über dem Basisraum $B$ gegeben, so bezeichne $\Hom{}{E}{E'}$ das Vektorbündel, das sich ergibt, indem man lokal die Homomorphismenräume der Vektorräume nimmt. Beachte, dass auf Ebene der Vektorräume und Vektorbündel gilt
	 \[ \Hom{}{V}{W} \isom{} V^* \otimes W. \]
	 
	 Ein Morphismus von Vektorraumbündeln entspricht dann einem Schnitt des Hom-Bündels.
\end{itemize}

Sei $E \pfeil{} B$ eine Whitney-Summe von Geradenbündeln, d.\,h. $E = \bigoplus_{i = 1}^n l_i$. Betrachte das Diagramm
\begin{center}
\begin{tikzcd}
L_i := \pi^* l_i \arrow[r] \arrow[d] & l_i \arrow[d]\\
\P(E) \arrow[r, "\pi"] & B
\end{tikzcd}
\end{center}

Es gilt
$\pi^*E = \bigoplus L_i$. Betrachte das tautologische Bündel $\gamma \pfeil{} \P E$ und
\begin{center}
	\begin{tikzcd}
	\gamma \arrow[r, hook] \arrow[rd, swap, "s_i :="] & \pi^*E = \bigoplus L_i \arrow[d, "Proj."]\\
	 & L_i
	\end{tikzcd}
\end{center}
Wir erhalten globale Schnitte
\[ s_i \in \Gamma(B, \Hom{}{\gamma}{L_i}) = \Gamma(B, \gamma^* \otimes L_i). \]
Setze
\[ U_i := \set{x \in \P E}{s_i(x) \neq 0}. \]
Es gilt
\[ c_1(\gamma^* \otimes L_i) = [\xi_i] \in H^2(\P E) \]
für passende $\xi \in \Omega^2(\P E)$. Ferner
\[ c_1(\gamma^* \otimes L_i)|_{U_i} = 0, \]
da $\e^1 \isom{} (\gamma^* \otimes L_i)|_{U_i}$ einen nirgendwo verschwindenden Schnitt $s_i$ hat.
Daraus folgt, dass $\omega_i \in \Omega^1(U_i)$ existieren mit
\[ \xi|_{U_i} = \d \omega_i. \]
Es gilt
\[ \P E = \bigcup_{i=1}^n U_i \]
denn für $x \in \P E$ existiert ein $i$ mit $s_i(x) \neq 0$, denn $\gamma$ ist in $\pi^* E$ enthalten.

Sei $\{ V_i\}_{i = 1,\ldots, n}$ eine offene Überdeckung von $\P E$ mit folgenden Eigenschaften:
\begin{itemize}
	\item $\overline{V_i} \subset U_i$
	\item Es gibt glatte Funktionen $f_i$, die auf $\overline{V_i}$ konstant 1 sind und deren Träger in $U_i$ enthalten sind.
\end{itemize}
Es gilt dann
\[{\xi_i}|_{V_i} - \d (f_i \omega_i)|_{V_i} = 0. \]
und
\[ c_1(\gamma^* \otimes L_i)|_{V_i} = [ (\xi_i - \d (f_i \omega_i))|_{V_i}], \]
folgt
\begin{align*}
&\prod_{i= 1}^nc_1(\gamma^* \otimes L_i) = 0\\
=&\prod_{i= 1}^n(c_1(\gamma^*) + c_1(L_i))\\
=&\prod_{i= 1}^n(y + c_1(L_i))\\
=& y^n + \sigma_1 y^{n-1} + \sigma_2 y^{n-2} + \ldots + \sigma_n
\end{align*}
für $y := c_1(\gamma^*)$. Die $\sigma_i$ heißen \df{elementarsymmetrische Polynome}.

Es folgt
\[ c(E) = c(\bigoplus_i l_i)
= \prod(1 + c_1(L_i)) = \prod c(L_i).
 \]
Die Produktformel von Whitney ist somit gezeigt für Summen für Geradenbündel.

\paragraph{Bemerkung:}
Seien $E,E'$ beliebige komplexe Vektorbündel über $B$. Mit dem Spaltungsprinzip erhalten wir eine Abbildung $f : B' \pfeil{} B$, sodass
\[ f^* E = \bigoplus_i l_i \text{ und } f^* E' = \bigoplus_i l_i'  \]
die Summen von Geradenbündel sind. Nun gilt
\begin{align*}
&f^*(
c(E) c(E') - c(E \otimes E')
)\\
=& f^*c(E) f^*c(E') - f^*c(E\otimes E')\\
=& c(f^*E) c(f^*E') - c(f^*E \otimes f^*E')\\
=&c(\bigoplus l_i) c(\bigoplus l'_i) - c((\bigoplus k_i) \otimes (\bigoplus l'_i))\\
=& \prod c(l_i) \cdot \prod c(l_i') - \prod c(l_i) \cdot \prod c(l_i') = 0
\end{align*}
Da $f^*$ injektiv ist, folgt nun die \df{Whitneysche Produktformel}
\[ c(E\oplus E') = c(E) \cdot c(E'). \]

\paragraph{Folgerung:}
$\e^k$ bezeichne das triviale Bündel von Rang $k$. Es gilt
\[ c(E\oplus \e^k) = c(E) \cdot c(\e^k) = c(E). \]
Diese Eigenschaft nennt man die \df{Stabilität der Chernklassen}. Das Hinzuaddieren eines trivialen Bündels nennt man \df{Stabilisieren}.