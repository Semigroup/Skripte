\marginpar{Vorlesung vom 04.07.18}
%23.te Vorlesung

Sei $\C^n \pfeil{} E \pfeil{} B$ ein komplexes Vektorraumbündel.
\paragraph{Projektivisierung:}
$\P^{n-1} = P\C^n \pfeil{} \P(E) \pfeil{\pi} B$.

Ist $A \in \mathrm{GL}_n(\C)$, so induziert $A$ einen Homöomorphismus
\begin{align*}
A : P\C^{n} & \Pfeil{} P\C^n\\
l & \longmapsto A(l).
\end{align*}
Betrachte den Pullback
\begin{center}
	\begin{tikzcd}
\gamma_{\P E} \arrow[r, hook] \arrow[rd]&	\pi^* E \arrow[r] \arrow[d] & E \arrow[d, "\pi"]\\
	&\P(E) \arrow[r, "\pi"] & B
	\end{tikzcd}
\end{center}
und das Geradenbündel über $\P E$
\[ \gamma_{\P E}E := \set{(l,v) }{v \in l} \subset \pi^* E. \]
$\gamma_{\P E}$ nennt man das \df{tautologische Geradenbündel}. In der algebraischen Geometrie ist $\gamma_{\P E}$ unter der Bezeichnung $\O(-1)$ geläufig.

\Prop{}
$\P E \pfeil{\pi} B$ ist kohomologisch gespalten.
\begin{Beweis}{}
Setze
\[ y:= c_1(\gamma_{\P E}) \in H^2(\P E) .\]
Die Inklusion
\[ j : \P^{n-1} \Inj{} \P E \]
induziert einen Homomorphismus
\[ j^* : H^*(\P E) \Pfeil{} H^*(\P^{n-1}) = \R[\mathrm{Erz.}] / (\mathrm{Erz.}^n = 0). \]
Nun gilt
\[ j^*(y) = j^*c_1(\gamma_{\P E}) = c_1(j^*\gamma_{\P E}) = c_1(\gamma_{\P E}|_{\mathrm{Faser}})
= c_1(\gamma_{\P^{n-1}}) = \mathrm{Erz.} \]
Da $j^*$ ein Ringhomomorphismus ist, gilt nun
\[ j^*(y^k) = \text{Erz.}^k. \]
Ergo muss $j^*$ surjektiv sein. Da es sich bei den Kohomologiegruppen um reelle Vektorräume handelt, können wir eine Spaltung $\beta$ wie folgt konstruieren.
\begin{align*}
\beta : H^*(\P^{n-1}) & \Pfeil{} H^*(\P E)\\
\mathrm{Erz.}^k & \longmapsto y^k.
\end{align*}
\paragraph{Bemerkung:} $\beta$ ist eine lineare Spaltung, aber im Allgemeinem kein Ringhomomorphismus. Zum Beispiel gilt
\[ \mathrm{Erz.}^n = 0, \]
aber
\[ y^n \neq 0 \]
im Allgemeinem.\\



Es gilt nun
\[ j^*\beta (\mathrm{Erz.}^k) = j^*(y^k) = \mathrm{Erz.}^k\]
und somit
\[ j^*\beta = \id{}. \]
\end{Beweis}

Aufgrund dieser Proposition können wir den Satz von Leray-Hirsch auf $\P E$ anwenden und erhalten folgenden Isomorphismus von linearen Vektorräumen
\[ H^*(\P E) \isom{}H^*(B) \otimes (\R[\mathrm{Erz.}]/ (\mathrm{Erz.}^n = 0)).  \]
Es bezeichne $\Phi$ den Isomorphismus $H^*(B) \otimes (\R[\mathrm{Erz.}]/ (\mathrm{Erz.}^n = 0)) \pfeil{} H^*(\P E)$. Es sei $n$ der Rang von $E$. Es gilt nun
\[ \Phi\i(y^n) = a_0\otimes 1  
+ a_1\otimes \mathrm{Erz.}
+ a_2\otimes \mathrm{Erz.}^2
+ a_3\otimes \mathrm{Erz.}^3
+ \ldots
+ a_{n-1}\otimes \mathrm{Erz.}^{n-1} \]
für $a_i \in H^{2(n-i)}(B)$. Wendet man nun $\Phi$ an, so gilt
\[ y^n = \pi^*(a_0) + \pi^*(a_1) \wedge y + \pi^*(a_2) \wedge y^2 + \ldots +\pi^*(a_{n-1}) \wedge y^{n-1}.  \]

\Def{}
Wir definieren die $i$-te \df{Chernklasse} von $E$ durch
\[ c_i(E) := a_{n-i} \in H^{2i}(B). \]
Insbesondere setzen wir
\[ c_0(E) := 1 \in H^0(B) \]
und
\[ c_i(E) := 0 \]
für $i > \text{Rang } E.$

\Bem{Chernklassen des trivialen Bündels}
Betrachte die Bündel
\begin{center}
\begin{tikzcd}
E = B\times \C^n \arrow[d, "\pi = \pi_1"] & \Impl{} & \P E = B\times \P^{n-1} \arrow[d, "\pi_1"] \arrow[r, "\pi_2"] & \P^{n-1}\\
B & & B 
\end{tikzcd}
\end{center}
Es gilt dann
\[ \pi_2^*\gamma_{\P^{n-1}} = \gamma_{\P E} \]
und
\[ y = c_1(\gamma_{\P E}) = c_1(\pi_2^*\gamma_{\P^{n-1}}) = \pi_2^* c_1(\gamma_{\P^{n-1}}) = \pi_2^*\mathrm{Erz.} \]
Somit
\[ y^n = (\pi_2^*\mathrm{Erz.})^n = \pi_2^*(\mathrm{Erz.}^n) = \pi_2^* 0 = 0. \]
Somit sind alle Chernklassen $c_i(E)$ gleich Null im trivialen Fall für alle $i> 0$.
\Bem{Natürlichkeit von $c_i$}
Betrachte das Diagramm
\begin{center}
\begin{tikzcd}
f^*E \arrow[r, "F"] \arrow[d] & E \arrow[d] & \rightsquigarrow & \P(f^*E)  \arrow[r] \arrow[d, "\overline{\pi}"]& \P(E) \arrow[d, "\pi"]\\
 X \arrow[r, "f"] & Y& & X \arrow[r, "f"] & Y
\end{tikzcd}
\end{center}
und
\begin{center}
\begin{tikzcd}
\C  \arrow[r] & \gamma_{\P f^*E} \arrow[d] \arrow[r, "\mathrm{kart.}"] & \gamma_{\P E} \arrow[d] & \arrow[l] \C\\
&\P(f^*E)  \arrow[r, "F"] & \P E  
\end{tikzcd}
\end{center}
D.\,h.
\[ \gamma_{\P f^*E} = F^*\gamma_{\P E}. \]
Setzt man
\[ x:= c_1(\gamma_{\P f^*E}) \in H^2(\P f^* E) .\]
Somit gilt
\[ F^*(y) = F^*c_1(\gamma_{\P E}) = c_1 (F^*\gamma_{\P E}) = c_1(\gamma_{\P f^*E}) = x. \]
Somit gilt
\[ F^*(y) = x. \]

Ferner gilt nun
\[ y^n = \pi^* c_n(E) + \pi^*c_{n-1}(E) y + \ldots \]
und somit
\[ F^*(y^n) = F^*\pi^*c_n(E) + F^*\pi^*c_{n-1}(E)F^*(y) + \ldots
= \overline{\pi}^* f^*c_n(E) + \overline{\pi}^* f^*c_{n-1}(E) F^*(y) + \ldots \]
Substituiert man $x = F^*(y)$, so ist dies gleich zu
\[ x^n = \overline{\pi}^*c_n(f^* E) + 
\overline{\pi}^*c_{n-1}(f^* E) x +
\overline{\pi}^*c_{n-2}(f^* E) x^2 +\ldots 
 \]
Somit gilt
\[ c_i(f^* E) = f^*c_i(E) \in H^{2i}(X) \]
für alle $i$.

\Prop{}
Hat $\pi : E \pfeil{} B$ einen nirgendwo verschwindenden Schnitt, so gilt $c_n(E) = 0$, wobei $n = \text{Rang }E$.
\begin{Beweis}{}
Sei $s : B \pfeil{} E$ ein Schnitt, der nirgendwo verschwindet. $s$ induziert einen nichtverschwindenden Schnitt in der Projektivisierung
\begin{align*}
\widetilde{s} : B & \Pfeil{} \P E\\
b & \longmapsto \text{Gerade erzeugt durch }s(b).
\end{align*}
Nun ist $\widetilde{s}^*(\gamma_{\P E})$ das triviale Geradenbündel, denn es hat einen nirgendwo verschwindenden Schnitt, nämlich $s$. Ferner gilt für $y = c_1(\gamma_{\P E})$
\[ \widetilde{s}^*(y) = c_1(\widetilde{s}^*\gamma_{\P E}) = c_1(\mathrm{triv.}) = 0. \]
Somit folgt
\[ \widetilde{s}^*(y^n) = 0. \]
Da
\[ y^n = \pi^* c_n(E) +
\pi^*c_{n-1}(E) y
+\pi^* c_{n-2}(E) y^2 + \ldots,
 \]
folgt
\[ 0 = \widetilde{s}^*(y^n)
= \widetilde{s}^*\pi^*c_n(E)
+ \widetilde{s}^*\pi^*c_{n-1}(E)\widetilde{s}^*(y) + \ldots
 \]
 Da gilt
 \[ \widetilde{s}^* \pi^* = \id{} \]
 und
 \[ \widetilde{s}^*(y) = 0, \]
 folgt
 \[ 0 = c_n(E) + ( \ldots) \cdot \widetilde{s}^*(y) = c_n(E). \]
\end{Beweis}


\Bem{Weitere Konstruktionen mit Vektorraumbündeln}
\begin{itemize}
	\item \textbf{Tensorprodukt: } Es seien $E$ und $E'$ zwei Vektorraumbündel über demselben Basisraum $B$.
	Wir definieren das Tensorprodukt $E \otimes E'$, indem wir faserweise Tensorprodukte bilden.	
	Um die Übergangsabbildungen zu erklären, führen wir das \df{Kroneckerprodukt} von Matrizen ein:
	
	Sind $A \in \mathrm{GL}_n(\C)$ und $B \in \mathrm{GL}_m(\C)$ gegeben, so definieren wir eine $(nm)\times (nm)$-Matrix durch
	\[A \otimes B := 
	\left(
	\begin{matrix}
	a_{1,1} B & \ldots & a_{1,n} B\\
	\vdots & & \vdots\\
	a_{n,1} B & \ldots & a_{n,n} B
	\end{matrix}
	\right). \]
	Sind nun Übergangsfunktionen $g_{\alpha \beta} : U_{\alpha \beta} \pfeil{} \mathrm{GL}_n(\C)$ für $E$ und
	$g_{\alpha \beta}' : U_{\alpha \beta} \pfeil{} \mathrm{GL}_n(\C)$ für $E'$. Die Übergangsfunktionen für $E\otimes E'$ sind dann die Kroneckerprodukte $g_{\alpha \beta} \otimes g_{\alpha \beta}' : U_{\alpha \beta} \pfeil{} \mathrm{GL}_{nm}(\C).$
	\paragraph{Beispiel:}
	Sei $n = m = 1$. Dann sind $E,E'$ Geradenbündel mit Übergangsfunktionen
	\begin{align*}
	g_{\alpha\beta} : U_{\alpha \beta} & \Pfeil{} \C^\times = \C - 0\\
	g_{\alpha\beta}' : U_{\alpha \beta} & \Pfeil{} \C^\times.\\
	\end{align*}
	Die Übergangsfunktionen von $E\otimes E'$ sind dann $g_{\alpha\beta}\cdot g_{\alpha\beta}'$. Es gilt dann
	
	\[c_1(E \otimes E') = \eu((E \otimes E')_{\R}) = 
	\frac{1}{2\pi i} \sum_\gamma \d (g_\gamma \cdot \d \log (g_{\alpha \beta}\cdot g_{\alpha \beta}')).
	\]
	Da
	\[ \log (g_{\alpha \beta}\cdot g_{\alpha \beta}') = \log (g_{\alpha \beta}) + \log (g_{\alpha \beta}'),\]
	folgt
	\[ c_1(E\otimes E') = c_1(E) + c_1(E'). \]
\end{itemize}