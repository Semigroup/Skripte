\marginpar{Vorlesung vom 02.07.18}
%22.te Vorlesung

\section{Die Thom-Klasse}
\Def{}
Sei $\R^2 \pfeil{} E \pfeil{\pi} M$ ein orientiertes Vektorraumbündel über einer kompakten Mannigfaltigkeit $M$. Wir erinnern uns daran, dass für die Karten aus dem vorhergehenden Abschnitt gilt
\[ \d \theta_\beta - \d \theta_\alpha = \d \phi_{\alpha\beta} = 2\pi (\xi_\beta - \xi_\alpha) \]
für
\[ \xi_\alpha = \frac{1}{2 \pi} \sum_\gamma g_\gamma \d \phi_{\gamma\alpha}. \]
Ferner gilt
\[ \d \theta_\beta - 2\pi \xi_\beta = \d \theta_\alpha - 2\pi \xi_\alpha \]
und somit
\[ \frac{\d \theta_\beta}{2 \pi} - \xi_\beta = \frac{\d \theta_\alpha}{2\pi} - \xi_\alpha \]
auf $U_{\alpha \beta} = U_\alpha \cap U_\beta$.

Die $\{\frac{\d \theta_\alpha}{2 \pi} - \xi_\alpha\}_\alpha$ definieren eine globale 1-Form $\Psi \in \Omega^1(E_0)$, die sogenannte \df{globale Winkelform}, für
\[ E_0 = \set{v\in E}{v\neq 0} = E - \mathrm{0-Schnitt}. \]

Wähle eine glatte monotone Funktion $f(r)$ mit folgenden Eigenschaften
\begin{align*}
f(r) = \left\lbrace
\begin{aligned}
&-1, && r \leq 0\\
&\in [0,1], && r \in [0, 1]\\
&0, && r \geq 1.
\end{aligned}
\right.
\end{align*}
Die Ableitung von $f$ ist dann eine bump function.

Wir setzen
\[ \Phi := \d (f \cdot \Psi). \]
Es gilt
\[ \d \Psi = \d (\frac{\d \theta_\alpha}{2 \pi} - \xi_\alpha) = \d \xi_\alpha = -\pi^* \eu, \]
d.\,h.,
\[ \d \Psi = -\pi^*\eu. \]
Somit gilt
\[ \Phi = \d (f \cdot \Psi) = \d f \wedge \Psi + f \d \Psi = \d f \wedge \Psi - f \pi^*\eu.  \]
Ergo hat $\Phi$ einen kompakten Träger, d.\,h.
\[ \Phi \in \Omega^2_c (E). \]
Ferner ist $\Phi$ offensichtlich geschlossen und repräsentiert somit eine Kohomologieklasse
\[ \Phi := [\Phi] \in H^2_c(E). \]
$\Phi$ nennt man die \df{Thom-Klasse} von $E \pfeil{} M$.



Sei $s : M \pfeil{} E$ der Nullschnitt, d.\,h.
\[ \pi \circ s = \id{M}, \]
woraus folgt
\[ s^* \pi^* = \id{H^*(M)}. \]
Betrachte nun
\[ s^*\Phi = s^* \d (f\Psi) = \d (s^* (f\Psi)) = \d ( (f\circ s)(r) \cdot s^*\Psi) = \d (f(0) \cdot s^*\Psi ) = - \d (s^* \Psi) = - s^*(\d \Psi) = s^*\pi^* \eu = \eu. \]
Somit folgt
\[ s^*(\text{Thom-Klasse}) = \text{Eulerklasse von E}. \]

\Satz{Isomorphiesatz von Thom}
Sei $M$ kompakt. Die Abbildung
\begin{align*}
H^*(M) & \Pfeil{} H^{*+2}_c(E)\\
[\omega] & \longmapsto [(\pi^*\omega) \wedge \Phi ]
\end{align*}
ist ein Isomorphismus. Er heißt auch \df{Thom-Isomorphismus}.
\begin{Beweis}{}
Wir besprechen zwei mögliche Beweise.
\begin{enumerate}[1.]
	\item Man benutzt wie gewohnt die Mayer-Vietoris-Sequenz, das Fünfer-Lemma, etc..
	\item Wir verwenden die Intergation entlang der Faser
	\begin{align*}
\pi_* : H^{* + 2}_c (E )  \Pfeil{} H^*(M).
\end{align*}
	Es gilt dann die sogenannte \df{Projektionsformel}:
	
	Sind $\omega \in H^*(M)$ und $\eta \in H^*_c(E)$, so gilt
	\[ \pi_*(\pi^* \omega \wedge \eta) = \omega \wedge \pi_*\eta.\]
	Für $\eta = \Psi$ gilt
	\[ \pi_*\Phi = 1. \]
	Und somit
	\[ 
	\pi_*\Phi_{|\text{Faser}} = \int_{0}^\infty \int_{0}^{2\pi} \d f \frac{\d \theta}{2\pi} = f(\infty) - f(0) = 0 - (-1) = 1.
	 \]
	 Mit der Projektionsformel gilt
	 \[ \pi_*(\pi^* \omega \wedge \Phi) = (\omega\wedge \pi_* \Phi) = \omega \wedge 1 = \omega. \]
	Daraus folgt, dass der Thom-Isomorphismus ist invers zu $\pi_*$, der Integration der Faser.
\end{enumerate}
\end{Beweis}

\section{Gysin-Sequenz}
Betrachte wieder das Bündel $E \pfeil{\pi} M$. Es trägt eine Riemannsche Metrik, insofern können wir das \df{Disk-Bündel} definieren durch
\[ D(E) := \set{v \in E}{\norm{v} \leq 1}. \]
Das \df{Sphären-Bündel} definiert man analog durch
\[ S(E) := \set{v \in E}{\norm{v} = 1}. \]
Wir erhalten dann Faserbündel.
\begin{center}
	\begin{tikzcd}
	D^2  \arrow[r]& D(E)\arrow[d, "\pi|"] & & S^1 \arrow[r] & S(E)\arrow[d, "\pi|"]\\
	& M & & & M  
	\end{tikzcd}
\end{center}
$D(E)$ ist dann eine glatte Mannigfaltigkeit mit $S(E)$. Ferner gilt
\[ D(E) \simeq M \]
durch eine von $\pi$ induziert Homotopie. Und
\[ S(E) \simeq E_0 \]
durch die Homotopie $\R^2 - 0 \simeq S^1$.

Wir betrachten
\[ H^*(D(E), S(E))  \isom{} H^*_c(D(E) - S(E)). \]
Da folgende Diffeomorphie vorliegt
\[ D(E) - S(E) \isom{} E, \]
gilt nun 
\[ H^*(D(E), S(E))  \isom{} H^*_c(D(E) - S(E)) \isom{} H^*_c(E). \]
Wir betrachten die lange exakte Sequenz
\begin{center}
\begin{tikzcd}
H^{j-1}(SE) \arrow[r, "\delta^*"] & H^j(DE, SE) \arrow[r] \arrow[d, "\isom{}"] & H^j(DE) \arrow[r]\arrow[d, "\pi^*; \isom{}"]  & H^j (SE) \arrow[r, "\delta^*"] \arrow[d, "\isom{}"]& H^{j+1}(DE, SE)\arrow[d, "\isom{}"]\\
& H^j_c(E) \arrow[d, "\isom{}"] & H^j(M) & H^j(E_0) & H^{j-1}(M)\\
& H^{j-2}(M) \arrow[ur, "\_ \wedge \eu"]
\end{tikzcd}	
\end{center}
Wir erhalten folgende exakte Sequenz, die sogenannte \df{Gysin-Sequenz}
\[ H^{j-1}(E_0) \Pfeil{\delta^*} H^{j-2} (M) \Pfeil{\_ \wedge \eu} H^j(M) \Pfeil{} H^j(E_0) \Pfeil{\delta^*} H^{j-1}(M). \]

\section{Euler-/Chernklasse des Tautologischen Geradenbündels über $\C\P^{n-1}$}
Wir schreiben in diesem Abschnitt kurz
\[ \P^{n-1} := \C P^{n-1}.\]
Es gilt
\[ \P^{n-1} = \set{l \subset \C^n \text{ UVR}}{
\dim_\C l = 1
} \]
Das \df{Tautologische Geradenbündel} ergibt sich durch
\[ \C \Pfeil{} \gamma := \set{
(l,v)
}{
v \in l
} \subset \P^{n-1} \times \C^n \Pfeil{} \P^{n-1}. \]

Betrachte
\begin{center}
	\begin{tikzcd}
	\gamma \arrow[r, hook] \arrow[rd, swap, "\sigma := "] & \P^{n-1} \times \C^n \arrow[d, "\text{pr}_2"]\\
	& \C^n
	\end{tikzcd}
\end{center}
Sei $v \in \C^n$.
\begin{itemize}
	\item Ist $v\neq 0$, so liegt $v$ in einem $l = \shrp{v}\in \P^{n-1}$. Es ist
	\[ \sigma\i(v) = \{l\} \]
	ein Punkt.
	\item Ist $v = 0$, so gilt
	\[ \sigma\i(\{0\})  = \P^{n-1}.\]
\end{itemize}
$\sigma : \gamma \pfeil{} \C^n$ nennt man einen \df{Blow-Up} von $\C^n$ entlang $\{0\} \subset \C^n$. Die Einschränkung
\[ \sigma| : \gamma - \text{0-Schnitt} \Pfeil{\isom{}} \C^n - \{0\} \]
ist ein Diffeomorphismus. Es folgt
\[ E_0 \isom{} S^{2n-1}. \]
Ferner wissen wir
\[ H^*(\P^{n-1}) = \R[x] /(x^n = 0), \]
wobei $x \in H^2(\P^{n-1})\isom{} \R$ ein Erzeuger von Rang 2 ist. Setzt man das in die Gysin-Sequenz ein, so erhält man
\[ H^{1}(S^{2n-1}) = 0 \Pfeil{\delta^*} H^0(\P^{n-1}) \Pfeil{\_ \wedge \eu} H^2(\P^{n-1}) \Pfeil{} H^2(S^{2n-1}) = 0\Pfeil{\delta^*}H^1(\P^{n-1}). \]
Dadurch erhalten wir eine Isomorphie
\begin{align*}
H^0(\P^{n-1}) & \Pfeil{} H^2(\P^{n-1})\\
1 \longmapsto \eu.
\end{align*}
Dadurch ist $\eu(\gamma_\R) = c_1(\gamma)$ der Erzeuger von $H^*(\P^{n-1})$.