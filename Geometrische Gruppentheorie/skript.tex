\documentclass{book}

\usepackage[T1]{fontenc}
\usepackage[UTF8]{inputenc}
\usepackage{amsmath}
\usepackage{amsfonts}
\usepackage{amssymb}
\usepackage{stmaryrd}
\usepackage{lmodern}
\usepackage{ngerman}
\usepackage{xcolor}
\usepackage{geometry}
\usepackage[siunitx]{circuitikz}
\usepackage{tikz-cd}
\usepackage{makeidx}
\usepackage{hyperref}

\newcommand{\df}[1]{\textbf{#1}\index{#1}}
\newcommand{\N}{\mathbb{N}}
\newcommand{\Z}{\mathbb{Z}}
\newcommand{\Q}{\mathbb{Q}}
\newcommand{\R}{\mathbb{R}}
\newcommand{\C}{\mathbb{C}}
\renewcommand{\H}{\mathbb{H}}
\newcommand{\T}{\mathcal{T}}
\newcommand{\Top}{\textbf{Top}}
\newcommand{\Mod}{\textbf{Mod}}

\newcommand{\set}[2]{\left\lbrace #1~|~#2 \right\rbrace}
\newcommand{\grp}[2]{\left\langle #1~|~#2 \right\rangle}

\newcommand{\qed}{$\square$}

\newcommand{\gl}[1]{\overset{#1}{=}}
\newcommand{\grgl}[1]{\overset{#1}{\geq}}
\newcommand{\klgl}[1]{\overset{#1}{\leq}}
\newcommand{\gr}[1]{\overset{#1}{>}}
\newcommand{\kl}[1]{\overset{#1}{<}}
\newcommand{\norm}[1]{\left|\left|#1\right|\right|}

\newcommand{\Matrix}[4]{\left(\begin{matrix}
		#1 & #2\\
		#3 & #4
	\end{matrix}\right)}

\begin{document}

% 15.05.15: 10.VL

Beweis $\Longleftarrow$ von 
\[G \text{ frei } \Longleftrightarrow G \curvearrowright_\text{frei} Baum \]
\[S' := \set{g_e \in G}{e \text{ wesentlich für }T_0} \]
wesentlich heißt
\[e = \{u,v\}, u \in T_0, v \notin T_0\]
$g_e$ so, dass $g_e^{-1} v \in V(T_0)$

\paragraph{2.Schritt}
Zeige $S'$ erzeugt $G$:\\

$g\in G$, Ziel: finde Elemente in $S'$ so, dass $g$ Produkt dieser ist.\\

Wähle Ecke $u \in T_0$, weil $T$ zusammenhängend, existiert Kantenpfad $p$ in $T$ von $u$ nach $g.u$.

Weil $V(T) = \bigcup_{g\in G} V(g.T_0)$, weil $T_0$ aus jedem $G$-Orbit eine Ecke enthält.\\
$\Longrightarrow$ $p$ durchläuft verschiedene Kopien $g_0T_0, ..., g_nT_0$ von $T_0$ mit $g_0 = 1, g_n = g$.\\
Es ist $g_{j+1} \neq g_j$ für $\forall j: k_0 \leq j \leq k_1$, wenn $p$ reduziert.\\
$\Rightarrow$ $g_jT_0$ und $g_{j+1}T_0$ sind für alle $j$ wie oben verbunden.\\
$g_j^{-1} e_j$ ist wesentliche Kante für $T_0$; $p = e_0...e_{n-1}$\\
Setze $s_j := g_j^{-1}g_{j+1} \in S'$.\\
Dann $g = g_0 \cdots g_{k_0}^{-1}g_{k_0 +1} g_{k_0 +1}^{-1} \cdots g_n = s_0 \cdots s_n \in \grp{S'}{}$

\paragraph{3.Schritt}
$\exists S \subset S'$, das G frei erzeugt.

aus 1.Schritt folgt, dass S' in Paare aufspaltet $\{s, s^{-1}$; für S wähle ein Element pro Paar aus.

Es reicht zu zeigen: Cay(G,S) enthält keine Kreise.

Annahme: Sei $g_0,\ldots, g_{n-1}, g_n = g_0$ Kreis in Cay(G,S)

Setze $s_j := g_j^{-1}g_{j+1} \forall j= 0, \ldots , n-1$

Es sei $s_j\in S \forall j$ (OE: S so wählbar)

Sei $e_j$ wesentliche Kante zw. $T_0$ und $s_jT_0$

Jede Kopie von $T_0$ ist zusammenhängender Teilbaum, daher können wir die Ecken der Kanten $g_je_j$ und $g_js_je_{j+1} = g_{j+1}e_{j+1}$, die in $g_{j+1}T_0$ liegen durch einen eindeutigen, reduzierten Weg in $g_{j+1}T_0$ verbinden.

Weil $g_n = g_0$, ist der erhaltene Weg geschlossen.\\
Starten und Enden in selber Kopie vom Baum $T_0$. Widerspruch zu $T$ ist Baum.

\subsubsection{Korollar 3.15 (Satz von Nielsen-Schreier)} Untergruppen freier Gruppen sind frei.
\paragraph{Beweis} Eine Untergruppe wirkt frei auf den Cayleygraphen seiner Obergruppe.

\subsubsection{Korollar 3.16} $F$ freie Gruppe, $Rang(F) = n$, $G < F$ UG vom Index $k$. Dann ist G frei und vom Rang $k(n-1) + 1$. Insbesondere sind Untergruppen vom endlichen Index in freien Gruppen vom endlichen Index endlich erzeugt.
\paragraph{Beweis} $S$ freies EZS von $F$, $\Gamma:= Cay(G,S)$, $G,F \curvearrowright_{frei} \Gamma$ durch Linksmult.

Bew 3.11: Rang(G) = $\frac{1}{2} E$, E = \# wesentlicher Kanten für Fund.-Baum $T_0$ von $G \curvearrowright T$

Weil $|F:G| = k$ hat $T_0$ genau $k$ Ecken.

Es gilt $d_T(v) = 2n$ für alle v in T.

Dann: (1) $\sum_{v\in V(T_0)} d_T(v) = k 2n$, andererseits ist $T_0$ endlicher Baum mit $k$ Ecken, also hat $T_0$ $k-1$ Kanten.

In (1) werden Kanten doppelt gezählt, d.h.
\[\sum_{v\in V(T_0)}d_T(v) = 2(k-1) + E\]
\[1/2 E = k(n-1) + 1 = Rang G\]

\subsubsection{Korollar 3.17}
F frei vom Rang $m \geq 2$, und $n\in \N$, Dann gibt es UG von $F$, die frei und vom Rang n ist.

\subsubsection{3.18 Ping-Pong Lemma (Felix Klein)}
$G$ Gruppe, erzeugt von $S = \{a,b\}$, wobei $a,b$ unendliche Ordnung.\\
$G \curvearrowright X$, X Menge, so dass für $\emptyset \neq A, B \subset X$
mit $B \not \subset A$ gilt:
\[a^nB \subset A \text{ und } b^nA \subset B, \forall n \in \Z\setminus\{0\} \]

dann ist G frei von S erzeugt.

\paragraph{Beweis}
Zu zeigen $G \cong F_{red}(a,b)$ via Isom, der S festhält.

UAE:
$\phi : F_{red}(a,b) \longrightarrow G$ mit $\phi|S = id$, dann ist $\phi$ surjektiv.

Zu zeigen: $\phi$ injektiv.

Annahme: $\phi$ nicht injektiv, dann existiert $w\in F_{red}(S)$ mit $\phi(w) = 1$

4 Fälle: 

\paragraph{1.Fall}
w beginnt mit nichttriv. Potenz von a und endet mit einer solchen:
\[w = a^{n_0}b^{m_0}...b^{m_k}a^{n_{k+1}}, n_i, m_i \in \Z- 0 \]

Nun ist $B = 1.B = \phi(w)B = a^{n_0}b^{m_0}...b^{m_k}a^{n_{k+1}}.B \subset A$. Widerspruch!

\paragraph{2.Fall}
w beginnt mit $b$ und endet mit $b$. konjugiere mit a: 1.Fall

\paragraph{3.Fall}
w beginnt mit $a$ und endet mit $b$. Konjugiere mit $a^k$ für k groß genug


\subsubsection{3.19 Beispiel}
freie UG von SL(2,Z)

\[SL(2,\Z) = \set{\begin{matrix}
	a & b\\
	c & d\\
	\end{matrix}}{det = 1}\]
Dann ist $G:= \grp{M_1, M_2}{}$ frei vom Rang 2, wobei 
\[M_1 =\begin{matrix}
1 & 2\\
0 & 1
\end{matrix}, M_1 =\begin{matrix}
1 & 0\\
2 & 1
\end{matrix}\]

\paragraph{Beweis}
Betrachte lineare Wirkung von $SL(2,\R) \curvearrowright \R^2$ definiert durch
\[(M, (x,y)) \longmapsto M.(x,y)\]
$\forall n \in \Z - 0$ und $(x,y) \in \R^2$: $M_1^n.(x,y) = (x+2ny,y)$

Sei $A = \set{(x,y)}{|x| > |y|}$, 
 $B = \set{(x,y)}{|y| > |x|}$,  $B \not \subset A$

Dann $|x + 2ny| \geq |2ny| - |x| > |2y| - |y| = |y|$, also $M_1^nB \subset A$, analog für $M_2$.

3.18 zeigt: G frei.

% 19.05.15, 11. VL
% Prüfungszeiträume: 27.7 - 30.7 & 8. - 11.9
% Listen vor Büro an Pinnwand: 1.005

\paragraph{Motivation}
Gruppe -> Geometrie
Ziel: Konzept finden, welches Cayleygraphen einer festgelegten Gruppe als gleich (äquivalent) auffasst

\subsection{Ein paar Definitionen}
Seien $(X,d),(Y,d)$ metrische Räume, $f:X\rightarrow Y$ eine stetige Abbildung.
\begin{itemize}
	\item $f$ heißt eine \df{isometrische Einbettung}, falls für alle $x,y \in X$ gilt
	\[d(f(x),f(y) = d(x,y) \]
	\item $f$ heißt eine \df{Isometrie}, falls $f$ eine surjektive isometrische Einbettung ist.
	\item $X$ und $Y$ heißen isometrisch, falls eine Isometrie $X\rightarrow Y$ existiert.
	\item $f$ heißt eine \df{Bilipschitz-Einbettung}, falls eine reelle Konstante $c \geq 1$ existiert, sodass für alle $x,y \in X$ gilt
	\[ \frac{1}{c} d(x,y) \leq  d(f(x),f(y) \leq c d(x,y) \]
	\item $f$ heißt eine \df{Bilipschitz-Äquivalenz}, falls $f$ eine surjektive Bilipschitz-Einbettung ist.
\end{itemize}

\subsection{Bemerkung 4.4}
\begin{itemize}
	\item Isometrie -> Bil.Äqu -> QI
	\item Umkehrung i.A. nicht richtig
	\item Quasi-Isometrisch sind $(R,d)$ und $(Z, d)$ und $(2Z, d)$ mit den euklidischen Metriken.
	Die Inklusionen sind quasi-isom. Einbettungen, aber keine Bilipschitzäqu., weiter sind
	\begin{align*}
	f:\R \longrightarrow \Z && x \longmapsto \lfloor x\rfloor\\
	g: \Z \longrightarrow 2\Z && x \longmapsto \{x, x -1\}\cap 2\Z
	\end{align*}
\end{itemize}
\subsection{Quiz 4.5}
\begin{itemize}
	\item Sind $\Z$ und $2\Z$ bilipschitz-äquivalent?
\end{itemize}

\subsection{4.6 Durchmesser metrischer Räume}
Jeder nichtleere, metrische Raum $(X,d)$ mit endlichen Durchmessern
\[diam(X) := \sup_{x,y\in X}(d(x,y)) \]
ist quasi-isometrisch zu einem Punkt.
\paragraph{Beweis}
Setze $D:= diam(X)$, sei $* \in X$ beliebig. definiere die Abbildung
\[f:X \longrightarrow X, x \longmapsto *\]
Dann gilt
\[d(f(x), f(y)) - D \leq d(f(x),f(y)) \leq d(f(x),f(y)) + D \]
Daraus folgt auch, dass $d(f^2(x), id(x)) \leq D$, ergo sind $X$ und $*$ quasi-isometrisch.
\qed
\subsection{Korollar}
Ist $X$ beschränkt und $Y$ quasi-isom. zu $X$, so ist auch $Y$ beschränkt.

\subsection{4.17 Satz}
$X,Y$ metrische Räume,$f:X\rightarrow Y$ eine quasi-isometrische Einbettung. Dann gilt:
\[f \text{ Quasi-Isometrie }\Longleftrightarrow f \text{ hat quasi-dichtes Bild in }Y \]
d.h. $f(X) \subset Y$ ist $\delta$-dicht für $\delta \geq 0$, d.h.
\[\forall y\in Y, \exists x \in X: d(y,f(x)) \leq \delta \]

\paragraph{Beweis}
$f$ Quasi-Isometrie, dann existiert quasi-Inverse $g : Y \rightarrow X$ und somit $\delta > 0$, s.d. $\forall y \in Y$ gilt
\[d((f\circ g)(y), y) \leq \delta \]
ergo quasi-Dichtes Bild.

Andere Richtung: $f$ sei (C,D)-q.i.Einbettung mit $\delta$-dichtem Bild, wir konstruieren quasi-Inverse via Auswahlaxiom\\
Setze $\lambda := \max\{C,D,\delta\} \geq 1$, dann gilt
\begin{itemize}
	\item $\forall x,y \in X:  \frac{1}{\lambda} d(x,y) - \lambda \leq d(f(x),f(y) \leq {\lambda} d(x,y) + \lambda  $
	\item $\forall y \in Y \exists x \in X: d(f(x),y) \leq \lambda$
\end{itemize}
Setze $g:Y\rightarrow X, y \longmapsto x_\lambda $; wähle $x_\lambda$ so, dass $d(f(x_\lambda), y) \leq \lambda$.\\

Zu Zeigen: $g$ ist quasi-invers zu $f$.\\
\[\forall y\in Y: d(f(g(y)), id(y)) = d(f(x_\lambda), y) \leq \lambda \]
\[\forall x\in X: d(g(f(x)), id(x)) = d(x_{f(x)}, x) \leq \lambda \cdot d(f(x_{f(x)}), f(x)) + \lambda^2 \leq 2\lambda^2 \]

Noch zu zeigen: $g$ ist quasi-isometrische Einbettung\\

Seien dazu $y,y' \in Y$
\[d(g(y),g(y')) = d(x_y, x_{y'}) \leq \lambda d(f(x_y), f(x_{y'})) + \lambda^2 \]
\[\leq \lambda \left( d(f(x_y),y) + d(y, y') + d(y', f(x_{y'})) \right) + \lambda^2\]
\[\leq \lambda^2 + \lambda d(y, y') + \lambda^2 + \lambda^2 \]

Setze $C = \lambda, D = 3\lambda^2$\\

Für $y,y' \in Y$ ist noch zu zeigen
\[d(g(y),g(y') \geq \frac{1}{C} d(y,y') - D \]

%% 13. VL, 26.05.15

\subsubsection{4.18 Definition: Geodäten}
Eine \df{Geodäte} ist ein eine isometrische Einbettung $\gamma : [0,L] \rightarrow X$ eines Intervalls in einen metrischen Raum.

\subsubsection{4.20 Definition Quasigeodäte}
Eine $(C,D)$-\df{Quasigeodäte} für $C\geq 1, D \geq 0$ ist eine $(C,D)$-Quasiisometrische Einbettung von $[0,L]$ nach $X$.

$X$ heißt $(C,D)$-quasigeodätisch, falls $\forall x,y \in X$ eine verbindende Quasigeodäte \[\gamma :[0,d(x,y)] \rightarrow X \] existiert.

\subsubsection{4.22 Satz von Schwarz-Milner}
$G$ Gruppe, $X$ metr. Raum, $G\curvearrowright X$ durch Isometrien.
Weiter gelte:
$X$ quasi-geod. für $(C,D)$ mit $D > 0$
$\exists B \subset X$ beschränkt mit $\bigcup_{g\in G}gB = X$
$S:=\set{g\in G}{gB' \cap B' \neq \emptyset}$ ist endlich mit $B' := \set{x\in X}{\exists y \in B: d(x,y) \leq 2D}$

Dann gilt:
$G$ wird von $S$ erzeugt
$\forall x \in X$ ist $(G,d_S) \rightarrow (X,d); g \mapsto g.x$ eine quasi-Isometrie.

\paragraph{Beweis}
ZZ: $S$ erzeugt $G$

Sei $g\in G$, $x\in B$. Dann existiert $(C,D)$-Quasigeodäte von $x$ nach $g.x$, $\gamma: [0,d(x,g.x)] \rightarrow X$.

Setze $n := \lceil\frac{CL}{D}\rceil$ und für alle $j = 0,\ldots, n-1$
Setzte $t_j = \frac{jD}{C}$ und $t_n := L$
$x_j := \gamma(t_j)$ für $j = 0,\ldots,n$

Die Translate von $B$ unter $G$ überdecken $X$, also existiert für alle $x_j$ ein $g_j$, s.d. $x_j \in g_j.B$, $g_0 = 1$, $g_n \in g$

Beh.: $\forall j = 1,\ldots, n$ ist $s_j := g_{j-1}^{-1}g_j \in S$
Bew.: $\gamma$ Quasi-Geodäte
$d(x_{j-1},x_j) \leq C |t_{j-1}-t_j| + D
\leq C \frac{D}{C} + D
= 2D$
also $x_j \in B_{2D}(g_{j-1}.B) \overset{G \curvearrowright X isom.}{=} g_{j-1}.B_{2D}(B) = g_{j-1}.B'$
andererseits ist $x_j \in g_j.B \subset g_j.B'$
also $g_j.B \cap g_{j-1}.B'\neq \emptyset$
also $g_{j-1}^{-1}g_j \in S$
\qed

Also $g = g_n = g_{n-1}(g_{n-1}^{-1}g_n) = g_{n-1}s_n = g_{n-2}(g_{n-2}^{-1}g_{n-1})s_n = s_1...s_n \in \grp{S}{}_G$


ZZ. $G\sim_{QI}X$:

Wir zeigen $\forall x \in X:$ $\phi: G \rightarrow X, g \mapsto g.x$ quasi-isom. Einbettung mit quasi-dichtem Bild.

OE: $x \in B$, weil $\bigcup_{g\in G}g.B = X$ und $G \curvearrowright X$ isom., sonst ersetze $B$ durch passendes Translat.

Sei $x'\in X$. Dann gibt es $g\in G$ mit $x' \in g.B$
$d(x', \phi(g)) = d(x', gx) \leq diam(gB) = diam(B) = \delta$
$\Longrightarrow \delta$-dichtes Bild

Noch ZZ: qi. Einbettung

Betrachte $(C,D)$-quasi-geodäte $\gamma : [0,L] \rightarrow X$ von $x$ nach $g.x$
Dann gilt
$d(\phi(e), \phi(g)) = d(x, g.x) = d(\gamma(0), \gamma(L)) \geq \frac{L}{C} - D
\geq \frac{1}{C}(\frac{D(n-1)}{C}D)
=\frac{D}{C^2}n - \frac{D}{C^2} - D
\geq \frac{D}{C^2} d_S(e,g) - (\frac{D}{C^2}+D)$

Abschätzung nach oben:
Setze $n = d_S(e,g)$
\[d(\phi(e), \phi(g))
= d(x,g.x)
\leq d(x,s_1.x) + d(s_1.x, s_1s_2.x) + ... + d(s_1...s_{n-1}.x, g.x)
\overset{G wirkt isom.}{=}
d(x,s_1.x) + d(x, s_2.x) + ... + s(x,s_n.x)
\leq 2n diam(B')
\leq 2n (diam(B)4D)
= d_S(e,g) 2 (diam B + 4D)
\]
wähle für $(C_0, D_0)$-qi Einbettung die Konstanten
$C_0 = \max\set{C^2/D, 2(...)}{}$
$D_0 = D/C^2 + D$
allgemeiner Fall folgt aus der Linksinvarianz von $d$ und $d_S$.
\qed

%% 14.VL, 29.05.15

\subsection{4.23 Definition}
Ein metrischer Raum $X$ heißt \df{eigentlich}, falls alle abgeschlossene Bälle von endlichem Radius kompakt sind.\\
Eine Wirkung $G\curvearrowright X$ ist \df{eigentlich}, wenn für alle kompakten Teilmengen $K \subset X$, die Menge
\[\set{g\in G}{g.K \cap K \neq \emptyset}\]
endlich ist.\\
Manchmal sagt man auch \df{eigentlich diskontinuierlich}.

\subsection{Bemerkung}
$f$ eigentlich, wenn Urbilder kompakter Mengen wieder kompakt sind.\\
Hier $G\curvearrowright X$ eigentlich
\[\Longleftrightarrow G\times X \longrightarrow X \]
\[(g,x) \longmapsto g.x\]
ist eigentliche Abbildung. (Wobei man auf $G$ die diskrete Topologie betrachtet.)

\subsection{4.24 Beispiel}
\begin{itemize}
	\item $\Z \curvearrowright \R$ via Translation ist eigentlich.
	\item $G\curvearrowright X$ eigentlich $\Longrightarrow$ $Stab_G(x)$ ist endlich für $x \in X$, d.h. $G$-Bahnen haben keinen Häufungspunkt
	\item $\Z \curvearrowright \R^2$ Rotation um Ursprung um Winkel mal $z$\\
	$(0,0)$ ist Fixpunkt, also kann diese Wirkung nicht eigentlich sein.
	\item $\Z \curvearrowright S^1$ via Rotation um $\alpha$ ist nicht eigentlich, da $S^1$ kompakt.
	\item $unendliche Gruppe \curvearrowright kompakter Raum$ ist nicht eigentlich
	\item $G$ erzeugt von $S$, $|S| < \infty$, dann ist $G \curvearrowright Cay(G,S) =: \Gamma$ eigentlich.
	\paragraph{Beweis}
	$K \subset \Gamma$ kompakt $\Longrightarrow$ $diam(K)< \infty$ $\Longrightarrow$ $\forall g\in G$ mit $d_S(e,g) = |g|_S > diam(K)$ gilt: $K\cap g.K = \emptyset$, sonst $\exists x \in K \cap g.K \Longrightarrow x \in K$ und $g^{-1}.x\in K$ mit $d_S(x,g^{-1}.x) = |g^{-1}|_S = |g|_S$ ein Widerspruch\\
	Insbesondere nur endlich viele $g$ mit $|g|_S \leq D$. \qed
\end{itemize}

\subsection{4.25 Erinnerung} $X$ topologischer Raum
\begin{itemize}
	\item $X$ \df{hausdorffsch}, g.d.w.
	\[\forall x \in X \exists U_x \subset_O, x \in U_x, U_y \subset_O, y \in U_y: U_x \cap U_y = \emptyset  \]
	\item $X$ \df{lokal kompakt}, g.d.w.
	Für alle $x \in X$ enthält jede offene Umgebung von $x$ eine kompakte Umgebung von $x$.
	\item $X$ metrischer Raum $\Longrightarrow$ hausdorffsch
	\item eigentliche metrische Räume $\Longrightarrow$ lokal kompakt
\end{itemize}

\subsection{4.26 Bemerkung/Lemma: Quotientenräume}

$(X,d)$ metrischer Raum, eigentlich
\[\alpha : G \rightarrow Isom(X) \text{ Wirkung von G auf X} \]
\[p : X \rightarrow X/G \text{ natürliche Projektion auf Quotienten} \]
Setze $f(x,y) := \inf\{d(x,y) | p(x) = x, p(y) = y\}$ für $x,y \in X/G$\\
Dann gilt:
\begin{enumerate}
	\item $inf = min$, d.h. $\exists x,y \in X: f(x,y) = d(x,y) \forall x,y \in X/G$
	\item $f$ ist Metrik auf $X/G$
\end{enumerate}

\paragraph{Beweis}
Seien $z,w \in X/G$, $x = p^{-1}(w)$; setze $R = f(z,w)$\\
Annahme: inf $\neq$ min\\
Dann existieren unendliche Folgen $(x_n,y_n)$ mit $d(x_n,y_n) \rightarrow R$ und $p(x_n) = w, p(y_n) = z$.\\
Weil $p(x_n) = p(x)$ gilt: $\exists h_n \in G$ mit $h_nx_n = x$\\
$\Longrightarrow$ $d(h_nx_n, h_ny_n) = d(x_n,y_n)$, da $\alpha$ isom.\\
daraus folgt $x_n$ kann durch konstante Folge $x$ und $y_n$ durch $y_nh_n$ ersetzt werden.\\
Daraus folgt $y_n \in B_{R+\epsilon}(x_n)$, $p(y_n) = z$\\
Weil $B_{R+ \epsilon}(x)$ kompakt ist, hat $(y_n)_n$ einen HP in $B_{R+ \epsilon}(x)$. Widerspruch zu 4.24\\
\\


$f$ nichtneg. und symmetrisch, da $d$ so.\\
$f(z,w) = 0 \Longrightarrow \exists x,y: d(x,y) = 0 \Longrightarrow x = y \Longrightarrow z = w$\\
Dreiecksungleichung: $u,v,w \in X/G$, wähle $x, y \in X$, s.d. $d(x,y) = f(u,v)$, $p(x) = u, p(y) = v$.\\
Wähle $y_1$ mit $d(x,y_1) = f(u,v)$, $p(y_1) = v$; $y_2$, $p(y_2) = v$ und $d(z,y_2) = f(v,w)$\\
weil $y_1, y_2 \in p^{-1}(v)$ existiert $g$ mit $g.y_2 =y_1$\\
$\Longrightarrow f(u,v) + f(v,w) = d(x,y_1)+d(y_2,z) = d(x,g.y_2) + d(g.y_2, g.z) \geq d(x,g.z) \geq f(u,w)$
\qed


\subsection{4.27 Definition}
Eine Gruppenwirkung $G\curvearrowright X$ heißt kokompakt, wenn $X/G$ kompakt.\\
Betrachte auf $X/G$ Topologie, die durch Quotientenmetrik $f$ induziert wird, wenn wir mit metrischen Raum gestartet sind.

\subsection{4.28 Beispiele}
\begin{itemize}
	\item $\Z \curvearrowright \R^2$ durch Translation längs $x$-Achse.\\
	$\R^2/\Z$ = Zylinder ist nicht kompakt, also keien kokompakte Wirkung.
	
	\item $X$ kompakt, wegzusammenhängend top. Raum, $\widetilde{X}$ universelle Überlagerung.\\
	$\pi_1(X) \curvearrowright \widetilde{X}$ durch Decktransformationen ist kokompakt und eigentlich\\
	$X = \widetilde{X}/ \pi_1(X)$
	
	\item $G \curvearrowright Cay(G,S) =: X$ mit kombinatorischer Metrik\\
	$n := |S|$, $X/G = R_n$, Rose mit n Blättern, kompakt
\end{itemize}

\subsection{4.29 (topologischer) Satz von Schwarz-Milner}
$G$ wirke eigentlich, kokompakt, durch Isometrien auf einen nichtleeren, eigentlichen, geodätischen metrischen Raum $(X,d)$
, dann gilt $G$ endlich erzeugt und für alle $x \in X$ ist
\[G \longrightarrow X, g \longmapsto g.x\]
eine Quasi-Isometrie.\\
Wenn $G\curvearrowright X$ eigentlich, kokompakt und durch Isometrien, so sagt man auch $G$ wirkt \df{geometrisch}.

\paragraph{Beweis}
Suche $B$.\\
\begin{itemize}
	\item nach Vorr. ist $X$ $\forall \epsilon > 0$, $(1, \epsilon)$-quasi-geodätisch.
	\item Sei für bel. $x_0 \in X$: $B:=\set{x \in X}{d(x,x_0) \leq D}$;\\
	$D:= diam(X/G) < \infty$, da $G\curvearrowright X$ kokompakt.
\end{itemize}
Dann gilt: $\bigcup_{g\in G}g.B= X$, $B' := B_{2\epsilon} (B)$ endlicher Radius, also kompakt, da $X$ eigentlich.\\
$G \curvearrowright X$ eigentlich, also $\set{g \in G}{g.B' \cap B' \neq \emptyset}$ endlich.

4.22 zeigt Beh.\qed

\subsection{Korollar}
Sei $H < G$, $G$ endlich erzeugt mit $(G: H )< \infty$.
Dann ist $H$ endlich erzeugt und quasi-isom. zu $G$.

\paragraph{Bew:}
$S$ sei endl. EZS von $G$\\
$\Longrightarrow$ $H \curvearrowright Cay(G,S) =: \Gamma$ mit Wortmetrik $d_S$ isom., eigentlich, kokompakt.\\
Sei $B$ endliches Vertretersystem von $G/H$, existiert, weil Anzahl Nebenklassen von $H$ in $G$ endlich ist.\\
Dann ist $HB = G$

$B' := B_2(B)$ endlich, $\set{h\in H}{h.B'\cap B' \neq \emptyset}$ endlich.\\
Schwarz-Milner: $H$ endlich erzeugt und $H\sim_{qi} \gamma \sim_{qi} G$\qed

%% 2.6.15, 15.Vl

\subsection{4.31 Definition}
\begin{enumerate}
	\item Zwei Gruppen $G,H$ heißen \df{kommensurabel}, wenn es Untergruppen $G'<G, H<H'$ mit endlichem Index gibt, s.d. $G' \cong H'$.
	\item Zwei Gruppen $G,H$ heißen \df{schwach kommensurabel}, wenn es Untergruppen $G'<G, H<H'$ mit endlichem Index gibt, s.d. normale Untergruppen $N \vartriangleleft H', M \vartriangleleft G'$ mit 
	\[H'/N \cong G'/M \]
\end{enumerate}

\subsection{Bemerkung}
$\sim_C, \sim_{WC}$ sind ÄQ (kommensurabel, schwach ...)
$G\sim_C H \Longrightarrow G\sim_{QI}H (falls G endlich erzeugt)$

\subsection{Korollar}
Sei $G$ eine Gruppe und 
\begin{enumerate}
	\item $G' < G$ eine UG mit endlichem Index. Dann gilt:
	\[G' \text{ endlich erzeugt } \Longleftrightarrow G \text{ endlich erzeugt} \]
	Falls $G,G'$ endlich erzeugt, dann $G \sim_{QI} G'$
	\item $N \vartriangleleft G$ ein endliche normale Untergruppe. Dann gilt:
	\[G/ N \text{ endlich erzeugt } \Longleftrightarrow \text{G endlich erzeugt} \]
	Falls $G,N$ endlich erzeugt, dann $G/N \sim_{QI} G$
\end{enumerate}
Insbesondere: Ist G endl. erz. und $H \sim_WC G$, dann ist $H$ endlich erzeugt und $G\sim_QI H$
\subsection{Bemerkung}
Man kann zeigen, dass nicht alle qi Gruppen kommensurabel sind. Z.Bsp.: $(F_3\times F_3)* F_3 \sim_QI (F_3\times F_3)*F_4$, aber die Gruppen sind nicht kommensurabel (Eulercharakteristik)

\subsection{4.33 Korollar}
Sei $M$ eine kompakte Mannigfaltigkeit ohne Rand mit Riemannscher Metrik und $M'$ die Riem. universelle Überlagerung. Dann gilt:
\begin{enumerate}
	\item $\pi_1(M)$ endl. erz.
	\item $\forall c \in M'$ ist $\pi_1(M) \rightarrow M'$, $g\longmapsto g.x$ eine QI
\end{enumerate}
\paragraph{Beweis}
Zeige mit Standard-Argumenten der Geometrie und alg. Topo, dass M' eig. und geod.
$\pi(M) \curvearrowright M'$ eig., kokompakt und durch Isom.
\section{Quasi-Isometrie-Invarianten}
\subsection{Definition}
Sei V eine menge. Eine \df{QI-Invariante} mit Werten in V ist eine Abb.
\[I:X \longrightarrow V\]
$X \subset \set{G: Gruppe}{G endl. erz}$, s.d. gilt
\[G\sim_QI H \Longrightarrow I(G) = I(H)\]
\subsection{Bemerkung}
\begin{enumerate}
	\item QI-Invarianten sind hilfreich, um $G \not\sim_QI H $ zu zeigen
	\item i.A. ist es nicht möglich zu entscheiden, ob $G \sim_QI H$ gilt
\end{enumerate}
\subsection{Beispiel}
\begin{enumerate}
	\item $V = \{1\}$, dann keine Infos
	\item $V = \{0,1\}$, $I(G) = 1, G unendl., sonst 0$ ist QIInv.
	\item $V = \N$, $I(F) = rang F$, $F$ endl. erz. freie Gruppe, ist keine QIInv., weil $F_n\sim_QI F_m$ für $n,m \geq 2$
\end{enumerate}
\subsection{Definition}
Eine Eigenschaft $P$ von endl. erz. Gruppen heißt \df{geometrisch}, wenn gilt: 
G hat P und H qi G, dann H hat P

\subsection{Beispiel}
\begin{enumerate}
	\item $\forall n \in \N$ ist die Eigenschaft \df{virtuell $\Z^n$} zu sein eine geom. ES.
	\item \df{endlich sein} ist geometrisch.
	\item \df{endlich erzeugt und virtuell frei} ist geometrisch ES.
	\item \df{abelsch} ist kein geom. ES.
\end{enumerate}
1 bis 3 ist schwer zu beweisen, wir zeigen:
\begin{enumerate}
	\item \df{endlich präsentiert} ist geom. ES.
	\item Wachstum von Gruppen liefert geom. ES.
	\item einige Ränder/Enden von einigen Gruppen liefert geom. ES.
\end{enumerate}

\subsection{Einschub: Simplizialkomplexe und CW-Komplexe}
\paragraph{Definition} Ein (abstrakter) \df{Simplizialkomplex} $\Delta$ ist eine Menge von Teilmengen einer Menge $V$, s.d. gilt:
\begin{enumerate}
	\item $\{v\} \in \Delta$ für alle $v \in V$
	\item $\emptyset\neq A \subset B \in \Delta \Longrightarrow A \in \Delta$
\end{enumerate}
Dimension von $a \in \Delta$ ist $dim(a) := |a| -1$
Dimension von $\Delta$ ist $dim(\Delta) = \sup_{a\in A}dim(a)$
Schreibe: $a$ ist $K$-Simplex, falls $dim(a) = K$

\paragraph{Beispiel}
\begin{enumerate}
	\item $V = \{1,2,3\}, \Delta = \{\{1\}, \{2\}, \{3\}, \{1,2\}, \{1,3\}\}$  ist Simplizialkomplex für V
	\item $V = \{1,2,3\}, \Delta = \{\{1\}, \{2\}, \{1,2\}\}$  ist kein Simplizialkomplex für V
	\item ungerichtete, einfache Graphen sind Simplizialkomplexe
	\item $V$ Menge, $\Delta = P(V)-\{\emptyset\} =: \grp{V}{} $ ist Simplizialkomplex;
\end{enumerate}

\paragraph{Allgemeiner: CW-Komplexe}
Ein CW-Komplex ist ein top. Raum, der schrittweise aus sog. Zellen zusammengeklebt worden ist. 
\paragraph{Definition}
Sei $X^{(0)}\subset \R^n$ eine diskrete Menge, diese Menge besteht aus den sogenannten \df{0-Zellen}.\\
Das \df{$n$-Skelett} $X^{(n)}$ entsteht aus den $X^{(n-1)}$ durch Ankleben von $n$-Zellen $D^n_i$ durch stetige Abb.
\[\phi_i : S^{n-1} = \partial D_n \longrightarrow X^{(n-1)} \]

Formal:
\[X^{(n)} = X^{(n-1)} \cup \bigcup_{i\in I}D^n_i/\sim \]
wobei $x\sim \phi_i(x)$ für $x \in \partial D_i^n$

Definiere den CW-Komplex durch $X = \bigcup_{n\geq 0} X^{(n)}$.

\paragraph{Beispiele}
\begin{enumerate}
	\item Graphen mit Doppelkanten sind CW-Komplexe
\end{enumerate}

%% 16.VL, 5.6.15

\subsubsection{Definition}
$G,H$ schwach kommensurabel, falls $\exists$
\[N \vartriangleleft G' \leq G\]
\[M \vartriangleleft H' \leq H\]
wobei $N,M, (G':G), (H':H)$ endlich sind.

\subsubsection{Satz 5.5}
$G$ endlich erzeugt von $S$ mit Relationen $R$, $R$ endlich. Sei $H$ endlich erzeugte Gruppe von $S'$ und $H\sim_{QI}G$, dann gilt:
$H$ ist endlich präsentiert und es existiert eine endliche Menge $R'$ von Relationen, s.d.
\[H = \grp{S'}{R'}\]
\paragraph{Idee}
Baue 2-dim. CW-Komplex, der die Darstellung kodiert (aufbauend auf Cayleygraphen).
\paragraph{Erinnerung}
$G = \grp{S}{R} = F(S)/ \grp{R}{}_G\vartriangleleft$\\
$\exists \pi: F(S) \rightarrow > \grp{S}{R}, kern \pi = \grp{R}{}_G\vartriangleleft$ 

\subsection{Definition 5.6: Präsentationskomplex}
OE: $1 \in S, G \cong \grp{S}{R}$ endlich präsentiert.
\[\Gamma := Cay(G,S)/\sim \]
wobei zwei Kanten $e, e'$ verklebt werden (äquiv. sind), wenn gilt $\delta(e) = \delta(e')$\\\\

Der \df{Präsentations(zwei)komplex} $K = K(S,R)$ von $G$ ist der Quotient $K'/G$ von folgendem 2-Komplex $K'$:\\
1-Skelett von $K'$ ist $\Gamma$\\
$\forall $ Kreise $\gamma$ in $\Gamma$ der Form $\gamma = g^{-1}.(1,s_1, s_1s_2, \ldots, s_1\cdots s_n)$ wobei $g \in G, s_1\cdots s_n \in R$; klebe 2-Zelle an $\gamma$ um $K'$ zu erhalten.\\
$K'$ heißt \df{Cayley-Komplex} von $\grp{S}{R}$

\paragraph{Bemerkung}
Man kann mittels Seifert-Van Kampen zeigen, dass $K'$ einfach zusammenhängend. $K'$ ist univ. Überlagerung und $G = \pi(K) = \pi(K'/G)$

\subsection{Beispiel 5.8}
\begin{enumerate}
	\item $G = \Z^2 = \grp{a,b}{aba^{-1}b^{-1}}$\\
		$K' = \R^2, K = T^2$
	\item Flächengruppen: $G:=\grp{a,b,c,d}{a^{-1}b^{-1}abc^{-1}d^{-1}cd }$
	$K'$ kann aufgefasst werden als Parkettierung von $H^2$\\
	$K$ ist Torus mit 2 Löchern, $S^2$-Fläche von Geschlecht 2
\end{enumerate}

\subsection{Bemerkung 5.9: alternative Definition von $K(S,R)$}
hier $K_G$, $K_G$ enthält
\begin{enumerate}
	\item eine 0-Zelle $v$
	\item eine 1-Zelle für jedes $s \in S$, die von $v$ nach $v$ führt, orientiere diese 1-Zellen
	\item eine 2-Zelle $d_r \forall r \in R$ verklebt so, dass Kanten $g \rightarrow gs$ orientierungserhaltend verklebt werden über $1 \rightarrow s_1 \rightarrow s_1s_2 \rightarrow \ldots \rightarrow s_1\cdots s_n$, wobei $r = s_1\cdots s_n, s_i \in S\cup S^{-1}$
\end{enumerate}
Man kann zeigen $K_G \cong K(S,R)$ und $K'$ ist univ. Überlagerung von $K_G$

\paragraph{Beweis von 5.5}
Setze $G_1 := G, G_2 := H, S_1 := S, S_2 := S', \Gamma = Cay(G_i, S_i)/\sim$ wie in 5.6.\\

Sei $\rho$ die Länge der längsten Relation in $R$
\begin{itemize}
	\item Cayleykomplex $K'_1$ ist einfach zusammenhängend
	\item Seien $f:\Gamma_2 \rightarrow \Gamma_1, f' : \Gamma_1 \rightarrow \Gamma 2$ $(C,D)$-quasi Isometrien (existieren, da $G \sim_{QI} H$)
\end{itemize}
Sei $\mu > 0$, s.d. $d(f'(f(v)), v) \leq \mu \forall v \in \Gamma_2$\\
Setze $m:= \max \set{\rho, \mu, C,D}{}, M := 3(3m^2 + 5m + 1)$.\\
Sei $K_2'$ 2-Komplex, den man durch Ankleben von 2-Zellen an jeden Kreis der Länge $\leq M$ in $\Gamma_2$ erhält.\\

Sei $l$ Kantenkreis in $\Gamma_2$, d.h. $l = (g_1,\ldots, g_n, g_1)$\\
Betrachte $l$ als Abb. $\partial D \rightarrow \Gamma_2$, $D$ ist hier eine 2-Zelle.\\

\paragraph{Zwischenlemma (Formalisierung der Bemerkung 5.7.2)}
$G$ erzeugt von $S$, $R \leq Kern \pi$, $\pi : F(S) \rightarrow G$; $X$ Komplex den man, durch Ankleben von 2-Zellen an Kantenkreisen geg. durch Wörtern in $R$ an $Cay(G,S)/\sim$ erhält. Dann gilt:
\[X \text{ einfach zusammenhängend } \Longleftrightarrow \grp{R}{}_G^\vartriangleleft = kern(\pi)  \]
\paragraph{Beweis von Zwischenlemma:} Lemma 8.9 in Bridson-Haefliger, S.135


Wir sind fertig, wenn wir zeigen können:\\
$l$ besitzt stetige Fortsetzung $l' : D \rightarrow K_2'$, d.h. $K'_2$ einfach zusammenhängend.\\
Seien $v_i$ Urbilder der $g_i$ unter $l$\\
Sei $\phi : \partial D \rightarrow \Gamma_1$ eine Abb., die $v_i$ auf $f(g_i)$ in $\Gamma_1$ und die Kante $\{v_i, v_{i+1}\}$ auf $\partial D$ auf Geodäten von $f(g_i)$ nach $f(g_{i+1})$.\\
$K_1'$ ist einfach zusammenhängend $\Longrightarrow$ $\phi$ erweitert zu $\phi': D \rightarrow K'_1$

\begin{itemize}
	\item $\forall x \in D$ definiere Elemente $h_x$ in $V(\Gamma_1) = G$ wie folgt:
	\begin{itemize}
		\item ist $\phi'(x)$ Ecke, so ist $h_x = \phi'(x)$
		\item ist $\phi'(x)$ in einer offenen Kante oder offenen 2-Zelle enthalten, so wähle nächste Ecke der Kante / 2-Zelle als $h_x$ 
	\end{itemize}
	Weil $\phi'$ stetig ist, ist $d(h_x, h_y) \leq \rho \forall x,y$, wenn $x,y$ nah genug aneinander sind in $D$.\\
	Es gilt $d(\phi(x), h_x) \leq \frac{1}{2} \forall x \in \partial D$ (alle Kanten in $\partial D$ haben Länge 1).
	\item Trianguliere $D$ so, dass $v_i \in \partial D$ wieder Ecken von $T$ sind und $\forall$ benachbarten $t,t' \in T$ gilt:
	\[d(h_t, h_{t'}) \leq \rho \]
	Metrik auf $D$ dazu so gewählt, dass $D$ reguläres $M$-Polygon in $R^2$ ist
	\item Setze $l'_{|\partial D} = l$ und $l'(x) = f'(h_x) \forall x \in D^o$
	\paragraph{Behauptung}
	Für alle benachbarten Ecken $t,t'$ in der Triangulierung $T$ gilt:
	\[d(t,t') \leq M/3 \]
	Gilt diese Behauptung, so erweitert $l'$ auf $D$ so, dass Kanten in $T$ auf Geodäten in $\Gamma_2$ geschickt werden und nach Konstruktion Kreise der Länge $\leq M$ eine 2-Zelle beranden. Daraus würde folgen, dass $l'$ eine stetige Fortsetzung wäre.
	\paragraph{Bew. Beh.:} einziger interessanter Fall: $t\in D^o, t' \in \partial D$. Sei $t'$ zwischen $v_i$ und $v_{i+1}$. Es gilt:
	\[d(l'(t), l'(t')) = d(f'(h_t), l(t')) \overset{ganz viele \triangle-Ugl.en}{\leq} d(f'(h_t), f'(h_t))  + d(f'(h_{t'}) , f'(\phi(t'))) + d(f'(\phi(t')) + f'(\phi(v_i)) ) + d(f'(\phi(v_i)), l(v_i) ) + d(l(v_i), l(t')) \leq \ldots \leq M/3 \]
\end{itemize}

%% 17.VL, 9.6.15

\section{Hyperbolische Gruppen}
\subsection{Oberes Halbebenenmodell von $\H^2$}
\[\H^2 := \set{z \in \C }{Im z > 0} \]
Riemannsche Struktur:
\[ds^2 = \frac{dx^2 + dy^2}{y^2} \]
hyberbolische Norm für Tangentenvektoren $v \in \T_z\H^2 = \R^2$
\[\norm{v}_{hyp} := \frac{\norm{v}_{eukl}}{im z} \]
direkte Definition einer Metrik auf $\H^2$:\\
Sei $\gamma: [0,1] \rightarrow \H^2 $ glatte Kurve, $\gamma(t) = x(t) + iy(t)$, dann ist die \df{Länge} von $\gamma$ definiert durch
\[L_{hyp}(\gamma) := \int_{0}^{1} \frac{\norm{\gamma'(t)}_{eukl}}{y(t)}dt\]
wir definieren die \df{hyperbolische Metrik} auf $\H^2$
\[d(z,w) := \inf_{\gamma : z \rightarrow w, glatt}L_\H(\gamma) \]

\subsection{Beispiel}
\begin{enumerate}
	\item
	$c : [0,1] \rightarrow \H^2, c(t) = i + (a-1)it$, $a \in \R$
	\[L_\H(c) = \ln(a) \]
	Außerdem gilt für beliebiges $\gamma : [0,1] \rightarrow \R^2$ von $i$ nach $a$
	\[L_\H = \int_0^1\frac{\sqrt{x'(t)^2 + y'(t)^2}}{y(t)}dt \geq \int_0^1\frac{y'(t)}{y(t)} = \ln a\]
	\[\Longrightarrow d(i,a) = \ln a\]
	\item
	$\gamma(t) = ai + t, a > 0, \gamma'(t) = 1, y(t) = a, x(t) =t$
	\[\Longrightarrow L^\H(\gamma) = \frac{1}{a} \]
	\[L(\gamma) \rightarrow 0, a \rightarrow \infty \]
	\[L(\gamma) \rightarrow \infty, a \rightarrow 1 \]
	Insbesondere ist $\gamma$ keine Geodäte.
\end{enumerate}

\subsection{Isometrien}
Isometrien von $\H^2$ sind die Möbiustransformationen. Eine \df{Möbiustransformation} (MT) ist eine Abbildung $\pi : \overline{\C} := \C\cup\{\infty\} \rightarrow \overline{\C}$ definiert durch
\[z \longmapsto \frac{az + b}{cz + d}, a,b,c,d \in \C \]

\subsection{Eigenschaften}
\begin{enumerate}
	\item MT sind dreifach transitiv auf $\overline{\C}$, d.h. sind $(z_1, z_2, z_3),(w_1, w_2, w_3) \in \overline{\C}^3$, dann existiert genau eine MT $T$ mit $T(z_i) = w_i$.
	\item MT bilden Kreise bzw. Geraden auf Kreise bzw. Geraden ab.
	\item $PSL(2,\R) = SL(2,\R)/\pm I $ operiert auf $\H^2$ durch Möbiustransformationen:
	\[A = \left(\begin{matrix}
	a & b \\
	c & d
	\end{matrix}\right) \longmapsto \frac{az + b }{cz + d}=: A.z\]
	\[Im(A.z) = \frac{Im z}{\left|cz + d\right|^2} > 0 \]
\end{enumerate}

\subsection{Satz}
Die Wirkung von $PSL(2,\R) \curvearrowright \H^2$ durch MT ist isometrisch und 
\[PSL(2,\R) \hookrightarrow Isom(\H^2) \]

\paragraph{Beweisskizze:}
\begin{itemize}
	\item Bestimme Erzeuger von $PSL(2, \R)$ (Gaußverfahren)
	\[\set{\Matrix{1}{r}{0}{1}, \Matrix{0}{-1}{0}{1}, \Matrix{\lambda}{}{}{\frac{1}{\lambda}}}{r \in \R, \lambda \in \R \setminus \{0\}}\]
	für Injektivität:\\
	\begin{itemize}
		\item betrachte: $\Matrix{-1}{}{}{-1} = id_{\H^2}$
		\item $\{I, -I\} \triangleleft SL(\R^2)$
		\item $T_A(z) = z \Longleftrightarrow A = \pm I$
	\end{itemize}	
\end{itemize}
\qed

\subsection{Bemerkung}
\[Isom(\H^2, d_\H) \cong PSL(2,\R) \cup \sigma \cdot PSL(\R^2) \]
wobei $\sigma = [z \mapsto -\overline{z}]$

\subsection{Satz: Geodätische}
Geodäten in $\H^2$ sind nach Bogenlänge parametrisierte Halbkreise mit Zentrum auf der $x$-Achse und Halbgeraden parallel zur $y$-Achse. Insbesondere gibt es für je zwei Punkte genau eine Geodätische, die diese verbindet.

\paragraph{Beweis}
Seien $z,w \in \H^2$
\begin{enumerate}
	\item Sei zunächst $z = ia, w= ib, b > a > 0$\\
	Man rechnet nach:
	\[L_\H (\gamma) = \ln(\frac{b}{a}) \]
	Für $\sigma$ gilt:
	\[L_\H (\sigma) = \ln\frac{b}{a} \]
	Ergo ist $\sigma$ Geodäte
	\item $z,w$ beliebig: betrachte 2 Teilfälle
	\begin{enumerate}
		\item $Re (z) \neq Re(w)$: Sei $C$ Kreis um Punkt $P$ auf $x$-Achse, der $z$ und $w$ enthält. ($P$ ist der Schnittpunkt der $x$-Achse und der Orthogonalen der Verbindungsstrecke $zw$, der den Mittelpunkt der Verbindungsstrecke enthält.)
		$(0, t_1)$ und $(0,t_2)$ seien die Randpunkte des Halbkreises $C$, setze dann
		\[g(u) := \frac{u - t_2}{u(t_2 - t_1)- t_1(t_2 - t_1)}\]
		\[g(t_1) = \infty, g(t_2) = 0\]
		Weiter bildet $g$ den Kreis $C$ auf die imaginäre Achse ab. (Nachrechnen mit Halbkreisparam.)
		
		\item $Re (z) = Re(w)$:\\
		Mit dreifach-Transitivität existiert MT mit $g(w) = w', g(z) = z'$ und $Re w = Re w', Im w' =0, Re z' = Re z, Im z' 0$. Nachrechnen: senkrechte Kurve $z \rightarrow w$ wird auf imaginäre Achse abbildet.
	\end{enumerate}
\end{enumerate}
\qed

\subsection{Bemerkung}
hyperbolische Kreise $S_{r,p} := \set{w\in \H^2}{d(w,p) = r}$ sehen exzentrisch aus

\subsection{Lemma}
Zu jeder Geodäten $\gamma$ und $\forall z \notin \gamma$ gibt es unendlich viele Geodäten $\sigma$ mit: $z \in \sigma $ und $\sigma || \gamma$, wobei 
\[\sigma || \gamma : \Leftrightarrow \sigma \cap \gamma = \emptyset \]


\subsection{Satz 6.9: Dreiecke sind dünn}
Jedes hyperbolische Dreieck hat Innenkreisradius $\leq \frac{1}{2} \ln 3$

\paragraph{Beweisskizze}
Sei ein Dreieck in $\H^2$ gegeben (die einzelnen Seiten sind Strecken von Geodäten). In diesem gibt es einen Hyperbolischen Kreis mit maximalen Radius.\\
Die beiden Seiten rechts und links, werden zu Geraden erweitert, wodurch der Innenkreisradius größer wird.\\
Es existiert eine MT, die die drei Ecken des neuen Dreiecks $p,q, \infty$ auf $-1, 1, \infty$ abbildet.\\
Das dadurch erhaltene Dreieck hat einen Innenkreisradius von $\leq \frac{1}{2} \ln 3$

\end{document}