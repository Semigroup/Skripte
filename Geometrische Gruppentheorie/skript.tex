\documentclass{article}

\usepackage{../Package/latexa}
\usepackage{../Package/algebra}
\usepackage{../Package/theorema}
\usepackage{../Package/diagramma}
\usepackage{../Package/categoria}

\newcommand{\qi}{\backsimeq_{\textsc{QI}}}
\newcommand{\ba}{\backsimeq_{\textsc{BA}}}
\newcommand{\normal}{\vartriangleleft}
\newcommand{\tm}{\subset}
\newcommand{\Stab}[2]{\textsf{Stab}_{#1}(#2)}
\newcommand{\Cay}[2]{\textsf{Cay}(#1,#2)}

\renewcommand{\C}{\mathcal{C}}
\newcommand{\Cz}{\mathbb{C}}

\begin{document}

\tableofcontents

\newpage
\section{Gruppen und Graphen}
\Def{Gruppe}
\Bsp{Gruppen}
\Def{Direktes Gruppenprodukt}
\Def{Homomorphismus}
\Bsp{Homomorphismen}
\Satz{Satz von Cayley}
Jede Gruppe ist Untergruppe einer symmetrischen Gruppe.
\begin{Beweis}{}
Ist $G$ eine Gruppe, so existiert folgender Monomorphismus
\Func{G}{\Sym{G}}{g}{\func{x}{gx}}
\end{Beweis}
\Def{Normale Untergruppe}
\Bsp{Normale Untergruppen}
\Satz{Faktorgruppe}
Ist $N\normal G $, so ist $G/N$ eine Gruppe mit folgender UAE:\\
Jeder Homomorphismus $\phi : G \pfeil{} H$ mit $\Ker \phi \tm N $ steigt eindeutig zu einem Homomorphismus $\widetilde{\phi} : G/N \pfeil{} H$ ab, der folgendes Diagramm zum Kommutieren bringt:
\begin{center}
\triag{G}{\phi}{H}{\exists_1 \widetilde{\phi}}{G/N}{\pi} 
\end{center}

\Def{Erweiterung}
Sind $Q,N,G$ Gruppen und ist folgende kurze exakte Sequenz gegeben
\kes{N}{G}{Q}
so heißt $G$ eine \df{Erweiterung} von $Q$ durch $N$.

\Def{Semidirektes Produkt}
Sind $Q,N$ Gruppen mit Homomorphismus $\phi : Q \pfeil{} \Aut{}{N}$, so definiere das \df{semidirekte Produkt} $Q\rtimes_\phi N$ von $Q$ mit $N$ durch die Menge $Q\times N$ und die Verknüpfung
\[ (q_1,n_1) \cdot (q_2, n_2) := (q_1q_2, n_1 \phi(q_1) n_2 ) \]
\paragraph{Bemerkung}
Spaltet eine Erweiterung $G$ von $Q$ durch $N$, so ist $G$ das semidirekte Produkt von $Q$ mit $N$.

\Def{Erzeugendensystem}
\Bsp{Erzeugendensysteme}

\Def{Freie Gruppe}
\Bsp{Freie Gruppen}
\Satz{Eindeutigkeit freier Gruppen}
Zwei Gruppen, die über derselben Menge frei erzeugt sind, sind isomorph.
\Satz{Existenz freier Gruppen}
\Kor{}
Ist $S$ ein freies Erzeugendensystem, so gilt
$\F(S) = \grp{S}{}$

\Satz{Rang freier Gruppen}
\begin{itemize}
\item Die freien Erzeugendensystem einer festen freien Gruppe haben dieselbe Kardinalität. Wir nennen diese den \df{Rang} der freien Gruppe.
\item Die Kardinalität jedes Erzeugendensystems einer freien Gruppe ist mindestens so mächtig wie der Rang der Gruppe. 
\end{itemize}

\Def{Freie Gruppe von Rang $n$}
$\F_n := \grp{s_1, \ldots, s_n}{}$
\Bem{Freie Gruppe in zwei Erzeugern}
Die freie Gruppe in zwei Erzeugern enthält alle endlich erzeugten freie Gruppen.

\Satz{}
Eine Gruppe ist genau dann endlich erzeugt, wenn sie Quotient einer freien Gruppe endlichen Ranges ist.

\Def{Reduzierte Wörter}
Sei $S$ eine Menge, dann ist $(S + S\i)^*$ eine Sprache. Ein Wort $s_1\cdots s_n$ dieser Sprache heißt \df{reduziert}, falls
\begin{itemize}
\item $s_i \neq s_{i+1}\i \forall i$
\item $s_i\i \neq s_{i+1} \forall i$
\end{itemize}

\Def{Erzeugte normale Untergruppe}
Sei $R \tm G$ Teilmenge einer Gruppe. Definiere den durch $R$ erzeugten Normalteiler durch
\[\N(R) := \bigcap_{R \tm U \normal G} U \] 

\Def{Darstellung}
Sei $S$ eine Menge, $R \subset (S + S\i)^*$. Definiere die Gruppe, die durch $S$ mit Relatoren $R$ erzeugt wird, durch
\[ \grp{S}{R} := \F(S) / \N(R) \]
Wir nennen $\grp{S}{R}$ eine \df{Darstellung}.

\Prop{Universelle Abbildungseigenschaft von Darstellungen}
Sei eine Darstellung $\grp{S}{R}$ und eine Abbildung $\phi : S \pfeil{} G$ gegeben, sodass $\phi(R) = 1$, so existiert genau ein Pfeil $\phi' :\grp{S}{R} \pfeil{} G$, sodass folgendes Diagramm kommutiert:
\begin{center}
\triag{S}{\phi}{G}{\exists_1 \phi'}{\grp{S}{R}}{}
\end{center}

\Bsp{}

\Bem{}
\begin{itemize}
\item Die Sprache aller endlichen trivialen Darstellungen ist nicht entscheidbar.
\item Die Sprache der trivialen Ausdrücke einer gegebenen Darstellungen ist im Allgemeinem nicht entscheidbar.
\end{itemize}

\Satz{}
Es gibt überabzählbar viele Isomorphieklassen von Gruppen, die von zwei Elementen erzeugt werden.

\Kor{}
Es gibt überabzählbar viele Isomorphieklassen von Gruppen, die endlich erzeugt, aber nicht endlich präsentiert sind.

\Prop{}
Es existiert ein Gruppe, die von zwei Elementen erzeugt wird und überabzählbar viele normale Untergruppen hat.
\begin{Beweis}{}
Betrachte $G = \grp{s,t}{R}$
\[ R = \set{[[s, t^nst^{-n}],s]}{n\in \Z} + \set{[[s, t^nst^{-n}],t]}{n\in \Z} \]
Definiere $C = \grp{[s, t^nst^{-n}]}{n \in \Z}$, dann ist $C$ zentral in $G$, da
\[sxs^{-1} = txt\i = x \]
für alle $x \in C$. Man kann einsehen, dass
\[ C \isom{} \sum_{\Z}\Z \]
D.h. $C$ enthält überabzählbar viele Untergruppen, die, da $C$ zentral ist, normal in $G$ sind.
\end{Beweis}

\Prop{}
Für eine endlich erzeugte Gruppe sind folgende Aussagen äquivalent:
\begin{itemize}
\item Die Gruppe hat überabzählbar viele normale Untergruppen.
\item Die Gruppe hat überabzählbar viele nicht-isomorphe Quotienten.
\end{itemize}
\begin{Beweis}{1 $\impl{}$ 2}
Wir zeigen die Kontraposition: Also habe die Gruppe $G$ höchstens abzählbar viele nicht-isomorphe Quotienten. Da $G$ endlich erzeugt ist, existieren nur abzählbar viele Homomorphismen von $G$ zu einem seiner Quotienten. Ergo gibt es nur abzählbar viele Normalteiler in $G$, die denselben Quotienten induzieren. Ergo hat $G$ nur endlich viele Normalteiler.
\end{Beweis}

\Bsp{Beispiele}

\newpage
\section{Cayleygraphen}
\Def{Graph}
Definiere einen \df{ungerichteten Graphen} durch ein Tripel $\Gamma = (V,E,\delta)$, wobei $V$ eine Knotenmenge, $E$ die Kantenmenge und $\delta : E \rightarrow \P_2(V)$ die Randabbildung ist.\\
Zwei Ecken $u,v \in V$ heißen \df{benachbart}, wenn es eine Kante $e \in E$ mit $\delta(e) = \set{u,v}{}$.\\
Eine \df{Schleife} ist eine Kante $e \in E$ mit $\delta(e) = \set{u}{}$.\\
Eine\df{Doppelkante} ist eine Kante $e \in E$ für die eine Kante $e' \neq e$ existiert, sodass $\delta(e) = \delta(e')$.\\
$\Gamma$ heißt \df{simplizial} bzw. \df{einfach} oder \df{kombinatorisch}, wenn keine Schleifen oder Doppelkanten existieren.

\Bsp{Graphen}

\Def{Graphenmorphismus}
Seien $\Gamma = (V,E, \delta), \Gamma' = (V', E', \delta')$ Graphen. Eine Abbildung
\Func{f : V}{V'}{}{}
heißt \df{Graphenmorphismus}, falls für alle $e \in E$ gilt:
\[ f(e) := f(\delta(e)) \in E' \]

\Def{Pfad}
Ein \df{Pfad} bzw. \df{Kantenweg} in $\Gamma$ der Länge $n \in \N_0$ ist definiert durch eine endliche Folge $P = v_0,\ldots, v_n \in V$, sodass $v_i$ und $v_{i+1}$ für jedes $i = 1,\ldots, n-1$ benachbart sind.\\
Wir sagen: $P$ \df{verbindet} $v_0$ mit $v_n$.\\
$P$ heißt \df{reduziert}, falls keine Kante mehrfach hintereinander im Pfad vorkommt.\\
Ist $v_0 = v_n$, so heißt $P$ \df{geschlossen}.\\
Ein \df{Kreis} bzw. \df{Zykel} der Länge $n \in \N$ ist definiert als ein geschlossener, reduzierter Pfad der Länge $n$.\\
Ein simplizialer Graph ohne Kreise heißt \df{Wald}; ist er zusammenhängend, so heißt er \df{Baum}.

\Satz{Charakterisierung von Bäumen}
Sei $\Gamma = (V,E)$ ein zusammenhängender, simplizialer Graph mit mindestens zwei Knoten, dann ist $\Gamma$ genau dann ein Baum, wenn für jedes Knotenpaar genau ein reduzierter Pfad existiert, der beide Knoten verbindet.

\Def{Cayleygraph}
Sei $G$ eine Gruppe mit Erzeugendensystem $S$. Der \df{Cayleygraph} von $G$ bzgl. $S$ ist definiert durch
\begin{align*}
\Cay{G}{S} & := (V,E,\delta)\\
V &= G\\
\end{align*}
\[E  = \set{(a,b)}{\exists s \in S: as = b} + \set{(a,b)}{\exists s \in S: as\i = b, s\i \in S}\]
\paragraph{Bemerkung}
\[E = \set{(s, (g, gs))}{ s\in S, g\in G } \]

\Bsp{Cayleygraphen}

\Prop{Eigenschaften der Cayleygraphen}
\begin{enumerate}
\item Cayleygraphen sind zusammenhängend.
\item Cayleygraphen sind \df{regulär}, d.h. jeder Knoten hat die gleiche Anzahl ausgehender Kante.
\item Cayleygraphen sind \df{lokal endlich}, d.h. jeder Knoten hat eine endliche Anzahl adjazenter Kante.
\end{enumerate}

\Satz{}
Sei $G$ eine Gruppe, $S \tm G$ ein Erzeugendensystem
\begin{itemize}
\item $Cay(G,S)$ besitzt Schleifen $\Gdw{}$ $1 \in S$
\item $Cay(G,S)$ besitzt Doppelkanten $\Gdw{}$ $\exists s \in S: s^2 = 1 \neq s$
\item $Cay(G,S)$ ist kombinatorisch $\Gdw{}$ $S\cap S\i = \set{s\in S}{s^2 = 1}$
\end{itemize}

\Satz{}
Für jede Menge $S$ ist $\Cay{\F(S)}{S}$ ein Baum.

\begin{Beweis}{}
Wir wissen bereits, dass $\Gamma = \Cay{\F(S)}{S}$ simplizial ist. Es bleibt zu zeigen, dass $\Gamma$ keine reduzierten Kreise besitzt. Angenommen $\Gamma$ besitzt einen reduzierten Kreis, dann existiert ein Wort $w \in (S + S\i)^*$ und ein $g \in \F(S)$, sodass $gw = g$, also $w = 1$.\\
Betrachte die Abbildung
\Func{f : S}{\Z}{s}{1}
dann existiert laut UAE eine Abbildung
\Func{g : \F(S)}{\Z}{}{}
sodass $f = g$ auf $S$. Nun folgt
\[ 0 = f(1) = f(w) \]
weswegen $w$ Länge 0 haben muss. Deswegen muss unser Kreis aber eine Schleife sein, was ein Widerspruch ist.\\
Ergo ist $\Gamma$ kreisfrei.
\end{Beweis}

\Satz{Beinahe Umkehrung von 2.10}
Sei $G$ eine Gruppe und $S$ ein \df{reduziertes Erzeugendensystem}, d.h. für alle $s,t \in S$ gilt:
\[ s\cdot t \neq 1 \]
Ist $\Cay{G}{S}$ ein Baum, dann ist $S$ ein freies Erzeugendensystem von $G$.

\begin{Beweis}{}
Sei $\iota : S \inj \F(S)$ die kanonische Inklusion. Sie induziert uns einen Homomorphismus
\Func{f : G}{\F(S)}{s_1\cdots s_n}{f(s_1\cdots s_n)}
$f$ ist wohldefiniert. Sind $w, v$ zwei Wörter in $(S + S\i)^*$, die in $G$ dasselbe Element erzeugen, so ergeben sich zwei nicht reduzierte Pfade im Cayleygraphen. Da dieser ein Baum ist, müssen beide Pfade übereinstimmen, weswegen $v,w$ als reduzierte Wörter und somit auch in $\F(S)$ übereinstimmen müssen.\\
Es ist jetzt nicht notwendig zu zeigen, dass $f$ bijektiv ist. Über ein Kommutativitätsargument kann man zeigen, dass man durch $f$ einen natürlichen Pfeil $h = f \circ g : \F(S) \rightarrow \F(S)$ erhält, der das UAE Diagramm von $\F(S)$ zum kommutieren bringt und deswegen die Identität ist. Ebenfalls erhält man einen Pfeil $G \pfeil{} \F(S) \pfeil{} G$, der die Identität ist.
\end{Beweis}

\Bsp{Beispiel}
Die Bedingung, dass $S$ in 2.11 reduziert ist, ist notwendig, da der Cayleygraph von
$\grp{s}{s^2 = 1}$
zwar ein Baum ist, aber die Gruppe offensichtlich nicht frei ist.

\Bem{}
Ist $S \tm G$ kein Erzeugendensystem von $G$, so
\begin{itemize}
\item nennt man $\Cay{G}{S}$ keinen Cayleygraphen.
\item ist $\Cay{G}{S}$ nicht zusammenhängend.
\end{itemize}

\newpage
\section{Gruppenwirkungen auf Cayleygraphen}
\Def{}
Sei $G$ eine Gruppe, $\C$ eine Kategorie, $X$ eine Menge. Eine \df{Wirkung} bzw. \df{Operation} von $G$ auf $X$ ist ein Homomorphismus
\Func{\rho : G}{\Aut{\mathcal{C}}{X}}{}{}

\Bsp{Beispiel}
\Bem{Alternative Schreibweise}

\Def{}
$G$ operiere auf $X$
\begin{itemize}
\item Die Gruppenwirkung heißt \df{frei}, falls $g.x = x \gdw{} g = 1$ für alle $g,x$. D.h. der Stabilisator
\[\Stab{G}{x} := \set{g \in G}{g.x = x}= \set{1}{}\]
für alle $x\in X, g \in G$. In diesem Fall wirkt jedes $g \in G- \set{1}{}$ \df{fixpunktfrei}.
\item Die Gruppenwirkung heißt \df{treu}, falls $\rho$ injektiv ist.
\item Die Gruppenwirkung heißt \df{transitiv}, falls für alle $x,y\in X$ ein $g\in G$ existiert, sodass $g.x = y$. In diesem Fall ist die \df{Bahn} bzw. der \df{Orbit} $G.x$ jedes $x$ gleich $X$.
\end{itemize}

\Bsp{Beispiel}

\Def{Freie Wirkungen auf Graphen}
Sei $G$ eine Gruppe, die auf einen Graphen $\Gamma = (V,E, \delta)$ wirkt. Diese Wirkung heißt \df{frei}, wenn für alle $x\in V, e \in E, g \in G-\set{1}{}$ gilt:
\begin{itemize}
\item $g.x \neq x$, d.h. die Wirkung ist \df{fixpunktfrei}.
\item $g.e \neq e$, d.h. die Wirkung ist \df{inversionsfrei}.
\end{itemize}

\Bsp{}

\Satz{}
Sei $G$ eine Gruppe mit Erzeuger $S, 1 \notin S$. Die Wirkung von $G$ auf $\Cay{G}{S}$ ist genau dann frei, wenn $S$ keine Elemente der Ordnung 2 enthält.
\begin{Beweis}{}
Wir müssen zeigen, dass die Wirkung genau dann inversionsfrei ist, wenn $S$ keine Elemente der Ordnung 2 enthält.\\
Sei $e = (g, gs) \in E, h \in G - 1$, es gelte
\[ \{hg, hgs\} = e.h = e = \{g, gs\} \]
das ist äquivalent dazu, dass
\[hgs = g \text{ und } hg = gs \]
was äquivalent ist zu $s^2 = 1$.
\end{Beweis}

\Def{Äquivalente Wirkungen}
Zwei Wirkungen $\phi : G \rightarrow \Aut{\C}{X}, \rho : G \pfeil{} \Aut{\C}{Y}$ heißen \df{äquivalent}, wenn es einen Isomorphismus $f : X \pfeil{} Y$ gibt, der mit der Wirkung verträglich ist, d.h.
\[ f(g.x) = g.f(x) \]

\Satz{Sabidussi}
$f : G \rightarrow Aut(\Gamma)$ Ghom., $\Gamma$ ein kombinatorischer Graph
\begin{enumerate}
	\item $f$ äquivalent zur Linkstranslation von $G$ auf der Kontenmenge von $Cay(G,S)$ für red. EZS $S$\\
	$\Longleftrightarrow$ $f_V$ ist transitiv und fixpunktfrei (heißt auch \df{einfach transitiv})
	\item Setzt man zusätzlich voraus, dass $G$ keine Element der Ordnung 2 hat, so ist $f_E$ inversionsfrei.
	\item $\Gamma$ ist ein Cayley-Graph $\Gdw{}$ $\Aut{}{\Gamma}$ enthält eine Untergruppe, die transitiv und fixpunktfrei auf $V(\Gamma)$ operiert.
\end{enumerate}

\paragraph{Beweis:}
Wähle feste Ecke $x\in \Gamma$\\
\[\overline{S} := \set{g\in G}{gx \text{ ist benachbart zu } x}\]

\paragraph{Behauptung 1}
$\overline{S}$ abgeschlossen unter Inversenbildung. Denn sind $x, gx$ Nachbarn, so müssen auch $g^{-1}x, g^{-1}gx$ Nachbarn sein, da $f(g)$ Graphhom. ist.\\

Sei $S$ die Menge bestehend aus allen Elementen $s\in \overline{S}$ mit $s^{-1} = s$ oder je eines der beiden Elemente $s \neq s^{-1}$ (Auswahlaxiom).
\[ x, sx \text{ benachbart} \Longleftrightarrow s \in S \vee s^{-1} \in S  \]
Rest genau so.

\paragraph{Aussage Teil 2}
\[\Gamma \text{ Cayley} \Longleftrightarrow Aut(\Gamma) \text{ enthält UG die transitiv und frei auf } V(\Gamma) \text{ wirkt} \]

\qed

\Satz{}
Eine Gruppe $G$ ist genau dann frei, wenn sie frei auf einen Baum wirkt.
\begin{Beweis}{$\Longleftarrow$ von 13.11}
\[G \text{ frei } \Longleftrightarrow G \curvearrowright_\text{frei} Baum \]
\[S' := \set{g_e \in G}{e \text{ wesentlich für }T_0} \]
wesentlich heißt
\[e = \{u,v\}, u \in T_0, v \notin T_0\]
$g_e$ so, dass $g_e^{-1} v \in V(T_0)$

\paragraph{2.Schritt}
Zeige $S'$ erzeugt $G$:\\

$g\in G$, Ziel: finde Elemente in $S'$ so, dass $g$ Produkt dieser ist.\\

Wähle Ecke $u \in T_0$, weil $T$ zusammenhängend, existiert Kantenpfad $p$ in $T$ von $u$ nach $g.u$.

Weil $V(T) = \bigcup_{g\in G} V(g.T_0)$, weil $T_0$ aus jedem $G$-Orbit eine Ecke enthält.\\
$\Longrightarrow$ $p$ durchläuft verschiedene Kopien $g_0T_0, ..., g_nT_0$ von $T_0$ mit $g_0 = 1, g_n = g$.\\
Es ist $g_{j+1} \neq g_j$ für $\forall j: k_0 \leq j \leq k_1$, wenn $p$ reduziert.\\
$\Rightarrow$ $g_jT_0$ und $g_{j+1}T_0$ sind für alle $j$ wie oben verbunden.\\
$g_j^{-1} e_j$ ist wesentliche Kante für $T_0$; $p = e_0...e_{n-1}$\\
Setze $s_j := g_j^{-1}g_{j+1} \in S'$.\\
Dann $g = g_0 \cdots g_{k_0}^{-1}g_{k_0 +1} g_{k_0 +1}^{-1} \cdots g_n = s_0 \cdots s_n \in \grp{S'}{}$

\paragraph{3.Schritt}
$\exists S \subset S'$, das G frei erzeugt.

aus 1.Schritt folgt, dass S' in Paare aufspaltet $\{s, s^{-1}$; für S wähle ein Element pro Paar aus.

Es reicht zu zeigen: Cay(G,S) enthält keine Kreise.

Annahme: Sei $g_0,\ldots, g_{n-1}, g_n = g_0$ Kreis in Cay(G,S)

Setze $s_j := g_j^{-1}g_{j+1} \forall j= 0, \ldots , n-1$

Es sei $s_j\in S \forall j$ (OE: S so wählbar)

Sei $e_j$ wesentliche Kante zw. $T_0$ und $s_jT_0$

Jede Kopie von $T_0$ ist zusammenhängender Teilbaum, daher können wir die Ecken der Kanten $g_je_j$ und $g_js_je_{j+1} = g_{j+1}e_{j+1}$, die in $g_{j+1}T_0$ liegen durch einen eindeutigen, reduzierten Weg in $g_{j+1}T_0$ verbinden.

Weil $g_n = g_0$, ist der erhaltene Weg geschlossen.\\
Starten und Enden in selber Kopie vom Baum $T_0$. Widerspruch zu $T$ ist Baum.
\end{Beweis}

\Def{Fundamentalbaum}
$G$ wirke auf einem zusammenhängenden, simplizialen Graphen $X = (V,E)$.\\
Ein \df{Fundamentalbaum} dieser Wirkung ist ein Unterbaum von $X$, der genau eine Ecke aus jeder $G$-Bahn von $V$ enthält.

\Satz{}
Jede Wirkung auf einem nichtleeren, zusammenhängenden, simplizialen Graphen besitzt einen Fundamentalbaum.

\begin{Beweis}{}
Wähle einen beliebigen Knoten $x \in V$. Dieser liegt in einer Bahn $G.x$. Da $\Gamma$ zusammenhängend ist, existiert für jede andere Bahn $G.v$ ein Pfad $P_v : x \mapsto v$. Dieser lässt sich reduzieren auf einen Pfad $P_{v'} : x \mapsto v'$ (für ein passend gewähltes  $v' \in G.v$), der ausschließlich Kanten enthält, die nicht in einer Bahn liegen, da $G$ auf $E$ wirkt. Der Fundamentalbaum ist nun der Spannbaum des Graphen, der durch alle $v', P_{v'}$ erzeugt wird. 
\end{Beweis}

\Def{}
$G$ wirke frei auf einem Baum $T$, $T_0$ sei ein Fundamentalbaum dieser Wirkung. Dann heißt eine Kante $e$ wesentlich, wenn
\[ e \notin E(T_0) \text{ und } \partial{e} \cap V(T_0) \neq \emptyset \]

% 15.05.15: 10.VL



\Kor{Satz von Nielsen-Schreier} 
Untergruppen freier Gruppen sind frei.
\begin{Beweis}{}
Eine Untergruppe wirkt frei auf den Cayleygraphen seiner Obergruppe.
\end{Beweis}

\Kor{Quantitative Version von 3.15} $F$ freie Gruppe, $Rang(F) = n$, $G < F$ UG vom Index $k$. Dann ist G frei und vom Rang $k(n-1) + 1$. Insbesondere sind Untergruppen vom endlichen Index in freien Gruppen vom endlichen Rang endlich erzeugt.
\paragraph{Beweis} $S$ freies EZS von $F$, $\Gamma:= Cay(G,S)$, $G,F \curvearrowright_{frei} \Gamma$ durch Linksmult.

Bew 3.11: Rang(G) = $\frac{1}{2} E$, E = \# wesentlicher Kanten für Fund.-Baum $T_0$ von $G \curvearrowright T$

Weil $|F:G| = k$ hat $T_0$ genau $k$ Ecken.

Es gilt $d_T(v) = 2n$ für alle v in T.

Dann: (1) $\sum_{v\in V(T_0)} d_T(v) = k 2n$, andererseits ist $T_0$ endlicher Baum mit $k$ Ecken, also hat $T_0$ $k-1$ Kanten.

In (1) werden Kanten doppelt gezählt, d.h.
\[\sum_{v\in V(T_0)}d_T(v) = 2(k-1) + E\]
\[1/2 E = k(n-1) + 1 = Rang G\]

\Kor{}
F frei vom Rang $m \geq 2$, und $n\in \N$, Dann gibt es UG von $F$, die frei und vom Rang n ist.

\Satz{Ping-Pong Lemma (Felix Klein)}
$G$ Gruppe, erzeugt von $S = \{a,b\}$, wobei $a,b$ unendliche Ordnung.\\
$G \curvearrowright X$, X Menge, so dass für $\emptyset \neq A, B \subset X$
mit $B \not \subset A$ gilt:
\[a^nB \subset A \text{ und } b^nA \subset B, \forall n \in \Z\setminus\{0\} \]

dann ist G frei von S erzeugt.

\paragraph{Beweis}
Zu zeigen $G \cong F_{red}(a,b)$ via Isom, der S festhält.

UAE:
$\phi : F_{red}(a,b) \longrightarrow G$ mit $\phi|S = id$, dann ist $\phi$ surjektiv.

Zu zeigen: $\phi$ injektiv.

Annahme: $\phi$ nicht injektiv, dann existiert $w\in F_{red}(S)$ mit $\phi(w) = 1$

4 Fälle: 

\paragraph{1.Fall}
w beginnt mit nichttriv. Potenz von a und endet mit einer solchen:
\[w = a^{n_0}b^{m_0}...b^{m_k}a^{n_{k+1}}, n_i, m_i \in \Z- 0 \]

Nun ist $B = 1.B = \phi(w)B = a^{n_0}b^{m_0}...b^{m_k}a^{n_{k+1}}.B \subset A$. Widerspruch!

\paragraph{2.Fall}
w beginnt mit $b$ und endet mit $b$. konjugiere mit a: 1.Fall

\paragraph{3.Fall}
w beginnt mit $a$ und endet mit $b$. Konjugiere mit $a^k$ für k groß genug


\Bsp{}
freie UG von SL(2,Z)

\[SL(2,\Z) = \set{\begin{matrix}
	a & b\\
	c & d\\
	\end{matrix}}{det = 1}\]
Dann ist $G:= \grp{M_1, M_2}{}$ frei vom Rang 2, wobei 
\[M_1 =\begin{matrix}
1 & 2\\
0 & 1
\end{matrix}, M_1 =\begin{matrix}
1 & 0\\
2 & 1
\end{matrix}\]

\paragraph{Beweis}
Betrachte lineare Wirkung von $SL(2,\R) \curvearrowright \R^2$ definiert durch
\[(M, (x,y)) \longmapsto M.(x,y)\]
$\forall n \in \Z - 0$ und $(x,y) \in \R^2$: $M_1^n.(x,y) = (x+2ny,y)$

Sei $A = \set{(x,y)}{|x| > |y|}$, 
 $B = \set{(x,y)}{|y| > |x|}$,  $B \not \subset A$

Dann $|x + 2ny| \geq |2ny| - |x| > |2y| - |y| = |y|$, also $M_1^nB \subset A$, analog für $M_2$.

3.18 zeigt: G frei.

\Bsp{Beispiel}

% 19.05.15, 11. VL

\newpage
\section{Quasi-Isometrien}

\paragraph{Motivation}
Gruppe -> Geometrie
Ziel: Konzept finden, welches Cayleygraphen einer festgelegten Gruppe als gleich (äquivalent) auffasst

\Def{}
Seien $(X,d),(Y,d)$ metrische Räume, $f:X\rightarrow Y$ eine Abbildung.
\begin{itemize}
	\item $f$ heißt eine \df{isometrische Einbettung}, falls für alle $x,y \in X$ gilt
	\[d(f(x),f(y)) = d(x,y) \]
	\item $f$ heißt eine \df{Isometrie}, falls $f$ eine surjektive isometrische Einbettung ist.
	\item $X$ und $Y$ heißen isometrisch, falls eine Isometrie $X\rightarrow Y$ existiert.
\end{itemize}

\Def{}
\begin{itemize}
	\item $f$ heißt eine \df{Bilipschitz-Einbettung}, falls eine reelle Konstante $c \geq 1$ existiert, sodass für alle $x,y \in X$ gilt
	\[ \frac{1}{c} d(x,y) \leq  d(f(x),f(y)) \leq c d(x,y) \]
	\item $f$ heißt eine \df{Bilipschitz-Äquivalenz}, falls $f$ eine surjektive Bilipschitz-Einbettung ist.
	\item Existiert eine Bilipschitz-Äquivalenz $X\pfeil{} Y$, so heißen $X$ und $Y$ \df{bilipschitz (äquivalent)}.
\end{itemize}

\Def{}
\begin{itemize}
	\item $f$ heißt eine $(C,D)$-\df{quasi-isometrische Einbettung}, falls reelle Konstanten $c \geq 1, D\geq 0$ existieren, sodass für alle $x,y \in X$ gilt
	\[ \frac{1}{C} d(x,y) - D \leq  d(f(x),f(y)) \leq C d(x,y) + D \]
	\item $f$ und $g$ haben endlichen Abstand, falls eine reelle Konstante $C \geq 0$ existiert, sodass
	\[d(f(x), g(x)) \leq C ~~ \forall x \in X \]
	\item $f$ heißt eine $(C,D)$-\df{Quasi-Isometrie}, falls $f$ eine $(C,D)$-{quasi-isometrische Einbettung} ist und eine weitere $(C,D)$-{quasi-isometrische Einbettung} $g : Y \pfeil{} X$ existiert, sodass $g \circ f$ und $f \circ g$ endlichen Abstand von der jeweiligen Identität haben.
	\item Existiert eine Quasi-Isometrie $X\pfeil{} Y$, so heißen $X$ und $Y$ \df{quasi-isometrisch}. In diesem Fall schreiben wir
	\[ X \qi Y \]
\end{itemize}

\Bem{}
\begin{itemize}
	\item Isometrie -> Bil.Äqu -> QI
	\item Umkehrung i.A. nicht richtig
	\item Quasi-Isometrisch sind $(\R,d)$ und $(\Z, d)$ und $(2\Z, d)$ mit den euklidischen Metriken.
	Die Inklusionen sind quasi-isom. Einbettungen, aber keine Bilipschitzäqu., weiter sind
	\begin{align*}
	f:\R \longrightarrow \Z && x \longmapsto \lfloor x\rfloor\\
	g: \Z \longrightarrow 2\Z && x \longmapsto \{x, x -1\}\cap 2\Z
	\end{align*}
\end{itemize}
\subsection{Quiz}
\begin{itemize}
	\item Sind $\Z$ und $2\Z$ bilipschitz-äquivalent?
\end{itemize}

\Satz{Durchmesser metrischer Räume}
Jeder nichtleere, metrische Raum $(X,d)$ mit endlichen Durchmessern
\[diam(X) := \sup_{x,y\in X}(d(x,y)) \]
ist quasi-isometrisch zu einem Punkt.
\paragraph{Beweis}
Setze $D:= diam(X)$, sei $* \in X$ beliebig. definiere die Abbildung
\[f:X \longrightarrow X, x \longmapsto *\]
Dann gilt
\[d(f(x), f(y)) - D \leq d(f(x),f(y)) \leq d(f(x),f(y)) + D \]
Daraus folgt auch, dass $d(f^2(x), id(x)) \leq D$, ergo sind $X$ und $*$ quasi-isometrisch.
\qed

\paragraph{Korollar}
Ist $X$ beschränkt und $Y$ quasi-isom. zu $X$, so ist auch $Y$ beschränkt.

\Satz{Alternative Charakterisierung von Quasi-Isometrie}
Sei $f : X \pfeil{} Y$ eine quasi-isometrische Einbettung. $f$ ist genau dann eine Quasi-Isometrie, wenn sein Bild \df{quasi-dicht} in $Y$ liegt, d.h. $\exists \delta > 0 \forall y \in Y \exists x \in f(X):$
\[ d(x,y) \leq \delta \]

\begin{Beweis}{}
$\Leftarrow:$\\
Ist $f$ eine QI, so existiert eine Inverse $g : Y \pfeil{} X$. Es gilt nun für alle $y \in Y:$
\[ d( y, f(g(y)) ) \leq \delta \]
für ein passendes $\delta$.


$\Rightarrow:$\\
Konstruiere $g$ wie folgt:
\[g(y) := x, \text{ so dass } d(y,f(x)) \leq \delta \]

\paragraph{Beweis}
$f$ Quasi-Isometrie, dann existiert quasi-Inverse $g : Y \rightarrow X$ und somit $\delta > 0$, s.d. $\forall y \in Y$ gilt
\[d((f\circ g)(y), y) \leq \delta \]
ergo quasi-Dichtes Bild.

Andere Richtung: $f$ sei (C,D)-q.i.Einbettung mit $\delta$-dichtem Bild, wir konstruieren quasi-Inverse via Auswahlaxiom\\
Setze $\lambda := \max\{C,D,\delta\} \geq 1$, dann gilt
\begin{itemize}
	\item $\forall x,y \in X:  \frac{1}{\lambda} d(x,y) - \lambda \leq d(f(x),f(y) \leq {\lambda} d(x,y) + \lambda  $
	\item $\forall y \in Y \exists x \in X: d(f(x),y) \leq \lambda$
\end{itemize}
Setze $g:Y\rightarrow X, y \longmapsto x_\lambda $; wähle $x_\lambda$ so, dass $d(f(x_\lambda), y) \leq \lambda$.\\

Zu Zeigen: $g$ ist quasi-invers zu $f$.\\
\[\forall y\in Y: d(f(g(y)), id(y)) = d(f(x_\lambda), y) \leq \lambda \]
\[\forall x\in X: d(g(f(x)), id(x)) = d(x_{f(x)}, x) \leq \lambda \cdot d(f(x_{f(x)}), f(x)) + \lambda^2 \leq 2\lambda^2 \]

Noch zu zeigen: $g$ ist quasi-isometrische Einbettung\\

Seien dazu $y,y' \in Y$
\[d(g(y),g(y')) = d(x_y, x_{y'}) \leq \lambda d(f(x_y), f(x_{y'})) + \lambda^2 \]
\[\leq \lambda \left( d(f(x_y),y) + d(y, y') + d(y', f(x_{y'})) \right) + \lambda^2\]
\[\leq \lambda^2 + \lambda d(y, y') + \lambda^2 + \lambda^2 \]

Setze $C = \lambda, D = 3\lambda^2$\\

Für $y,y' \in Y$ ist noch zu zeigen
\[d(g(y),g(y') \geq \frac{1}{C} d(y,y') - D \]

\end{Beweis}

\Satz{Quasi-Isometriegruppe}
\begin{itemize}
\item Die Komposition zweier Quasi-Einbettungen ist wieder eine Quasi-Einbettung.
\item Definiere $QI(X)$ als die Menge aller Quasi-Isometrien $X\pfeil{} X$ und führe folgende die Äquivalenzrelation ein:
\[ f\sim g \Gdw{} f,g \text{ haben endlichen Abstand} \]
Dann ist $QI(X)/\sim$ bzgl. der Komposition von Abbildungen eine Gruppe, die sogenannte \df{Quasi-Isometrie-Gruppe}.
\item Eine Wirkung von $G$ auf $X$ \df{mittels Quasi-Isometrien} ist ein Morphismus
\[G \Pfeil{} QI(X)/\sim \]
\end{itemize}

\Bsp{Beispiel}

\Def{}
Sei $\Gamma = (V,E, \d)$ ein zusammenhängender Graph.
\begin{itemize}
\item Definiere die \df{kombinatorische Metrik} auf $\Gamma$ durch
\begin{align*}
d : V^2 &\Pfeil{} \N_0\\
(v,w) & \longmapsto inf \{n \in \N_0 ~~|~~\text{es existiert ein }n \text{-Pfad von }v \text{ zu }w \}
\end{align*}
\item Sei $G$ Gruppe mit Erzeuger $S$. Die kombinatorische Metrik auf $\Cay{G}{S}$ heißt \df{Wortmetrik}.
\end{itemize}

\Bsp{Beispiel}

\Satz{Alle Cayleygraphen einer Gruppe sind quasi-isometrisch}
Sei $G$ eine Gruppe, $S,T$ endliche Erzeugermengen von $G$.
Dann ist $\id{G}$ eine Bilipschitz-Äquivalenz von $(G,d_S)$ und $(G,d_T)$.

\Def{QI-Typen von endlich erzeugten Gruppen}
Sei $G$ eine endlich erzeugte Gruppe.\\
$G$ heißt \df{bilipschitz-äquivalent} bzw. \df{quasi-isometrisch} zu einem metrische Raum $X$, falls sein Cayleygraph {bilipschitz-äquivalent} bzw. {quasi-isometrisch} zu $X$ ist.\\
Zwei endlich erzeugte Gruppen heißen \df{bilipschitz-äquivalent} bzw. \df{quasi-isometrisch}, falls ihre Cayleygraphen {bilipschitz-äquivalent} bzw. {quasi-isometrisch} sind.\\

\Kor{}
Eine endlich erzeugte Gruppe ist genau dann quasi-isometrisch zu einer endlichen Gruppe, wenn sie selber endlich ist.

\Satz{}
Sei $G$ eine Gruppe, die von einem EZS $S$ erzeugt wird.\\
$S$ endlich $\Gdw{}$ $d_S$ ist \df{eigentlich}, d.h. Bälle von endlichem Radius enthalten endlich viele Elemente. 


\Bsp{}

\Def{Virtuelle Eigenschaften}
$G$ ist virtuell $P$, wenn $G$ eine Untergruppe von endlichem Index besitzt, die $P$ hat.


\Satz{}
Eine endlich erzeugte Gruppe ist virtuell $\Z$, wenn sie quasi-isometrisch zu $\Z$ ist.

%\subsection{4.17 Satz}
%$X,Y$ metrische Räume,$f:X\rightarrow Y$ eine quasi-isometrische Einbettung. Dann gilt:
%\[f \text{ Quasi-Isometrie }\Longleftrightarrow f \text{ hat quasi-dichtes Bild in }Y \]
%d.h. $f(X) \subset Y$ ist $\delta$-dicht für $\delta \geq 0$, d.h.
%\[\forall y\in Y, \exists x \in X: d(y,f(x)) \leq \delta \]



%% 13. VL, 26.05.15

\Def{Geodäten}
Eine \df{Geodäte} ist eine isometrische Einbettung $\gamma : [0,L] \rightarrow X$ eines Intervalls in einen metrischen Raum.\\
$X$ heißt \df{geodätisch (vollständig)}, falls je zwei Punkte durch eine Geodäte verbunden werden.

\Bsp{Beispiel}

\Def{Quasigeodäte}
Eine $(C,D)$-\df{Quasigeodäte} für $C\geq 1, D \geq 0$ ist eine $(C,D)$-Quasiisometrische Einbettung von $[0,L]$ nach $X$.

$X$ heißt $(C,D)$-quasigeodätisch, falls $\forall x,y \in X$ eine verbindende Quasigeodäte \[\gamma :[0,d(x,y)] \rightarrow X \] existiert.

\Bsp{Beispiel}

\Satz{Schwarz-Milner}
$G$ Gruppe, $X$ metr. Raum, $G\curvearrowright X$ durch Isometrien. Weiter gelte:
\begin{itemize}
\item $X$ $(C,D)$-quasi-geodätisch
\item $\exists B \subset X$ beschränkt mit $\bigcup_{g\in G}gB = X$
\item $S:=\set{g\in G}{gB' \cap B' \neq \emptyset}$ ist endlich mit $B' := B_{2D}(B) = \set{x\in X}{\exists y \in B: d(x,y) \leq 2D}$
\end{itemize}
Dann gilt:\\
$G$ wird von $S$ erzeugt und
$\forall x \in X$ ist 
\Func{(G,d_S)}{(X,d)}{g}{g.x}
eine Quasi-Isometrie.

\paragraph{Beweis}
ZZ: $S$ erzeugt $G$

Sei $g\in G$, $x\in B$. Dann existiert $(C,D)$-Quasigeodäte von $x$ nach $g.x$, $\gamma: [0,d(x,g.x)] \rightarrow X$.

Setze $n := \lceil\frac{CL}{D}\rceil$ und für alle $j = 0,\ldots, n-1$
Setzte $t_j = \frac{jD}{C}$ und $t_n := L$
$x_j := \gamma(t_j)$ für $j = 0,\ldots,n$

Die Translate von $B$ unter $G$ überdecken $X$, also existiert für alle $x_j$ ein $g_j$, s.d. $x_j \in g_j.B$, $g_0 = 1$, $g_n \in g$

Beh.: $\forall j = 1,\ldots, n$ ist $s_j := g_{j-1}^{-1}g_j \in S$
Bew.: $\gamma$ Quasi-Geodäte
$d(x_{j-1},x_j) \leq C |t_{j-1}-t_j| + D
\leq C \frac{D}{C} + D
= 2D$
also $x_j \in B_{2D}(g_{j-1}.B) \overset{G \curvearrowright X isom.}{=} g_{j-1}.B_{2D}(B) = g_{j-1}.B'$
andererseits ist $x_j \in g_j.B \subset g_j.B'$
also $g_j.B \cap g_{j-1}.B'\neq \emptyset$
also $g_{j-1}^{-1}g_j \in S$
\qed

Also $g = g_n = g_{n-1}(g_{n-1}^{-1}g_n) = g_{n-1}s_n = g_{n-2}(g_{n-2}^{-1}g_{n-1})s_n = s_1...s_n \in \grp{S}{}_G$


ZZ. $G\sim_{QI}X$:

Wir zeigen $\forall x \in X:$ $\phi: G \rightarrow X, g \mapsto g.x$ quasi-isom. Einbettung mit quasi-dichtem Bild.

OE: $x \in B$, weil $\bigcup_{g\in G}g.B = X$ und $G \curvearrowright X$ isom., sonst ersetze $B$ durch passendes Translat.

Sei $x'\in X$. Dann gibt es $g\in G$ mit $x' \in g.B$
$d(x', \phi(g)) = d(x', gx) \leq diam(gB) = diam(B) = \delta$
$\Longrightarrow \delta$-dichtes Bild

Noch ZZ: qi. Einbettung

Betrachte $(C,D)$-quasi-geodäte $\gamma : [0,L] \rightarrow X$ von $x$ nach $g.x$
Dann gilt
$d(\phi(e), \phi(g)) = d(x, g.x) = d(\gamma(0), \gamma(L)) \geq \frac{L}{C} - D
\geq \frac{1}{C}(\frac{D(n-1)}{C}D)
=\frac{D}{C^2}n - \frac{D}{C^2} - D
\geq \frac{D}{C^2} d_S(e,g) - (\frac{D}{C^2}+D)$

Abschätzung nach oben:
Setze $n = d_S(e,g)$
\[d(\phi(e), \phi(g))
= d(x,g.x)
\leq d(x,s_1.x) + d(s_1.x, s_1s_2.x) + ... + d(s_1...s_{n-1}.x, g.x)
\overset{G wirkt isom.}{=}
d(x,s_1.x) + d(x, s_2.x) + ... + s(x,s_n.x)
\leq 2n diam(B')
\leq 2n (diam(B)4D)
= d_S(e,g) 2 (diam B + 4D)
\]
wähle für $(C_0, D_0)$-qi Einbettung die Konstanten
$C_0 = \max\set{C^2/D, 2(...)}{}$
$D_0 = D/C^2 + D$
allgemeiner Fall folgt aus der Linksinvarianz von $d$ und $d_S$.
\qed

%% 14.VL, 29.05.15

\subsection{4.23 Definition}
Ein metrischer Raum $X$ heißt \df{eigentlich}, falls alle abgeschlossene Bälle von endlichem Radius kompakt sind.\\
Eine Wirkung $G\curvearrowright X$ ist \df{eigentlich}, wenn für alle kompakten Teilmengen $K \subset X$, die Menge
\[\set{g\in G}{g.K \cap K \neq \emptyset}\]
endlich ist.\\
Manchmal sagt man auch \df{eigentlich diskontinuierlich}.

\paragraph{Bemerkung}
$f$ eigentlich, wenn Urbilder kompakter Mengen wieder kompakt sind.\\
Hier $G\curvearrowright X$ eigentlich
\[\Longleftrightarrow G\times X \longrightarrow X \]
\[(g,x) \longmapsto g.x\]
ist eigentliche Abbildung. (Wobei man auf $G$ die diskrete Topologie betrachtet.)

\subsection{4.24 Beispiel}
\begin{itemize}
	\item $\Z \curvearrowright \R$ via Translation ist eigentlich.
	\item $G\curvearrowright X$ eigentlich $\Longrightarrow$ $Stab_G(x)$ ist endlich für $x \in X$, d.h. $G$-Bahnen haben keinen Häufungspunkt
	\item $\Z \curvearrowright \R^2$ Rotation um Ursprung um Winkel mal $z$\\
	$(0,0)$ ist Fixpunkt, also kann diese Wirkung nicht eigentlich sein.
	\item $\Z \curvearrowright S^1$ via Rotation um $\alpha$ ist nicht eigentlich, da $S^1$ kompakt.
	\item $unendliche Gruppe \curvearrowright kompakter Raum$ ist nicht eigentlich
	\item $G$ erzeugt von $S$, $|S| < \infty$, dann ist $G \curvearrowright Cay(G,S) =: \Gamma$ eigentlich.
	\paragraph{Beweis}
	$K \subset \Gamma$ kompakt $\Longrightarrow$ $diam(K)< \infty$ $\Longrightarrow$ $\forall g\in G$ mit $d_S(e,g) = |g|_S > diam(K)$ gilt: $K\cap g.K = \emptyset$, sonst $\exists x \in K \cap g.K \Longrightarrow x \in K$ und $g^{-1}.x\in K$ mit $d_S(x,g^{-1}.x) = |g^{-1}|_S = |g|_S$ ein Widerspruch\\
	Insbesondere nur endlich viele $g$ mit $|g|_S \leq D$. \qed
\end{itemize}

\subsection{4.25 Erinnerung} $X$ topologischer Raum
\begin{itemize}
	\item $X$ \df{hausdorffsch}, g.d.w.
	\[\forall x \in X \exists U_x \subset_O, x \in U_x, U_y \subset_O, y \in U_y: U_x \cap U_y = \emptyset  \]
	\item $X$ \df{lokal kompakt}, g.d.w.
	Für alle $x \in X$ enthält jede offene Umgebung von $x$ eine kompakte Umgebung von $x$.
	\item $X$ metrischer Raum $\Longrightarrow$ hausdorffsch
	\item eigentliche metrische Räume $\Longrightarrow$ lokal kompakt
\end{itemize}

\subsection{4.26 Bemerkung/Lemma: Quotientenräume}

$(X,d)$ metrischer Raum, eigentlich
\[\alpha : G \rightarrow Isom(X) \text{ Wirkung von G auf X} \]
\[p : X \rightarrow X/G \text{ natürliche Projektion auf Quotienten} \]
Setze $f(x,y) := \inf\{d(x',y') | p(x') = x, p(y') = y\}$ für $x,y \in X/G$\\
Dann gilt:
\begin{enumerate}
	\item $inf = min$, d.h. $\exists x,y \in X: f(x,y) = d(x,y) \forall x,y \in X/G$
	\item $f$ ist Metrik auf $X/G$
\end{enumerate}

\paragraph{Beweis}
Seien $z,w \in X/G$, $x = p^{-1}(w)$; setze $R = f(z,w)$\\
Annahme: inf $\neq$ min\\
Dann existieren unendliche Folgen $(x_n,y_n)$ mit $d(x_n,y_n) \rightarrow R$ und $p(x_n) = w, p(y_n) = z$.\\
Weil $p(x_n) = p(x)$ gilt: $\exists h_n \in G$ mit $h_nx_n = x$\\
$\Longrightarrow$ $d(h_nx_n, h_ny_n) = d(x_n,y_n)$, da $\alpha$ isom.\\
daraus folgt $x_n$ kann durch konstante Folge $x$ und $y_n$ durch $y_nh_n$ ersetzt werden.\\
Daraus folgt $y_n \in B_{R+\epsilon}(x_n)$, $p(y_n) = z$\\
Weil $B_{R+ \epsilon}(x)$ kompakt ist, hat $(y_n)_n$ einen HP in $B_{R+ \epsilon}(x)$. Widerspruch zu 4.24\\
\\


$f$ nichtneg. und symmetrisch, da $d$ so.\\
$f(z,w) = 0 \Longrightarrow \exists x,y: d(x,y) = 0 \Longrightarrow x = y \Longrightarrow z = w$\\
Dreiecksungleichung: $u,v,w \in X/G$, wähle $x, y \in X$, s.d. $d(x,y) = f(u,v)$, $p(x) = u, p(y) = v$.\\
Wähle $y_1$ mit $d(x,y_1) = f(u,v)$, $p(y_1) = v$; $y_2$, $p(y_2) = v$ und $d(z,y_2) = f(v,w)$\\
weil $y_1, y_2 \in p^{-1}(v)$ existiert $g$ mit $g.y_2 =y_1$\\
$\Longrightarrow f(u,v) + f(v,w) = d(x,y_1)+d(y_2,z) = d(x,g.y_2) + d(g.y_2, g.z) \geq d(x,g.z) \geq f(u,w)$
\qed


\subsection{4.27 Definition}
Eine Gruppenwirkung $G\curvearrowright X$ heißt \df{kokompakt}, wenn $X/G$ kompakt.\\
Betrachte auf $X/G$ Topologie, die durch Quotientenmetrik $f$ induziert wird, als wir mit einem metrischen Raum gestartet sind.

\subsection{4.28 Beispiele}
\begin{itemize}
	\item $\Z \curvearrowright \R^2$ durch Translation längs $x$-Achse.\\
	$\R^2/\Z$ = Zylinder ist nicht kompakt, also keien kokompakte Wirkung.
	
	\item $X$ kompakt, wegzusammenhängend top. Raum, $\widetilde{X}$ universelle Überlagerung.\\
	$\pi_1(X) \curvearrowright \widetilde{X}$ durch Decktransformationen ist kokompakt und eigentlich\\
	$X = \widetilde{X}/ \pi_1(X)$
	
	\item $G \curvearrowright Cay(G,S) =: X$ mit kombinatorischer Metrik\\
	$n := |S|$, $X/G = R_n$, Rose mit n Blättern, kompakt
\end{itemize}

\subsection{4.29 (topologischer) Satz von Schwarz-Milner}
$G$ wirke eigentlich, kokompakt, durch Isometrien auf einen nichtleeren, eigentlichen, geodätischen metrischen Raum $(X,d)$
, dann gilt $G$ endlich erzeugt und für alle $x \in X$ ist
\[G \longrightarrow X, g \longmapsto g.x\]
eine Quasi-Isometrie.\\
Wenn $G\curvearrowright X$ eigentlich, kokompakt und durch Isometrien, so sagt man auch $G$ wirkt \df{geometrisch}.

\paragraph{Beweis}
Suche $B$.\\
\begin{itemize}
	\item nach Vorr. ist $X$ $\forall \epsilon > 0$, $(1, \epsilon)$-quasi-geodätisch.
	\item Sei für bel. $x_0 \in X$: $B:=\set{x \in X}{d(x,x_0) \leq D}$;\\
	$D:= diam(X/G) < \infty$, da $G\curvearrowright X$ kokompakt.
\end{itemize}
Dann gilt: $\bigcup_{g\in G}g.B= X$, $B' := B_{2\epsilon} (B)$ endlicher Radius, also kompakt, da $X$ eigentlich.\\
$G \curvearrowright X$ eigentlich, also $\set{g \in G}{g.B' \cap B' \neq \emptyset}$ endlich.

4.22 zeigt Beh.\qed

\subsection{Korollar}
Sei $H < G$, $G$ endlich erzeugt mit $(G: H )< \infty$.
Dann ist $H$ endlich erzeugt und quasi-isom. zu $G$.

\paragraph{Bew:}
$S$ sei endl. EZS von $G$\\
$\Longrightarrow$ $H \curvearrowright Cay(G,S) =: \Gamma$ mit Wortmetrik $d_S$ isom., eigentlich, kokompakt.\\
Sei $B$ endliches Vertretersystem von $G/H$, existiert, weil Anzahl Nebenklassen von $H$ in $G$ endlich ist.\\
Dann ist $HB = G$

$B' := B_2(B)$ endlich, $\set{h\in H}{h.B'\cap B' \neq \emptyset}$ endlich.\\
Schwarz-Milner: $H$ endlich erzeugt und $H\sim_{qi} \Gamma \sim_{qi} G$\qed

%% 2.6.15, 15.Vl

\subsection{4.31 Definition}
\begin{enumerate}
	\item Zwei Gruppen $G,H$ heißen \df{kommensurabel}, wenn es Untergruppen $G'<G, H<H'$ mit endlichem Index gibt, s.d. $G' \cong H'$.
	\item Zwei Gruppen $G,H$ heißen \df{schwach kommensurabel}, wenn es Untergruppen $G'<G, H<H'$ mit endlichem Index gibt, s.d. normale \textbf{endliche} Untergruppen $N \vartriangleleft H', M \vartriangleleft G'$ mit 
	\[H'/N \cong G'/M \]
\end{enumerate}

\subsection{Bemerkung}
$\sim_C, \sim_{WC}$ sind ÄQ (kommensurabel, schwach ...)
$G\sim_C H \Longrightarrow G\sim_{QI}H (falls G endlich erzeugt)$

\subsection{Korollar}
Sei $G$ eine Gruppe und 
\begin{enumerate}
	\item $G' < G$ eine UG mit endlichem Index. Dann gilt:
	\[G' \text{ endlich erzeugt } \Longleftrightarrow G \text{ endlich erzeugt} \]
	Falls $G,G'$ endlich erzeugt, dann $G \sim_{QI} G'$
	\item $N \vartriangleleft G$ ein endliche normale Untergruppe. Dann gilt:
	\[G/ N \text{ endlich erzeugt } \Longleftrightarrow \text{G endlich erzeugt} \]
	Falls $G,N$ endlich erzeugt, dann $G/N \sim_{QI} G$
\end{enumerate}
Insbesondere: Ist G endl. erz. und $H \sim_WC G$, dann ist $H$ endlich erzeugt und $G\sim_QI H$
\subsection{Bemerkung}
Man kann zeigen, dass nicht alle qi Gruppen kommensurabel sind. Z.Bsp.: $(F_3\times F_3)* F_3 \sim_QI (F_3\times F_3)*F_4$, aber die Gruppen sind nicht kommensurabel (Eulercharakteristik)

\subsection{4.33 Korollar}
Sei $M$ eine kompakte Mannigfaltigkeit ohne Rand mit Riemannscher Metrik und $M'$ die Riem. universelle Überlagerung. Dann gilt:
\begin{enumerate}
	\item $\pi_1(M)$ endl. erz.
	\item $\forall c \in M'$ ist $\pi_1(M) \rightarrow M'$, $g\longmapsto g.x$ eine QI
\end{enumerate}
\paragraph{Beweis}
Zeige mit Standard-Argumenten der Geometrie und alg. Topo, dass M' eig. und geod.
$\pi(M) \curvearrowright M'$ eig., kokompakt und durch Isom.

\newpage
\section{Quasi-Isometrie-Invarianten}
\subsection{Definition}
Sei V eine menge. Eine \df{QI-Invariante} mit Werten in V ist eine Abb.
\[I:X \longrightarrow V\]
$X \subset \set{G: Gruppe}{G endl. erz}$, s.d. gilt
\[G\qi H \Longrightarrow I(G) = I(H)\]
\paragraph{Bemerkung}
\begin{enumerate}
	\item QI-Invarianten sind hilfreich, um $G \not\qi H $ zu zeigen
	\item i.A. ist es nicht möglich zu entscheiden, ob $G \qi H$ gilt
\end{enumerate}
\subsection{Beispiel}
\begin{enumerate}
	\item $V = \{1\}$, dann keine Infos
	\item $V = \{0,1\}$, $I(G) = 1, G unendl., sonst 0$ ist QIInv.
	\item $V = \N$, $I(F) = rang F$, $F$ endl. erz. freie Gruppe, ist keine QIInv., weil $F_n\qi F_m$ für $n,m \geq 2$
\end{enumerate}
\subsection{Definition}
Eine Eigenschaft $P$ von endl. erz. Gruppen heißt \df{geometrisch}, wenn gilt: 
$G$ hat $P$ und $H\qi G$, dann $H$ hat $P$

\subsection{Beispiel}
\begin{enumerate}
	\item $\forall n \in \N$ ist die Eigenschaft \df{virtuell $\Z^n$} zu sein eine geom. ES.
	\item \df{endlich sein} ist geometrisch.
	\item \df{endlich erzeugt und virtuell frei} ist geometrisch ES.
	\item \df{abelsch} ist kein geom. ES.
\end{enumerate}
1 bis 3 ist schwer zu beweisen, wir zeigen:
\begin{enumerate}
	\item \df{endlich präsentiert} ist geom. ES.
	\item Wachstum von Gruppen liefert geom. ES.
	\item einige Ränder/Enden von einigen Gruppen liefert geom. ES.
\end{enumerate}

\Satz{endlich präsentiert}
Sei $G = \grp{S}{R}$ endlich präsentierte Gruppe. Sei $H$ endlich erzeugt durch $S'$.\\
Gilt $G \qi H$, so existiert ein endliches $R' \tm \F(S')$, sodass $H \isom{} \grp{S'}{R'}$.

\Def{Präsentationskomplex}
$G = \grp{S}{R}$, $S$ reduziert.\\
Definiere den \df{Cayleykomplexen} $K'$ als den Komplex, der aus $\Cay{G}{S}$ durch Anwenden folgender Schritte entsteht:
\begin{itemize}
\item Für jedes selbstinverse $s\in S$ füge eine zweite Kante $gs \pfeil{} g$ für jedes $g \in G$ hinzu.
\item Orientiere alle anderen Kanten $(g, gs)$, so dass $g \pfeil{} gs$.
\item Füge für jedes $r \in R, x \in \Cay{G}{S}$ folgende Zweizellen hinzu:\\
Fasse $r = s_1\cdots s_n \in (S + S\i)^*$ als \textbf{orientierten} Kreis $x \pfeil{} x.s_1 \pfeil{\ldots} x.r$ auf und klebe entlang ihm orientierungserhaltend eine 2-Zelle.
\end{itemize}
Definiere den \df{Präsentationskomplex} als $K(S,R) := K' / G$.

\paragraph{Bemerkung}
Man kann mittels Seifert-Van Kampen zeigen, dass $K'$ einfach zusammenhängend. $K'$ ist univ. Überlagerung und $G = \pi(K) = \pi(K'/G)$

\Bem{}

\Bsp{Beispiel}
\begin{enumerate}
	\item $G = \Z^2 = \grp{a,b}{aba^{-1}b^{-1}}$\\
		$K' = \R^2, K = T^2$
	\item Flächengruppen: $G:=\grp{a,b,c,d}{a^{-1}b^{-1}abc^{-1}d^{-1}cd }$
	$K'$ kann aufgefasst werden als Parkettierung von $H^2$\\
	$K$ ist Torus mit 2 Löchern, $S^2$-Fläche von Geschlecht 2
\end{enumerate}

\Bem{alternative Definition von $K(S,R)$}
$G = \grp{S}{R}$, $S$ reduziert.\\
Definiere den Komplex $K_G$ durch
\begin{itemize}
	\item eine 0-Zelle $v$
	\item eine 1-Zelle für jedes $s \in S$, die von $v$ nach $v$ führt, orientiere diese 1-Zellen
	\item eine 2-Zelle $d_r \forall r \in R$ verklebt so, dass Kanten $s_i$ orientierungserhaltend verklebt werden über $v \pfeil{s_1}  v \pfeil{s_2}  \ldots \pfeil{s_n} v$, wobei $r = s_1\cdots s_n, s_i \in S\cup S^{-1}$
\end{itemize}
Man kann zeigen $K_G \cong K(S,R)$ und $K'$ ist univ. Überlagerung von $K_G$



\begin{Beweis}{von Satz 5.5}
Setze $G_1 := G, G_2 := H, S_1 := S, S_2 := S', \Gamma = Cay(G_i, S_i)/\sim$ wie in 5.6.\\

Sei $\rho$ die Länge der längsten Relation in $R$
\begin{itemize}
	\item Cayleykomplex $K'_1$ ist einfach zusammenhängend
	\item Seien $f:\Gamma_2 \rightarrow \Gamma_1, f' : \Gamma_1 \rightarrow \Gamma 2$ $(C,D)$-quasi Isometrien (existieren, da $G \sim_{QI} H$)
\end{itemize}
Sei $\mu > 0$, s.d. $d(f'(f(v)), v) \leq \mu \forall v \in \Gamma_2$\\
Setze $m:= \max \set{\rho, \mu, C,D}{}, M := 3(3m^2 + 5m + 1)$.\\
Sei $K_2'$ 2-Komplex, den man durch Ankleben von 2-Zellen an jeden Kreis der Länge $\leq M$ in $\Gamma_2$ erhält.\\

Sei $l$ Kantenkreis in $\Gamma_2$, d.h. $l = (g_1,\ldots, g_n, g_1)$\\
Betrachte $l$ als Abb. $\partial D \rightarrow \Gamma_2$, $D$ ist hier eine 2-Zelle.\\

\paragraph{Zwischenlemma (Formalisierung der Bemerkung 5.7.2)}
$G$ erzeugt von $S$, $R \leq Kern \pi$, $\pi : F(S) \rightarrow G$; $X$ Komplex den man, durch Ankleben von 2-Zellen an Kantenkreisen geg. durch Wörtern in $R$ an $Cay(G,S)/\sim$ erhält. Dann gilt:
\[X \text{ einfach zusammenhängend } \Longleftrightarrow \grp{R}{}_G^\vartriangleleft = kern(\pi)  \]
\paragraph{Beweis von Zwischenlemma:} Lemma 8.9 in Bridson-Haefliger, S.135


Wir sind fertig, wenn wir zeigen können:\\
$l$ besitzt stetige Fortsetzung $l' : D \rightarrow K_2'$, d.h. $K'_2$ einfach zusammenhängend.\\
Seien $v_i$ Urbilder der $g_i$ unter $l$\\
Sei $\phi : \partial D \rightarrow \Gamma_1$ eine Abb., die $v_i$ auf $f(g_i)$ in $\Gamma_1$ und die Kante $\{v_i, v_{i+1}\}$ auf $\partial D$ auf Geodäten von $f(g_i)$ nach $f(g_{i+1})$.\\
$K_1'$ ist einfach zusammenhängend $\Longrightarrow$ $\phi$ erweitert zu $\phi': D \rightarrow K'_1$

\begin{itemize}
	\item $\forall x \in D$ definiere Elemente $h_x$ in $V(\Gamma_1) = G$ wie folgt:
	\begin{itemize}
		\item ist $\phi'(x)$ Ecke, so ist $h_x = \phi'(x)$
		\item ist $\phi'(x)$ in einer offenen Kante oder offenen 2-Zelle enthalten, so wähle nächste Ecke der Kante / 2-Zelle als $h_x$ 
	\end{itemize}
	Weil $\phi'$ stetig ist, ist $d(h_x, h_y) \leq \rho \forall x,y$, wenn $x,y$ nah genug aneinander sind in $D$.\\
	Es gilt $d(\phi(x), h_x) \leq \frac{1}{2} \forall x \in \partial D$ (alle Kanten in $\partial D$ haben Länge 1).
	\item Trianguliere $D$ so, dass $v_i \in \partial D$ wieder Ecken von $T$ sind und $\forall$ benachbarten $t,t' \in T$ gilt:
	\[d(h_t, h_{t'}) \leq \rho \]
	Metrik auf $D$ dazu so gewählt, dass $D$ reguläres $M$-Polygon in $\R^2$ ist
	\item Setze $l'_{|\partial D} = l$ und $l'(x) = f'(h_x) \forall x \in D^o$
	\paragraph{Behauptung}
	Für alle benachbarten Ecken $t,t'$ in der Triangulierung $T$ gilt:
	\[d(t,t') \leq M/3 \]
	Gilt diese Behauptung, so erweitert $l'$ auf $D$ so, dass Kanten in $T$ auf Geodäten in $\Gamma_2$ geschickt werden und nach Konstruktion Kreise der Länge $\leq M$ eine 2-Zelle beranden. Daraus würde folgen, dass $l'$ eine stetige Fortsetzung wäre.
	\paragraph{Bew. Beh.:} einziger interessanter Fall: $t\in D^o, t' \in \partial D$. Sei $t'$ zwischen $v_i$ und $v_{i+1}$. Es gilt:
	\begin{align*}
	d(l'(t), l'(t')) &= d(f'(h_t), l(t'))\\
	&\overset{ganz viele \triangle-Ugl.en}{\leq} d(f'(h_t), f'(h_t))  + d(f'(h_{t'}), f'(\phi(t')))\\
	&+ d(f'(\phi(t')) + f'(\phi(v_i)) ) + d(f'(\phi(v_i)), l(v_i) ) + d(l(v_i), l(t'))\\
	&\leq \ldots \leq M/3
	\end{align*}
\end{itemize}
\end{Beweis}




\subsection{Einschub: Simplizialkomplexe und CW-Komplexe}
\paragraph{Definition} Ein (abstrakter) \df{Simplizialkomplex} $\Delta$ ist eine Menge von Teilmengen einer Menge $V$, s.d. gilt:
\begin{enumerate}
	\item $\{v\} \in \Delta$ für alle $v \in V$
	\item $\emptyset\neq A \subset B \in \Delta \Longrightarrow A \in \Delta$
\end{enumerate}
Dimension von $a \in \Delta$ ist $dim(a) := |a| -1$
Dimension von $\Delta$ ist $dim(\Delta) = \sup_{a\in A}dim(a)$
Schreibe: $a$ ist $K$-Simplex, falls $dim(a) = K$

\paragraph{Beispiel}
\begin{enumerate}
	\item $V = \{1,2,3\}, \Delta = \{\{1\}, \{2\}, \{3\}, \{1,2\}, \{1,3\}\}$  ist Simplizialkomplex für V
	\item $V = \{1,2,3\}, \Delta = \{\{1\}, \{2\}, \{1,2\}\}$  ist kein Simplizialkomplex für V
	\item ungerichtete, einfache Graphen sind Simplizialkomplexe
	\item $V$ Menge, $\Delta = P(V)-\{\emptyset\} =: \grp{V}{} $ ist Simplizialkomplex;
\end{enumerate}

\paragraph{Allgemeiner: CW-Komplexe}
Ein CW-Komplex ist ein top. Raum, der schrittweise aus sog. Zellen zusammengeklebt worden ist. 
\paragraph{Definition}
Sei $X^{(0)}\subset \R^n$ eine diskrete Menge, diese Menge besteht aus den sogenannten \df{0-Zellen}.\\
Das \df{$n$-Skelett} $X^{(n)}$ entsteht aus den $X^{(n-1)}$ durch Ankleben von $n$-Zellen $D^n_i$ durch stetige Abb.
\[\phi_i : S^{n-1} = \partial D_n \longrightarrow X^{(n-1)} \]

Formal:
\[X^{(n)} = X^{(n-1)} \cup \bigcup_{i\in I}D^n_i/\sim \]
wobei $x\sim \phi_i(x)$ für $x \in \partial D_i^n$

Definiere den CW-Komplex durch $X = \bigcup_{n\geq 0} X^{(n)}$.

\paragraph{Beispiele}
\begin{enumerate}
	\item Graphen mit Doppelkanten sind CW-Komplexe
\end{enumerate}

%% 16.VL, 5.6.15

%% 17.VL, 9.6.15

\newpage
\section{Hyperbolische Gruppen}
\subsection{Oberes Halbebenenmodell von $\H^2$}
\[\H^2 := \set{z \in \Cz }{Im z > 0} \]
Riemannsche Struktur:
\[ds^2 = \frac{dx^2 + dy^2}{y^2} \]
hyberbolische Norm für Tangentenvektoren $v \in \T_z\H^2 = \R^2$
\[\norm{v}_{hyp} := \frac{\norm{v}_{eukl}}{im z} \]
direkte Definition einer Metrik auf $\H^2$:\\
Sei $\gamma: [0,1] \rightarrow \H^2 $ glatte Kurve, $\gamma(t) = x(t) + iy(t)$, dann ist die \df{Länge} von $\gamma$ definiert durch
\[L_{hyp}(\gamma) := \int_{0}^{1} \frac{\norm{\gamma'(t)}_{eukl}}{y(t)}dt\]
wir definieren die \df{hyperbolische Metrik} auf $\H^2$
\[d(z,w) := \inf_{\gamma : z \rightarrow w, glatt}L_\H(\gamma) \]

\subsection{Beispiel}
\begin{enumerate}
	\item
	$c : [0,1] \rightarrow \H^2, c(t) = i + (a-1)it$, $a \in \R$
	\[L_\H(c) = \ln(a) \]
	Außerdem gilt für beliebiges $\gamma : [0,1] \rightarrow \R^2$ von $i$ nach $a$
	\[L_\H = \int_0^1\frac{\sqrt{x'(t)^2 + y'(t)^2}}{y(t)}dt \geq \int_0^1\frac{y'(t)}{y(t)} = \ln a\]
	\[\Longrightarrow d(i,a) = \ln a\]
	\item
	$\gamma(t) = ai + t, a > 0, \gamma'(t) = 1, y(t) = a, x(t) =t$
	\[\Longrightarrow L^\H(\gamma) = \frac{1}{a} \]
	\[L(\gamma) \rightarrow 0, a \rightarrow \infty \]
	\[L(\gamma) \rightarrow \infty, a \rightarrow 1 \]
	Insbesondere ist $\gamma$ keine Geodäte.
\end{enumerate}

\subsection{Isometrien}
Isometrien von $\H^2$ sind die Möbiustransformationen. Eine \df{Möbiustransformation} (MT) ist eine Abbildung $\pi : \overline{\C} := \Cz\cup\{\infty\} \rightarrow \overline{\Cz}$ definiert durch
\[z \longmapsto \frac{az + b}{cz + d}, a,b,c,d \in \Cz \]

\subsection{Eigenschaften}
\begin{enumerate}
	\item MT sind dreifach transitiv auf $\overline{\Cz}$, d.h. sind $(z_1, z_2, z_3),(w_1, w_2, w_3) \in \overline{\Cz}^3$, dann existiert genau eine MT $T$ mit $T(z_i) = w_i$.
	\item MT bilden Kreise bzw. Geraden auf Kreise bzw. Geraden ab.
	\item $PSL(2,\R) = SL(2,\R)/\pm I $ operiert auf $\H^2$ durch Möbiustransformationen:
	\[A = \left(\begin{matrix}
	a & b \\
	c & d
	\end{matrix}\right) \longmapsto \frac{az + b }{cz + d}=: A.z\]
	\[Im(A.z) = \frac{Im z}{\left|cz + d\right|^2} > 0 \]
\end{enumerate}

\subsection{Satz}
Die Wirkung von $PSL(2,\R) \curvearrowright \H^2$ durch MT ist isometrisch und 
\[PSL(2,\R) \hookrightarrow Isom(\H^2) \]

\paragraph{Beweisskizze:}
\begin{itemize}
	\item Bestimme Erzeuger von $PSL(2, \R)$ (Gaußverfahren)
	\[\set{\Matrix{1}{r}{0}{1}, \Matrix{0}{-1}{0}{1}, \Matrix{\lambda}{}{}{\frac{1}{\lambda}}}{r \in \R, \lambda \in \R \setminus \{0\}}\]
	für Injektivität:\\
	\begin{itemize}
		\item betrachte: $\Matrix{-1}{}{}{-1} = id_{\H^2}$
		\item $\{I, -I\} \triangleleft SL(\R^2)$
		\item $T_A(z) = z \Longleftrightarrow A = \pm I$
	\end{itemize}	
\end{itemize}
\qed

\subsection{Bemerkung}
\[Isom(\H^2, d_\H) \cong PSL(2,\R) \cup \sigma \cdot PSL(\R^2) \]
wobei $\sigma = [z \mapsto -\overline{z}]$

\subsection{Satz: Geodätische}
Geodäten in $\H^2$ sind nach Bogenlänge parametrisierte Halbkreise mit Zentrum auf der $x$-Achse und Halbgeraden parallel zur $y$-Achse. Insbesondere gibt es für je zwei Punkte genau eine Geodätische, die diese verbindet.

\paragraph{Beweis}
Seien $z,w \in \H^2$
\begin{enumerate}
	\item Sei zunächst $z = ia, w= ib, b > a > 0$\\
	Man rechnet nach:
	\[L_\H (\gamma) = \ln(\frac{b}{a}) \]
	Für $\sigma$ gilt:
	\[L_\H (\sigma) = \ln\frac{b}{a} \]
	Ergo ist $\sigma$ Geodäte
	\item $z,w$ beliebig: betrachte 2 Teilfälle
	\begin{enumerate}
		\item $Re (z) \neq Re(w)$: Sei $C$ Kreis um Punkt $P$ auf $x$-Achse, der $z$ und $w$ enthält. ($P$ ist der Schnittpunkt der $x$-Achse und der Orthogonalen der Verbindungsstrecke $zw$, der den Mittelpunkt der Verbindungsstrecke enthält.)
		$(0, t_1)$ und $(0,t_2)$ seien die Randpunkte des Halbkreises $C$, setze dann
		\[g(u) := \frac{u - t_2}{u(t_2 - t_1)- t_1(t_2 - t_1)}\]
		\[g(t_1) = \infty, g(t_2) = 0\]
		Weiter bildet $g$ den Kreis $C$ auf die imaginäre Achse ab. (Nachrechnen mit Halbkreisparam.)
		
		\item $Re (z) = Re(w)$:\\
		Mit dreifach-Transitivität existiert MT mit $g(w) = w', g(z) = z'$ und $Re w = Re w', Im w' =0, Re z' = Re z, Im z' 0$. Nachrechnen: senkrechte Kurve $z \rightarrow w$ wird auf imaginäre Achse abbildet.
	\end{enumerate}
\end{enumerate}
\qed

\subsection{Bemerkung}
hyperbolische Kreise $S_{r,p} := \set{w\in \H^2}{d(w,p) = r}$ sehen exzentrisch aus

\subsection{Lemma}
Zu jeder Geodäten $\gamma$ und $\forall z \notin \gamma$ gibt es unendlich viele Geodäten $\sigma$ mit: $z \in \sigma $ und $\sigma || \gamma$, wobei 
\[\sigma || \gamma : \Leftrightarrow \sigma \cap \gamma = \emptyset \]


\subsection{Satz 6.9: Dreiecke sind dünn}
Jedes hyperbolische Dreieck hat Innenkreisradius $\leq \frac{1}{2} \ln 3$

\paragraph{Beweisskizze}
Sei ein Dreieck in $\H^2$ gegeben (die einzelnen Seiten sind Strecken von Geodäten). In diesem gibt es einen Hyperbolischen Kreis mit maximalen Radius.\\
Die beiden Seiten rechts und links, werden zu Geraden gebogen, wodurch der Innenkreisradius größer wird.\\
Es existiert eine MT, die die drei Ecken des neuen Dreiecks $p,q, \infty$ auf $-1, 1, \infty$ abbildet.\\
Das dadurch erhaltene Dreieck hat einen Innenkreisradius von $\leq \frac{1}{2} \ln 3$\qed

%% 18.V, 12.6.15

\subsection{Definition 6.10: Fuchssche Gruppen}
Eine Untergruppe $\Gamma < Isom(\H^2)$ heißt genau dann \df{Fuchssche Gruppe}, wenn sie eigentlich diskontinuierlich auf $\H^2$ wirkt.

\subsection{Definition 6.11: Fundamentalbereich}
Sei $G \curvearrowright X$ eigentliche Wirkung auf einem lokal kompaktem Hausdorffraum.\\
Ein (abgeschlossenes) $F\subset X$ heißt \df{Fundamentalbereich} von $G \curvearrowright X$, wenn gilt:
\begin{enumerate}
	\item[(a)] $G.F = X$
	\item[(b)] $g.F^o \cap F^o = \emptyset \forall g \in G\setminus\{e\}$
\end{enumerate}

\subsection{Bemerkung}
Jede Untergruppe einer Fuchsschen Gruppe ist wieder eine Fuchssche Gruppe.

\subsection{Beispiel: 6.12}
\begin{enumerate}
	\item $T_A$ MT zu 
	\[\Matrix{1}{1}{0}{1} \]
	$T_A(z) = z +1, \Gamma:= \grp{T_A}{}$
	Bahnen haben keinen Häufungspunkt. Stabilisatoren sind trivial. $\set{z \in \H^2}{Re z \in [0,1)}$ ist ein Fundamentalbereich. $\H^2 / \Gamma \cong (D^2)^o\setminus\{0\} \cong S^1\times \R$.
	
	\item $A = \Matrix{\lambda}{0}{0}{1/\lambda}, \Gamma = \grp{T_A}{}$ wirkt eigentlich.\\
	$\H^2\setminus \Gamma \cong S^1\times \R$
	
	\item $PSL(2, \Z) < SL(2, \R) < Isom (\H^2)$, $PSL(2, \Z)$ ist Fuchssche Gruppe\\
	Poincare: Theorie des groups fuchsiens (1882)
	
	\item alle abelschen Fuchsschen Gruppen sind zyklisch\\
	insbesondere is also keine Fuchssche Gruppe isomorph zu $\Z^n$ für $n > 1$
\end{enumerate}

\subsection{Definition 6.13: Hyperbolische metrische Räume}
Sei $(X,d)$ ein metrischer Raum. Ein geodätisches Dreieck $\Delta$ in $X$ ist ein Tripel von Geodäten (den Seiten des Dreiecks) $\gamma_1, \gamma_2,\gamma_3 : [0,l_i] \rightarrow X$ so, dass:
\[\gamma_1(0) = \gamma_3(l_3), \gamma_2(0) = \gamma_1(l_1), \gamma_3(0) = \gamma_2(l_2)\]
Ein geodätisches Dreieck $\Delta$ ist $\delta$-dünn für ein $\delta \geq 0$, falls
\[\gamma_i \subset U_j\cup U_k \]
wobei $U_i := \set{x \in X}{d(x, \gamma_i) \leq \delta}$ für $i,j,k$ verschieden.

\subsection{Beispiel}
\begin{itemize}
	\item Alle Dreiecke in $\R$ sind 0-dünn.
	\item Ebenfalls 0-dünn sind alle Dreiecke in simplizialen Bäumen.
	\item Dreiecke in $\H^2$ sind $\delta$-dünn.
\end{itemize}

\subsection{Definition 6.14: }
Ein metrischer Raum $(X,d)$ heißt \df{$\delta$-hyperbolisch}, wenn er geodätisch ist und alle geodätischen Dreiecke in $X$ $\delta$-dünn sind. Wir sagen $X$ ist \df{(Gromor)-hyperbolisch}, wenn es ein $\delta \geq 0$ gibt, s.d. $X$ $\delta$-hyperbolisch ist.

\subsection{Bemerkung}
$\delta$-hyperbolisch $\Longrightarrow$ $\delta'$-hyperbolisch $\forall \delta' \geq \delta$

\subsection{Beispiel 6.15}
\begin{itemize}
	\item $\R^n$ ist nicht hyperbolisch für $n > 1$.
	\item Geodätische Räume mit endlichem Durchmesser sind hyperbolisch.
	\item 0-hyperbolische Räume sind genau $\R$-Bäume.
\end{itemize}

\subsection{Satz 6.16: iterierte Dünnheit}
Sei $(X,d)$ $\delta$-hyperbolischer Raum, $P$ sei ein hyperbolisches Polygon mit Kanten $s_i : X_{i-1} \rightarrow X_{i}, i = 1\ldots n$. Setze
\[Y:= \bigcup_{i=1}^n Bild(s_i) \]
Dann gilt $\forall x \in S_0$
\[d(x,Y) \leq k\cdot \delta \]
mit $k := \lceil\log_2n \rceil$

\paragraph{Beweis}
\begin{itemize}
	\item Sei $n = 2^l$ für ein $l \in \N$, zu zeigen:
	\[d(x,Y) \leq l\delta \]
	Induktion über $l$:
	\paragraph{I.A.:} $l = 1$, dann $n = 2$, $P$ Dreieck $\Longrightarrow$ Behauptung.
	\paragraph{Induktionsschritt:} $l \rightarrow l + 1$\\
	$n = 2^{l+1}$, wähle $x \in Bild S_0$ und geodätische Segmente $[X_0, X_{2^l}], [X_n, X_{2^l}] $\\
	Das Dreieck $X_0, X_{2^l}, X_n$ ist $\delta$-dünn, also existiert $t \in [X_0, X_{2^l}]$ mit $d(t,x) \leq \delta$.\\
	Nach I.V. ist
	\[d(t, Y') \leq l\delta \]
	für $Y' = \bigcup_{i= 1}^{2^l} s_i$
	\[\Rightarrow \exists t' \in Y' \text{ mit } d(t,t') = d(t,Y') \]
	\[\Rightarrow d(x,Y) \leq d(x,t') \leq d(x,t) + d(t,t') \leq \delta + l \delta \]
	\item Sei $n$ beliebig: füge $r$ Zwischenpunkte auf $Y$ ein, sodass
	\[n + r = 2^l\]
	\qed
\end{itemize}

\subsection{Definition 6.17: }
$(X,d)$ : metrischer Raum, $c \geq 1, b \geq 0$\\
Ein \df{$(c,b)$-quasi-geodätisches Dreieck} ist analog zu geodätisches Dreieck definiert mit $\gamma_i$ $(c,b)$-quasi-Geodäten.\\
$\delta$-dünn genau analog definiert.

\subsection{Definition 6.18: }
$X$ heißt \df{$(c,b,\delta)$-quasi-hyperbolisch}, falls $X$ $(c,b)$-quasi-geodätisch und alle $(c,b)$-quasi-geodätisch Dreiecke $\delta$-dünn sind.\\
$X$ heißt \df{$(c,b)$-quasi-hyperbolisch}, wenn $\delta \geq 0$, s.d. $X$ $(c,b,\delta)$-quasi-hyperbolisch.

\subsection{Bemerkung}
Im Allgemeinem ist es schwieriger zu zeigen, dass $X$ quasi-hyperbolisch ist, als, dass $X$ hyperbolisch ist.

\subsection{Proposition 6.19: QI-Invarianz von quasi-hyperbolisch}
$(X,d), (Y, e)$ metrische quasi-isometrische Räume, dann:
\begin{itemize}
	\item $X$ quasi-geod. $\Longleftrightarrow$ $Y$ quasi-geod.
	\item $X$ quasi-hyp. $\Longleftrightarrow$ $Y$ quasi-hyp.
\end{itemize}


%% 16.06.15, 19.VL

\paragraph{Beweis}
$f:X \rightarrow Y$ sei eine $(c,c)$-Quasi-Isometrie mit $c$-dichtem Bild.
\begin{itemize}
	\item Ohne Einschränkung sei $Y$  $(c,c)$-quasi-geod., seien $x,x'\in X$. Dann existiert eine $(c,c)$-Quasi-Geodäte $\gamma$ von $f(x)\rightarrow f(x')$.\\
	Mit Auswahlaxiom finden wir $\gamma':[0,l] \rightarrow X$ mit 
	\[\gamma'(0) = x, \gamma'(l)  = x' \text{ und } e(f(\gamma'(t)), \gamma(t)) \leq c \forall t \]
	Wie in Satz 4.7 zeigt man, dass $\gamma'$ $(c, max(3c^2, 3))$-quasi-Geodäte ist.\\
	Daraus folgt: $X$ ist quasi-geodätisch.
	
	\item Ohne Einschränkung sei $Y$  quasi-hyp., der obere Teil zeigt, dass $X$ quasi-geodätisch ist. Es gibt also Konstanten $c\geq 1, d \geq 0$, s.d. $Y$ $(c,d)$-quasi-hyp. und $X$ $(c,d)$-quasi-geod.\\
	Seien $c' \geq  c, d' \geq d$ und $\gamma_i$, $i = 1,2,3$, $(c',d')$-quasi-geodätisches Dreieck in $X$.\\
	Das Bild $(f\circ \gamma_1, f\circ \gamma_2, f\circ \gamma_3)$ unter $f$ ist $(c'', d'')$-quasi-geod. Dreieck in $Y$.\\
	Weil $Y$ quasi-hyp., ist $(f\circ \gamma_1, f\circ \gamma_2, f\circ \gamma_3 )$ $\delta$-dünn, für $\delta \geq 0$.\\
	Man rechne nach:
	\[Bild(\gamma_1) \subset \bigcup_{c\delta + cd}(Bild\gamma_2 \cup Bild \gamma_3) \]
	weil $f$ eine $(c,d)$- quasi-isom. Einbettung ist mit $\delta$-dichtem Bild.\\
	Also ist $X$ quasi-hyperbolisch mit Konstanten $(c', d', c\delta + cd)$. \qed
\end{itemize}

\subsection{Bemerkung}
In 6.19.2 reicht es vorauszusetzen: $X,Y$ metr., $Y$ quasi-hyp., $X$ quasi-geod., $f: X \rightarrow Y$ qi. Einbettung, dann ist $X$ quasi-hyp.

\subsection{Satz 6.20}
$(X,d)$ geod. metr. Raum.
\[X \text{ hyperbolisch } \Longleftrightarrow X \text{ quasi-hyperbolisch }  \]

\paragraph{Beweis unter Benutzung von 6.21}
$\Longleftarrow$: klar.\\
$\Longrightarrow$: Sei $X$ $\delta$-hyperbolisch, $c \geq 0,b \geq 1$. Wir zeigen:\\
$\exists \delta' \geq 0$, s.d. $X$ $(c,b,\delta')$-quasi-hyp.\\
Sei $\Delta := (\gamma_1, \gamma_2, \gamma_3)$ ein $(c,b)$-quasi-geod. Dreieck.\\
Weil $X$ geodätisch, existieren Geodäten $\gamma_i'$ mit selben Endknoten wie $\gamma_i$.\\
$X$ hyp. $\Longrightarrow \Delta' =(\gamma_i')_{i = 1,\ldots, 3}$ $delta$-dünn.\\
Mit (6.21) folgt:
\[Bild(\gamma_i') \subset U_k(Bild \gamma_i) \]
\[Bild(\gamma_i) \subset U_k(Bild \gamma_i') \]
$X$ $\delta$-hyp. $\Longrightarrow$ $Bild(\gamma_i') \subset U_\delta(Bild \gamma_j \cup Bild \gamma_k)$\\
$\Longrightarrow Bild(\gamma_i) \subset U_k(Bild(\gamma_i')) \subset  U_k(U_\delta (Bild\gamma_j \cup Bild \gamma_k)) \subset U_{2k + \delta}(Bild\gamma_j \cup Bild \gamma_k)$\qed


\subsection{Satz 6.21 (Stabilität von Quasi-Geodäten)}
Seien $c,\delta \geq 0, b \geq 1$ Konstanten. Dann existiert $k = k(\delta, c,b) \geq 0$, s.d. gilt:\\
Ist $X$ $\delta$-hyp. Raum, $\gamma : [0,l] \rightarrow X$ eine $(c,b)$-quasi-Geodäte und $\gamma' : [0,l'] \rightarrow X$ eine Geodäte mit $\gamma'(0) = \gamma(0)$ und $\gamma'(l') = \gamma(l)$.\\
Dann gilt: $Bild(\gamma') \subset U_k(Bild \gamma)$ und $Bild(\gamma) \subset U_k(Bild \gamma')$

\paragraph{Bemerkung} Die Voraussetzung $X ~\delta$-hyperbolisch ist hier wesentlich.

\paragraph{Beweis unter Benutzung von 6.23 und 6.24}
Ohne Einschränkung sei $\gamma$ stetig (geht wg. 6.24).\\
Sei $\gamma'$ Geodäte, wir schätzen zunächst Abstand ab:
\[D:= \sup\set{d(x, Bild(\gamma))}{ x\in Bild\gamma' } \]
Sei $x_0$ Punkt, an dem das Supremum angenommen wird. Der offene Ball mit Radius $d$ um $x_0$ schneidet $\gamma$ nicht.\\
Sei $y$ Punkt auf $\gamma_{|[0,t]}$, ($t$ so, dass $\gamma'(t) = x_0$), so gewählt, dass $d(y,x_0) = 2D$ oder, wenn $d(\gamma'(0), x_0)< 2D$, dann setze $y = \gamma'(0)$.\\
Wähle $y',z' \in \gamma$ mit $d(y,y') \leq D$ und $d(z,z') \leq D$. Verbinde $y$ mit $y'$, $z$ mit $z'$ durch geod. Segmente. Betrachte jetzt den Weg\\
$\sigma := $
\begin{enumerate}
	\item Geodäte $y\mapsto y'$, dann Teilstück  
	\item dann Teilstück von $\gamma$ von $y'$ nach $z'$
	\item Geodäte $z'\mapsto z$
\end{enumerate}
Nach Konstruktion und 6.23:
\[D \overset{D vermeidet den D-Ball um x_0 }{\leq} d(x_0, \sigma) \leq \delta \norm{\log_2(L(\sigma))} + 1 \]
\[\Longrightarrow d(y',z') \leq d(y',y) + d(y,z) + d(z,z') \leq D + 4D+ D \]
Mit 6.24 ist
\[L(\sigma) \leq cd(y',z') + b + 2D \leq c6D + b +2D\]
\[\overset{6.23}{\Longrightarrow} D = d(x_0, \sigma) \leq \sigma \norm{\log_2(L(\delta))} + 1 \]
\[ D- 1 \leq \delta \norm{\log_2(D(6c + 2) + b)} \]
$\log_2$ wächst langsamer als $D-1$, also ist $D$ beschränkt durch eine Konstante $D_0(c,b,\delta)$.\\
Wir müssen noch zeigen: 
\[\gamma \subset U_{D_1}(\gamma') \]
für $D_1 = D_1(D_0, c,b)$\\
Sei $[r,s] \subset [0,l]$ maximales Teilintervall, s.d. $\gamma_{|[r,s]}$ außerhalb der $D_0$-Umgebung von $\gamma'$ liegt;\\
wenn so ein $[r,s]$ nicht existiert, setze $D_1 = D_0$, fertig.\\
Jeder Punkt von $\gamma'$ liegt in $U_{D_0} (\gamma)$.\\
Somit gilt
\[ \gamma' \subset U_{D_0}(\gamma_{|[0,r]} \cup \gamma_{|[s,l]}) \]
Weil $\gamma, \gamma'$ stetig und $[0,l']$ Definitionsbereich von $\gamma'$ zusammenhängend, existiert $t'\in [0,l']$, sowie $r'\in [0,r']$ und $s' \in [s,l]$ mit $d(\gamma'(t), \gamma(r')) \leq D_0$ und
\[ d(\gamma'(t'), \gamma(s')) \leq D_0 \]
\[\Longrightarrow d(\gamma(r'), \gamma(s')) \leq 2D_0 \]
Und wir erhalten: 
\[ L(\gamma_{|[r,s]}) \leq L(\gamma_{|[r',s']}) \overset{6.24}{\leq} cd(\gamma(r'), \gamma(s')) + b \leq c2D_0 + b=:D_1 \]
\[\Longrightarrow \gamma_{|[r,s]} \subset U_{D_1}(\gamma')  \]

Man überlegt sich $\forall p,q$ mit $\gamma_{|[p,q]} \not \subset U_{D_0}(\gamma')$, dass
\[\gamma_{|[p,q]} \subset U_{D_1} (\gamma')\]
also gilt wegen $D_1 > D$, dass
\[\gamma \subset U_{D_1}(\gamma') \]
\qed



\subsection{6.22 Korollar von 6.20}
$X,Y$ geod. metr. Räume, $X \qi Y$.\\
Dann:
\[X \text{ hyperbolisch} \Longleftrightarrow Y \text{ hyperbolisch} \]

\subsection{6.23 Hilfslemma (Abstand von Kurven und Geodäten)}
$\delta \geq 0$, $(X,d)$ $\delta$-hyp., sei $\gamma : I \rightarrow X$ stetige Kurve in $X$ mit Endpunkten $p,q$. Sei $\gamma'$ Geodäte von $p$ nach $q$. Dann gilt $\forall t\in I$
\[d(\gamma'(t), Bild(\gamma)) \leq \delta \norm{\log_2(L(\gamma))} + 1 \]
wobei 
\[L(\gamma) := \sup \set{\sum d(\gamma(t_i), \gamma(t_{i+1})}{(t_0, \ldots, t_n) \text{ Zerlegung von } I} \]

\paragraph{Beweisskizze}
Ohne Einschränkung: $1 < L(\gamma) =: L' < \infty$ und $\gamma: [0,l] \rightarrow X$ nach Bogenlänge parametrisiert, $p = \gamma(0), q = \gamma(l)$. Wähle $N \in \N_0$ groß genug, sodass
\[\frac{L(\gamma)}{2^{N+1}} < 1 \leq \frac{L(\gamma)}{2^N} \]
Sei $\gamma' : [0, l'] \rightarrow X$ und $t$ fest in $[0, l']$.\\
$X$ hyperbolisch $\Longrightarrow$ wir finden Punkte $x_1, \ldots, x_n$ in $X$, s.d. gilt:
\[d(\gamma'(t), x_1 ) , d(x_i, x_{i+1}) \leq \delta \]
mit $x_i$ auf einer Geodäten der Länge $L(\gamma)/2^i$ mit Endpunkten in $Bild \gamma$.\\
Sei $y$ der Endpunkt der Geodäten, die $x_i$ enthält, der näher an $\gamma'$ ist.\\
Es gilt:
\[d(\gamma'(t) , Bild \gamma) \leq d(\gamma'(t), y) \leq d(\gamma'(t), x_i) + d(x_i,y) \leq i \delta + L(\gamma)/(2^{i+1}) \leq \delta \norm{\log_2(L(\gamma))} + 1 \]
weil $\frac{L(\gamma)}{2^{N+1}} < 1$, $N \leq \norm{\log_2(L(\gamma))}$.\qed

%% 19.06.15, 20.VL

\subsection{Satz 6.24: Hilfslemma: Approximation durch stetige Quasi-Geodäten}
$(X,d)$ geod. metr. Raum, $c \geq 1, b \geq 0$, dann existiert $c' \geq 1, b \geq 0$, s.d. gilt:\\
Ist $\gamma$ $(c,b)$-quasi-Geodäte $\gamma : [0,l] \rightarrow X$, dann existiert eine stetige $(c',b')$-quasi-Geodäte $\gamma' : [0,l] \rightarrow X$ mit $\gamma'(0) = \gamma(0), \gamma'(l) = \gamma(l)$ und für die gilt
\begin{enumerate}
	\item $\forall s \leq t $ in $[0,1]$:
	\[L(\gamma'_{|[s,t]}) \leq c'd(\gamma'(s) , \gamma'(t)) + b' \]
	\item 
	\[Bild(\gamma') \subset U_{c+ b}(Bild(\gamma))  \]
	\[Bild(\gamma) \subset U_{c+ b}(Bild(\gamma'))  \]
\end{enumerate}

\paragraph{Beweisskizze}
$I := [0,l) \cap \Z$
\begin{enumerate}
	\item[1.Schritt] $\gamma'(t) := \gamma(t) \forall t \in I$
	\item[2.Schritt] Erweitere $\gamma'$ auf $[0,l]$ durch $\gamma'(l) := \gamma(l)$ und zwischen benachbarten $t,t+1$ in $I$ bzw. zw. $\max I$ und $l$ durch Geodäten
	\end{enumerate}
\qed

\subsection{Definition 6.25: Hyperbolische Gruppen}
Eine endlich erzeugte Gruppe $G$ heißt (Gromor)-\df{hyperbolisch}, wenn für ein (und somit für alle) endliche Erzeugersysteme $S$ der Cayleygraph von $G$ bzgl. $S$ hyperbolisch ist.

\paragraph{Bemerkung}
Auf dem Cayleygraphen betrachten wir die geodätische Graphmetrik (geom. Realisierung) definiert durch die Eigenschaft, dass alle Kanten die Länge 1 haben.

\subsection{Satz 6.24 (hyperbolisch ist geometrisch)}
Seien $G, H $ endlich erzeugt, $G \qi H$, so gilt:\\
\[G \text{ hyperbolisch } \Leftrightarrow H \text{ hyperbolisch} \]
\paragraph{Beweis} folgt aus 6.22, weil \grqq hyperbolisch\glqq\ QI-Invar. für metr. Räume.\qed

\subsection{Beispiele 6.25}
\begin{enumerate}
	\item freie Gruppen
	\item endliche Gruppen
	\item $\Z, \R$
	\item Fuchssche Gruppen
	\item $\Z^n, \R^n$ nicht hyperbolisch für $n > 1$
	\item $\C$ nicht hyperbolisch
	
	%%23.06.15, 21.VL
	\item Parkettierungsgruppen von ($\H^2$)\\
	$P$ konvexes, kompaktes (d.h. keine Ecken auf dem Rand) Polygon in $\H^2$ mit Ecken in $x_1, \ldots, x_n, n\geq 3$\\
	\paragraph{Annahme} $\forall i$ ist der Innenwinkel an $x_i$ in $P$ von der Form $\pi/p_i, p_i \in \Z_{\geq 2}$.\\
	Falls $\sum_{j=1}^n \frac{1}{p_i} < n - 2$, dann existiert ein solches Polygon (dünnes Polygon in $\H^2$)\\
	\paragraph{Annahme} $P$ \df{regulär}, d.h. $p_i = p \forall i$\\
	betrachte Spiegelungen $s_i$ an Seite $x_i \rightarrow x_{i+1}$ in $P$, diese erzeugen eine Gruppe $G$\\
	\[G = \grp{s_1, \ldots, s_n}{s_i^2, (s_is_{j-1})^{p}, \forall i \forall j }\]
	
	\paragraph{Satz (Poincare)}
	$G$ wirkt eigentlich  auf $\H^2$ mit $P$ als Fundamentalbereich.\\
	Für $x \in P^o$ ist $G.x$ quasi-isom. zu $\H^2$
\end{enumerate}

\newpage
\section{Wortproblem für hyperbolische Gruppen}

\subsection{Definition}
$G = \grp{S}{R}$ endl. präsentierte Gruppe.\\
Wir sagen:
\[das~Wortproblem ~ist ~für ~G ~lösbar\]
wenn es einen Algorithmus gibt, der für alle $w \in (S\cup S^{-1})^*$ entscheidet, ob $w= 1$ in $G$ oder nicht.\\
\paragraph{Genauer} Die Mengen 
\[\set{w \in (S\cup S^{-1})^*}{w = 1} \text{ und } \set{w \in (S\cup S^{-1})^*}{w \neq 1}\]
sind rekursiv aufzählbar.

\paragraph{Beispiele}
\begin{enumerate}
	\item freie Gruppen und freie abelsche Gruppen sind sogar Chomsky 2
	\item alle endlichen Gruppen sind Chomsky 3
	\item Gruppen sind genau dann Chomsky 3, dann, wenn $((S\cup S^{-1})^* : L) < \infty$
\end{enumerate}

\paragraph{Bemerkung}
\begin{enumerate}
	\item Für eine beliebige Gruppe ist das Wortproblem im Allgemeinem nicht lösbar.
	\item Folgende Klassen haben lösbares WP:
	\begin{enumerate}
		\item Coxetgruppen
		\item Zopfgruppen
		\item Fundamentalgruppen geschlossener orientierter Flächen
		\item hyperbolische Gruppen
	\end{enumerate}
\end{enumerate}

\subsection{Satz (Gromor, Olshanskii 1992)}
$k \in \Z_{\geq 2}, A = S\cup S^{-1}, S = \set{s_1,\ldots,s_n}{}, i \geq 0$ und Folge von natürlichen Zahlen $n_1,\ldots,n_i$\\
Sei $\mathcal{N} := \mathcal{N}(k,i,n_1,\ldots, n_i)$ die Anzahl der Präsentationen $\grp{s_1,\ldots, s_k}{r_1,\ldots, r_i}$, $r_j$ reduziertes Wort in $A$ mit Länge $l(r_j) = n_j \forall j = 1,\ldots, i$.\\
$N_h \leq \mathcal{N}$ sei die Anzahl der hyperbolischen Gruppen dieser Form.\\
Für $n := \min\set{n_1,\ldots, n_i}{}$ gilt:
\[\lim\limits_{n \rightarrow \infty} \frac{N_h}{N} = 1 \]

\subsection{Satz}
Hyperbolische Gruppen haben lösbares Wortproblem, d.h. für ein endliches Erzeugendensystem $S$ von $G$ existiert $R \subset (S\cup S^{-1})^*$ so, dass $G \cong \grp{S}{R}$ und das Wortproblem für $\grp{S}{R}$ lösbar ist.

\paragraph{Beweis unter Benutzung von 7.4 und 7.5} Offensichtlich.

\subsection{Definition}
Eine endliche Präsentierung $\grp{S}{R}$ ist eine \df{Dehn-Präsentation}, wenn für ein $n \in \N$ Wörter $u_1,\ldots, u_n, v_1,\ldots, v_n$ existieren mit
\begin{enumerate}
	\item $R = \set{u_1v_1^{-1},\ldots, u_nv_n^{-1}}{}$
	\item $\forall j$ gilt: $v_j$ ist echt kürzer als $u_j$
	\item $\forall w \in (S\cup S^{-1})^*\setminus\{\epsilon\}$, die das neutrale Element präsentieren, existiert ein $j \in \{1,\ldots, n\}$, s.d. $u_j$ ein Teilwort von $w$ ist.
\end{enumerate}


\subsection{7.4 Satz: Dehn-Algorithmus}
Ist $\grp{S}{R}$ eine Dehn-Präsentation, dann ist das Wortproblem lösbar.

\paragraph{Beweis}
$R = \set{u_iv_i^{-1}}{i = 1,\ldots, n}$, sei $w$ ein beliebiges Wort und führe strukturelle Induktion nach der Länge von $w$.
\begin{enumerate}
	\item[I.A.] Offensichtlich ist $w$ trivial.
	\item[I.S.] Existiert ein $u_i$, sodass $u_i$ ein Teilwort von $w$ ist, so ersetze dieses durch $v_i$. Das dadurch erhaltene Wort hat eine echt kleinere Länge, weshalb man nach Induktion-Voraussetzung entscheiden kann, ob dieses das Neutralelement präsentiert.\\
	Ist kein Teilwort von $w$ gleich einem $u_i$, so kann $w$ nach Dehn-Eigenschaft nicht das Neutralelement präsentieren.\qed
\end{enumerate}

\subsection{7.5 Satz}
$G$ hyperbolische, von $S$ endlich erzeugte Gruppe; dann existiert $R \subset (S\cup S^{-1})^*$ so, dass $\grp{S}{R}$ Dehn-Präsentation und $G \cong \grp{S}{R}$.

\paragraph{Beweis unter Benutzung von 7.6}
Sei $R := \set{uv^{-1}}{u,v \in (S\cup S^{-1})^*, d_S(1, \pi(u)) < \norm{u} \leq D, \norm{v} = d_S(1, \pi(u)), \pi(u) = \pi(v) } $ wobei $D:= \lceil8\delta \rceil$, $\pi$ ist kanonische Projektion $F(S) \rightarrow G$\\ %%\cup \set{ss^{-1}e}{s\in (S\cup S^{-1})^*}
Betrachte Abbildung
\[\phi : \grp{S}{R} \rightarrow G \]
induziert durch $\phi_{|S} = id_{S}$, $\grp{S}{} = G$, also ist $\phi$ surjektiv.\\

\subparagraph{Behauptung} $\phi$ injektiv und $\grp{S}{R}$ Dehn-Präsentation.
\[\Longleftrightarrow \ker \pi  = \grp{R}{}_{F(S)}^\vartriangleleft \]
\subparagraph{Beweis}
Definition von $R$ liefert
\[ \grp{R}{}_{F(S)}^\vartriangleleft \subset \ker \pi \]
Sei jetzt $w \in (S\cup S^{-1})^*$ mit $\pi(w) = 1$\\
Zu Zeigen durch Induktion über Wortlänge: $w \in \grp{R}{}_{F(S)}^\vartriangleleft$ und $w$ besitzt Teilwörter nach Definition von Dehn-Präsentation.\\
\begin{enumerate}
	\item[$l(w) = 0$ ] $w = e$
	\item[$l(W) > 0$] 
	\begin{enumerate}
		\item $w$ nicht reduziert, dann besitzt $w$ ein Teilwort der Form $ss^{-1}e$ mit $s \in S\cup S^{-1}$.\\
		$ss^{-1}e \in R$ mit $u = ss^{-1}, v = e$.
		\item $w$ reduziert, dann entspricht $w$ einem geschlossenem Kantenpfad in $Cay(G,S)$\\
		7.6 $\Longrightarrow$ $\exists$ nicht-geod. Teilpfad und somit nicht-geod. Teilwort, das abgekürzt werden kann.\\
		$\Longrightarrow$ $w$ besitzt Zerlegung $w = w'uw''$, wobei $u$ nicht geodätisch und $d(1, \pi(u)) < l(u) \overset{7.7}{\leq} D$\\
		Wähle $v \in (S\cup S^{-1})^*$, sodass $\pi(u) = \pi(v)$ und $l(v) = d_S(1, \pi(u))< l(U)$\\
		Definition von $R$: $1 = \pi(w) = \pi(w') \pi(u) \pi(w'') \overset{\pi(u) = \pi(v)}{=} \pi(w'vw'') $\\
		Nach IV. ist $w'vw'' \in \grp{R}{}_{F(S)}^\vartriangleleft$ und besitzt gesuchte Teilwörter.\\
		$\Longrightarrow$ $\exists k \in \ker \pi $ mit $w = w'vw''k$ \qed
	\end{enumerate}
\end{enumerate}

\subsection{7.6 Lemma: Abkürzungslemma}
$G$ hyperbolische, von $S$ endlich erzeugte Gruppe; $Cay(G,S)$ $\delta$-hyperbolisch für ein $\delta > 0$.\\
Falls $\gamma : [0,n] \rightarrow Cay(G,S)$ ein stückweiser linearer nach Bogenlänge parametrisierter geschlossener Weg ist, dann existieren $t,t'\in [0,n]$ mit 
\[l(\gamma_{|[t,t']}) \leq 8\delta \text{ und } \gamma_{|[t,t']} \text{ nicht geodätisch} \]

\paragraph{Beweis unter Benutzung von 7.7}
\begin{enumerate}
	\item Zeige: $\gamma$ kann für $c > 8\delta$ keine $c$-lokale Geodäte sein.\\
	\paragraph{Annahme} $\exists c > 8 \delta$, sodass $\gamma$ eine $c$-lokale Geodäte in $Cay(G,S)$.\\
	Weil $\gamma(0) = \gamma(n)$ muss $n > 8\delta$\\
	Lemma 7.7 $: \gamma 2\delta$-nahe an jeder Geodäte von $\gamma(0) \rightarrow \gamma(n) = \gamma(0)$,\\
	also ist $\gamma$ $2\delta$-nahe am Punkt $\gamma(0) = $ konstante Geodäte\\
	\[\Longrightarrow \gamma \subset U_{2\delta}(\gamma(0)) = B_{2\delta}(\gamma(0)) \]
	\[ 4\delta \geq diam B_{2\delta}(\gamma(0)) \overset{\gamma \text{ geschlossen, nach BL param.}}{\geq} d_S(\gamma(0), \gamma(5\delta)) \overset{BL, c\text{-lokale Geodäte}}{=} 5\delta \]
	Was ein Widerspruch ist.\\
	$\Longrightarrow$ $\gamma$ kann keine $c$-lokale Geodäte sein für $c > 8\delta$
	\item Also $\exists t, t' \in [0,n]$ mit
	\[ \norm{t - t'} \leq 8\delta \text{ und } d(\gamma(t), \gamma(t')) \neq \norm{t -t'} \]
	Insbesondere ist $\gamma_{|[t,t']}$ keine Geodäte. \qed
\end{enumerate}


\subsection{Lemma 7.7: Trapping für lokale Geodäten}
$\delta \geq 0, c > 8 \delta$ und $(X,d)$ $\delta$-hyperbolisch.\\
$\gamma:[0,l] \rightarrow X$ \df{$c$-lokale} Geodäte, d.h.\\
$\forall t, t' \in [0,l]$ mit $\bet{t - t'} < c$ gilt
\[d(\gamma(t), \gamma(t')) = \bet{t -t'}\]
Ist $\gamma':[0,l'] \rightarrow X$ Geodäte mit $\gamma'(0) = \gamma(0), \gamma'(l) = \gamma(l)$, so gilt
\[\gamma \subseteq U_{2\delta}(\gamma') \]

%% 26.06.15, 22. VL

\subsection{Konjugationsproblem}
Problem: Gegeben zwei Elemente $u,v$; existiert ein $g$, s.d.
\[gug^{-1} =v \]

\paragraph{Beispiel: freie Gruppen}
Wort $w = s_0\cdots s_n$ in $F(S)$ ist \df{zyklisch reduziert}, wenn $s_i \neq s_{i+1}^{-1} \forall i = 0,\ldots, n-1$ und $s_0 \neq s_n^{-1}$, d.h. alle zyklischen Permutationen von $w$ sind reduziert.\\
Gegeben beliebiges Wort $w$, so kann $w$ wie folgt zyklisch reduziert werden:\\
Wiederhole
\begin{itemize}
	\item Lösche Teilwörter $s^{-1}s$, $s \in S\cup S^{-1}$
	\item Lösche $s_0$ und $s_n$, falls $s_0^{-1} = s_n$
\end{itemize}
Das zyklisch reduzierte Wort, das man so erhält, ist eindeutig bis auf zyklische Permutationen.\\
Z.Bsp. $bab^{-1}babb^{-1} \longrightarrow baa \text{ oder } aab$\\
$u$ und $v$ sind konjugiert genau dann, wenn ihre zyklischen Reduktionen zyklische Permutationen voneinander sind.\\
Algorithmus:
\begin{enumerate}
	\item reduziere $u,v$ zyklisch
	\item betrachte alle zyklischen Permutationen von $u$ und $v$ und prüfe, ob Schnitt nicht leer ist. (Hierzu muss man das Wortproblem lösen können.)
\end{enumerate}

\subsection{Definition 7.9}
Ein Wort $w$ über $S$, $S$ endl. EZS von $G$, heißt \df{vollständig reduziert}, wenn $w$ und alle seine zyklischen Permutationen Geodäten in $Cay(G,S)$ sind.

\subsection{Lemma 7.10}
$G$ $\delta$-hyp. bzgl. $S$, $\norm{S} < \infty$.\\
Repräsentieren zwei vollständige reduzierte Wörter $u,v$ zwei konjugierte Elemente in $G$, dann gilt:
\begin{enumerate}
	\item $\max\{l(u), l(v)\} \leq 8\delta + 1$ oder
	\item $\exists$ zykl. Permutationen $u',v'$ von $u,v$ und ein Wort $w$ mit
	\[l(w)  \leq 2\delta +1  \]
	sodasss $wu'w^{-1} = v'$
\end{enumerate}

\paragraph{Beweis}
Sei $w$ geodätisches Wort, s.d. $wuw^{-1} = v$.\\
Betrachte geod. Viereck $Q $ mit Kanten$ w,u, w^{-1},v^{-1}$ in $Cay(G,S)$:\\
Man kann zeigen, dass es zyklische Permutationen von $u$ und $v$ gibt, s.d. jede Ecke auf $u$ Abstand $\geq l(w)$ zur Seite $v$ hat.\\
Jede Ecke auf $u$ hat Abstand $\geq l(w)$ zu jeder Ecke $v$.\\
Sei $P$ Mitte von Pfad $u$, dann existiert $q$ auf eines der drei anderen Kanten mit $d(p,q) \leq 2\delta$ (weil $G$ $\delta$-hyp.).\\
Ist $q$ auf $v$, dann haben die am nächsten an $p$ und $q$ liegenden Ecken Abstand $\leq 2\delta + 1$.\\
Dann ist $l(w) \leq 2\delta + 1$.\\
Ist $q$ auf einer der vertikalen Seiten mit Enden $x$ (auf $u$) und $y$ (auf $v$), dann gilt:
\[ l(w) - \frac{1}{2} \leq d(p,y) \leq 2\delta + d(q,y)  \]
und
\[ d(q,y) = l(w) = d(x,q) \]
\[\Longrightarrow d(x,q)  \leq 2\delta + 1 \]
\[ d(x,p)  \leq d(p,q) + d(q,x) \leq 4\delta + \frac{1}{2} \]
\[l(w) = 2d(x,p) \]
\[ \Longrightarrow l(u) \leq 8\delta + 1 \]
Analog zeige
\[l(v) \leq 8\delta + 1 \]
\qed

%%% 23.VL, 7.7.15

\subsection{Lemma 7.11 (Trapping von $c$-lokalen-Geodäten)}
$X$ $\delta$-hyperbolisch metr. Raum, dann $\exists C = C(\delta)$, s.d. $\forall$ 4-Ecke $Q$ in $X$ mit $(8\delta + 1)$-lokal-geodätische Seiten gilt:\\
Jede Seite von $Q$ liegt in der $C$-Umgebung der drei anderen Seiten.

\subsection{Lemma 7.12 (Verallgemeinerung von 7.10 mit Hilfe von 7.11)}
$G$ $\delta$-hyperbolische Gruppe bzgl. EZS $S$, d.h. $Cay(G,S)$ ist $\delta$-hyperbolisch.\\
Dann $\exists k > 0$, $k = k(\delta)$ konstant, s.d. $\forall$ Wörter $u,v$, die in $G$ konjugierte Elemente repräsentieren, gilt:\\
Sind $u,v$ und alle ihre zyklischen Permutationen $(8\delta + 1)$-lokal-geodätisch, dann ist entweder
\begin{enumerate}
	\item $\max\{ l(u),l(v) \} \leq k$ oder
	\item $\exists $ Wort $w$, $l(w) \leq k$, s.d.
	\[wuw^{-1} =v^{-1} \]
	für $u',v'$ geeignete zykl. Permutation von $u$ bzw. $v$.
\end{enumerate}

\subsection{7. 13: Algorithmus für Konjugationsproblem}
Input: 2 Wörter $u,v$
\begin{itemize}
	\item Finde in $u$ und $v$ und allen zykl. Permutationen Teilwörter der Länge $\leq 8\delta + 1$, die nicht Geod. sind und ersetze durch äquivalente Teilwörter, die geod. sind.
	\item Wiederhole 1.Schritt bis $u', v'$ und alle zykl. Permutationen $(8\delta + 1)$-loakal-geodätisch.
	\item mit Lemma 7.12 existiert endl. Menge $\Sigma \subset (S+S^{-1})^*$, s.d. $u$ konjugiert ist, gdw. $\exists w \in \Sigma$ mit $wu''w^{-1} = v''$ für zykl. Permutationen $u'',v''$ von $u,v$.\\
	(Wähle: $\Sigma = \set{\text{Wörter }w}{l(w) \leq k} \sqcup \set{k}{ku_0k^{-1} = v_0 \forall \text{ konjugierte Paare } (u_0,v_0) \text{ und }\max\{l(u_0), l(v_0)\}} \leq k $)
		\item Verwende Dehn-Algorithmus, um zu prüfen, ob ein $w \in \Sigma$ existiert mit $w^{-1}u''w = v''$ in $G$
\end{itemize}
\paragraph{Fazit} Konjugationsproblem lässt sich für hyperbolische Gruppen auf das Wortproblem reduzieren.

\newpage
\section{Enden von Gruppen}
\subsection{Definition}
Sei $\Gamma = (V,E, \delta)$ ein Graph.
\begin{itemize}
	\item Ein \df{(eigentlicher) Strahl} in $\Gamma$ ist eine unendlicher Pfad $\gamma = (x_i)_{i \geq 0}$, s.d. $\forall$ beschränkten Teilmengen $B \leq V$ nur endlich viele $x_i$ in $B$ liegen.\\
	(Graphen mit endlich vielen Ecken besitzen keine eigentlichen Strahlen).
	\item Eine Menge $C$ von Ecken heißt \df{zusammenhängend}, wenn $\forall x,y \in C$ ein Pfad von $x$ nach $y$ existiert, der nur über Ecken in $C$ läuft.
	\item \df{Zusammenhangskomponenten} von $C \subset V$ sind die maximalen zusammenhängenden Teilmengen $D$ von $C$.\\
	\item Wir sagen $A,B \subset V$ werden von $S \subset V$ \df{separiert}, falls $\forall x\in A, y \in B$ gilt:\\
	$x,y$ liegen in unterschiedlichen Zusammenhangkomponenten von $V- S$.

\end{itemize}

\subsection{Definition 8.2}
\begin{itemize}
	\item Ein \df{Unterstrahl} eines Strahles $(x_i)_{i \geq 0}$ ist ein Strahl $(y_j)_{j\geq 0}$ mit $y_i = x_i+k$ für alle $i$, $k\in \N$ fest.
	
	\item Zwei Strahlen $(x_i)_{i\geq 0}, (y_j)_{j\geq 0}$ werden von einer endlichen Menge $S$ \df{separiert}, wenn es zwei disjunkte Unterstrahlen $(x'_i), (y'_j)$ in $V-S$ gibt, deren Eckenmengen von $S$ separiert werden.
	
	%%10.07.15, 24.VL
	
	%%Online Prüfungsanmeldung nicht vergessen!
	
	\item  Zwei Strahlen  heißen \df{äquivalent} (definieren das selbe Ende), wenn sie nicht durch eine endliche Menge von Ecken separiert werden können.
	\item Die Äquivalenzklasse eines Strahles $\gamma$ heißt \df{Ende von $\gamma$} $end(\gamma)$.
	\item $Ends(\Gamma) := \set{end(\gamma)}{\gamma \text{ Strahl in }\Gamma}$
	\item $E(\Gamma) := \# Ends(\Gamma)$
\end{itemize}

\subsection{8.3 Beispiel}
\begin{enumerate}
	\item $\Z$ hat zwei Enden
	\item $\Z^2$ hat nur eine Ecke, da $\Z^2-B$ für $B$ endlich immer genau eine Zusammenhangkomponente hat.
	\item Jeder endliche Graph hat 0 Enden.
	\item Der Cayleygraph einer freien Gruppe hat unendlich viele Enden.
\end{enumerate}

\subsection{Bemerkung 8.4}
Man kann für geodätische metrische Räume Enden wie folgt definieren:
\begin{itemize}
	\item Ein \df{(eigentlicher) Strahl} $\gamma$ in $X$ ist eine stetige Abbildung $\gamma:[0,\infty) \rightarrow X$, s.d. $\forall B \subset X$ beschränkt das Urbild $\gamma^{-1}(B)$ beschränkt ist.
	\item Zwei eigentliche Strahlen $\gamma, \gamma'$ sind \df{äquivalent}, wenn $\forall B \subset X$ beschränkt gilt:
	\[\exists t \in [0,\infty): \gamma([t,\infty)) und \gamma'([t,\infty)) \text{ liegen in derselben Zusammenhangskomponente von } X- B \]
	\item $end, Ends$ und $E$ definiert man analog wie für Graphen.
\end{itemize}

\subsection{8.5 Definition}
Ein Strahl ist \df{geodätisch}, wenn der zugehörige Strahl in der geometrischen Realisierung eine Geodäte ist.

\subsection{Bemerkung}
$B(v_0, R) := \set{v \in V}{d(v_0, v) \leq R}$

\subsection{8.7 Beobachtung}
$\Gamma$ lokal endlicher, zusammenhängender Graph
\begin{enumerate}
	\item $end((x_i)) = end((y_j)) \Longleftrightarrow \forall S \subset V \text{ endl. } \exists N \in \N: (x_i)_{i \geq N},(y_j)_{j \geq N}$ liegen in derselben Zusammenhangkomponente von $V-S$
	\item $end((x_i)) = end((y_j)) \Longleftrightarrow \forall B = B(v_0, R) \exists N \in \N: (x_i)_{i \geq N}, (y_j)_{j \geq N}$ liegen in derselben Zusammenhangkomponente von $V-B$
\end{enumerate}
\paragraph{Beweis}
Jede endliche Menge $S$ ist in einem $B(V,R)$ enthalten. Jeder Ball ist endlich, also selbst endliches $S \subset V$.\qed

\subsection{8.8 Lemma}
$\Gamma$ lokal endlicher, zusammenhängender Graph, $v_0 \in V$ fest.\\
Sei $G_{v_0}(\Gamma) := \set{\text{geod. Strahl } \gamma}{\gamma \df{ startet bei }v_0}$. Dann ist die Abb.
\[G_{v_0}(\Gamma ) \longrightarrow Ends(\Gamma) \]
\[\gamma \longmapsto end(\gamma) \]
surjektiv.

\paragraph{Beweis-Skizze}
$\sigma$ (eigentlicher Strahl) in $\Gamma$, d.h. $end(\sigma) \in Ends(\Gamma)$.\\
Zu Zeigen: $\exists$ geod. Strahl $\gamma$ mit $end(\gamma) = end(\sigma)$.\\
Setze $\gamma_n := $Geodäte von $v_0$ nach $x_n$ wobei $\sigma = (x_i)_{i\geq 0}$ verlängert bis unendlich durch die konstante Folge $(x_n)_i$\\

Man kann zeigen, dass eine Teilfolge von $(\gamma_n)_n$ existiert, die gegen einen geod. Strahl $\gamma$ konvergiert, der nach Konstruktion und 8.7 (b) dasselbe Ende wie $\sigma$ besitzt.

\paragraph{Satz von Arzela-Ascoli}
$X$ kompakter metrischer Raum, $Y$ separabler metrischer Raum, dann hat jede Folge gleichmäßig stetiger Abbildungen $f_n : Y \rightarrow X$ eine Teilfolge, die gegen eine stetige Abbildung $f : Y \rightarrow X$ konvergiert.
\qed

\subsection{8.9 Lemma}
Ein \df{$k$-Pfad} von $x$ nach $y$, $x,y \in V$, ist endliche Folge von Ecken $x = x_1,\ldots, x_n = y$ mit $d(x_i, x_{i+1}) \leq k$.\\
Sei $\Gamma$ lokal endlicher Graph, $v_0$ feste Ecke. Seien $\gamma_1 = (x_i)_{i \geq 0}$ und $\gamma_2 = (y_i)_{i \geq 0}$ Strahlen in $\Gamma$. Dann gilt:
\[end(\gamma_1) = end(\gamma_2) \Longleftrightarrow\]
\[\forall R > 0 \exists T > 0\forall i > T: x_i \in V- B(v_0, R) \text{ kann mit } y_i \text{ durch }k\text{-Pfad verbunden werden} \]

\paragraph{Beweis}
$\Longrightarrow: $ $x_i$ kann mit $y_i$ durch einen 1-Pfad verbunden werden für $i$ groß (wg. 8.7 (b)).\\
$\Longleftarrow: $ für einen $k$-Pfad $x = x_1, \ldots, x_n = y$ wähle Geodäten der Länge $d(x_i,x_{i+1})$ in $V- B(v_0,R)$ um $x_i$ mit $y_i$ zu verbinden. Verkettung dieser Geodäten ist Pfad von $x_i$ nach $y_i$\\
$\Longrightarrow$ $x_i, y_i$ liegen in derselben Zusammenhangkomponente $\forall i > T$.\qed

%%27.VL, 14.07.15

\subsection{8.10 Satz}
Seien $\Gamma_1 = (V_1, E_1, \delta_1)$, $\Gamma_2 = (V_2, E_2, \delta_2)$ lokal endliche Graphen.\\
Jede Quasi-Isometrie $f:\Gamma_1 \rightarrow \Gamma_2$ induziert eine bijektive Abbildung $f_E : Ends(\Gamma_1) \rightarrow Ends(\Gamma_2)$, d.h. $E(\Gamma)$ ist eine Quasi-Isometrie-Invariante.

\paragraph{Beweis}
Sei $\gamma = (x_i)_{i\geq 0}$ eigentlicher Strahl in $\Gamma_1$, d.h. $end(\gamma) \in Ends(\Gamma_1)$, sei $f$ eine $(c,d)$ quasi-Isometrie.\\
Definiere Folge $(y_j)$ in $\Gamma_2$ wie folgt:\\
Verbinde Bilder $f(x_i), f(x_{i+1})$ durch einen stückweise geodätischen Weg in $\Gamma_2$ und erhalte so Eckenfolge $(y_i)$ mit $d(y_i, y_{i+1}) = 1$.\\
Weil $f$ Quasi-Isometrie, ist $(y_i)$ ein Strahl, weil beschränkte Mengen von $(y_i)$ nach endl. Zeit verlassen werden.\\
Definiere $f_*((x_i)_{i \geq 0}) = (y_j)_{j\geq 0}$. Offensichtlich ist $end(f_*((x_i)))$ unabhängig von der Wahl der geodätischen Segmente von $f(x_i)$ nach $f(x_{i+1})$.\\
Definiere $f_E : Ends(\Gamma_1) \rightarrow Ends(\Gamma_2), end(\gamma) \mapsto end(f_*(\gamma))$ für \textbf{geodätische} Strahlen $\gamma$.\\
Wegen Lemma 8.8 ist $f_*$ auf ganz $Ends(\Gamma_1)$ definiert.\\
Seien $\gamma = (x_i), \sigma = (y_j)$ zwei Strahlen in $\Gamma_1$ mit $end(\gamma) = end(\sigma)$. Dann gilt mit Lemma 8.9, dass $\forall R > 0$ ein $T> 0$ existiert, s.d. $\forall i > T$ $x_i$ durch einen $k$-Pfad mit $y_i$ verbunden werden kann außerhalb von $B(1,R)$.\\\\

Das Bild eines $k$-Pfades unter einer $(c,d)$-Quasi-Isometrie ist ein $(cd+k)$-Pfad.\\
$\Longrightarrow end(f_*(\gamma)) = end(f_*(\sigma))$, d.h. $f_E$ wohldefiniert und insbes. injektiv.\\
Für Quasi-Isometrien $f: \Gamma_1 \rightarrow \Gamma_2 \overset{f'}{\rightarrow} \Gamma_1$ gilt $(f'\circ f)_E = f'_e \circ f_E$.\\
Wenn $f'$ Quasi-Inverse von $f$ ist, dann ist $(f'\circ f)_E = id_{Ends(\Gamma_1)}$.\\
$\Longrightarrow f_E$ surjektiv.\qed


\subsection{Definition 8.11}
Die (Anzahl der) Enden einer endlich erzeugten Gruppe ist die (Anzahl der) Enden eines ihrer Cayley-Graphen.
\paragraph{Beispiele}
\begin{itemize}
	\item Jede endliche Gruppe hat 0 Enden.
	\item $\Z^2$ hat ein Ende.
	\item $\Z$ hat zwei Enden.
	\item $F_2$ hat unendlich viele Enden.
\end{itemize}

\subsection{8.12 Satz von Freudenthal-Hopf (1930)}
$G$ endlich erzeugte Gruppe, dann hat $G$ entweder $0,1,2$ oder überabzählbar viele Enden.\\
$G$ hat 0 Enden $\Longrightarrow$ $G$ endlich\\
$G$ hat 2 Enden $\Longrightarrow$ $\Z$ eine Untergruppe von endlichem Index in $G$ ist.

\paragraph{Beweis}
$G$ endlich erzeugt, $\Gamma = Cay(G,S)$, $S$ endliches EZS von $G$.

\paragraph{Annahme:} $k := E(\Gamma) \geq 3$.
\paragraph{Annahme:} $\norm{G} = \infty$, sonst Widerspruch zu $E(G) = 0$ (8.7 (b)).\\
Wähle $g\in G$ mit $d_S(1,g) > 2n$ und $g$ enthalten in einer unbeschränkten Zusammenhangkomponente von $V- B(1,n)$, $n$ groß genug, dass $V-B(1,n)$ $k$ unbeschränkte Zusammenhangkomponente hat (geht wegen 8.7 (b)).\\
Es gilt:
\[g.B(1,n) \cap B(1,n) = \emptyset \]
Weiter ist $g.B(1,n)$ in einer unbeschränkten Zusammenhangkomponente von $V-B(1,n)$ weil $B(1,n)$ und somit auch $g.B(1,n)$ zusammenhängend ist.\\
$\Gamma$ homogen (sieht überall gleich aus) also teilt $g.B(1,n)$ die unbeschränkte Zusammenhangkomponente von $g$ in mind. $k$ zusammenhängende Teile, von denen mind. $k-1$ unbeschränkt sind.\\
Die Anzahl der unbeschränkten Zusammenhangkomponenten von $V- (B(1,n) \cup g.B(1,n))$ ist $\geq (k-1) + (k-1) = 2k - 2 > k$ weil $k \geq 3$.\\
Widerspruch zur Annahme, dass $k$ endlich wäre.\\

\paragraph{Beweisskizze} für 2 Enden:\\
$\Longleftarrow$: $\Z$ UG von endlichem Index in $G$:
\[Cay(\Z, T)  \qi Cay(G,S) \]
$\Longrightarrow E(G) = E(\Z)$\\

$\Longrightarrow$: 
Sei jetzt $E(G) = 2$. $\Gamma$ Cayleygraph von $G$, dann gilt:
\[E(\Gamma) = 2\]
und mit 8.10 induziert jede Quasi-Isom. auf $\Gamma$ eine Bijektion der beiden Enden.\\
$G \curvearrowright \Gamma$ durch Isometrien $\rightarrow$ Jedes $g \in G$ induziert Bijektion $f_g : Ends(\Gamma) \rightarrow Ends(\Gamma)$.\\
Wähle $g \in G$ so, dass $f_g = id$.\\
Sei $B:=B(1,k)$ so, dass $V-B$ zwei Zusammenhangkomponenten hat.\\
OE sei $g$ so gewählt, dass $g\in V-B$ und $g$ liegt in einer unbeschränkten Zusammenhangkomponente von $V-B$. \\
Wähle einen Strahl $\gamma$ geodätisch und leerene Schnitt mit $B$, $\gamma$ starte in $g$.\\
$g^n.\gamma$ ist $\forall n$ wieder Strahl. Insbesondere ist (nach Wahl von $g$) $end(\gamma) = end(g.\gamma)$.\\
Nachrechnen: $g^n.\gamma$ und $g^{-n}.\gamma$ konvergieren gegen 2 verschiedene Enden. $\rightarrow \grp{g}{} = \cong \Z$\\
Noch zu zeigen: alle Elemente aus $V$ haben beschränkten Abstand zu $\grp{g}{}$.\qed

\subsection{Satz von Stallings (1968)}
$G$, endlich erzeugt, hat $\infty$-viele Enden $\Longleftrightarrow$\\
$G$ zerfällt als HNN-Erweiterung oder freies amalgiertes Produkt, d.h. $G \cong A *_C B$ oder $\cong A_{*B}$ mit $C$ endlich und $\norm{A/C} \geq 3, \norm{B/C} \geq 2$.
\begin{itemize}
	\item Insbesondere wirkt $G$ dann auf einem Baum.
	\item Ist $G$ torsionsfrei (d.h. $\not \exists$ $g\in G$ endlicher Ordnung), dann hat $G$ $\infty$-viele Enden $\Longleftrightarrow$ $G$ freies Produkt, $G \cong H * K$ und $H,K$ nicht-trivial.
\end{itemize}

%%%% 28.VL, 17.07.15


\end{document}