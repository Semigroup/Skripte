\section{De Rham-Kohomologie von Glatten Mannigfaltigkeiten}
\marginpar{Vorlesung vom 8.1.18}
\Def{}
Sei $M$ eine (kompakte) glatte Mannigfaltigkeit der Dimension $n$. Durch Whitneys Einbettungssatz erhalten wir eine glatte Einbettung $M \subset \R^{2n+1}$. Durch den Satz über Tubenumgebungen wissen wir um die Existenz einer offenen Umgebung $U$ von $M$ in $\R^{2n+1}$, sodass
\[ U \isom{} E(\nu) \]
wobei $E \surj{p} M $ der Totalraum des Normalenbündels ist. Dieser induziert einen Deformationsretrakt $r : U \pfeil{} M$.\\
$H^*$ soll eine Homotopieinvariante sein. Dies würde einen Isomorphismus
\[ r^* : H^*(M) \Pfeil{\sim} H^*(U) \]
implizieren. In diesem Sinne definieren wir die \df{Kohomologiegruppen} von $M$ durch
\[ H^k(M) := H^k(U) \]
Diese Definition hängt von der Einbettung von $M$ ab. Insofern wäre es wünschenswert eine intrinsische Definition von $H^*(M)$ zu finden.

\Def{}
Sei $M$ eine glatte Mannigfaltigkeit der Dimension $n$.\\
Wir betrachten Familien $\omega = \{ \omega_p\}_{p\in M}$ mit $\omega_p \in Alt^k(T_pM)$. Sei eine glatte Karte 
\[\phi : U' \off M \pfeil{\sim} U \off \R^n \]
gegeben. Betrachte für jedes $p \in U'$ die Abbildung
\[ Alt^k(\phi\i_{*,\phi(p)}) : Alt^k(T_pM) \Pfeil{\sim} Alt^k(T_{\phi(p)}U) \isom{} Alt^k(\R^n) \]
Wir definieren den \df{Pullback} von $\omega$ durch
\begin{align*}
(\phi\i)^*\omega : U & \Pfeil{} Alt^k(\R^n)\\
x & \longmapsto Alt^k(\phi\i_{x,*})(\omega_{\phi\i(x)})
\end{align*}
$\omega$ heißt eine \df{glatte Differential-$k$-Form} auf $M$, wenn $(\phi\i)^*\omega$ für jede Karte $\phi$ glatt ist.\\
Es bezeichne $\Omega^k(M)$ den reellen Vektorraum aller glatter Differential-$k$-Formen auf $M$.

\Def{}
Die Karte $\phi$ von $M$ induziert einen Isomorphismus
\[ \phi_{*,p} : T_pM \Pfeil{\sim} T_{x}U \]
für $x = \phi(p)$. Dadurch erhalten wir einen Isomorphismus
\[ Alt^{k+1} ( \phi_{*,p}) : Alt^{k+1}(T_xU) \Pfeil{\sim} Alt^{k+1}(T_pM) \]
Wir können so folgende $k+1$-Form definieren
\[ \d \omega_p:= Alt^{k+1}(\phi_{*,p}) \klam{ \d ((\phi\i)^*\omega) (\phi(p)) } \]
Dadurch erhalten wir eine glatte Differentialform $\d \omega$ auf $M$. Diese Definition ist unabhängig von der Wahl der Karte $\phi$. Es gilt $\d^2 = 0$ auf $\Omega^*(M)$. Wir erhalten folglich einen Kokettenkomplex $(\Omega^*(M), \d)$.

\Def{}
Die \df{de Rham-Kohomologie} von $M$ ist definiert als die Kohomologie des Kokettenkomplexes $(\Omega^*(M), \d)$, d.\,h.
\[ H^k(M) := H^k(\Omega^*(M), \d) = \frac{\Ker (\d : \Omega^k(M) \pfeil{}\Omega^{k+1}(M))}{\Img \d : (\Omega^{k-1}(M) \pfeil{} \Omega^k(M))} \]

\Def{}
Zu glatten Differentialformen $\omega = \{ \omega_p\}, \eta = \{\eta_p\}$ auf $M$ definieren wir das \df{äußere Produkt} punktweise durch
\[ (\omega\wedge \eta)_p := \omega_p \wedge \eta_p \]
$\omega \wedge \eta$ ist wieder eine glatte Differentialform auf $M$. Ferner gilt hierfür offensichtlich wieder die Produktregel, d.\,h.
\[ \d(\omega \wedge \eta) = (\d \omega) \wedge \eta + (-1)^k\omega \wedge \d(\eta) \]
für $\omega \in \Omega^k(M)$. Wir erhalten dadurch wieder eine graduiert kommutative Algebra $\Omega^*(M)$.\\
Das äußere Produkt auf $\Omega^*(M)$ steigt wie im affinen Fall wohldefiniert auf $H^*(M)$ ab. Dadurch wird auch $H^*(M)$ zu einer graduiert kommutativen Algebra.

\Def{}
Sei $\phi : M \pfeil{} N$ eine glatte Abbildung und $\omega = \{ \omega_q\}_{q\in N} \in \Omega^k(N)$ eine glatte Differentialform. Für einen Punkt $p \in M$ erhalten wir eine lineare Abbildung
\begin{align*}
Alt^k(\phi_{*,p}) : Alt^k(T_{\phi(p)}M) & \Pfeil{} Alt^k(T_pM)
\end{align*}
In diesem Sinn setzen wir
\[ (\phi^*\omega)_p := Alt^k(\phi_{*,p})(\omega_{\phi(p)}) \]
und erhalten eine glatte Differentialform
\[ \phi^*\omega := \{ (\phi^*\omega)_p\}_{p\in M} \]
auf $M$. Dadurch erhalten wir eine lineare Abbildung
\[ \phi^* : \Omega^k(N) \pfeil{} \Omega^k(M) \]
Wie im affinen Fall ist die Zuweisung
\begin{align*}
M & \longmapsto \Omega^k(M)\\
\phi & \longmapsto \phi^*
\end{align*}
ein kontravarianter Funktor.\\
Es gilt wieder
\begin{itemize}
	\item $\phi^*(\omega \wedge \eta) = (\phi^*\omega) \wedge (\phi^*\eta)$
	\item $\d_M \circ \phi^* = \phi^* \circ \d_N$
\end{itemize}
$\phi^*$ steigt wohldefiniert auf die Kohomologie ab und liefert Abbildungen
\[ \phi^* : H^k(N) \Pfeil{} H^k(M) \]
Dadurch wird die de Rham-Kohomologie zu einem kontravarianten Funktor von der Kategorie der glatten Mannigfaltigkeiten in die Kategorie der reellen Vektorräume.

\section{Integration auf Glatten Mannigfaltigkeiten}

\Def{}
Sei $U \subset \R^n$ offen, $\omega \in \Omega^n(U)$ sei eine $n$-Form.\\
Wir definieren den \df{Träger} von $\omega$ durch
\[ \supp \omega := Cl_U{\set{x \in U}{\omega_x \neq 0}} \]
wobei $Cl_U$ den Abschluss einer Menge in $U$ bezeichnet.\\
Hat $\omega$ einen kompakten Träger, so hat $\omega$ eine glatte Fortsetzung durch Null auf $\R^n$.\\
Da $\omega$ eine $n$-Form ist, hat es die Gestalt
\[ \omega = f(x_1, \ldots, x_n) \d x_1 \wedge \ldots \wedge \d x_n \]
Wir definieren folgendes Integral zu $\omega$
\[ \int_U \omega := \int_{\R^n}\omega := \int_{\R^n} f \d x_1 \ldots \d x_n \]

\Bem{}
Seien $V,U \subset \R^n$ offen, $\theta : V \pfeil{} U$ ein Diffeomorphismus.\\
Sei ferner $\omega = f \d x_1 \wedge \ldots \d x_n\in \Omega^n(U)$ mit kompakten Träger. Dies induziert uns eine Differentialform $\theta^*\omega \in \Omega^n(V)$, welche ebenfalls kompakten Träger hat. Es gilt
\[ \theta^*\omega= \theta^*(f\d x_1 \wedge \ldots \wedge \d x_n) = f\circ \theta \cdot \theta^*(\d x_1 \wedge \ldots \wedge \d x_n) = f\circ \theta \cdot \det(J_\theta) \cdot \d x_1 \wedge \ldots \wedge \d x_n \]
Es ergibt sich nun
\begin{align*}
\int_V \theta^*\omega &= \int_{\R^n} f\circ \theta \cdot \det(J_\theta)~ \d x_1 \ldots \d x_n\\
&\gl{\text{Traforegel in }\R^n} \pm \int_{\R^n} f~ \d x_1 \ldots \d x_n = \pm \int_U \omega
\end{align*}
Ist $U$ zusammenhängend, so ist das Vorzeichen hier gerade das Vorzeichen der Jacobi-Determinante.\\
Ist $\theta$ orientierungserhaltend und $U$ zusammenhängend, so gilt also
\[ \int_V \theta^*\omega = \int_U \omega \]

\Def{}
Sei nun $M$ eine glatte, orientierte Mannigfaltigkeit der Dimension $n$ und $\omega \in \Omega^n(M)$.\\
Sei $\phi : U' \pfeil{} U$ eine orientierte Karte von $M$. Es gelte ferner
\[ \supp(\omega) \subset U' \text{ ist kompakt} \]
Wir setzen dann
\[ \int_M  \omega := \int_U (\phi\i)^*\omega \]
Dies ist wohldefiniert. Ist nämlich $\psi : U' \pfeil{} U$ eine weitere orientierte Karte, so gilt
\[ (\phi\i)^* = (\phi\i)^*\psi^* (\psi\i)^* \omega \]
Setzt man $\psi \circ \phi\i =: \theta$, so folgt mit obiger Bemerkung
\[ \int_U (\phi\i)^*\omega = \int_U (\psi\i)^*\omega \]

\Def{}
Sei nun $\omega \in \Omega^n(M)$ mit kompakten Träger. Im Allgemeinem liegt $\supp f$ nicht in einer einzelnen Karte von $M$.\\
Deswegen sei $(f_i)_i$ eine glatte Partition der Eins auf $M$, sodass $\supp f_i \subset U_i$, wobei die Paare $(U_i, \phi_i)_i$ orientierte Karten seien, die $\supp f$ überdecken, und, bei denen $\overline{U_i}$ kompakt ist.\\
Wir setzen
\[ \int_M \omega := \sum_i\int_M f_i \omega \]
Dies ist ein wohldefiniertes Integral. Es ist unabhängig von der Wahl der Partition, denn sei $(g_j)_j$ eine weitere Partition der Eins, dann gilt ja
\[ f_i = \sum_j f_i g_j \text{ und } g_j = \sum_i f_i g_j \]
Ergo folgt
\[ \sum_i \int f_i \omega = \sum_{i,j} \int f_i g_j \omega = \sum_j \int g_j \omega \]