%
% -------------------------------------------------------------------------------------------
% "THE BEER-WARE LICENSE"
% Jo, Ihr könnt mit diesem Code machen, was Ihr wollt, solange keinerlei Verantwortlichkeiten auf mich zurückfallen...
% Ansonsten kann ich Euch nur anregen, Wissen frei zu gestalten und allen Menschen zugänglich zu machen.
% Ja, und trinkt ein Bier oder eine Chai Latte oder einen Grünen Tee auf das Skript hier (oder ladet mich ein, wenn Ihr wollt).
% LG Akin
% -------------------------------------------------------------------------------------------
%

\documentclass[12pt]{article}

\usepackage[T1]{fontenc}
\usepackage[utf8]{inputenc}
\usepackage[ngerman]{babel}

\usepackage{tikz-cd}
\usetikzlibrary{babel}

\usepackage{amsfonts}
\usepackage{amssymb}
\usepackage{amsmath}
\usepackage{mathtools}
\usepackage{wasysym}
\usepackage{dsfont}
\usepackage{geometry}
\usepackage{makeidx}
\usepackage{booktabs}
\usepackage{hyperref}

\usepackage{enumerate}
\usepackage{adjustbox}

\newcommand{\ifLeer}[3]{\ifx&#1&\relax#2\relax\else\relax#3\relax\fi\relax}

\newcommand{\Def}[1]{\subsection{Definition\ifLeer{#1}{}{: #1}}}
\newcommand{\Bsp}[1]{\subsection{Beispiel\ifLeer{#1}{}{: #1}}}
\newcommand{\Lem}[1]{\subsection{Lemma\ifLeer{#1}{}{: #1}}}
\newcommand{\Bem}[1]{\subsection{Bemerkung\ifLeer{#1}{}{: #1}}}
\newcommand{\Kor}[1]{\subsection{Korollar\ifLeer{#1}{}{: #1}}}
\newcommand{\Satz}[1]{\subsection{Satz\ifLeer{#1}{}{: #1}}}
\newcommand{\Prop}[1]{\subsection{Proposition\ifLeer{#1}{}{: #1}}}

\newcommand{\QED}{\hfill $\square$}
\newcommand{\qed}{\hfill $\blacksquare$}

\newenvironment{Beweis}[1]{\paragraph{Beweis\ifLeer{#1}{}{: #1}\\}}{\QED}
\newenvironment{Beweisskizze}[1]{\paragraph{Beweisskizze\ifLeer{#1}{}{: #1}\\}}{\qed}

\newcommand{\df}[1]{\index{#1}\textbf{#1}}

\newcommand{\klam}[1]{\left(#1\right)}
\newcommand{\bet}[1]{\left|#1\right|}
\newcommand{\norm}[1]{\bet{\bet{#1}}}
\newcommand{\brak}[1]{\left[#1\right]}
\newcommand{\curv}[1]{\left\lbrace#1\right\rbrace}
\newcommand{\shrp}[1]{\left<#1\right>}
\newcommand{\quot}[1]{\glqq #1 \grqq\relax}
\newcommand{\set}[2]{\curv{\ifLeer{#2}{#1}{#1 ~ | ~ #2}}}
\newcommand{\grp}[2]{\shrp{\ifLeer{#2}{#1}{#1 ~ | ~ #2}}}

\newcommand{\A}{\mathcal{A}}
\newcommand{\B}{\mathcal{B}}
\newcommand{\C}{\mathbb{C}}
\newcommand{\D}{\mathcal{D}}
\newcommand{\E}{\mathcal{E}}
\newcommand{\F}{\mathcal{F}}
\newcommand{\G}{\mathcal{G}}
\renewcommand{\H}{\mathbb{H}}
\newcommand{\I}{\mathcal{I}}
\newcommand{\J}{\mathcal{J}}
\newcommand{\K}{\mathbb{K}}
\renewcommand{\L}{\mathcal{L}}
\newcommand{\M}{\mathcal{M}}
\newcommand{\N}{\mathbb{N}}
\renewcommand{\O}{\mathcal{O}}
\renewcommand{\P}{\mathcal{P}}
\newcommand{\Q}{\mathbb{Q}}
\newcommand{\R}{\mathbb{R}}
\renewcommand{\S}{\mathcal{S}}
\newcommand{\T}{\mathcal{T}}
\newcommand{\U}{\mathcal{U}}
\newcommand{\V}{\mathcal{V}}
\newcommand{\W}{\mathcal{W}}
\newcommand{\X}{\mathcal{X}}
\newcommand{\Y}{\mathcal{Y}}
\newcommand{\Z}{\mathbb{Z}}

\newcommand{\id}[1]{\text{Id}_{#1}}
\newcommand{\Ker}{\textsf{Kern}}
\newcommand{\Coker}{\textsf{Kokern}}
\newcommand{\Img}{\textsf{Bild}}
\newcommand{\Coimg}{\textsf{Kobild}}
\newcommand{\Hom}[3]{\textsf{Hom}_{#1}\left(#2, #3\right)}
\newcommand{\Aut}[2]{\textsf{Aut}_{#1}\left(#2\right)}
\newcommand{\Sym}[1]{\textsf{Symm}_{#1}}

\newcommand{\e}{\varepsilon}

\newcommand{\Pfeil}[1]{\overset{#1}{\longrightarrow}}
\newcommand{\pfeil}[1]{\overset{#1}{\rightarrow}}
\newcommand{\inj}[1]{\overset{#1}{\hookrightarrow}}
\newcommand{\Inj}[1]{\overset{#1}{\lhook\joinrel\longrightarrow}}
\newcommand{\surj}[1]{\overset{#1}{\twoheadrightarrow}}

\newcommand{\impl}[1]{\overset{#1}{\Rightarrow}}
\newcommand{\Impl}[1]{\overset{#1}{\Longrightarrow}}
\newcommand{\gdw}[1]{\overset{#1}{\Leftrightarrow}}
\newcommand{\Gdw}[1]{\overset{#1}{\Longleftrightarrow}}

\newcommand{\off}{\overset{o}{\subset}}
\newcommand{\abg}{\overset{c}{\subset}}

\newcommand{\gl}[1]{\overset{#1}{=}}
\newcommand{\grgl}[1]{\overset{#1}{\geq}}
\newcommand{\klgl}[1]{\overset{#1}{\leq}}
\newcommand{\gr}[1]{\overset{#1}{>}}
\newcommand{\kl}[1]{\overset{#1}{<}}
\newcommand{\isom}[1]{\overset{#1}{\cong}}

\newcommand{\supp}{\text{supp}}

\renewcommand{\i}{^{-1}}
\renewcommand{\phi}{\varphi}
\renewcommand{\d}{\text{d}}

\newcommand{\rot}{\text{rot}}

\renewcommand{\epsilon}{\varepsilon}
\newcommand{\sgn}{\text{sign}}

\setlength{\marginparwidth}{20mm}

\begin{document}

\section{Hauptklausur}
\subsection{Aufgabe (4 Punkte)}
Sei $f :M \pfeil{} N$ eine glatte Abbildung glatter geschlossener Mannigfaltigkeiten der Dimension $n$. $W$ sei eine glatte kompakte $(n+1)$-dimensionale Mannigfaltigkeit mit Rand $\partial W = M$. $F$ sei eine glatte Fortsetzung von $f$ auf $W$. $p \in N$ sei regulär für $F$ und $f$.
\paragraph{Zeigen Sie:}
\[ \deg_p f \equiv 0 \mod 2 \]

\subsection{Aufgabe (4 Punkte)}
Sei $f :M \pfeil{} \R^n$ eine glatte Abbildung. $N\subset \R^n$ sei eine glatte Untermannigfaltigkeit. $\epsilon > 0$ sei beliebig.
\paragraph{Zeigen Sie:} Es gibt ein $v \in \R^n$ mit $\norm{v} < \epsilon$, sodass die Abbildung
\[ g(x) := f(x) + v \]
transversal zu $N$ steht.
\paragraph{Hinweis:} Benutzen Sie die Abbildung
\begin{align*}
F: M\times N &\Pfeil{} \R^n\\
(x,y) & \longmapsto y - f(x)
\end{align*}

\subsection{Aufgabe (4 Punkte)}
Sei $f : S^3 \pfeil{} S^2$ glatt. $\omega \in \Omega^2(S^2)$ sei eine glatte 2-Form.
\subsubsection{Teilaufgabe (1 Punkt)}
\paragraph{Zeigen Sie:} Es gibt ein $\beta \in \Omega^1(S^3)$ mit
\[ \d \beta = f^*(\omega) \]

\subsubsection{Teilaufgabe (3 Punkte)}
\paragraph{Zeigen Sie:} Das Integral $\int_{S^3} f^*(\omega) \wedge \beta$ ist unabhängig von der Wahl von $\beta \in \Omega^1(S^3)$ mit $\d \beta = f^*(\omega)$.

\subsection{Aufgabe (4 Punkte)}
Es bezeichne $T^2$ den zweidimensionalen Torus. $p \in T^2$ sei ein beliebiger Punkt.
\paragraph{Berechnen Sie:} Die Kohomologiegruppen $H^k(T^2 - \{p\})$ der zweidimensionalen Mannigfaltigkeit $T^2 - \{p\}$ für $k = 0,1,2$.

\subsection{Aufgabe (4 Punkte)}
\subsubsection{Teilaufgabe (1 Punkt)}
Geben sie Definition von \textit{Immersion} und \textit{Submersion} wieder.
\subsubsection{Teilaufgabe (1 Punkt)}
Geben sie Definition von \textit{Parallelisierbar} wieder.
\subsubsection{Teilaufgabe (1 Punkt)}
Ist $S^2$ parallelisierbar? Warum / Warum nicht?
\subsubsection{Teilaufgabe (1 Punkt)}
Ist $S^2$ parallelisierbar? Warum / Warum nicht?

\end{document}