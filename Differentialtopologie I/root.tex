%
% -------------------------------------------------------------------------------------------
% "THE BEER-WARE LICENSE"
% Jo, Ihr könnt mit diesem Code machen, was Ihr wollt, solange keinerlei Verantwortlichkeiten auf mich zurückfallen...
% Ansonsten kann ich Euch nur anregen, Wissen frei zu gestalten und allen Menschen zugänglich zu machen.
% Ja, und trinkt ein Bier oder eine Chai Latte oder einen Grünen Tee auf das Skript hier (oder ladet mich ein, wenn Ihr wollt).
% LG Akin
% -------------------------------------------------------------------------------------------
%

\documentclass[12pt]{book}

\usepackage[T1]{fontenc}
\usepackage[utf8]{inputenc}
\usepackage[ngerman]{babel}

\usepackage{tikz-cd}
\usetikzlibrary{babel}

\usepackage{amsfonts}
\usepackage{amssymb}
\usepackage{amsmath}
\usepackage{mathtools}
\usepackage{wasysym}
\usepackage{dsfont}
\usepackage{geometry}
\usepackage{makeidx}
\usepackage{booktabs}
\usepackage{hyperref}

\usepackage{enumerate}
\usepackage{adjustbox}

\newcommand{\ifLeer}[3]{\ifx&#1&\relax#2\relax\else\relax#3\relax\fi\relax}

\newcommand{\Def}[1]{\subsection{Definition\ifLeer{#1}{}{: #1}}}
\newcommand{\Bsp}[1]{\subsection{Beispiel\ifLeer{#1}{}{: #1}}}
\newcommand{\Lem}[1]{\subsection{Lemma\ifLeer{#1}{}{: #1}}}
\newcommand{\Bem}[1]{\subsection{Bemerkung\ifLeer{#1}{}{: #1}}}
\newcommand{\Kor}[1]{\subsection{Korollar\ifLeer{#1}{}{: #1}}}
\newcommand{\Satz}[1]{\subsection{Satz\ifLeer{#1}{}{: #1}}}
\newcommand{\Prop}[1]{\subsection{Proposition\ifLeer{#1}{}{: #1}}}

\newcommand{\QED}{\hfill $\square$}
\newcommand{\qed}{\hfill $\blacksquare$}

\newenvironment{Beweis}[1]{\paragraph{Beweis\ifLeer{#1}{}{: #1}\\}}{\QED}
\newenvironment{Beweisskizze}[1]{\paragraph{Beweisskizze\ifLeer{#1}{}{: #1}\\}}{\qed}

\newcommand{\df}[1]{\index{#1}\textbf{#1}}

\newcommand{\klam}[1]{\left(#1\right)}
\newcommand{\bet}[1]{\left|#1\right|}
\newcommand{\norm}[1]{\bet{\bet{#1}}}
\newcommand{\brak}[1]{\left[#1\right]}
\newcommand{\curv}[1]{\left\lbrace#1\right\rbrace}
\newcommand{\shrp}[1]{\left<#1\right>}
\newcommand{\quot}[1]{\glqq #1 \grqq\relax}
\newcommand{\set}[2]{\curv{\ifLeer{#2}{#1}{#1 ~ | ~ #2}}}
\newcommand{\grp}[2]{\shrp{\ifLeer{#2}{#1}{#1 ~ | ~ #2}}}

\newcommand{\A}{\mathcal{A}}
\newcommand{\B}{\mathcal{B}}
\newcommand{\C}{\mathbb{C}}
\newcommand{\D}{\mathcal{D}}
\newcommand{\E}{\mathcal{E}}
\newcommand{\F}{\mathcal{F}}
\newcommand{\G}{\mathcal{G}}
\renewcommand{\H}{\mathbb{H}}
\newcommand{\I}{\mathcal{I}}
\newcommand{\J}{\mathcal{J}}
\newcommand{\K}{\mathbb{K}}
\renewcommand{\L}{\mathcal{L}}
\newcommand{\M}{\mathcal{M}}
\newcommand{\N}{\mathbb{N}}
\renewcommand{\O}{\mathcal{O}}
\renewcommand{\P}{\mathcal{P}}
\newcommand{\Q}{\mathbb{Q}}
\newcommand{\R}{\mathbb{R}}
\renewcommand{\S}{\mathcal{S}}
\newcommand{\T}{\mathcal{T}}
\newcommand{\U}{\mathcal{U}}
\newcommand{\V}{\mathcal{V}}
\newcommand{\W}{\mathcal{W}}
\newcommand{\X}{\mathcal{X}}
\newcommand{\Y}{\mathcal{Y}}
\newcommand{\Z}{\mathbb{Z}}

\newcommand{\id}[1]{\text{Id}_{#1}}
\newcommand{\Ker}{\textsf{Kern}}
\newcommand{\Coker}{\textsf{Kokern}}
\newcommand{\Img}{\textsf{Bild}}
\newcommand{\Coimg}{\textsf{Kobild}}
\newcommand{\Hom}[3]{\textsf{Hom}_{#1}\left(#2, #3\right)}
\newcommand{\Aut}[2]{\textsf{Aut}_{#1}\left(#2\right)}
\newcommand{\Sym}[1]{\textsf{Symm}_{#1}}

\newcommand{\e}{\varepsilon}

\newcommand{\Pfeil}[1]{\overset{#1}{\longrightarrow}}
\newcommand{\pfeil}[1]{\overset{#1}{\rightarrow}}
\newcommand{\inj}[1]{\overset{#1}{\hookrightarrow}}
\newcommand{\Inj}[1]{\overset{#1}{\lhook\joinrel\longrightarrow}}
\newcommand{\surj}[1]{\overset{#1}{\twoheadrightarrow}}

\newcommand{\impl}[1]{\overset{#1}{\Rightarrow}}
\newcommand{\Impl}[1]{\overset{#1}{\Longrightarrow}}
\newcommand{\gdw}[1]{\overset{#1}{\Leftrightarrow}}
\newcommand{\Gdw}[1]{\overset{#1}{\Longleftrightarrow}}

\newcommand{\off}{\overset{o}{\subset}}
\newcommand{\abg}{\overset{c}{\subset}}

\newcommand{\gl}[1]{\overset{#1}{=}}
\newcommand{\grgl}[1]{\overset{#1}{\geq}}
\newcommand{\klgl}[1]{\overset{#1}{\leq}}
\newcommand{\gr}[1]{\overset{#1}{>}}
\newcommand{\kl}[1]{\overset{#1}{<}}
\newcommand{\isom}[1]{\overset{#1}{\cong}}

\newcommand{\supp}{\text{supp}}

\renewcommand{\i}{^{-1}}
\renewcommand{\phi}{\varphi}
\renewcommand{\d}{\text{d}}

\newcommand{\rot}{\text{rot}}

\renewcommand{\epsilon}{\varepsilon}
\newcommand{\sgn}{\text{sign}}

\setlength{\marginparwidth}{20mm}

\makeindex
\date{\today}
\author{\href{mailto:tensor.produkt@gmx.de}{tensor.produkt@gmx.de}}

\makeindex

\begin{document}
\title{Mitschrieb: Differentialtopologie I\\
WS 17 / 18}
\maketitle
\section*{Vorwort}
Dies ist ein Mitschrieb der Vorlesungen vom 1.12.17 bis zum 22.01.18 des Kurses \textsc{Differentialtopologie I} an der Universität Heidelberg.\\
Dieses Dokument wurde \glqq{live}\grqq\ in der Vorlesung getext. Sämtliche Verantwortung für Fehler übernimmt alleine der Autor dieses Dokumentes.\\
Auf Fehler kann gerne hingewiesen werden bei folgende E-Mail-Adresse
\begin{center}
	\href{mailto:tensor.produkt@gmx.de}{tensor.produkt@gmx.de}
\end{center}
Ferner kann bei dieser E-Mail-Adresse auch der Tex-Code für dieses Dokument erfragt werden.

\setcounter{tocdepth}{1}
\tableofcontents

\chapter{\textsc{Abriss}}
\marginpar{Abriss vorhergehender Vorlesungen}
\section{Topologie}
\Lem{Lebesgue}
Sei $X$ ein kompakter Raum mit Metrik $d$. $(U_i)_{i\in I}$ sei eine Überdeckung von $X$ durch offene Mengen. Dann gibt es eine Konstante $\delta > 0$, die sogenannte \df{Lebesgue-Konstante}, sodass für jede Teilmenge $A \subset X$ gilt
\[ \text{diam}(A) = \sup \{d(a,b)~|~a,b \in A\} < \delta \Impl{} \exists i \in I: A \subset U_i \]
\begin{Beweis}{}
Für jedes $x \in X$ wählen wir ein $\epsilon(x) > 0$ und ein $i(x) \in I$ mit
\[ B_{2\epsilon(x)}(x) \subset U_{i(x)} \]
Die Menge $\{ B_{\epsilon(x)}(x) \}_{x\in X}$ ist eine Überdeckung von $X$ durch offene Mengen und eine Verfeinerung von $\{U_i\}_{i\in I}$. Da $X$ kompakt ist, erhalten wir eine endliche Teilüberdeckung $\{ B_{\epsilon(x_i)}(x_i) \}_{i=1}^n$ von $X$. Setze
\[ \delta := \min\{ \epsilon(x_1), \ldots, \epsilon(x_n) \} \]
Sei nun $A \subset X$ mit
\[ \text{diam}(A) \subset \delta \]
Dann gibt es ein $x \in \{x_1,\ldots, x_n\}$ mit $a_0 \in B_{\epsilon(x)}(x)\cap A \neq \emptyset$. Es gilt dann für alle $a \in A$
\[ d(a,x) \leq d(a, a_0) + d(a, x) < \delta + \epsilon(x)\leq 2 \epsilon(x) \]
Daraus folgt
\[ A \subset B_{2\epsilon(x)}(x) \subset U_{i(x)} \]
\end{Beweis}

%\section{Parakompaktheit}
\Def{}
Eine Überdeckung eines topologischen Raumes durch offene Mengen heißt \df{lokal endlich}, wenn jeder Punkt des Raumes eine Umgebung besitzt, die nur endlich viele Elemente der Überdeckung schneidet.

\Def{}
Ein topologischer Raum heißt \df{parakompakt}, wenn jede Überdeckung durch offene Mengen eine lokal endliche Verfeinerung besitzt.

\Bem{}
\begin{itemize}
	\item Ist ein Raum parakompakt, so ist er auch \df{normal}, d.\,h., zwei disjunkte abgeschlossene Teilmengen besitzen in diesem Raum zwei disjunkte Umgebungen.
	\item Jeder metrische Raum ist parakompakt.
	\item Im Allgemeinem impliziert Parakompaktheit \textbf{nicht} Metrisierbarkeit.
\end{itemize}

\Def{}
Sei $X$ ein topologischer Raum mit einer Überdeckung $\{U_i\}_{i\in I}$ durch offene Mengen. Eine \df{Zerlegung der Eins} bzgl. $\{U_i\}_{i\in I}$ ist eine Familie $\{f_i\}_{}$ von stetigen Funktionen
\begin{align*}
f_i : X & \Pfeil{} \R
\end{align*}
mit
\begin{enumerate}[i.)]
	\item $\supp f_i := Cl(\set{x\in X}{f(x) \neq 0}) \subset U_i$,
	\item Für alle $x \in X$ gilt
	\[ f_i(x) = 0 \]
	für fast alle $i \in I$
	\item und
	\[\sum_{i\in I}f_i(x) = 1\]
	für alle $x \in X$.
\end{enumerate}

\Satz{}
Ein parakompakter Raum besitzt bzgl. jeder Überdeckung durch offene Mengen eine Zerlegung der Eins.

\section{Mannigfaltigkeiten}
Jede Mannigfaltigkeit, die sich durch zwei Karten mit zusammenhängendem Schnitt überdecken lässt, ist orientierbar.


\Def{}
Sei $\phi : M \pfeil{} N$ eine glatte Abbildung glatter Mannigfaltigkeiten.
\begin{enumerate}[1.)]
	\item $p \in M$ heißt ein \df{kritischer Punkt}, falls
	\[\phi_{p,*} : T_pM \pfeil{} T_{\phi(p)}N\]
	nicht surjektiv ist.
	\item $q \in N$ heißt ein \df{kritischer Wert}, falls es einen kritischen Punkt $p\in \phi\i(q)$ gibt.
	\item Ist $q \in N$ nicht kritisch, so nennen wir $q$ einen \df{regulären Wert}. 
\end{enumerate}
\Bem{}
\begin{itemize}
	\item Ist $\dim M < \dim N$, so gilt für $q \in N$
	\[q \text{ ist regulär } \Gdw{} q \notin \phi(M) \]
	\item Ist $\dim M \geq \dim N$, so gilt für $q \in N$
	\[ q \text{ ist regulär } \Gdw{} \forall p \in \phi\i(q): \phi_{p,*} \text{ hat als lineare Abbildung einen Rang von } \dim N \]
	\item Alle $q\in N - \phi(M)$ sind reguläre Werte.
\end{itemize}

\Satz{Sard}
Sei $f : U \off \R^n \pfeil{} \R^p$ glatt. Setze
\[ C:= \set{x \in \R^n}{x \text{ ist kein regulärer Punkt für }f} \]
Dann ist $f(C)$ eine {Nullmenge}.
\begin{Beweis}{}
Wir führen eine Induktion nach $n$:
Setze
\[ C_k := \set{x \in U}{\text{alle partiellen Ableitungen von }f\text{ der Ordnung } k\text{ verschwinden in }x} \]
Es gilt dann
\[ C \supseteq C_1 \supseteq C_2 \supseteq \ldots \]
Wir proklamieren folgende Dinge
\begin{enumerate}[(1)]
	\item $f(C\setminus C_1)$ ist eine Nullmenge.
	\item $f(C_k\setminus C_{k+1} )$ ist eine Nullmenge.
	\item Es gibt ein $k$, sodass $f(C_k)$ eine Nullmenge ist.
\end{enumerate}
Hieraus folgt dann, dass $f(C)$ eine Nullmenge ist.\\
Wir zeigen nun die proklamierten Dinge
\begin{enumerate}[(1)]
	\item $f(C\setminus C_1)$ ist eine Nullmenge:\\
	Sei $x' \in C\setminus C_1$. Dann gilt
	\[ \frac{\partial f_1}{\partial x_1}(x') \neq  0 \]
	Betrachte
	\begin{align*}
	h : U & \Pfeil{} \R^n\\
	x & \longmapsto (f_1(x), x_2, \ldots, x_n)
	\end{align*}
	$h$ ist dann in einer Umgebung von $x'$ invertierbar. Betrachte für ein passendes $V \off \R^n$
	\[ g:= f\circ h\i : V \Pfeil{} \R^p \]
	Betrachte
	\[ C' := h(C\cap V) \]
	$C'$ ist gerade die Menge der kritischen Punkte von $g$. Ferner genügt es zu zeigen, dass $g(C') = f(C\cap V)$ eine Nullmenge ist.\\
	Betrachte die Einschränkung
	\begin{align*}
	g_t : V \cap \{t\} \times \R^{n-1} & \Pfeil{} \R^n
	\end{align*}
	$g$ ist gerade so definiert, dass gilt
	\[ g(x_1, \ldots, x_n) = (x_1, y_2,\ldots, y_p) \]
	Deswegen ist der erste Eintrag der Jacobimatrix von $g$ eine Eins. Insofern gilt
	\[ C' = \bigcup_t C_t \]
	wobei $C_t$ die kritischen Punkte von $g_t$ sind. Nach der Induktionsvoraussetzung haben aber alle
	\[ g_t(C_t) \]
	Maß 0. Nach dem Satz von Fubini hat damit auch $g(C')$ Maß Null. Damit hat auch $f(C - C_1)$ Maß Null.
	\item $f(C_k\setminus C_{k+1} )$ ist eine Nullmenge:\\
	Sei $x' \in C_k\setminus C_{k+1}$. Dann gilt ohne Einschränkung
	\[ \frac{\partial^{k+1} f_1}{\partial x_1\ldots \partial x_{k+1}}(x') \neq  0 \]
	Betrachte
	\begin{align*}
	h : U & \Pfeil{} \R^n\\
	x & \longmapsto (\frac{\partial^{k} f_1}{\partial x_1\ldots \partial x_{k}}(x), x_2, \ldots, x_n)
	\end{align*}
	Nach der vorherigen Überlegung folgt nun für $h$
	\[ h(C^h - C_1^h) \text{ hat Maß Null} \]
	Hieraus folgt die Behauptung.
	\item Es gibt ein $k$, sodass $f(C_k)$ eine Nullmenge ist:\\
	Sei $I^n \subset \R^n$ ein Würfel mit Seitenlängen $\delta$. Es genügt zu zeigen, dass $f(C_k \cap I^n)$ Nullmaß hat.\\
	Seien $x$ und $x+h$ aus $I^n \cap C$. Durch eine Taylorentwicklung von $f$ bei $x$ sieht man ein, dass
	\[ \norm{f(x+h) - f(x)} \leq c \cdot \norm{h}^{k+1} \]
	Setze $k = n$. Durch Unterteilung von $I^n$ erhält man $2^n$ viele neue Unterwürfel (jedes $I$ wird halbiert). Dadurch wird die maximale Länge von $h$ halbiert. Ergo wird die maximale Distanz von Bildwerten von $f$ eines Unterwürfels um den Faktor $2^{n+1}$ reduziert. Hieraus folgt nun, dass $f(C_n\cap I^n)$ Maß Null haben muss, da wir sonst einen Widerspruch erhalten.
\end{enumerate}
\end{Beweis}

\Kor{Satz von Brown}
Sei $f : M \pfeil{} N$ glatt. Dann ist die Menge der regulären Werte von $f$ in $N$ dicht.

\Def{}
Sei $\phi : M \pfeil{} N$ eine glatte Abbildung glatter Mannigfaltigkeiten.
\begin{enumerate}[1.)]
	\item $\phi$ heißt \df{Submersion}, falls $\phi_{p,*} : T_pM \pfeil{} T_{\phi(p)}N $ für alle $p \in M$ surjektiv ist.
	\item $\phi$ heißt \df{Immersion}, falls $\phi_{p,*} : T_pM \pfeil{} T_{\phi(p)}N $ für alle $p \in M$ injektiv ist.
	\item $(M,\phi)$ heißt eine \df{Untermannigfaltigkeit} von $N$, falls $\phi$ eine injektive Immersion ist.
	\item $(M,\phi)$ heißt eine \df{Einbettung} in $N$, falls sie eine Untermannigfaltigkeit ist und ein $\phi$ einen Homöomorphismus von $M$ auf ihr Bild ist.
\end{enumerate}

\Prop{}
Sei $\phi : M \pfeil{} N$ eine glatte Abbildung glatter Mannigfaltigkeiten der Dimensionen $m$ bzw. $n$. $p\in M$ sei ein beliebiger Punkt.
\begin{itemize}
	\item Ist $\phi$ immersiv bei $p$, so existieren Karten $U\subset M, V\subset N$, um $p$ bzw. $\phi(p)$ und eine Abbildung
	\begin{align*}
	\widetilde{\phi} : \R^m& \Pfeil{} \R^n\\
	(x_1,\ldots, x_m) & \longmapsto (x_1,\ldots, x_m, 0,\ldots, 0)
	\end{align*}
	sodass folgendes Diagramm kommutiert
	\begin{center}
		\begin{tikzcd}
			U \arrow[r, "\phi"] \arrow[d, "\isom{}"]	& V \arrow[d, "\isom{}"] \\
			\R^m  \arrow[r, "\widetilde{\phi}"] 	& \R^n
		\end{tikzcd}
	\end{center}
	\item Ist $\phi$ submersiv bei $p$, so existieren Karten $U\subset M, V\subset N$, um $p$ bzw. $\phi(p)$ und eine Abbildung
\begin{align*}
\widetilde{\phi} : \R^m& \Pfeil{} \R^n\\
(x_1,\ldots, x_m) & \longmapsto (x_1,\ldots, x_n)
\end{align*}
sodass folgendes Diagramm kommutiert
\begin{center}
	\begin{tikzcd}
		U \arrow[r, "\phi"] \arrow[d, "\isom{}"]	& V \arrow[d, "\isom{}"] \\
		\R^m  \arrow[r, "\widetilde{\phi}"] 	& \R^n
	\end{tikzcd}
\end{center}
\end{itemize}

\Kor{}
Sei $\phi : M\pfeil{} N$ eine glatte Abbildung glatter Mannigfaltigkeiten. Ist $q\in N$ regulär, so ist $\phi\i(q)\subset M$ eine eingebettete Untermannigfaltigkeit der Dimension $\dim M - \dim N$.

\Satz{}
Seien $U,V \subset M$ glatte eingebettete Untermannigfaltigkeit. $U$ und $V$ schneiden sich \df{transversal}, wenn für alle $x\in U\cap V$ gilt
\[ T_xU +T_xV = T_xM \]
In einem solchen Fall ist $U\cap V$ eine eingebettete Untermannigfaltigkeit der Dimension $\dim U + \dim V - \dim M$.

\section{Vektorraumbündel}
\Def{}
Ein \df{Vektorraumbündel} von Rang $n$ ist Tripel $(p,E,B)$, bei der $E,B$ topologisch Räume und $ p:E\pfeil{} B$ eine stetige Abbildung sind, die folgende Eigenschaften erfüllen
\begin{itemize}
	\item $p$ ist \df{lokal trivial}, d.\,h., jeder Punkt $b \in B$ hat eine Umgebung $U\subset B$ zusammen mit einem Diffeomorphismus
	\[ \phi_U : p\i(U) \Pfeil{} U \times \R^n \]
	sodass folgendes Diagramm kommutiert
		\begin{center}
		\begin{tikzcd}
			p\i(U) \arrow[rr, "\phi_U"] \arrow[dr, "p"] &	& U\times \R^n \arrow[dl, "\pi"] \\
			& U &  
		\end{tikzcd}
	\end{center}
	\item Obiges $\phi_U$ induziert \df{faserweise} Isomorphismen, d.\,h., für alle $x\in U$ hat $p\i(x)$ eine gegebene Vektorraumstruktur, für die
	\[ \phi_{|x} : p\i(x) \Pfeil{}  \{x\}\times \R^n \]
	ein Isomorphismus von Vektorräumen ist.
\end{itemize}
In diesem Setting heißt $B$ der \df{Basisraum}, $E$ der \df{Totalraum}, $p$ die lokal triviale \df{Projektion} und $p\i(b)$ die \df{Faser} über $b \in B$.

\Bem{}
Ist ein Vektorraumbündel wie oben gegeben, so erhalten wir für zwei Karten $U,V\subset B$ folgendes Diagramm
\begin{center}
\begin{tikzcd}
	& (U\cap V) \times \R^n \arrow[dd, "\theta_{UV}"]\\
	p\i(U\cap V) \arrow[ru, "\phi_U"] \arrow[dr, "\phi_V"] & \\
	&  (U\cap V) \times \R^n
\end{tikzcd}
\end{center}
$\theta_{UV}$ ist dabei von der Gestalt
\[ \theta_{UV}(x,y) = (x, \mathfrak{g}_{U,V}(x)\cdot y) \]
mit $\mathfrak{g}_{U,V} : U\cap V \pfeil{} GL_n(\R)$ stetig. $GL_n(\R)$ nennt man hier die \df{Strukturgruppe} von $(p,E,B)$ und die $\mathfrak{g}_{U,V}$ nennt man die \df{Übergangsfunktionen}. Diese erfüllen funktorielle Eigenschaften:
\begin{itemize}
	\item $\mathfrak{g}_{U,U} = \id{}$
	\item $\mathfrak{g}_{V,W} \cdot \mathfrak{g}_{U,V}  = \mathfrak{g}_{U,W}$
\end{itemize}

\Def{}
Seien $(p,E,B)$ und $(p',E',B')$ zwei Vektorraumbündel. Eine \df{Homomorphismus} von Vektorraumbündeln ist ein kommutatives Diagramm
\begin{center}
	\begin{tikzcd}
	E\arrow[r,"F"]\arrow[d, "p"]	& E' \arrow[d, "p'"] \\
	B \arrow[r, "f"] & B'
	\end{tikzcd}
\end{center}
wobei $F$ und $f$ stetig sind, und $F$ faserweise linear ist, d.\,h.
\[ F_{|p\i(b)} : p\i(b) \Pfeil{} {p'}\i(f(b)) \]
ist ein Homomorphismus von Vektorräumen für alle $b\in B$.

\Bem{}
Ein Homomorphismus $(F,f)$ von Vektorraumbündeln ist genau dann ein Isomorphismus, wenn $f$ homöomorph ist und $F$ auf jeder Faser einen Isomorphismus induziert.

\Def{}
$(p,E,B)$ heißt \df{trivial}, falls $E \isom{} B\times \R^n$.

\Def{}
Eine glatte Mannigfaltigkeit heißt \df{parallelisierbar}, wenn ihr Tangentialbündel trivial ist.

\Satz{Einbettungssatz von Whitney}
Sei $M$ eine glatte, geschlossene Mannigfaltigkeit der Dimension $n$. Dann existiert eine Einbettung $M \subset \R^{2n+1}$ von $M$ als Untermannigfaltigkeit.
\begin{Beweis}{}
\begin{itemize}
	\item Sei $U_1, \ldots, U_k$ eine Überdeckung von $M$ durch Karten mit Diffeomorphismen $\phi_1,\ldots, \phi_k$. Wir wählen zusätlich offene Mengen $V_1,\ldots, V_k$ so, dass diese $M$ überdecken und dass gilt
	\[ \overline{V_i} \subset U_i \]
	Ferner wählen wir glatte Funktionen $\lambda_i : M \pfeil{} \R$ mit
	\begin{align*}
	\lambda_{i|V_i} \equiv 1 \text{  und  } \supp \lambda_i \subset U_i
	\end{align*}
	Definiere nun glatte Abbildungen
	\begin{align*}
	\psi_i : M & \Pfeil{} \R\\
	x &\longmapsto \left\lbrace
	\begin{aligned}
	\lambda_i(x)\phi_i(x) && x\in U_i\\
	0 && x \notin U_i
	\end{aligned}
	\right.
	\end{align*}
	Wir erhalten nun eine glatte Abbildung
	\begin{align*}
	\Theta : M &\Pfeil{} (\R^{n})^k \times \R^k\\
	x & \longmapsto (\psi_1(x), \ldots, \psi_k(x), \lambda_1(x), \ldots, \lambda_k(x))
	\end{align*}
	\item Wir wollen zeigen, dass $\Theta$ eine Einbettung ist.\\
	Sei $p \in V_i, 0\neq v \in T_pM$. Angenommen es gilt
	\[ \Theta_{*,p}(v) = 0 \]
	Dann gilt insbesondere
	\[ \psi_{i,*,p}(v) = 0 \]
	$\lambda_i$ ist in einer Umgebung von $p$ konstant 1, ergo gilt
	\[ \phi_{j,p,*}(v) = 0 \]
	Aber $\phi_{j,p,*}$ ist ein Diffeomorphismus, ergo erhalten wir einen Widerspruch. Insofern ist $\Theta$ immersiv.\\
	Anhand der Definition sieht man auch ein, dass $\Theta$ injektiv ist. Ferner ist $\Theta$ ein Homöomorphismus auf sein Bild, da $M$ kompakt und $\R^{nk + k}$ ein Hausdorffraum ist.
	\item Wir haben nun eine Einbettung
	\[ \Theta : M \Pfeil{} \R^N \]
	und wollen $N$ auf $2n+1$ verringern. Dazu nehmen wir an, dass es ein $0\neq w \in \R^N$ gibt mit
	\begin{align*}
	&w \text{ ist tangential zu } \Theta(M)\\
	\forall x,y \in \Theta(M):~& x\neq y \Impl{} x-y \text{ ist nicht parallel zu } w
	\end{align*}
	In diesem Fall ergibt sich folgendes Diagramm
	\begin{center}
		\begin{tikzcd}
			\R^N\arrow[rr]	&  & w^\bot \\
			M \arrow[u, "\Theta"]\arrow[rru, "\Theta'"]  & &
		\end{tikzcd}
	\end{center}
wobei $\Theta'$ wieder eine Einbettung liefert.
\item Wir wollen die Existenz von oben proklamierten $w$s zeigen und betrachten die Projektion
\begin{align*}
\R^N \Pfeil{} P^{N-1}\R
\end{align*}
Wir erhalten zwei Abbildungen
\begin{align*}
\tau : \T M - M \subset \R^N & \Pfeil{} P^{N-1}\R\\
v & \longmapsto [v]\\
\sigma : M\times M - \Delta(M) & \Pfeil{} P^{N-1}\R\\
(x,y) & \longmapsto [x-y]
\end{align*}
Laut dem Satz von Sard besitzen beide Abbildungen einen gemeinsamen regulären Wert $[w]$. Da
\[ \dim M\times M= \dim \T M = 2n< N-1 = \dim P^{N-1}\R \]
kann dieser Wert nicht in den Bildern von $\tau$ und $\sigma$ liegen, ergo erfüllt $w$ obige Eigenschaften.
\end{itemize}
\end{Beweis}

\Def{}
Sei $M$ eine $n$-dimensionale glatte Mannigfaltigkeit mit einer Einbettung $M \subset \R^{n+k}$. Setze
\[ E:= \set{ (p,v) \in M\times \R^{n+k} }{ x\bot T_pM } \]
Dann ist $(\pi, E, M)$ das \df{Normalenbündel} von Rang $k$ bzgl. $M \subset \R^{n+k}$.

\Satz{Tubenumgebung}
Ist $M$ kompakt im obigen Setting,\\
so existiert eine offene \df{Tubenumgebung} $U\subset \R^{n+k}$ von $M$ mit
\[ U \isom{} E \]
Inbesondere kommt $U$ mit einem Deformationsretrakt $r : U \pfeil{} M$ einher.
\begin{Beweis}{}
Wir setzen für $\e > 0$
\[ E(\e) := \set{(p,v) \in E}{\norm{v}< \epsilon} \]
Offensichtlich liegt dann folgende Isomorphie vor
\[ E(\e) \isom{} E \]
Durch die Exponentialabbildung von $\R^{n+k}$ erhalten wir eine glatte Abbildung
\begin{align*}
\exp : E(\e) & \Pfeil{} \R^{n+k}\\
(p,v) & \longmapsto \exp_p(v) = p +v 
\end{align*}
Da $M$ kompakt ist, können wir $\e$ so klein wählen, dass $\exp$ zu einer Einbettung der Untermannigfaltigkeit $E(\e)$ wird. Dann setzen wir
\[ U:= \exp(E(\e)) \]
\end{Beweis}
%Vorstufe zum glatten Approximationssatz
\setcounter{chapter}{3}
\chapter{\textsc{Vektorraumbündel}}
\marginpar{Vorlesung vom 1.12.17}
\section{Glatter Approximationssatz}

\Prop{}
Sei $M$ eine glatte Mannigfaltigkeit, $A \abg M$, $f : M \pfeil{} \R^k$ stetig, sodass $f_{|A}$ glatt ist.\\
Dann existiert für alle $\e > 0$ eine Abbildung $g : M \pfeil{{}} \R^k$ mit:
\begin{enumerate}[1.)]
	\item $g$ ist glatt
	\item $g_{|A} = f_{|A}$
	\item $\norm{f(x) - g(x)} < \e ~~\forall x \in M$
	\item $g \simeq f$ relativ $A$ durch eine $\epsilon$-kleine Homotopie, d.\,h., es existiert eine Homotopie $H : M \times I \pfeil{} \R^k$ mit
	\begin{enumerate}
		\item $H(x,t) = H(x,0) ~~~\forall t \in I,x \in A $
		\item $H(x,0) = f(x)~~~\forall x \in A$
		\item $H(x,1) = g(x)~~~\forall x \in A$
		\item $d(H(x,t_1), H(x, t_2)) < \e ~~~ \forall x \in M, t_1, t_2 \in I $
	\end{enumerate} 
\end{enumerate}
\begin{Beweis}{}
	Für alle $x \in M$ wählen wir:
	\begin{enumerate}[\text{Fall} 1]
		\item $x \in A$:\\
		Dann existiert eine offene Umgebung $V_x\subset M$ und eine glatte Abbildung $h_x : V_x \pfeil{} \R^k$ mit
		\[ {h_x}_{|V_x \cap A} = f_{|V_x\cap A} \]
		\item $x\notin A:$\\
		Wähle $V_x\off M$ mit
		\[ V_x \cap A = \emptyset \]
		und wähle $h_x : V_x \pfeil{} \R^k$ glatt mit
		\[ h_x(y) = f(x) \]
		für alle $y \in V$. Außerdem stellen wir sicher, dass die $V_x$ so klein sind, dass für $x, x' \notin A$ gilt
		\[ \norm{h_x(y) - f(x')} < \frac{\e}{2}, ~~~ \norm{f(y) - f(x)}< \frac{\e}{2} \]
	\end{enumerate}
Sei $(U_\alpha)_\alpha$ eine lokal endliche Verfeinerung von $(V_x)_x$ mit
\[ U_{\alpha} \subset V_{x(\alpha)} \]
Sei $(\lambda_\alpha)_\alpha$ eine glatte Partition der Eins mit $\text{supp}\lambda_\alpha \subset U_\alpha$.\\
Wir setzen
\[ g(y) := \sum_{\alpha}\lambda_\alpha(y) h_{x(\alpha)}(y) \]
Dann ist $g : M \pfeil{} \R^k$ bereits glatt.\\
Sei $y \in A$. Wenn $y \notin V_{x(\alpha)}$, dann ist $\lambda_\alpha(y) = 0$, denn $\text{supp}(\lambda_\alpha) \subset V_{x(\alpha)}$. Daraus folgt
\[ g(y) = \sum_{\alpha : y \in V_{x(\alpha)}\cap A} \lambda_\alpha(y) h_{x(\alpha)}(y) = \sum_{\alpha : y \in V_{x(\alpha)}\cap A} \lambda_\alpha(y)f(y) = f(y) \]
bzw.
\[ f_{|A} = g_{|A} \]
Sei $y \notin A$
\[ g(y) - f(y) = \sum_{\alpha} \lambda_\alpha(y) (h_{x(\alpha)}(y) - f(y)) \]
Da $\norm{h_{x(\alpha)}(y) - f(y)} \leq \norm{h_{x(\alpha)}(y) - f(x)} + \norm{f(x)- f(y)} \leq \e$, folgt
\[ \norm{g(y) - f(y)} \leq \e \]
Wir definieren nun die Homotopie zwischen $f$ und $g$ durch
\begin{align*}
H(x,t) := t\cdot f(x) + (1- t) g(x)
\end{align*}
\end{Beweis}

\Satz{Glatter Approximationssatz}
Sei $M$ eine glatte Mannigfaltigkeit der Dimension $m$, sei $N$ eine glatte, kompakte und metrische Mannigfaltigkeit der Dimension $n$. Sei $A\abg M$, $f : M\pfeil{} N$ stetig. $f$ sei auf $A$ eingeschränkt glatt. Dann gilt:\\
Für jedes $\e > 0$ existiert eine glatte Abbildung $h : M \pfeil{} N$, sodass gilt:
\begin{enumerate}
	\item $h$ stimmt auf $A$ mit $f$ überein.
	\item $f$ und $h$ sind durch eine $\e$-kleine Homotopie relativ zu $A$ verbunden.
\end{enumerate}
\begin{Beweis}{}
$N$ habe eine glatte Einbettung $\iota : N \inj{} \R^k$. Da $N$ kompakt ist, existiert für jedes $\e > 0$ ein $\delta > 0$, sodass für alle $p,q \in N$ gilt
\[ \norm{\iota(p)- \iota(q)} < \delta \Impl{} d(p,q) < \e \]
d.\,h., $\iota\i$ ist gleichmäßig stetig.\\
Dies motiviert im Folgenden $\norm{\cdot}$ auf $\R^k$ statt $d$ auf $N$ zu betrachten.\\
Wir fixieren ein $\e > 0$. Der Satz über Tubenumgebungen impliziert die Existenz einer $\frac{\delta}{2}$-Umgebung $U \off \R^k$ von $\iota(N)$, sodass $U \isom{} E(\frac{\delta}{2})$, wobei $E$ das Normalenbündel zu $\iota$ war.\\
Aus der vorhergenden Proposition folgt nun die Existenz einer glatten Abbildung $g : M\pfeil{} \R^k$, die $\frac{\delta}{2}$-klein und relativ zu $A$ homotop zu $\iota \circ f$ ist. Das Bild von $g$ liegt dann ganz in $U$.\\
Sei $r : U \pfeil{} N$ ein glatter Deformationsrektrakt. Wir können fordern, dass diese eine $\frac{\e}{2}$-kleine Homotopie induziert. Dann ist $r \circ g$ glatt und homtop zu $f$ via einer $\e$-kleinen Homotopie relativ zu $A$.
\end{Beweis}

\Bem{}
\begin{itemize}
\item Der Metrisierbarkeitssatz von Smirnov besagt.
\begin{center}
	Ist $X$ ein parakompakter, lokal metrisierbarer Hausdorffraum, so ist $X$ global metrisierbar.
\end{center}
Insbesondere sind Mannigfaltigkeiten immer metrisierbar.
\item Sei $ f: M \pfeil{} S^n$ eine stetige Abbildung. $m = \dim M < n$. Der Glatte Approximationssatz impliziert nun die Existenz einer glatten Abbildung $f : M \pfeil{} S^n$, die homotop zu $f$ ist.\\
Der Satz von Sard proklamiert nun die Existenz eines regulären Wert $p \in S^n$ von $g$. Da $m < n$, folgt aber hieraus
\[ p \notin g(M) \]
$S^n- p \isom{} \R^n$, ergo erhalten wir eine glatte Abbildung $ g : M \pfeil{} \R^n$. Hieraus folgt aber, dass $g$ nullhomotop ist. Insbesondere ist auch $f$ nullhomoptop.\\
D.\,h., eine stetige Abbildung von einer glatten Mannigfaltigkeit in eine höherdimensionale Sphäre ist immer null-homotop.
\item $\partial D^{n+1} = S^n$ ist kein Retrakt von $D^{n+1}$.\\
Denn angenommen, es gäbe eine Retraktion $r : D^{n+1} \pfeil{} S^n$. Definiere
\[ D^{n+1}_{\leq \frac{1}{2}} = \set{ x \in \R^{n+1} }{\norm{x} \leq \frac{1}{2}} \text{ und } \partial D^{n+1}_{\leq \frac{1}{2}} = S^n_{\frac{1}{2}} \]
Analog erhalten wir $r_{\frac{1}{2}} : D^{n+1}_{\leq\frac{1}{2}} \pfeil{} S^n_{\frac{1}{2}}$. Betrachte ferner
\[ p:\R^{n+1} \pfeil{} S^n, x \mapsto \frac{x}{\norm{x}} \]
Definiere nun
\begin{align*}
f : \R^{n+1} & \Pfeil{} S^n\\
x & \longmapsto \left\lbrace
\begin{aligned}
p(r_{\frac{1}{2}}(x)) && \norm{x} \leq \frac{1}{2}\\
p(x) && \norm{x} \geq \frac{1}{2}
\end{aligned}
\right.
\end{align*}
$f$ ist glatt in einer kleinen Umgebung vom $S^n$. Betrachte
\[ f_{|D^{n+1}} \Pfeil{} S^n \]
Diese Abbildung ist stetig, ergo homotop zu einer glatten Abbildung $g : D^{n+1} \pfeil{} S^n$, wobei $f_{|S^n} = g_{|S^n}$.\\
Mit dem Satz von Sard existiert ein regulärer Wert für $g$ (und $g_{|S^n}$).\\
$g\i(p)$ ist dann eine glatte, kompakte Untermannigfaltigkeit der Dimension 1 mit Rand. Es gilt folgende Randformel
\[ \partial g\i(p) = (g\i(p)) \cap \partial D^{n+1} \]
Dann ist $g\i(p)$ eine endliche Vereinigung von Kreisen in $\text{int}(D^n)$ und kompakten Intervallen mit Randpunkten in $S^n$. Allerdings gilt
\[ \partial g\i(p) = \{ p\} \]
da $g$ die ganze Faser $g\i(p)$ auf $p$ schickt und $g$ auf $S^n$ die Identität ist.
Deswegen kann die Zahl der Randpunkte von $g\i(p)$ nicht gerade sein.
\end{itemize}

\Lem{}
Sei $(M, \partial M)$ eine berandete Mannigfaltigkeit, $g : M\pfeil{} N$ glatt.\\
$p \in N$ sei regulär für $g$ und für $g_{|\partial M}$. Dann gilt
\[ \partial g\i(p) = g\i(p) \cap \partial M \]

\Bsp{}
Betrachte $g:D^2 \pfeil{} \R$ durch
\[ g(x,y)  = x^2 + y^2 \]
$p = 1$ ist ein regulärer Wert für $g$, aber nicht für $g_{|S^1} \gl{\text{konst.}} 1$. Es gilt
\[ g\i(p) \cap \partial D^2 = S^1 \cap S^1 = S^1 \]
aber
\[ \partial g\i(p) = \partial S^1 = \emptyset \]
\Kor{Brownscher Fixpunktsatz}
\marginpar{Vorlesung vom 4.12.17}
Jede stetige Abbildung $f : D^n \pfeil{} D^n$ hat einen Fixpunkt.
\begin{Beweis}{}
	Wir nehmen an, $f : D^n \pfeil{} D^n$ habe keinen Fixpunkt. Wir definieren dann folgende stetige Abbildung
	\begin{align*}
	r : D^n & \Pfeil{} S^{n-1}\\
	x & \longmapsto x + t (x - f(x))
	\end{align*}
	s.d. $x + t(x- f(x)) \in S^{n-1}$.
Er ist insbesondere ein Retrakt auf $S^{n-1}$, da $r$ die Identität auf $S^{n-1}$ ist. Dies steht im Widerspruch zum obigen Satz.
\end{Beweis}

\Kor{}
Sphären sind nicht zusammenziehbar.
\begin{Beweis}{}
Angenommen, $S^n$ wäre zusammenziehbar. Dann existiert eine Homotopie
\[H : S^n \simeq p \in S^n\]
der folgendes Diagramm induziert
\begin{center}
\begin{tikzcd}
	S^n\times I \arrow[r, "H"] \arrow[d, "Quot"]	& S^n \\
	S^n \times I/S^n \times \{1\} \arrow[ur, dashed, "\exists_1 \overline{H}" ]		& 
\end{tikzcd}
\end{center}
Wir haben also eine stetige Abbildung
\[ \overline{H} : D^{n+1} \Pfeil{} S^n \]
wobei gilt
\[ \overline{H}_{|S^n} = \id{S^n} \]
Ergo ist $\overline{H}$ ein Retrakt von $D^{n+1}$ auf $S^n$. Dies ist ein Widerspruch.
\end{Beweis}

\section{Homogenität von Mannigfaltigkeiten}
Wir wollen Folgendes zeigen in diesem Kapitel.
\Satz{}
\label{SatzHomogenitat}
Sei $M$ eine zusammenhängende, geschlossene Mannigfaltigkeit und $p,q \in M$ beliebige Punkte. Dann existiert ein Diffeomorphismus $\phi:M\pfeil{} M$, der sogar isotop zur Identität ist, mit
\[ \phi(p) = q \]

\Lem{}
Für $p = 0 \in \R^n$ und $q \in \R^n$ mit $\norm{q} < 1$ existiert ein Diffeomorphismus $\tau : \R^n \pfeil{} \R^n$ mit
\begin{enumerate}[i.]
	\item $\tau(p) = q$
	\item $\tau(x) = x$ für alle $x \in \R^n$ mit $\norm{x} \geq 1$
	\item $\tau$ ist isotop zu $\id{\R^n}$
\end{enumerate}
\begin{Beweis}{}
Ohne Einschränkung liege $q \in [0,1)$ auf einer Achse.\\
Wähle eine glatte Funktion $\lambda : \R^n \pfeil{} \R$ mit
\begin{align*}
\lambda(x) > 0 & \text{ für } \norm{x} < 1\\
\lambda(x) = 0 & \text{ für } \norm{x} \geq 1
\end{align*}
Sei $v_0 \in S^{n-1}$. Wir definieren folgendes Vektorfeld auf $\R^n$
\[ v(x) := \lambda(x) \cdot v_0 \]
Dann ist $v(x) = 0$ für $\norm{x} \geq 1$.\\
Wir betrachten nun folgende gewöhnliche Differentialgleichung
\begin{align*}
\left\lbrace
\begin{aligned}
x'(t) &= v(x(t))\\
x(0) &=x_0
\end{aligned}
\right.
\end{align*}
Diese hat lokal eine eindeutige Lösung $x(t)$, die glatt von $x_0$ abhängt. Hier existiert die Lösung für alle $t \in \R$, da $v(x)$ außerhalb einer kompakten Menge verschwindet. Definiere
\begin{align*}
\tau_t(x_0) &:= x(t)\\
\tau_0(x_0) &:= x_0
\end{align*}
Die $\{\tau_t~|~ t\in \R\} \subset \text{Diffeo}(\R^n)$ bilden dann eine Einparametergruppe von Diffeomorphismen. Es gilt
\[ \set{ \tau_t(0) }{t \in \R^n} = [0,1) \]
Ergo erfüllt eines der $\tau_t$ die Voraussetzungen.
\end{Beweis}

\begin{Beweis}{Satz \ref{SatzHomogenitat}}
Seien $p, q \in M$.
\begin{itemize}
	\item Liegen $p,q$ im Definitionsbereich einer Karte $U \subset M$, dann konstruieren wir ein Koordinatensystem $x$ um $p$ mit
	\begin{align*}
	x(p) &= 0\\
	x(q) &= (\frac{1}{2}, 0, \ldots, 0)
	\end{align*}
	Dann verwenden wir das $\tau : U \pfeil{} U$ aus dem vorhergehenden Lemma, um $p$ auf $q$ abzubilden und setzen $\tau$ durch die Identität zu einem Diffeomorphismus auf $M$ fort.
	\item Sind $p,q$ beliebig auf $M$ verteilt, so können wir $M$ mit endlich vielen Karten überdecken und eine Sequenz von Punkten
	\[ p = p_0 \pfeil{} p_1 \pfeil{} p_2  \pfeil{} \ldots \pfeil{} p_k = q \]
	finden, bei denen zwei hintereinander folgende Punkte in einer Karte liegen. Wir konstruieren nun induktiv Diffeomorphismen $\tau : p \mapsto p_{i}$.
\end{itemize}
\end{Beweis}

\newpage
\section{Theorie der Abbildungsgrade}
Seien $M,N$ glatte, geschlossene Mannigfaltigkeiten derselben Dimension $n$. \\
Sei $f : M \pfeil{} N$ eine glatte Abbildung.\\
Wir wollen $f$ einen \df{Abbildungsgrad} $\deg f \in \Z/ 2\Z$ zuordnen, sodass gilt
\begin{align*}
\deg f = \deg g 
\end{align*}
für $f \simeq g$. Dadurch wird der Abbildungsgrad zu einer Homotopieinvariante von $f$.\\
Sei $p \in N$ ein regulärer Wert von $f$. Dann ist $f\i(p)$ eine nulldimensionale, kompakte Untermannigfaltigkeit von $M$, also eine endliche Menge von Punkten. Wir setzen
\[ \deg_pf := \# f\i(p) \mod 2 \]

\Lem{}
\label{GradLemma1}
Seien $f,g : M \pfeil{} N$ glatt und sei $H :M\times I \pfeil{} N$ eine Homotopie von $f$ nach $g$. Ist $p$ ein regulärer Wert für $f,g$ und $H$, so gilt
\[ \deg_pf = \deg_p g \mod 2 \]

\begin{Beweis}{}
	$H\i(p)$ ist eine kompakte eindimensionale Untermannigfaltigkeit von $M \times I$ mit Rand
	\[ \partial H\i(p) = \partial(M\times I) \cap H\i(p) = f\i(p) \times \{0\} \cup g\i(p) \times \{1\} \]
	$H\i(p)$ ist eine disjunkte Vereinigung von endlich vielen Kreisen und kompakten Intervallen. Daraus folgt
	\[ \# \partial H\i(p) \equiv 0 \mod 2 \]
	Nun gilt aber
	\[ \# \partial H\i(p) = \# f\i(p) + \# g\i(p) \]
	Daraus folgt die Behauptung.
\end{Beweis}

\Lem{}
In Lemma \ref{GradLemma1} genügt es anzunehmen, dass $p$ ein regulärer Wert für $f$ und $g$ ist.
\begin{Beweis}{}
	Auf einer hinreichend kleinen Umgebung eines Urbildes von $p$ unter $f$ ist $f$ ein lokaler Diffeomorphismus, da $p$ regulär ist und $\dim M = \dim N$.\\
	Also ist jeder Punkt, der hinreichend nahe bei $p$ liegt, auch ein regulärer Wert von $f$ und $g$.\\
	Laut dem Satz von Sard existiert ein $p'$ hinreichend nahe bei $p$, sodass $p'$ ein regulärer Wert von $H$ ist. $p'$ ist dann insbesondere regulär für $f$ und $g$.
\end{Beweis}

\Lem{}
Sei $N$ zusammenhängend. Seien $p,q$ reguläre Werte von $f$. Dann gilt
\[ \deg_pf = \deg_qf \mod 2  \]
\begin{Beweis}{}
	Aus der Homogenität von $N$ folgt die Existenz eines Diffeomorphismus $\tau : p \mapsto q$, der isotop zur Identität ist. Es gilt
	\[ (\tau f)\i(q) = f\i(p) \]
	Nun gilt
	\[ \deg_q f \equiv \deg_q \tau \circ f = \deg_p f \mod 2 \]
\end{Beweis}
\section{Orientierungen}
\marginpar{Vorlesung vom 8.12.17}
Seien $V,W$ reelle Vektorräume der Dimension $n$.
\Def{}
Eine \df{Orientierung} von $V$ ist die Äquivalenzklasse einer geordneten Basis $(\alpha_1, \ldots, \alpha_n)$ von $V$, wobei zwei geordnete Basen $(\alpha_1, \ldots, \alpha_n)$ und $(\beta_1, \ldots, \beta_n)$ genau dann äquivalent sind, wenn die Determinante des Isomorphismus $\Phi$, der $\alpha_i$ auf $\beta_i$ abbildet, positiv ist.
\Bem{}
Ist $n> 0$, so hat $V$ genau eine Orientierung. Anderenfalls hat $V$ genau eine Orientierung.
\Def{}
Die \df{kanonische Orientierung} von $\R^1$ ist gegeben durch
\[ [(+1)] \]

\Bem{}
Sind $V$ und $W$ orientiert\footnote{Ein \df{orientierter} Vektorraum ist ein Vektorraum zusammen mit einer fixierten Orientierung.}, so ist auch $V\oplus W$ orientiert. D.\,h.
%Diese Zuordnung ist kanonisch. Wir schreiben dies (aus sich mir nicht erschließenden Gründen) im Typenkalkül
\begin{equation*}
\begin{aligned}
V : &[(\alpha_1, \ldots, \alpha_n)]\\
 W : &[(\beta_1, \ldots, \beta_n)]\\
\midrule
V\oplus W :& [(\alpha_1,\ldots,  \alpha_n, \beta_1, \ldots, \beta_n)]
\end{aligned}
\end{equation*}
Insbesondere ist $\R^n$ kanonisch orientiert.\footnote{Haben $V\oplus W$ und $W \oplus V$ dieselbe Orientierung?}

\Def{}
Eine \df{stabile Orientierung} von $V$ ist eine Orientierung $V \oplus \R^1$. \footnote{Also eigentlich eine von $V \oplus \R^n$ für $n > 0$. Beachte, dass $\R^n$ kanonisch orientiert ist für alle $n > 0$, und diese Orientierungen können kompatibel gewählt werden, d.\,h., $\R^n \oplus \R^m \isom{} \R^{n+m}$ ist kanonisch orientiert, wenn $\R^n$ und $\R^m$ beide kanonisch bzw. antikanonisch orientiert sind.}

\Bem{}
Jeder Vektorraum hat genau zwei stabile Orientierungen.

\Bsp{}
Der nulldimensionale Vektorraum hat die beiden stabilen Orientierungen
\[ [(+1)] \text{ und } [(-1)] \]
die wir kurz auch einfach nur als $+$ und $-$ bezeichnen werden.

\subsection{Prinzip}
Sind zwei der Elemente aus $\{V, W, V\oplus W\}$ stabil orientiert, so ist auch das dritte auf kanonische Weise stabil orientiert.

\Bsp{}
Es sei $V = 0$. $W = \R^1$ kanonisch orientiert. $V\oplus W = \R^1$ sei antikanonisch orientiert\footnote{An dieser Stelle ist es wichtig zu erwähnen, dass der kanonische Isomorphismus $V\oplus W \isom{} \R^1 $ hier orientierungserhaltend gewählt ist.}. Dann erhält $V$ die stabile Orientierung $-$.

\Def{}
Ein Vektorraumhomomorphismus $\Phi : V \pfeil{} W$ heißt \df{orientierungserhaltend}, falls
\[ [ (\Phi(\alpha_1), \ldots, \Phi(\alpha_n) )] = [ (\beta_1, \ldots, \beta_n)] \]
Anderenfalls heißt $\Phi$ \df{orientierungsumkehrend}.

\Def{}
Sei $p : E \pfeil{} B$ ein Vektorraumbündel von Rang $n$ mit lokaler Trivialisierung
\[p\i (U_\alpha) \Pfeil{\isom{\phi_\alpha} } U_\alpha \times \R^n\]
Eine \df{Orientierung} von $p$ ist eine Familie von Orientierungen der $p\i(b), b\in B$, sodass alle
\[ p\i(b) \Pfeil{ \phi_\alpha } \{b\} \times \R^n  \]
orientierungserhaltend sind, wobei $\{b\} \times \R^n$ kanonisch orientiert ist.\\
Dies ist genau dann der Fall, wenn die Übergangsfunktionen
\[ \phi_{\alpha \beta} : U_\alpha \cap U_\beta \Pfeil{} \text{GL}_n(\R) \]
nur orientierungserhaltende Isomorphismen als Bilder annehmen, d.\,h.
\[ \det \phi_{\alpha \beta}(x) > 0 ~~~~\forall x \in U_\alpha \cap U_\beta \]

\Bem{}
Ein orientierbares Vektorraumbündel mit einem zusammenhängenden Basisraum hat genau zwei stabile Orientierungen.\\
Denn, seien $\alpha, \beta$ zwei Orientierungen von $p : E \pfeil{} B$. Die Menge der $b \in B$ an denen $\alpha$ und $\beta$ übereinstimmen ist offen. Die Menge, in denen sich $\alpha$ und $\beta$ unterscheiden ist aber auch offen. Allgemein können wir eine Orientierung als eine stetige Abbildung
\[ \text{or} : B \Pfeil{} \{+, -\} \]
auf"|fassen.

\Bsp{}
Das Möbiusbündel ist nicht orientierbar.

\subsection{Prinzip}
Sind zwei Elemente aus der Menge der Vektorbündel ${E, E', E\oplus E'}$ über $B$ orientiert, so bestimmt dies eindeutig und kanonisch eine Orientierung des Dritten.

\Bem{}
Vektorraumbündel über nullhomotopen Basisräumen sind immer orientierbar.
\begin{Beweis}{}
Definiere
\[ \widehat{X} := \set{(x, \alpha)}{x \in X, \alpha : \text{ Orientierung von }p\i(x)} \]
Dies liefert eine Überlagerung von Grad 2.
\end{Beweis}

\Prop{}
Sei $M$ eine glatte Mannigfaltigkeit. $M$ ist genau dann orientierbar, wenn $\T M$ orientierbar ist.
\begin{Beweis}{}
	Die $\phi_{\alpha \beta}$ sind gerade die Jacobimatrizen der Kartenwechsel.
\end{Beweis}

\Bem{}
Sei $(M,\partial M)$ eine glatte berandete Mannigfaltigkeit. Es gilt
\[ \T M_{|\partial M} = \T(\partial M) \oplus \R^1 \]
Denn auf $\partial M$ können wir einen Schnitt angeben, indem wir einen Vektor im Tangentialraum identifizieren, der vom Rand nach \textbf{innen} rein geht.\\
Ist $M$ orientiert, so ist es auch $\T M$, dadurch auch $\T\partial M$ und ergo auch $\partial M$. Beachte, dass wir immer den nach innen weisenden Randvektor für $\R^1$ instrumentalisieren.

\Def{}
Seien $M, N$ geschlossene orientierte glatte Mannigfaltigkeiten der Dimension $n$.\\
$f : M\pfeil{} N $ sei glatt.\\
Sei ferner $p \in N$ ein regulärer Wert von $f$. Definiere
\[ \deg_p f := \sum_{q \in f\i(p)} \e_q \]
wobei
\[
\e_q :=
\left\lbrace
\begin{aligned}
+1 && \d f_q : T_q M \pfeil{} T_p N \text{ ist orientierungserhaltend}\\
-1 && \d f_q : T_q M \pfeil{} T_p N \text{ ist orientierungsumkehrend}
\end{aligned}
\right.\]

\Bem{}
Wie im nicht orientierten Fall zeigt man, dass $\deg_p f$ unabhängig von $p$ und eine Homotopieinvariante bzgl. $f$ ist.\\
Denn ist $M$ orientiert, $f\simeq g$, so ist $M \times I$ orientiert, da $I$ kanonisch orientiert ist. Ergo ist auch $f\i(p)$ orientiert, da $\partial H\i(p)$ orientiert ist. Alle Punkte in $f\i(p)$ und $g\i(p)$ haben eine Orientierung. Ihre Epsilonwerte subtrahieren sich zu Null. 
\section{Anwendungen des Abbildunggrades}
\marginpar{Vorlesung vom 11.12.17}
Im Folgenden seien $M,N$ geschlossene glatte orientierte Mannigfaltigkeiten der Dimension $n$.

\Lem{}
Sei $f : M \pfeil{} N$ glatt. Ist $\deg f \neq 0$, dann ist $f$ surjektiv.
\begin{Beweis}{}
	Wäre $f$ nicht surjektiv, dann wählen wir $p\in N - f(M)$. Damit ist $p$ regulär und es gilt
	\[ \deg f = \deg_p f = 0 \]
\end{Beweis}

\Satz{Fundamentalsatz der Algebra}
Jedes nichtkonstante komplexe Polynom hat eine Nullstelle.
\begin{Beweis}{}
Sei $f : \C \pfeil{} \C$ ein nichtkonstantes komplexes Polynom. Ohne Einschränkung hat $f$ folgende Gestalt
\[ f(z) = z^n + a_{n-1} z^{n-1}+\ldots + a_1z + a_0 \]
Es gilt notorischerweise
\[ \lim\limits_{z\pfeil{} \infty} f(z) = \infty \]
$f$ induziert somit eine stetige Fortsetzung $\overline{f}$ auf der Ein-Punkt-Kompaktifizierung von $\C$.
\begin{align*}
\overline{f} : \C \cup \{\infty\} & \Pfeil{}\C \cup \{\infty\}\\
z \in \C & \longmapsto f(z)\\
\infty & \longmapsto \infty
\end{align*}
Wir fassen im Folgenden $\C \cup \infty$ als $S^2$ auf. Wir wollen den Abbildungsgrad von $\overline{f}$ bestimmen. Betrachte hierzu folgende Homotopie
\begin{align*}
H : S^2 \times I & \Pfeil{} S^2\\
(z,t) & \longmapsto z^n + a_{n-1}tz^{n-1} + \ldots + a_1tz + a_0t
\end{align*}
hierdurch werden $\overline{f}$ und $\overline{g}$ homotop für $g(z) := z^n$. Ergo haben $\overline{f}$ und $\overline{g}$ den selben Abbildungsgrad. Der Abbildungsgrad von $\overline{g}$ ist gerade $n$. Betrachte zum Beispiel den regulären Wert 1. Dieser hat $n$ $n$-te Einheitswurzeln.\\
Ergo verschwindet der Abbildungsgrad von $\overline{f}$ nicht, ergo ist $\overline{f}$ surjektiv, ergo ist $f$ surjektiv, ergo hat $f$ eine Nullstelle.
\end{Beweis}

\Bem{}
Es liegt folgender Isomorphismus vor
\begin{align*}
\left\lbrace
\begin{aligned}
\text{punktierte Homotopieklassen von}\\
\text{stetigen Abbildungen }f : S^n \pfeil{} S^n
\end{aligned}
\right\rbrace
&\Pfeil{\isom{}}
\Z\\
[f] & \longmapsto \deg f
\end{align*}

\chapter[\textsc{De Rham Kohomologie}]{{\textsc{Glatte Differentialformen und De Rham Kohomologie}}}

\section*{Motivation}
Für jede glatte Funktion $f : U\off \R \pfeil{} \R$ existiert eine glatte \df{Stammfunktion} $F : U \pfeil{} \R$, d.\,h.
\[ \frac{\d}{\d x} F = f \]
Kann das auf Funktionen in mehreren Veränderlichen verallgemeinert werden?\\
Betrachten wir hierzu eine glatte Abbildung
\[ f : U \off \R^2 \Pfeil{} \R^2 \]
\paragraph{Frage} Existiert ein \df{Potential} $F : U \pfeil{} \R$ sodass
\[ f = (\frac{\d}{\d x} F, \frac{\d}{\d y} F) =: (f_1, f_2) \]
Wenn ja, dann gilt auch
\[ \frac{\d f_1}{\d y} = \frac{\d^2 F}{\d x\d y} = \frac{\d^2 F}{\d y\d x} =  \frac{\d f_2}{\d x}  \]
Dadurch erhalten wir folgende notwendige Bedingung
\[ \frac{\d f_1}{\d y} = \frac{\d f_2}{\d x} \]
\paragraph{Frage} Ist diese Bedingung hinreichend?\\
Schauen wir uns dazu folgendes Beispiel an:
\begin{align*}
f : \R^2 - 0 & \Pfeil{} \R^2\\
(x_1, x_2) & \longmapsto \frac{1}{x_1^2 + x_2^2}(-x_2, x_1)
\end{align*}
$f$ erfüllt obige Bedingung. Angenommen es gäbe ein Potential $F : \R^2 - 0 \pfeil{} \R$ für $f$. Betrachte
\[ \int_{0}^{2\pi} \frac{\d}{\d \theta} F(\cos \theta, \sin \theta) \d \theta = F(\cos 2\pi , \sin 2\pi) - F(\cos 0, \sin 0) = 0 \]
Andererseits gilt
\begin{align*}
\frac{\d}{\d \theta} F(\cos \theta, \sin \theta) &= - \sin \theta\frac{\partial F}{\partial x_1}(\cos \theta, \sin \theta) + 
\cos \theta\frac{\partial F}{\partial x_2}(\cos \theta, \sin \theta)\\
& = \frac{\sin^2 \theta}{\cos^2 \theta + \sin^2 \theta} +  \frac{\cos^2 \theta}{\cos^2 \theta + \sin^2 \theta} =1
\end{align*}
woraus folgen würde
\[\int_{0}^{2\pi} \frac{\d}{\d \theta} F(\cos \theta, \sin \theta) \d \theta = \int_{0}^{2\pi} 1 \d \theta = 2\pi \]
Dies ist ein Widerspruch, ergo ist obige Bedingung nicht hinreichend.

\Def{}
Eine Menge $U \subset \R^n$ heißt \df{sternförmig} bzgl. $x_0\in U$, wenn für alle $x \in U$ die Strecke
\[\set{tx + (1-t)x_0}{t\in [0,1]} \]
in $U$ enthalten ist.

\Prop{}
Ist $U \subset \R^2$ sternförmig und erfüllt die glatte Funktion $f : U \pfeil{} \R^2$ die Bedingung
\[ \frac{\d f_1}{\d x_2} = \frac{\d f_2}{\d x_1} \]
dann hat $f$ ein Potential auf $U$.
\begin{Beweis}{}
Ohne Einschränkung ist $x_0 = 0$ das Zentrum von $U$. Dann setzen wir
\[ F(x_1, x_2) = \int^{1}_0 x_1f_1(tx_1, tx_2) + x_2 f_2(tx_1, tx_2) \d t \]
\end{Beweis}

\Bem{}
Die Existenz eines Potentials hängt also irgendwie von der Topologie der Definitionsmenge ab.

\subsection*{Umformulierung}
Sei $U \subset \R^2$. Wir definieren den \df{Gradient} durch
\begin{align*}
\nabla : C^\infty(U,\R) & \Pfeil{} C^\infty (U, \R^2)\\
f & \longmapsto (\frac{\d f}{\d x_1}, \frac{\d f}{\d x_2})
\end{align*}
Die \df{Rotation} definieren wir durch
\begin{align*}
\rot : C^\infty(U,\R^2) & \Pfeil{} C^\infty (U, \R)\\
f_1, f_2 & \longmapsto \frac{\d f_1}{\d x_2} - \frac{\d f_2}{\d x_1}
\end{align*}
Es gilt dann 
\[ \rot \circ \nabla = 0 \]
D.\,h., $\Img \nabla \subset \Ker~ \rot$. Wir definieren die \df{erste Kohomologiegruppe} von $U$ durch
\[ H^1(U) := \Ker~ \rot / \Img \nabla \]

\Bsp{}
Wir wissen bereits
\[ H^1(\text{sternförmig}) = 0 \]
und
\[ H^1(\R^2 - 0) \neq 0 \]

\section{Äußere Algebren}
Sei $V$ ein endlich dimensionaler reeller Vektorraum. Es bezeichne $V^k$ das $k$-fache kartesische Produkt von $V$ mit Vektorraumstruktur.
\Def{}
Eine $k$-lineare Abbildung $\omega : V^k \pfeil{} \R$ heißt \df{alternierend}, wenn
\[ \omega(v_1, \ldots, v_k) = 0 \]
für alle $v_1,\ldots, v_k$, in denen ein Vektor $v_i$ mindestens an zwei Stellen vorkommt. Das ist äquivalent dazu zu fordern, dass $\omega$ für alle linear abhängige System $v_1,\ldots, v_k$ verschwindet.

\Bem{}
Für ein alternierendes $\omega$ gilt
\[ \omega(v_1, \ldots, v_i, \ldots, v_j, \ldots, v_k) = - \omega(v_1,\ldots, v_j, \ldots, v_i, \ldots, v_k) \]

\Def{}
Unter $Alt^k(V) \subset \Hom{\R}{V^k}{\R}$ verstehen wir den reellen Vektorraum der alternierenden Formen. Wir legen ferner folgende Konvention fest
\[ Alt^0(V) := \R \]

\Bsp{}
Ist $k = \dim V$, so ist $Alt^k(V)$ eindimensional und wird von der Determinante erzeugt.

\Lem{}
\begin{enumerate}[1.)]
	\item $Alt^k(V) = 0$ für $k > \dim V$
	\item $\omega(v_1, \ldots, v_n) = \text{sgn}(\sigma) \cdot \omega(v_{\sigma(1)}, \ldots, v_{\sigma(n)})$\\
	für eine Permutation $\sigma \in S_k = \text{Bij}(\{1,\ldots, k\}, \{1,\ldots,k\})$
\end{enumerate}

\section{Äußeres Produkt}
Wir wollen ein Produkt auf dem System der $Alt^k(V)$ konstruieren.
\[ \wedge : Alt^p(V) \times Alt^q(V) \Pfeil{} Alt^{p+q} (V) \]

Für $p = q = 1$ legen wir fest
\[ (\omega_1 \wedge \omega_2) (v_1, v_2) = \omega_1(v_1) \omega_2(v_2) - \omega_1(v_2) \omega_2(v_1) \]

\Def{}
Eine Permutation $\sigma \in S_{p+q}$ heißt $(p,q)$-\df{Shuffle}, wenn gilt
\begin{align*}
& \sigma(1) < \sigma(2) < \ldots < \sigma(p)\\
\text{ und }& \sigma(p+1) < \sigma(p+2) < \ldots \sigma(p+q)
\end{align*}
Ein $(p,q)$-Shuffle ist eindeutig durch sein Verhalten auf $\{1,\ldots, p\}$ festgelegt. Daraus folgt
\[ \# S_{p,q}  = \binom{p+q}{p} \]
wobei $S_{p,q} \subset S_{p+q}$ die Menge aller $(p,q)$-Shuffles bezeichnet.

\Def{}
Seien $p,q$ beliebig, $\omega_1 \in Alt^p(V), \omega_2 \in Alt^q(V)$. Wir definieren das \df{Wedge-Produkt} der beiden Funktionale durch
\[ (\omega_1 \wedge \omega_2) (v_1,\ldots, v_{p+q}) := \sum_{\sigma \in S_{p,q}}\text{sgn}(\sigma) \omega_1(v_{\sigma(1)}, \ldots, v_{\sigma(p)}) \omega_2( v_{\sigma(p+1)}, \ldots, v_{\sigma(p+q)} ) \]
Dann erhalten wir eine bilineare Abbildung
\[ \wedge : Alt^p(V) \otimes Alt^q(V)  \Pfeil{} Alt^{p+q}(V) \]
Es gilt ferner
\[ (\omega_1 \wedge \omega_2) (v_1,\ldots, v_{p+q}) = \frac{1}{p!q!} \sum_{\sigma \in S_{p+q}}\text{sgn}(\sigma) \omega_1(v_{\sigma(1)}, \ldots, v_{\sigma(p)}) \omega_2( v_{\sigma(p+1)}, \ldots, v_{\sigma(p+q)} ) \]


%\section{Anwendungen des Abbildunggrades}
%Im Folgenden seien $M,N$ geschlossene glatte orientierte Mannigfaltigkeiten der Dimension $n$.
%
%\Lem{}
%Sei $f : M \pfeil{} N$ glatt. Ist $\deg f \neq 0$, dann ist $f$ surjektiv.
%\begin{Beweis}{}
%	Wäre $f$ nicht surjektiv, dann wählen wir $p\in N - f(M)$. Damit ist $p$ regulär und es gilt
%	\[ \deg f = \deg_p f = 0 \]
%\end{Beweis}
%
%\Satz{Fundamentalsatz der Algebra}
%Jedes nichtkonstante komplexe Polynom hat eine Nullstelle.
%\begin{Beweis}{}
%Sei $f : \C \pfeil{} \C$ ein nichtkonstantes komplexes Polynom. Ohne Einschränkung hat $f$ folgende Gestalt
%\[ f(z) = z^n + a_{n-1} z^{n-1}+\ldots + a_1z + a_0 \]
%Es gilt notorischerweise
%\[ \lim\limits_{z\pfeil{} \infty} f(z) = \infty \]
%$f$ induziert somit eine stetige Fortsetzung $\overline{f}$ auf der Ein-Punkt-Kompaktifizierung von $\C$.
%\begin{align*}
%\overline{f} : \C \cup \{\infty\} & \Pfeil{}\C \cup \{\infty\}\\
%z \in \C & \longmapsto f(z)\\
%\infty & \longmapsto \infty
%\end{align*}
%Wir fassen im Folgenden $\C \cup \infty$ als $S^2$ auf. Wir wollen den Abbildungsgrad von $\overline{f}$ bestimmen. Betrachte hierzu folgende Homotopie
%\begin{align*}
%H : S^2 \times I & \Pfeil{} S^2\\
%(z,t) & \longmapsto z^n + a_{n-1}tz^{n-1} + \ldots + a_1tz + a_0t
%\end{align*}
%hierdurch werden $\overline{f}$ und $\overline{g}$ homotop für $g(z) := z^n$. Ergo haben $\overline{f}$ und $\overline{g}$ den selben Abbildungsgraden. Der Abbildungsgrad von $\overline{g}$ ist gerade $n$. Betrachte zum Beispiel den regulären Wert 1. Dieser hat $n$ $n$-te Einheitswurzeln.\\
%Ergo verschwindet der Abbildungsgrad von $\overline{f}$ nicht, ergo ist $\overline{f}$ surjektiv, ergo ist $f$ surjektiv, ergo hat $f$ eine Nullstelle.
%\end{Beweis}
%
%\Bem{}
%Es liegt folgender Isomorphismus vor
%\begin{align*}
%\left\lbrace
%\begin{aligned}
%\text{punktierte Homotopieklassen von}\\
%\text{stetigen Abbildungen }f : S^n \pfeil{} S^n
%\end{aligned}
%\right\rbrace
%&\Pfeil{\isom{}}
%\Z\\
%[f] & \longmapsto \deg f
%\end{align*}
%
%\chapter{{\textsc{Glatte Differentialformen und De Rham Kohomologie}}}
%
%\section*{Motivation}
%Für jede glatte Funktion $f : U\off \R \pfeil{} \R$ existiert eine glatte \df{Stammfunktion} $F : U \pfeil{} \R$, d.\,h.
%\[ \frac{\d}{\d x} F = f \]
%Kann das auf Funktionen in mehreren Veränderlichen verallgemeinert werden?\\
%Betrachten wir hierzu eine glatte Abbildung
%\[ f : U \off \R^2 \Pfeil{} \R^2 \]
%\paragraph{Frage} Existiert ein \df{Potential} $F : U \pfeil{} \R$ sodass
%\[ f = (\frac{\d}{\d x} F, \frac{\d}{\d y} F) =: (f_1, f_2) \]
%Wenn ja, dann gilt auch
%\[ \frac{\d f_1}{\d y} = \frac{\d^2 F}{\d x\d y} = \frac{\d^2 F}{\d y\d x} =  \frac{\d f_2}{\d x}  \]
%Dadurch erhalten wir folgende notwendige Bedingung
%\[ \frac{\d f_1}{\d y} = \frac{\d f_2}{\d x} \]
%\paragraph{Frage} Ist diese Bedingung hinreichend?\\
%Schauen wir uns dazu folgendes Beispiel an:
%\begin{align*}
%f : \R^2 - 0 & \Pfeil{} \R^2\\
%(x_1, x_2) & \longmapsto \frac{1}{x_1^2 + x_2^2}(-x_2, x_1)
%\end{align*}
%$f$ erfüllt obige Bedingung. Angenommen es gäbe ein Potential $F : \R^2 - 0 \pfeil{} \R$ für $f$. Betrachte
%\[ \int_{0}^{2\pi} \frac{\d}{\d \theta} F(\cos \theta, \sin \theta) \d \theta = F(\cos 2\pi , \sin 2\pi) - F(\cos 0, \sin 0) = 0 \]
%Andererseits gilt
%\[ \frac{\d}{\d \theta} F(\cos \theta, \sin \theta) = \frac{\partial F}{\partial x_1}(- \sin \theta) + 
%\frac{\partial F}{\partial x_2}(\cos \theta) = \frac{\sin^2 \theta}{\cos^2 \theta + \sin^2 \theta} +  \frac{\cos^2 \theta}{\cos^2 \theta + \sin^2 \theta} =1 \]
%woraus folgenden würde
%\[\int_{0}^{2\pi} \frac{\d}{\d \theta} F(\cos \theta, \sin \theta) \d \theta = \int_{0}^{2\pi} 1 \d \theta = 2\pi \]
%Dies ist ein Widerspruch, ergo ist obige Bedingung nicht hinreichend.
%
%\Def{}
%Eine Menge $U \subset \R^n$ heißt \df{sternförmig} bzgl. $x_0\in U$, wenn für alle $x \in U$ die Strecke
%\[\set{tx + (1-t)x_0}{t\in [0,1]} \]
%in $U$ enthalten ist.
%
%\Prop{}
%Ist $U \subset \R^2$ sternförmig und erfüllt die glatte Funktion $f : U \pfeil{} \R^2$ die Bedingung
%\[ \frac{\d f_1}{\d x_2} = \frac{\d f_2}{\d x_1} \]
%dann hat $f$ ein Potential auf $U$.
%\begin{Beweis}{}
%Ohne Einschränkung ist $x_0 = 0$. Dann setzen wir
%\[ F(x_1, x_2) = \int^{1}_0 x_1f_1(tx_1, tx_2) + x_2 f_2(tx_1, tx_2) \d t \]
%\end{Beweis}
%
%\Bem{}
%Die Existenz eines Potentials hängt also irgendwie von der Topologie der Definitionsmenge ab.
%
%\subsection*{Umformulierung}
%Sei $U \subset \R^2$. Wir definieren den \df{Gradient} durch
%\begin{align*}
%\nabla : C^\infty(U,\R) & \Pfeil{} C^\infty (U, \R^2)\\
%f & \longmapsto (\frac{\d f}{\d x_1}, \frac{\d f}{\d x_2})
%\end{align*}
%Die \df{Rotation} definieren wir durch
%\begin{align*}
%\rot : C^\infty(U,\R^2) & \Pfeil{} C^\infty (U, \R)\\
%f_1, f_2 & \longmapsto \frac{\d f_1}{\d x_2} - \frac{\d f_2}{\d x_1}
%\end{align*}
%Es gilt dann 
%\[ \rot \circ \nabla = 0 \]
%D.\,h., $\Img \nabla \subset \Ker~ \rot$. Wir definieren die \df{erste Kohomologiegruppe} von $U$ durch
%\[ H^1(U) := \Ker~ \rot / \Img \nabla \]
%
%\Bsp{}
%Wir wissen bereits
%\[ H^1(\text{sternförmig}) = 0 \]
%und
%\[ H^1(\R^2 - 0) \neq 0 \]
%
%\section{Äußere Algebren}
%Sei $V$ ein endlich dimensionaler reeller Vektorraum. Es bezeichne $V^k$ das $k$-fache kartesische Produkt von $V$ mit Vektorraumstruktur.
%\Def{}
%Eine $k$-lineare Abbildung $\omega : V^k \pfeil{} \R$ heißt \df{alternierend}, wenn
%\[ \omega(v_1, \ldots, v_k) = 0 \]
%für alle $v_1,\ldots, v_k$, in denen ein Vektor $v_i$ mindestens an zwei Stellen vorkommt. Das ist äquivalent dazu zu fordern, dass $\omega$ für alle linear abhängige System $v_1,\ldots, v_k$ verschwindet.
%
%\Bem{}
%Für ein alternierendes $\omega$ gilt
%\[ \omega(v_1, \ldots, v_i, \ldots, v_j, \ldots, v_k) = - \omega(v_1,\ldots, v_j, \ldots, v_i, \ldots, v_k) \]
%
%\Def{}
%Unter $Alt^k(V) \subset \Hom{\R}{V^k}{\R}$ verstehen wir den reellen Vektorraum der alternierenden Formen. Wir legen ferner folgende Konvention fest
%\[ Alt^0(V) := \R \]
%
%\Bsp{}
%Ist $k = \dim V$, so ist $Alt^k(V)$ eindimensional und wird von der Determinante erzeugt.
%
%\Lem{}
%\begin{enumerate}[1.)]
%	\item $Alt^k(V) = 0$ für $k > \dim V$
%	\item $\omega(v_1, \ldots, v_n) = \text{sgn}(\sigma) \cdot \omega(v_{\sigma(1)}, \ldots, v_{\sigma(n)})$ für eine Permutation $\sigma \in S_k = \text{Bij}(\{1,\ldots, k\}, \{1,\ldots,k\})$
%\end{enumerate}
%
%\section{Äußeres Produkt}
%Wir wollen ein Produkt auf dem System der $Alt^k(V)$ konstruieren.
%\[ \wedge : Alt^p(V) \times Alt^q(V) \Pfeil{} Alt^{p+q} (V) \]
%
%Für $p = q = 1$ legen wir fest
%\[ (\omega_1 \wedge \omega_2) (v_1, v_2) = \omega_1(v_1) \omega_2(v_2) - \omega_1(v_2) \omega_2(v_1) \]
%
%\Def{}
%Eine Permutation $\sigma \in S_{p+q}$ heißt $(p,q)$-\df{Shuffle}, wenn gilt
%\begin{align*}
%& \sigma(1) < \sigma(2) < \ldots < \sigma(p)\\
%\text{ und }& \sigma(p+1) < \sigma(p+2) < \ldots \sigma(p+2)
%\end{align*}
%Ein $(p,q)$-Shuffle ist eindeutig durch sein Verhalten auf $\{1,\ldots, p\}$ festgelegt. Daraus folgt
%\[ \# S_{p,q}  = \binom{p+q}{p} \]
%wobei $S_{p,q} \subset S_{p+q}$ die Menge aller $(p,q)$-Shuffles bezeichnet.
%
%\Def{}
%Seien $p,q$ beliebig, $\omega_1 \in Alt^p(V), \omega_2 \in Alt^q(V)$. Wir definieren das \df{Wedge-Produkt} der beiden Funktionale durch
%\[ (\omega_1 \wedge \omega_2) (v_1,\ldots, v_{p+q}) := \sum_{\sigma \in S_{p,q}}\text{sgn}(\sigma) \omega_1(v_{\sigma(1)}, \ldots, v_{\sigma(p)}) \omega_2( v_{\sigma(p+1)}, \ldots, v_{\sigma(p+q)} ) \]
%Dann erhalten wir eine bilineare Abbildung
%\[ \wedge : Alt^p(V) \otimes Alt^q(V)  \Pfeil{} Alt^{p+q}(V) \]
%Es gilt ferner
%\[ (\omega_1 \wedge \omega_2) (v_1,\ldots, v_{p+q}) = \frac{1}{p!q!} \sum_{\sigma \in S_{p+q}}\text{sgn}(\sigma) \omega_1(v_{\sigma(1)}, \ldots, v_{\sigma(p)}) \omega_2( v_{\sigma(p+1)}, \ldots, v_{\sigma(p+q)} ) \]
\Lem{}
\marginpar{Vorlesung vom 15.12.17}
Das Wedge-Produkt ist assoziativ, bilinear und ein graduiert kommutatives Produkt, d.\,h.
\begin{enumerate}[i.]
		\item $(\omega_1 \wedge \omega_2) \wedge \omega_3 = \omega_1 \wedge (\omega_2 \wedge \omega_3)$
	\item $(\omega_1 + \omega_2) \wedge \omega_3 = \omega_1 \wedge \omega_3 + \omega_2 \wedge \omega_3$
	\item $(\lambda \omega_1)\wedge \omega_2 = \lambda (\omega_1 \wedge \omega_2) = \omega_1 \wedge (\lambda \omega_2)$
	\item $\omega_1 \wedge \omega_2 = (-1)^{pq} (\omega_2 \wedge \omega_1)$
\end{enumerate}
für $\omega_1 \in Alt^p(V), \omega_2 \in Alt^q(V), \omega_3 \in Alt^t(V), \lambda \in \R$.

\Def{}
Setze
\[ Alt^*(V) = \bigcup_{p\geq 0} Alt^p(V) \]
$(Alt^*(V), +,\wedge)$ bildet eine graduierte $\R$-Algebra, die sogenannte \df{Äußere Algebra} von $V$. Sie ist graduiert kommutativ.

\Lem{}
Für 1-Formen $\omega_1, \ldots, \omega_p \in Alt^1(V)$ gilt
\[ (\omega_1 \wedge \ldots \wedge \omega_p)(v_1,\ldots, v_p) = \det 
\left(
\begin{matrix}
\omega_1(v_1) & \cdots & \omega_1(v_p)\\
\vdots & & \vdots\\
\omega_p(v_1) & \cdots & \omega_p(v_p)
\end{matrix}
\right)\]
\begin{Beweis}{}
Wir beweisen dies durch Induktion nach $p$.\\
Es sei $p = 2$. Dann gilt
\[ (\omega_1 \wedge \omega_2)(v_1, v_2) = \omega_1(v_1) \omega_2(v_2) - \omega_2 (v_1) \omega_1(v_2) = 
\det 
\left(
\begin{matrix}
\omega_1(v_1) & \omega_1(v_2)\\
\omega_2(v_1) &  \omega_2(v_2)
\end{matrix}
\right)
 \]
Im Induktionsschritt rechnen wir nun
\begin{align*}
&\omega_1 \wedge ( \omega_2 \wedge \ldots \wedge \omega_p) (v_1, \ldots, v_p)
= \sum_{j} (-1)^{j+1} \omega_1(v_j) (\omega_2 \wedge \ldots \wedge \omega_p) (v_1, \ldots, \widehat{v_j}, \ldots, v_p)
\end{align*}
die Aussage ergibt sich nun, indem man
\[\left(
\begin{matrix}
\omega_1(v_1) & \cdots & \omega_1(v_p)\\
\vdots & & \vdots\\
\omega_p(v_1) & \cdots & \omega_p(v_p)
\end{matrix}
\right)\]
nach der ersten Zeile entwickelt.
\end{Beweis}

\Lem{}
Sei $\{e_1, \ldots, e_n\}$ eine Basis von $V$ und $\{\epsilon_1, \ldots, \epsilon_n\}$ die dazu duale Basis von $Alt^1(V) = V^*$. Dann ist
\[ \left\lbrace\epsilon_{i_1} \wedge \epsilon_{i_2} \wedge \ldots \wedge \epsilon_{i_p} ~|~ 1 \leq i_1 < i_2 < \ldots < i_p \leq n \right\rbrace\]
eine Basis für $Alt^p(V)$. Insbesondere gilt
\[ Alt_p(V) = \binom{n}{p} \]
\begin{Beweis}{}
Es gilt nach Lemma 1 für $1 \leq j_1< \ldots < j_p \leq n$
\begin{align*}
(\epsilon_{i_1} \wedge \epsilon_{i_2} \wedge \ldots \wedge \epsilon_{i_p})(e_{j_1},\ldots, e_{j_p})
=&
\det \left(
\begin{matrix}
\epsilon_{i_1}(e_{j_1}) & \cdots & \epsilon_{i_1}(e_{j_p})\\
\vdots & & \vdots\\
\epsilon_{i_p}(e_{j_1}) & \cdots & \epsilon_{i_p}(e_{j_p})
\end{matrix}
\right)\\
=
\det \left(
\begin{matrix}
\delta_{i_1,j_1} & \cdots & \delta_{i_1,j_p}\\
\vdots & & \vdots\\
\delta_{i_p,j_1} & \cdots & \delta_{i_p,j_p}
\end{matrix}
\right)
=& \left\lbrace
\begin{aligned}
\text{sign}(\sigma) && \{i_1, \ldots, i_p\} = \{j_1, \ldots, j_p \} \text{ und } \exists \sigma \in S_p : \sigma(i_k) = j_k\\
0 && \text{ sonst}
\end{aligned}
\right.
\end{align*}
Insbesondere gilt für eine $p$-Form $\omega \in Alt^p(V)$
\[ \omega(e_{j_1}, \ldots, e_{j_p}) = \sum_{i_1<\ldots < i_p} \omega(e_{i_1},\ldots, e_{i_p}) \cdot (\epsilon_{i_1} \wedge \ldots \wedge \epsilon_{i_p})(e_{j_1}, \ldots, e_{j_p}) \]
Definiert man $c_{i_1, \ldots, i_p} = \omega(e_{i_1},\ldots, e_{i_p})$, so folgt mit der Linearität von $\omega$
\[ \omega = \sum_{i_1 < \ldots <i_p}c_{i_1, \ldots, i_p} (\epsilon_{i_1} \wedge \ldots \wedge \epsilon_{i_p})  \]
Ergo wird $Alt^p(V)$ linear von den Produkten $\epsilon_{i_1} \wedge \ldots \wedge \epsilon_{i_p}$ erzeugt. Diese Produkte sind linear unabhängig, denn wenn
\[\sum_{i_1 < \ldots <i_p}b_{i_1, \ldots, i_p} (\epsilon_{i_1} \wedge \ldots \wedge \epsilon_{i_p})= 0\]
für Koeffizienten $b_{i_1, \ldots, i_p} \in \R$, dann gilt
\[ b_{j_1, \ldots, j_p} = \sum_{i_1 < \ldots <i_p}b_{i_1, \ldots, i_p} (\epsilon_{i_1} \wedge \ldots \wedge \epsilon_{i_p})(e_{j_1}, \ldots, e_{j_p}) = 0 \]
für alle $1 \leq j_1<\ldots j_p\leq n$.
\end{Beweis}

\Bsp{}
Für $p = n$ ist $Alt^p(V)$ eindimensional und erzeugt durch $\epsilon_1 \wedge \ldots \wedge \epsilon_n$.

\section{Glatte Differentialformen auf Offenen Mengen im $\R^n$}
Sei $U \subseteq \R^n$ offen.
\Def{}
Eine glatte \df{Differentialform} vom Grad $p$ auf $U$ ist eine glatte Abbildung
\[ \omega : U \pfeil{} Alt^p(\R^n) \isom{} \R^{\binom{n}{p}} \]
Es gilt
\[ \omega = \sum_{i_1 < \ldots <i_p} f_{i_1,\ldots, i_p}(x) (\epsilon_{i_1} \wedge \ldots \wedge \epsilon_{i_n}) \]
für glatte Funktionen $f_{i_1,\ldots,i_p} \in C^\infty (U,\R)$.\\
Es bezeichne $\Omega^p(U)$ den $\R$-Vektorraum aller glatten Differentialformen vom Grad $p$ auf $U$.

\paragraph{Schreibweise} Schreibt man für $1\leq i_1< \ldots i_p\leq n$
\[ I = (i_1, \ldots, i_p)  \]
so schreibe man weiterhin
\[ f_I = f_{i_1,\ldots, i_p} \]
und
\[ \epsilon_I = \epsilon_{i_1} \wedge \ldots \wedge \epsilon_{i_n} \]
und
\[ \omega = \sum_{I}f_I \epsilon_I \]

\Def{}
Wir definieren \df{Richtungsableitungen} bei $x \in U$
\begin{align*}
\D_x\omega : \R^n & \Pfeil{} Alt^p(\R^n)\\
e_i & \longmapsto  \frac{\d \omega}{\d x_i}(x) := \frac{\d}{\d t}_{t = 0} \omega(x + te_i)
\end{align*}
Für $\omega = \sum_{I}f_I \epsilon_I$ ist
\[ (\D_x \omega)(e_i) = \sum_{I} \frac{\partial f_I}{\partial x_i}(x) \epsilon_I \]

\Def{}
Mithilfe von $\D_x$ definieren wir die \df{äußere Ableitung}
\begin{align*}
\d : \Omega^p(U) & \Pfeil{} \Omega^{p+1}(U)\\
\omega & \longmapsto \d \omega
\end{align*}
mit
\[ (\d \omega)(x)(v_1, \ldots, v_{p+1}) := \sum_{j=1}^{p+1} (-1)^{j+1} (\D_x \omega) (v_j) (v_1, \ldots, \widehat{v_j}, \ldots, v_{p+1}) \]

\Bsp{}
$\d : \Omega^0(U) = C^\infty(U,\R) \Pfeil{} \Omega^1 (U)$ mit
\[ \d f = \sum_{i = 1}^{n} \frac{\partial f}{\partial x_i} \epsilon_i \]
Insbesondere gilt für $f = x_j$
\[ \d x_j = \epsilon_j \]
Insofern werden wir in Zukunft $\d x_i$ statt $\epsilon_i$ schreiben und ferner
\[ \epsilon_I = \epsilon_{i_1} \wedge\ldots \wedge \epsilon_{i_p} = \d x_{i_1} \wedge\ldots \wedge \d x_{i_p} =: \d x_I \]

\Lem{}
Für $\omega = f \epsilon_I$ gilt
\[ \d \omega = (\d f) \wedge \epsilon_I \]
\begin{Beweis}{}
\[ (\d \omega)(x)(v_1, \ldots, v_{p+1}) = \sum_{j} (-1)^{j+1} (\D_x \omega) (v_j) (v_1, \ldots, \widehat{v_j}, \ldots, v_p) \]
es gilt dabei mit $v = \sum_{i= 1}^{n} \epsilon_i(v) e_i$
\[ (\D_x \omega)(v) = \sum_{i= 1}^{n} \epsilon_i(v) \D_x(f\epsilon_I)(e_i) \]
Ferner gilt
\[ \sum_{i= 1}^{n} \epsilon_i(v)\frac{\partial f}{\partial x_i}(x) \epsilon_I = \d f(x)(v) \epsilon_I \]
Daraus folgt
\begin{align*}
 (\d \omega) (x) (v_1, \ldots, v_{p+1}) &= \sum_{j} (-1)^{j+1} \d f(x) (v_j)  \epsilon_I(v_1, \ldots, \widehat{v_j}, \ldots, v_{p+1}) \\
 &= ( (\d f) (x) \wedge \epsilon_I) (v_1, \ldots, v_{p+1}) 
\end{align*}
\end{Beweis}

\Lem{}
Die Zusammensetzung
\[ \Omega^p(U) \Pfeil{\d} \Omega^{p+1}(U) \Pfeil{ \d } \Omega^{p+2}(U) \]
verschwindet.
\begin{Beweis}{}
Es genügt dies für Formen der Gestalt $\omega = f\epsilon_I$ zu zeigen. Es gilt laut vorhergehendem Lemma
\[ \d \omega = \d f \wedge \epsilon_I = \sum_{i = 1}^n \frac{\partial f}{\partial x_i} \epsilon_i \wedge \epsilon_I \]
und
\[ \d (\d \omega) = \sum_{i,j} \frac{\partial^2 f}{\partial x_i \partial x_j} \epsilon_i \wedge \epsilon_j \wedge \epsilon_I
= \sum_{i<j} ( \frac{\partial^2 f}{\partial x_i \partial x_j} - \frac{\partial^2 f}{\partial x_j \partial x_i}) \epsilon_j\wedge \epsilon_i \wedge \epsilon_I
 \]
 Das verschwindet aber, da
 \[\frac{\partial^2 f}{\partial x_i \partial x_j} = \frac{\partial^2 f}{\partial x_j \partial x_i}\]
\end{Beweis}
\Def{}
\marginpar{Vorlesung vom 18.12.17}
Wir definieren ein äußeres Produkt auf der Menge der glatten Differentialformen durch
\begin{align*}
 \wedge : \Omega^p(U) \times \Omega^q(U) & \Pfeil{{}} \Omega^{p+q}(U)\\
 (\omega,\eta) & \longmapsto [x \in U \mapsto \omega(x)\wedge \eta(x)] 
\end{align*}
Dies ist zulässig, da $\omega\wedge \eta$ tatsächlich eine glatte Abbildung $U \pfeil{} Alt^{p+q}(\R^n)$ ist.\\
Es gilt dann
\[ (f\cdot \omega)\wedge \eta = f(\omega \wedge \eta) = \omega \wedge (f\eta) \]
und
\[ f\wedge \omega = f \cdot\omega\]

\Lem{}
Seien glatte Differentialformen $\omega\in \Omega^p(U), \eta \in \Omega^q(U)$. Dann gilt folgende \df{Produktregel} für Differentialformen
\[ \d(\omega \wedge \eta) = (\d \omega) \wedge \eta +  (-1)^{p} \omega \wedge (\d \eta) \]
\begin{Beweis}{}
Es genügt dies auf Ebene von erzeugenden Formen der Gestalt
\[ \omega = f\epsilon_I \text{ und } \eta = g\epsilon_J \]
zu zeigen für $I = (i_1,\ldots, i_p), J = (j_1, \ldots, j_q)$. Es gilt
\[ \omega \wedge \eta = (fg)\epsilon_I \epsilon_J \]
und deswegen
\begin{align*}
\d (\omega \wedge \eta) &= \d (fg) \wedge \epsilon_I \wedge \epsilon_J\\
&= \sum_{i} \frac{\partial (fg)}{\partial x_i} \epsilon_i \wedge \epsilon_I \wedge \epsilon_J\\
&= \sum_{i} \klam{\frac{\partial f}{\partial x_i} g + f \frac{\partial g}{\partial x_i}} \epsilon_i \wedge \epsilon_I \wedge \epsilon_J\\
&= \sum_{i} \frac{\partial f}{\partial x_i} g  \epsilon_i \wedge \epsilon_I \wedge \epsilon_J
+ \sum_{i} f\frac{\partial g}{\partial x_i}  \epsilon_i \wedge \epsilon_I \wedge \epsilon_J\\
&=g \cdot (\d f) \wedge \epsilon_I \wedge \epsilon_J +  f \cdot (\d g) \wedge \epsilon_I \wedge \epsilon_J\\
&=(\d f) \wedge \epsilon_I \wedge (g \cdot \epsilon_J) +  (-1)^p(f \cdot \epsilon_I) \wedge (\d g \wedge\epsilon_J)\\
&= \d \omega  \wedge \eta + (-1)^p\omega \wedge (\d \eta)
\end{align*}
\end{Beweis}

\Def{}
Wir definieren die Algebra der glatten Differentialformen durch
\[ (\Omega^*(U) = \bigoplus_{p\geq 0} \Omega^p(U), +, \wedge, \d) \]
$(\Omega^*(U) = \bigoplus_{p\geq 0} \Omega^p(U), +, \wedge)$ ist eine graduiert-kommutative graduierte $C^\infty(U,\R)$-Algebra. $(\Omega^*(U),\d)$ ist ferner ein Komplex von $\R$-Vektorräumen, d.\,h. $\d^2 = 0$.\\
Zwischen diesen beiden Strukturen existiert eine Interaktion\footnote{Im Englischen nennt man eine solche Struktur\textsl{ \textbf{D}ifferential \textbf{g}raded \textbf{a}lgebra}.}, nämlich ist $\d$ eine Derivation auf der Algebra, d.\,h., es gilt
\[ \d(\omega\wedge \eta) = (\d \omega) \wedge \eta + (-1)^p \omega \wedge (\d \eta) \]

\Prop{Eindeutigkeit von $\d$}
Es existiert genau eine Familie linearer Abbildungen
\[ d : \Omega^p(U) \Pfeil{} \Omega^{p+1}(U) \]
sodass gilt:
\begin{enumerate}[(i)]
	\item $d f= \sum_{i} \frac{\partial f}{\partial x_i} \d x_i$ für $f \in \Omega^0(U)$
	\item $d^2 = 0$
	\item $d(\omega\wedge \eta) = (d \omega) \wedge \eta + (-1)^p\omega \wedge (d\eta)$ für $\omega \in \Omega^p(U), \eta \in \Omega^q(u)$
\end{enumerate}

\Bsp{Klassische Integralsätze von Green, Stokes und Gauß}
Betrachte $U \off \R^2$. Alle nicht verschwindenden Gruppen von Differentialformen sind $\Omega^0(U),\Omega^1(U),\Omega^2(U)$. Ab $p\geq 3$ verschwinden die Gruppen, da ab da die alternierenden Räume verschwinden. Es gilt für $f \in \Omega^0(U) = C^\infty(U,\R)$
\[ \d f = \frac{\partial f}{\partial x_1} \d x_1 +  \frac{\partial f}{\partial x_2} \d x_2 = \nabla f \cdot 
\left(
\begin{matrix}
\d x_1\\
\d x_2
\end{matrix}
\right) \]
Für $\omega = f_1 \d x_1 + f_2 \d x_2 \in \Omega^1(U)$ gilt
\begin{align*}
\d \omega &= \d f_1 \wedge \d x_1 + \d f_2 \wedge \d x_2\\
&= (\frac{\partial f_1}{\partial x_1} \d x_1 + \frac{\partial f_1}{\partial x_2} \d x_2)\wedge \d x_1
+(\frac{\partial f_2}{\partial x_1} \d x_1 + \frac{\partial f_2}{\partial x_2} \d x_2)\wedge \d x_2\\
&= (\frac{\partial f_2}{\partial x_1} - \frac{\partial f_1}{x_2}) \d x_1 \wedge \d x_2\\
&= \text{rot}(f_1, f_2) \d x_1 \wedge \d x_2
\end{align*}
Sei nun $U \off \R^3$. Wir betrachten $\d : \Omega^1(U) \pfeil{} \Omega^2(U)$ und $\omega = f_1 \d x_1 + f_2 \d x_2 + f_3 \d x_3$. Es gilt
\begin{align*}
\d \omega &= \klam{\frac{\partial f_2}{\partial x_1} -\frac{\partial f_1}{\partial x_2}}  \d x_1 \wedge \d x_2
+\klam{\frac{\partial f_3}{\partial x_2} -\frac{\partial f_2}{\partial x_3}}  \d x_2 \wedge \d x_3
+\klam{\frac{\partial f_1}{\partial x_3} -\frac{\partial f_3}{\partial x_1}}  \d x_3 \wedge \d x_1\\
&= \text{rot}(f_1, f_2, f_3) \cdot
\klam{\begin{matrix}
	\d x_1 \wedge \d x_2\\
	\d x_2 \wedge \d x_3\\
	\d x_3 \wedge \d x_1
	\end{matrix}}
\end{align*}
Betrachte nun $\omega = g_3 \d x_1 \wedge \d x_2+ g_1 \d x_2 \wedge \d x_3 +  g_2 \d x_3 \wedge \d x_1\in \Omega^2(U)$. Es gilt
\[ \d \omega = \klam{ \frac{\partial g_1}{\partial x_1} + \frac{\partial g_2}{\partial x_2 } + \frac{\partial g_3}{\partial x_3} } \d x_1 \wedge \d x_2 \wedge \d x_3 = \text{div}(g)  \d x_1 \wedge \d x_2 \wedge \d x_3  \]
In der klassischen Physik gilt, was hier wg. $\d^2 = 0$ offensichtlich ist
\begin{align*}
\rot \circ \nabla &= 0\\
\text{div} \circ \rot &= 0
\end{align*}

\Def{}
Wir definieren die $p$-te Kohomologiegruppe der \df{de Rham-Kohomologie} durch
\[ H^p(U) := \ker (\d : \Omega^p(U) \pfeil{} \Omega^{p+1} (U)) / \Img(\d : \Omega^{p-1}(U) \pfeil{} \Omega^p(U) ) \]
Wir setzen ferner $H^p(U) = \Omega^p(U) = 0$ für $p < 0$.

\Bem{}
\begin{align*}
 H^0(U) =& \ker \d = \{f \in C^\infty (U,\R)~|~\d f = 0\} =  \{f \in C^\infty (U,\R)~|~ \frac{\partial f}{\partial x_i} = 0 \}\\
  =& \{f \in C^\infty (U,\R)~|~ f\text{ ist lokal konstant auf }U \}
\end{align*}
Daraus folgt
\[ \dim_\R H^0(U) = \text{ Zahl der Wegzshgkomp. von }U \]

\section{Das Äußere Produkt auf der Kohomologie}
\Def{}
Wir definieren auf den Kohomologiegruppen ein Produkt durch
\begin{align*}
\wedge : H^p(U) \times H^q(U) & \Pfeil{} H^{p+q}(U)\\
([\omega], [\eta]) & \longmapsto [\omega\wedge \eta]
\end{align*}
Dies ist wohldefiniert, denn für $\omega\in \Omega^p(u), \eta\in \Omega^q(U)$ mit $\d \omega = 0, \d \eta = 0$ gilt
\[ \d (\omega \wedge \eta) = (\d \omega) \wedge \eta + (-1)^p \omega \wedge (\d \eta) = 0 \]
ergo liegt $\omega \wedge \eta$ ebenfalls im Kern von $\d$. Ferner gilt für andere Repräsentanten $[\omega'] = [\omega], [\eta'] = [\eta]$
\[ \omega' = \omega + \d \alpha \text{ und }\eta' = \eta + \d \beta \]
und somit
\begin{align*}
\omega'\wedge \eta' &= (\omega + \d \alpha)\wedge (\eta + \d \beta) \\
&= \omega \wedge \eta + \d \alpha \wedge \eta + \omega \wedge \d \beta + \d \alpha \wedge \d \beta\\
 &= \omega \wedge \eta + \d(\alpha \wedge \eta + (-1)(\omega \wedge \beta) + (\alpha\wedge \d \beta) )
\end{align*}
Insofern bildet $(H^*(U), +, \wedge)$ eine graduiert-kommutative graduierte Algebra. $\wedge$ nennt man in diesem Zusammenhang auch \textit{Cup-Produkt}.

\section{Funktorialität}
\Bsp{Lineare Algebra}
Seien $V,W$ reelle Vektorräume und $A : V\pfeil{} W$ eine lineare Abbildung. Sei $\eta \in Alt^p(W), v_1, \ldots, v_p \in V$. Dann setze
\[ \omega(v_1, \ldots, v_p) := \eta( A(v_1), \ldots, A(v_p) ) \]
Dann ist $\omega \in Alt^p(V)$. Setze $Alt^p(A)(\eta) := \omega$. Dadurch erhalten wir eine lineare Abbildung
\begin{align*}
Alt^p(A) : Alt^p(W) \Pfeil{} Alt^p(V)
\end{align*}
Für eine weitere lineare Abbildung $B : W \pfeil{} P$ gilt
\[ Alt^p(B\circ A) = Alt^p(A) \circ Alt^p(B) \]
Ferner gilt
\[ Alt^p(\id{V}) = \id{Alt^p(V)} \]
Insofern liefern die $(Alt^p)_p$ eine Familie kontravarianter Funktoren von der Kategorie der reellen Vektorräume in die Kategorie der reellen Vektorräume.
\Def{}
\marginpar{Vorlesung vom 22.12.17}
Seien $U_1\subset \R^n,U_2\subset \R^m$ offen und $\phi : U_1 \pfeil{}U_2$ eine glatte Abbildung. Wir definieren folgenden Vektorraumhomomorphismus
\begin{align*}
\phi^* :  \Omega^p(U_2) & \Pfeil{} \Omega^p(U_1)\\
\eta & \longmapsto Alt^p( \phi_*) ( \eta \circ \phi)
\end{align*}
D.\,h., für $x \in U_1$ ist folgender Vektorraumhomomorphismus gegeben
\begin{align*}
\phi^*(\eta)(x) : (\T_x U_1)^p & \Pfeil{} \R\\
(v_1, \ldots, v_p) & \longmapsto \eta(\phi(x))( \phi_{*,x} v_1, \ldots, \d \phi_{*,x} v_p)
\end{align*}
\Bem{}
Die Zuweisung $\phi \mapsto \phi^*$ ist funktoriell, d.\,h., es gilt
\[ (\phi \circ \psi)^* = \psi^* \circ \phi^* \text{ und } \id{U}^* = \id{\Omega^p(U)} \]

\Bem{}
Die Zuweisung $\phi \mapsto \phi^*$ ist eindeutig durch folgende Rechenregeln bestimmt:
\begin{itemize}
	\item $\phi^*(f\omega) = f \phi^*(\omega)$ für $f \in \Omega^0(U_2)$
	\item $\phi^*(\omega_1 \wedge \omega_2) = \phi^*(\omega_1) \wedge \phi^*(\omega_2)$
	\item $\d \circ \phi^* = \phi^* \circ \d $
\end{itemize}

\Bsp{}
\begin{itemize}
	\item $\phi^*(\d x_i) = \d(\phi^*(x_i)) = \d (x_i \circ \phi ) = \d \phi_i$
	\item Sei $\gamma : (a,b) \pfeil{} U \off \R^n$ eine glatte Kurve, $\omega = f_1\d x_1 + \ldots +f_n \d x_n$ sei eine 1-Differentialform auf $U$. Es gilt
	\begin{align*}
	\gamma^*(\omega) &= \gamma^*(f_1) \wedge \gamma^*(\d x_1) + \ldots + \gamma^*(f_n) \wedge \gamma^*(\d x_n)\\
	&=f_1(\gamma(t)) \d \gamma_1 + \ldots + f_n(\gamma(t)) \d \gamma_n\\
	&= \shrp{f(\gamma(t)), \gamma'(t)} \d t
	\end{align*}
	\item Für die Volumenform $\d x_1 \wedge \ldots \wedge \d x_n$ von $U_2$ gilt
	\begin{align*}
	\phi^*(\d x_1 \wedge \ldots \wedge \d x_n) &= \d\phi_1 \wedge \ldots \wedge \d \phi_n\\
	&= \klam{\sum_{i = 1}^n \frac{\partial \phi_1}{\partial x_i}\d x_1} \wedge \ldots \wedge\klam{\sum_{i = 1}^n \frac{\partial \phi_n}{\partial x_i}\d x_1}\\
	&= \det\klam{ (\frac{\partial \phi_i}{\partial x_j})_{i,j} } \d x_1 \wedge \ldots \wedge \d x_n\\
	&= \det( J_\phi )\d x_1 \wedge \ldots \wedge \d x_n\\
	\end{align*}
	\item Betrachte die glatte Abbildung $\phi : U \times \R \pfeil{} U, U \off \R^n,$ mit
	\[ \phi(x,t) = \psi(t)\cdot x \] 
	für eine glatte Funktion $\psi : \R \pfeil{} \R$. Es gilt
	\[ \phi^*(\d x_i) = \d \phi_i = \d (\psi(t) \cdot x_i) = x_i \cdot \d \psi(t) + \psi(t) \cdot \d x_i = x_i \psi'(t)\d t + \psi(t) \d x_i \]
\end{itemize}

\section{Pullback auf die de Rham-Kohomologie}
Im Folgenden sei $\phi : U_1\subset \R^n\pfeil{} U_2\subset \R^m$ immer eine glatte Abbildung.

\Def{}
Eine Form $\omega \in \Omega^p(U)$ heißt \df{geschlossen}, falls $\d \omega = 0$.\\
$\omega$ heißt \df{exakt}, wenn es ein $\eta \in \Omega^{p-1}(U)$ gibt, mit $\omega = \d \eta$.

\Bem{}
$\phi : U_1 \pfeil{} U_2$ induziert einen Ringhomomorphismus
\[ \phi^* : H^*(U_2) \Pfeil{} H^*(U_1) \]
da $\phi^*$ den Kern und das Bild von $\d$ erhält. Dadurch folgt, dass die de Rham-Kohomologie ein kontravarianter Funktor ist.

\Satz{Poincare-Lemma}
Sei $U \subset \R$ offen und sternförmig. Dann gilt
\[ 
H^p(U) \isom{} \left\lbrace
\begin{aligned}
0 && p > 0\\
\R && p =0
\end{aligned}
\right.
\]

\begin{Beweis}{}
\begin{itemize}
	\item Ohne Einschränkung sei $0$ der Mittelpunkt des sternförmigen Gebietes $U$. Setze dann
	\begin{align*}
	ev : \Omega^0(U) & \Pfeil{} \R \\
	\omega & \longmapsto \omega(0)
	\end{align*}
	Wir wollen im Folgenden eine Kettenhomotopie $s_p : \Omega^p(U) \pfeil{} \Omega^{p-1}(U)$ konstruieren, für die gilt
	\[\d s_p + s_{p+1} \d = \left\lbrace
	\begin{aligned}
	\id{} && p > 0\\
	\id{} - ev && p = 0
	\end{aligned}
	\right. \]
	Dann folgt nämlich für $\omega \in \Omega^p(U), p > 0,$
	\[ \d \omega = 0 \Impl{} \d s_p (\omega) = \d s_p(\omega) +s_{p+1}\d\omega = \omega \]
	also $[\omega] = 0$, da $\omega \in \Img\ \d$. Ferner gilt für $p = 0$
	\[\omega - \omega(0) = s_1\d \omega = 0\]
	also $\omega = \omega(0)$ ist konstant.
	\item Eine Differentialform $\omega \in \Omega^p(U\times \R)$ hat die Gestalt
	\begin{align*}
	\omega = \sum_I f_I(x,t) \d x_I + \sum_J g_J(x,t) \d t \wedge \d x_J
	\end{align*}
	Definiere daher folgende Abbildung
	\begin{align*}
	\widehat{S}_p : \Omega^p(U\times \R ) &\Pfeil{} \Omega^{p-1}(U)\\
	\omega & \longmapsto \sum_J (\int_0^1 g_J(x,t) \d t ) \d x_J
	\end{align*}
	Dann gilt
	\[	\d \widehat{S}_p(\omega) + 	\widehat{S}_{p+1}(\d\omega) =
	\sum_I \klam{ \int_0^1 \frac{\partial f_I}{\partial t} \d t } \d x_I 
	=
	\sum_I \klam{ f_I(x,1) - f_I(x,0)} \d x_I   \]
	\item Sei nun $\psi : \R \pfeil{} \R$ eine glatte Funktion mit
	\[ \psi(t) \in
	\left\lbrace
	\begin{aligned}
	\{0\} && t \leq 0\\
	[0,1] && t \in [0,1]\\
	\{1\} && t \geq 1
	\end{aligned}
	\right.
	 \]
	 Definiere dann
	 \begin{align*}
	 \phi : U \times \R &\Pfeil{} U\\
	 (x,t) & \longmapsto \psi(t) \cdot x
	 \end{align*}
	 und setze
	 \[ s_p(\omega) := \widehat{S}_p \circ \phi^*(\omega) \]
	 Die so definierte Funktion tut das Gewünschte.
\end{itemize}
\end{Beweis}

\section{De Rham-Kohomologie von Glatten Mannigfaltigkeiten}
\marginpar{Vorlesung vom 8.1.18}
\Def{}
Sei $M$ eine (kompakte) glatte Mannigfaltigkeit der Dimension $n$. Durch Whitneys Einbettungssatz erhalten wir eine glatte Einbettung $M \subset \R^{2n+1}$. Durch den Satz über Tubenumgebungen wissen wir um die Existenz einer offenen Umgebung $U$ von $M$ in $\R^{2n+1}$, sodass
\[ U \isom{} E(\nu) \]
wobei $E \surj{p} M $ der Totalraum des Normalenbündels ist. Dieser induziert einen Deformationsretrakt $r : U \pfeil{} M$.\\
$H^*$ soll eine Homotopieinvariante sein. Dies würde einen Isomorphismus
\[ r^* : H^*(M) \Pfeil{\sim} H^*(U) \]
implizieren. In diesem Sinne definieren wir die \df{Kohomologiegruppen} von $M$ durch
\[ H^k(M) := H^k(U) \]
Diese Definition hängt von der Einbettung von $M$ ab. Insofern wäre es wünschenswert eine intrinsische Definition von $H^*(M)$ zu finden.

\Def{}
Sei $M$ eine glatte Mannigfaltigkeit der Dimension $n$.\\
Wir betrachten Familien $\omega = \{ \omega_p\}_{p\in M}$ mit $\omega_p \in Alt^k(T_pM)$. Sei eine glatte Karte 
\[\phi : U' \off M \pfeil{\sim} U \off \R^n \]
gegeben. Betrachte für jedes $p \in U'$ die Abbildung
\[ Alt^k(\phi\i_{*,\phi(p)}) : Alt^k(T_pM) \Pfeil{\sim} Alt^k(T_{\phi(p)}U) \isom{} Alt^k(\R^n) \]
Wir definieren den \df{Pullback} von $\omega$ durch
\begin{align*}
(\phi\i)^*\omega : U & \Pfeil{} Alt^k(\R^n)\\
x & \longmapsto Alt^k(\phi\i_{x,*})(\omega_{\phi\i(x)})
\end{align*}
$\omega$ heißt eine \df{glatte Differential-$k$-Form} auf $M$, wenn $(\phi\i)^*\omega$ für jede Karte $\phi$ glatt ist.\\
Es bezeichne $\Omega^k(M)$ den reellen Vektorraum aller glatter Differential-$k$-Formen auf $M$.

\Def{}
Die Karte $\phi$ von $M$ induziert einen Isomorphismus
\[ \phi_{*,p} : T_pM \Pfeil{\sim} T_{x}U \]
für $x = \phi(p)$. Dadurch erhalten wir einen Isomorphismus
\[ Alt^{k+1} ( \phi_{*,p}) : Alt^{k+1}(T_xU) \Pfeil{\sim} Alt^{k+1}(T_pM) \]
Wir können so folgende $k+1$-Form definieren
\[ \d \omega_p:= Alt^{k+1}(\phi_{*,p}) \klam{ \d ((\phi\i)^*\omega) (\phi(p)) } \]
Dadurch erhalten wir eine glatte Differentialform $\d \omega$ auf $M$. Diese Definition ist unabhängig von der Wahl der Karte $\phi$. Es gilt $\d^2 = 0$ auf $\Omega^*(M)$. Wir erhalten folglich einen Kokettenkomplex $(\Omega^*(M), \d)$.

\Def{}
Die \df{de Rham-Kohomologie} von $M$ ist definiert als die Kohomologie des Kokettenkomplexes $(\Omega^*(M), \d)$, d.\,h.
\[ H^k(M) := H^k(\Omega^*(M), \d) = \frac{\Ker (\d : \Omega^k(M) \pfeil{}\Omega^{k+1}(M))}{\Img \d : (\Omega^{k-1}(M) \pfeil{} \Omega^k(M))} \]

\Def{}
Zu glatten Differentialformen $\omega = \{ \omega_p\}, \eta = \{\eta_p\}$ auf $M$ definieren wir das \df{äußere Produkt} punktweise durch
\[ (\omega\wedge \eta)_p := \omega_p \wedge \eta_p \]
$\omega \wedge \eta$ ist wieder eine glatte Differentialform auf $M$. Ferner gilt hierfür offensichtlich wieder die Produktregel, d.\,h.
\[ \d(\omega \wedge \eta) = (\d \omega) \wedge \eta + (-1)^k\omega \wedge \d(\eta) \]
für $\omega \in \Omega^k(M)$. Wir erhalten dadurch wieder eine graduiert kommutative Algebra $\Omega^*(M)$.\\
Das äußere Produkt auf $\Omega^*(M)$ steigt wie im affinen Fall wohldefiniert auf $H^*(M)$ ab. Dadurch wird auch $H^*(M)$ zu einer graduiert kommutativen Algebra.

\Def{}
Sei $\phi : M \pfeil{} N$ eine glatte Abbildung und $\omega = \{ \omega_q\}_{q\in N} \in \Omega^k(N)$ eine glatte Differentialform. Für einen Punkt $p \in M$ erhalten wir eine lineare Abbildung
\begin{align*}
Alt^k(\phi_{*,p}) : Alt^k(T_{\phi(p)}M) & \Pfeil{} Alt^k(T_pM)
\end{align*}
In diesem Sinn setzen wir
\[ (\phi^*\omega)_p := Alt^k(\phi_{*,p})(\omega_{\phi(p)}) \]
und erhalten eine glatte Differentialform
\[ \phi^*\omega := \{ (\phi^*\omega)_p\}_{p\in M} \]
auf $M$. Dadurch erhalten wir eine lineare Abbildung
\[ \phi^* : \Omega^k(N) \pfeil{} \Omega^k(M) \]
Wie im affinen Fall ist die Zuweisung
\begin{align*}
M & \longmapsto \Omega^k(M)\\
\phi & \longmapsto \phi^*
\end{align*}
ein kontravarianter Funktor.\\
Es gilt wieder
\begin{itemize}
	\item $\phi^*(\omega \wedge \eta) = (\phi^*\omega) \wedge (\phi^*\eta)$
	\item $\d_M \circ \phi^* = \phi^* \circ \d_N$
\end{itemize}
$\phi^*$ steigt wohldefiniert auf die Kohomologie ab und liefert Abbildungen
\[ \phi^* : H^k(N) \Pfeil{} H^k(M) \]
Dadurch wird die de Rham-Kohomologie zu einem kontravarianten Funktor von der Kategorie der glatten Mannigfaltigkeiten in die Kategorie der reellen Vektorräume.

\section{Integration auf Glatten Mannigfaltigkeiten}

\Def{}
Sei $U \subset \R^n$ offen, $\omega \in \Omega^n(U)$ sei eine $n$-Form.\\
Wir definieren den \df{Träger} von $\omega$ durch
\[ \supp \omega := Cl_U{\set{x \in U}{\omega_x \neq 0}} \]
wobei $Cl_U$ den Abschluss einer Menge in $U$ bezeichnet.\\
Hat $\omega$ einen kompakten Träger, so hat $\omega$ eine glatte Fortsetzung durch Null auf $\R^n$.\\
Da $\omega$ eine $n$-Form ist, hat es die Gestalt
\[ \omega = f(x_1, \ldots, x_n) \d x_1 \wedge \ldots \wedge \d x_n \]
Wir definieren folgendes Integral zu $\omega$
\[ \int_U \omega := \int_{\R^n}\omega := \int_{\R^n} f \d x_1 \ldots \d x_n \]

\Bem{}
Seien $V,U \subset \R^n$ offen, $\theta : V \pfeil{} U$ ein Diffeomorphismus.\\
Sei ferner $\omega = f \d x_1 \wedge \ldots \d x_n\in \Omega^n(U)$ mit kompakten Träger. Dies induziert uns eine Differentialform $\theta^*\omega \in \Omega^n(V)$, welche ebenfalls kompakten Träger hat. Es gilt
\[ \theta^*\omega= \theta^*(f\d x_1 \wedge \ldots \wedge \d x_n) = f\circ \theta \cdot \theta^*(\d x_1 \wedge \ldots \wedge \d x_n) = f\circ \theta \cdot \det(J_\theta) \cdot \d x_1 \wedge \ldots \wedge \d x_n \]
Es ergibt sich nun
\begin{align*}
\int_V \theta^*\omega &= \int_{\R^n} f\circ \theta \cdot \det(J_\theta)~ \d x_1 \ldots \d x_n\\
&\gl{\text{Traforegel in }\R^n} \pm \int_{\R^n} f~ \d x_1 \ldots \d x_n = \pm \int_U \omega
\end{align*}
Ist $U$ zusammenhängend, so ist das Vorzeichen hier gerade das Vorzeichen der Jacobi-Determinante.\\
Ist $\theta$ orientierungserhaltend und $U$ zusammenhängend, so gilt also
\[ \int_V \theta^*\omega = \int_U \omega \]

\Def{}
Sei nun $M$ eine glatte, orientierte Mannigfaltigkeit der Dimension $n$ und $\omega \in \Omega^n(M)$.\\
Sei $\phi : U' \pfeil{} U$ eine orientierte Karte von $M$. Es gelte ferner
\[ \supp(\omega) \subset U' \text{ ist kompakt} \]
Wir setzen dann
\[ \int_M  \omega := \int_U (\phi\i)^*\omega \]
Dies ist wohldefiniert. Ist nämlich $\psi : U' \pfeil{} U$ eine weitere orientierte Karte, so gilt
\[ (\phi\i)^* = (\phi\i)^*\psi^* (\psi\i)^* \omega \]
Setzt man $\psi \circ \phi\i =: \theta$, so folgt mit obiger Bemerkung
\[ \int_U (\phi\i)^*\omega = \int_U (\psi\i)^*\omega \]

\Def{}
Sei nun $\omega \in \Omega^n(M)$ mit kompakten Träger. Im Allgemeinem liegt $\supp f$ nicht in einer einzelnen Karte von $M$.\\
Deswegen sei $(f_i)_i$ eine glatte Partition der Eins auf $M$, sodass $\supp f_i \subset U_i$, wobei die Paare $(U_i, \phi_i)_i$ orientierte Karten seien, die $\supp f$ überdecken, und, bei denen $\overline{U_i}$ kompakt ist.\\
Wir setzen
\[ \int_M \omega := \sum_i\int_M f_i \omega \]
Dies ist ein wohldefiniertes Integral. Es ist unabhängig von der Wahl der Partition, denn sei $(g_j)_j$ eine weitere Partition der Eins, dann gilt ja
\[ f_i = \sum_j f_i g_j \text{ und } g_j = \sum_i f_i g_j \]
Ergo folgt
\[ \sum_i \int f_i \omega = \sum_{i,j} \int f_i g_j \omega = \sum_j \int g_j \omega \]
\section{Der Allgemeine Satz von Stokes}
\marginpar{Vorlesung vom 12.1.18}
\Def{}
Sei $(M,\partial M)$ eine $n$-dimensionale Mannigfaltigkeit mit Rand. $M$ sei orientiert, dadurch ist auch $\partial M$ orientiert. $\iota : \partial M \pfeil{} M$ bezeichne die Inklusion des Randes.\\
Für $\omega \in \Omega^*(M)$ erhalten wir unter $\iota$ eine Pullbackform
\[ \omega_{|\partial M} := \iota^*\omega \in \Omega^*(\partial M) \]
Wir nennen dies die \df{Einschränkung auf den Rand} von $\omega$.

\Satz{Allgemeiner Satz von Stokes}
Sei $(M,\partial M)$ eine $n$-dimensionale orientierte Mannigfaltigkeit mit Rand. $\omega \in \Omega^{n-1}(M)$ sei eine $n-1$-Form auf $M$ mit kompakten Träger. Dann hat $\d \omega \in \Omega^n(M)$ kompakten Träger und es gilt
\[ \int_M \d \omega = \int_{\partial M} \iota^*\omega \]
\begin{Beweis}{}
Wir führen den Beweis in lokalen Koordinaten $x_1, \ldots, x_n$. Der Rand soll lokal beschrieben werden durch
\[ \partial M = \set{ (x_1,\ldots, x_n) }{ x_1 = 0 } \]
wobei allgemeine Punkte $x_1 \leq 0$ erfüllen.\\
$\omega$ habe die Gestalt
\[ \omega = \sum_{j = 1}^n f_j \d x_1 \wedge \ldots \wedge \widehat{\d x_j} \wedge \ldots \wedge \d x_n  \]
Es gilt nun
\begin{align*}
\iota^*\omega &= \sum_{j = 1}^n (f_j \circ \iota) (\d  x_1 \circ \iota) \wedge \ldots \wedge \widehat{(\d x_j\circ \iota)} \wedge \ldots \wedge (\d x_n\circ \iota) \\
&= (f_1)_{|\partial M} \d x_2 \wedge \ldots \wedge \d x_n \\
&= f_1(0, x_2, \ldots, x_n) \d x_2 \wedge \ldots \wedge \d x_n
\end{align*}
da $x_1 \circ \iota = 0$.\\
Ohne Einschränkung nehmen wir nun an, dass $\omega$ folgende Form hat
\[ \omega = f \d x_1 \wedge \ldots \wedge \widehat{\d x_j} \wedge \ldots \wedge \d x_n  \]
Es gilt nun
\begin{align*}
\d \omega &= \sum_k \frac{\partial f}{\partial x_k} \d x_k \wedge d x_1 \wedge \ldots \wedge \widehat{\d x_j} \wedge \ldots \wedge \d x_n\\
 &= (-1)^{j-1} \frac{\partial f}{\partial x_j} \d x_1 \wedge \ldots \wedge \d x_n
\end{align*}
Wir nehmen nun ferner an, dass $\omega$ kompakten Träger folgender Gestalt habe
\[ \supp \omega \subset \set{x}{-a \leq x_1 \leq 0, \bet{x_j}\leq a \forall j = 2,\ldots,n} \] 
Es gilt nun
\[ \int_M \d \omega = (-1)^{j-1} \int_{-a}^{+a} \ldots \int_{-a}^{+a} \int_{-a}^{0} \frac{\partial f}{\partial x_j} \d x_1 \ldots \d x_n \]
Wir unterscheiden nun zwei Fälle. Beachte hierbei, dass $f$ auf dem Rand von $\supp \omega$ verschwindet.
\begin{enumerate}[\text{Fall} 1]
	\item $j\neq 1$: In diesem Fall gilt
	\[ \int_M \d \omega = (-1)^{j-1} \int_{-a}^{+a} \ldots \int_{-a}^{+a} \int_{-a}^{0} f|_{x_j = -a}^{x_j = a} \d x_1 \ldots \widehat{\d x_j} \ldots \d x_n = 0  \]
	da $f|_{x_j = -a}^{x_j = a} = 0$, da $f_{| x_j = a \neq 0} = 0$.
	\item $j= 1$: In diesem Fall gilt
	\[ \int_M \d \omega = \int_{-a}^{+a} \ldots \int_{-a}^{+a} f|_{x_1 = -a}^{x_1 = 0} \d x_2 \ldots \d x_n = 
	  \int_{-a}^{+a} \ldots \int_{-a}^{+a} f(0, x_2, \ldots, x_n) \d x_2 \ldots \d x_n = \int \iota^*\omega  \]
	da $f(-a, x_2, \ldots, x_n) = 0$.
\end{enumerate}
Es sei $\omega$ nun beliebig. Sei $\{g_j\}_j$ eine glatte Partition der Eins mit
\[ \supp g_j \subset \{ x~|~ -a \leq x_1 \leq 0, \bet{x_j}\leq a \forall j = 2, \ldots, n \} \]
Es gilt $\omega = \sum_j g_j \omega$ und somit
\[ \int_M \d \omega = 
\int_M \d (\sum_j g_j \omega) = \sum_j  \int_M \d (g_j \omega) = \sum_{j} \int_{\partial M} \iota^*(g_j\omega) = \int_{\partial M}\iota^*(\sum_{j} g_j \omega) = \int_{\partial M} \iota^*\omega \]
\end{Beweis}

\newpage
\section{Das Homotopieaxiom für de Rham-Kohomologie}
\Def{}
Ein \df{Komplex} bzw. \df{Kokettenkomplex} $(C^*,\d)$ ist eine Familie $(C^p)_{p\in \Z}$ reeller Vektorräume mit einer Familie linearer Abbildungen
\[ \d : C^p \Pfeil{} C^{p+1} \]
für die gilt
\[ \d \circ \d = 0 \]

\Def{}
Sind $(C^*,\d_C)$ und $(D^*, \d_D)$ Komplexe, so ist ein \df{Homomorphismus} von Komplexen $\phi : C^* \pfeil{} D^*$ eine Familie von linearen Abbildungen
\[ \phi^p : C^p \Pfeil{} D^p \]
so, dass folgendes Diagramm kommutiert
\begin{center}
\begin{tikzcd}
	C^p \arrow[r, "\phi^p"] \arrow[d, "\d_C"]	& D^p \arrow[d, "\d_D"] \\
	C^{p+1}  \arrow[r, "\phi^{p+1}"] 	& D^{p+1} 
\end{tikzcd}
\end{center}

\Def{}
Unter einer \df{(Ketten)-Homotopie} zwischen Morphismen $\phi, \psi : C^* \pfeil{} D^*$ verstehen wir einen Kettenmorphismus $K : C^* \pfeil{} D^{*-1}$, d.\,h.
\[ K^p : C^p \Pfeil{} D^{p-1} \]
sodass gilt
\[ \d K - K \d = \phi - \psi \]
Wir schreiben in diesem Fall
\[ \phi \simeq \psi \]

\Lem{}
Gilt $\phi \simeq \psi$, so folgt
\[ H(\phi) = H(\psi) : H^*(C^*) \Pfeil{} H^*(D^*) \]
\begin{Beweis}{}
Sei $[c] \in H^*(C^*)$, d.\,h., $\d c = 0$. Es gilt
\begin{align*}
H(\phi)[c] - H(\psi)[c] 
= [\phi(c)] - [\psi(c)]
= [(\phi - \psi)(c)]
= [(\d K - K \d)(c)]
= [\d K c ]
= 0
\end{align*}
\end{Beweis}

\Bem{}
Betrachte die Projektion
\begin{align*}
\pi : \R^n \times \R^1 & \Pfeil{} \R^n\\
(x,t) & \longmapsto x
\end{align*}
und den Schnitt
\begin{align*}
s : \R^n & \Pfeil{} \R^n \times \R^1\\
x & \longmapsto (x,0)
\end{align*}
Formen auf $\R^n \times \R^1$ sind Linearkombinationen von Formen des Types
\begin{enumerate}[(i)]
	\item $f(x,t)(\pi^*\eta)  $ 
	\item $(\pi^*\eta) \wedge f(x,t) \d t$
\end{enumerate}
wobei $\eta \in \Omega^*(\R^n)$.

\Bem{}
Wir definieren eine Homotopie $K : \Omega^*(\R^n \times \R^1) \Pfeil{} \Omega^{* - 1}(\R^n \times \R^1)$ durch
\begin{align*}
K( f(x,t)(\pi^*\eta)) &= 0\\
K((\pi^*\eta) \wedge f(x,t) \d t) &= (\pi^*\eta)\cdot \int_0^t f(x,t)\d t
\end{align*}
Durch die Rechnung im Beweis des Poincare-Lemmas wissen wir nun, dass gilt
\[ \d K - K \d = \pm(\id{\Omega^*(\R^n \times \R^1)} - \Omega(\pi) \circ \Omega(s)) \]
Daraus folgt
\[ H^*(\pi) \circ H^*(s) = \id{} \]
Trivialerweise gilt
\[ H^*(s) \circ H^*(\pi) = \id{} \]
da $\pi \circ s = \id{\R^n}$. Daraus folgt, dass $H^*(\pi) : H^*(\R^n ) \pfeil{} H^*(\R^n\times \R^1)$ ein Isomorphismus ist.

\Kor{}
\[ H^p(\R^n) \isom{} H^p(\R^{n-1}) \isom{} \ldots \isom{} H^p(\R^0) \isom{} \left\lbrace 
\begin{aligned}
\R && p = 0\\
0 && p > 0
\end{aligned}
\right. \]


\Bem{}
Sei nun $M$ eine glatte Mannigfaltigkeit. Betrachte die Projektion
\[ \pi : M \times \R \Pfeil{} M \]
und den dazu gehörenden Schnitt $s : M \pfeil{} M\times \R$. Es gilt dann bereits
\[ H^*(s) H^*(\pi) = \id{} \]
Seien Karten $\{U_\alpha\}_\alpha$ auf $M$ gegeben. Dann ist $\{U_\alpha \times \R\}_\alpha$ ein korrespondierender Atlas für $M\times \R$. Mithilfe dieser Karten kann $K$ wie zuvor definiert werden, d.\,h., wir erhalten
\begin{align*}
K : \Omega^*(M\times \R) & \Pfeil{} \Omega^{*-1}(M\times \R)
\end{align*}
mit
\[ \d K - K \d = \pm (\id{\Omega^*(M\times \R)} - \Omega(\pi) \Omega(s)) \]
Dadurch folgt wieder
\[ H^*(\pi) H^*(s) = \id{} \]
Und $H^*(\pi) : H^*(M) \pfeil{} H^*(M\times \R)$ ist ein Isomorphismus.

\Satz{Homotopieaxiom}
Seien $f,g : M \pfeil{} N$ zueinander homotope glatte Abbildungen. Dann gilt
\[ H^*(f) = H^*(g) : H^*(N) \pfeil{} H^*(M) \]
\begin{Beweis}{}
Sei $F: M\times \R^1 \pfeil{} N$ eine glatte Homotopie, $F(x,t) = f(x)$ für alle $t\geq 1$ und $F(x,t)= g(x)$ für alle $t\leq 0$.\\
Betrachte ferner $s_0, s_1 : M \pfeil{} M \times \R^1$ mit $s_0(x) = (0,x)$ und $s_1(x) = (1,x)$.\\
Da $H(\pi) H(s_0) = \id{} = H(\pi)H(s_1)$ und $H(\pi)$ isomorph ist, folgt
\[ H(s_0) = H(s_1) \]
Da $F\circ s_0 = g$ und $F\circ s_1 = f$, gilt nun
\[ H(g) = H(s_0) \circ H(F) = H(s_1) \circ H(F) = H(f) \]
\end{Beweis}
\Kor{}
\marginpar{Vorlesung vom 15.1.18}
Sind $M,N$ Homotopie-äquivalente, glatte Mannigfaltigkeiten, so gilt
\[ H^*(M) \isom{} H^*(N)  \]

\section{Exakte Sequenzen}
\Def{}
Im folgenden seien $A,B$ und $C$ reelle Vektorräume und $f : A\pfeil{} B, g : B \pfeil{} C$ lineare Abbildungen.\\
Die Sequenz
\[ A \Pfeil{f} B \Pfeil{g} C \]
heißt \df{exakt} bei $B$, falls
\[ \Ker g = \Img f \]
Eine \df{kurze exakte Sequenz} ist eine Sequenz der Form
\[ 0 \Pfeil{} A \Pfeil{} B \Pfeil{} C \Pfeil{} 0 \]
die exakt bei $A,B$ und $C$ ist.

\Bsp{}
Sind $U \subset V$ Vektorräume, so ist folgende kurze exakte Sequenz gegeben
\[ 0 \Pfeil{} U \Pfeil{} V \Pfeil{} V/U \Pfeil{} 0 \]

\Def{}
Seien $A^*,B^*$ und $C^*$ Koketten-Komplexe reeller Vektorräume zusammen mit Morphismen $i : A^* \pfeil{} B^*$ und $j : B^* \pfeil{} C^*$.\\
Die Sequenz
\[ A^* \Pfeil{i} B^* \Pfeil{j} C^* \]
heißt \df{exakt} bei $B^*$, falls sie gradweise exakt ist, d.\,h.
\[ A^p \Pfeil{} B^p \Pfeil{} C^p \]
ist exakt bei $B^p$ für alle $p\in \Z$.

\Lem{Zick-Zack- bzw. Schlangenlemma}
Eine kurze exakte Sequenz
\[ 0 \Pfeil{} A^* \Pfeil{i} B^* \Pfeil{j} C^* \Pfeil{} 0  \]
von Koketten-Komplexen induziert eine lange exakte Sequenz der Kohomologiegruppen
\[ \ldots \pfeil{j^*} H^{p-1}(C^*) \pfeil{\delta} H^p(A) \pfeil{i^*} H^p(B) \pfeil{j^*} H^p(C) \pfeil{\delta} H^{p+1}(A) \pfeil{i^*} \ldots  \]
\begin{Beweis}{}
	Benutze Diagrammjagd.
\end{Beweis}

\section{Die Mayer-Vietoris-Sequenz}

Sei $M$ eine glatte Mannigfaltigkeit, $U,V \subset M$ offen mit $M = U\cup V$. Es ergeben sich folgende kommutative Diagramm
\begin{center}
	\begin{tikzcd}
		U\cap V \arrow[hookrightarrow, r, "i_U"] \arrow[hookrightarrow, d, "i_V"]	& U \arrow[hookrightarrow, d, "j_U"] \\
		V  \arrow[hookrightarrow, r, "j_V"] 	& M 
	\end{tikzcd}
$\rightsquigarrow$
\begin{tikzcd}
	\Omega^*(U\cap V) 	& \Omega^*(U) \arrow[l, "i_U^*"] \\
	\Omega^*(V)  \arrow[u, "i_V^*"] 	& \Omega^*(M)  \arrow[l, "j_V^*"] \arrow[u, "j_U^*"]
\end{tikzcd}
\end{center}

\Prop{}
Definiert man im obigen Setting folgende Abbildungen
\begin{align*}
\Omega^*(M) & \Pfeil{(i^*_U, j^*_V)} \Omega^*(U) \oplus \Omega^*(V) & \Omega^*(U) \oplus \Omega^*(V) & \Pfeil{} \Omega^*(U\cap V)\\
\omega & \longmapsto (j^*_U \omega, j^*_V \omega) & (\omega, \eta) & \longmapsto i^*_V\eta - i^*_U\omega 
\end{align*}
So liegt folgende kurze exakte Sequenz vor
\[ 0 \Pfeil{} \Omega^*(M) \Pfeil{} \Omega^*(U) \oplus \Omega^*(V) \Pfeil{} \Omega^*(U\cap V) \Pfeil{} 0 \]
\newpage
\begin{Beweis}{}
\begin{enumerate}[1)]
	\item $i_V^*-i_U^*$ ist surjektiv:\\
	Sei $\{f_U,f_V\}$ eine Partition der Eins auf $M$ bzgl. $\{U,V\}$. Ist $\omega $ in $ \Omega^*(U\cap V)$, so ist $f_U\omega $ in $\Omega^*(V)$ und $f_V \omega$ in $\Omega^*(U)$. Es gilt
	\[ (i_V^* - i_U^*)(-f_V\omega, f_U \omega) = f_U\omega_{|U\cap V} - (-f_V \omega)_{|U\cap V} = (f_U + f_V)(\omega) = \omega \]
	\item $(j_U^*, j_V^*)$ ist injektiv:\\
	Sei $\omega \in \Omega^*(M)$ mit $j_U^*(\omega) = 0$ und $j_V^*(\omega) = 0$. Dann verschwindet $\omega$ auf $U$ und $V$, also auch auf $M = U\cup V$. Ergo $\omega = 0$.
	\item Exaktheit in der Mitte:\\
	Ist $(\omega, \eta) \in \Omega^*(U) \oplus \Omega^*(V)$, s.\,d.
	\[ \eta_{|U\cap V} - \omega_{|U\cap V} = 0 \]
	dann stimmen $\omega$ und $\eta$ auf $U\cap V$ überein. Dann ist es möglich $\omega$ durch $\eta$ auf $V$ fortzusetzen und dadurch eine glatte Form $\tau \in \Omega^*(M)$ zu erhalten mit
	\[ \tau_{|U}= \omega \text{ und } \tau_{|V} = \eta \]
	Ist umgekehrt ein $\tau \in \Omega^*(M)$ gegeben, so gilt offensichtlich
	\[ (i^*_V - i_U^*)\circ (j_U^*, j_V^*)(\tau) = (i^*_V - i_U^*) (\omega_{|U}, \omega_{|V}) = \omega_{|U\cap V} - \omega_{|V\cap U} = 0 \]
\end{enumerate}
\end{Beweis}
\Bem{}
Durch das Zick-Zack-Lemma erhalten wir folgende lange exakte Sequenz der Kohomologie-Gruppen
\[ \ldots \pfeil{} H^{p-1}(U\cap V) \pfeil{} H^p(M) \pfeil{} H^p(U) \oplus H^p(V) \pfeil{} H^p(U\cap V) \pfeil{} H^{p+1}(M) \pfeil{} \ldots \]

\Bsp{}
\marginpar{Vorlesung vom 19.1.18}
Wir wollen die Kohomologiegruppen von $\R^2-\{0\} \sim S^1$ berechnen. Dazu sollen $U$ und $V$ Umgebungen von zwei Hälften von $S^1$ sein. Dann besteht $U\cap V$ aus zwei Wegzusammenhangkomponenten.\\
$U,V$ sind homotop zu Punkten und $U\cap V$ ist homotop zu zwei Punkten. Es ergibt sich folgende Mayer-Vietoris-Sequenz:
\begin{align*}
0\Pfeil{}H^0(S^1) & \Pfeil{} H^0(U) \oplus H^0(V) \Pfeil{} H^0(U\cap V) \\
\Pfeil{}  H^1(S^1) & \Pfeil{} H^1(U) \oplus H^1(V) \Pfeil{}  H^1(U\cap V)\\
 \Pfeil{}  H^2(S^1) & \Pfeil{} H^2(U) \oplus H^2(V) \Pfeil{}  H^2(U\cap V)\Pfeil{} \ldots
\end{align*}
Wir kennen nun folgende Kohomologiegruppen
\[
H^p(U) \isom{} H^p(V)\isom{}
\left\lbrace
\begin{aligned}
\R && p = 0\\
0 && \text{ sonst}
\end{aligned}
\right.
\]
Da $H^0(U\cap V)$ genau von den Wegzusammenhangkomponenten von $U\cap V$ abhängt, folgt
\[ H^0(U\cap V) = \R \oplus \R \]
Es ergibt sich nun folgende exakte Sequenz
\begin{align*}
0\Pfeil{}H^0(S^1) & \Pfeil{} \R \oplus \R \Pfeil{f} \R^2 \\
\Pfeil{}  H^1(S^1) & \Pfeil{} 0 \oplus 0 \Pfeil{}  H^1(U\cap V)\\
\Pfeil{}  H^2(S^1) & \Pfeil{} 0 \oplus 0 \Pfeil{}  H^2(U\cap V)\Pfeil{} \ldots
\end{align*}
$f$ ist dabei gegeben durch
\[ f(u,v) = u - v \]
Ergo hat $f$ einen eindimensionalen Kern und ein eindimensionales Bild. Es folgt aufgrund der Exaktheit
\[ H^0(S^1) \isom{} \Ker f \isom{}\R \text{ und } H^1(S^1) \isom{}  \R^2 / \Img f \isom{} \R \]
Für den Rest gilt nun
\[ H^{p+1}(S^1) \isom{} H^p(U\cap V) = 0 \]
da $\Omega^{p+1}(S^1) = 0$ für $p\geq 1$. Unterm Strich erhalten wir
\[
H^p(\R^2 - 0) \isom{} H^p(S^1) \isom{} 
\left\lbrace
\begin{aligned}
\R && p = 0, 1\\
0 && \text{ sont}
\end{aligned}
\right.
\]

\chapter{\textsc{Kohomologie mit Kompakten Trägern}}
\Def{}
Für $U \subset \R^n$ offen definieren wir den Raum der \df{Differentialformen mit kompakten Träger} durch
\[ \Omega_c^p(U) := \set{\omega \in \Omega^p(U)}{\supp\ \omega \text{ ist kompakt}} \]
Ist $\supp\ \omega$ kompakt, so ist auch $\supp\ \d \omega$ kompakt. Dadurch können wir $\d$ wohldefiniert auf $\Omega^*_c(U)$ einschränken. Dadurch erhalten wir folgenden Koketten-Komplex von reellen Vektorräumen
\[ \d| : \Omega_c^p(U) \Pfeil{} \Omega_c^{p+1}(U) \]
Die Kohomologiegruppen dieses Komplexes definieren die \df{Kohomologiegruppen mit kompakten Trägern}
\[ H^p_c(U) := H^p(\Omega^*_c(U)) \]

\Bsp{}
Wir wollen $H^1_c(\R^1)$ bestimmen. Betrachte dazu die Abbildung
\begin{align*}
\int : \Omega_c^1(\R^1) & \Pfeil{} \R\\
\omega &\longmapsto \int_{\R^1}\omega
\end{align*}
$\int$ ist offensichtlich linear und surjektiv. Wir wollen den Kern bestimmen. Zuerst zeigen wir
\[ \Img\ \d| \subset \ker \int \]
Sei dazu $\omega = \d f, f\in \Omega_c^0(\R)$. Dann gilt
\[\int \omega = \int \d f = \int_{-\infty}^{\infty} \frac{\partial f}{\partial x} \d x \]
Der kompakte Träger von $f$ sei enthalten in $[a,b]$. Dann haben wir
\[\int \omega = \int \d f = \int_{a}^{b} \frac{\partial f}{\partial x} \d x = f(a) - f(b) = 0-0 = 0 \]
Ferner behaupten wir
\[ \ker \int \subset \Img\ d| \]
Denn sei $\omega = g(x)\d x \in \Omega_c^1(\R^1)$ mit $\int_{-\infty}^{\infty}g(x)\d x = 0$. Ferner hat $g$ kompakten Träger. Wir definieren dann die wohldefinierte Funktion
\[ G(x) := \int_{-\infty}^xg(x) \d x \]
Es gilt dann
\[ \d G = G' \d x = g \d x = \omega \]
$G$ hat tatsächlich einen kompakten Träger, denn $g$ ist kompakt und $\int_{-\infty}^{\infty}g(x)\d x = 0$. Insofern ist $G$ nur auf einem kompakten Intervall ungleich Null.\\
Es gilt nun
\[ H^1_c(\R^1) = \ker \d| / \Img \d| = \Omega_c^1(\R^1) / \ker \int \isom{} \R^1 \]
Dies unterscheidet sich von der gewöhnlichen De-Rham-Kohomologie, für die gilt
\[H^1(\R^1) = 0\]
Insbesondere ist $H^*_c$ \textbf{keine} Homotopieinvariante.

\Def{}
Ist $M$ eine beliebige, glatte Mannigfaltigkeit, dann definieren wir
\[ \Omega_c^p(M) := \set{\omega \in \Omega^p(M)}{\supp \omega \text{ ist kompakt}} \]
und
\[H^p_c(M) := H^p(\Omega_c^*(M))\]

\Bem{}
Sei $\Phi : M \pfeil{} N$ eine glatte Abbildung. Der Pullback unter $\phi$ induziert im Allgemeinem \textbf{keine} Abbildung auf $H^*_c$.\\
Z.\,Bsp. kann man
\[ \phi : \R^1 \Pfeil{} \ast  \]
betrachten. Der Pullback einer Differentialform auf dem einpunktigen Raum $\ast$ gibt eine konstante Abbildung auf $\R^1$, die im Allgemeinem keinen kompakten Träger hat.

\Bem{}
Wir klassifizieren zwei Abbildungen von glatten Mannigfaltigkeiten, die trotzdem Abbildungen auf den Kohomologiegruppen induzieren:
\begin{enumerate}[1)]
	\item \Def{}
	Eine stetige Abbildung $f : X \pfeil{} Y$ topologischer Raum heißt \df{eigentlich}\footnote{Im Englischen \textit{proper}.}, falls das Urbild jeder kompakten Menge wieder kompakt ist, d.\,h.
	\[ A \subset Y \text{ kompakt } \Impl{} f\i(A)  \subset X\text{ kompakt} \]
	\Bsp{}
	Ist $F$ ein kompakter Raum, so ist
	\[ \R^n\times F \Pfeil{} \R^n \]
	eigentlich.\\\\
	Die Abbildung
	\[\R^n \Pfeil{} \ast \]
	ist nicht eigentlich.\\\\
	Eigentliche, glatte Abbildungen $\phi : M\pfeil{} N$ von Mannigfaltigkeiten induzieren \textbf{kontravariant} Abbildungen
	\begin{align*}
	\phi^* : H^*_c(N) & \Pfeil{} H^*_c(M)\\
	[\omega] & \longmapsto [\phi^*\omega]
	\end{align*}
	\item Sei $\iota : U \inj{} M$ die Inklusion einer offenen Teilmenge. Dann wird \textbf{kovariant} eine Abbildung
	\[ \iota_* : H^*_c(U) \Pfeil{} H^*_c(M)\]
	induziert, indem $\omega \in \Omega_c^*(U)$ durch die Null auf $M$ fortgesetzt wird.
\end{enumerate}

\section{Mayer-Vietoris für $H^*_c$}
Sei $M$ eine glatte Mannigfaltigkeit, $U,V \subset M$ offen mit $U\cup V = M$. Wir erhalten folgende kommutative Diagramme
\begin{center}
	\begin{tikzcd}
		U\cap V \arrow[hookrightarrow, r, "i_U"] \arrow[hookrightarrow, d, "i_V"]	& U \arrow[hookrightarrow, d, "j_U"] \\
		V  \arrow[hookrightarrow, r, "j_V"] 	& M 
	\end{tikzcd}
	$\rightsquigarrow$
	\begin{tikzcd}
		\Omega^*_c(U\cap V) \arrow[r, "i_{U,*}"] \arrow[d, "i_{V,*}"]	& \Omega_c^*(U) \arrow[d, "j_{U,*}"] \\
		\Omega_c^*(V)   \arrow[r, "j_{V,*}"]	& \Omega_c^*(M) 
	\end{tikzcd}
\end{center}
Dadurch erhalten wir die kurze exakte Sequenz
\[ 0 \Pfeil{} \Omega_c^*(U\cap V) \Pfeil{} \Omega_c^*(U) \oplus \Omega_c^*(V) \Pfeil{} \Omega_c^*(M) \Pfeil{} 0 \]
und die folgende lange exakte Sequenz
\[ \ldots \Pfeil{} H^p_c(U\cap V)  \Pfeil{} H_c^p(U) \oplus H_c^p(V) \Pfeil{} H^p(M)_c  \Pfeil{} H^{p+1}_c(U\cap V) \Pfeil{} \ldots \]
Beachte, diese Sequenz ist analog zu der langen exakten Sequenz auf den normalen Kohomologiegruppen bis auf die Tatsache, dass $M$ und $U\cap V$ hier die Positionen getauscht haben.
\section{Poincare-Lemma für $H^*_c$}
\marginpar{Vorlesung vom 22.1.18}
Betrachte die Projektion
\[ \pi : M \times \R^1 \Pfeil{} M \]
$\pi$ ist nicht eigentlich. Trotzdem behaupten wir, dass $\pi$ kovariant eine Abbildung
\[ \pi_* : \Omega^p_c(M \times \R^1) \Pfeil{} \Omega^{p-1}_c(M) \]
induziert. Diese Abbildung nennt man \df{Integration entlang der Faser}. Wir unterscheiden dazu zwei Typen von Elementen in $\Omega_c^*(M\times \R^1)$
\begin{enumerate}[(1)]
	\item $\pi^*\eta \cdot f(x,t)$ mit $\eta \in \Omega^*(M)$ und $f$ hat kompakten Träger
	\item $(\pi^*\eta)\wedge f(x,t) \d t$ mit $\eta \in \Omega^{*-1}(M)$
\end{enumerate}
Wir definieren $\pi_*$ durch
\begin{align*}
\pi_*(\pi^*\eta \cdot f(x,t)) &= 0\\
\pi_*((\pi^*\eta)\wedge f(x,t) \d t) &= \eta \int_{-\infty}^{\infty}f(x,t) \d t
\end{align*}
Es bleibt nun nachzuprüfen
\[ \d \pi_* = \pi_* \d \]
Dadurch erhalten wir eine wohldefinierte Abbildung
\[ \pi_* : H^*_c(M\times \R^1) \Pfeil{} H^{*-1}_c(M) \]
Wir behaupten, dass $\pi_*$ ein Isomorphismus auf den Kohomologiegruppen ist, und wollen eine Umkehrabbildung konstruieren.\\
Sei hierzu $e(t)$ eine glatte Funktion auf $\R$ mit $\int_{-\infty}^{\infty}e(t) \d t = 1$ und $\supp\ e(t)\subset \R$ kompakt. Definiere
\[ e:= e(t) \d t \in \Omega_c^1(\R^1) \]
und
\begin{align*}
e_* : \Omega_c^{*-1}(M) & \Pfeil{} \Omega_c^*(M\times \R)\\
\eta & \longmapsto (\pi^*\eta) \wedge e(t) \d t
\end{align*}
Auch in diesem Fall rechnet man nach
\[ \d \circ e_* = e_* \circ \d \]
Dadurch ergibt sich eine wohldefinierte Abbildung
\[ e_* : H_c^{*-1}(M) \Pfeil{} H_c^*(M\times \R^1) \]
Es bleibt nun zu zeigen, dass $e_*$ und $\pi_*$ auf Ebene der Kohomologiegruppen tatsächlich invers zueinander sind.\\
Wir erhalten hierdurch folgende Isomorphie
\[ H^k_c(M\times \R^1) \isom{\pi_*} H^{k-1}_c(M) \]

\Lem{Poincare für $H^*_c$}
\[ H^n_c(\R^n) = H^0_c(\R^0) = \R \]

\Satz{}
Sei $M$ eine glatte, zusammenhängende, orientierbare Mannigfaltigkeit der Dimension $n$ ohne Rand. Dann gilt
\[ H^n_c(M) \isom{} \R \]
\begin{Beweis}{}
\begin{enumerate}[\text{Schritt} 1]
	\item $\dim_\R H^n_c(M)\geq 1$:\\
	Sei dazu $U\subset M$ eine Karte $U \isom{} \R^n$. $\omega \in \Omega_c^n(U)$ sei eine $n$-Form mit
	\[ \int_{U}\omega \neq 0 \]
	Der Satz von Stokes impliziert nun, dass $\omega$ ist nicht exakt, d.\,h., $\omega \notin \Img\ \d$. Damit folgt aber auch
	\[ 0\neq [\omega] \in H^n_c(M) \]
	\item $\dim_\R H^n_c(M) \leq 1$:\\
	Sei $\omega' \in \Omega_c^n(M)$. Wir müssen zeigen, dass es ein $c\in \R$ und ein $\eta \in \Omega_c^{n-1}(M)$ gibt, sodass
	\[ \omega' = c\omega + \d \eta \]
	Dazu nehmen wir uns Karten $U_1, \ldots, U_k \subset M$ mit $U_i \isom{}\R^n$ und
	\[\supp\ \omega' \subset U_1\cup \ldots \cup U_k\]
	Sei ferner $f_1,\ldots, f_k$ eine glatte Partition der Eins mit $\supp f_i \subset U_i$ kompakt. Dann gilt
	\[ \omega' = \sum_{i} f_i \omega' \]
	Angenommen, es gäbe $c_i \in \R$ und $\eta\in \Omega_c^{n-1}(U_i)$ mit
	\[ f_i \omega' = c_i\omega + \d \eta_i \]
	für alle $i$. Dann ergäbe sich
	\[ \omega' = (\sum_{i}c_i) \omega + \d (\sum_{i} \eta_i) \]
	Insofern genügt es also, die Existenz von $c_i$ und $\eta_i$ nachzuweisen. D.\,h., wir können ohne Einschränkung annehmen, dass $\omega'$ kompakten Träger in einer Karte $V \subset M$, $V \isom{} \R^n$, hat.\\
	Beachte, dass
	\[ \supp\ \omega \subset U \text{  und  } \supp\ \omega' \subset V \]
	Da $M$ zusammenhängend ist, finden wir offene Karten $U_1, \ldots, U_r \subset M$ mit
	\begin{align*}
		U_i             & \isom{} \R     & U_1 & = U \\
		U_i\cap U_{i+1} & \neq \emptyset & U_r & = V
	\end{align*}
	Sei $\omega_1 \in \Omega_c^n(U_1)$ mit $\emptyset \neq \supp\ \omega_1 \subset U_1 \cap U_2$. Dies führen wir für $i = 2, \ldots, r-1$ fort und erhalten
	\[ 0\neq \omega_i \in \Omega^n_c(U_i\cap U_{i+1})  \]
	Mit dem Poincare Lemma folgt nun
	\[ H^n_c(U_i) \isom{} \R \]
	Deswegen existieren $c_1 \in \R$ und $\eta_1 \in \Omega^{n-1}_c(U_1)$ mit
	\[ \omega_1 = c_1 \omega + \d \eta_1 \]
	Und analog folgt die Existenz von $c_2, \ldots, c_{r-1} \in \R$ und $\eta_2, \ldots, \eta_r$ mit
	\[ \omega_i = c_i \omega_{i-1} + \d \eta_i \]
	Zusammenfügen ergibt
	\[ \omega' = c_1 \cdots c_{r-1} \omega + \d (\eta_{r-1} +c_{r-1}\eta_{r-2} + \ldots + c_{1}\cdots c_{r-2}\eta_1) \]
\end{enumerate}
\end{Beweis}
\Bem{}
Ist $M$ nicht orientierbar, so gilt
\[ H^n_c(M) = 0 \]
Der Beweis hierfür wird ähnlich geführt wie oben.

\section{Zurück zum Abbildungsgrad}
\Def{}
Sei $\phi : M \pfeil{} N$ eine glatte Abbildung glatter, orientierter, zusammenhängender, geschlossener Mannigfaltigkeiten der Dimension $n$.\\
Sei $\omega_0 \in \Omega^n_c(N)$ eine Form mit
\[ \int_M \omega_0 \neq 0 \]
Dann ist $\phi^* \omega$ eine $n$-Form auf $M$. Definiere
\[ d(\phi) := \frac{\int_{M}\phi^*\omega_0}{\int_N\omega_0} \]
$d(\phi)$ ist unabhängig von der Wahl von $\omega_0$, denn ist $\omega \in \Omega^n_c(N)$ eine weitere $n$-Form, so gilt
\[ \omega = c \omega_0 + \d \eta \]
und es gilt
\[ \frac{\int_M \phi^*\omega }{\int_N \omega} = \frac{\int_M c\phi^*\omega_0 + \phi^*\d \eta}{\int_N c \omega_0 + \d \eta} = \frac{\int_M \phi^*\omega_0}{\int_N\omega_0} = d(\phi) \]
da $\int_N \d \eta= 0$, da $N$ keinen Rand hat.

\Bem{}
$d(\phi)$ ist auch dann wohldefiniert, wenn $M,N$ nicht kompakt sind, aber $\phi$ eigentlich ist.

\Satz{}
In obiger Situation gilt
\[ d( \phi) = \deg (\phi) \]
Insbesondere ist $d(\phi)$ immer eine ganze Zahl.
\begin{Beweis}{}
	Sei $p \in N$ ein regulärer Wert von $\phi$ und $\phi\i(p) = \{q_1, \ldots, q_k\} \subset M$. Wir haben Isomorphismen
	\[ \phi_{*,q_i} : T_{q_i}M \Pfeil{\isom{}} T_pN \]
	Nach dem Satz über umkehrbare Funktionen ist $\phi$ lokal in der Nähe der $q_i$ ein Diffeomorphismus. Ergo existiert eine Karte $\R^n \isom{} V\subset N$ um $p$ und weitere Karten $\R^n \isom{} U_i \subset M$ um $q_i$, sodass
	\[ \phi_{|U_i} : U_i \Pfeil{} V \]
	ein Diffeomorphismus ist für alle $i$.\\
	Sei $\omega_0 \in \Omega_c^n(V)$ mit
	\[ \int_N \omega_0 \neq 0 \]
	Dann ist der Träger $\supp\ \phi^*\omega_0$ in $U_0 \cup \ldots \cup U_k$ enthalten. Wir erhalten $k$ Kopien von $\omega_0$
	\[ \phi^*_{|U_i}\omega_0 \in \Omega_c^n(U_i) \]
	Es folgt
	\[ \int_M \phi^*\omega_0 =
	\sum_{i=1}^k\int_{U_i} \phi^*_{|U_i}\omega_0 \]
	und es gilt
	\[\int_{U_i} \phi^*_{|U_i}\omega_0 = \epsilon_i \int_V \omega_0 \]
	wobei
	\[ \epsilon_i = \left\lbrace
	\begin{aligned}
	+1 && \phi_{*,q_i} \text{ ist orientierungserhaltend}\\
	-1 && \phi_{*,q_i} \text{ ist orientierungsumkehrend}
	\end{aligned}
	\right. \]
	Dadurch folgt
	\[ d(\phi) = \frac{\int_M \phi^*\omega_0 }{ \int_N \omega_0 }
	= \frac{\sum_{i=1}^k\epsilon_i \int_N \omega_0}{\int_N \omega_0} = \sum_{i=1}^k\epsilon_i  = \deg(\phi)
	 \]
\end{Beweis}


\section{Endlich-Dimensionalität der Kohomologie}
\marginpar{Vorlesung vom 26.1.18}
Sei $M$ eine glatte Mannigfaltigkeit der Dimension $n$.
\Def{}
Eine offene Überdeckung $\{U_\alpha\}_\alpha$ von $M$ heißt \df{gut}, wenn alle nichtleeren endlichen Schnitte $U_{\alpha_1}\cap U_{\alpha_2} \cap \ldots \cap U_{\alpha_k}$ diffeomorph zu $\R^n$ sind.

\Lem{}
$M$ besitzt eine gute Überdeckung. Diese kann endlich gewählt werden, wenn $M$ kompakt ist.
\begin{Beweisskizze}{}
Wähle eine Riemannsche Metrik auf $M$. Man kann zeigen, dass jeder Punkt von $M$ eine geodätisch konvexe Umgebung besitzt. Der Durchschnitt geodätischer konvexer Mengen ist wieder geodätisch konvex, sofern die beiden Mengen klein genug sind. Außerdem sind geodätische konvexe Mengen diffeomorph zu $\R^n$.\\
Insofern genügt es eine Überdeckung von $M$ durch geodätisch konvexe Mengen zu wählen, die klein genug sind.
\end{Beweisskizze}

\Satz{}
Hat $M$ eine gute endliche Überdeckung, so gilt für alle $p$
\[ \dim_\R H^p(M) < \infty \]
\begin{Beweis}{}
\begin{enumerate}[\text{Schritt} 1:]
	\item Seien $U,V \subset M$ offen mit
	\begin{align*}
	\dim H^p(U) &< \infty\\
	\dim H^p(V) &< \infty\\
	\dim H^p(U\cap V) &< \infty
	\end{align*}
	für alle $p$. Betrachte folgende exakte Sequenz
	\[ 
	H^p(U\cap V)
	\Pfeil{\delta^*}
	H^p(U\cup V)
	\Pfeil{\iota^*}
	H^p(U) \oplus H^p(V)
	\Pfeil{}
	H^p(U\cap V)
	 \]
	 Es gilt dann
	 \[ H^p(U\cup V) \isom{} \Img \iota^* \oplus \Img \delta^* \]
	 Da $H^p(U\cap V), H^p(U)$ und $H^p(V)$ endliche Dimension haben, haben dies auch die Bilder von $\iota^*$ und $\delta^*$. Ergo auch $H^p(U\cap V)$.
	 \item Wir führen eine vollständige Induktion nach der Kardinalität $\kappa$ einer endlichen guten Überdeckung:
	 \begin{itemize}
	 	\item $\kappa = 1$:\\
	 	Dann ist $M \isom{} \R^n$. Damit folgt auch
	 	\[ H^p(M) \isom{} H^p(\R^n) \]
	 	$H^p(\R^n)$ ist nach dem Poincare-Lemma endlich dimensional.
	 	\item $\kappa - 1 \pfeil{} \kappa$:\\
	 	Sei $U_1, \ldots, U_\kappa$ eine gute Überdeckung von $M$. Setze
	 	\[ U:= U_1 \cup \ldots \cup U_{\kappa-1} \]
	 	und
	 	\[ V:= U_\kappa \isom{} \R^n \]
	 	Durch die Induktionsannahme folgt
	 	\[ \dim H^p(U)< \infty \text{  und  } \dim H^p(V)<\infty \]
	 	für alle $p$. Betrachte
	 	\[ U\cap V = (U_1\cap U_\kappa) \cup \ldots \cup (U_\kappa\cap U_\kappa) \]
	 	$U\cap V$ besitzt die gute Überdeckung
	 	\[ U_1\cap U_\kappa, \ldots, U_{\kappa - 1} \cap U_\kappa \]
	 	der Kardinalität $\kappa - 1$. Mit der Induktionsannahme gilt also
	 	\[ \dim H^p(U\cap V) < \infty \]
	 	für alle $p$. Mit Schritt 1 folgt nun
	 	\[ \dim H^p(M) = \dim H^p(U\cup V) < \infty \]
	 	für alle $p$.
	 \end{itemize}
\end{enumerate}
\end{Beweis}\\
Wir wollen Folgendes zeigen: \textit{Ist $M$ orientiert und hat keinen Rand, so gilt}
\[ \dim H^p(M) = \dim H_c^{n-p}(M) \]
Dies nennt man \df{Poincare-Dualität}.\\

Wir wollen eine Beschreibung des Verbindungshomomorphismus
\[ \delta^* : H^p(U\cap V) \Pfeil{} H^{p+1}(U\cup V) \]
in der Mayer-Vietoris-Sequenz erarbeiten. Sie ergibt sich durch eine Diagrammjagd aus folgendem Diagramm
\begin{center}
	\begin{tikzcd}
		0 \arrow[r] &
		\Omega^p(U\cup V) \arrow[r] \arrow[d, "\d"]	& 
		\Omega^p(U) \oplus \Omega^p(V) \arrow[r] \arrow[d, "\d\oplus \d"] &
		\Omega^p(U\cap V) \arrow[r] \arrow[d, "\d"] &
		0 \\
		0 \arrow[r] &
		\Omega^{p+1}(U\cup V) \arrow[r]	& 
		\Omega^{p+1}(U) \oplus \Omega^{p+1}(V) \arrow[r] &
		\Omega^{p+1}(U\cap V) \arrow[r] &
		0 
	\end{tikzcd}
\end{center}
Es gilt somit für $\omega \in \Omega^p(U\cap V)$
\begin{align*}
\delta^*[\omega]_{|U} &= - [\d (f_V \cdot \omega)]\\
\delta^*[\omega]_{|V} &= [\d (f_U \cdot \omega)]
\end{align*}
wobei
\[ \omega = (f_U\omega)_{|V} + (f_V \omega)_{|U} \]
Wir können $\delta^*: H^{n-p-1}_c(U\cup V) \pfeil{} H^{n-p}_c(U\cap V)$ beschreiben, indem wir für ein
$[\omega] \in H^{n-p-1}_c(U\cup V)$ eine Fortsetzung durch Null von $\delta^*[\omega]$ auf $U$ und $V$ erhalten durch jeweils
\[ -[\d (f_V\omega)] \text{  und  } [\d (f_U \omega)] \]

\Lem{Fünfer-Lemma}
\begin{center}
	\begin{tikzcd}
		A \arrow[r]\arrow[d,"\alpha"] & B \arrow[r]\arrow[d,"\beta"] & C \arrow[r]\arrow[d,"\gamma"] & D \arrow[r]\arrow[d,"\delta"] & E \arrow[d,"\epsilon"]\\
		A' \arrow[r] & B' \arrow[r] & C' \arrow[r] & D' \arrow[r] & E'\\
	\end{tikzcd}
\end{center}
Kommutiert obiges Diagramm und sind $\alpha, \beta, \delta$ und $\epsilon$ Isomorphismen, so ist auch $\gamma$ isomorph.
\begin{Beweis}{}
	Diagrammjagd.
\end{Beweis}

\Bem{}
Sei $\shrp{\cdot ~|~ \cdot} : V\times W \pfeil{} \R$ eine Bilinearform, $V,W$ seien endlich-dimensionale Vektorräume.\\
Man erinnere sich daran, dass
\[ V^* = \Hom{\R}{V}{\R} \]
gilt. $\shrp{\cdot ~|~\cdot }$ induziert lineare Abbildungen
\begin{align*}
V & \Pfeil{} W^*\\
v & \longmapsto \shrp{v~|~\cdot}
\end{align*}
und
\begin{align*}
W & \Pfeil{} V^*\\
w & \longmapsto \shrp{\cdot~|~w}
\end{align*}
Man erinnere sich daran, dass $\shrp{\cdot ~|~\cdot}$ \df{nicht-ausgeartet} heißt, wenn für alle $0\neq v \in V$ und $0\neq w \in W$ Vektoren $v'\in V, w'\in W$ existieren mit
\[ \shrp{v~|~w'} \neq 0 \text{  und  } \shrp{v'~|~w} \neq 0 \]
\Lem{}
$\shrp{\cdot ~|~ \cdot}$ ist genau dann {nicht-ausgeartet}, wenn die beiden obigen induzierten Abbildungen Isomorphismen sind.

\Def{}
Sei $M$ eine orientierte, glatte Mannigfaltigkeit ohne Rand der Dimension $n$. Wir definieren eine Paarung durch
\begin{align*}
\shrp{\cdot ~|~\cdot} :H^p(M) \times H^{n-p}_c(M) &\Pfeil{} \R\\
([\omega], [\eta]) & \longmapsto \int_M \omega \wedge \eta
\end{align*}
$\omega \wedge \eta$ ist in $\Omega^n_c(M)$, da $\eta$ kompakten Träger hat. Insofern ist obiges Integral wohldefiniert.\\
Die Paarung ist unabhängig von der Wahl der Repräsentanten $\omega, \eta$. Dies folgt aus der Produktregel und dem Satz von Stokes, da $M$ keinen Rand hat.

\Satz{Poincare-Dualität}
Sei $M$ eine orientierte glatte Mannigfaltigkeit ohne Rand der Dimension $n$, die eine endliche gute Überdeckung besitzt.\\
Dann ist
\begin{align*}
\shrp{\cdot ~|~\cdot} :H^p(M) \times H^{n-p}_c(M) &\Pfeil{} \R
\end{align*}
nicht-ausgeartet für alle $p$.
\begin{Beweis}{}
Seien $U,V \subset M$ offen. Die Poincare-Dualität gelte für $U,V$ und $U\cap V$.
\begin{center}
	\begin{tikzcd}
		H^p(U\cup V) \arrow[r, "\iota^*"] \arrow[d, dash, "\oplus"]	& 
		H^p(U) \oplus H^p(V) \arrow[r]  \arrow[d, dash, "\oplus"] &
		H^p(U\cap V) \arrow[r, "\delta^*"]\arrow[d, dash, "\oplus"]&
		H^{p+1}(U\cup V)\arrow[d, dash, "\oplus"] \\
		H^{n-p}_c(U\cup V) \arrow[d, "\shrp{\cdot~|~\cdot}"]	& 
		H^{n-p}_c(U) \oplus H^{n-p}_c(V) \arrow[l, "\iota_*"]  \arrow[d, "\shrp{\cdot~|~\cdot}"] &
		H^{n-p}_c(U\cap V) \arrow[l]  \arrow[d, "\shrp{\cdot~|~\cdot}"]&
		H^{n-p-1}_c(U\cup V) \arrow[l, "\delta^*"]  \arrow[d, "\shrp{\cdot~|~\cdot}"]\\ 
		\R	& 
		\R  &
		\R &
		\R\\ 
	\end{tikzcd}
\end{center}
Indem wir den kontravarianten, exakten Funktor $\_^*$ auf die untere Zeile anwenden, erhalten wir\\\\
\adjustbox{scale=0.8,center}{%
	\begin{tikzcd}
	H^{p-1}(U) \oplus H^{p-1}(V) \arrow[r]  \arrow[d, "\isom{}"] &
	H^{p-1}(U\cap V) \arrow[r, "\delta^*"] \arrow[d, "\isom{}"]	&
	H^p(U\cup V) \arrow[r, "\iota^*"] \arrow[d, "f"]	& 
	H^p(U) \oplus H^p(V) \arrow[r]  \arrow[d, "\isom{}"] &
	H^p(U\cap V) \arrow[d, "\isom{}"] \\
	H_c^{n-p+1}(U)^* \oplus H_c^{n-p+1}(V)^* \arrow[r] &
	H_c^{n-p+1}(U\cap V)^* \arrow[r, "\delta^*"] &
	H_c^{n-p}(U\cup V)^* \arrow[r, "\iota^*"] 	& 
	H_c^{n-p}(U)^* \oplus H^p(V)^* \arrow[r]   &
	H_c^{n-p}(U\cap V)^*
\end{tikzcd}
}\\\\
Mit dem Fünferlemma würde nun folgen, dass $f$ ein Isomorphismus ist, unter der Voraussetzung, dass obiges Diagramm kommutiert.\\
Für die Kommutativität ist zu zeigen
\[
\shrp{\delta^* \omega, \eta} = \shrp{\omega, \delta^* \eta}
\]
Tatsächlich gilt
\begin{align*}
\shrp{\delta^* \omega, \eta} = \int_{U\cap V} \delta^*\omega \wedge \eta 
&= \int \d(f_U\omega) \wedge \eta
= \int \d f_U \wedge \omega \wedge \eta
\end{align*}
und
\begin{align*}
\shrp{\omega, \delta^* \eta}=\int \omega \wedge \delta_* \eta 
&= \int \omega \wedge \d  (f_V \eta)
 = \int \omega \wedge \d f_U \wedge \omega
 =(-1)^p \int \d f_U \wedge \omega \wedge \eta
\end{align*}
Somit kommutiert obiges Diagramm bis auf Vorzeichen. Ergo ist 
\[f : H^p{(U\cup V)} \pfeil{} H^{n-p}_c(U\cup V)^*\]
ein Isomorphismus.\\
Um den Beweis abzuschließen, führen wir wieder Induktion nach der Kardinalität einer endlichen guten Überdeckung von $M$.\\
Der Induktionsanfang ist hierbei gegeben durch die Poincare-Lemmata.
\end{Beweis}

\Kor{}
Insbesondere folgt aus obigem Satz
\[ H^p(M) \isom{} H^{n-p}_c(M)^* \]
für alle $p$.
\section{Der Kohomologiering von $\C P^n$}
\marginpar{Vorlesung vom 29.1.18}
\Def{}
Wir definieren den \df{komplexen projektiven Raum} durch
\[ \C P^n := (\C^{n+1} - \{0\})/\sim \]
wobei
\[ (x_1,\ldots, x_{n+1}) \sim (\lambda x_1,\ldots, \lambda x_{n+1}) \]
für alle $(x_1,\ldots, x_{n+1}) \in \C^{n+1} - \{0\}$ und $\lambda \in \C - \{0\}$.\\
Die Punkte von $\C P^n$ lassen sich durch \df{homogene Koordinaten} beschreiben:
\[ (x_1 : x_2 : \ldots : x_{n+1}) := [ (x_1,\ldots, x_{n+1}) ]_\sim \]

\Bsp{}
Betrachte $\C P^1$ in $\C P^2$ gegeben durch
\[ \C P^1 = \set{ (0: x_1 : x_2) }{} \subset \C P^2 \]
Es gilt nun
\[ \C P^2 - \C P^1 = \set{(1 : x_1 : x_2)}{x_1, x_2 \in \C} \isom{} \C^2 \]
Dies zeigt, dass $\C P^2$ eine glatte reelle Mannigfaltigkeit der Dimension 4 ist. Allgemeiner gilt
\[ \dim_\R \C P^n = 2n \]
Im Detail haben wir für $\C P^2$ folgende Karten
\begin{align*}
\{(1 : u : v)~|~u,v \in \C\} & \Pfeil{} \C^2\\
(1 : u : v) & \longmapsto (u,v)\\
\{(u : 1 : v)~|~u,v \in \C\} & \Pfeil{} \C^2\\
(u : 1 : v) & \longmapsto (u,v)\\
\{(u : v : 1)~|~u,v \in \C\} & \Pfeil{} \C^2\\
( u : v : 1 ) & \longmapsto (u,v)\\
\end{align*}

\Def{}
Wir führen \df{Polarkoordinaten} auf $\C^2 = \C P ^2 - \C P ^1$ ein
\begin{align*}
u &= r e^{2\pi i \theta }\\
v &= s e^{2\pi i \phi }
\end{align*}
für $(u,v) \in \C^2$ und $r,s \geq 0, \theta, \phi \in [0,1)$.\\
Wir deklarieren folgende 1-Formen auf $\C P^2 - \C P^1$
\[ \eta(u,v) = \frac{r^2 \d \theta + s^2 \d \phi}{ 1 + r^2 + s^2 } \]
Allerdings ist $r$ im Allgemeinem nicht glatt in Abhängigkeit von $u$, aber $r^2= x^2 + y^2 = u \cdot \overline{u}$ ist glatt für $x = Re(u), y = Im(u)$.\\
$\theta$ ist nicht einmal stetig. Aber $2\pi r^2 \d \theta$ ist glatt, denn
\begin{align*}
2\pi r^2 \d \theta &= 2\pi (x^2 + y^2) \d (\frac{1}{2\pi} \text{atan}(\frac{y}{x}) )\\
&= (x^2 + y^2)( \frac{\partial}{\partial x} \text{atan}(\frac{y}{x}) \d x +
\frac{\partial }{\partial y} \text{atan}(\frac{y}{x}) \d y )\\
&= (x^2 + y^2) \frac{1}{1 + (\frac{y}{x})^2} (-\frac{y}{x^2}\d x + \frac{1}{x}\d y)\\
&= (x^2 + y^2) \frac{x^2}{x^2 + y^2} (- y \frac{\d x}{x^2} + x \frac{\d y}{x^2})\\
&= x \d y - y \d x
\end{align*}
D.\,h., $\eta \in \Omega^1(\C P^2 - \C P^1)$. Setze
\[ \omega := \d \eta \in \Omega^2(\C P^2 - \C P^1) \]
$\omega$ ist dann geschlossen auf $\C P^2 - \C P ^1$. Ferner lässt sich $\omega$ glatt auf $\C P^2$ fortsetzen. Betrachte die Karte $(u_1 : 1 : v_1) \mapsto (u_1, v_1)$:
\begin{align*}
(u_1 : 1 : v_1) &= (r_1 e^{2\pi i \theta_1} : 1 : s_1 e^{2\pi i \phi_1} )
\end{align*}
Ist zum Beispiel $u_1 \neq 0$, so gilt
\begin{align*}
(u_1 : 1 : v_1) &= (1 : \frac{1}{r_1 e^{2\pi i \theta_1}} : \frac{s_1 e^{2\pi i \phi_1}}{r_1 e^{2\pi i \theta_1}} )\\
&=  (1 : \frac{1}{r_1} e^{-2\pi i \theta_1} : \frac{s_1}{r_1} e^{2\pi i(\phi_1- \theta_1)} )
\end{align*}
Dies lässt sich in obige Formel einsetzen. Da $\C P^2 - \C P^1$ dicht in $\C P^2$ liegt und $\omega$ stetig ist, ist die Fortsetzung auf $\C P^2$ eindeutig. Aus dem selben Grund gilt
\[ \d \omega = 0 \]
auf ganz $\C P ^2$.
Wir erhalten so $\omega \in \Ker\ \d \subset \Omega^2(\C P^2)$. Ergo ist $\omega$ geschlossen auf ganz $\C P^2$. 
$\eta$ lässt sich nicht glatt auf $\C P^2$ fortsetzen, insofern ist $\omega$ nicht exakt auf ganz $\C P^2$.\\\\


Es gilt nun
\[ \omega \wedge \omega = \frac{8rs}{(1+ r^2 + s^2)^3}\d r\d\theta \d s \d \phi \in \Omega^4(\C P ^2)  \]
und
\begin{align*}
\int_{\C P^2} \omega \wedge \omega  &= 8
\int_{0}^{\infty}\int_{0}^{1}\int_{0}^{\infty}\int_{0}^{1}
\frac{8rs}{(1+ r^2 + s^2)^3}\d r\d\theta \d s \d \phi \in \Omega^4(\C P ^2)\\
&= \int_{0}^{\infty}\int_{0}^{\infty} \frac{rs}{(1 + r^2 +s^2)^3} \d r \d s = 1
\end{align*}
Daraus folgt $\omega^2$ ist nicht exakt auf $\C P^2$. Es gilt somit
\begin{align*}
0\neq [\omega] &\in H^2(\C P^2)\\
0\neq [\omega]^2 &\in H^4(\C P^2)
\end{align*}
Wir wissen, dass $\C P^2$ geschlossen und orientiert ist. Mit der Poincare-Dualität folgt nun
\[\dim H^4(\C P^2) = \dim H^0(\C P^2) = 1 \]
da $\C P ^2$ zusammenhängend ist. Insofern wird $H^4(\C P^2)$ von $[\omega^2]$ als reeller Vektorraum erzeugt.\\
Ferne folgt aus obigem
\[ \dim H^2(\C P^2) \geq 1 \]

Um $\dim H^2(\C P^2)$ genau zu bestimmen, brauchen wir nun \textit{relative Kohomologie}.

\Def{Relative Kohomologie}
Sei $\iota : N \inj{} M$ eine geschlossene glatte eingebettete Untermannigfaltigkeit. Setze
\[ \Omega^k(M, N) :=
\set{\omega \in \Omega^k(M)}{ \iota^*(\omega) = 0 }
= \Ker (\iota^* : \Omega^k(M) \pfeil{} \Omega^k(N) ) \]
Ist $\omega \in \Omega^k(M,N)$, so gilt
\[ \iota^*(\d \omega) = \d (\iota^*\omega) = 0 \]
d.\,h., $\d$ steigt wohldefiniert auf $\Omega^*(M,N)$ ab. Dadurch erhalten wir den \df{relativen de Rham-Komplex} $(\Omega^*(M,N), \d)$.\\
Wir definieren die $k$-te \df{relative Kohomologiegruppe} durch
\[ H^k(M,N) := H^k(\Omega^*(M,N), \d) \]
Wir erhalten insbesondere folgende kurze exakte Sequenz
\[ 0 \Pfeil{} \Omega^*(M,N) \Pfeil{} \Omega^*(M) \Pfeil{\iota^*} \Omega^*(N) \Pfeil{} 0 \]
Die Surjektivität von $\iota^*$ gilt, denn:\\
Lokal ist $\iota$ gegeben durch
\[ \iota : \R^n \Inj{} \R^m = \R^n \times \R^{m-n} \]
mit einer Projektion
\[ \pi : \R^n \times \R^{m-n} \Pfeil{} \R^n \]
Daraus folgt
\[ \pi \circ \iota = \id{} \]
und damit
\[ \iota^* \circ \pi^* = \id{} \]
Daraus folgt die Surjektivität von $\iota^*$.\\
Global folgt die Surjektivität durch eine Zerlegung der Eins.\\


Wir erhalten dadurch folgende lange exakte Sequenz
\[
\ldots \Pfeil{\delta^*} H^k(M,N)
\Pfeil{  } H^k(M)
\Pfeil{ \iota^* }
H^k(N)
\Pfeil{\delta^*} 
H^{k+1}({M,N})
\Pfeil{  }\ldots
\]

\Prop{Alternative Beschreibung der Relativen Kohomologie}
Durch Fortsetzung durch Null erhält man eine Abbildung
\[ \Omega^*_c(M- N ) \Pfeil{} \Omega^*(M,N) \]
Diese vertauscht mit $\d$. Sind $M,N$ kompakt, so erhalten wir einen Isomorphismus
\[ H^*_c (M-N) \Pfeil{\isom{}} H^*(M,N) \]

\Bem{Zurück zu $\C P^2$}
Betrachte $N = \C P^1 \inj{\iota} \C P^2 = M$. Es liegt folgende exakte Sequenz vor
\begin{align*}
H^2(\C P^2, \C P^1) \Pfeil{} H^2 (\C P^2) \Pfeil{  }H^2(\C P^1) \Pfeil{  }H^3(\C P^2 , \C P^1) 
\end{align*}
Ferner gilt\footnote{Anmerkung des Autors: Die Isomorphie $H^3_c(\C P^2 - \C P^1) \isom{} H^3_c(\R ^4)$ gilt, obwohl $H^*_c$ keine Homotopie-Invariante ist, weil $\C P^2 - \C P^1 \isom{} \R ^4$ ein eigentlicher Diffeomorphismus ist.}
\[ H^3(\C P^2, \C P^1) \isom{} H^3_c(\C P^2 - \C P^1) \isom{} H^3_c(\R ^4) = 0 \]
und
\[ H^2(\C P^2, \C P^1) = H^2_c(\R^4) = 0 \]
Daraus folgt
\[ H^2(\C P^1) \isom{} H^2(\C P^2) \]
Es gilt nun
\[ H^2(\C P^2) \isom{} H^0(\C P^1)^* \isom{} \R \]
Daraus folgt
\[ H^2(\C P^2) \isom{} \R\shrp{\omega} \]

Betrachte ferner
\[ H^1(\C P^2, \C P^1) \Pfeil{} H^1(\C P^2) \Pfeil{\iota^*} H^1(\C P^1) \]
Es gilt nun
\[ H^1(\C P^2, \C P^1) \isom{} H^1_c(\R^4) = 0 \]
und
\[ H^1(\C P^1) \isom{} H^1(S^2) = 0 \]
Daraus folgt
\[ H^1(\C P^2) = 0\]
Mit der Poincare-Dualität folgt nun, da $\C P^2$ orientierbar ist.
\[ H^3(\C P^2) = 0 \]
Unterm Strich erhalten wir folgende Isomorphie von graduierten $\R$-Algebren
\[ H^*(\C P^2) \isom{} \R[~[\omega]~]/([\omega]^3 = 0) \]
wobei $[\omega]$ Grad 2 hat.\\

Allgemeiner gilt
\begin{align*}
H^*(\C P^n) \isom{} \R[ ~[\omega]~] /([\omega]^{n+1} = 0)
\end{align*}

\section{Kartesische Produkte}
Seien $M,N$ glatte Mannigfaltigkeiten. Wir fragen uns, wie wir die Kohomologie von $H^*(M\times N)$ berechnen können.

\Bem{Tensorprodukte}
Seien $V,W$ reelle endlich-dimensionale Vektorräume. Wir definieren das \df{Tensoprodukt} $V\otimes W$ durch
\begin{align*}
V \otimes W := \R\shrp{V\times W}/I
\end{align*}
wobei $I$ der Untervektorraum ist, der durch folgende Elemente erzeugt wird
\begin{align*}
(v+v', w + w') &- (v, w) - (v, w') - (v', w) - (v', w')\\
(\lambda v, \eta w) &- \lambda \eta (v, w)
\end{align*}
für $v,v'\in V, w, w'\in W, \lambda, \eta \in \R$. Die Klasse von $(v,w)$ in $V\otimes W$ bezeichnen wir mit $v\otimes w$. Es gilt dann
\begin{align*}
(v+v') \otimes (w + w') &= v \otimes w + v' \otimes w + v\otimes w' + v'\otimes w'\\
(\lambda v) \otimes (\eta w) &= \lambda \eta (v\otimes w)
\end{align*}

\marginpar{Vorlesung vom 02.02.18}
Elemente von $V\oplus W$ sind endliche Linearkombinationen der Gestalt
\[ \lambda_1 \cdot v_1 \otimes w_1 + \ldots + \lambda_k\cdot v_k \otimes w_k \]
für $\lambda_i \in \R, v_i \in V$ und $w_i \in W$. In diesem Sinne nennt man Elemente der Gestalt $v\oplus w$ \df{Elementartensoren}.\\
Ist $v_1, \ldots, v_n$ eine Basis von $V$ und $w_1, \ldots, w_m$ eine Basis von $W$, so ist
\[ \set{v_i \otimes w_j}{i = 1, \ldots, n,~~j=1, \ldots, m} \]
eine Basis von $V\otimes W$.\\
Es liegt eine kanonische Abbildung
\begin{align*}
V\times W & \Pfeil{} V\otimes W\\
(v,w) & \longmapsto v\otimes w
\end{align*}
vor. Diese ist bilinear, aber nicht linear. Tatsächlich liegt folgende Äquivalenz vor:
\begin{align*}
\left\lbrace
\begin{aligned}
\text{Bilineare }&\text{Abbildungen}\\
V\times W &\Pfeil{} U\\
(v,w) & \longmapsto \beta(v,w)
\end{aligned}
\right\rbrace
\longleftrightarrow
\left\lbrace
\begin{aligned}
\text{Lineare }&\text{Abbildungen}\\
V\otimes W &\pfeil{} U\\
v\otimes w & \longmapsto \beta(v,w)
\end{aligned}
\right\rbrace
\end{align*}

\paragraph{Zurück zu Produktmannigfaltigkeiten\\}
Es seien wieder $M,N$ glatte Mannigfaltigkeiten. Dann ist $M\times N$ ebenfalls eine glatte Mannigfaltigkeit mit Projektionen
\begin{align*}
\pi : M\times N & \Pfeil{} M\\
\rho : M\times N & \Pfeil{} N
\end{align*}
Betrachte die bilineare Abbildung
\begin{align*}
H^p(M) \times H^q(N) & \Pfeil{} H^{p+q}(M\times N)\\
([\omega],[\eta]) & \longmapsto [\pi^*\omega \wedge \eta^*\eta] 
\end{align*}
bzw. die lineare Abbildung
\begin{align*}
\kappa_{p,q} : H^p(M) \otimes H^q(N) & \Pfeil{} H^{p+q}(M\times N)\\
[\omega]\otimes [\eta] & \longmapsto [\pi^*\omega \wedge \eta^*\eta] 
\end{align*}
Diese induzieren uns eine lineare Abbildung
\begin{align*}
\kappa = \sum_{p+q=k}\kappa_{p,q} : \bigoplus_{p+q = k} H^p(M) \otimes H^q(N) & \Pfeil{} H^{k}(M\times N)
\end{align*}

\Satz{Satz von Künneth}
$\kappa$ ist ein Isomorphismus, d.\,h.
\begin{align*}
\bigoplus_{p+q = k} H^p(M) \otimes H^q(N) \isom{} H^{k}(M\times N)
\end{align*}
\begin{Beweis}{}
Wir zeigen die Aussage nur im Fall, dass $M$ eine endliche gute Überdeckung hat.\\
Wir führen wieder eine Induktion nach der Kardinalität einer endlichen guten Überdeckung.\\
Induktionsbasis: $M \isom{} \R^n$\\
Daraus folgt $M\times N = \R^n \times N$. In diesem Fall folgt die Behauptung aus dem Poincare-Lemma.\\
Induktionsschritt: Wir wollen wieder ein Argument via Mayer-Vietoris-Sequenz und Fünferlemma machen. Dazu seien $U,V\subset M$ gegeben, dann erhalten wir folgende exakte Sequenz
\begin{center}
	\begin{tikzcd}
		H^p(U\cup V) \arrow[r] & H^p(U)\oplus H^p(V) \arrow[r] & H^p(U\cap V) \arrow[r, "\delta^*"] & H^{p+1}(U\cup V)
	\end{tikzcd}
\end{center}
Tensorieren mit $H^q(N)$ erhält die Exaktheit, da Vektorräume flach sind. Dadurch erhalten wir folgende exakte Sequenz
\begin{scriptsize}
	\begin{center}
	\begin{tikzcd}
		H^p(U\cup V)\otimes H^q(N)  \arrow[r] & H^p(U)\otimes H^q(N)\oplus H^p(V)\otimes H^q(N)  \arrow[r] & H^p(U\cap V)\otimes H^q(N)  \arrow[r, "\delta^*"] & H^{p+1}(U\cup V)\otimes H^q(N) 
	\end{tikzcd}
\end{center}
\end{scriptsize}
Wir bilden für alle $p+q = k$ die direkte Summe der Sequenzen. Dies erhält weiterhin die Exaktheit, ergo erhalten wir folgende exakte Sequenz
\begin{tiny}	
\begin{center}
	\begin{tikzcd}
		\bigoplus\limits_{p+q = k}H^p(U\cup V)\otimes H^q(N)  \arrow[r] 
		& \bigoplus\limits_{p+q = k}H^p(U)\otimes H^q(N)\oplus \bigoplus\limits_{p+q = k}H^p(V)\otimes H^q(N)  \arrow[r] 
		&\bigoplus\limits_{p+q = k} H^p(U\cap V)\otimes H^q(N)  \arrow[r, "\delta^*"] 
		&\bigoplus\limits_{p+q = k+1} H^{p}(U\cup V)\otimes H^q(N) 
	\end{tikzcd}
\end{center}
\end{tiny}
Da $U\times N$ und $V\times N$ offene Teilmengen von $M\times N$ sind, haben diese ihrerseits eine exakte Mayer-Vietoris-Sequenz
\begin{center}
\begin{small}
\begin{tikzcd}
H^k((U\cup V)\times N) \arrow[r] 
& H^k(U\times N) \oplus H^k(V\times N)  \arrow[r] 
&H^k((U\cap V) \times N)  \arrow[r, "\delta^*"] 
& H^{k+1}((U\cup V)\times N)
\end{tikzcd}
\end{small}
\end{center}
Ferner erhalten wir durch die Abbildungen $\kappa$ ein Diagramm\\\\
\adjustbox{scale=0.6,center}{%
\begin{tikzcd}
	\bigoplus\limits_{p+q = k}H^p(U\cup V)\otimes H^q(N)  \arrow[r] \arrow[d, "\kappa"]
	& \bigoplus\limits_{p+q = k}H^p(U)\otimes H^q(N)\oplus \bigoplus\limits_{p+q = k}H^p(V)\otimes H^q(N)  \arrow[r] \arrow[d, "\kappa\oplus \kappa"]
	&\bigoplus\limits_{p+q = k} H^p(U\cap V)\otimes H^q(N)  \arrow[r, "\delta^*"] \arrow[d, "\kappa"]
	&\bigoplus\limits_{p+q = k+1} H^{p}(U\cup V)\otimes H^q(N)\arrow[d, "\kappa"] \\
	H^k((U\cup V)\times N) \arrow[r] 
	& H^k(U\times N) \oplus H^k(V\times N)  \arrow[r] 
	&H^k((U\cap V) \times N)  \arrow[r, "\delta^*"] 
	& H^{k+1}((U\cup V)\times N)
\end{tikzcd}
}\\\\
Wegen der Induktionshypothese können wir annehmen, dass alle $\kappa$ außer
\[ \kappa : \bigoplus\limits_{p+q = k}H^p(U\cup V)\otimes H^q(N) \Pfeil{} H^k((U\cup V) \otimes N) \]
Isomorphismen sind. Wir wollen die Kommutativität des Diagramms zeigen. Für alle Quadrate, die nicht das $\delta^*$ involvieren, ist dies klar. Für
\begin{center}
	\begin{tikzcd}
		\bigoplus\limits_{p+q = k} H^p(U\cap V)\otimes H^q(N)  \arrow[r, "\delta^*"] \arrow[d, "\kappa"]
		&\bigoplus\limits_{p+q = k+1} H^{p}(U\cup V)\otimes H^q(N)\arrow[d, "\kappa"] \\
		H^k((U\cap V) \times N)  \arrow[r, "\delta^*"] 
		& H^{k+1}((U\cup V)\times N)
	\end{tikzcd}
\end{center}
gilt
\begin{align*}
\kappa \delta^*(\omega \otimes \eta) 
&= \kappa((\delta^*\omega) \otimes \eta)\\
&= \kappa(\d (f_U\omega) \otimes \eta )\\
&= \pi^*(\d (f_U\omega)) \wedge \rho^*\eta\\
&= \d (\pi^*f_U \cdot \pi^*\omega \wedge \rho^*\eta)\\
&= \d (\pi^*f_U \cdot \kappa(\omega \otimes \eta))\\
&= \delta^*\kappa(\omega \times \eta)
\end{align*}
wobei $f_U\circ \pi, f_V \circ \pi$ eine Partition der Eins bzgl. $U\times N$ und $V\times N$ ist.\\
Daraus folgt, dass
\[ \kappa : \bigoplus\limits_{p+q = k}H^p(U\cup V)\otimes H^q(N) \Pfeil{} H^k((U\cup V) \otimes N) \]
ein Isomorphismus ist.
\end{Beweis}

\Bsp{}
\begin{align*}
H^*(S^3 \times S^2)&=H^*(S^3)\otimes H^*(S^2)\\
&= (H^0(S^3)\oplus H^3(S^3)) \otimes (H^0(S^2) \oplus H^2(S^2)\\
&= H^0(S^3)\otimes H^0(S^2) \oplus H^0(S^3) \otimes H^2(S^2) 
\oplus H^3(S^3) \otimes H^0(S^3) \oplus H^3(S^3)\otimes H^2(S^2)
\end{align*}
Daraus folgt
\begin{align*}
H^p(S^3 \times S^2) =
\left\lbrace
\begin{aligned}
H^0(S^3)\otimes H^0(S^2) = \R && p= 0\\
H^0(S^3)\otimes H^2(S^2) = \R && p= 2\\
H^3(S^3)\otimes H^0(S^2) = \R && p= 3\\
H^3(S^3)\otimes H^2(S^2) = \R && p= 5\\
0 && \text{ sonst}
\end{aligned}
\right.
\end{align*}

\section{Die Signatur einer Mannigfaltigkeit}
Sei $M$ eine glatte orientierte geschlossene Mannigfaltigkeit der Dimension $n = 4k$.\\
Betrachte die Bilinearform
\begin{align*}
\shrp{\cdot ~|~\cdot}: H^{2k}(M) \times H^{2k} & \Pfeil{} \R\\
([\omega], [\eta]) &\longmapsto \int_M \omega \wedge \eta
\end{align*}
Diese Bilinearform ist symmetrisch und nicht ausgeartet wegen der Poincare-Dualität.\\
Deswegen existiert eine Basis $v_1, \ldots, v_{s+t}$ von $H^{2k}(M)$, in der die Matrixdarstellung von $\shrp{\cdot~|~\cdot}$ diagonal ist. D.\,h.
\[ \shrp{v_i|v_j}_{i,j} = \left(
\begin{matrix}
	p_1 &        &     &     &        &  0   \\
	    & \ddots &     &     &        &  \\
	    &        & p_s &     &        &     \\
	    &        &     & n_1 &        &     \\
	    &        &     &     & \ddots &     \\
	0    &        &     &     &        & n_t
\end{matrix}
\right) \]
mit $p_i >0$ und $n_i< 0$. Die \df{Signatur} 
\[\sigma(\shrp{\cdot~|~\cdot}):= s-t\]
ist dann definiert als die Anzahl der positiven Eigenwerte von $\shrp{\cdot~|~\cdot}$ minus die Anzahl der negativen Eigenwerte von $\shrp{\cdot~|~\cdot}$.\\
Wir definieren die \df{Signatur} der Mannigfaltigkeit $M$ durch
\[ \sigma(M):= \sigma(\shrp{\cdot~|~\cdot} : H^{2k}(M) \otimes H^{2k}(M) \pfeil{} \R)= s-t\]
Die Signatur einer Mannigfaltigkeit ist offenbar eine orientierte Homotopieinvariante.

\paragraph{Frage}
Wann ist $\sigma(\shrp{\cdot ~|~\cdot})$ gleich Null für eine beliebige symmetrische nicht-ausgeartete Bilinearform $\shrp{\cdot~|~\cdot} : V\otimes V \pfeil{} \R$?

\Def{}
Ein Untervektorraum $L\subset V$ heißt \df{Lagrangscher Untervektorraum} bzgl. $\shrp{\cdot ~|~\cdot}$, falls
\begin{enumerate}[i.)]
	\item $\shrp{\cdot ~|~\cdot}_{L} = 0$
	\item $\dim L = \frac{1}{2} \dim V$
\end{enumerate}

\Satz{}
$\sigma(\shrp{\cdot ~|~\cdot})$ ist genau dann Null, wenn $V$ einen Lagrangschen Untervektorraum hat.

\marginpar{Vorlesung vom 05.02.18}

\subsection{Konvention}
Wir setzen die Signatur einer Mannigfaltigkeit gleich Null, falls die Dimension der Mannigfaltigkeit nicht von Vier geteilt wird. D.\,h.,
\[ \sigma(M) := 0\text{  falls  } \dim M = 0 \mod{4} \]

\Bsp{}
\begin{itemize}
	\item Für alle $n> 0$ gilt
	\[ \sigma(S^n) = 0 \]
	\item $\sigma(\text{Punkt}) = 1$
	\item Wir betrachten $M = \C P^2$. Betrachte die 2-Form $\omega \in \C P^2$, die $H^2(\C P^2)$ erzeugt und für die gilt
	\[ \int_{\C P^2} \omega \wedge \omega = \shrp{[\omega], [\omega]} = +1 \]
	Daraus folgt, dass die Matrix von $\shrp{\cdot ~|~\cdot}$ gerade durch $+1$ dargestellt wird. Insofern folgt
	\[ \sigma(\C P^2) = +1 \]
\end{itemize}

\begin{Beweis}{Satz \ref{SatzLagrangeRaum}}
\begin{enumerate}[1.)]
\item Sei $\sigma(\shrp{\cdot ~|~\cdot}) = 0$. Dann existiert eine Basis $v_1, \ldots, v_s, w_1,\ldots, w_s$ von $V$, sodass gilt
\begin{align*}
\left(
\begin{matrix}
\shrp{v_i|v_j} & \shrp{v_i|w_j}\\
\shrp{w_i|v_j} & \shrp{w_i | w_j}
\end{matrix}
\right)
=
\left(
\begin{matrix}
	p_1 &        &     &     &        &  \\
	    & \ddots &     &     &        &     \\
	    &        & p_s &     &        &     \\
	    &        &     & n_1 &        &     \\
	    &        &     &     & \ddots &     \\
	    &        &     &     &        & n_s
\end{matrix}
\right)
\end{align*}
mit
\[ p_i > 0 \text{  und  } n_i < 0 \]
Setzt man
\begin{align*}
&e_i := \frac{v_i}{\sqrt{p_i}}
&f_i := \frac{w_i}{\sqrt{\bet{n_i}}}
\end{align*}
so gilt
\begin{align*}
\left(
\begin{matrix}
\shrp{e_i|e_j} & \shrp{e_i|f_j}\\
\shrp{f_i|e_j} & \shrp{f_i | f_j}
\end{matrix}
\right)
=
\left(
\begin{matrix}
	1 &        &   &    &        &  \\
	  & \ddots &   &    &        &    \\
	  &        & 1 &    &        &    \\
	  &        &   & -1 &        &    \\
	  &        &   &    & \ddots &    \\
	  &        &   &    &        & -1
\end{matrix}
\right)
\end{align*}
Setze nun
\begin{align*}
e_i' &:= \frac{e_i + f_i}{\sqrt{2}} &f_i' := \frac{e_i - f_i}{\sqrt{2}}
\end{align*}
Dann folgt
\begin{align*}
\shrp{e_i'|e_i'} &= \frac{1}{2}\shrp{e_i+f_i|e_i + f_i}\\
&= \frac{1}{2} \klam{ \shrp{e_i|e_i} + 2 \shrp{e_i|f_i} + \shrp{f_i|f_i} } = 0
\end{align*}
und
\begin{align*}
\shrp{e_i'|e_j'} &= \frac{1}{2}\shrp{e_i+f_i|e_j + f_j}\\
&= \frac{1}{2} \klam{ \shrp{e_i|e_j} + \shrp{e_i|f_j}+\shrp{e_j|f_i} + \shrp{f_i|f_j} } = 0
\end{align*}
Setze ergo
\[ L:= \R \shrp{e_1',\ldots, e_s'} \]
$L$ ist dann der gesuchte Lagrangsche Untervektorraum.
\item Sei nun $L$ ein Lagrangscher Untervektorraum. Sei $\{e_1, \ldots, e_s\}$ eine Basis für $L$. Dann existieren Vektoren $\{f_1, \ldots, f_s\}$ mit
\[ \shrp{e_i|f_j} = \delta_{i,j} = \left\lbrace
\begin{aligned}
1 && i = j\\
0 && i\neq j
\end{aligned}
\right. \]
Daraus folgt, dass $\{e_1, \ldots, e_s, f_1, \ldots, f_s \}$ eine Basis von $V$ ist. Es ergibt sich nun
\begin{align*}
\left(
\begin{matrix}
\shrp{e_i|e_j} & \shrp{e_i|f_j}\\
\shrp{f_i|e_j} & \shrp{f_i | f_j}
\end{matrix}
\right)
=
\left(
\begin{matrix}
	0 &   &   & 1 &   &   \\
	  & 0 &   &   & 1 &   \\
	  &   & 0 &   &   & 1 \\
	1 &   &   & * &   &   \\
	  & 1 &   &   & * &   \\
	  &   & 1 &   &   & *
\end{matrix}
\right)
\end{align*}
Setze
\[\widehat{f}_j \in f_j +L \]
so für alle $j$, dass
\[ \shrp{f_i|f_j} = 0 \]
für alle $i,j$. Es gilt dann
\begin{align*}
\left(
\begin{matrix}
\shrp{e_i|e_j} & \shrp{e_i|f_j}\\
\shrp{f_i|e_j} & \shrp{f_i | f_j}
\end{matrix}
\right)
=
\left(
\begin{matrix}
0 &   &   &  1   &        &     \\
& 0 &   &     &   1     &     \\
&   & 0 &     &        &  1   \\
1  &   &   & 0 &        &     \\
& 1  &   &     & 0 &     \\
&   & 1  &     &        & 0
\end{matrix}
\right)
\end{align*}
Wir ordnen die Basis $\{e_1, \ldots, e_s, \widehat{f}_1, \ldots, \widehat{f}_s \}$ zu $\{e_1, \widehat{f}_1, \ldots, e_s, \widehat{f}_s \}$ um und erhalten folgende Darstellung von $\shrp{~|~}$
\begin{align*}
\left(
\begin{matrix}
	0 & 1 &   &   &   &   &   &  \\
	1 & 0 &   &   &   &   &   &  \\
	  &   & 0 & 1 &   &   &   &  \\
	  &   & 1 & 0 &   &   &   &  \\
	  &   &   &   & \ddots & \ddots &   &  \\
	  &   &   &   & \ddots & \ddots &   &  \\
	  &   &   &   &   &   & 0 & 1 \\
	  &   &   &   &   &   & 1 & 0
\end{matrix}
\right)
\end{align*}
Ersetzt man $e_i, \widehat{f}_i$ durch $\frac{e_i + \widehat{f}_i}{\sqrt{2}},\frac{e_i - \widehat{f}_i}{\sqrt{2}}$, so verwandeln sich die Diagonalblöcke von $\left(
\begin{matrix}
0 & 1 \\
1 & 0 
\end{matrix}
\right)$
zu  $\left(
\begin{matrix}
1 & 0 \\
0 & -1 
\end{matrix}
\right)$
Die Anzahl der positiven Diagonaleinträge minus die Anzahl der negativen Diagonaleinträge von einer Matrix von $\shrp{~|~}$ ist nun Null.
\end{enumerate}
\end{Beweis}

\paragraph{Geometrische Frage}
Wann ist eine geschlossene (orientierte) Mannigfaltigkeit der Dimension $n$ Rand einer kompakten (kompatibel orientierten) Mannigfaltigkeit der Dimension $(n+1)$.

\Bem{}
Die Forderung, dass der Korand kompakt ist, ist in obiger Frage wesentlich, da $M$ immer der Rand von $M \times [0,1)$ ist.

\Bsp{}
\begin{itemize}
	\item Sei $n = 1$ und $M = S^1$. Dann ist $M$ der Rand von $D^2$. Diese Wahl ist nicht eindeutig, da $M$ als Rand einer Fläche mit beliebigen Geschlecht dargestellt werden kann.
	\item Sei $n=2$. Betrachten Sie eine Fläche von Geschlecht $g$ Ihres Vertrauens. Im nicht orientierten Fall kann man sich die Kleinsche Flasche vorstellen als unorientierbare Fläche von Geschlecht Eins. Die Kleinsche Flasche ist tatsächlich der Rand einer dreidimensionalen Mannigfaltigkeit, denn:
	\begin{center}
		\textit{Stellen Sie sich hier eine Grafik vor, die alles erklärt.}
	\end{center}
	\item Sei $n = 3$. Jede dreidimensionale Mannigfaltigkeit lässt sich als Rand einer vierdimensionalen Mannigfaltigkeit darstellen. Dies besagt der Satz von Rohlin.
\end{itemize}

\Satz{Satz von R. Thom}
Sei $n = 4k$. Existiert eine $(n+1)$-dimensionale orientierte, kompakte Mannigfaltigkeit $W$ mit $\partial W = M$, so gilt
\[ \sigma(M) = 0 \]
\begin{Beweis}{}
Sei $M$ der Rand von $W$. Betrachte das Paar $(W,M)$:
\begin{align*}
H^{2k}(W)
\Pfeil{\iota^*} H^{2k}(M)
\Pfeil{\delta^*} H^{2k+1}(W,M) 
\end{align*}
Mit der Poincare-Dualität folgen Isomorphismen\\\\
\adjustbox{scale=1,center}{%
	\begin{tikzcd}
		H^{2k}(W) \arrow[d, "\sim"] \arrow[r,"\iota^*"]
		& H^{2k}(M) \arrow[d, "\sim"] \arrow[r, "\delta^*"]
		& H^{2k+1}(W,M) \arrow[d, "\sim"]\\
		H^{2k+1}(W,M)^*\arrow[r]
		& H^{2k}(M)^*  \arrow[r, "(\iota^*)^*"]
		& H^{2k+1}(W)^*
	\end{tikzcd}
}\\\\
wobei gilt
\[ H^k(W, \partial W) = H^{k}_c(W - \partial W) = H^{n-k}(W)^* \]
für jede kompakte, $n$-dimensionale Mannigfaltigkeit. Obiges Diagramm kommutiert bis auf Vorzeichen.\\
Wir behaupten nun, dass
\[ L:= \Img \iota^* \subset H^{2k}(M) \]
ein Lagrangscher Untervektorraum ist. Es gilt
\begin{itemize}
	\item
	\[H^{2k}(M)/\Ker \delta^* \isom{} \Img \delta^* \isom{} \Img (\iota^*)^* = L^*\]
	und
	\[H^{2k}(M)/\Ker \delta^* = H^{2k}(M) / \Img \iota^* \isom{} H^{2k}(M)/L \isom{} L^*\]
	Daraus folgt
	\[ \dim L = \frac{1}{2} \dim H^{2k}(M) \]
	\item Wir müssen zeigen, dass $\shrp{\cdot~|~\cdot}$ auf $L$ verschwindet. Seien $\omega, \eta \in \Img i^*$. Dann existieren $\omega', \eta' \in H^{2k}(W)$ mit
	\begin{align*}
	\omega = \iota^* \omega' && \eta = \iota^*\eta'
	\end{align*}
	Es gilt nun
	\begin{align*}
	\shrp{\omega, \eta} &=
	\int_M \omega \wedge \eta
	= \int_{\partial W} \iota^*\omega' \wedge \iota^* \eta'
	= \int_{\partial W} \iota^*(\omega'\wedge \eta')
	= \int_W \d \iota^*(\omega' \wedge \eta')
	=\int_W  \iota^*\d(\omega' \wedge \eta')
	= 0
	\end{align*}
	
	
\end{itemize}
\end{Beweis}


\Kor{}
Ist $\partial W = M \dot{\cup} N$, so heißen $M$ und $N$ \df{kobordant}. In diesem Fall gilt
\[ \sigma(M) = \sigma(N) \].


\printindex
\end{document}