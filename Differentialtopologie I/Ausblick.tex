\chapter{\textsc{Differentialtopologie II}, SS18}
\begin{itemize}
	\item Fortsetzung: Bordismustheorie
	\item Riemannsche Geometrie:
	\begin{itemize}
		\item Riemannsche Metrik
		\item Zusammenhänge
		\item Paralleltransport
		\item Kovariante Ableitung
		\item Geodätische
		\item Exponentialabbildung
		\item Riemannscher Krümmungstensor
		\item Schnittkrümmung, Ricci-Krümmung, Gauss-Krümmung für Flächen
		\item vollständige Metriken
		\item Satz von Hopf-Rinow
		\item Satz von Hadamard $(\kappa \leq 0)$
		\item {Gauss-Bonnet}:
		\[ \int_{M^2} \kappa = 2\pi \chi(M) \]
		\item Jacobi-Felder
	\end{itemize}
	\item Morse-Theorie:
	\[ f: M \Pfeil{} \R \]
	mit isolierten kritischen Punkten (und Zusatzeigenschaften) heißt Morse-Funktion.\\
	Für diese Funktion betrachtet man die kritischen Punkte und ordnet diesen je einen Index zu. Es gilt so in etwa
	\begin{align*}
	\left\lbrace
	\begin{aligned}
	\text{kritische Punkte}\\
	+ \text{Indizes}
	\end{aligned}
	\right\rbrace
	\longleftrightarrow
	\left\lbrace
	\begin{aligned}
	\dim H^i(M)\\
	+\text{Zellstruktur}
	\end{aligned}
	\right\rbrace
	\end{align*}
	
	\item Charakteristischen Klassen für Vektorraumbündel:
	\begin{itemize}
		\item \begin{tikzcd}
			\R^n, \C^n \arrow[r] & E\arrow[d]\\
			& B
			\end{tikzcd}
		Dies induziert ein $c(E) \in H^*(B)$.
		\item Eulerklasse
		\item Chernklassen
		\item Satz von Levay-Hirsch (Faserbündel)
	\end{itemize}
\end{itemize}