\marginpar{Vorlesung vom 02.02.18}
Elemente von $V\oplus W$ sind endliche Linearkombinationen der Gestalt
\[ \lambda_1 \cdot v_1 \otimes w_1 + \ldots + \lambda_k\cdot v_k \otimes w_k \]
für $\lambda_i \in \R, v_i \in V$ und $w_i \in W$. In diesem Sinne nennt man Elemente der Gestalt $v\oplus w$ \df{Elementartensoren}.\\
Ist $v_1, \ldots, v_n$ eine Basis von $V$ und $w_1, \ldots, w_m$ eine Basis von $W$, so ist
\[ \set{v_i \otimes w_j}{i = 1, \ldots, n,~~j=1, \ldots, m} \]
eine Basis von $V\otimes W$.\\
Es liegt eine kanonische Abbildung
\begin{align*}
V\times W & \Pfeil{} V\otimes W\\
(v,w) & \longmapsto v\otimes w
\end{align*}
vor. Diese ist bilinear, aber nicht linear. Tatsächlich liegt folgende Äquivalenz vor:
\begin{align*}
\left\lbrace
\begin{aligned}
\text{Bilineare }&\text{Abbildungen}\\
V\times W &\Pfeil{} U\\
(v,w) & \longmapsto \beta(v,w)
\end{aligned}
\right\rbrace
\longleftrightarrow
\left\lbrace
\begin{aligned}
\text{Lineare }&\text{Abbildungen}\\
V\otimes W &\pfeil{} U\\
v\otimes w & \longmapsto \beta(v,w)
\end{aligned}
\right\rbrace
\end{align*}

\paragraph{Zurück zu Produktmannigfaltigkeiten\\}
Es seien wieder $M,N$ glatte Mannigfaltigkeiten. Dann ist $M\times N$ ebenfalls eine glatte Mannigfaltigkeit mit Projektionen
\begin{align*}
\pi : M\times N & \Pfeil{} M\\
\rho : M\times N & \Pfeil{} N
\end{align*}
Betrachte die bilineare Abbildung
\begin{align*}
H^p(M) \times H^q(N) & \Pfeil{} H^{p+q}(M\times N)\\
([\omega],[\eta]) & \longmapsto [\pi^*\omega \wedge \eta^*\eta] 
\end{align*}
bzw. die lineare Abbildung
\begin{align*}
\kappa_{p,q} : H^p(M) \otimes H^q(N) & \Pfeil{} H^{p+q}(M\times N)\\
[\omega]\otimes [\eta] & \longmapsto [\pi^*\omega \wedge \eta^*\eta] 
\end{align*}
Diese induzieren uns eine lineare Abbildung
\begin{align*}
\kappa = \sum_{p+q=k}\kappa_{p,q} : \bigoplus_{p+q = k} H^p(M) \otimes H^q(N) & \Pfeil{} H^{k}(M\times N)
\end{align*}

\Satz{Satz von Künneth}
$\kappa$ ist ein Isomorphismus, d.\,h.
\begin{align*}
\bigoplus_{p+q = k} H^p(M) \otimes H^q(N) \isom{} H^{k}(M\times N)
\end{align*}
\begin{Beweis}{}
Wir zeigen die Aussage nur im Fall, dass $M$ eine endliche gute Überdeckung hat.\\
Wir führen wieder eine Induktion nach der Kardinalität einer endlichen guten Überdeckung.\\
Induktionsbasis: $M \isom{} \R^n$\\
Daraus folgt $M\times N = \R^n \times N$. In diesem Fall folgt die Behauptung aus dem Poincare-Lemma.\\
Induktionsschritt: Wir wollen wieder ein Argument via Mayer-Vietoris-Sequenz und Fünferlemma machen. Dazu seien $U,V\subset M$ gegeben, dann erhalten wir folgende exakte Sequenz
\begin{center}
	\begin{tikzcd}
		H^p(U\cup V) \arrow[r] & H^p(U)\oplus H^p(V) \arrow[r] & H^p(U\cap V) \arrow[r, "\delta^*"] & H^{p+1}(U\cup V)
	\end{tikzcd}
\end{center}
Tensorieren mit $H^q(N)$ erhält die Exaktheit, da Vektorräume flach sind. Dadurch erhalten wir folgende exakte Sequenz
\begin{scriptsize}
	\begin{center}
	\begin{tikzcd}
		H^p(U\cup V)\otimes H^q(N)  \arrow[r] & H^p(U)\otimes H^q(N)\oplus H^p(V)\otimes H^q(N)  \arrow[r] & H^p(U\cap V)\otimes H^q(N)  \arrow[r, "\delta^*"] & H^{p+1}(U\cup V)\otimes H^q(N) 
	\end{tikzcd}
\end{center}
\end{scriptsize}
Wir bilden für alle $p+q = k$ die direkte Summe der Sequenzen. Dies erhält weiterhin die Exaktheit, ergo erhalten wir folgende exakte Sequenz
\begin{tiny}	
\begin{center}
	\begin{tikzcd}
		\bigoplus\limits_{p+q = k}H^p(U\cup V)\otimes H^q(N)  \arrow[r] 
		& \bigoplus\limits_{p+q = k}H^p(U)\otimes H^q(N)\oplus \bigoplus\limits_{p+q = k}H^p(V)\otimes H^q(N)  \arrow[r] 
		&\bigoplus\limits_{p+q = k} H^p(U\cap V)\otimes H^q(N)  \arrow[r, "\delta^*"] 
		&\bigoplus\limits_{p+q = k+1} H^{p}(U\cup V)\otimes H^q(N) 
	\end{tikzcd}
\end{center}
\end{tiny}
Da $U\times N$ und $V\times N$ offene Teilmengen von $M\times N$ sind, haben diese ihrerseits eine exakte Mayer-Vietoris-Sequenz
\begin{center}
\begin{small}
\begin{tikzcd}
H^k((U\cup V)\times N) \arrow[r] 
& H^k(U\times N) \oplus H^k(V\times N)  \arrow[r] 
&H^k((U\cap V) \times N)  \arrow[r, "\delta^*"] 
& H^{k+1}((U\cup V)\times N)
\end{tikzcd}
\end{small}
\end{center}
Ferner erhalten wir durch die Abbildungen $\kappa$ ein Diagramm\\\\
\adjustbox{scale=0.6,center}{%
\begin{tikzcd}
	\bigoplus\limits_{p+q = k}H^p(U\cup V)\otimes H^q(N)  \arrow[r] \arrow[d, "\kappa"]
	& \bigoplus\limits_{p+q = k}H^p(U)\otimes H^q(N)\oplus \bigoplus\limits_{p+q = k}H^p(V)\otimes H^q(N)  \arrow[r] \arrow[d, "\kappa\oplus \kappa"]
	&\bigoplus\limits_{p+q = k} H^p(U\cap V)\otimes H^q(N)  \arrow[r, "\delta^*"] \arrow[d, "\kappa"]
	&\bigoplus\limits_{p+q = k+1} H^{p}(U\cup V)\otimes H^q(N)\arrow[d, "\kappa"] \\
	H^k((U\cup V)\times N) \arrow[r] 
	& H^k(U\times N) \oplus H^k(V\times N)  \arrow[r] 
	&H^k((U\cap V) \times N)  \arrow[r, "\delta^*"] 
	& H^{k+1}((U\cup V)\times N)
\end{tikzcd}
}\\\\
Wegen der Induktionshypothese können wir annehmen, dass alle $\kappa$ außer
\[ \kappa : \bigoplus\limits_{p+q = k}H^p(U\cup V)\otimes H^q(N) \Pfeil{} H^k((U\cup V) \otimes N) \]
Isomorphismen sind. Wir wollen die Kommutativität des Diagramms zeigen. Für alle Quadrate, die nicht das $\delta^*$ involvieren, ist dies klar. Für
\begin{center}
	\begin{tikzcd}
		\bigoplus\limits_{p+q = k} H^p(U\cap V)\otimes H^q(N)  \arrow[r, "\delta^*"] \arrow[d, "\kappa"]
		&\bigoplus\limits_{p+q = k+1} H^{p}(U\cup V)\otimes H^q(N)\arrow[d, "\kappa"] \\
		H^k((U\cap V) \times N)  \arrow[r, "\delta^*"] 
		& H^{k+1}((U\cup V)\times N)
	\end{tikzcd}
\end{center}
gilt
\begin{align*}
\kappa \delta^*(\omega \otimes \eta) 
&= \kappa((\delta^*\omega) \otimes \eta)\\
&= \kappa(\d (f_U\omega) \otimes \eta )\\
&= \pi^*(\d (f_U\omega)) \wedge \rho^*\eta\\
&= \d (\pi^*f_U \cdot \pi^*\omega \wedge \rho^*\eta)\\
&= \d (\pi^*f_U \cdot \kappa(\omega \otimes \eta))\\
&= \delta^*\kappa(\omega \times \eta)
\end{align*}
wobei $f_U\circ \pi, f_V \circ \pi$ eine Partition der Eins bzgl. $U\times N$ und $V\times N$ ist.\\
Daraus folgt, dass
\[ \kappa : \bigoplus\limits_{p+q = k}H^p(U\cup V)\otimes H^q(N) \Pfeil{} H^k((U\cup V) \otimes N) \]
ein Isomorphismus ist.
\end{Beweis}

\Bsp{}
\begin{align*}
H^*(S^3 \times S^2)&=H^*(S^3)\otimes H^*(S^2)\\
&= (H^0(S^3)\oplus H^3(S^3)) \otimes (H^0(S^2) \oplus H^2(S^2)\\
&= H^0(S^3)\otimes H^0(S^2) \oplus H^0(S^3) \otimes H^2(S^2) 
\oplus H^3(S^3) \otimes H^0(S^3) \oplus H^3(S^3)\otimes H^2(S^2)
\end{align*}
Daraus folgt
\begin{align*}
H^p(S^3 \times S^2) =
\left\lbrace
\begin{aligned}
H^0(S^3)\otimes H^0(S^2) = \R && p= 0\\
H^0(S^3)\otimes H^2(S^2) = \R && p= 2\\
H^3(S^3)\otimes H^0(S^2) = \R && p= 3\\
H^3(S^3)\otimes H^2(S^2) = \R && p= 5\\
0 && \text{ sonst}
\end{aligned}
\right.
\end{align*}

\chapter{\textsc{Bordismus-Theorie}}

\section{Die Signatur einer Mannigfaltigkeit}
Sei $M$ eine glatte orientierte geschlossene Mannigfaltigkeit der Dimension $n = 4k$.\\
Betrachte die Bilinearform
\begin{align*}
\shrp{\cdot ~|~\cdot}: H^{2k}(M) \times H^{2k}(M) & \Pfeil{} \R\\
([\omega], [\eta]) &\longmapsto \int_M \omega \wedge \eta
\end{align*}
Diese Bilinearform ist symmetrisch und nicht ausgeartet wegen der Poincare-Dualität.\\
Deswegen existiert eine Basis $v_1, \ldots, v_{s+t}$ von $H^{2k}(M)$, in der die Matrixdarstellung von $\shrp{\cdot~|~\cdot}$ diagonal ist. D.\,h.
\[ \shrp{v_i|v_j}_{i,j} = \left(
\begin{matrix}
	p_1 &        &     &     &        &  0   \\
	    & \ddots &     &     &        &  \\
	    &        & p_s &     &        &     \\
	    &        &     & n_1 &        &     \\
	    &        &     &     & \ddots &     \\
	0    &        &     &     &        & n_t
\end{matrix}
\right) \]
mit $p_i >0$ und $n_i< 0$. Die \df{Signatur} 
\[\sigma(\shrp{\cdot~|~\cdot}):= s-t\]
ist dann definiert als die Anzahl der positiven Eigenwerte von $\shrp{\cdot~|~\cdot}$ minus die Anzahl der negativen Eigenwerte von $\shrp{\cdot~|~\cdot}$.\\
Wir definieren die \df{Signatur} der Mannigfaltigkeit $M$ durch
\[ \sigma(M):= \sigma(\shrp{\cdot~|~\cdot} : H^{2k}(M) \otimes H^{2k}(M) \pfeil{} \R)= s-t\]
Die Signatur einer Mannigfaltigkeit ist offenbar eine orientierte Homotopieinvariante.

\paragraph{Frage}
Wann ist $\sigma(\shrp{\cdot ~|~\cdot})$ gleich Null für eine beliebige symmetrische nicht-ausgeartete Bilinearform $\shrp{\cdot~|~\cdot} : V\otimes V \pfeil{} \R$?

\Def{}
Ein Untervektorraum $L\subset V$ heißt \df{Lagrangscher Untervektorraum}\footnote{Eine andere Bezeichnung ist \textit{maximaler isotroper} Untervektorraum.} bzgl. $\shrp{\cdot ~|~\cdot}$, falls
\begin{enumerate}[i.)]
	\item $\shrp{\cdot ~|~\cdot}_{L} = 0$
	\item $\dim L = \frac{1}{2} \dim V$
\end{enumerate}

\Satz{}
\label{SatzLagrangeRaum}
Sei $V$ endlich dimensional.\\
$\sigma(\shrp{\cdot ~|~\cdot})$ ist genau dann Null, wenn $V$ einen Lagrangschen Untervektorraum hat.
